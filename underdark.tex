\section{The Thermodynaminomicon} %: The G of Life Continues
\vspace*{-10pt}
\quot{"Seriously, this cave is a mile below the surface, what do they eat?"}

When you drop an egg on the ground, it breaks. But when you drop a broken egg it doesn't reform a perfect shell. That's entropy: the simple fact that it requires energy inputs to maintain an open system, and a closed system only degrades over time. If there is to be Life, let alone civilization, there has to be some way of getting energy into the system. For the surface worlders, that's not even a problem: the Sun shines energy down on the surface all day. But for those who live in the dark realms: be it sewers, the ocean floor, or the classic dungeon complex; there has got to be a source of energy in the D\&D world that just doesn't exist in ours. You don't see lush forests in marine trenches because the energy inputs just aren't there. But in D\&D neither the ocean floor not the underdark is a desert -- it's a vibrant ecology. Here's why:

%\vspace*{-\lineskip}
\subsection{Those Whacky Mushrooms}

Ask any DM what people eat down there in the Underdark and they'll probably say "Mushrooms" because it is a well known fact that mushrooms are a fungus and not a plant and they don't need sunlight to grow. What they do need, however, is a source of chemical energy. That can be dead bodies or otherwise digested material, sure, but it still has to come from somewhere. When fungus grabs some organic material and converts it into more fungus, that's an inefficient process. You actually have less energy worth of fungus than you had energy worth of whatever it was that the fungus was eating.

Don't get me wrong -- edible fungus will happily consume things that are inedible (like woodchips) or even poisonous (like waste) and turn it into something you can eat. It just won't make something out of nothing. So mushrooms are an excuse for why Underdark dwellers have something to eat if and only if there is some other way that energy is coming into the system. It could be anything really, just as the trees will turn useless sunlight into tasty peaches, the fungus will turn useless chemical waste into delicious mushrooms. But while mushrooms can handwave the problem of converting energy that you can't use into energy that you can, they don't explain how that energy gets in to the Underdark in the first place.

\subsection{Portals}

The Inner and Outer Planes are infinite in scope, and don't follow the same thermodynamic constraints as Prime worlds do. A portal to the plane of Fire simply heaps energy into whatever cavern it happens to be in, which is like having a little sun right in your closet. Plants and algae can use that energy input to grow, and ooze monsters can eat those plant and civilizations can eat the ooze monsters. As long as those portals stay open, energy can keep entering the system even underground.

\subsection{Magic Deposits}

Magical locations exist all over the place in the D\&D world. Some of them are on the surface, and these become fairy rings or magical castles, or whatever. The point is, almost all of them are put to use, because even the ones in deadly jungles or on top of treacherous mountains simply are not hard to get to. One teleport and you're there. But the world isn't flat (unless you're in Bytopia or the Abyss), and actually there's a lot more volume of the planet that's under the ground than there is on the surface. And that means that the vast majority of magical locations are underground somewhere.

If you consider that in the D\&D world, magic is approximately as important as the Sun is to life on the planet, that means that a significant amount of the total energy inputs in the system are underground to the exclusion of being above ground. The underdark has irregularly spaced Nausica-style gardens down there that are supported by magic upwellings. Each of these locations is massively more productive than a field or forest above ground of similar size, and the immutable fact that the surrounding territory is lifeless barren stone causes fights over these locations to be extremely brutal.

\subsection{Subduction}

The world above is filled with living things, and even in D\&D the vast majority of living things die. When living things die, they leave behind a body that other living things can eat. Some of this actually gets eaten, while other bodies end up sinking into the land. In our world, the organic material sucked into the ground eventually becomes fossil fuels, but in D\&D there's actually stuff down there that will eat it. This is a relatively minor input into the dungeon ecosystem, but it essentially means that there aren't any oil deposits to be found in the D\&D world.

\section{The Bionomicon} %: Biodiversity in Top Predators
\vspace*{-10pt}
\quot{"Where do all these monsters come from? How do they persist generation after generation?"}

Resources are limited, and thus only a finite number of creatures can be supported on any particular diet within any area. D\&D has a biodiversity that would make a modern ecologist sing and dance -- Greyhawk has every single species that Earth has, and then it also has thousands of additional monsters, many of which are technically top predators. That's hard to manage. Remember that to support a single top predator requires a huge amount of energy inputs.

\subsection{The Chicken and the Egg}

In our own world, the question of the chicken and the egg is one put out mostly to confuse the very young. It actually has a definitive answer, the egg came first. It doesn't even matter where you draw the line as to what is a chicken and what is some other creature, because wherever that line is drawn, the creature in question was first an egg and its own parents were not chickens. But in the D\&D world, that's an open question because a lot of creatures are made rather than born and appear in the world fully formed. Golems simply don't need a stable pool of Golems in order to maintain genetic diversity. They don't even have genetic diversity.

Lots of other creatures in D\&D reproduce in a completely magical way. Demons simply spawn out of flaming pits of rebirth, Aleaxes are created from the fact that a god spent some divine focus (it comes somewhere between "Place Papal Magnet" and "Earthquake" I think), and chimeras are assembled out of parts by mad wizards.

\subsection{Small Isolated Populations are Bad}

Bad things happen to a species if there aren't many instances of them in the world. Cheetahs can all accept skin grafts from each other and are all weak to the same diseases. Seriously, all cheetahs could go extinct next year, their existence is that fragile. In D\&D there are numerous species that are less well represented in the world than cheetahs, so why don't they have the same problems?

Truth be told, some of them do. Many times a dungeon will contain a "unique monster" that the player characters will go and kill. That's an extinction event right there. There will be no more generations of "five-armed fire troll" once you kill Togor the five-armed fire troll. But some of them do not, and the reason is that D\&D has a wealth of completely magical ways to keep a lineage from drying up. Even if a creature can't find a mate of its own kind anywhere in the world, there's always the planes and the realms of sorcery. An Archon or Devil can provide a means for any creature to create a new generation. In doing so, the new creature is sometimes pretty much indistinguishable from its mortal parent, and sometimes it shows up as a half-whatever. The point is, no one has to be doing anything particularly weird for the world to have half-fiend dire tigers in it. Which is a load off our minds, because the D\&D world does have half-fiend dire tigers in it.

\subsection{We Eat What We Like}

It has been noted by many an observer that actually humans are a top predator themselves, that it takes nearly 20 years for a human to grow to full size, and they're only 70 kilograms at that point. Thus, the concept of a creature persisting on a diet of human flesh is pretty much absurd. Especially if it lives in an out-of-the-way area like a mountain top or the bottom of a forgotten cave, there's just no way that something can live on manflesh alone.

Some monsters however, get the vast majority of their sustenance through magical means and only need small influxes of real food from their human food. The classic example, of course, is Vampires. They consume much less in food energy than they use up in maintaining their undead existence. Most of their energy is actually siphoned off the negative energy plane, and the drinking of human blood is just a symbolic evil act that they need to perform in order to keep that juice flowing. Similarly, mindflayers are sustained not by the nutritive value of brains, but by their own psychic powers. They need to eat the brains of intelligent creatures to keep their psychic powers sharp, and they need their psychic powers to be sharp in order to survive day to day without eating enough calories to keep themselves alive in the normal way.

Secondarily, lots of creatures have a perfectly fine diet of normal food and simply happen to attack and eat humans if they encounter them. A gelatinous cube, for example, lives just fine off of the lichens and offal that it scrapes off the walls and ceilings with its passing. But it certainly won't turn down a meal of 70 kilos of meat if it comes to that.

\section{Empirinomicon}
\vspace*{-10pt}
\quot{"What difference does it make? We're just going to kill them anyway\ldots"}

The Underdark is filled with empires. It is notable not only that there is civilization down here at all, but that there are actual empires. There's enough space for there to be tribes of creatures that people can unite as part of their nation-building exercises.

But more than that, these disparate tribes are more different than any group of humans have ever been one from another. While there is only one race of humans on Earth, there can be literally dozens of races in a single cavern settlement in a D\&D world.

\subsection{The Myconids: Apathy Writ Large}

The Myconids don't need anybody, have no enemies, and don't care what you do. The first appearance of the Myconid empire was in an adventure otherwise filled with hostile evil humanoids, with the concept being that the Myconids were completely indifferent and would actually return aggression with aggression or soft words with a weird hallucinogenic telepathic mind-meld. The theme was that the Myconids were there as some sort of bizarre intelligence test for your players -- if the players "figured out" that they could get past the Myconids without resorting to conflict the adventure would be easier and otherwise it would be more difficult.

But the years have passed by, and Myconids are no longer the new kids on the block. Players actually know where they stand with Myconids, and subsequent attempts to write adventures with the same setup have had to make up entirely new Neutral monsters to fill the same role (like the desmodu and their stupid Buck Rogers style fighter planes).

The Myconids have a pretty good world conquest strategy. They don't need anything at all to reproduce themselves and they don't really have to interact with the economies that the other races have bought into. They have an army of the dead and huge piles of crazy potions that they make free of course, but they aren't even interested in fighting the other races. The extent of the Myconid empire and population is limited by the farthest reaches of their mushroom fields, which simply grow a little every year. Eventually the Myconids might push the limits of their fields into areas other races intended to use, and then the Myconids will go ape and start throwing armies of zombies and themselves (every Myconid is completely replaceable) at whatever is in their way -- but for now they just hang out and groove on the telepathy spores and share their dreams.

\subsection{The Aboleth: Inheritance of the Memory Fish}

The signature ability of the Aboleth is that they remember everything that was known to any creature they eat. There's no game mechanics for that ability, it just happens. And while Aboleth can create layered full-sensory illusions whenever they want, and dominate enemies, and turn humanoid opponents into lame deep one clones, the memory devouring ability is the really memorable one. It means that Aboleths remember intimate details about ancient events that go back to the very origins of the Aboleth race, and it means that Aboleths have access to all kinds of special magical sites and gadgets that are not available to other races.

The D\&D world is filled with weird one-off locations that under certain circumstances do potentially awesome things. Over the course of their adventuring lives, a party of adventurers is liable to encounter several of them; and the Aboleth remembers all of the ones found by any of the adventurers that it or any of its ancestors back to the beginning of time ever ate -- which is a tremendously large amount. So any Aboleth plot is going to be facilitated by magical architecture and supernatural convergeances that happen once every hundred years and all kinds of crazy crap. Aboleth periodically plot to take over the world, and otherwise they pretty much sit and fester at the bottom of the nastiest, stinkiest pools in the Underdark.

Keep in mind that even by itself, an Aboleth is badly under CRed. They have actual dominate with a pretty decent DC and they aren't half bad in combat. Also, they have long duration images available as spell-like abilities. So any Aboleth area is going to be covered in layered illusions. If an Aboleth attacks, chances are that it's going to have several turns of doing pretty much anything it wants as the PCs sit shell-shocked on the other side of an illusory wall.

\subsection{The Illithid: Slaves of the Elder Brain}

The illithid have a bad reputation among the other sentient races: the mind flayers see them as food, and most races take offense to that viewpoint. It should be the fuel for a war of extinction on the illithid race, but three things protect the illithid as a race: each is a powerful artillery piece surrounded by hordes of charmed minions; the race is led by powerful elder brains who are each the equal of a powerful sorcerers; and they make good neighbors\ldots\ . that's right, they're good neighbors.

Each mindflayer can potentially control a small army of charmed slaves, can defeat a small army with their powerful stunning blast ability and resistance to magic, and can negotiate any conflict into peace with the ability to telepathically communicate and read minds, but their best ability is the ability to plane shift. As a race that can naturally use this ability, they can hop between planes and be within five miles of any location that they can imagine on their own plane or any other. This key fact means that when they go rampaging for brains and slaves, they not only do it in some place far from their home, but they might do it on some other plane entirely. Due to the fact that few races can mount an extraplanar war, the flayers generally are too far and too difficult to find to ever face retaliation for their acts. Because of the limits of their travel ability, the mind flayers will clear and patrol an area about ten miles from their home, removing any potential threat and keeping dangerous predators away. In this way, they can return to their homes in relative peace, and by scrupulously not preying on their neighbors, they avoid any retaliation on single illithid walking home. Add in their mind-reading and telepathy ability, they are naturally suited to making mutual defense pacts with nearby races so that they can establish a peaceful dominance in their own territory.

The fact that every mind flayer enclave is controlled by a powerful elder brain is another fact that makes their enclaves safe and their culture vital. As a powerful, but generally stationary creature, it has every incentive to make its home as well-defended as possible, drawing on its own powers to equip its home with wondrous architecture and traps befitting a powerful sorcerer (or psionicist). Add in the hordes of slaves and the illithid themselves, this means that even moderately-sized enclaves can bring to bear enough force to make taking the city an extremely unprofitable enterprise.

One final note about the illithid: as planar travelers with an innate ability to travel to any plane, they often gain access to technology and magic from cultures beyond counting. While the mind flayers are geniuses in their own right, they often store knowledge of these devices in the minds of their slaves, a practice that leads them to losing that knowledge when hunger or carelessness takes that slave away. Even so, expect the average illithid to be a font of secrets dredged from dozens of extraplanar cultures, its home filled with odd artifacts and devices culled from those far-off places. If their own powers and hungers weren't so great, they might even be drawn to the exploitation of this knowledge. Luckily for the races of the worlds, the mind flayer's total confidence in their own abilities and need expend time feeding on difficult-to-acquire fare makes them ignore all but the most obviously useful things stolen from other cultures.\\



\subsection{The Drow: A Higher Technology Setting}
Everyone knows that the dark elves are hardcore. Even in the bad old days of D\&D's conception, dark elves were "mirror matches" to the party with class levels of their own, crazy magic items of their own, and good tactics. The real question is: "why are the dark elves so hardcore?" The answer is simple. Dark elves are living at a higher technology level than the rest of the D\&D world; their society only exists because, as a society, they cheat. Rather than grow food like surface races, they eat magic mushrooms as the basis of the food chain, they enslave other races for menial positions rather than work, and rather than mine or gather their own resources, they take them from other races. They don't even have to work that hard on defense as Underdark caverns are naturally easy to defend with small numbers of troops stationed at chokepoints

This means that your average dark elf has free time to spare. While some take that time to indulge in the pleasures of their society, most dark elves are the products of a very odd world view: if only dark elves are your peers and everyone else is a slave, then the only real power worth having is power over other dark elves. That being the case, this means that dark elves have both the free time and the inclination to attempt to enslave each other all the time. This breeds great internal strife with each noble house being an armed camp designed to use stealth, power, and manipulation in order to both resist the efforts of other dark elves and attempt to enslave them.

Like any heavily-automated wartime culture, the dark elves spend considerable resources on weapons research, espionage, and cultural misinformation. This means that every noble house or other organization is constantly looking for an "edge" in their dealings with other dark elves and other races. This leads them to kidnap experts from other races, engage in spell research, experiment with weird magic or exotic technology, forge partnerships with magically or technologically-advanced races, and otherwise do whatever it takes to grow in power. In any particular drow city you can expect to see dozens of competing forms of magic, odd inventions ranging from mechanical limbs to powered gliders, exotic troops like demon-bred orcs or elite espionage races like skulkers, and constructions with magical architecture or resonances. Since every drow is attempting to master his peers, these magics and technologies are tightly controlled, meaning that when the individual or organization that controls them is killed off, these secrets are often lost, meaning that any particular drow might be using relics from a previous generation (that he may well lack the ability to understand or reproduce).

Other races in the Underdark realize that the dark elves truly only want to control each other, so they allow the occasional resource and slave raids of the dark elves. They know that the dark elves are ill-suited to any form of large-scale conquest due to their particular style of command, so placating the drow is often the best way to conserve the resources of your society. Since the other Underdark races tithe goods to the drow and the drow are smart enough to see the value in trade relationships, Underdark races of note are allowed to use dark elf cities as major trading posts between their own kind and other races. The dark elves see all races as being underneath them, so as long as the other races show deference to them and bring in a profit in trade, they allow this enterprise to continue.

The average drow city is thus a hornet's nest of power, full of indolent, wildly dangerous, and spoiled aristocrats. Even the lowliest of drow lives in a level of luxury suitable to the most powerful of nobles on the surface, and each one of them has reached adulthood in an atmosphere of distrust and manipulation with the weakest dying early. As individuals this makes them powerful and cruel, but as a race it keeps them inwardly looking and less of a threat than more ambitious warrior races, a fact that actually prevents other races from gathering their forces and destroying the drow outright.

\subsection{The Eye Tyrants: Lingering Hatred}

Beholders are among the most magically capable of races; this is a fact that's well accepted. So why don't they own everyone? They can charm people at will, kill people, inspire terror by turning people to stone or just inspire terror with real magical fear. Even the lowliest beholder can attempt to destroy the world one ten foot cube at a time.

The truth? Beholders are paranoid jerks. Beholders recognize that they each have the ability to destroy each other with at least four different effects, and they also have to live with the fact that any beholder can charm his lessers and betters if he happens to get the jump on them. This means that, as a race, the beholders are like gunslingers from the American Old West. They know that to associate with any other beholder is to risk disintegrate or charm rays in the back with the winner taking the loser's treasure and slaves. Actual beholder meetings involve both parties agreeing to aim their anti-magic rays on each other, and only then can negotiations or exchanges can take place. This means that actual organizations of beholders are practically impossible as life breaks down as soon as you can't cover all your enemies in your anti-magic eye (with many eyes, they can spot ambushes by thralls pretty easily, so it's only your peers you fear).

That still doesn't explain why beholders don't go on bloody rampages on the surface races. The reason is simple: longbows. The average beholder is a tough customer that can expect to wreak a bloody swath of destruction if he chooses, but he's painfully weak against long-range weapons. Any race with even a passing knowledge of the beholders knows that they charm people, so they also know that killing the beholder frees the slaves. This is why the beholders prefer underground areas. With ranged weapons blocked by the limits of doors, walls and corridors, beholders can reign as kings in underground or indoor environments.

While the Spelljammer universe posits "nations" of beholders held together by racial hatred of other beholders and everyone else, this is really a fallacy. Racial pride or nationalism come to far seconds when you realize that beholder nations are actually held together by single individuals who routinely charm every other beholder on their ship and force them to "play nice" with the other beholders. Any "Hive" metaphor talking about beholders actually talks about the layers of charm effects building a top-down command structure where the Queen controls everyone, then the second-in-commands has control of everyone else in order to serve as secondary leaders but unable to defy the Queen. Like other kinds of dictatorships, killing the leader figure causes the nation to fall apart into bloody factions headed by second-in-commands attempting to assert control of the others, and these power plays work through the bonds of charm effects and personal charisma. Some Hives are actually controlled by a racial variant called a Hive Mother, but these creatures are merely biological extensions of relationships that already exist within beholder society.

\subsection{The Kuo-Toans: Opportunities Slip By}

The Kuo-Toans are extremely aware that things used to be pretty awesome if you were a Kuo-Toa, and now they suck. They are actually a deep ocean race and they don't live in the ocean at all anymore. That's because long ago they lost a war to the Sahuagin. And they lost it badly. Now they live in pools of water that often as not are fresh water in the bottoms of caves, and they hate it here. The lack of pressure and salinization of the water makes the Kuo-Toans unhealthy and uncomfortable, and they end up stinking of rotting fish as their skin becomes diseased and crumbly.

Every generation of Kuo-Toa is a little sicker than the one before it, and everyone understands and accepts that the race is dying out. Every Kuo-Toa expects the future to be worse than the present, and the Whips (the Clerics of the Kuo-Toa) do nothing to forestall that process or convince their people otherwise. Legends say that the Great Evils they left behind at the bottom of the seas will eventually return to destroy the whole world, but only once they've successfully fed them with enough of the misery of the Kuo-Toan people. No one in Kuo-Toa society wants to become a leader, because the world will become even more unpleasant every year and the leaders are always blamed. A Kuo-Toa gains a position of leadership when the old leader is finally killed and eaten for failure and the Whips draw lots for who has to be the next leader. Most Kuo-Toans believe that these lots are fixed in advance, and they're right.

Despite the utter hatred that all Kuo-Toans hold for all other races, they are perfectly willing to trade with them. The Kuo-Toans are badly out of their element, and need nutritional supplementation from far away just to survive. They need to receive goods from the Drow, and they know it. They hate the Drow, as they hate everyone, but that doesn't stop them from trading. The Kuo-Toans understand that the Aboleth know where every single one of their spawning pools are and that only laziness on the part of the Aboleth has left the Kuo-Toan people with any territory at all. Still, they wait in the darkness for the cataclysm to come that will put them out of their misery and slaughter all the other creatures of the land and the sea. Their one hope is that just before the last Kuo-Toa is finally slain, that they will see with their own eyes the horrible vengeance wreaked on the other empires.\\



\subsection{The Troglodytes: Persecution Complex}

Everybody hates Troglodytes. Everybody. They don't necessarily do anything that horrible in the scheme of things, they just happen to stink so bad that they can cause other races to collapse from nausea. So while the dwarves have a very complicated relationship with the hobgoblins where they have long periods of intermittent strife punctuated by flourishing trade relations and shared artistic histories and stuff -- the dwarves literally don't have anything nice to say about the troglodytes at all. Their entire history with the Trogs is one where sometimes they fought and sometimes they didn't fight. There's never been real peace between the Troglodytes and anyone. That's hard on a culture, and their isolation has made them intensely barbaric and xenophobic by the standards of any other race. Troglodytes can't even use the other races as slaves, and open lines of communication do not exist so the Troglodytes can't trade captives back to other races for concessions on the bargaining table. There isn't even a bargaining table at the end of any conflict.

So if you get captured by Troglodytes, you're going to be eaten or sacrificed to their dark gods. The Troglodytes literally have no other use for captives. So the only reason for any of the other races to surrender to Troglodytes is if they think there is a chance they will be rescued. Troglodytes themselves will generally not surrender in battle because they believe that other races will treat them the same way that they treat others.

A natural result of all this, is that the Troglodyte tribes are much lower tech than the rest of the setting. They have no trade in equipment or ideas with the other races, so the only steel equipment that Troglodytes have is what they looted off of fallen enemies. Most troglodyte weapons are just sharp rocks. Troglodytes can be useful to a campaign because they have a legitimate reason to still be "cave men" even while the rest of the world is putting together portal highways and overshot water mills.
