
\section{The Mass Combat Mini-game}

Mass combat in D\&D tends to be a terribly overlooked part of the game. While it's expected for PCs to potentially join or own armies, the granularity of D\&D has been set to skirmish-level events, meaning that larger battles become a monster of paperwork and bookkeeping. The fun part of armies, such as tactics on the battlefield, city-taking, and dragonkilling is so buried by the raw nature of individually statted out armies that any system except for the most abstract is just not fun\ldots\ if you wanted to play Warhammer 40K, then you wouldn't be playing D\&D.

So here are some rules for abstracting mass combat. It's a mini-game inside regular D\&D that has been designed for simplicity and a minimum of bookkeeping. We've kept it as basic as a mini-game so that learning and playing it doesn't distract from the experience of playing D\&D. You'll really only be worrying yourself over this sort of thing at infrequent intervals. The player characters will still spend most of their time bringing down enemy monsters and assassinating dark lords. But having a running tally of how battles are proceeding around the players can be pretty entertaining.

\subsection{The Basics}

\ability{The Mass Combat Map:}{In the Mass Combat Mini-game, troop movements take place in 50' by 50' squares. For players wishing to play out a mass battle, they simply take standard miniatures and battlemats and draw out the map so that it reflects the new scale. Characters of Gargantuan or larger size take up their own squares, but its otherwise assumed that each square contains as normal individual characters can fit in such a space as normal when in regular formations. Medium sized PCs in a square take damage and are affected by spell effects that affect their square.}


\ability{Units:}{Each unit has a Move, HPs, Damage, and Morale, and they may have special abilities or attacks based on their race, training, or armament. Each unit also has a Leader with normal racial CR as a base, the advanced by CR equal to the normal damage of the unit. Each individual unit's stats will be discussed in the Building a Unit section, but for now we have a brief overview of the terms.}

Move is the number of 50' by 50' squares a unit can move each Combat Turn. Units that are mounted tend to have faster movement rates. Moving through difficult terrain takes more movement points, and sloped terrain takes more or less.
HPs represent a rough estimation of the number of casualties a unit can take before its morale breaks and the individuals in that unit begin to flee the combat. Heavily armored or tougher units have more HPs, while cowardly or lightly armored units will have less. Units with 0 or less HPs are Broken (or Destroyed, depending on your specific unit's attributes), and its individual members flee the battlefield and not longer counts a functional battlefield units for any purpose.

Morale is a number used to calculate your Army Morale Rating. Add the Morale score of each unit in your whole army together, then half it. The resulting number is your Morale Rating for your Army. As units in your army are destroyed, their Morale is subtracted from the Morale Rating of your Army in the same manner as HPs. At a Morale Rating of 0, the remaining units in your army separate from your army and flee the battlefield in an orderly fashion. At this point they are considered Uncontrolled units, and they leave the battlefield by the clearest and fastest route available. Uncontrolled units no longer follow orders from former Commanders, and may return to the home of their government for reassignment, turn to banditry, seek a mercenary contract, or attempt to form independent armies under a new commander, as following their nature and circumstances.

Damage is the overall amount of punishment a unit can inflict on another unit in melee combat. When a unit enters the square next to an unfriendly unit, both inflict damage on each other immediately, with no rolls. Certain Tactics can increase a unit's damage, such as Charging, Fortifying/Setting, and Rushing.

\subsection{Battles}

\subsubsection{Commander} A Commander is the overall leader of an army, and he has a Commander Rating which affects his ability to lead troops in battle and surprise enemy commanders with his battle tactics. A Commander's Command Rating is added to his Army's Army Morale Rating, and if he is killed this rating is subtracted from his Army Morale Rating as normal and control of an Army falls to the character in that army with the next highest Command Rating.

During the Battle Order, a Commander may move or give Tactics commands to a number of units equal to his Commander Rating + 1d4 in units each Turn.

\subsubsection{Battle Turns} At the beginning of each battle, assume Armies move simultaneously with each Commander moving one unit at a time until at least one unit can attack an enemy unit, then determine Battle Order.

Battle Order starts with the Army who is capable of attacking first, then descends in order for each Army that can enter the battle after that (uncommitted Armies simultaneously move last in the Battle Order until they commit to at least one attack). For example, an Elven Army and a Human Army is going into battle with an Orc Army, Gnoll Army, and a Necromancer Army. Since the elves can attack first due to the range of their bow attack with their Elven Longbowmen, they are first in the Battle Order. The Necromancer Army is second due to its position near the Elven Army, then the Human Army enters the battle third with their Shortbowman, then the Gnolls enter fourth with their Gnoll Levies due to their position on the battlefield and finally the Orc Army enters last due to poor positioning on the battlefield. (One may note that an army that is positioned well on a battlefield or has ranged attacks gains a slight advantage by having a better Battle Order. This is intentional.)

\subsubsection{Tactics} Certain tactics are not formal actions for a combat unit, and others are. Pincer maneuvers, battlefield deception or skillful maneuvering, taking advantage of higher ground or forests, etc., are normal consequences of well or badly played battles. Other tactics are commands that can be given to a unit that have special effects, and are noted below.

\subsubsection{Charge} A Charging unit gains several things from a Charge: +1 Move, +1 damage bonus in melee combat during their Battle Order, and suffers 1 point of damage at the end of any movement(a unit must move during a Charge).

\subsubsection{Rush} A Rush is when a unit enters another unit's square at the cost of 1 additional Movement. When units have done this maneuver, neither unit can move again until all enemy units in that square are Broken or Destroyed. Friendly units may enter or exit another friendly unit's square, but this action costs 1 movement each time.

\subsubsection{Fortify/Set} A unit in a building, ruins, or other fortified position like entrenched ground can Fortify, gaining Damage Reduction equal to the DR rating of the building(usually 1-3) and doing an additional +1 point of damage on their turn. Units that cannot Fortify due to their position may instead Set, and gain the +1 damage bonus.

A unit cannot Fortify/Set and move on the same turn. Some units gain additional abilities when they Fortify/Set.\\

\noindent \textbf{Base Terrain/DR}

\listone
\item Light forest, ruins, swamp, dense smoke: 1
\item Sturdy wooden buildings, Light stone buildings, dense forests: 2
\item Small Stone Keep, Heavy Stone building: 3
\item Stone Castle: 4
\end{list}

\subsubsection{Movement} Moving through different terrain has different costs, as shown below. If a unit cannot spend the required number of movement costs to enter a square, they instead pay what they can each turn until they have entered a square, but count as being in their old square until that time. The costs are below:

\listone
     \item Light forests: +1
     \item Heavy Forests/Jungle: +2
     \item Ruins/Very Rocky: +2
     \item Building: +3
     \item Castle: +4
     \item Down a Slope: -1(minimum of 1)
     \item Up a Slope: +1 to +3, depending on grade
     \item Swamp: +3
     \item Sand or Rocky Desert: +1
\end{list}



\subsection{Player Characters in the Mass combat Mini-Game}

Player characters have a special role in the mass combat mini-game, as they have access to abilities far beyond that of lesser troops. PCs wishing to lead a friendly unit must have a Commander Rating of at least 1 and a level equal to the CR of the Leader of that friendly unit. They may attach themselves to that unit, effectively assuming direct control of it in battle.

\subsubsection{Attacking a Unit in Combat} PCs fight units not by killing all the members of that unit, but by killing the Leader of that unit. Killing or rendering that Leader ineffective (like teleporting him away, petrifying him, etc.) will cause that unit to break. PCs must be in that unit's square to attack the Leader, and they automatically take damage from the enemy unit equal to its damage times 10. Units attached to a PC and who have use the Rush Tactic to enter the square with the PC take this damage instead of the PC.

PCs in a square next to an enemy unit take no damage. Ranged attacks that target a PC's square allow the PC a Reflex save for half equal to 10 + the ranged attack's damage. Spells or other effects that can target more than 50\% of a PCs square, or specifically target the PC, effect him as normal.

\subsubsection{Attacking a Unit with Spells or Effects} For every 10 normal HPs a PC can do to over 50\% of a unit's square, assume that the unit takes 1 HP of damage. Spells that don't do damage (like fear effects or other effects) but can effect more that half the members of unit do the spell's level in damage to the unit with no save.

\subsubsection{Building a Unit} Units have a Levy cost in GP to hire and train them, a Time cost to complete the training, and they assume the stats of their unit at the end of the training. Each unit is assumed to have twenty members of each race, and only races with at least an Int of 8 and a language can form combat units(unless they are the mounts).

\featnamelist{Raw Unit Stats:}
\listone
\itemability{HPs:}{Units gain permanent HPs for the following reasons:}
\listtwo
\item Racial HD: HD times 2
\item Medium armor: +1
\item Heavy Armor: +2
\item Elite Training: +1-3
\item Mounted: half HD of mount
\item Conscripted: -1
\end{list}

\itemability{Damage:}{}\listtwo
\item Base 1
\item BAB: +BAB
\item Str: Half Str bonus
\item Poorly Armed: -1
\item Mounted$^*$: +half damage of mount
\end{list}
$^*$ {\small When calculating Mount and rider damage, use the Mount as base damage if the Mount would have a higher base damage, then add half the Rider's damage.}

\itemability{Move:}{}
\listtwo
\item Base move: 1, +1 for every additional 30' of base movement
\item Mounted: Mounted units use the mount's movement to calculate this number.
\end{list}

\itemability{Morale:}{Morale is calculated like this:}
\listtwo
\item Base: equal to HD
\item Intelligence: Add Wisdom modifer of Race
\item Mounts do not add to morale in any way.
\item Conscripts: -1
\end{list}
 $^*$ {\small Mindless or undead creatures have a morale or "--," meaning that they do not contribute to Army Morale, and they do not stop fighting when an army is defeated}
\end{list}


\subsubsection{Levy/Time}

The cost to form a unit is equal to its damage times its HPs times 10. Add 100 for every extra ability of the unit (races that naturally have an ability like Elven Longbowmen and the Longbow do not incur this extra ability Levy cost).  They take a number of months to train equal to its Levy cost divided by 100 +/- the Int mod of the race.  The base cost to maintain and pay a unit is equal to its Levy Cost divided by 10 each month.

\subsubsection{Special Qualities}

\listone
\itemability{Lancer:}{Lancers don't take damage from the Charge Tactic.}
\itemability{Ranged:}{This is the tag for units with bows, spears, and other throwing weapons. A ranged unit's ranged attack does normal damage at its range increment, then -1 for each additional range increment. Only units using bows or thrown weapons calculate Str for the damage of this attack. Most units with ranged attacks can only use ranged attacks when Set (for example, Longbowmen have a Ranged 2/Set tag), and the Set Tactic extra damage is assumed to be calculated in this figure.}
\itemability{Damage Reduction:}{Some units have damage reduction from their race; this converts to Mass Combat damage resistance on a 5 to 1 ratio.}
\end{list}
