\chapter{Playing the Game}

\section{What's that Noise?! Playing at Low Level}

\desc{There is a reason that the XP charts in the DMG completely fudge character levels 1-3. That is because those levels genuinely don't have a good consistent rubric for how powerful things are. There are damn few first level PCs that wouldn't go down if they took a lucky crit from a kobold's small light crossbow, and a first level Wizard has a pretty reasonable chance of taking down an orcish warrior by hitting him with a club. At first through third level, combat really is anyone's game and it is strongly advisable that the PCs outnumber their foes in the majority of confrontations at this level of conflict.}

The TPK (Total Party Kill) is a very real concern for 2nd level characters, because the success or failure of actions is so very random. A run of bad luck can quite plausibly wipe out even a well-played low level team of adventurers quite easily and it is recommended that DMs use discrete encounters at these low levels in order to minimize the effects of having characters getting dropped by allowing the remaining characters to consistently revive fallen comrades.

\section{The Rigors of Command: Playing at High Levels}

\desc{A high level party isn't really ``adventuring'' in the traditional sense any more, or at least they probably shouldn't be. Instead, they are playing a whole different game -- a \emph{strategic} game. Characters who make it into the Epic landscape can in fact become gods according to long standing D\&D tradition. Along the way it behooves you to conquer and administer stuff in order to propel yourself to victory.}

\desc{More detail will be gone into in the Tome of Virtue, as the high level world is a really strange place. Almost all the source material from Arthur and Beowulf to Theseus and Ulysses involves characters who are somewhere between 1st and 6th level in D\&D terminology. Stories which involve a 10th level adventure are extremely rare. Perseus killed Medusa (CR 7), and Bellerophon killed Chimera (also CR 7), but they both pulled some fancy equipment and cheesy tactics to pull it off (Bellerophon seriously had a flying mount that was faster than Chimera and shot arrows at the beast until it died).}

If one insists upon continuing with powerful characters in an adventuring role, there is a primary conceit which must be embraced: all adventures must be timed adventures. A 14th level Wizard can, with sufficient preparation, kill any challenge in D\&D without exception. And while sitting around planning the perfect murder of a red dragon or the perfect heist of a major artifact is interesting as an intellectual exercise, there is no way that represents an ``adventure'' in the way we use that word to describe 4th level characters breaking into pantries and stabbing people in the face for money.