\chapter{Magic Items: Swag That You Brag About}
\quot{``No� \emph{This} is a knife.''}

\desc{Any man on the street with a few nasty scars and good tale or two can call himself an adventurer, but there are a few true tests that can determine the difference between a talented liar and the kind of person who considers fighting dragons a slow day at the office. It's not demonstrable skills, or nerve, or even a history of past accomplishments. It's magic items.}

\desc{I know that this sounds counterintuitive, but work it out for a second. Put a fighting guy with just better than average stats, some class features, and some HD out on the front line, and what do you have? Basically, you have a giant, which means ``NPC''. Without a magic weapon to bypass DR, good armor to avoid being clobbered, healing magic to recover for the next fight, and crazy extra effects to surprise an enemy like dimension dooring with the Cloak of the Montebank or reflecting a spell with a Ring of Spell Turning, you just don't have enough mojo to call yourself a PC. Monsters have bigger raw stats and better recharge times on their abilities, so if you don't have something extra you aren't going to be able to compete.}

\desc{Magic items are the true test of the adventurer because they say ``I'm trying to grow my power asymmetrically and I'm willing to do it by stealing it from other people who are also growing their power asymmetrically.'' Anyone can fire a bow at a manticore in flight, but only an adventurer is so concerned with power that he'll track that manticore to its lair and risk getting boxed in by a family of manticores just for the opportunity to root through its dropping on the off chance some would-be hero got eaten by the thing and a magical trinket or two survived passing through its innards.}

\desc{Some would called that ``greedy'', but in fact that's ``hardcore.'' Real adventurers are willing and able to risk their life on just the hope that their efforts will bring magical loot.... and its worth it. The more magical loot one gains, the more able an adventurer is to survive the next terrible risk that might offer magical loot. Heck, just holding onto any reasonable-sized pile of magical loot means that one is tough enough to face off against most people who would want that stuff.}

That being said, magic item creation and ownership is a big deal, and should not be the abbreviated (and broken) process that you see in the DMG. Here are some rules to make it sane and easy.

\section{The Core of Magic Item Design: Don't Do It Like Diablo}

Diablo II is a great game, but literally every single thing it does with magic items is bad for a table top role playing game.

\section{Magic Items with Class(ifications)}

It's all well and good to talk about ``Magic Items'' as a whole, but there really is a very big difference between piles of scrolls (which have a modest effect on a single adventure) and a flaming sword (which has a modest effect on all your adventures). Not as much as the writers of the DMG seem to think -- but it's certainly there. An item with ``unlimited'' charges is actually \emph{going} to be used a specific and finite number of times before the character stops adventuring, the item is destroyed, or the character starts using something else. While there is no specific limits to how many times you \emph{can} swing a sword, fundamentally there is a limit to how many times you are \emph{going} to swing that sword.

\subsection{Activation vs. Constant}

\desc{Walking around in a suit of magic plate assumes that as long as it's worn properly, then without any prompting on the part of the character the suit is providing an enhancement bonus to AC. It's the same with a Ring of Fire Resistance, an Amulet of Natural Armor, and a host of other items. Similarly there are items such as magical swords that can be used round after round generating their effects time and time again without rest or recharge. It's the same with wands, most rods, the vast majority of rings, and collapsible animated ice swans. In either case these items are Constant items. Item providing a Constant effect (or usable in a Constant fashion) must be specifically targeted by a \spell{dispel magic} to be affected.}

Other items need to be activated before they work. Scrolls and potions are classic examples, but a good percentage of magic items fall into that category. These are Activation items. Activation items have to be in some way prepped up before they are used. A scroll must be read and deciphered; a potion must be shaken up and opened. Any Activation effect can be dispelled in an Area \spell{dispel magic} or person-targeted effect (as appropriate).

\subsection{Ownership is a Privilege, Not a Right}

\desc{Several systems of magic item ownership have been attempted in the past. The current system is a pseudo chakra-based BS where magic power is limited by one's body parts where some items are dedicated to a specific body part (magic helmets like a Helm of Telepathy) and others are supposed to be put on the body but get to ignore this system (Ioun Stones are a classic example, as they float around your head and just give you some magic powers but you can have a dozen or a hundred doing that job and it's no problem). Other magic items generally sit in your pocket until you use them, and its assumed that your backpack is stuffed with them (staffs, wands, rods, most rings, scroll, potions, special-use weapons like \spell{ghost touch} swords, and about half of the wondrous magic items).}

One of the dumber parts of D\&D has been the tally sheets of items where determining the effects and bonuses on a single character starts to look like doing your taxes. That's lame and slows down the game, and together that's unacceptable. Since we have removed the GP and XP rules from magic items, which were previously the only limiting factor on magic item abuse (which we did because they didn't really work), we have instead have these new rules for magic item ownership:

\begin{enumerate}
	\bolditem{Eight Item Limit:}{Adventurers can have up to eight Constant effect magic items operating on their body at one time. Any items past that limit (8), and the most recent items won't work. This can be any combination of items, but available space on the body is a limiting factor, meaning that you definitely can't wear two sets of chain armor at the same time (no way to get two torsos), but you can wear several amulets (assuming you have a neck, which most oozes don't) or even two helmets (assuming you have two heads like an ettin).
\vspace{10pt}

	Carrying around Activation magic items is no problem though. You can have bandoleers of potions across your chest, a scrollcase full of scrolls, or a magic arrow hidden up every seam in your clothes and every body cavity, but only eight items can currently be providing Constant benefits.
\vspace{10pt}

	A constant item must be worn/used and working properly for it to count against the Eight Item limit, and activation items can only be used one at a time. For example, Tommy of the Twelve Magic Daggers can wear a constant effect magic armor, a constant effect magic cloak, and five constant effect magic rings and still throw/activate his daggers one at a time in a round (assuming he can throw more than one each round), but if he tried to use two at a time with Two-Weapon Fight (for example: to benefit from qualities like Defending), then one of those daggers is not working and is basically a non-magical dagger. Some situations may arise where it is difficult to decide if a character is exceeding his limit; and in those cases, use your best judgment (meaning that if you are a DM, be consistent). For example, Tommy might be holding a magic longsword by the blade in his hand, so it's not ``active'' since he can't take AoOs with it and get its bonus and its not providing him with any Constant benefit.}

	\bolditem{True Ownership:}{A person has to willingly put on a magic item and intend to activate it for it to count as active. That means that clever people can't trick you into putting on weak magic items so that your good magic items won't work when you try to use them. Unconscious or helpless characters can have items activated on their behalf (remember that in D\&D unconscious creatures are always ``willing''), so you can put a Ring of Regeneration on an unconscious buddy or put Dimensional Shackles on a sleeping wizard. Command word and spell completion items cannot be activated on someone's behalf (though you are welcome to use them on another character by dint of pointing the wand at your opponent and shooting lasers at them as normal).
\vspace{10pt}

	The flip side of this is that when you put an item down, it still counts as being one of your items for a period of time. This means that when you throw your magic spear it retains any benefits that are dependent upon your level while it is arcing trough the air into the dragon's chest; and it also means that it is not practical to pull a magic skirt off in the middle of combat and replace it with some really cute bike shorts. That's actually a good thing, because while if you're specifically playing Final Fantasy X 2 D20 it is setting appropriate to change your clothes in the middle of combat, in all other settings that sort of thing is just really dumb. Once it leaves your person, a constant magic item generally stops being one of your eight in a d4 minutes. If you're actually dead, your magic items stop counting as being yours the next round.
\vspace{10pt}

	Cursed items are the same. You have to try to use a cursed item before it can affect you. Otherwise, you can just keep it in a box labeled ``Cursed sword: Do not use for stab-ination.''}

	\bolditem{An End to Bonuses:}{Andy Collins talks a lot about the ``big items'' that players need to get in the door at high levels. Mostly swords and shields with bonuses on them. And while he is correct that people \emph{do} need them, I personally think that constantly taking up time worrying about getting another uninteresting ``slightly more magical sword'' is bad for the game. The solution is truly that for magic items to fulfill their duty within the game without being really annoying, they just have to scale by level. So the ``+2 Sword'' is dead. Now there's just a ``Magic Sword''. If you happen to be 6th level when you use that sword, it'll be +2.}

	\bolditem{Artifacts have a Level:}{What makes Artifacts special? Mostly it's that they are a source of power that is completely asymmetric and well outside what the user could be ``expected'' to have. This is represented by an artifact simply being a magic item that has a level on its own time. That means that the first level farmer's daughter who picks up Excalibur (an artifact with an inherent level of 15) gets all the benefits that she would had she actually been 15th level herself (a +5 enhancement bonus, being king of England, the whole deal). A character who holds an artifact of a lower level than herself still treats it as a magic item of her level -- the Artifact's level is a minimum, not a maximum.}
\end{enumerate}

\subsection{Wanna Take Some Body Slots?}

The slot system of traditional D\&D is more than a little bit insulting and carrying it over into this document would be a tragic failure of our design goal to make things not be like Diablo II. So yes, if you want to have every single one of your eight items be a ring, or an ioun stone, that's fine. Heck, you really could plausibly wear eight rings on one hand, there are people who do that sort of thing. If it's really important that you use three different magic crowns, we welcome you to run around calling yourself \emph{The Thrice Crowned King}. Nevertheless, items do have classes that they fit into fairly neatly:

\begin{enumerate}
	\bolditem{Wielded Items -}{These are held in a hand and brandished, swung, or otherwise triggered to activate their power.}
	\bolditem{Worn Items -}{These are placed somewhere on the body in order to unleash their power. While it is possible for someone to wear multiple sets of clothes, or armor over clothes, or even armor over other armor, only the heaviest armor counts as the armor you are wearing for purposes of AC, special abilities, etc.}
	\bolditem{Miscellaneous Items -}{These are items that are used in some other arbitrary way. Their power continues even when not held or worn, which is good because a lot of these items are things like thrones, golems, or crystal spheres that simply cannot be placed on the body at all.}
\end{enumerate}

% Magic Item Creation
\section{Magic Item Creation}

\desc{Building a magic item is a big deal. It is a way to expand one's power and a way to transfer power to your lessers, and in many ways the life of an adventurer revolves around the acquisition of magical loot. If magic item creation is too easy, adventuring is less fun, and if its too hard then people won't do it and resent the system and DM.}

\desc{We know that the current rules don't work. GP and XP costs are things that have little meaning in even a low-level game, and players are notorious for finding ways around them by taking metagame classes like the Artificer, by having cohorts pay those costs, or even by giving morals the finger and having mindcontrolled captured spellcasters do it. That's before we even get to \spell{wishes}, powerful outsiders, or craziness like the Dark craft and soul rules.}

\desc{There is one thing that hurts characters: time. Adventures and stories happen along a timeline, and players may or may not be able to stop during an adventure to build just the right item for an adventure. Even ``downtime'', the time between adventures, is limited because powerful characters attract powerful enemies and predators. Heroes that say ``we'll just take a year off and make magic cloaks for everyone'' are basically saying ``we'll sit in the open and let our potential and actual enemies pick the time and place for any battles.'' DMs can throw enough intrigues in someone's way during that time so that before the first cloak is built that the campaign is over.}

Creating magic items just requires time. There's work that goes into enchanting a sword, forging a blade, smelting the steel, mining the ore, and all that just takes time. If a character is really dedicated, he really seriously can wander off into the hills collecting reddish stones and then heating them up until iron comes out and then hammering the molten metal into a blade and then enchanting it with his power and walking out of the hills with a magic sword. Various portions of this can be expedited by, for example, \emph{hiring other people} to do a lot of that -- so a character can reasonably expect to throw down gold and buy himself a lot of that time back. But if you just have time; time will suffice. Exactly what magical goods are needed or helpful in magic item creation is highly variable campaign to campaign.

\listone
	\item \textbf{Questing for Reagents}\\It is a classic story for those fantasy settings that have on-camera magic item creation that characters must go quest for magical ingredients they need to make whatever the hell it is that they want to make
\end{list}

\subsection{Building a Better Magic Item: the Minor Magic Item}

A Minor Magic Item is one which can be produced in quantity by NPC apprentice factories and can thus be in some sense standardized or expected to exist in major city bazaars. Most minor magic items just provide some sort of unimpressive numeric bonus. The magnitude of that bonus is dependent upon the level of the character who is using that magic item. The rate at which the bonus scales to level varies depending upon what the item is giving a bonus to, and when magic items would provide a fractional bonus always round that fraction up. There are no caps on any of these bonuses. If you're a 19th level guy your sword simply provides a +7 enhancement bonus and that's fine. You're 19th level and you don't even really care.

\listone
	\bolditem{Enhancement Bonus to Weapons ::}{+1/3 per character level.}
	\bolditem{Enhancement Bonus to Armor ::}{+1/3 per character level.}
	\bolditem{Enhancement Bonus to Attributes ::}{+1/3 per character level.}
	\bolditem{Resistance Bonus to Saving Throws ::}{+1/3 per character level.}
	\bolditem{Competence Bonus to Skills ::}{+1 per character level.}
	\bolditem{Energy Resistance to any Energy Type ::}{+1 per character level.}
	\bolditem{Deflection Bonus to AC ::}{\fourth per character level.}
	\bolditem{Enhancement Bonus to Some Other Thing (Natural Armor, DR, SR, whatever) ::}{+1/3 per character level.}
\end{list}

\desc{Non-standard bonus types, or as we like to call them around the office: \emph{bullshit} bonus types do not exist. No, you can't have a Sacred Bonus to your AC or an Insight Bonus to your skills. That stuff is straight up broken and will push characters right off the random number generator. If all of your eight magic items are providing a bonus of some sort, they most definitely should not be providing a bonus to the same number -- that sort of thing really does make the d20 system fall apart.}

\desc{Minor Magic Items which do not provide a numeric benefit usually reproduce the effect of a spell, and are caster level 5 or less. A Minor Magic Item may potentially be traded in the turnip economy. It is conceivable that a man might trade a wand of \spell{cure light wounds} for food or shelter directly. Nonetheless, these items are much more commonly traded for gold, and anything more powerful than a Minor Magic Item is actually less than worthless in the turnip market -- a \magicitem{Frost Brand} or \magicitem{Stone Sphere of Shaz} is really going to draw more fire for a peasant than it's worth.}

More powerful magic items begin with a Minor Magic Item base and layer additional abilities on top. In this way a Sword of Sharpness always provides the basic level appropriate enhancement bonus to attack and damage even while it is chopping the heads off of dudes.

\subsubsection{Building a Better Magic Item: the Magic Weapon}

\desc{Generally speaking, magic weapons start with the basic minor magic item chassis: "Weapon with an Enhancement Bonus" and items more powerful weapons also have an ability. There are two kinds of magical weapon ability: Spell-Like Abilities and Supernatural Abilities. An example of the first type is a Rod of Fire and an example of the second type is a Vorpal Sword.}

A Spell-like ability is just a spell that having that weapon allows you to use. Using this spell-like ability is a Standard Action, so Quickened Spells aren't particularly interesting.

\abox{Behind the Scenes: What Spells Can Rods and Swords use?}{D\&D has literally thousands of \emph{pages} devoted to spells and it is entirely impractical to go through the list and find all the spells that would be appropriate from an activation ability for a magical rod or staff. Instead, here are some basic ideas of things which are \emph{not} a good idea to put into weapons:

	\listone
		\bolditem{Long Casting Times:}{Spells like \spell{major creation} can make stuff like Adamantine Boxes, which is all fine and dandy until you start making them in combat time by having them be used as a spell-like ability. Then it's suddenly battlefield control with no save allowed and that's just messed up. So while having a Rod of Summoning that allows you to throw down a Fullround spell like \spell{summon monster} whenever you feel the urge is fine -- sources of spells like \spell{move earth} and \spell{planar binding} are deeply problematic.}
		\bolditem{Swift or Quickened Spells:}{\spell{swift fly} is a really crappy spell except for the part where it's castable with a swift action. Even then it's not that great. When used as a Standard Action, it's just crap.}
		\bolditem{Juggling Spells:}{This last category is by far the hardest to nail down, because it isn't precisely quantifiable. But a spell effect that delays an opponent is really a crap tonne more effective if it is repeatable time and time again by autofiring the go button on a magic rod. Spells like \spell{frost breath} and \spell{color spray} are amazingly effective anyway, and if you can just throw them every round you go to straight up unfair territory.}
	\end{list}}

Supernatural abilities, on the other hand, are just things that your weapon does. Like a monster ability, your weapon simply has some effect going all the time. In many cases, this involves inflicting a status effect on enemies struck with the weapon. Status effects will be inflicted on the following circumstances:

\listone
	\item A target is struck at least once during a round (so figuring out some way to scam lots of attacks doesn't give extra statuses).
	\item The target fails a saving throw. The DC is 10 + \half the wielder's character level + the wielder's Charisma Modifier.
\end{list}

Here are some supernatural weapon qualities:

\listone
	\item \textbf{Lesser Qualities:}
	\listtwo
		\bolditem{Defender -}{A defending weapon moves itself to intercept attacks made on the wielder. While wielding a defending weapon, the character has an armor bonus of 5, enhanced by the enhancement bonus of the weapon.}
		\bolditem{Dispelling -}{A weapon of dispelling destroys magic. Anything struck that fails a Willpower Save is targeted by a targeted \spell{dispel magic}, with a dispelling check of d20 + wielder's character level (no cap).}
		\bolditem{Flame -}{A flaming weapon sets things touched by it \emph{on fire}. A victim who fails a Reflex Save is on fire and will suffer a d6 of fire damage every round until they extinguish themselves.}
		\bolditem{Ghost Touch -}{A ghost touch weapon spans the material and ethereal planes. It can be wielded by any standard, incorporeal, or ethereal being and can be used to attack any standard, incorporeal, or ethereal being with no miss chances due to the difference (if any).}
		\bolditem{Terror -}{A weapon of terror strikes fear into the hearts of its foes. A victim who fails a Willpower Save becomes \condition{shaken} for an hour. This is a [Fear] effect.}
		\bolditem{Thunder -}{A thundering weapon makes a whole lot of noise. A victim who fails a Fortitude Save becomes \condition{deafened}, and an object struck by a thundering weapon has its hardness ignored.}
		\bolditem{Time Distortion, Lesser -}{A weapon of lesser time distortion cuts time away from the target. A victim who fails a Willpower Save becomes \spell{slowed} for 5 rounds.}
		\bolditem{Berserking -}{A berserking weapon causes the wielder to go into a red rage of mindless fury. Whenever the user makes an attack with the weapon, the user is immune to mind affecting and fear effects for three rounds. However, during this period the character also cannot cast spells or activate magic items.}
	\end{list}
	\item \textbf{Moderate Qualities}
	\listtwo
		\bolditem{Cursed -}{A cursed weapon cannot be gotten rid of. A character who uses a cursed item will find that it continues to count against her 8 item limit for some time after being set aside, and it can be willed into her hand as a free action regardless of distance. Even if it is destroyed, the cursed weapon reforges itself once every day and continues to count against the wielder's item limit until a successful \spell{remove curse} is used to sever the connection.}
		\bolditem{Disruption -}{A disrupting weapon damages the necromantic animating force of the undead. An undead victim who fails a Fortitude Save is instantly destroyed.}
		\bolditem{Frost -}{A frost weapon freezes things quite severely. A victim who fails a Fortitude Save becomes \condition{fatigued}, normal fires are extinguished, and liquid objects freeze out to a 5' radius. Within 5 feet of an unsheathed frost weapon, the temperature cannot rise above \emph{cold}.}
		\bolditem{Lifestealing -}{A lifestealing weapon damages the souls of the living. A victim who fails a Willpower Save gains a negative level.}
		\bolditem{Planar -}{A planar weapon ironically is infused with the power of the Prime Material and is named thus because it's a good thing to have when traveling the planes. When a victim who is not a native of the Prime or whatever plane you happen to be on fails a Willpower Save it is instantly banished to its home plane, from which it may not leave for 24 hours (treat as a \spell{dimensional anchor}). In addition, an outsider victim must make a Fortitude Save or be \condition{dazed} for 1 round, regardless of what their native plane happens to be.}
		\bolditem{Sharpness -}{A weapon of sharpness cuts stuff into pieces. A victim who fails a Fortitude Save loses a limb (chosen at random), and an object struck by a weapon of sharpness has its hardness ignored. This enhancement is only available for sharp weapons, other weapons should use Withering instead.}
		\bolditem{Sun -}{A sun weapon sheds tremendous amounts of light. A victim who fails a Reflex Save becomes \condition{blind} for 1 round. Such a weapon sheds more light than normal, and is surrounded by a \spell{daylight} effect when in use.}
		\bolditem{Withering -}{This quality is exactly like "sharpness" except that the special effect is that limbs wither and objects crumble. It is used for blunt weapons.}
		\bolditem{Wounding -}{A wounding weapon causes brutal and horribly bleeding wounds. Damage caused by a wounding weapon is vile physical damage even if the victim has Regeneration. A living victim who fails a Fortitude Save becomes \condition{staggered} for one round.}
	\end{list}
	\item \textbf{Greater Qualities}
	\listtwo
		\bolditem{Petrification -}{A petrifying weapon causes living tissue to transform into stone. A living victim who fails a Fortitude Save is \condition{petrified}.}
		\bolditem{Ruin -}{A ruinous weapon destroys pretty much anything. A ruin weapon ignores all hardness, DR, and resistance to critical hits of any target it strikes.}
		\bolditem{Soul Prison -}{A soul prison weapon absorbs the soul of any enemy slain with it. A victim who is dropped by a soul prison weapon has their soul immediately drawn into the weapon, where it remains until used. A soul prison weapon can hold up to nine such souls at a time, and not even a \spell{wish} can restore the life of a foe whose soul is therein contained. Nominally there is a Willpower save is involved, but since a dropped foe is considered "willing" that doesn't normally come up. This is a necromantic effect.}
		\bolditem{Time Distortion, Greater -}{A weapon of greater time distortion cuts time away from the target. A victim who fails a Willpower Save becomes affected by \spell{temporal stasis} for ever.}
		\bolditem{Vorpal -}{A vorpal weapon kills things outright with a "snicker-snack" noise. A victim who fails a Fortitude Save is killed, this is a death effect.}
	\end{list}
\end{list}

\subsubsection{Magic Ammunition}

\desc{A magic arrow is indeed a special thing. The only kind of magic arrow that doesn't make us feel really bad about ourselves is the Spell Arrow, so that's the only one that exists. Every magical arrow (or crossbow bolt, or whatever) has one spell in it which is chosen when it is made and which will be cast when it is fired. A spell arrow is not recoverable after the fact because the spell only goes off once. In order to actually get a magical arrow to ``go off'', you have to spend a standard action firing it. Otherwise it's just an incredibly expensive arrow.}

Magical arrows have the spell go off in whatever way would be most awesome looking. So if you fire an arrow which contains a touch ranged spell like \spell{cure serious wounds} or \spell{incite love} then the spell takes effect on whoever gets hit by the arrow. On the other hand, if you have a spell with a cone or line area of effect like \spell{lightning bolt} or \spell{color spray} it starts the line right in front of the bow. Bursts or Emanations come from wherever the arrow lands, and Personal or 0-range spells can't be made into Spell Arrows at all. In any case, the arrow itself is completely consumed by this process and doesn't do any actual damage (so curative arrows aren't as stupid as they might sound). Hitting a specific target with a Spell Arrow is a ranged touch attack.

\subsubsection{Magic Armor, Clothing, and Accessories}
\vspace*{-8pt}
\quot{``He's the man with the magic pants.''}

\desc{Heavy plate armor, racks upon racks of Mr. T style gold chains, shiny pants, and magic belts, these are a small set of examples of the crazy crap that people wear in the D\&D universe. The only difference between wearing, for example, a bunch of gold chains and a sleek set of leather armor is that the leather armor counts as \emph{armor} and has a tendency to provide some sort of Armor Bonus, Armor Check Penalty, and level appropriate bonuses (see Races of War). The gold chains just make you look like a circa-1986 rap star. But basic bonuses aside, all such items are simply a minor magic item unless they have some special ability above and beyond the standard level appropriate effect.}

\desc{Special abilities on such items can be spell-like or supernatural, exactly as per weapons. The activated spells on a cloak or a belt function exactly like the activated spell-likes provided by magic weapons.}

Here are some supernatural Worn-item qualities:

\listone
	\item \textbf{Lesser Qualities:}
	\listtwo
		\bolditem{All Around Vision -}{Iconically placed upon helmets and Spot-bonus items, this enhancement gives the user the ability to see in all directions, preventing enemies from flanking.}
		\bolditem{Aquatic -}{Iconically placed upon any worn item, this enhancement gives the user the aquatic subtype, allowing them to breath water and swim easily.}
		\bolditem{Dark Vision -}{Iconically placed upon helmets and Spot-bonus items, this enhancement gives the user the ability to see without light, as darkvision out to 60'.}
		\bolditem{Tremorsense -}{Iconically placed upon boots and Listen-bonus items, this enhancement gives the user the ability to detect things within 30' who are touching the ground as with tremor sense.}
	\end{list}
	\item \textbf{Moderate Qualities:}
	\listtwo
		\bolditem{Blindsense -}{Iconically placed upon helmets and Listen-bonus items, this enhancement gives the user the ability to detect things within 60' as with blind sense.}
		\bolditem{Madness -}{This enhancement surrounds the user the with maddening trills and whispers, causing all sane creatures within 10 feet of the user to have to save vs. a \spell{hypnotism} effect each round that the item is active and uncovered.}
		\bolditem{Spell Resistance -}{Iconically placed upon protective items and cloaks, this enhancement gives the user Spell Resistance of 10 + Character Level. Spell Resistance from multiple items with this enhancement do not stack.}
		\bolditem{Telepathy -}{Iconically placed upon helmets and Sense Motive-bonus items, this enhancement gives the user the ability to silently communicate with any creature which has a language out to 100' regardless of line of effect.}
	\end{list}
\end{list}

\abox{Armor Bonuses and Natural Armor Bonuses}{Yes, Armor Bonuses and Natural Armor Bonuses stack, but they don't 100\% stack. If you have both an Armor Bonus and a Natural Armor Bonus, you only benefit from half of the smaller bonus (round up). So if you have a +8 Armor bonus and a +5 Natural Armor Bonus, you are getting a total of +11 from Armor and Natty Armor, not +13. The reason for this is because Natural Armor gets \emph{very large} on a number of creatures. Originally this was because writing in a big natural armor bonus is really easy and gives level appropriate overall bonuses for the stuff in the Monster Manual, but when you mix in regular armor it pushes defenses straight off the random number generator.}

\subsubsection{Magic Rings}
\vspace*{-8pt}
\quot{``In brightest day, in darkest night\ldots''}

There is nothing special about Rings. At this point there is enough fantasy material available that there are people deeply immersed in the genre who have never read the Nibelungenlied or Lord of the Rings. When Arneson and Gygax made D\&D back in the day, LotR really was primary inspiration and the natural result was to put rings on some sort of whacky pedestal. Well, nowadays we have people for whom the iconic Magic Item of Vast Power is a lamp (Aladdin), a gem (Dark Crystal), an orb (Castle of Llyr), or whatever. So a Ring is just like any other piece of clothing, save that it rarely provides an enhancement bonus to armor.

\subsubsection{Constant Miscellaneous Magic Objects}

There are a number of objects in D\&D land that are neither worn nor wielded and yet count as constant items. Crystal Balls, Handy Haversacks, and Braziers that call fire elementals are all powerful items that do count against a character's eight item limit. What they don't do is actually stay connected to the user in a physical sense between uses. In order to use one of these items, one must \emph{attune} it, at which point that item is connected to the character who did so. Attuning such an item takes 15 minutes, and it takes that long for it to stop being attuned as well. It takes an act of will to make a magic item of this sort stop working for you, and this act of will can be taken either by you or someone who holds the actual object. So if someone snags your decanter of endless water,

\abox{Behind the Scenes: Attuning Crystal Balls.}{When you draw a flame tongue it bursts into flame immediately upon leaving its sheath -- granting a level appropriate bonus to attack and damage while setting stuff on fire. However, the same does not happen when you uncover a crystal ball. And the reason for this is honestly that items like collapsible bridges, bags of holding, and iron flasks are almost never used in combat time and yet they \emph{do} have a serious impact on your success or failure in an adventure.

It takes longer to swap these objects into and out of your bat cave simply because it is assumed that when you would be doing this you actually have more time to swap things in and out. In fact, it might be pointed out that it takes precisely as long to attune such an item as it does to fill an open spell slot on the fly -- that's not an accident.}

\subsubsection{Building a Better Magic Item: Intelligent Items}
\vspace*{-8pt}
\quot{``Hello computer!''}

\desc{In every edition of D\&D, the intelligent item has been listed as something that happened quite frequently. Seriously, even in the 3.5 DMG it says that fully 1\% of all Amulets of Health and Rings of Featherfall have intelligence. Were you to actually roll that up for every item you found it seems a virtual lock that every campaign would have one or more Intelligent Items in them. Since the vast majority of campaigns include \emph{zero} talking swords rather than the 1-5 expected by purely random chance, it seems extremely clear that something is wrong with the way Intelligent Items have been handled in the last 40 years of D\&D.}

\desc{An Intelligent Item is like having a cohort, and if it is the same level as you are that's really unbalancing to the game. While previous editions have tried to keep track of ego points, we're going to try to make this as simple as possible: An Intelligent Item is a Sorcerer who happens to be a dagger or a pair of shoes. Like any Sorcerer, an Intelligent Item has a character level, and if that character level is more than 2 less than your character's level, it \emph{will not be your cohort}.}

\desc{And that's it. An Intelligent Item is "just" a magic item that happens to have one or more levels of Sorcerer, and an Int, Wis, and Cha. If it is within one step of alignment of your character, and is at least two levels lower than your character, and it is attuned as one of your eight items, it will work with you -- casting its spells on your behalf. An Intelligent Item never needs to worry about somatic components, which is just as well because a lot of them don't have moving parts.}

That being said, an Intelligent Item is still an extremely powerful, game altering item. An extra spiderweb cloak that is throwing down \spell{web} in pitched battles can make the difference between life and death even at very high levels.

\subsection{The Appearance of Magic Items}
\vspace*{-8pt}
\quot{``Don't touch that sword.''\\
``Why? Because it's on fire? Because it has glowing runes?''\\
``Because the glowing runes say `Don't touch this sword.' ''}

\desc{Magic items do not normally require a casting of \spell{detect magic} to uncover. The DC of an appraise check to determine that something is in fact magical is 20 \emph{minus} the object's caster level. A powerful item bends space around it and glows with unearthly soulflame and such and really can be noted as magical by the untrained eye. But what exactly a magic item \emph{looks like} is contingent upon who made it and what they made it out of. Broadly speaking, the magic items made by the Drow really are black and covered with spider motifs; the magic items made by the Hobazad Khanate are generally lacquered in red and black with decorative leafing of gold and brass; the artificer mages of Bladereach make their magic items by etching them with hydra saliva so they look all melty and marbled.}

\desc{Minor Magic Items of any sort can usually be identified by regular people who are familiar with the culture which produced them. If you're a Drow you've \emph{seen} the cloaks of resistance that the tailors in your society make. You might even own one. It's really not any kind of mystery to you.}

Artifacts of course, follow their own set of rules. Some artifacts are instantly identifiable as powerful magical objects by people remotely in the vicinity (good examples of this are the Rod of Orcus and the Machine of Lum the Mad), while others really do adequately disguise themselves as mundane, even commonplace items (good examples of this second type are Aladdin's Lamp and the Pillowcase of Storms).

\subsubsection{Iconic Forms}

\desc{Let's face it, magic items are more fun when they come in recognizable forms. See a wizard waving around a stick and knowing that its a wand is more fun than trying to guess the effects of a glowing stone in his hand. That being said, here are additional rules to bridge the gap between our creation system and 3.X D\&D.}

Iconic Form bonus: Any item made in both its iconic form (ring, wand, sroll, etc) and enchantment as shown in the DMG or other published source recieves can be created as if it was -2 its normal caster level after creation. This means that if you make a Cloak of the Manta Ray rather than a Ring of the Manta Ray, it takes you the amount of time it would take for a 7th level item instead of its normal 9th level, and it counts as a 7th level item for item creation limits. Note that this does mean that casters can create iconic items by using a lower caster level (so a 7th level caster can create a Cloak of Manta Ray, but not a Ring of the Manta Ray), assuming they can cast (or have cast) the required spells.

% This subsection was moved from later on in the book, because it definitely
% belongs under the "Magic Items" section.
\subsection{Disposing of Magic Items}
\vspace*{-8pt}
\quot{``You're going to have to throw The Ring into Mount Doom. Probably those pants as well.''}

\desc{Magic items are \emph{really dangerous}. Not just to use, but also to leave lying around. Or to destroy. Really anything you happen to do or not do with magic items carries significant consequences down the line.}

The Bat Cave or Sword Rack is a relatively simple storage system for magical objects, and works fairly well.