\chapter{Created Monsters: Forged and Bred}
\vspace*{-8pt}
\quot{``It's alive! Or at least animate\ldots\ it's not an object anymore, that's my point.''}

There are three entire types of creatures in D\&D that are to one degree or another created. The obvious one of course is the Construct. It's a creature which was never alive and created by sorcery. Well, most of them are like that. The Flesh Golem is kind of hard to explain actually, but whatever. The point is that every Construct is \emph{constructed}. That's the whole point. And of course there are a lot of Undead that are pretty much the same thing except that they are animated with Negative Energy channeled into them. There's a lot you can say about those guys, and we actually \emph{did} that in the Tome of Necromancy. So we aren't touching that one here. The other one of course, is the Vermin. In D\&D land, ``Vermin'' doesn't mean anything vaguely approaching its meaning in natural English. Rats and cockroaches are vermin because they live in your pantry and poop on your food stores, but they aren't \emph{Vermin} because they aren't enormous biological constructs that mindlessly follow the programming planted in them by an ancient race of long departed mage kings. Really. That's what the Vermin type means in D\&D land. Actual giant insects are just animals in the same way that dire toads and weasels are animals.

\section{Vermin: Remnants of a Fallen Empire}
\vspace*{-8pt}
\quot{``Great holes secret are digged where earth's pores ought to suffice. Things have learnt to walk which ought to crawl\ldots''}

\desc{Ants track by smell and follow trails left by other ants and bees see deep into the ultraviolet spectrum and perceive a beautiful tapestry of gorgeous colors that escape the eye of the man and the mouse. And when dealing with Vermin type creatures that all means precisely \emph{nothing}, because Vermin in D\&D don't do any of that. It's not because the scent ability was ``left off'' the Monstrous Ant description, it's because the Ant described in the Monster Manual genuinely doesn't have a good sense of smell. It does have Darkvision out to 60 feet like an outsider or a Construct, and that's not an accident either despite the fact that Earthly ants really demonstrably don't do that regardless of size.}

The Monstrous Scorpion isn't a super sized scorpion \emph{at all}. It has a set of abilities which are on the face of it completely bizarre from the context of what actual scorpions do, because it's actually a living construct created by a long fallen empire for use in war. That's why it's immune to hallucinatory poisons and can see in perfect darkness. It's actually created from biomass by powerful magic and not by the interaction of natural and magical mutation across a thousand generations and a harsh selection process hastened by unpredictable climate and predation by manticores. The Vermin have a couple of neat things going for them which is why they were created as war machines in the first place:

\listone
	\bolditem{Mindless -}{Unlike actual or even giant spiders, the \emph{monstrous} spider has no mind at all. It cannot be influenced with magic or confused with poisons. It can't even be detected with \spell{detect thoughts}.}
	\bolditem{Brainless -}{Vermin are subject to critical hits because they have segments and organs, but they don't have any \emph{brains}. That means that they can be blinded, but not killed, by decapitation.}
	\bolditem{Darkvision -}{Vermin can see even in complete darkness, making them quite useful in cave fighting.}
	\bolditem{Aggressive -}{The vast majority of predators will retreat from battles where they are presented with even a chance at serious injury. Yet Vermin fight until they are dead. That's a really bad plan for an individual or even a species, but it's \emph{great} for a battle platform.}
\end{list}

\subsection{Who made the Vermin?}
\vspace*{-8pt}
\quot{``They did not \emph{know} that steal marks flesh, and they did not \emph{know} that flesh does not mark steal. In their ignorance they continued to do one task after another in the old ways. They did not \emph{know} what we \emph{know}.''}

Vermin come from the before time. The time when metals were not made and words were not written down. It's quite a feat of construction talent and a testimony to the power and ingenuity of these ancient flesh crafters that these devices are still running, still attempting to fulfill their programming to this day. The answer is not known in the days that D\&D is normally set. They are a product of a bygone age and their origin is a mystery to all but the Aboleth and the memory fish are being \emph{extremely quiet} on this subject. And yet, their conspicuous silence is probably more telling than anything they could possibly say. The Vermin were constructed during the days when aberrations ruled the world, and they were quite obviously designed to fight against aberrations.

\abox{Getting the Program}{All Vermin have a program that they follow at all times, usually involving a spiral search pattern in groups of one to six until they encounter a creature, at which point they will attack it until it is dead. If confronted with more than one type of creature, they will target them in the following order:
\listone
	\item Any Aberration (except their specific non-targeted group)
	\item Any Humanoid
	\item Any other moving creature they can detect
	\item Anything especially edible
\end{list}\vspace{2pt}

This behavior is entirely comprehensible from the standpoint of the wars in the before time -- the Ilithid and Aboleth both sent slave troopers to their death by the millions in their quest for world domination.}

\desc{Individual groups of Vermin will usually have one type of aberration that they will not ever attack. It may be an entire race of aberrations (such as Kopru or Neogi), or it may be a specific clan of aberrations (such as the Aboleth spawn of the Great Mother of the Howling Wells, or the Ilithid of the Tallow Halls). In any case, determining the type of Aberration that is completely safe from any group of Vermin can be done by observing the markings on the beast. Extracting that information is a DC 30 Knowledge Nature check.}

Vermin eggs persist apparently indefinitely and are produced by the hundred score. A starving Vermin cocoons itself and goes into a state of hibernation so deep that it is essentially mummified. When in the presence of magical auras, the eggs of Vermin progress steadily towards hatching, and the cocoons burst forth their contents. Thus it is not weird or unexplainable for areas that recently have been subjected to incursion by adventurers or mind flayers to spontaneously develop invasions by tiny monstrous centipedes or giant cocooned spiders.

\subsection{Vermin Alchemy}
\vspace*{-8pt}
\quot{``The old ways are the good ways.''}

Vermin cannot think for themselves, nor would they have been better at their job given that ability. So it is not surprising that one can severely adjust the behavior of Vermin through the use of chemicals and sounds. Identifying the sounds and smells that a particular group of Vermin will respond to is difficult (requiring a DC 30 Knowledge (Nature) check), but actually producing them is not particularly. Here is a list of possible behavior modifications one can achieve and the Perform or Craft (Alchemy) check required:

\listone
	\bolditem{DC10 - Rampage}{It's a very simple behavior modification to cause a rampage. The spiral search pattern ends entirely and all affected Vermin take off in a random direction and move at full speed or until their path is blocked by a creature.}
	\bolditem{DC15 - Ignore}{There are chemicals that cause Vermin creatures to simply ignore}
	\bolditem{DC20 - Attack}{}
	\bolditem{DC20 - Shut Down}{}
	\bolditem{DC35 - Command}{}
\end{list}

\section{Constructs: Durability at a Price}

\desc{Like the Undead, the Constructs suffer tremendously from the fact that they have been over generalized. It is of course thematically appropriate for a Golem to be tireless and work day and night on whatever its last command was for as long as day follows night and night follows day. But it is also thematically appropriate for a clockwork beast to wind down and ``pass out'' as it continues to work long or strenuous schedules. Similarly, while it is fine and more than fine for an implacable lump of animated steel to be immune to critical hits, the very idea that there aren't key locations on a geared robot or a colossus given life by a mystical forehead rune is patently ridiculous. The construct type, therefore is filled to the brim with stuff that has no business being there, and this harms the game. The immersiveness of the story is depleted when players cannot rationally deduce what effects a being is resistant or vulnerable to, and anyone who's ever slapped washing machine or tripped over a playstation knows that there's no excuse for a machine to be immune to stunning.}

So here it is, the Construct Type. Pared down to the things it should actually do. Remembering of course that the Type itself should contain only those effects that one would want to be a universal law for all constructed beings, rather than rules one could imagine being situationally appropriate for one construct or another:

\listone
	\bolditem{Low Light Vision:}{Sees twice as far in limited illumination.}
	\bolditem{Dark Vision:}{60'}
	\bolditem{Poor Healing:}{Constructs can be healed by any of a number of means but do not heal for periods of rest. A construct's daily healing rate is 0 hp (though of course a construct with Fast Healing has a healing rate \emph{per round} and likely doesn't care).}
	\bolditem{Mindless:}{Even an intelligent construct has a synthetic mind that is unreachable by sorcery. A construct is not affected by [Mind Affecting] effects and cannot be detected with \spell{detect thoughts}.}
	\bolditem{Never Alive:}{A construct cannot be raised or resurrected. A construct is likewise immune to energy drain.}
	\bolditem{Repairable:}{A construct does not become staggered at 0 hit points, nor does it die at -10. If for some reason you are using the ``Death by Massive Damage Rule'', constructs aren't affected by it. As soon as a Construct hits zero hit points it becomes inert, and any abilities it may have cease to function (including fast healing abilities). However, a construct in this state can still be brought to working order again with a Craft check with a DC equal to the DC to make it in the first place with a base amount of time of one hour per hit point below 1 the construct was left at.}
	\bolditem{Nonbiological:}{Constructs do not eat or breathe, constructs do not age.}
	\bolditem{Lacks Squishy Bits:}{A construct is not affected by any effect that allows a Fort save unless that effect affects objects or is a (Harmless) effect. For example, a clockwork horror is not going to catch red fever or become nauseated by a stinking cloud. But it is not outside the realm of possibility for an eidolon to be afflicted with a totally magical disease that functions off of Willpower saves.}
\end{list}

All the stuff about constructs being ``immune to necromancy'' is out the window (because we all know that you can use \spell{magic jar} to put your soul into a statue); all the stuff about constructs being immune to ability damage is out the window (because we all know that you can slow down a lumber construct); and of course the immunity to critical hits is \emph{totally} out the window (if you have the name of Pelor on your forehead there is at least one critical location that probably won't go well for you if it is hit).

\subsection{Controlling Constructs: Robot Armies and Statuary Servants}

\desc{Time and time again adventurers report finding constructs that have been left attending temples and castles long after those buildings have fallen into ruin. The reason for this is twofold: First, constructs don't age; and Second, constructs don't count as one of your eight constant magical items if they are set to guarding a location. This means that powerful wizards are actually encouraged to leave their golems places with patrol or sentry orders and then of course these sentry golems will have a tendency to outlive the wizards, and even the buildings that they guard.}

Of course, it's entirely possible to make your constructs follow you around. If you do, they count against your 8 item limit.

\abox{Behind the Curtain: Why the Lower CRs?}{A cohort, or a planarly bound outsider, or a necromantically crafted monster could all plausibly be of a CR that is just 2 less than your character level without particularly disrupting play. So it may seem pretty weird that the constructs one can order around are weaker than that. The reasoning is ironically because the tactical role of a construct is so different from that of a Ghoul or Jarilith. While many potential servant creatures are simply weaker versions of normal characters or dangerous and fragile glass cannons -- in almost all cases a construct is an offensively anemic unit with a highly powerful defense. For those of you who have played tactical games or MMOs, that makes the average construct an ideal ``pet''. A strong defense is disproportionately useful for secondary characters expected to travel in front, and the fact that characters aren't allowed to fill their magic item cap with cohort level constructs is no accident.}

\subsection{Specific Constructs Under the New Rules}

% BIG HONKIN' NOTE:
% This uses \subsubsection instead of \monster. This is because we put 
% The Book of Gears in its entirety in its own appendix instead of moving 
% stuff around (I did it this way because the book is unfinished and it would 
% be awfully weird and illogical to move stuff around until it is finished).
% 
% So, when we finally move stuff around and put the simulacrum in the "Monsters" 
% chapter, we're gonna want to use \monster (which creates a subsection and a label 
% for hyperlinking) instead of \subsubsection.
% -Surgo
\subsubsection{Simulacrum}
Whether created by an Effigy Master, a mystic location or some other powerful source of illusion magic, a simulacrum is a construct made of ice and snow which appears to be a normal living creature through the power of illusion. Some 

\noindent\monstersizetype{Medium}{Construct}
\monsterline{Hit Dice}{6d10+6 (39 hit points)}
\monsterline{Initiative}{+1}
\monsterline{Speed}{30'}
\monsterline{AC}{11 (+1 Dex); Flat-footed 10; Touch 11}
\monsterline{BAB/Grapple}{+4/+5}
\monsterline{Attack}{Glamersword +5 melee (1d8+1)}
\monsterline{Full Attack}{Glamersword +5 melee (1d8+1)}
\monsterline{Space/Reach}{5'/5'}
\monsterline{Special Abilities}{Glamered, Imprinting}
\monsterline{Ability Scores}{Str 13; Dex 13; Con 13; Int 15; Wis 15; Cha 15}
\monsterline{Saves}{Fort +3; Reflex +3; Will +4}
\monsterline{Skills}{Bluff +11; Disguise +13 (+23 when Imprinted); Gather Information +11; Sense Motive +11}
\monsterline{Feats}{Impersonation}
\monsterline{Alignment}{As creator}
\monsterline{Organization}{Thrall}
\monsterline{Challenge Rating}{3}

It is important to note, however, that simulacra are entirely capable of using equipment, and usually will do so. Like most constructs, a simulacrum's true power comes to the fore when gifted with some basic mundane and magical equipment. Here is a sample simulacrum which has been given a magic shield, a magic breastplate, and a Frost Sword -- all equipment which is well within the capabilities of an Effigy Master to acquire or produce. While the simulacrum is still a ``CR 3 Creature" -- once it has been armed and equipped it is \textit{much} more formidable.

\noindent\textbf{Simulacrum with Equipment}\\
\monstersizetype{Medium}{Construct}
\monsterline{Hit Dice}{6d10+6 (39 hit points)}
\monsterline{Initiative}{+1}
\monsterline{Speed}{30'}
\monsterline{AC}{22 (+1 Dex, +7 Armor (Magic Breastplate), +4 Shield (Magic Shield)); Flat-footed 21; Touch 11}
\monsterline{BAB/Grapple}{+4/+5}
\monsterline{Attack}{Frost Sword +7 melee (1d8+3, +5 Cold Damage)}
\monsterline{Full Attack}{Frost Sword +7 melee (1d8+3, +5 Cold Damage)}
\monsterline{Space/Reach}{5'/5'}
\monsterline{Special Abilities}{Glamered, Imprinting, Ignore first 5 points of nonlethal damage (from armor), +2 bonus on bull rush attempts (from shield)}
\monsterline{Ability Scores}{Str 13; Dex 13; Con 13; Int 15; Wis 15; Cha 15}
\monsterline{Saves}{Fort +3; Reflex +3; Will +4}
\monsterline{Skills}{Bluff +11; Disguise +13 (+23 when Imprinted); Gather Information +11; Sense Motive +11}


\section{Denizens of the Planes of Law}

When you think avatars of Evil in D\&D it is no trouble at all to conjure up images of spiteful devils and destructive demons; but when you talk about a being of \emph{Law} the image that comes up is simply not the same from one person to another. Part of that is because Law doesn't really mean anything consistent in D\&D nomenclature. And part of that is because the actual description of the inhabitants of Mechanus has changed wildly through the generations and editions.

\subsection{Modrons: Singularity of Purpose}
% By Frank Trollman

\desc{For those of you who don't remember: Modrons are the original creatures of Law from the old days of AD\&D. They haven't been seen very often because they were originally written as a joke. Their very existence is as offensive to many players as the fact that they were essentially retconned out of existence is to others. And what's that all about? It's because the Modrons were originally written up as giant dice. Yes, really. The different types of basic Modron are shaped like four sided dice, six sided dice, 8 sided dice, the whole thing.}

\desc{So if your DM jumps on the ``let's forget this ever happened'' bandwagon, we understand. The original write up of the Modrons was actually pretty insulting. But since then there have been a number of variously successful attempts to rehabilitate them and make them independently awesome. Different Modron art has been made by Tony DiTerlizzi and Eric Campanella that looks pretty darned awesome � and not like your DM put a 6 sided die on the battle mat at all. Instead each Modron looks like a ghastly hybrid of metal and flesh covered with cogs and wheels where spindly appendages emerge from a solid (though not rollable) core.}

\desc{So assuming that you use some of the reform Modrons from late in 2nd Edition, the Modrons are actually pretty cool. They represent the idea of Law as an implacable and incomprehensible force. They are at their best when portrayed as being so single mindedly focused on some long term goal that they actually don't even care about you. Sometimes they destroy your village, sometimes they don't, and there's really no predicting that sort of thing unless you're knowledgeable about the Big Plan. Now I know what you're thinking� that having a plan so convoluted and far ranging that mortal minds cannot grasp it or predict its unfolding is actually indistinguishable from not having a plan at all and just performing actions at random. And yeah� that's true. That's DnD alignment for you.}

\desc{The Modrons come from a city in the Clockwork Nirvana called Regulus and have a rigid caste system where more powerful Modrons are told more of ``the plan'' than less powerful Modrons are told less. Each Modron is told exactly as much as it needs to know to complete its assigned tasks. And in the face of a long term plan of this magnitude, that pretty much means that every Modron is kept entirely in the dark about just what the heck it is doing or why it is doing it. The Modrons are arranged into a rigid caste system with no possibility (or concept) of personal advancement. That would be pretty stultifying if they were like humans where they all started out equal, but they aren't like that at all and Modrons of each caste sincerely don't have any desire to move up to another caste. At the very bottom (or at least, ``most numerous and least clued in to the plan'') there are the cogs. Cogs look like little gears and have the ability to transform into any object of roughly a cubic foot or less. When properly supervised by a higher level drone, they can take the shape of quite complicated pieces of clockwork and frequently do so. Above them are various drones of various shapes and sizes, each constructed of materials mechanical and biological to fulfill its role in the great plan.}

\desc{At the top there is Primus, who has been variously described as everything from an Intelligent Item to a reasonably powerful Outsider to a guy working the machinery behind a curtain. Seriously, your campaign can potentially reveal anything you want it to about what is really in the center of Regulus because that's been retconned so many times that noone knows what the official answer even is right now. Primus has been killed off several times in official continuity as part of various authors attempting to delete the Modron race from D\&D. However, someone always brings Primus back, because D\&D never really throws anything away (except the pygmies, that was too racist even for the 1980s). If you demand continuity in your life, then it seems that Primus simply can be killed time and time again, each time getting replaced with a new Primus who is in turn enlightened as to the nature of the big plan and granted the authority to turn the wheels of Regulus.}

\desc{The Modrons mostly sit around and operate the machinery in Regulus that apparently keeps all the giant gears and pistons running in the Clockwork Nirvana of Mechanus. So even if they exist in your game's continuity, chances are good that you'll never encounter them. Every so often, a whole lot of Modrons open up gates to other planes and start wandering around in big groups doing� stuff. The amount of time that Modrons spend between their ``March'' is supposed to be constant but actually every single Modron March that has ever been mentioned in any published adventure or story has taken place out of sequence so one is forced to conclude that actually the Modron March happens whenever the big plan calls for it and rumors of a great schedule are just rumors. The Modron March is generally not harmful and the Modrons don't seem to go out of their way to chase anyone who gets out of the way, so it doesn't seem to be an invasion. Although who knows? The plan of the Modron is so incredibly far reaching that it is entirely possible that they simply walk around in huge armed groups wandering seemingly aimlessly through the planes at irregular intervals doing no harm to anyone so that at some later date they can do the same thing and then just destroy some enemy that is predicted in the distant future.}

\desc{Unfortunately, now I'm going to have to talk about the Hierarchs. These are a layer of bad asses that live in the eco niche between drone and god. They are roughly equivalent to the high end fiends and celestials � coming in various flavors and power levels like Mariliths and Balors do for the demons. Unfortunately, noone has ever overhauled the art on these guys to the point where they don't look like ass. Sorry, the Hierarchs of Regulus look like they were drawn by Napoleon Dynamite and there's nothing nice I can say about them. If I were personally inclined to run with these guys at all I would be forced at gun point by the players to have them look like� something else. Putting them back into the theme of the Modron Drones is probably best, because at least then they appear to be Modrons in the same way that a Trumpet Archon is readily identifiable as an Archon. Which means that honestly the Hierarchs are much cooler if they look like Daleks. And the truth of that statement is probably the single greatest argument for walking away from the whole thing, as advised by Monte Cook.}

\desc{So what do Modrons do in a story? Mostly they show up with a specific set of instructions that they attempt to fulfill. They speak their own language but the more powerful ones also speak additional planar languages (hope you speak Formian). If the player characters attempt to prevent the Modrons from doing their thing, the Modrons will fight. Otherwise they'll simply complete their task and leave. If other creatures attempt to stop the Modrons, they'll fight them. Modrons hold no grudges and have no loyalty but to the plan. They will seamlessly switch sides in a battle if a different group proves more detrimental to their mission. The Modrons don't know why they are doing what they are doing, only Primus does (assuming Primus exists, in some versions The Plan is actually a flaw in Mechanus and there is no reason for any of it). And yet they will fight to the death to complete their mission. With good art and weird dialogue, the Modrons can serve as dynamic antagonists or useful allies.}

\desc{Like an amazingly intricate puzzle box, the Modron Race unfolds, collapses, and progresses in countless ways, but at the end, it's still a puzzle box. The entire point of the Modrons is that you never get clued in to what they are doing or why they are doing it. But if you don't want to use them at all, I understand.}

\subsection{Inevitables: Enforcing Natural Law}
%By Quantumboost

The Inevitables are unusual among ``exemplary'' creatures from Mechanus in that they actually don't have the Outsider (Law) type; they're straight-up Constructs. They're still Extraplanar, and they still have the Law subtype, but they're forged from steel and other metals, or at least the closest Outer Planar equivalents. What they represent among the creatures of Law is the idea of Law as ``universal rules'' -- that there is some code which should be obeyed by all beings, and that if you don't abide by it super fighting robots from another dimension will come and stop you from disobeying if and when they find out.

\subsection{Formians: In Many, One}
%By Quantumboost

\quot{``There is no good for the bee that is bad for the hive.''}

\desc{The Formians are the current Lawful Outsiders, and true to their name are giant ants -- and that's most of what you need to know about them. Formians represent Law as Conformity -- conformity to their society, and removal of anyone who won't conform. They're expansionistic, militaristic, and each hive effectively acts as one organism as they spread across the gears. Like most portrayals of giant ant-people, Formians are telepathic and have a hive-mind. However, they are giant ant \emph{people}, and each individual Formian above Worker level has a level of mental ability at least what you might expect from a human. So interacting with them is a lot like working with the Borg - you can totally }

\desc{The most straightforward way to use the Formians is to have them show up and want to conquer your favorite hangout -- either taking away the people to work in their hive (which is a lot like what the Hobgoblins will do, except that there probably isn't any chance of advancement from the slave thing) or turning it into a new hive, superintelligent mind-enslaving queen ant and all. But you can also use them as more mundane antagonists or even as potential allies. Unlike Modrons, Formians actually have discernable goals and motivations, their motivations just all happen to be ``what is good for the hive''. If you're working towards a goal or offering a reward that they think will help the hive, they'll totally be willing to work with you and might not even bother you about the whole assimilation thing beyond handing out some pamphlets.}

