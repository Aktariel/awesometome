\documentclass[10pt]{report}

\usepackage{appendix}
\usepackage{floatflt}
\usepackage{fancyhdr}
\usepackage{textcomp}
\usepackage[usenames]{color}
\usepackage{isoent}    %\sfrac
%\usepackage{palatino} %font
\usepackage{newcent}   %font
\usepackage{sectsty} %custom section headings

\newcommand{\normalsections}{
\sectionfont{\noindent\rule{\textwidth}{0.015in}\\\nohang}
\subsectionfont{\noindent\rule{\textwidth}{0.005in}\\\nohang}
}

\newcommand{\columnsections}{
\sectionfont{\vspace*{-20pt}\noindent\rule{3.5in}{0.015in}\\\nohang}
\subsectionfont{\vspace*{-20pt}\noindent\rule{3.5in}{0.005in}\\\nohang}
}


\sectionfont{\noindent\rule{\textwidth}{0.015in}\\\nohang}
\subsectionfont{\noindent\rule{3.5in}{0.005in}\\\nohang}

\usepackage[Bjarne]{fncychap} %Canned chapter headings


\usepackage{multicol}
\usepackage[bookmarks=true,colorlinks,linkcolor=cyan,breaklinks]{hyperref}



%%%%Margins%%%
\topmargin 0pt
\advance \topmargin by -\headheight
\advance \topmargin by -\headsep
\textheight 8.9in
\oddsidemargin -0.25in
\evensidemargin \oddsidemargin
\textwidth 7in
\oddsidemargin -0.25in
%\setlength {\parindent} {0pt}



%%%Formatting%%%
\newcommand{\ability}[2]{\smallskip \noindent \textbf{#1} #2}
\newcommand{\shortability}[2]{\noindent\textbf{#1} #2\\}
\newcommand{\bolded}[1]{\noindent\textbf{#1}}
\newcommand{\itemability}[2]{\item \textbf{#1} #2}
\newcommand{\featname}[1]{\vspace*{0.1cm plus 0.2cm minus 0.05cm}\noindent\textbf{#1}\\}
\newcommand{\featnamelist}[1]{\vspace*{0.1cm plus 0.2cm minus 0.05cm}\noindent\textbf{#1}}

\newcommand{\descfeat}[2]{\featname{#1}\emph{#2}\\}

\newcommand{\classname}[1]{\section{#1}}
\newcommand{\condition}[1]{\emph{#1}}
\newcommand{\quot}[1]{\emph{#1}\medskip}
\newcommand{\desc}[1]{#1 \medskip}
\newcommand{\example}[1]{\emph{#1}}
\newcommand{\magicitem}[1]{\emph{#1}}
\newcommand{\monster}[1]{\subsection{#1} \label{monster:#1}}
\newcommand{\monsterline}[2]{\textbf{#1:} #2\\}
\newcommand{\monstersizetype}[2]{\textbf{#1 #2}\\}
\newcommand{\spell}[1]{\emph{#1}}
\newcommand{\spelllist}[1]{\smallskip \noindent \underline{\textbf{#1}}}

% For the feats -- moved here from feats.tex because we added another 
% section(s) for feats. -Surgo
\newcommand{\minitabular}[1]{\begin{tabular}{p{0.25in}p{2.9in}} #1\\ \end{tabular}}
\newcommand{\babfeat}[7]{
	\noindent\minitabular{\multicolumn{2}{l}{\parbox{3in}{\textbf{#1}}}}
	\minitabular{\multicolumn{2}{l}{\parbox{3in}{\small #2}}}
	\minitabular{\raggedleft\textbf{\small \textbf{+0:}}& {\small #3}}
	\minitabular{\raggedleft\textbf{\small +1:} & {\small #4}}
	\minitabular{\raggedleft\textbf{\small +6:} & {\small #5}}
	\minitabular{\raggedleft\textbf{\small +11:} & {\small #6}}
	\minitabular{\raggedleft\textbf{\small +16:} &{\small #7}}
}
\newcommand{\skillfeat}[8]{
	\noindent\minitabular{\multicolumn{2}{l}{\parbox{3in}{\textbf{#1}}}}
	\minitabular{\multicolumn{2}{l}{\parbox{3in}{\small #2}}}
	\minitabular{\multicolumn{2}{l}{\parbox{3in}{\small\textbf{#3}}}}
	\minitabular{\raggedleft\textbf{\small \textbf{0:}} &{\small #4}}
	\minitabular{\raggedleft\textbf{\small 4:} &{\small #5}}
	\minitabular{\raggedleft\textbf{\small 9:} &{\small #6}}
	\minitabular{\raggedleft\textbf{\small 14:} &{\small #7}}
	\minitabular{\raggedleft\textbf{\small 19:} &{\small #8}}
}

\newcommand{\boxit}[1]{\frame{\parbox{\textwidth}{#1}}}

% A new box command -- the first argument is the box's header, the second the 
% contents of the box. This looks a lot better than \boxit. -Surgo
\newcommand{\abox}[2]{\vspace{5pt}
	\fbox{\begin{minipage}{0.8\linewidth}
	\setlength{\parindent}{0.15in}
	\textbf{#1}

	#2
\end{minipage}}
\vspace{5pt}}

\newcommand{\itemspace}{\setlength{\itemsep}{-1mm}\setlength{\topsep}{-1mm} }

\newcommand{\listone}{\begin{list}{$\bullet$}{\itemspace}}
\newcommand{\listprereq}{\begin{list}{\vspace*{-2pt}}{\itemspace}}
\newcommand{\listtwo}{\begin{list}{$\triangleright$}{\itemspace}}
\newcommand{\listthree}{\begin{list}{--}{\itemspace}}

% Bold the first argument in the listitem
\newcommand{\bolditem}[2]{\item \textbf{#1} #2}

\newcommand{\slashfrac}[2]{${}^{#1}$\hspace*{-1pt}\makebox[2pt]{$\diagup$}${}_{#2}$}
%\newcommand{\half}[0]{\slashfrac{1}{2}}
\newcommand{\half}[0]{\ensuremath{\sfrac{1}{2}} }
\newcommand{\third}[0]{\ensuremath{\sfrac{1}{3}} }
\newcommand{\fourth}[0]{\ensuremath{\sfrac{1}{4}} }
%\newcommand{\half}[0]{\ensuremath{\scriptscriptstyle 1/2}}
%\newcommand{\threefourths}[0]{\ensuremath{\sfrac{3}{4}}}


%\rule{\textwidth}{0.005in}

%\renewcommand{\sectionmark}[1]{\markright{\thesection\ \boldmath\emph{#1}\unboldmath}\\\rule{\textwidth}{0.005in}\\}

\setcounter{tocdepth}{1}

\begin{document}

\pagestyle{plain}
%Cover Page

\begin{center} \Huge

\textsc{The Book of Gears}\end{center}

\vspace{2cm}
\begin{center}\large By The Gaming Den\end{center}

\newpage

\vspace*{4in}

\noindent Please address all complaints and comments about balance to the authors at\\
{\color{blue} \href{http://tgdmb.com/viewforum.php?f=1}{http://tgdmb.com/viewforum.php?f=1}}

\vspace{0.2in}

%\noindent For a hypertext version of some of this information, especially classes, please look at\\
%{\color{blue} \href{http://www.d20ragon.com/frank/}{http://www.d20ragon.com/frank/}}

%\vspace{0.2in}

\noindent Amateur Typesetting by Joshua Middendorf, updated by Morgon ``Surgo'' Kanter, ``Aktariel'', and Stephen ``Quantumboost'' Smith.

\vspace{0.15in}

\noindent Please address all comments regarding the quality (or lack thereof) of the typesetting (that is, formatting of the pdf) to Joshua Middendorf (\href{mailto:middendorfproject@gmail.com}{middendorfproject@gmail.com}), Morgon Kanter (\href{mailto:morgon.kanter@gmail.com}{morgon.kanter@gmail.com}), Aktariel (\href{mailto:aktariel@gmail.com}{aktariel@gmail.com}), or simply comment in the above forum.


%\vspace{0.2in}

%\emph{Enjoy!}


\vspace{1in}
\noindent Published on \today, version 0.6\\
\noindent You may find the most recent version of this document at:\\
{\color{blue} \href{http://www.tgdmb.com/viewtopic.php?t=36046}{http://www.tgdmb.com/viewtopic.php?t=36046}}

\newpage


\pagestyle{fancy}
%Fix the spacing so it's all reasonable
\linespread{.9}  \small  \normalsize \itemspace \normalsections

\tableofcontents

\chapter*{Preface to the Book of Gears}

\desc{Cooperative Storytelling is essentially all about artifice. The stories we create are created, the shared narrative is an illusion which fills our mind and pushes us forward. So it is no surprise that creating things within that narrative is so very contentious. Building a house in the game or creating an illusion in the story is a}

\desc{An illusion is something that isn't real inside a story that isn't real. Forging a sword is creating something within a tale that is being created around it. These actions, while very integral to the source material upon which our cooperative storytelling games are based, are yet one more step removed from reality when contrasted with the old standards of pretending to be a knight who kills imaginary dragons to save fictitious princesses.}

\desc{So it seems not at all surprising in retrospect that the rules we have used to represent the creation of stuff within the game world have historically been extremely unsatisfactory. Creating things takes time, which is a problematic concern in a game where time passages narratively. That means that the time a character spends nailing boards together for his dream house may be spent in a montage that ends in subtitles reading ``six months later'' and it may happen interspersed with a rollicking adventure where seconds count and the hammering essentially never gets done.}

\desc{The result has been that previous editions have attempted to put additional or alternate costs on crafting of all sorts. From Constitution points to years off your life to XP, D\&D has experimented with about a dozen different rubrics by which characters could trade one part of their character for more magic items. In almost all cases this allowed players to trade things they weren't using anyway for powerful artifacts that allowed them to conquer worlds, although in a few cases the flip side showed up and made item creation so crappy that people seriously didn't do it at all. Needless to say, this has been unsatisfying, and it is our intention to help remedy these problems.}

\desc{The rules presented here present a different take entirely. Creating magic items is something that takes only time, and adventures can be expected to be completed without ever doing it at all...}

\chapter{Character Advancement: Power and Wealth}
\vspace{-12pt}

\quot{``Assuming that I make the use of most of our spells, I should be able to advance a circle of magic every week or so, which essentially means that the optimum solution to this difficulty to simply scare up minor tangential difficulties in the woods for two months so that I can go back in time and solve this problem retroactively.''}
\vspace{12pt}

\desc{While we're talking about magical items, we really have to talk about XP at the same time. And that's not just because the DMG asks us to pay small amounts of XP to create them. D\&D is based on two kinds of advancement: XP and GP. Both of them have failed, because we're actually playing a cooperative storytelling game and not Diablo multiplayer. We know that a high level guy can whack low level stuff again and again at virtually no risk, and that this can be repeated endlessly for levels. We know that people can take off downtime to just plain \emph{farm} to get GP endlessly. Seriously, ``XP Grind'' is extremely boring and players should not be exposed to it under any circumstances.}

\desc{Noone wants to hear about the time you threw a \emph{cloudkill} into a Satyr tavern and then teleported home so that you could try out the new spells that appeared in your book because you just dinged to 10th level. That's a story that is \emph{dumb}, and the current rules pretty much expect you to do it over and over again. If we're going to have a rational system for magic items, we can't have it work that way.}

\section{XP: Beer Me.}
\quot{``Boil an Anthill: Go Up One Level.''}

\desc{The rubrics for challenge and advancement as depicted in the DMG have to go. We've looked at them from every direction, and they don't work. At all. And no, I'm not talking about the classic problems like the variable difficulty inherent in fighting a giant scorpion (an interesting intellectual exercise for a 4th level horse archer or a brutal melee slugfest for a 14th level swordsman). That's a real problem, but we are talking about the basic structure of fighting monsters of increasing CR, getting increased piles of XP, and moving on with your life. That's got to end.}

\desc{Here's why: according to the DMG you are supposed to face about 4 equal-level challenges per day of adventuring. Further, going by the XP chart, your 4-person party will go up a level every time you defeat 13.3 of those encounters -� which is less than 4 days worth of encounters according to the first idea. So if you adventure ``like you're supposed to'' -- you'll go up 2 levels a week. And of course, if you encounter less than 4 enemies a day, spell-slot characters like Wizards and Druids are crazy good. Essentially, this means that D\&D characters go from 1st level to 20th level in half the time as it takes to bring a pregnancy to term.}

\desc{Indeed, D\&D society is essentially impossible. Not because Wizards are producing expensive items with their minds or because high level Clerics can raise the dead -- but because the character advancement posited in the DMG is so fast that it is literally impossible for anyone to keep tabs on what the society even is. High level characters are the military, economic, and social powerbases of the world. And they apparently rise from nothing in\dots about 2\half months. That means that if a peasant goes home to plant his crops, then when he gets back to the city with his harvest in the fall the city will have seen the rise of a group of hearty adventurers who attempt to conquer the world and achieve godhood four times while he's gone. The city will have been conquered by a horde of Dao and sucked into the Elemental Plane of Earth and then returned to the prime material as a group of escaped Dao slaves achieved their freedom and themselves became powerful plane hopping adventurers who graduated to the Epic landscape. Then a team of renegade soldiers from the Dao army will have run off into the countryside and survived in the Spider Woods long enough to return with the Spear of Ankhut to return the city to the Dao Sultan in exchange for a gravy train of concubines and wishes. Then a squad of frustrated concubines will have turned on their masters and engaged in a web of intrigue culminating in the poisoning of the Dao Sultan with Barghest Bile and ultimately turned the city into a matriarchal magocracy run by ex-concubine sorceresses. So when the peasant returns with his harvest of wheat, he returns to� a black edifice of magical stone done up in Arabian styles and bedecked with weaponry from Olympus that is all controlled by epically subtle and powerful wizards who are themselves the masters of a setting created from the fallout of the destruction of a setting that is itself the fallout of the destruction of a setting that was in turn created out of the destruction of the setting that our peasant walked away from with a bag of grain come planting time last year.}

\desc{And while purely intellectual exercises in a universe that is essentially a giant lava lamp of crazy can be interesting, satisfying storytelling is impossible. If the players can't make lasting impact, the game has no meaning. And if players are seriously going from 1st to 20th in a single season, lasting impact of any kind is absurd to even contemplate. It behooves players and DMs to come to a consensus about how they want their campaign to be structured. There is no single best way to handle character advancement in a cooperative storytelling game, and there are a lot of ways to really piss off the other players at the table if you aren't all on the same page to begin with.}

\section{Reach for the Stars: Character Advancement}

\desc{All classic fantasy adventures take place in D\&D terms somewhere between 1st and 10th level. Seriously. Conan is like 3rd level, Theseus is about 3rd level too. Adventures for 13th level in literature of any kind are hard to come by and generally involve wearing capes or being a god. However, D\&D is not a game about modeling tales of legendary knights, skilled samurai, or barbarian chiefs � it is a game about adventuring in the world of D\&D. And in D\&D, characters do become 20th level, at which point they either become honorary Olympians or join the Justice League. Within that context, character advancement should follow a few basic principles:}

\desc{\listone
    \item\ability{Stagnant Characters are frustrating.}{That is, in a game which offers so much potential for advancement, it is frustrating to be in the position where you don't actually get to do any of it. Sure, in a game like Shadowrun there's no disappointment to be had from not being able to achieve godhood and in a game like Champions you don't need to advance your character at all to have a good time. But D\&D is a leveled system and not getting those levels makes us sad.}
    \item\ability{Advancement of Characters shouldn't destroy the setting.}{If you're playing a ``pirate game'' then you shouldn't get to the point where there is no longer a purpose served in piracy as long as you still play that game. Furthermore, you shouldn't be adverse to downtime on the grounds that waiting a month or two for a storm to go by will leave your enemies driving air cars powered by t-rexes on bicycles.}
    \item\ability{Players should be able to play with their toys.}{Too often, a character will get a shiny new trick only to go up in level and have no further use for it long before he has had a chance to actually use it. And that defeats the entire purpose of leveling up in the first place.}
    \item\ability{Characters should not be rewarded for doing stupid crap.}{Seriously. Your goal is to rescue the princess, so what should you do? Rescue the princess, or� run around the compound she's being held in punching out the baron's attack dogs? An army is heading for your city, should you sneak in and kill the enemy general or should you try to wrestle the army's horses one at a time?}
\end{list}}

\vspace{12pt}

\noindent \desc{This leads us to several conclusions of varying palatability:}

\subsubsection{Wealth By Level Has Got to Go.}

\desc{This hurts a lot of people, but it's true. If you can turn a pile of silver into increases to your natural armor bonus, the setting is going to be destroyed. Quite literally, and with crowbars. Fantasy settings are filled with bridges made of opal and castles faced with blue ice that sty forever cold and stuff. This fantastic scenery is awesome, and it contributes to the feel of fantasy that should permeate the cooperative stories we tell within a D\&D game. If player character power is determined by ``wealth'' in any directly measurable fashion, you can bank on PCs ripping all the expensive facing off the castles they conquer � and then we all lose.}

\desc{See, it's pragmatic and even sort of reasonable to rip the marble off the Great Pyramid at Giza and use it to build fancy houses in Cairo. But for all the future generations, it sucks. There really is a correlation here: if we don't allow people to trade blocks of marble for extra spells per day and more powerfully magical swords, then people will leave our pyramids alone. Otherwise, future generations will look at another unfaced ziggurat and wonder what wonders the ancient battlefields possessed before vandals came and destroyed our fantasy world.}

\subsubsection{Encounter XP Has Got to Go.}

\desc{XP rewards are a form of incentive towards heroic behavior. The problem is that individual challenges don't make things more heroic, they just make things more time consuming. By parting out XP per \emph{encounter} rather than per \emph{quest} the game is actually discouraging intelligent play. Avoiding difficulties is supposed to get you XP according to the DMG but we all know that doesn't actually happen in any game or published module.}

\desc{Adventurers respond very rapidly to incentives. If you give incentives for painstakingly stabbing minotaur after minotaur in the face the players will do that. If you incentivize running past the horde of minotaurs and rescuing the princess the players will do that instead. So if the XP comes from quest completion, players will \emph{complete quests}. If XP comes from Final Fantasy style XP dancing in the woods � the players will do that instead. Since one of them makes for awesome stories, and the other is a rote repetition of the worst kind of World of Warcraft nonsense, we know what has to be done.}

\begin{list}{}{\itemspace}
\item \subsubsection{A Little Note on XP Costs}
\item \desc{I know that you're probably thinking ``If XP rewards are handed out in a less per-diem manner, doesn't that mean that XP costs would be more noticeable and even actually have meaning?'' And of course the answer is ``yes''\dots. Sort of.}
\item \desc{The problem with XP costs isn't just that they don't really cost anything ``in the long run'' (which they don't), the problem is that they are bad for the game. Like Age increases before them, an XP cost is essentially running up a credit card bill. You get whatever it is that you were buying with the XP cost \desc{now}, and you pay \desc{later} (by death from old age or not going up in level when you otherwise would). That's never balanced, because there's no guaranty that the character in question will still be being played when that credit card comes due.}
\item \desc{So even though staggering XP gains out longer as suggested in this book \emph{would} make XP costs more meaningful than the hoax they are in the basic rules, we still strongly aadvise you to do away with them in your home games as we have in ours.}
\end{list}

\section{Strategies of Advancement}

\desc{Having determined the core problems with advancement in the manner described in the DMG, let's talk about some of the ways you could do it that might be satisfying. Like the handling of alignment and necromancy that we're talked of in the past, there really is no right answer -- it really depends upon what your group wants to do.}

\subsection{Steady State 1: Serial Heroism}

\quot{``We have another mission for you.....''}

\desc{Let's face it: in a lot of fictional source material, the characters don't really change between their adventures at all. In fact, that's kind of the \emph{point} of a lot of stories. The hero is the one fixed point and the story is just the fixed character reacting to different situations. You read about Conan or Hercules fighting the Moon Men or the Ice Jarls, but you don't really read the story set after Hercules got a laser gun and grew wings. Even the books where Conan is an old man rarely reference specific events from previous books.}

\desc{In the serial heroism campaign, characters begin play at the level that depicts their abilities appropriately. Characters have signature equipment and a collection of levels and skills that are integral to their character. Over the course of the adventure, the characters may well find new equipment and learn special crap and be blessed by Nymph Pools and whatever -� but at the beginning of the next adventure they will be back to exactly the same place they were last time. Even characters getting married or having limbs whacked off doesn't have any effect on the next episode.}

\desc{There are a lot of ways to explain this. Adventurers spend money profligately and put equipment into bat caves and bequeath magic swords to temples and favored wenches. Major wounds can be healed, and we all know how rarely things work out between men and women �- especially when one is a halfling rogue and the other is a giant iguana. You can either begin each episode by coming up with an amusing off-the-cuff answer to why you begin the next adventure just like you began the last one or you can just ignore it the way Saturday morning cartoons do. It's not a big problem.}

\desc{There are a lot of advantages to this sort of thing. If the characters already do what they are supposed to (generally about level 6 or so with a couple of standard magic items and an artifact), then advancement of any kind just makes the character less like himself. He Man didn't become a better show when Prince Adam got a plane -� it just lost focus. But there are pitfalls as well. Certainly it is the case that games like World of Warcraft or Everquest can be remarkably unsatisfying \emph{precisely} because no real accomplishment can occur. It is a fine line between a character not changing and a character's actions not mattering �- walking that line is sometimes quite difficult. Certainly, before such a sweeping change is implemented, very frank discussions must be had between players and the DM. The game is essentially now a series of once off adventures that happen to have the same characters in them.}

\desc{In the Serial Heroism game the character's core abilities are the same in every tale. That can be mythic. Like the Robin Hood songs. But it can also be retarded. Like the Smurfs.}

\subsection{Steady State 2: Trophy Hunting}

\quot{``So� where \emph{are} we putting the giant penny?''}

\desc{Characters like Conan and He-Man are pretty much the same between issues or episodes. But what of characters like Angel and Buffy who really do pick up and use equipment found in previous episodes? This is also a very plausible setup of ``nearly steady state'' storytelling with limited character development. The character stays relatively recognizable one adventure to another. Chapter after chapter goes by without the player ever growing wings, learning to fly, shooting laser eye beams or in any other way having obviously gained a level of Bard. Important plot points and devices are referenced in later installments, allowing the characters to use the Doom Glaive after they took it off the cooling body of Bruc Avec Piti'e both immediately in that adventure and subsequently in later adventures as well. While in the true Serial the characters would have destroyed the Doom Glaive at the end of the adventure, in the Trophy Hunting model it stays in the Bat Cave only until it is needed for a later adventure.}

\desc{In this model of steady state dynamics, the players gradually increase in power -- though they do so in an asymmetric fashion that is not level dependent. This means that the amount of Ogres that the party can successfully dispatch \bolded{will} increase considerably over time. But it won't increase \emph{dramatically} and the players may never be able to take on a really hardcore monster like a Cranium Rat Swarm or a Pit Fiend.}

\desc{In this model then, it is expected that even the \emph{idea} of ``Wealth by Level'' be tossed in the trash. The players are literally gaining as many as infinity magic items per level because by and large they aren't going up levels \emph{at all}, while magic items are accumulating slowly. Characters can bathe in magic puddles that increase their stats or find statues that transform into giant frogs; but this can also happen pretty slowly and still be fine because players aren't being forced into situations where they necessarily face higher leveled opposition all the time.}

\subsection{Rapid Advancement: Level a Session}

\quot{``That was last week. This week I am a master of fire.''}

\desc{It is entirely plausible to play a game where the characters go up a level every adventure or even every session. While this sort of rapid advancement scenario is often dismissed as ``munchkin'', it actually does capture the feel of many stand-alone books and movies quite well. There are a lot of stories like The Wheel of Time or The Matrix which are actually ruined by having sequels at all �- they are much better as a single progression where the characters begin as youngsters who don't even know about the major Evil that threatens the world and progress briskly into becoming world straddling badasses who control reality with willpower alone.}

\desc{In this set up it is highly recommended that the DM hand out magic items like candy. After all, while the players are fighting hill giants today, they'll be up against a swarm of bloodfiend locusts next week and a rogue deva the week after that. The players will need new swag to face their new enemies just as they'll need new class abilities.}

\desc{Many players feel that this sort of play environment is simple minded, but really nothing could be further from the truth. In fact, players have no chance to get acquainted with their new abilities before they are laden with even newer abilities. With only a single adventure to make use of each new level of powers it is entirely possible that the Wizard will \emph{never} get a chance to use one or even both of the shiny new spells he picks up each level. Indeed, since both the characters and the opposition is coming up with more power and options each week, the game is actually \emph{really hard}.}

\desc{And that, ironically, is the most major drawback of this gaming style. Some of the players who gravitate the most towards this advancement system are actually the least able to successfully juggle a new class level and two new magic items every week. Sure, there are difficulties to be had in this scenario when players miss a session or three (nothing says ``suck'' like finding out that Fighter's girlfriend the sorceress cohort is actually a more powerful magician than you are). But that can be worked around in a number of ways: the DMG suggests giving out experience bonuses to people who fall behind until they catch up in level and that works well enough. Of course, to actually make use of that you'd have to chuck the idea of not being able to level more than once per session -� which makes characters even more confusing -� but there you go.}

\subsection{Attenuating Advancement: Diminished Returns.}

\quot{``You youngsters have no concept of how difficult it was to get the Doom Glaive.''}

\desc{If one considers advancement at face value: a direct method to prevent adventuring from becoming ``stale'', then it is entirely reasonable to question its inclusion in the game at all. After all, a sixth level party could very plausibly encounter a manticore, a summoning ooze, a dragon, a war party of ogres, a troll, an evil wizard, a dinosaur, a nymph, a mud slaad, a nerra facechanger, a medusa, a circle of myconid, a cathedral protected by a stained glass golem, a cadre of yak folk, an infestation of ash rats, a room full of hammerers, a spawn of Kyuss, or a dreadful cleric with some orcish minions. Or whatever. The point is, you could very plausibly face different opposition every week until half the players move out of town before you ever run out of monsters to fight. The staleness then, comes not at the hands of the players in any case, but for the DM. After all, once the DM has thrown the adventure where an ancient cathedral of Pelor has been taken over by an evil group of Yak Folk who have bound a Janni and forced her to tell them the secret password that allows them to break into the inner cloister without having the stained glass tear itself out of the wall and attack them in order to conduct a foul ritual to transform the daughter of the old king into a medusa and set up some zombie ogres to protect themselves while the mighty ritual commences � that leaves some of the DM's favorite monsters used up out of that level. More importantly however, the players are presenting essentially the same skill set so long as their skill set doesn't change � meaning that the DM can become bored finding challenges for the PCs unless the PCs demonstrably change over time.}

\desc{Be that as it may, the fact is that higher level characters with more magical swag have more abilities than do lower level characters and quite definitely present a face to team monster with more attachments on their Swiss Army knives.}

\chapter{The Manual of Making Things}

\section{Why a Revision to the Crafting Rules?}

\desc{An overhaul to the Craft rules may sound fairly unbalancing, as the current Craft rules were created to prevent characters from making a lot of money and potentially destabilizing their games with an influx of magic items. Unfortunately, like Level Allowance, the heavy nerfing to Crafting resulted in a lot of characters simply becoming unviable, a lot of very dumb things happening all around, and it still doesn't actually stop characters from breaking the game if they really want to. If the party is made out of Elves, they can simply set a single skill rank on fire and announce that they're going to spend 100 years farming, making trained Profession (Farmer) checks every week. That'll get them about 6 gp a week for the next 5,200 weeks � for a total of 31,200 gp at first level before they even start adventuring. And as elves, they can honestly just spend 200 years farming or spend some real skill ranks on that to get even more money.}

\desc{If the DM is willing to simply let players roll dice, have downtime, and purchase magic items of unlimited power, the game is already broken on first principles at first level using the PHB alone. If the DM wants to keep sanity going \emph{at all}, then something in that equation is going to have to go. Probably everything in that equation should go. As discussed in the Dungeonomicon, there is an inherent limit to what players could reasonably be expected to be able to purchase with pieces of gold, so to a very real extent crafting for money is simply multiplying the amount of low-level equipment you have � it doesn't particularly get you more powerful equipment. And of course there's no reason for players to be able to do all of this 9 to 5 working without having on-camera adventures. An adventure where you are running a silk factory and will make a bunch of money as soon as you can stop the goblin syndicate from extorting all your profits is pretty much the same as the adventure where you run off into a dungeon, fight the goblins, and take the money they stole from the silk merchants home in a sack.}

\noindent \desc{So the nerfs on Crafting just aren't necessary. But what actually needs to change?}

\vspace{6pt}

\listone
	\bolditem{Valuable Raw Materials Aren't Valuable:}{This is a part of the rules that makes me cry. Since the amount of value you make each day is based on the \textit{difficulty} of working the material and not on the \textit{value} of said material, there is no way for a goldsmith to stay in business. Gold is very easy to work and therefore the DC to work it is very low, and therefore a goldsmith makes very little in the way of finished product each week. A five pound gold candle holder is roughly four ounces and fits into the palm of your hand, but it'll take a master goldsmith (+10 Craft Bonus) almost a year to finish one (500 gp value, at DC 5 = 50 weeks).}
	\bolditem{The Costs of Materials are WHAT?}{Remember that five pound gold candle holder? It's worth 500 gp and therefore requires 167 gp worth of materials to make it. But it's worth 250 gp just as a lump of gold. So you can buy things as raw materials and sell them as trade goods and make \textit{lots of money}. The reverse happens when you make complex or finely worked items. A masterwork sword is made out of pretty much the same materials as a normal sword and is much more expensive because it's better made. But because the higher quality crafting will make it sell for more down the line, the cost of the materials goes up by a 100 gp. Where does that money go? What are you getting for 2 pounds of gold? Sure, maybe you get some better coal or something, but really, that doesn't even begin to cover it.}
	\bolditem{Field Fortifications Cannot Happen:}{Even the simplest of traps (such as a bucket with some acid in it balanced on a partially open door) has a cost that is very high -- in the hundreds of gp. That means even the most gifted craftsman is going to take weeks to boobytrap a room or lay down some field fortifications. When longbowmen want to hammer some stakes into the ground to protect themselves from the knight stampede that's going to come when the battle starts, the Craft rules essentially tell them that they can't do it. Which for those of us who have seen Henry V, seems unlikely.}
	\bolditem{Risky and Illegal Trades are Pointless:}{Some products are expensive because producing them is risky (poison, flower arrangements from the Bane Mires). Some products are expensive because their production and sale is in some manner restricted by the authorities (shrunken dwarf heads, disrespectful puppets of the king). In the real world, people produce these things because they can charge inflated prices because of the risk. It's a gamble, where sometimes you make big money and sometimes you get killed by hydras or agents of King Ronard. But with craft times directly dependent upon resale value, these crafts are gambles where sometimes you make the same amount of money you would have making night stands, and sometimes you get killed by your own poison or Clerics of Torm.}
\end{list}
\vspace{6pt}

\noindent And with that, here's how we propose to fix this.

\section{The Craft Skill}

\desc{Having multiple Craft skills that each require an independent investment of skill points put an illusory emphasis on the crafting system that is roughly equivalent to what you'd expect from the Rogue's \emph{entire non-combat contribution}. Making a system that actually does have that sort of impact -- or even enough to justify multiple sets of skill point investment -- is just unworkable, so we have to go with the alternative: unify everything under a single Craft skill, and have subdivisions of that skill not cost anything in themselves.}

\subsection{Subskills: Breaking Things Down}

\desc{On the other hand, we do definitely want to have some distinction between making different kinds of things. That was a lot of the point behind the original subskill system, and we're actually okay with there being a difference between low-level people who are good at blacksmithing versus people who are good at tailoring, just like we prefer to have people who are good at singing but can't play a piano nearly as well.}

\desc{Your number of ranks in the Craft skill directly affects how good you are at the various Craft checks, and it \emph{also} affects how many areas of item creation your character is skilled in. If you have a particular subskill you can use your full Craft ranks on any Craft checks involving that subskill. You can \emph{also} combine the checks to make a more complex item which involves multiple Craft checks if you have all the pertinent subskills. Otherwise, you add only half your Craft ranks and have to perform any checks individually. You can also qualify for anything which has a particular subskill of Craft as a requirement; for instance, a PrC which required Craft (weaponsmithing) could only be taken if you had the Blacksmithing subskill and enough Craft ranks.}

\noindent There are, of course, a finite number of subskills you can meaningfully take:
\listone
    \item \ability{Alchemy --}{Basic alteration of chemical materials; obscure liquids, metallurgy, pretty much anything that would fall under the heading of ``chemistry'' or ``materials science'' today.}
    \item \ability{Blacksmithing --}{Metal weapons and armor, utilitarian metal objects. This covers the working of iron, but it also covers other metals which are valuable because they're durable rather than pretty.}
    \item \ability{Ceramics --}{Hard and brittle stuff that isn't quite rock. Pretty much anything made of clay or glass would use this subskill.}
    \item \ability{Jewelry --}{Gemstones and soft metals. Basically, if it's used just because it's pretty, it probably falls under here.}
    \item \ability{Mechanics --}{You can make intricate and complex devices out of component parts. Clockwork devices and siege engines are the main sorts of things you might make with this subskill.}
    \item \ability{Papermaking \& Bookbinding --}{You can make the material for books and other documents. Actually putting information into them is another matter.}
    \item \ability{Stonework --}{Making things out of rock. This includes things like architecture and also things like special Dwarven armors.}
    \item \ability{Weaving/Tailoring --}{Cloth, leather, and other flexible materials.}
    \item \ability{Woodworking --}{Wooden weapons and armor, bows, carpentry, whittling. Pretty much anything with reasonably sturdy plant material.}
\end{list}

\vspace{6pt}

\noindent Yes, a character who has 8 ranks in Craft has all the subskills. We're okay with that, someone who's working at a post-human level of craftsmanship \emph{should} disregard that sort of thing.

\subsection{Using Craft}

\desc{Linking the creation time of a object to its cost is dumb. It results in a lot of the nonsense mentioned earlier, and it hurts suspension of disbelief that something made out of gold is harder to make than the exact same thing made out of silver.}

\desc{So here's how this works: making something has a basic DC at low levels, and as things get more complex the DC increases. The basic DC for making an item is 10. Modifiers are added to this based on how complex the object being made is and how many specializations the various parts fall under.}

\desc{Making a Masterwork item adds 5 to the DC, and it also takes additional time. Combining multiple Craft subskills also adds a +5, such as making a thinaun dagger out of iron (Blacksmithing to make it a dagger, Alchemy to transform the iron into thinaun). Significant changes to a material like making glass an unbreakable substance adds +10 to the DC.}

Magic Items 
Magic item properties come in minor, moderate, and major varieties, and making a magic item takes only time and the ability to actually produce it. These are the rough times it takes to create a magic item, per property:

\listone
\item Minor Magic--1 day 
\item Moderate Magic--5 days 
\item Major Magic--50 days. 
\end{list}


\subsubsection{Mastercraft}

Okay, so there are people who are really good at making high-quality goods. Let's talk about that a little bit, shall we? 

You can enhance and enchant non-masterwork items. I've never seen any particularly compelling reason why an item has to be well-made for it to hold magic. 

Making a Masterwork item adds 5 to the Craft DC. 

But, there are people who can make *really* good items. They've taken the Mastercraft feat.

\subsection{Crafting Feats}
There's several things you can do with Craft, so it makes a certain amount of sense that there are several Craft skill feats. Some of them are multiple-dependency feats. 

\begin{multicols}{2}

\skillfeat{Craft Magic [Skill]}
{What you make is simply magical.}
{Craft and Spellcraft Ranks:}
{You can craft magic items with a caster level equal to your character level. You Get Scribe Scroll and Brew Potion. Spellcraft is a class skill for you.}
{Craft Wand, Craft Wondrous Item}
{Craft Magic Arms and Armor, and Craft Rod}
{Craft Staff and Forge Ring}
{You can craft artifacts, with an inherent level of up to your character level.}

\skillfeat{Swift Crafting [Skill]}
{By knowing the secret time-saving techniques of the master craftsmen, you can take significantly less time to make things.}
{Craft Ranks:}
{You may take 10 on a craft check without increasing the amount of time you spend working.}
{Creation time is decreased to 80\%}
{Creation time is decreased to 60\%}
{Creation time is decreased to 40\%}
{Creation time is decreased to 20\%}

\skillfeat{Alchemy [Skill]}
{}
{Craft (alchemy) Ranks:}
{+3 to Craft checks involving Alchemy}
{You can make alchemical items such as antitoxin, acid flasks, and so on. You do not have to be a wizard to make any of these items. You also get Brew Potion.}
{Your understanding of matter lends itself to alchemical combinations of base materials. So one could imbue glass with the strength of adamantine, or adamantine with the lightness of mithril. Anything goes, so have fun and be creative.}
{Your skill with alchemy lets you make one material wholly into another, as long as they are vaguely similar--so you can turn steel into adamantine or wood into darkwood (or whatever). Yea, ye may turn heavy lead into bright gold, even. However, you're level 11 and have hit the Wish economy, so that's basically good for scamming the people lower down the economic latter.}
{You can make absolute materials--completely unbreakable, or cuts anything, or weightless, or what have you. Have fun with that.}

\skillfeat{Mastercraft [Skill]}
{A feat that almost all serious craftsmen aspire to.}
{Craft Ranks:}
{You get a +3 to Craft checks}
{You can put that point of Masterwork bonus to anything weapon, armor property you want. 
Melee Weapon Properties: 
Attack (Supposedly, its balance) 
Damage 
Critical threat range (+1. Added *after* any doubling is done) 
Hardness/HP (2 points of Hardness and 5 HP per Masterwork point) 

Ranged Weapon Properties: 
Attack 
Damage 
Critical multiplier (+1) 
Range Increment (+50\% range increment. Adds up with the Sniper feat, so someone with a long-range bow and Sniper has double the range) 

Armor Properties: 
AC 
ACP (-1 per point of bonus) 
ASP (ditto) 
Weight (-10\% per point). 

Tools/Items: 
+2 bonus to relevant activities per masterwork point. The point cap of Craft Ranks/4 is still in effect.}
{You can add multiple masterwork bonuses to a piece of equipment, at the extra effort of +5 for each point of masterwork bonus. Masterwork bonuses stack with magical enhancement bonuses, but there are a couple of rules concerning their use: 

-Weapon/armor properties, once enhanced, cannot be improved until all the other properties have been improved. So, no, you can't make a weapon with an attack bonus of (Craft check - 10)/5 or add a ton of AC to a set of armor. 

-The highest masterwork bonus you can produce is equal to your ranks in Craft/4. There's nothing to stop you from making *every* item property have that bonus, though. 

-Masterwork bonuses also take skill to use. The best set of tools isn't much help to a rank amateur, after all. So the highest masterwork bonus you can use is 1/3 your character level; this only applies to weapons and tools (and, even then, it doesn't apply to weapon durability). If you get a weapon or item you don't know how to fully use at the moment, you will gain more bonuses as you gain more skill (i.e., gain levels).}
{When you make something, you can add Masterwork points equal to your Intelligence modifier, without increasing the check DC. You're just that good. You can also still take the time to add normal Masterwork points (with the same DC increase).}
{Masterwork points are equal to 3/2 your Intelligence modifier (round down).}

\end{multicols}

Craft Bonuses

In some figuring I did today, I worked out that a level 16 character who whores Intelligence can get a +63 bonus to Craft. That actually seems about right. I used a couple of rules for figuring bonus: 

-You can only get bonuses from one set of tools involved in the crafting (use the highest) 

-You can only get two bonuses from a feat or specialization involved (these are all +3 bonuses, so it's a total of +6). 

So for a level 16 craftsman of the right race who's serious about getting his work: 
19 ranks 
+3 (Mastercraft) 
+3 (some feat bonus or specialization) 
+10 (Masterwork tools from someone who was at least level 17) 
+16 (Enhancement bonus on tools) 
+12 Intelligence bonus (18 + 2 (Racial) + 4 Ability Boost + 5 (Wish) + 6 (Enhancement)) 

So I suggest you follow the same guidelines.


Materials: 
Having the necessary materials is essential, of course. But you can also use materials to speed the process along. Gemstones, oils, minerals, metals, the body parts of weird creatures...These should be used to speed up crafting or just because they sound cool. 

Magic: 
Whether you're casting Burning Hands into a sword until it understands what's expected of it, or etching the runes for Burning Blade onto it, magic helps things along. This is the standard Craft Magic Arms and Armor, but I *will* expand on runes. 

Drama: 
This is not worked nearly enough in the SRD crafting rules. Remember when you read The Crystal Shard for the first time and Bruenor was crafting the hammer? He'd found what he believed to be a magic place, and he worked during the full moon, around the high point of summer, and was able to make a kickass magic weapon in three night's work. Although he did have runes. So, you know what? Without anything else, someone should be able to make a magic weapon by making the creation awesome enough. For example, if someone crafts a sword and then prays hard enough over it, or writes prayers to Pelor or whatever on the blade as he sings, he should be able to get Pelor's attention and get a blessing on the item. Seriously, the gods are actually there and you ought to be able to get a hand from them if it's dramatically appropriate. There's other ways, too. If you slew the Sun Emperor or the Pale King of the Shades with your rapier, it's pretty awesome if your rapier was was changed in the process and burned undead or sucked the life out of people. 

Edit: Other methods of getting some Drama going-- 

Timing: Working by a full, half, or new moon, or during a special set of holidays or around the time of an eclipses, or making a weapon specifically for a purpose or something...that's totally cool. 

Events: Sometimes an item is cursed or blessed based on the events it was involved in. 




\chapter{Dangerous Locations: When the Floor has a CR}

\desc{It is an undeniable truth that hunting goblins in a dank warren filled with dead falls and snares is both more exciting and more dangerous than hunting the same goblins in an open field. However, it must be stressed that the way 3rd edition D\&D has traditionally dealt with this -- to give CRs to individual traps as if they were enemy monsters in their own right -- is both unsatisfying and unplayable. The fact is that you probably are \emph{never} going to tell a story about the time your party was shot at by an arrow trap, it just isn't interesting in the same way that overcoming an evil necromancer or slaying a greedy dragon is.}

\desc{And why is that? It's essentially because an arrow trap is not an encounter, it's an \emph{attack}. Just a single salvo in an ongoing battle between you and the dungeon, not a battle in and of itself. And when looked at in that manner, the problem becomes obvious: a \spell{glyph of warding} is a single spell. Overcoming it is like making your saving throw against the same Cleric's \spell{hold person}, it's simply wildly inappropriate to stop the action and play the battle complete music at that point.}

So what do we do about it? Well, just as one does not stop and record a victory every time you bypass a summoned monster or overcome an opponent's thrown javelin, we shouldn't be worrying about the CR of individual traps. No, we should be concerned only with the CR of \emph{areas} that have traps in them. For one thing, this means that we don't have to have endless arguments over whether people should get the XP for bypassing the \spell{symbol of pain} on the door if they came in through the floor. For another, the act of avoiding \emph{that} stupid argument really helps to encourage characters to play things a bit smarter and not simply run through the ``Hallway of Leveling'' where they go up a level every thirteen traps and get to 20th level in less than 150 doors.

\section{Location CRs: Quality and Quantity}
\vspace*{-8pt}
\quot{``Why check the door? Maybe because there was a trap on \textbf{every single other door in this entire complex?!}''}

\desc{To a limited extent an area can become more dangerous by making traps more ubiquitous. We say a ``limited'' extent because there is a profound sense of diminishing returns when the chance of encountering a trap equals one. Our classic example is the Citadel of Fire, the castle that is the home of the Efreeti King. It's \emph{on fire}. Every square is \emph{on fire}. Every door is \emph{on fire}. And if you go there, \emph{you'll be on fire}. To an extent, that means that the kind of dangerous area that you might have seen in the Lizard Temple when you were 4th level is now every square on the battlemat. That's bad. But it's not unconquerably bad. It doesn't take a whole lot of Fire Resistance to survive in that kind of environment, and you don't have to be amazingly high level to get your grubby mitts on that kind of fire resistance. The fact that every single doorknob and chair is on fire in the Citadel essentially just means ``Only Adventurers with Fire Resistance can Adventure here'' or even ``You must be at least as tall as this sign to attack the Citadel.''}

\desc{And the effect would be pretty much the same if you just had to wade through a \emph{moat} of Fire. There are literally dozens of rooms in the Citadel of Fire that are on fire without this increasing the difficulty of your assault in any way. And that's OK. In fact, people would be slightly offended if large amounts of the Citadel of Fire were not in fact on fire, which would be the logical way to do it if you were handing out XP or construction costs on a per flaming room basis. It adds to the immersion to have some relatively homogenous fantasy environments.}

Practically speaking, this means that by the time you have put in enough of a single type of difficulty that the players will not plausibly be able to complete their quest without taking appropriate precautions, the CR of the location shouldn't rise any more by adding more of the same difficulty. And that goes for more than just places being on fire. If there are enough pressure plates linked to arrows that the PCs aren't going to get through alive without the Rogue taking 20 on her Search checks, throwing in some more arrow traps (or tripwires, or anything else that the Rogue can find and bypass by just taking the time to search thoroughly) doesn't make the area any more difficult. A cave at the bottom of the sea isn't any more difficult when it's \emph{completely} full of water than when it's \emph{mostly} full of water -- you still need \spell{water breathing} just to get there.

\section{WWMD? Disabling Traps.}
\vspace*{-8pt}
\quot{``A paperclip can be a wondrous thing. More times than I can remember, one of these has gotten me out of a tight spot.''}

\desc{The Disable Device Skill is extremely powerful and amazingly bizarre. You don't need it to bypass a trap, there are dungeons full of Kuo Toans who have no more Disable Device than you do who bypass traps every day. What Disable Device \emph{does} do is allow you to interfere with the mechanisms of mechanical and magical devices such that they don't get in your stuff when you \emph{don't} have access to the special catch or magic word or whatever it is that you're supposed to have. In short, any fool can press an off switch or simply not step on an on-switch; Disable Device allows you to shut things down \emph{without} access to those things.}

\desc{Once you have found a trap with the Spot skill, it requires no skill roll at all to simply walk around it. If you discover a pressure plate, you can normally expect to simply step or jump over it without even making a Disable Device check. What Disable Device let's you do is set the plate to not trigger if you do walk on it. Often that's pretty pointless, but sometimes it's pretty useful, especially if you're up against a "trap" that is a siege defense or hostile spell (such that its normal deactivation trigger is far away). Remember however, that you can still activate traps by any of a number of means without actually being in harm's way. Summoned monsters, tossed barrels and the ubiquitous 10' pole have been used by generations of adventurers to activate traps from 10' or more away. Again, that totally works and requires \emph{zero} ranks in disable device. However, sometimes you don't want a trap to go off at all or a trap can go off virtually limitless numbers of times -- that's where disable device comes in.}

\desc{So what counts as a device? Well\ldots\ \emph{everything}. Every mechanical or magical effect is a device. A \spell{Wall of Force} is a device as is a giant stone block that is set to fall down on a foolish intruder who breaks a trip wire. A character with sufficient Disable Device can successfully turn off any magical effect or prevent virtually any cause and effect chain from occurring. You can stop an avalanche (DC 15) even after it has begun (DC 35). You can remove any permanent magic effect, even curses like \spell{Cause Blindness} (DC 32). What you \emph{can't} do is disable instantaneous effects. \spell{Flesh to Stone}, therefore, is out of bounds for disabling, as is \spell{Wall of Stone}. Sorry, once an instantaneous effect has gone off, there's nothing left to disable.}

\desc{How does that work? I have no frickin idea. Rogues, Thief Acrobats, Ninjas, and Gadgeteers are capable of simply turning off \spell{Geas} and there's no physical explanation for how it is that they do it. The fact is that most of the devices in D\&D are beyond my understanding. I don't know how a \spell{symbol of death} works, I don't know how the magical energies stay in place for weeks or years until activated, so I don't know how a Ninja goes about making those magical energies dissipate harmlessly without entering the kill zone. I do know that he can do it, and if required I can make something up that sounds cool. That's a DM's job, after all.}

\abox{Item Spotlight: Bag of Flour}{The bag of flour can be used to disable any rune or sigil without meaningful risk. A magical rune can only detonate if it is uncovered. So if you throw some flour on it, the symbol can't ever explode and is now completely safe. You may want to put the flour on the end of a pole because moving your hand \emph{close} to a rune may trigger it before the flour lands.}

\section{I \emph{live} here: Setting off Traps}
\vspace*{-8pt}
\quot{``How did those gnolls run through that hallway if the whole thing collapses when people are in it?''}

\desc{The common conceit of trap placement is that they automatically go off against player characters who don't find them and automatically don't go off against Team Monster. Needless to say, that's ridiculous, and it actually harms the game when you implement it. While there are magical traps that are virtually guaranteed to go off against certain kinds of creatures and are nonetheless bypassable with something as simple as a command word, those are not PC/NPC selective. A command word bypassed Symbol will go off against any creature that doesn't say the magic word. That means that creatures without language capabilities like bears holding sharks or remorhazz will set those traps exactly as PCs who don't know any better would. It also means that any player character in the correct position can simply \emph{listen} for the command words that Goblins use when safely passing over the danger zone and use it themselves. The base DC is only 15 so the challenge here is actually getting into position to observe enemies bypassing magical traps rather than the replication of the technique itself. The bypass words on magical symbols are pretty forgiving, they can be spoken by blink dogs, Sahuagin and Xorn without serious risk of misunderstanding.}

But what of other traps? Mechanical traps go off mechanically, which means that to make them go off you have to \emph{do} something to make it go off. And that means that there is a chance that even someone who doesn't have a clue what they are doing might simply happen to not set off the trap. Life is filled with Mr. McGoos and if there is \emph{any} path to walk across an area without setting off a pressure plate there is a chance that people will happen to do so. And yet, if there isn't a way to move past a trap, there's a whole area that the residents of an area have to avoid altogether (or just be immune to the effect of the trap). Here are some common trap triggers:

\listone
	\bolditem{Opening a Door:}{This is a common and fun one because unless someone decides to go through the wall (and sometimes even then) the trap will go off any time the door is opened. This can either be placed on "fake" doors that the occupants have no intention of ever opening, or it can be put on doors that are used frequently if there is a separate switch to deactivate the trap (be sure to get buzzed in). The important part about this is that an opening trigger will go off any time the door is opened normally. If you cut a hole in the middle of the door and squeeze through it, you're probably safe. After all, the door itself is acting as a switch in this case, methods of entrance that don't literally involve turning that hinge often don't involve pulling the switch.}
	\bolditem{Tripping a Wire:}{Strings and wires can be strung in walkways at anything from ground to eye level. A trip wire sets off a trap when it is broken or pulled upon, and thus won't go off at all if creatures shorter than the wire run underneath it (barring polearms and the like). A tripwire lower to the ground is more likely to be randomly stepped over than is a higher tripwire, but less likely to be seen. Several trip wires can be run in tandem across a walkway to virtually guaranty that a passerby will sever them, but in doing so they become a lot more visible. In general, a trip wire can go off 25\% of the time when someone moves through its space and have a spot DC of 20, go off 50\% of the time and have a spot DC of 15, or go off 100\% of the time and have a spot DC of only 10. A trip wire can be severed without triggering the trap by holding both ends of the wire and slicing out the middle -- but this requires a Disable Device check (DC 20). Failure triggers the trap. A tripwire can be triggered from range by throwing a chair at the problem, or with an arrow (against projectile weapons a tripwire has an AC of 13, against a larger object such as a barrel or a couple of cabbages tied together the AC is negligible).}
	\bolditem{Pressing a Plate:}{Bizarrely complex mechanisms can be hidden inside of walls and a pressure plate is as good a manner as any to get those mechanisms up and working. I seriously don't have any idea what the mechanical pieces under the floor look like, and neither do you. And that's generally OK. Mostly players won't respond to pressure plates by breaking the floor or walls open to get at the clockwork (though that is a viable option), mostly players will gamely accept whatever fate the pressure plate has in store for them. Without tearing up the scenery, characters can disable a pressure plate with a Disable Device check (generally DC 20, though more awesome plates exist). Pressure plates can be disguised as regular floor and are often quite difficult to spot (DC 16-30). A pressure plate can be as small as a single out of place brick or floorboard and may go off quite rarely (1-5 times out of 20 when someone moves through the space), this has the advantage that characters ``in the know'' can step over it (though enemies are presented with the same option). Alternately, pressure plates can cover entire squares, being triggered automatically if any creature heavier than a specific cutoff enters the square. In any case, characters can fly over a pressure plate or climb along the wall and simply never activate it.}
	\bolditem{Getting Stabbed:}{The old ones are the good ones, and many a trap has been simply to put pointy bits on areas that a character might step on, touch, or fall into. One can with exaggerated care simply step over such things, but in the heat of battle this may be pretty difficult. A single caltrop or blade is rather unlikely for someone to step on (a 1 on a d20 unless the character is crawling or otherwise stepping on more of the square than one might expect), and can be quite difficult to find unless one is specifically looking for it (DC 18 to spot). An area covered with spikes, caltrops, or blades is generally pretty obvious (DC 5 to spot), but it is generally assumed that anyone who moves into a covered square will step on one unless they take some sort of precautions. Caltrop covered terrain is difficult terrain, and characters who move through it at faster than a \half speed walk are going to step on something they'd rather not unless they make a Reflex Save (DC 20). Characters standing in an area covered with caltrops or the like are denied their Dex bonus to AC unless they have 5 ranks in Balance or allow themselves to step on something every time they are attacked.}
	\bolditem{Offending a Glyph:}{Magical runes have at times been implied to have the power to determine a character's alignment, their level, their class, even what they've eaten recently. That's not good for anyone, and we cannot suggest that it be allowed. So here's what Runes do: first, they are constantly taking 20 on a Listen check. That means that you need to make a Stealth check DC 21 to sneak past one. It also means that they will, generally speaking, hear a command word to turn off or turn on. A Magic Rune can also have a detection spell imbedded in them, which last until the rune triggers. So a rune might be set to go off as soon as a source of ``Good" was brought to within 10 feet of the Rune. A Rune might also simply be set to go off whenever any creature moves through its area while it is active (being activated and deactivated with command words set when the rune is). The parameters of a rune can be determined with a DC 20 + Spell Level Knowledge (Arcana) check.}
\end{list}

\section{Facing the Architect: The CR of Locations}

\desc{When you adventure in a dangerous or exotic location you are essentially encountering the architect of that location. Each trap, obstacle, and danger of the region can be looked at as the contingent spells and attacks of the force that put that together. Sometimes a devious maze is engineered by a mad architect or fabricated by an elusive wizard and this is in fact literally true. Other times the Forest of Dread is just really dangerous on its own lookout and the only ``architect" involved is just the DM.}

\desc{The importance here is that an individual \spell{fire trap} isn't really an encounter. It's a single attack, and a pretty ineffective one at that. When the wizard tries to soften you up with his \spell{explosive runes}, that's a lot like the same wizard softening you up by conjuring some celestial badgers and sending them around the corner to engage your forces.}

So while we definitely do not suggest doing something dumb like giving out XP for each trap bypassed, we do encourage you to consider the traps in an area to collectively be an opponent. An opponent that spends a lot of time hiding and taking opportunistic attacks. The Kobold Warrens, for example, have a number of trip wires set to launch crossbow bolts at anyone tall enough to pass through them. In an ideal world, the trip wires would be fairly visible, but in the heat of battle characters may feel compelled to chase after kobolds through the strings.

\subsection{Structuring Encounters in a Day}

\desc{Challenge Ratings have a real utility as a DM, but do not substitute for having a decent idea of what your party is capable of. We're going to go back to the Giant Scorpion a few times, because it's a very poignant example, but we could just as easily be talking about Fairies or Elementals. The Monstrous Scorpion comes in a variety of CRs based on its size and overall awesomeness. Don't be fooled: in reality a monstrous scorpion is essentially of identical difficulty regardless of size based entirely upon what the players are capable of tactically. The Monstrous Scorpion has no intelligence, no ranged attacks, and no interesting abilities -- it's just a biological construct that happens to be exceptionally tough in its one-dimensional way. If you can simply get to longish range (or \emph{fly}) and use ranged attacks, you win. It'll take a while, but you will win. It doesn't really matter what level you are, or how strong your ranged attacks are, victory will be yours. On the other hand, if the Scorpion is presented as a closet troll, it'll mess you right up.}

What the CR grants you as DM then is a basic idea of how much ``resources'' an encounter is liable to use up. The Scorpion, for example, will use up a lot of arrows and not a small amount of time. It probably won't cause any damage if the players play it smart, but it will drag things out for a bit. Higher CRs will take a bite out of the arrows of higher level parties and so on. Still, the fact is that in no way will facing an appropriately CRed monster use up the 20\% of your resources specified by the DMG. Not at any level. What kinds of resources will be used up will depend upon the types of opposition:

\begin{list}{}{\itemspace}
\bolditem{Traps:} Trapped locations of an appropriate CR are generally speaking time sinks more than anything else. At levels 1-6, the characters will normally Search regions that are known to contain traps, which reduces the character's speed through the area to 5' per 6 seconds (about \half MPH or 0.9 KPH).
\item
\item So even though we're looking to completely toss the idea that players should actually \emph{get} anything for necessarily killing ``Ogre Thug \#2'' that doesn't mean that he shouldn't be there.
\item
\item As player characters become higher level they can take on more opposition. This does not necessarily mean they should be confronted with \emph{more powerful} opposition, but they should certainly encounter more of it. A Lunar Ravager and a Sand Giant are basically two large sized men with funny colored skin and a bad attitude. The fact that one is massively more powerful than the other is a staple of the D\&D system, but doesn't make an extremely exciting story. Having just looked up the stats of a Lunar Ravager and a Sand Giant I am confident that defeating a Sand Giant is a more difficult feat -- though of course it is not a more \emph{impressive} feat since as previously described both opponents are just 3 meter tall dudes with funny colored skin and a sword. Taking on 45 bug bears, which is something the stronger party could easily accomplish is however much more impressive than defeating 15 gnolls, as would be a light romp for the party who might otherwise face the Lunar Ravager.
\item
\item It is therefore important to note that parties should generally speaking not run into level appropriate opposition until quite late in an adventure. It's fine for a boss to be a True Fiend, Wizard, or Androsphix who is 2 or 3 CRs higher than the average character level in the party, but the vast majority of opposition should be several levels lower and a crap tonne more numerous than the PCs. This isn't just because this sort of thing keeps cleaving and \spell{fireballs} as reasonably viable tactics, but because high level combats really do involve lots of participants on both sides of the combat kicked out of the battle from time to time and if there's only one enemy it gets really anticlimactic.
\end{list}

\chapter{Playing the Game}

\section{What's that Noise?! Playing at Low Level}

\desc{There is a reason that the XP charts in the DMG completely fudge character levels 1-3. That is because those levels genuinely don't have a good consistent rubric for how powerful things are. There are damn few first level PCs that wouldn't go down if they took a lucky crit from a kobold's small light crossbow, and a first level Wizard has a pretty reasonable chance of taking down an orcish warrior by hitting him with a club. At first through third level, combat really is anyone's game and it is strongly advisable that the PCs outnumber their foes in the majority of confrontations at this level of conflict.}

The TPK (Total Party Kill) is a very real concern for 2nd level characters, because the success or failure of actions is so very random. A run of bad luck can quite plausibly wipe out even a well-played low level team of adventurers quite easily and it is recommended that DMs use discrete encounters at these low levels in order to minimize the effects of having characters getting dropped by allowing the remaining characters to consistently revive fallen comrades.

\section{The Rigors of Command: Playing at High Levels}

\desc{A high level party isn't really ``adventuring'' in the traditional sense any more, or at least they probably shouldn't be. Instead, they are playing a whole different game -- a \emph{strategic} game. Characters who make it into the Epic landscape can in fact become gods according to long standing D\&D tradition. Along the way it behooves you to conquer and administer stuff in order to propel yourself to victory.}

\desc{More detail will be gone into in the Tome of Virtue, as the high level world is a really strange place. Almost all the source material from Arthur and Beowulf to Theseus and Ulysses involves characters who are somewhere between 1st and 6th level in D\&D terminology. Stories which involve a 10th level adventure are extremely rare. Perseus killed Medusa (CR 7), and Bellerophon killed Chimera (also CR 7), but they both pulled some fancy equipment and cheesy tactics to pull it off (Bellerophon seriously had a flying mount that was faster than Chimera and shot arrows at the beast until it died).}

If one insists upon continuing with powerful characters in an adventuring role, there is a primary conceit which must be embraced: all adventures must be timed adventures. A 14th level Wizard can, with sufficient preparation, kill any challenge in D\&D without exception. And while sitting around planning the perfect murder of a red dragon or the perfect heist of a major artifact is interesting as an intellectual exercise, there is no way that represents an ``adventure'' in the way we use that word to describe 4th level characters breaking into pantries and stabbing people in the face for money.

\chapter{Magic}

\section{Illusion Magic: I Don't Believe This Crap}

\desc{Illusion magic has the distinguishing characteristic of being either the most powerful school of magic, or the least -- entirely at the whims of your playgroup. Illusions can be used as distractions, threats, enticements, concealment, modes of communication, prisons, attacks, disguises, false targets, entertainments, misdirections, religious inspiration, incitements to riot, madness provokers, commercial fraud, redecoration, time wasters, limited-use ability wasters (like prepared spells, scroll spells, or use-per-day spell-like abilities), or traps (in conjunction with dangerous terrain, monsters, substances, events, or magical effects). And that's just using the 1st level spell \spell{silent image}.}

\desc{People just don't expect their senses to lead them wrong, even in a world where people know that illusions exist. I mean, if a wall of fire suddenly pops up out of nowhere, it's actually more likely to actually be a real damaging wall made out of magical fire than it is to be an illusion of the same thing. And truthfully, who wants to pop a hand in to check? Not me either.}

\desc{What this means is that illusions are incredibly powerful because they allow such perfect forgeries of the real world. The downside of this is that lots of DMs try to counter the efforts of creative players by using a particularly harsh interpretation of the Disbelief rules in order to nerf illusions out of existence. It works like this: by the rules, you get a Will save vs. an illusion if you ``interact'' with it. DMs looking to throw salt in an illusionist's game usually allow that to mean ``in the same square as an illusion'' or ``looking at it.'' You also automatically make a save if you have ``proof that an illusion isn't real.'' What that means is anyone's guess, because in D\&D even the most unlikely circumstances could quite plausibly occur without illusionary influence. A silent orc moving through the grass might be a \spell{silent image} of an orc, an orc in a \spell{silence} effect, an incorporeal orc, or just an orc who happens to be really sneaky. Once you disbelieved the illusion, you suddenly got to see through its like it was transparent.}

\desc{Usually, DMs looking to punish illusionists will give multiple saves per turn, and then at some point just say that the target has automatically disbelieved the illusion, and this is possible only because the rules regarding illusions were written in the style of previous editions of D\&D called ``Rule 0'' where playing a pick-up game of D\&D involved a few hours of discussion about how the DM handled most effects. The current edition of D\&D (3.X) mostly did away with this because it sucks up valuable game time to have arguments about D\&D rules and it was the worst part of playing the game; however, illusions were never fully overhauled, so we are still stuck with this noise.}

\desc{Potential effects of illusions are also hotly debated. Some genius at WotC has laid down the law and said that the various \spell{image} and \spell{illusion} spells don't cause darkness, but that doesn't stop them from creating opaque mist or smoke or dust, obscuring objects, or even autumn leaves that drift around a person's head and float away from his touch, effectively blinding a person from dangers as well as complete darkness. Additionally, there are DM vs Player wars where DMs try to interpret the ``single object, creature, or force'' line to mean ``no more than one person or a monster in the illusion'' and players respond with things like ``its an illusion of a single force that summoned many monsters like the spell \spell{summon monster} or \spell{gate}'' or ``its one object connected by many invisible threads.'' Other DMs and players are convinced that you control all visual information in the Area of Effect, while others agree but say things like ``you can't trap a creature in a bubble with visual information on the inside that mimics the world except for some key creatures/object/terrain/effects, but people outside see him as normal because his image is on the outside of bubble.''}

In the end, it's a mess because the current rules can be made to do amazing things by creative people, but those amazing things break the level system and that means that DMs are forced to punish players for their creativity, thus hurting everyone. That being said, here are some playable rules regarding illusions that won't cause you to stab out your own eyes.

\chapter{Magic Items: Swag That You Brag About}
\quot{``No� \emph{This} is a knife.''}

\desc{Any man on the street with a few nasty scars and good tale or two can call himself an adventurer, but there are a few true tests that can determine the difference between a talented liar and the kind of person who considers fighting dragons a slow day at the office. It's not demonstrable skills, or nerve, or even a history of past accomplishments. It's magic items.}

\desc{I know that this sounds counterintuitive, but work it out for a second. Put a fighting guy with just better than average stats, some class features, and some HD out on the front line, and what do you have? Basically, you have a giant, which means ``NPC''. Without a magic weapon to bypass DR, good armor to avoid being clobbered, healing magic to recover for the next fight, and crazy extra effects to surprise an enemy like dimension dooring with the Cloak of the Montebank or reflecting a spell with a Ring of Spell Turning, you just don't have enough mojo to call yourself a PC. Monsters have bigger raw stats and better recharge times on their abilities, so if you don't have something extra you aren't going to be able to compete.}

\desc{Magic items are the true test of the adventurer because they say ``I'm trying to grow my power asymmetrically and I'm willing to do it by stealing it from other people who are also growing their power asymmetrically.'' Anyone can fire a bow at a manticore in flight, but only an adventurer is so concerned with power that he'll track that manticore to its lair and risk getting boxed in by a family of manticores just for the opportunity to root through its dropping on the off chance some would-be hero got eaten by the thing and a magical trinket or two survived passing through its innards.}

\desc{Some would called that ``greedy'', but in fact that's ``hardcore.'' Real adventurers are willing and able to risk their life on just the hope that their efforts will bring magical loot.... and its worth it. The more magical loot one gains, the more able an adventurer is to survive the next terrible risk that might offer magical loot. Heck, just holding onto any reasonable-sized pile of magical loot means that one is tough enough to face off against most people who would want that stuff.}

That being said, magic item creation and ownership is a big deal, and should not be the abbreviated (and broken) process that you see in the DMG. Here are some rules to make it sane and easy.

\section{The Core of Magic Item Design: Don't Do It Like Diablo}

Diablo II is a great game, but literally every single thing it does with magic items is bad for a table top role playing game.

\section{Magic Items with Class(ifications)}

It's all well and good to talk about ``Magic Items'' as a whole, but there really is a very big difference between piles of scrolls (which have a modest effect on a single adventure) and a flaming sword (which has a modest effect on all your adventures). Not as much as the writers of the DMG seem to think -- but it's certainly there. An item with ``unlimited'' charges is actually \emph{going} to be used a specific and finite number of times before the character stops adventuring, the item is destroyed, or the character starts using something else. While there is no specific limits to how many times you \emph{can} swing a sword, fundamentally there is a limit to how many times you are \emph{going} to swing that sword.

\subsection{Activation vs. Constant}

\desc{Walking around in a suit of magic plate assumes that as long as it's worn properly, then without any prompting on the part of the character the suit is providing an enhancement bonus to AC. It's the same with a Ring of Fire Resistance, an Amulet of Natural Armor, and a host of other items. Similarly there are items such as magical swords that can be used round after round generating their effects time and time again without rest or recharge. It's the same with wands, most rods, the vast majority of rings, and collapsible animated ice swans. In either case these items are Constant items. Item providing a Constant effect (or usable in a Constant fashion) must be specifically targeted by a \spell{dispel magic} to be affected.}

Other items need to be activated before they work. Scrolls and potions are classic examples, but a good percentage of magic items fall into that category. These are Activation items. Activation items have to be in some way prepped up before they are used. A scroll must be read and deciphered; a potion must be shaken up and opened. Any Activation effect can be dispelled in an Area \spell{dispel magic} or person-targeted effect (as appropriate).

\subsection{Ownership is a Privilege, Not a Right}

\desc{Several systems of magic item ownership have been attempted in the past. The current system is a pseudo chakra-based BS where magic power is limited by one's body parts where some items are dedicated to a specific body part (magic helmets like a Helm of Telepathy) and others are supposed to be put on the body but get to ignore this system (Ioun Stones are a classic example, as they float around your head and just give you some magic powers but you can have a dozen or a hundred doing that job and it's no problem). Other magic items generally sit in your pocket until you use them, and its assumed that your backpack is stuffed with them (staffs, wands, rods, most rings, scroll, potions, special-use weapons like \spell{ghost touch} swords, and about half of the wondrous magic items).}

One of the dumber parts of D\&D has been the tally sheets of items where determining the effects and bonuses on a single character starts to look like doing your taxes. That's lame and slows down the game, and together that's unacceptable. Since we have removed the GP and XP rules from magic items, which were previously the only limiting factor on magic item abuse (which we did because they didn't really work), we have instead have these new rules for magic item ownership:

\begin{enumerate}
	\bolditem{Eight Item Limit:}{Adventurers can have up to eight Constant effect magic items operating on their body at one time. Any items past that limit (8), and the most recent items won't work. This can be any combination of items, but available space on the body is a limiting factor, meaning that you definitely can't wear two sets of chain armor at the same time (no way to get two torsos), but you can wear several amulets (assuming you have a neck, which most oozes don't) or even two helmets (assuming you have two heads like an ettin).
\vspace{10pt}

	Carrying around Activation magic items is no problem though. You can have bandoleers of potions across your chest, a scrollcase full of scrolls, or a magic arrow hidden up every seam in your clothes and every body cavity, but only eight items can currently be providing Constant benefits.
\vspace{10pt}

	A constant item must be worn/used and working properly for it to count against the Eight Item limit, and activation items can only be used one at a time. For example, Tommy of the Twelve Magic Daggers can wear a constant effect magic armor, a constant effect magic cloak, and five constant effect magic rings and still throw/activate his daggers one at a time in a round (assuming he can throw more than one each round), but if he tried to use two at a time with Two-Weapon Fight (for example: to benefit from qualities like Defending), then one of those daggers is not working and is basically a non-magical dagger. Some situations may arise where it is difficult to decide if a character is exceeding his limit; and in those cases, use your best judgment (meaning that if you are a DM, be consistent). For example, Tommy might be holding a magic longsword by the blade in his hand, so it's not ``active'' since he can't take AoOs with it and get its bonus and its not providing him with any Constant benefit.}

	\bolditem{True Ownership:}{A person has to willingly put on a magic item and intend to activate it for it to count as active. That means that clever people can't trick you into putting on weak magic items so that your good magic items won't work when you try to use them. Unconscious or helpless characters can have items activated on their behalf (remember that in D\&D unconscious creatures are always ``willing''), so you can put a Ring of Regeneration on an unconscious buddy or put Dimensional Shackles on a sleeping wizard. Command word and spell completion items cannot be activated on someone's behalf (though you are welcome to use them on another character by dint of pointing the wand at your opponent and shooting lasers at them as normal).
\vspace{10pt}

	The flip side of this is that when you put an item down, it still counts as being one of your items for a period of time. This means that when you throw your magic spear it retains any benefits that are dependent upon your level while it is arcing trough the air into the dragon's chest; and it also means that it is not practical to pull a magic skirt off in the middle of combat and replace it with some really cute bike shorts. That's actually a good thing, because while if you're specifically playing Final Fantasy X 2 D20 it is setting appropriate to change your clothes in the middle of combat, in all other settings that sort of thing is just really dumb. Once it leaves your person, a constant magic item generally stops being one of your eight in a d4 minutes. If you're actually dead, your magic items stop counting as being yours the next round.
\vspace{10pt}

	Cursed items are the same. You have to try to use a cursed item before it can affect you. Otherwise, you can just keep it in a box labeled ``Cursed sword: Do not use for stab-ination.''}

	\bolditem{An End to Bonuses:}{Andy Collins talks a lot about the ``big items'' that players need to get in the door at high levels. Mostly swords and shields with bonuses on them. And while he is correct that people \emph{do} need them, I personally think that constantly taking up time worrying about getting another uninteresting ``slightly more magical sword'' is bad for the game. The solution is truly that for magic items to fulfill their duty within the game without being really annoying, they just have to scale by level. So the ``+2 Sword'' is dead. Now there's just a ``Magic Sword''. If you happen to be 6th level when you use that sword, it'll be +2.}

	\bolditem{Artifacts have a Level:}{What makes Artifacts special? Mostly it's that they are a source of power that is completely asymmetric and well outside what the user could be ``expected'' to have. This is represented by an artifact simply being a magic item that has a level on its own time. That means that the first level farmer's daughter who picks up Excalibur (an artifact with an inherent level of 15) gets all the benefits that she would had she actually been 15th level herself (a +5 enhancement bonus, being king of England, the whole deal). A character who holds an artifact of a lower level than herself still treats it as a magic item of her level -- the Artifact's level is a minimum, not a maximum.}
\end{enumerate}

\subsection{Wanna Take Some Body Slots?}

The slot system of traditional D\&D is more than a little bit insulting and carrying it over into this document would be a tragic failure of our design goal to make things not be like Diablo II. So yes, if you want to have every single one of your eight items be a ring, or an ioun stone, that's fine. Heck, you really could plausibly wear eight rings on one hand, there are people who do that sort of thing. If it's really important that you use three different magic crowns, we welcome you to run around calling yourself \emph{The Thrice Crowned King}. Nevertheless, items do have classes that they fit into fairly neatly:

\begin{enumerate}
	\bolditem{Wielded Items -}{These are held in a hand and brandished, swung, or otherwise triggered to activate their power.}
	\bolditem{Worn Items -}{These are placed somewhere on the body in order to unleash their power. While it is possible for someone to wear multiple sets of clothes, or armor over clothes, or even armor over other armor, only the heaviest armor counts as the armor you are wearing for purposes of AC, special abilities, etc.}
	\bolditem{Miscellaneous Items -}{These are items that are used in some other arbitrary way. Their power continues even when not held or worn, which is good because a lot of these items are things like thrones, golems, or crystal spheres that simply cannot be placed on the body at all.}
\end{enumerate}

% Magic Item Creation
\section{Magic Item Creation}

\desc{Building a magic item is a big deal. It is a way to expand one's power and a way to transfer power to your lessers, and in many ways the life of an adventurer revolves around the acquisition of magical loot. If magic item creation is too easy, adventuring is less fun, and if its too hard then people won't do it and resent the system and DM.}

\desc{We know that the current rules don't work. GP and XP costs are things that have little meaning in even a low-level game, and players are notorious for finding ways around them by taking metagame classes like the Artificer, by having cohorts pay those costs, or even by giving morals the finger and having mindcontrolled captured spellcasters do it. That's before we even get to \spell{wishes}, powerful outsiders, or craziness like the Dark craft and soul rules.}

\desc{There is one thing that hurts characters: time. Adventures and stories happen along a timeline, and players may or may not be able to stop during an adventure to build just the right item for an adventure. Even ``downtime'', the time between adventures, is limited because powerful characters attract powerful enemies and predators. Heroes that say ``we'll just take a year off and make magic cloaks for everyone'' are basically saying ``we'll sit in the open and let our potential and actual enemies pick the time and place for any battles.'' DMs can throw enough intrigues in someone's way during that time so that before the first cloak is built that the campaign is over.}

Creating magic items just requires time. There's work that goes into enchanting a sword, forging a blade, smelting the steel, mining the ore, and all that just takes time. If a character is really dedicated, he really seriously can wander off into the hills collecting reddish stones and then heating them up until iron comes out and then hammering the molten metal into a blade and then enchanting it with his power and walking out of the hills with a magic sword. Various portions of this can be expedited by, for example, \emph{hiring other people} to do a lot of that -- so a character can reasonably expect to throw down gold and buy himself a lot of that time back. But if you just have time; time will suffice. Exactly what magical goods are needed or helpful in magic item creation is highly variable campaign to campaign.

\listone
	\item \textbf{Questing for Reagents}\\It is a classic story for those fantasy settings that have on-camera magic item creation that characters must go quest for magical ingredients they need to make whatever the hell it is that they want to make
\end{list}

\subsection{Building a Better Magic Item: the Minor Magic Item}

A Minor Magic Item is one which can be produced in quantity by NPC apprentice factories and can thus be in some sense standardized or expected to exist in major city bazaars. Most minor magic items just provide some sort of unimpressive numeric bonus. The magnitude of that bonus is dependent upon the level of the character who is using that magic item. The rate at which the bonus scales to level varies depending upon what the item is giving a bonus to, and when magic items would provide a fractional bonus always round that fraction up. There are no caps on any of these bonuses. If you're a 19th level guy your sword simply provides a +7 enhancement bonus and that's fine. You're 19th level and you don't even really care.

\listone
	\bolditem{Enhancement Bonus to Weapons ::}{+1/3 per character level.}
	\bolditem{Enhancement Bonus to Armor ::}{+1/3 per character level.}
	\bolditem{Enhancement Bonus to Attributes ::}{+1/3 per character level.}
	\bolditem{Resistance Bonus to Saving Throws ::}{+1/3 per character level.}
	\bolditem{Competence Bonus to Skills ::}{+1 per character level.}
	\bolditem{Energy Resistance to any Energy Type ::}{+1 per character level.}
	\bolditem{Deflection Bonus to AC ::}{\fourth per character level.}
	\bolditem{Enhancement Bonus to Some Other Thing (Natural Armor, DR, SR, whatever) ::}{+1/3 per character level.}
\end{list}

\desc{Non-standard bonus types, or as we like to call them around the office: \emph{bullshit} bonus types do not exist. No, you can't have a Sacred Bonus to your AC or an Insight Bonus to your skills. That stuff is straight up broken and will push characters right off the random number generator. If all of your eight magic items are providing a bonus of some sort, they most definitely should not be providing a bonus to the same number -- that sort of thing really does make the d20 system fall apart.}

\desc{Minor Magic Items which do not provide a numeric benefit usually reproduce the effect of a spell, and are caster level 5 or less. A Minor Magic Item may potentially be traded in the turnip economy. It is conceivable that a man might trade a wand of \spell{cure light wounds} for food or shelter directly. Nonetheless, these items are much more commonly traded for gold, and anything more powerful than a Minor Magic Item is actually less than worthless in the turnip market -- a \magicitem{Frost Brand} or \magicitem{Stone Sphere of Shaz} is really going to draw more fire for a peasant than it's worth.}

More powerful magic items begin with a Minor Magic Item base and layer additional abilities on top. In this way a Sword of Sharpness always provides the basic level appropriate enhancement bonus to attack and damage even while it is chopping the heads off of dudes.

\subsubsection{Building a Better Magic Item: the Magic Weapon}

\desc{Generally speaking, magic weapons start with the basic minor magic item chassis: "Weapon with an Enhancement Bonus" and items more powerful weapons also have an ability. There are two kinds of magical weapon ability: Spell-Like Abilities and Supernatural Abilities. An example of the first type is a Rod of Fire and an example of the second type is a Vorpal Sword.}

A Spell-like ability is just a spell that having that weapon allows you to use. Using this spell-like ability is a Standard Action, so Quickened Spells aren't particularly interesting.

\abox{Behind the Scenes: What Spells Can Rods and Swords use?}{D\&D has literally thousands of \emph{pages} devoted to spells and it is entirely impractical to go through the list and find all the spells that would be appropriate from an activation ability for a magical rod or staff. Instead, here are some basic ideas of things which are \emph{not} a good idea to put into weapons:

	\listone
		\bolditem{Long Casting Times:}{Spells like \spell{major creation} can make stuff like Adamantine Boxes, which is all fine and dandy until you start making them in combat time by having them be used as a spell-like ability. Then it's suddenly battlefield control with no save allowed and that's just messed up. So while having a Rod of Summoning that allows you to throw down a Fullround spell like \spell{summon monster} whenever you feel the urge is fine -- sources of spells like \spell{move earth} and \spell{planar binding} are deeply problematic.}
		\bolditem{Swift or Quickened Spells:}{\spell{swift fly} is a really crappy spell except for the part where it's castable with a swift action. Even then it's not that great. When used as a Standard Action, it's just crap.}
		\bolditem{Juggling Spells:}{This last category is by far the hardest to nail down, because it isn't precisely quantifiable. But a spell effect that delays an opponent is really a crap tonne more effective if it is repeatable time and time again by autofiring the go button on a magic rod. Spells like \spell{frost breath} and \spell{color spray} are amazingly effective anyway, and if you can just throw them every round you go to straight up unfair territory.}
	\end{list}}

Supernatural abilities, on the other hand, are just things that your weapon does. Like a monster ability, your weapon simply has some effect going all the time. In many cases, this involves inflicting a status effect on enemies struck with the weapon. Status effects will be inflicted on the following circumstances:

\listone
	\item A target is struck at least once during a round (so figuring out some way to scam lots of attacks doesn't give extra statuses).
	\item The target fails a saving throw. The DC is 10 + \half the wielder's character level + the wielder's Charisma Modifier.
\end{list}

Here are some supernatural weapon qualities:

\listone
	\item \textbf{Lesser Qualities:}
	\listtwo
		\bolditem{Defender -}{A defending weapon moves itself to intercept attacks made on the wielder. While wielding a defending weapon, the character has an armor bonus of 5, enhanced by the enhancement bonus of the weapon.}
		\bolditem{Dispelling -}{A weapon of dispelling destroys magic. Anything struck that fails a Willpower Save is targeted by a targeted \spell{dispel magic}, with a dispelling check of d20 + wielder's character level (no cap).}
		\bolditem{Flame -}{A flaming weapon sets things touched by it \emph{on fire}. A victim who fails a Reflex Save is on fire and will suffer a d6 of fire damage every round until they extinguish themselves.}
		\bolditem{Ghost Touch -}{A ghost touch weapon spans the material and ethereal planes. It can be wielded by any standard, incorporeal, or ethereal being and can be used to attack any standard, incorporeal, or ethereal being with no miss chances due to the difference (if any).}
		\bolditem{Terror -}{A weapon of terror strikes fear into the hearts of its foes. A victim who fails a Willpower Save becomes \condition{shaken} for an hour. This is a [Fear] effect.}
		\bolditem{Thunder -}{A thundering weapon makes a whole lot of noise. A victim who fails a Fortitude Save becomes \condition{deafened}, and an object struck by a thundering weapon has its hardness ignored.}
		\bolditem{Time Distortion, Lesser -}{A weapon of lesser time distortion cuts time away from the target. A victim who fails a Willpower Save becomes \spell{slowed} for 5 rounds.}
		\bolditem{Berserking -}{A berserking weapon causes the wielder to go into a red rage of mindless fury. Whenever the user makes an attack with the weapon, the user is immune to mind affecting and fear effects for three rounds. However, during this period the character also cannot cast spells or activate magic items.}
	\end{list}
	\item \textbf{Moderate Qualities}
	\listtwo
		\bolditem{Cursed -}{A cursed weapon cannot be gotten rid of. A character who uses a cursed item will find that it continues to count against her 8 item limit for some time after being set aside, and it can be willed into her hand as a free action regardless of distance. Even if it is destroyed, the cursed weapon reforges itself once every day and continues to count against the wielder's item limit until a successful \spell{remove curse} is used to sever the connection.}
		\bolditem{Disruption -}{A disrupting weapon damages the necromantic animating force of the undead. An undead victim who fails a Fortitude Save is instantly destroyed.}
		\bolditem{Frost -}{A frost weapon freezes things quite severely. A victim who fails a Fortitude Save becomes \condition{fatigued}, normal fires are extinguished, and liquid objects freeze out to a 5' radius. Within 5 feet of an unsheathed frost weapon, the temperature cannot rise above \emph{cold}.}
		\bolditem{Lifestealing -}{A lifestealing weapon damages the souls of the living. A victim who fails a Willpower Save gains a negative level.}
		\bolditem{Planar -}{A planar weapon ironically is infused with the power of the Prime Material and is named thus because it's a good thing to have when traveling the planes. When a victim who is not a native of the Prime or whatever plane you happen to be on fails a Willpower Save it is instantly banished to its home plane, from which it may not leave for 24 hours (treat as a \spell{dimensional anchor}). In addition, an outsider victim must make a Fortitude Save or be \condition{dazed} for 1 round, regardless of what their native plane happens to be.}
		\bolditem{Sharpness -}{A weapon of sharpness cuts stuff into pieces. A victim who fails a Fortitude Save loses a limb (chosen at random), and an object struck by a weapon of sharpness has its hardness ignored. This enhancement is only available for sharp weapons, other weapons should use Withering instead.}
		\bolditem{Sun -}{A sun weapon sheds tremendous amounts of light. A victim who fails a Reflex Save becomes \condition{blind} for 1 round. Such a weapon sheds more light than normal, and is surrounded by a \spell{daylight} effect when in use.}
		\bolditem{Withering -}{This quality is exactly like "sharpness" except that the special effect is that limbs wither and objects crumble. It is used for blunt weapons.}
		\bolditem{Wounding -}{A wounding weapon causes brutal and horribly bleeding wounds. Damage caused by a wounding weapon is vile physical damage even if the victim has Regeneration. A living victim who fails a Fortitude Save becomes \condition{staggered} for one round.}
	\end{list}
	\item \textbf{Greater Qualities}
	\listtwo
		\bolditem{Petrification -}{A petrifying weapon causes living tissue to transform into stone. A living victim who fails a Fortitude Save is \condition{petrified}.}
		\bolditem{Ruin -}{A ruinous weapon destroys pretty much anything. A ruin weapon ignores all hardness, DR, and resistance to critical hits of any target it strikes.}
		\bolditem{Soul Prison -}{A soul prison weapon absorbs the soul of any enemy slain with it. A victim who is dropped by a soul prison weapon has their soul immediately drawn into the weapon, where it remains until used. A soul prison weapon can hold up to nine such souls at a time, and not even a \spell{wish} can restore the life of a foe whose soul is therein contained. Nominally there is a Willpower save is involved, but since a dropped foe is considered "willing" that doesn't normally come up. This is a necromantic effect.}
		\bolditem{Time Distortion, Greater -}{A weapon of greater time distortion cuts time away from the target. A victim who fails a Willpower Save becomes affected by \spell{temporal stasis} for ever.}
		\bolditem{Vorpal -}{A vorpal weapon kills things outright with a "snicker-snack" noise. A victim who fails a Fortitude Save is killed, this is a death effect.}
	\end{list}
\end{list}

\subsubsection{Magic Ammunition}

\desc{A magic arrow is indeed a special thing. The only kind of magic arrow that doesn't make us feel really bad about ourselves is the Spell Arrow, so that's the only one that exists. Every magical arrow (or crossbow bolt, or whatever) has one spell in it which is chosen when it is made and which will be cast when it is fired. A spell arrow is not recoverable after the fact because the spell only goes off once. In order to actually get a magical arrow to ``go off'', you have to spend a standard action firing it. Otherwise it's just an incredibly expensive arrow.}

Magical arrows have the spell go off in whatever way would be most awesome looking. So if you fire an arrow which contains a touch ranged spell like \spell{cure serious wounds} or \spell{incite love} then the spell takes effect on whoever gets hit by the arrow. On the other hand, if you have a spell with a cone or line area of effect like \spell{lightning bolt} or \spell{color spray} it starts the line right in front of the bow. Bursts or Emanations come from wherever the arrow lands, and Personal or 0-range spells can't be made into Spell Arrows at all. In any case, the arrow itself is completely consumed by this process and doesn't do any actual damage (so curative arrows aren't as stupid as they might sound). Hitting a specific target with a Spell Arrow is a ranged touch attack.

\subsubsection{Magic Armor, Clothing, and Accessories}
\vspace*{-8pt}
\quot{``He's the man with the magic pants.''}

\desc{Heavy plate armor, racks upon racks of Mr. T style gold chains, shiny pants, and magic belts, these are a small set of examples of the crazy crap that people wear in the D\&D universe. The only difference between wearing, for example, a bunch of gold chains and a sleek set of leather armor is that the leather armor counts as \emph{armor} and has a tendency to provide some sort of Armor Bonus, Armor Check Penalty, and level appropriate bonuses (see Races of War). The gold chains just make you look like a circa-1986 rap star. But basic bonuses aside, all such items are simply a minor magic item unless they have some special ability above and beyond the standard level appropriate effect.}

\desc{Special abilities on such items can be spell-like or supernatural, exactly as per weapons. The activated spells on a cloak or a belt function exactly like the activated spell-likes provided by magic weapons.}

Here are some supernatural Worn-item qualities:

\listone
	\item \textbf{Lesser Qualities:}
	\listtwo
		\bolditem{All Around Vision -}{Iconically placed upon helmets and Spot-bonus items, this enhancement gives the user the ability to see in all directions, preventing enemies from flanking.}
		\bolditem{Aquatic -}{Iconically placed upon any worn item, this enhancement gives the user the aquatic subtype, allowing them to breath water and swim easily.}
		\bolditem{Dark Vision -}{Iconically placed upon helmets and Spot-bonus items, this enhancement gives the user the ability to see without light, as darkvision out to 60'.}
		\bolditem{Tremorsense -}{Iconically placed upon boots and Listen-bonus items, this enhancement gives the user the ability to detect things within 30' who are touching the ground as with tremor sense.}
	\end{list}
	\item \textbf{Moderate Qualities:}
	\listtwo
		\bolditem{Blindsense -}{Iconically placed upon helmets and Listen-bonus items, this enhancement gives the user the ability to detect things within 60' as with blind sense.}
		\bolditem{Madness -}{This enhancement surrounds the user the with maddening trills and whispers, causing all sane creatures within 10 feet of the user to have to save vs. a \spell{hypnotism} effect each round that the item is active and uncovered.}
		\bolditem{Spell Resistance -}{Iconically placed upon protective items and cloaks, this enhancement gives the user Spell Resistance of 10 + Character Level. Spell Resistance from multiple items with this enhancement do not stack.}
		\bolditem{Telepathy -}{Iconically placed upon helmets and Sense Motive-bonus items, this enhancement gives the user the ability to silently communicate with any creature which has a language out to 100' regardless of line of effect.}
	\end{list}
\end{list}

\abox{Armor Bonuses and Natural Armor Bonuses}{Yes, Armor Bonuses and Natural Armor Bonuses stack, but they don't 100\% stack. If you have both an Armor Bonus and a Natural Armor Bonus, you only benefit from half of the smaller bonus (round up). So if you have a +8 Armor bonus and a +5 Natural Armor Bonus, you are getting a total of +11 from Armor and Natty Armor, not +13. The reason for this is because Natural Armor gets \emph{very large} on a number of creatures. Originally this was because writing in a big natural armor bonus is really easy and gives level appropriate overall bonuses for the stuff in the Monster Manual, but when you mix in regular armor it pushes defenses straight off the random number generator.}

\subsubsection{Magic Rings}
\vspace*{-8pt}
\quot{``In brightest day, in darkest night\ldots''}

There is nothing special about Rings. At this point there is enough fantasy material available that there are people deeply immersed in the genre who have never read the Nibelungenlied or Lord of the Rings. When Arneson and Gygax made D\&D back in the day, LotR really was primary inspiration and the natural result was to put rings on some sort of whacky pedestal. Well, nowadays we have people for whom the iconic Magic Item of Vast Power is a lamp (Aladdin), a gem (Dark Crystal), an orb (Castle of Llyr), or whatever. So a Ring is just like any other piece of clothing, save that it rarely provides an enhancement bonus to armor.

\subsubsection{Constant Miscellaneous Magic Objects}

There are a number of objects in D\&D land that are neither worn nor wielded and yet count as constant items. Crystal Balls, Handy Haversacks, and Braziers that call fire elementals are all powerful items that do count against a character's eight item limit. What they don't do is actually stay connected to the user in a physical sense between uses. In order to use one of these items, one must \emph{attune} it, at which point that item is connected to the character who did so. Attuning such an item takes 15 minutes, and it takes that long for it to stop being attuned as well. It takes an act of will to make a magic item of this sort stop working for you, and this act of will can be taken either by you or someone who holds the actual object. So if someone snags your decanter of endless water,

\abox{Behind the Scenes: Attuning Crystal Balls.}{When you draw a flame tongue it bursts into flame immediately upon leaving its sheath -- granting a level appropriate bonus to attack and damage while setting stuff on fire. However, the same does not happen when you uncover a crystal ball. And the reason for this is honestly that items like collapsible bridges, bags of holding, and iron flasks are almost never used in combat time and yet they \emph{do} have a serious impact on your success or failure in an adventure.

It takes longer to swap these objects into and out of your bat cave simply because it is assumed that when you would be doing this you actually have more time to swap things in and out. In fact, it might be pointed out that it takes precisely as long to attune such an item as it does to fill an open spell slot on the fly -- that's not an accident.}

\subsubsection{Building a Better Magic Item: Intelligent Items}
\vspace*{-8pt}
\quot{``Hello computer!''}

\desc{In every edition of D\&D, the intelligent item has been listed as something that happened quite frequently. Seriously, even in the 3.5 DMG it says that fully 1\% of all Amulets of Health and Rings of Featherfall have intelligence. Were you to actually roll that up for every item you found it seems a virtual lock that every campaign would have one or more Intelligent Items in them. Since the vast majority of campaigns include \emph{zero} talking swords rather than the 1-5 expected by purely random chance, it seems extremely clear that something is wrong with the way Intelligent Items have been handled in the last 40 years of D\&D.}

\desc{An Intelligent Item is like having a cohort, and if it is the same level as you are that's really unbalancing to the game. While previous editions have tried to keep track of ego points, we're going to try to make this as simple as possible: An Intelligent Item is a Sorcerer who happens to be a dagger or a pair of shoes. Like any Sorcerer, an Intelligent Item has a character level, and if that character level is more than 2 less than your character's level, it \emph{will not be your cohort}.}

\desc{And that's it. An Intelligent Item is "just" a magic item that happens to have one or more levels of Sorcerer, and an Int, Wis, and Cha. If it is within one step of alignment of your character, and is at least two levels lower than your character, and it is attuned as one of your eight items, it will work with you -- casting its spells on your behalf. An Intelligent Item never needs to worry about somatic components, which is just as well because a lot of them don't have moving parts.}

That being said, an Intelligent Item is still an extremely powerful, game altering item. An extra spiderweb cloak that is throwing down \spell{web} in pitched battles can make the difference between life and death even at very high levels.

\subsection{The Appearance of Magic Items}
\vspace*{-8pt}
\quot{``Don't touch that sword.''\\
``Why? Because it's on fire? Because it has glowing runes?''\\
``Because the glowing runes say `Don't touch this sword.' ''}

\desc{Magic items do not normally require a casting of \spell{detect magic} to uncover. The DC of an appraise check to determine that something is in fact magical is 20 \emph{minus} the object's caster level. A powerful item bends space around it and glows with unearthly soulflame and such and really can be noted as magical by the untrained eye. But what exactly a magic item \emph{looks like} is contingent upon who made it and what they made it out of. Broadly speaking, the magic items made by the Drow really are black and covered with spider motifs; the magic items made by the Hobazad Khanate are generally lacquered in red and black with decorative leafing of gold and brass; the artificer mages of Bladereach make their magic items by etching them with hydra saliva so they look all melty and marbled.}

\desc{Minor Magic Items of any sort can usually be identified by regular people who are familiar with the culture which produced them. If you're a Drow you've \emph{seen} the cloaks of resistance that the tailors in your society make. You might even own one. It's really not any kind of mystery to you.}

Artifacts of course, follow their own set of rules. Some artifacts are instantly identifiable as powerful magical objects by people remotely in the vicinity (good examples of this are the Rod of Orcus and the Machine of Lum the Mad), while others really do adequately disguise themselves as mundane, even commonplace items (good examples of this second type are Aladdin's Lamp and the Pillowcase of Storms).

\subsubsection{Iconic Forms}

\desc{Let's face it, magic items are more fun when they come in recognizable forms. See a wizard waving around a stick and knowing that its a wand is more fun than trying to guess the effects of a glowing stone in his hand. That being said, here are additional rules to bridge the gap between our creation system and 3.X D\&D.}

Iconic Form bonus: Any item made in both its iconic form (ring, wand, sroll, etc) and enchantment as shown in the DMG or other published source recieves can be created as if it was -2 its normal caster level after creation. This means that if you make a Cloak of the Manta Ray rather than a Ring of the Manta Ray, it takes you the amount of time it would take for a 7th level item instead of its normal 9th level, and it counts as a 7th level item for item creation limits. Note that this does mean that casters can create iconic items by using a lower caster level (so a 7th level caster can create a Cloak of Manta Ray, but not a Ring of the Manta Ray), assuming they can cast (or have cast) the required spells.

% This subsection was moved from later on in the book, because it definitely
% belongs under the "Magic Items" section.
\subsection{Disposing of Magic Items}
\vspace*{-8pt}
\quot{``You're going to have to throw The Ring into Mount Doom. Probably those pants as well.''}

\desc{Magic items are \emph{really dangerous}. Not just to use, but also to leave lying around. Or to destroy. Really anything you happen to do or not do with magic items carries significant consequences down the line.}

The Bat Cave or Sword Rack is a relatively simple storage system for magical objects, and works fairly well.

\chapter{Treasure and the World}

\section{Finding Treasure}
\vspace*{-8pt}
\quot{``There's nothing here but worthless gold!''}

\desc{It is an absolutely necessary step in the entire process of dragon slaying that one cart off the pile of gold. Indeed, previous editions oft as not required that one employ literal carts to carry off the fantastic wealth that a single Dragon might hoard. This was made possible by the letters G and H and by the number nine \emph{thousand}. And while it is true that the old alphabetical treasure types may have been a \emph{bit} overboard with the tremendous piles of loot that they handed out, the reverse trend of giving characters piles of gold that fit in one's pocket is fairly unfortunate.}

It is absolutely the case that any dragon worth its salt should be worth enough in gold that it can actually \emph{sleep} on said gold. For reference, that's about 760 pounds of gold for a minimal medium-sized dragon, and about three tonnes of gold for a large dragon. But it is equally the case that when you encounter a group of gnolls or bugbears or even hill giants they generally don't have a big pile of gold and more often than not they don't have any magical items. Even more importantly, owlbears don't have any treasure \emph{at all}. Their digestive systems really will destroy all of the valuables they eat, and most of the time they won't even eat valuables because \emph{they're owlbears}! They live in the woods and they kill things and they don't participate in any economic activity at all.

\subsection{Books}

\desc{One of the most important and interesting things one can find in a cooperative storytelling game is a book. It's a story within a story, a source of potentially needed information and it's not really game breaking for your character to have it. It is for this very reason that virtually every Dungeon Magazine includes at least one book that the characters can find.}

So what do books actually do? Well, the obvious thing is that if there are any spells in them, you can copy those down into your spellbook (or your \spell{secret page} manifest pad, if you're a modern wizard). But even if they are completely mundane they can still be useful. If you have enough of them on a subject you can have a \emph{Library}, which allows you to take 20 on Knowledge Checks. And a book about a specific subject can allow a character to spend an hour in study to make a knowledge check as if you had an appropriate Area of Expertise. So if you are confronted with a hobgoblin wartabard, and you can't make a sufficient Knowledge Geography or Knowledge History check to figure out where it comes from -- you can bust out a copy of \underline{Bastions of the Goblin Khanate} and try to find a match -- then you can make another knowledge check with the much lower DC.\\

\textbf{One Hundred Books that you can find in a Fantasy Setting:}
\begin{enumerate}
	\topsep=0pt
	\itemsep=-2pt
	\item The Ascendancy of Fire
	\item Abjuring Minor Demons
	\item Alterationism and Revisionism
	\item August and Winter
	\item Anatomie d'Ghoule
	\item Book of the Wars of Pelor
	\item Bastions of the Goblin Khanate
	\item Bees: Keeping and Secrets
	\item Benevolence and the Duchess
	\item Blzht's Personal Notes on the Badger Kingdoms
	\item Birthrights
	\item Crumbling Shadow
	\item The Cruelty of Healing Magic
	\item Carbuncles in an Elvish Context
	\item Cyclopean Constructions of the Vanished Ones
	\item Cults of the Maggot God
	\item De Vermis Mysteris
	\item The Draconomicon
	\item The Diary of Jakkar the Mad
	\item The Diary of Prince Olaf
	\item Dangerous Plants of the Bane Mires
	\item Djinn Fermentation Techniques
	\item Donjon Menagerie
	\item 101 Secrets of Devilcraft
	\item Eternal Subjects: Stasis and Crystal
	\item Evil
	\item Etherealness, Property, and Government
	\item Extreme Cold: A Goblin's Tale
	\item Ettercap: The Terrible Secret Reality
	\item The Crawling Darkness: Practical Necromancies
	\item Fabled Lands and Mythic Locales
	\item Fairy Courts: Sun and Shadow
	\item Fear in Hoburg
	\item Remnant Cities and Constructions of the Ancients
	\item The Bestiary Arcane
	\item Giant Crab!
	\item The Giant Kingdom: A Traveler's Perspective
	\item Gargoyle Physiology
	\item Grafting Flesh and Lead
	\item Gold: Providence and Necessity
	\item Gnome Lore
	\item Horror and Birds
	\item Harpy Statecraft
	\item The Harvest of Sorrows
	\item Blood of the Innocent
	\item The Asmodeus Gambit
	\item Blood and Silk: Danger Rises as the Sun
	\item The Cutting Edge: A Warrior's Tale
	\item The Dark History of Bladereach: A shocking and true revelation
	\item Fantastic Economie
	\item The Fly and the Serpent: Against the Giant Frog
	\item Cults: Demons: Dark Miracles
	\item Five Beans You Can Eat
	\item The Broken Mask: A Practical Guide to Hunting Shapeshifters
	\item The Book of Odamma
	\item Children of a Dark Star
	\item The Horrible Reality: The Devouring Darkness Unavoidable
	\item Last of its Kind: Twelve Dying Races
	\item Dark Revelations V through IX
	\item The House of Fiery Justice
	\item Aboleth Memories
	\item Industrial Uses of Slimes, Molds and Jellies
	\item Balance and Leverage: Druidic Construction and the Natural Order
	\item The Path of Blood
	\item Plains of Dust on the Planes of Water
	\item The Properties of 120 Magical Plants available anywhere on the planes
	\item Political Maneuvers of the Depraved
	\item Playing with Fire: The Dangers of the Vilest Necromantic Arts
	\item The Planar Political Primer
	\item The Physiology of Pain
	\item Prophecies of Profleggathron the Ever Burning
	\item Ash on the Wind: the Conquest of Valdrana
	\item Stone Unyielding, Impressions and Sand
	\item Servants of Leaf and Branch: Dryads, Nixies, and Nymphes.
	\item Secrets of Life and Death
	\item Slaves to the Black Tower
	\item The Complete Dwarven Histories: Volume XVIII: The Seventh Bugbear Confrontation
	\item A Transcript of the Trial of Harakhdar the Forsaken
	\item Taxidermy for Profiteers
	\item A Treatise on the Efficacy of Fungal Remedies
	\item Unaussprechlichen Kulten
	\item Ur Priest: Eating the Gods
	\item The Void and the Flame: The Story of Elothar
	\item Tactica Implacable: A Primer for Dwarven Strategists
	\item Land Grants of the Wendish Borderlands
	\item Surprisingly Delicious Things
	\item Tracking the Wily Displacer Beast
	\item Potion Miscibility
	\item The Worst the Banemires Can Do
	\item Wanderers of the Void: Giant Frog
	\item Six Problems of Classic Philosophy
	\item Twelve Ninja Clans
	\item The Xorn and the Unicorn: Root and Stone
	\item Path of the Mud Sorcerer:
	\item The Wish Economy and the Brass Sultan
	\item Metallurgical Properties of Mithril and its Common Alloys
	\item The Precepts of Hruggek
	\item Your Word Against Mine: The Kobold Problem
	\item Anathema
	\item Zone Agents
\end{enumerate}

\section{The Three (or so) Economies}
\vspace*{-8pt}
\quot{``I'll give you five pounds of gold, the soul of Karlack the Dread King, and three onions for your boat, the Sword of the Setting Sun, and that cabbage\ldots''}

\desc{Life in D\&D land is not like life in a capitalist meritocracy with expense accounts and credit cards. There is no unified monetary system and there are no marked prices. \emph{All} transactions are essentially barter, and you can only trade things for goods and services if people genuinely believe that the things you are trading have intrinsic value and the people you are trading to actually want those specific things. Gold can be traded to people only because people in the world genuinely think that gold is intrinsically valuable and that they want to own piles of gold.}

That means that in places where people don't want gold -- such as the halfling farming collective of Feddledown, you can't buy anything with it. It's just a heavy, soft metal. But for most people in the fantasy universe, gold has a certain mystique that causes people to want it. That means that they'll trade things they don't need for gold. But no matter what they are giving up they aren't ``selling'' things because money as we understand the concept doesn't really exist. They are \emph{trading} some goods or services directly for a physical object -- an actual lump of gold. Not a unit of value equivalency, not a promise of future gold, not a state guaranty of an amount of labor and productive work -- but an actual physical object that is being literally traded. And yeah, that's totally inefficient, but that's what you get when John Locke hasn't been born yet, let alone modern economic theorists like Adam Smith, Karl Marx, or Benito Mussolini. If you really want to get into the \emph{progressive} economic theories that people are throwing around with a straight face, go ahead and check out theoreticians like Martin Luther, Thomas Aquinas, Sir Thomas Moore, or Zheng He. If you want to see what \emph{conservative} opinions look like in D\&D land, go ahead and read up on your Draconis, Li Ssu, Aristotle, or Tamerlain.

\subsection{The Turnip Economy}
\vspace*{-8pt}
\quot{``We got rats! Rats on sticks!''}

\desc{Most settlements in a D\&D setting are really small and completely unable to sustain any barter for such frivolities as gold or magical goods. The blacksmith of a hamlet does not trade his wares for silver, he trades them for \emph{food}. He does this because the people around him are farmers and they don't make enough surplus to hoard valuable metals. So if he took gold for his services, he would get something he couldn't spend, and then he wouldn't be able to eat. So even though people in the tiny villages you fly over when you get your first gryphon will freely acknowledge that your handful of silver is worth very much more than their radishes, or their tin cups, or whatever it is that they produce for the market, they still won't trade for your metal because they know that by doing so they run the risk of starving to death as rich men.}

\desc{The economy of your average gnomish village is so depressed by modern standards that even the \emph{idea} of wealth accumulation and currency is incomprehensible. But the idea of \emph{slacking off} is universal. There is a static amount of work that needs to be done on the farm each year and the peasants are perfectly willing to put you up if you do some of their chores. Seriously, they won't let you stay in their house for a copper pfennig or a silver ducat, but they \emph{will} give you food and shelter if you cleanout the pig trough. They have no use for your ``money'', but they do need the poop out of the pig pen and they don't want to do it. On the other hand, they also don't want to be eaten by a manticore, so if you publicly slay one that has been terrorizing the village the people will feed you for free pretty much as long as you live. That's why people pay money to bards. Bards spend a lot of time in cities and actually will take payment in copper and gold. And if they sing songs about you, your fame increases. And fame really is something that you can use to buy yourself food and shelter from people in the turnip economy.}

\desc{``Costs'' in the turnip economy are extremely variable. In lean times, the buying power of a carrot is relatively high and in fat times the buying power of a cabbage is very low. It is in this way that the people in tiny hamlets get so very screwed. No matter how much they produce or don't produce, they are pretty much going to get just enough nails and ladders and such to continue the operations of their farms. However, such as there is a unit of currency in the barter economy of the turnip exchange -- it's a unit of 1000 Calories. That's enough food to keep one peasant alive for one day. It's not enough to feed them well, and it's not enough to make them grow big and strong, but it's enough so that they don't actually die (for reference, a specialist eats 2000 Calories a day to stay sharp and an actual adventurer eats 5000 Calories a day to maintain fighting shape). In Rokugan, that's called a Koku, and in much of Faerun it is called a ``ration''. It works out to about 2 cups of dry rice (435 mL), or a 12 oz. steak (340 g), or 5 cups of black beans (1.133 kg), or 4.4 ounces of cooking oil (125 g).}

\desc{Higher Calorie foods like meat and oil are more valuable and lower calorie foods like celery or spinach are less valuable because a lot of people exist on the razor's edge of starvation. The really fatty cuts of meat are the most valuable of all (it's like you're in Japan or Africa in that way). The practical effect of all of this is that people who have a skilled position such as blacksmith or scribe get enough food to grow up big, healthy, and intelligent. The peasants actually are weak and stupid because they only get 1000 Calories a day -- they won't die on that but they don't grow as people. This also means that the blacksmith's son becomes the next blacksmith -- he's the guy in the village who gets enough food to get the muscles you need to actually be a blacksmith.}

When you start a party of adventurers, note the really tremendous expenditures that were required to make your characters. A 16 year old first level character didn't just get a longsword from somewhere, he's also been fed a non-starvation diet for 5844 days. That means that at some point your newly trained Fighter or Rogue seriously had someone invest thousands of Koku into him to allow him to get to that point. If your character is a street rat or a war orphan, consider where this food may have come from. Perhaps when the orcs destroyed your village leaving your character alone in the world the granary survived and your character had a huge supply of millet to sustain himself until he could hunt and kill deer to augment his diet.

\abox{A Note on Peasant Uprisings}{\desc{Peasants may seem like they get a crap deal out of life. That's because they do. And regardless of whatever happy peasant propaganda you may have seen, peasants aren't really happy with their life even under Good or Lawful rulership. That's because they work hard hours all year and get nothing to show for it. So the fact that they don't get \emph{beaten} by Good regimes or \emph{stolen from} by Lawful regimes doesn't really make them particularly rich or pleased.}

\desc{In Earth's history, peasant uprisings happened about every other generation in every single county from Europe all the way to China all the way through the entire feudal era (all 1500 years of it). It is not unreasonable to expect that feudal regions in D\&D land would have even more peasant uprisings because the visible wealth discrepancies between Rakshasa overlords and halfling dirt farmers is that much more intense. Sure, as in the real world's history these uprisings would rarely win, and even more rarely actually hold territory (if lords can agree on nothing else, it is that the peasants should not be allowed to rise up and kill the lords). The lords are all powerful adventurers, or the family and friends of powerful adventurers, so the frequent peasant revolts are usually put down with \spell{fireball}s and even \spell{cloudkill}s.}

Students of modern economic thought may notice that cutting the remote regions in on a portion of the central government's wealth in order to buy actual loyalty from the hinterlands could quite easily pay itself off in greater stability and the ability to invest in the production of the hinterlands causing the central government's coffers to swell with the enhanced overall economy and making the entire region safer and stronger in times of war ? but as noted elsewhere such talk is considered laughable even by Lawfully minded theorists in the D\&D world. After all, since abstract currency doesn't see use and the villagers don't have any \emph{gold}, it is ``well known'' that it is \emph{impossible} to make a profit on investment in the villages. The only possible choices involve taking more or less of their food as taxes/loot as that is all they produce.}

\subsection{The Gold Economy}
\vspace*{-8pt}
\quot{``What pleasures can I get for a diamond?''\\
``We'll\ldots\ have to get the book.''}

\desc{People who live in cities mostly trade in gold. This is not just because living so far away from the dirt farmers makes the hoarding of turnips as a trade commodity a dangerous undertaking -- but because people living in cities are surrounded by a lot of \emph{people} who provide a wide variety of goods and services they are willing and able to trade for substances generally acknowledged to be valuable rather than trading directly for the goods and services that they actually want. These valuable substances range from precious metals (copper, silver, gold, platinum) to gems (pearls, rubies, onyx, diamond) to spices (salt, myconid spores, hellcandy flowers). In any case, these trade goods are traded back and forth many times before they are ever used for anything.}

\desc{When someone sells an item or a service for trade goods they are doing it for one of two reasons. The first is that they want \emph{something} that the buyer doesn't have. For example, a man might want a barrel of lard or a bolt of silk -- but they'll accept silver coins or something else that they are reasonably certain they can trade to a third party for whatever it is that they are actually interested in. Whoever is using the trade goods is at a disadvantage in the bargaining therefore, because while they are getting something they actually want, the other trader is essentially getting the \emph{potential} to purchase something they want once they walk around and find someone who will take the silver for their goods. It is for this reason that the purchasing power of gold is shockingly low in rural areas: a prospective trader would have to walk for days to get to another place he might actually spend a gold coin -- so all negotiation essentially starts with buying several days of the man's labor and attention. The second reason for accepting a trade good is the belief that the trade good may itself become more valuable. Indeed, when were crocodiles take over a nearby village all the silver becomes a lot more interesting. This sort of speculation happens all the time and is incredibly bad for the economy. People and dragons take enormous amounts of currency out of circulation and the resulting economic downturns are part of what makes the dark ages so\ldots\ \emph{dark}.}

Gold and jewels \emph{can} be used to purchase magic items that aren't amazingly impressive. No wizard is ever going to make a masterpiece just to sell it for slips of silver. However, there are more than a few magicians who would be willing to invest some time in order to get a handful of gold that they can use to live their lives easier with. Making even Minor magic items is hard work, and wizards demand piles of gold to be heaped on them for producing even magical trinkets. And because these demands actually work, there's really no chance to purchase anything that would take a Magician a long time to make. That means that Major magic items cannot be purchased with standard trade goods \emph{at all}. There's literally no artificer anywhere who is going to sit down and make a Ring of Spellstoring or a Helm of Brilliance in order to sell it for gold -- because the same artificer can acquire as much gold as he can carry just by making Rings of \spell{Featherfall} or \magicitem{Cloaks of Resistance}.

\subsection{The Wish Economy}
\vspace*{-8pt}
\quot{``They scour the land searching for relics of the age of legends. Scant remnants they believe will grant them the powers of the Vanished Ones. I do not. The Age of Legends lives in me.''}

\desc{Magicians can only produce a relatively small number of truly powerful magic items. While a magician can produce any number of magic items that hold requirements at least 4 levels below their own -- a wizard is permitted only one masterpiece at each level of their progression. It is no surprise, therefore, that characters would be vastly interested in acquiring magic items produced by others that are even of near equivalence to the mightiest items that a character could produce. A character could plausibly bind 8 magic items, and yet they can only create one which is of their highest level of effect. Gaining powerful magic items from other sources is a virtual requirement of the powerful adventurer.}

So it is of no surprise that there is a brisk -- if insanely risky -- trade in magical equipment amongst the mighty. All the ingredients are there: characters are often left holding onto items that they can't use (for example: a third fire scimitar) and they are totally willing to exchange them for other items that they might want (magical teapots that change the weather or helmets that allow a man to see in all directions). And while the mutual benefit of such trades is not to be downplayed, it is similarly obvious that the benefits of betrayal in such arrangements are amazingly amazing. Killing people and taking their magical stuff is what adventurers do, so handing magic items back and forth in a seedy bar in a planar metropolis is an obviously dangerous undertaking.

\subsection{Tamerlain's Economy: The Murderocracy}
\vspace*{-8pt}
\quot{``The soldier may die, but he must receive his pay.''}

\desc{Let's say that you don't want to exchange goods and services for other goods and services at all. Well, it's medieval times baby, there's totally another option. See, if you \emph{kill} people by stabbing them in the face when they want to be paid for things, you \emph{don't have to pay for things}. Indeed, if you have a big enough pack of gnolls at your back, you don't have to pay anything to anyone except your own personal posse of gnolls.}

\desc{The disadvantages of this plan are obvious -- people get super pissed when they find out that you murdered their daughter because it was that or pay for a handful of radishes. But let's face it: if that old man can't do anything about it because you've \emph{got a pack of gnolls} -- then seriously what's he going to do? And while this sort of thing is often as not the source for an adventure hook (some guy comes to you and whines about how his whole family was killed by orcs/gnolls/your mom/ ogres/demons/or whatever and suddenly you have to strike a blow for great justice), it is also a cold harsh reality that everyone in D\&D land has to live with. Remember: noone has written \underline{The Rights of Man}. Heck, no one has even written \underline{Leviathan}. The fact that survivors of an attack may appeal to the better nature of adventurers is pretty much the only recompense that our gnoll posse might fear should they simply forcibly dispossess everyone in your village.}

\desc{So people who have something that the \emph{really powerful} people want are in a lot of danger. If a dirt farmer who does all of his bargaining in and around the turnip economy suddenly finds himself with a pile of rubies that's \emph{bad news}. It's not that there aren't people who would be willing to trade that farmer fine clothing, good food, and even minor magic items for those rubies -- there totally are. But a pile of rubies is just big enough that a Marilith might take time out of her busy schedule to teleport in and murder his whole family for them. And he's a dirt farmer -- there's no way he has the force needed to even \emph{pretend} to have the force needed to stop her from doing it. So if you have planar currencies or powerful artifacts, you can't trade them to innkeepers and prostitutes. You can't even give them away save to other powerful people and organizations.}

That doesn't mean that there isn't a peasant who runs around with a ring that casts \spell{charm person} once a day or there isn't a minor bandit chief who happens to have a magic sword. Those guys totally exist and they may well wander the lands trying to parlay their tiny piece of asymmetric power into something more. But the vast majority of these guys don't go on to become famous adventurers or dark lords -- they get their stuff taken away from them the first time they go head to head with someone with real power. Good or Evil, Lawful or Chaotic, \emph{noone} wants some idiot to be running around with a ring that \spell{charms} people -- because frankly that's the kind of dangerous and an accident waiting to happen. If you happen to be powerful and see some small fry running around with some magic -- your natural inclination is to take it from them. It doesn't matter what your alignment is, it doesn't matter if the guy with the wand of \spell{lightning bolt} is currently ``abusing'' it, the fact is that if you don't take magic items away from little fish one of your enemies will. There is no right to private property. Noone owns anything, they just hold on to it until someone takes it from them.

\subsection{Beelzebub's Economy: The Trade in Favors}
\vspace*{-8pt}
\quot{``I'm certain that there's something we can do to help you\ldots\ but eventually you'll have to help us.''}

\desc{Every transaction in D\&D land is essentially barter. People trade a cloth sack for a handful of peas, people trade an embroidered silken sack for a handful of silver, and people trade a powerful magical sack for a handful of raw power. But in any of these cases, the exchange is a one-time swap of goods that one person wants more for goods the other person desires. But there is no reason it has to work like that. Modern economies abstract all of the exchanges by creating ``money'' that is an arbitrary tally of how much goods and services one can expect society to deliver -- thereby allowing everyone to ``trade'' for whatever they want regardless of what they happen to produce. Nothing nearly that awesome exists anywhere in the myriad worlds of Dungeons and Dragons.}

\desc{What one \emph{can} see in heavy use is the trade in \emph{favors}. This is just like getting paid in money except that your money is only good with the guy who paid it to you. So you can see why people might be reluctant to sell you things for it. And yet despite the extremely obvious disadvantages of this system, it is in extremely wide use at every level of every economy. And the reason is because it's really convenient. There is no guaranty that a King will have anything you want right now when he needs you to kill the dragon that is plaguing his lands. In fact, with a dragon plaguing his lands, the King is probably in the worst possible position to pay you anything. But once the lands aren't on fire and taxes start rolling in, he can probably pay you quite handsomely. Heck, in two years or so his daughter will be marrying age and since she's just going to end up as an aristocrat unless she becomes the apprentice and cohort of a real adventurer\ldots}

Failing to pay one's debts can have disastrous consequences in D\&D land. We're talking ``sold to hobgoblin slavers'' levels of bad. Heck, this is a world in which you can seriously go into a court of law and present ``He needed killing'' as an excuse for premeditated homicide, so people who renege on their favors owed are in actual mortal danger. Of course, everyone is in mortal danger all the time because in D\&D land you actually can have land shark attacks in your home town -- so it isn't like there are any less people who flake on duties and favors. Of course, if people know you let favors slide they might be less likely to pull you out of the way of oncoming land sharks. Even in Chaotic areas, pissing off your neighbors is rarely a great plan.

% Monsters
\chapter{Created Monsters: Forged and Bred}
\vspace*{-8pt}
\quot{``It's alive! Or at least animate\ldots\ it's not an object anymore, that's my point.''}

There are three entire types of creatures in D\&D that are to one degree or another created. The obvious one of course is the Construct. It's a creature which was never alive and created by sorcery. Well, most of them are like that. The Flesh Golem is kind of hard to explain actually, but whatever. The point is that every Construct is \emph{constructed}. That's the whole point. And of course there are a lot of Undead that are pretty much the same thing except that they are animated with Negative Energy channeled into them. There's a lot you can say about those guys, and we actually \emph{did} that in the Tome of Necromancy. So we aren't touching that one here. The other one of course, is the Vermin. In D\&D land, ``Vermin'' doesn't mean anything vaguely approaching its meaning in natural English. Rats and cockroaches are vermin because they live in your pantry and poop on your food stores, but they aren't \emph{Vermin} because they aren't enormous biological constructs that mindlessly follow the programming planted in them by an ancient race of long departed mage kings. Really. That's what the Vermin type means in D\&D land. Actual giant insects are just animals in the same way that dire toads and weasels are animals.

\section{Vermin: Remnants of a Fallen Empire}
\vspace*{-8pt}
\quot{``Great holes secret are digged where earth's pores ought to suffice. Things have learnt to walk which ought to crawl\ldots''}

\desc{Ants track by smell and follow trails left by other ants and bees see deep into the ultraviolet spectrum and perceive a beautiful tapestry of gorgeous colors that escape the eye of the man and the mouse. And when dealing with Vermin type creatures that all means precisely \emph{nothing}, because Vermin in D\&D don't do any of that. It's not because the scent ability was ``left off'' the Monstrous Ant description, it's because the Ant described in the Monster Manual genuinely doesn't have a good sense of smell. It does have Darkvision out to 60 feet like an outsider or a Construct, and that's not an accident either despite the fact that Earthly ants really demonstrably don't do that regardless of size.}

The Monstrous Scorpion isn't a super sized scorpion \emph{at all}. It has a set of abilities which are on the face of it completely bizarre from the context of what actual scorpions do, because it's actually a living construct created by a long fallen empire for use in war. That's why it's immune to hallucinatory poisons and can see in perfect darkness. It's actually created from biomass by powerful magic and not by the interaction of natural and magical mutation across a thousand generations and a harsh selection process hastened by unpredictable climate and predation by manticores. The Vermin have a couple of neat things going for them which is why they were created as war machines in the first place:

\listone
	\bolditem{Mindless -}{Unlike actual or even giant spiders, the \emph{monstrous} spider has no mind at all. It cannot be influenced with magic or confused with poisons. It can't even be detected with \spell{detect thoughts}.}
	\bolditem{Brainless -}{Vermin are subject to critical hits because they have segments and organs, but they don't have any \emph{brains}. That means that they can be blinded, but not killed, by decapitation.}
	\bolditem{Darkvision -}{Vermin can see even in complete darkness, making them quite useful in cave fighting.}
	\bolditem{Aggressive -}{The vast majority of predators will retreat from battles where they are presented with even a chance at serious injury. Yet Vermin fight until they are dead. That's a really bad plan for an individual or even a species, but it's \emph{great} for a battle platform.}
\end{list}

\subsection{Who made the Vermin?}
\vspace*{-8pt}
\quot{``They did not \emph{know} that steal marks flesh, and they did not \emph{know} that flesh does not mark steal. In their ignorance they continued to do one task after another in the old ways. They did not \emph{know} what we \emph{know}.''}

Vermin come from the before time. The time when metals were not made and words were not written down. It's quite a feat of construction talent and a testimony to the power and ingenuity of these ancient flesh crafters that these devices are still running, still attempting to fulfill their programming to this day. The answer is not known in the days that D\&D is normally set. They are a product of a bygone age and their origin is a mystery to all but the Aboleth and the memory fish are being \emph{extremely quiet} on this subject. And yet, their conspicuous silence is probably more telling than anything they could possibly say. The Vermin were constructed during the days when aberrations ruled the world, and they were quite obviously designed to fight against aberrations.

\abox{Getting the Program}{All Vermin have a program that they follow at all times, usually involving a spiral search pattern in groups of one to six until they encounter a creature, at which point they will attack it until it is dead. If confronted with more than one type of creature, they will target them in the following order:
\listone
	\item Any Aberration (except their specific non-targeted group)
	\item Any Humanoid
	\item Any other moving creature they can detect
	\item Anything especially edible
\end{list}\vspace{2pt}

This behavior is entirely comprehensible from the standpoint of the wars in the before time -- the Ilithid and Aboleth both sent slave troopers to their death by the millions in their quest for world domination.}

\desc{Individual groups of Vermin will usually have one type of aberration that they will not ever attack. It may be an entire race of aberrations (such as Kopru or Neogi), or it may be a specific clan of aberrations (such as the Aboleth spawn of the Great Mother of the Howling Wells, or the Ilithid of the Tallow Halls). In any case, determining the type of Aberration that is completely safe from any group of Vermin can be done by observing the markings on the beast. Extracting that information is a DC 30 Knowledge Nature check.}

Vermin eggs persist apparently indefinitely and are produced by the hundred score. A starving Vermin cocoons itself and goes into a state of hibernation so deep that it is essentially mummified. When in the presence of magical auras, the eggs of Vermin progress steadily towards hatching, and the cocoons burst forth their contents. Thus it is not weird or unexplainable for areas that recently have been subjected to incursion by adventurers or mind flayers to spontaneously develop invasions by tiny monstrous centipedes or giant cocooned spiders.

\subsection{Vermin Alchemy}
\vspace*{-8pt}
\quot{``The old ways are the good ways.''}

Vermin cannot think for themselves, nor would they have been better at their job given that ability. So it is not surprising that one can severely adjust the behavior of Vermin through the use of chemicals and sounds. Identifying the sounds and smells that a particular group of Vermin will respond to is difficult (requiring a DC 30 Knowledge (Nature) check), but actually producing them is not particularly. Here is a list of possible behavior modifications one can achieve and the Perform or Craft (Alchemy) check required:

\listone
	\bolditem{DC10 - Rampage}{It's a very simple behavior modification to cause a rampage. The spiral search pattern ends entirely and all affected Vermin take off in a random direction and move at full speed or until their path is blocked by a creature.}
	\bolditem{DC15 - Ignore}{There are chemicals that cause Vermin creatures to simply ignore}
	\bolditem{DC20 - Attack}{}
	\bolditem{DC20 - Shut Down}{}
	\bolditem{DC35 - Command}{}
\end{list}

\section{Constructs: Durability at a Price}

\desc{Like the Undead, the Constructs suffer tremendously from the fact that they have been over generalized. It is of course thematically appropriate for a Golem to be tireless and work day and night on whatever its last command was for as long as day follows night and night follows day. But it is also thematically appropriate for a clockwork beast to wind down and ``pass out'' as it continues to work long or strenuous schedules. Similarly, while it is fine and more than fine for an implacable lump of animated steel to be immune to critical hits, the very idea that there aren't key locations on a geared robot or a colossus given life by a mystical forehead rune is patently ridiculous. The construct type, therefore is filled to the brim with stuff that has no business being there, and this harms the game. The immersiveness of the story is depleted when players cannot rationally deduce what effects a being is resistant or vulnerable to, and anyone who's ever slapped washing machine or tripped over a playstation knows that there's no excuse for a machine to be immune to stunning.}

So here it is, the Construct Type. Pared down to the things it should actually do. Remembering of course that the Type itself should contain only those effects that one would want to be a universal law for all constructed beings, rather than rules one could imagine being situationally appropriate for one construct or another:

\listone
	\bolditem{Low Light Vision:}{Sees twice as far in limited illumination.}
	\bolditem{Dark Vision:}{60'}
	\bolditem{Poor Healing:}{Constructs can be healed by any of a number of means but do not heal for periods of rest. A construct's daily healing rate is 0 hp (though of course a construct with Fast Healing has a healing rate \emph{per round} and likely doesn't care).}
	\bolditem{Mindless:}{Even an intelligent construct has a synthetic mind that is unreachable by sorcery. A construct is not affected by [Mind Affecting] effects and cannot be detected with \spell{detect thoughts}.}
	\bolditem{Never Alive:}{A construct cannot be raised or resurrected. A construct is likewise immune to energy drain.}
	\bolditem{Repairable:}{A construct does not become staggered at 0 hit points, nor does it die at -10. If for some reason you are using the ``Death by Massive Damage Rule'', constructs aren't affected by it. As soon as a Construct hits zero hit points it becomes inert, and any abilities it may have cease to function (including fast healing abilities). However, a construct in this state can still be brought to working order again with a Craft check with a DC equal to the DC to make it in the first place with a base amount of time of one hour per hit point below 1 the construct was left at.}
	\bolditem{Nonbiological:}{Constructs do not eat or breathe, constructs do not age.}
	\bolditem{Lacks Squishy Bits:}{A construct is not affected by any effect that allows a Fort save unless that effect affects objects or is a (Harmless) effect. For example, a clockwork horror is not going to catch red fever or become nauseated by a stinking cloud. But it is not outside the realm of possibility for an eidolon to be afflicted with a totally magical disease that functions off of Willpower saves.}
\end{list}

All the stuff about constructs being ``immune to necromancy'' is out the window (because we all know that you can use \spell{magic jar} to put your soul into a statue); all the stuff about constructs being immune to ability damage is out the window (because we all know that you can slow down a lumber construct); and of course the immunity to critical hits is \emph{totally} out the window (if you have the name of Pelor on your forehead there is at least one critical location that probably won't go well for you if it is hit).

\subsection{Controlling Constructs: Robot Armies and Statuary Servants}

\desc{Time and time again adventurers report finding constructs that have been left attending temples and castles long after those buildings have fallen into ruin. The reason for this is twofold: First, constructs don't age; and Second, constructs don't count as one of your eight constant magical items if they are set to guarding a location. This means that powerful wizards are actually encouraged to leave their golems places with patrol or sentry orders and then of course these sentry golems will have a tendency to outlive the wizards, and even the buildings that they guard.}

Of course, it's entirely possible to make your constructs follow you around. If you do, they count against your 8 item limit.

\abox{Behind the Curtain: Why the Lower CRs?}{A cohort, or a planarly bound outsider, or a necromantically crafted monster could all plausibly be of a CR that is just 2 less than your character level without particularly disrupting play. So it may seem pretty weird that the constructs one can order around are weaker than that. The reasoning is ironically because the tactical role of a construct is so different from that of a Ghoul or Jarilith. While many potential servant creatures are simply weaker versions of normal characters or dangerous and fragile glass cannons -- in almost all cases a construct is an offensively anemic unit with a highly powerful defense. For those of you who have played tactical games or MMOs, that makes the average construct an ideal ``pet''. A strong defense is disproportionately useful for secondary characters expected to travel in front, and the fact that characters aren't allowed to fill their magic item cap with cohort level constructs is no accident.}

\subsection{Specific Constructs Under the New Rules}

% BIG HONKIN' NOTE:
% This uses \subsubsection instead of \monster. This is because we put 
% The Book of Gears in its entirety in its own appendix instead of moving 
% stuff around (I did it this way because the book is unfinished and it would 
% be awfully weird and illogical to move stuff around until it is finished).
% 
% So, when we finally move stuff around and put the simulacrum in the "Monsters" 
% chapter, we're gonna want to use \monster (which creates a subsection and a label 
% for hyperlinking) instead of \subsubsection.
% -Surgo
\subsubsection{Simulacrum}
Whether created by an Effigy Master, a mystic location or some other powerful source of illusion magic, a simulacrum is a construct made of ice and snow which appears to be a normal living creature through the power of illusion. Some 

\noindent\monstersizetype{Medium}{Construct}
\monsterline{Hit Dice}{6d10+6 (39 hit points)}
\monsterline{Initiative}{+1}
\monsterline{Speed}{30'}
\monsterline{AC}{11 (+1 Dex); Flat-footed 10; Touch 11}
\monsterline{BAB/Grapple}{+4/+5}
\monsterline{Attack}{Glamersword +5 melee (1d8+1)}
\monsterline{Full Attack}{Glamersword +5 melee (1d8+1)}
\monsterline{Space/Reach}{5'/5'}
\monsterline{Special Abilities}{Glamered, Imprinting}
\monsterline{Ability Scores}{Str 13; Dex 13; Con 13; Int 15; Wis 15; Cha 15}
\monsterline{Saves}{Fort +3; Reflex +3; Will +4}
\monsterline{Skills}{Bluff +11; Disguise +13 (+23 when Imprinted); Gather Information +11; Sense Motive +11}
\monsterline{Feats}{Impersonation}
\monsterline{Alignment}{As creator}
\monsterline{Organization}{Thrall}
\monsterline{Challenge Rating}{3}

It is important to note, however, that simulacra are entirely capable of using equipment, and usually will do so. Like most constructs, a simulacrum's true power comes to the fore when gifted with some basic mundane and magical equipment. Here is a sample simulacrum which has been given a magic shield, a magic breastplate, and a Frost Sword -- all equipment which is well within the capabilities of an Effigy Master to acquire or produce. While the simulacrum is still a ``CR 3 Creature" -- once it has been armed and equipped it is \textit{much} more formidable.

\noindent\textbf{Simulacrum with Equipment}\\
\monstersizetype{Medium}{Construct}
\monsterline{Hit Dice}{6d10+6 (39 hit points)}
\monsterline{Initiative}{+1}
\monsterline{Speed}{30'}
\monsterline{AC}{22 (+1 Dex, +7 Armor (Magic Breastplate), +4 Shield (Magic Shield)); Flat-footed 21; Touch 11}
\monsterline{BAB/Grapple}{+4/+5}
\monsterline{Attack}{Frost Sword +7 melee (1d8+3, +5 Cold Damage)}
\monsterline{Full Attack}{Frost Sword +7 melee (1d8+3, +5 Cold Damage)}
\monsterline{Space/Reach}{5'/5'}
\monsterline{Special Abilities}{Glamered, Imprinting, Ignore first 5 points of nonlethal damage (from armor), +2 bonus on bull rush attempts (from shield)}
\monsterline{Ability Scores}{Str 13; Dex 13; Con 13; Int 15; Wis 15; Cha 15}
\monsterline{Saves}{Fort +3; Reflex +3; Will +4}
\monsterline{Skills}{Bluff +11; Disguise +13 (+23 when Imprinted); Gather Information +11; Sense Motive +11}


\section{Denizens of the Planes of Law}

When you think avatars of Evil in D\&D it is no trouble at all to conjure up images of spiteful devils and destructive demons; but when you talk about a being of \emph{Law} the image that comes up is simply not the same from one person to another. Part of that is because Law doesn't really mean anything consistent in D\&D nomenclature. And part of that is because the actual description of the inhabitants of Mechanus has changed wildly through the generations and editions.

\subsection{Modrons: Singularity of Purpose}
% By Frank Trollman

\desc{For those of you who don't remember: Modrons are the original creatures of Law from the old days of AD\&D. They haven't been seen very often because they were originally written as a joke. Their very existence is as offensive to many players as the fact that they were essentially retconned out of existence is to others. And what's that all about? It's because the Modrons were originally written up as giant dice. Yes, really. The different types of basic Modron are shaped like four sided dice, six sided dice, 8 sided dice, the whole thing.}

\desc{So if your DM jumps on the ``let's forget this ever happened'' bandwagon, we understand. The original write up of the Modrons was actually pretty insulting. But since then there have been a number of variously successful attempts to rehabilitate them and make them independently awesome. Different Modron art has been made by Tony DiTerlizzi and Eric Campanella that looks pretty darned awesome � and not like your DM put a 6 sided die on the battle mat at all. Instead each Modron looks like a ghastly hybrid of metal and flesh covered with cogs and wheels where spindly appendages emerge from a solid (though not rollable) core.}

\desc{So assuming that you use some of the reform Modrons from late in 2nd Edition, the Modrons are actually pretty cool. They represent the idea of Law as an implacable and incomprehensible force. They are at their best when portrayed as being so single mindedly focused on some long term goal that they actually don't even care about you. Sometimes they destroy your village, sometimes they don't, and there's really no predicting that sort of thing unless you're knowledgeable about the Big Plan. Now I know what you're thinking� that having a plan so convoluted and far ranging that mortal minds cannot grasp it or predict its unfolding is actually indistinguishable from not having a plan at all and just performing actions at random. And yeah� that's true. That's DnD alignment for you.}

\desc{The Modrons come from a city in the Clockwork Nirvana called Regulus and have a rigid caste system where more powerful Modrons are told more of ``the plan'' than less powerful Modrons are told less. Each Modron is told exactly as much as it needs to know to complete its assigned tasks. And in the face of a long term plan of this magnitude, that pretty much means that every Modron is kept entirely in the dark about just what the heck it is doing or why it is doing it. The Modrons are arranged into a rigid caste system with no possibility (or concept) of personal advancement. That would be pretty stultifying if they were like humans where they all started out equal, but they aren't like that at all and Modrons of each caste sincerely don't have any desire to move up to another caste. At the very bottom (or at least, ``most numerous and least clued in to the plan'') there are the cogs. Cogs look like little gears and have the ability to transform into any object of roughly a cubic foot or less. When properly supervised by a higher level drone, they can take the shape of quite complicated pieces of clockwork and frequently do so. Above them are various drones of various shapes and sizes, each constructed of materials mechanical and biological to fulfill its role in the great plan.}

\desc{At the top there is Primus, who has been variously described as everything from an Intelligent Item to a reasonably powerful Outsider to a guy working the machinery behind a curtain. Seriously, your campaign can potentially reveal anything you want it to about what is really in the center of Regulus because that's been retconned so many times that noone knows what the official answer even is right now. Primus has been killed off several times in official continuity as part of various authors attempting to delete the Modron race from D\&D. However, someone always brings Primus back, because D\&D never really throws anything away (except the pygmies, that was too racist even for the 1980s). If you demand continuity in your life, then it seems that Primus simply can be killed time and time again, each time getting replaced with a new Primus who is in turn enlightened as to the nature of the big plan and granted the authority to turn the wheels of Regulus.}

\desc{The Modrons mostly sit around and operate the machinery in Regulus that apparently keeps all the giant gears and pistons running in the Clockwork Nirvana of Mechanus. So even if they exist in your game's continuity, chances are good that you'll never encounter them. Every so often, a whole lot of Modrons open up gates to other planes and start wandering around in big groups doing� stuff. The amount of time that Modrons spend between their ``March'' is supposed to be constant but actually every single Modron March that has ever been mentioned in any published adventure or story has taken place out of sequence so one is forced to conclude that actually the Modron March happens whenever the big plan calls for it and rumors of a great schedule are just rumors. The Modron March is generally not harmful and the Modrons don't seem to go out of their way to chase anyone who gets out of the way, so it doesn't seem to be an invasion. Although who knows? The plan of the Modron is so incredibly far reaching that it is entirely possible that they simply walk around in huge armed groups wandering seemingly aimlessly through the planes at irregular intervals doing no harm to anyone so that at some later date they can do the same thing and then just destroy some enemy that is predicted in the distant future.}

\desc{Unfortunately, now I'm going to have to talk about the Hierarchs. These are a layer of bad asses that live in the eco niche between drone and god. They are roughly equivalent to the high end fiends and celestials � coming in various flavors and power levels like Mariliths and Balors do for the demons. Unfortunately, noone has ever overhauled the art on these guys to the point where they don't look like ass. Sorry, the Hierarchs of Regulus look like they were drawn by Napoleon Dynamite and there's nothing nice I can say about them. If I were personally inclined to run with these guys at all I would be forced at gun point by the players to have them look like� something else. Putting them back into the theme of the Modron Drones is probably best, because at least then they appear to be Modrons in the same way that a Trumpet Archon is readily identifiable as an Archon. Which means that honestly the Hierarchs are much cooler if they look like Daleks. And the truth of that statement is probably the single greatest argument for walking away from the whole thing, as advised by Monte Cook.}

\desc{So what do Modrons do in a story? Mostly they show up with a specific set of instructions that they attempt to fulfill. They speak their own language but the more powerful ones also speak additional planar languages (hope you speak Formian). If the player characters attempt to prevent the Modrons from doing their thing, the Modrons will fight. Otherwise they'll simply complete their task and leave. If other creatures attempt to stop the Modrons, they'll fight them. Modrons hold no grudges and have no loyalty but to the plan. They will seamlessly switch sides in a battle if a different group proves more detrimental to their mission. The Modrons don't know why they are doing what they are doing, only Primus does (assuming Primus exists, in some versions The Plan is actually a flaw in Mechanus and there is no reason for any of it). And yet they will fight to the death to complete their mission. With good art and weird dialogue, the Modrons can serve as dynamic antagonists or useful allies.}

\desc{Like an amazingly intricate puzzle box, the Modron Race unfolds, collapses, and progresses in countless ways, but at the end, it's still a puzzle box. The entire point of the Modrons is that you never get clued in to what they are doing or why they are doing it. But if you don't want to use them at all, I understand.}

\subsection{Inevitables: Enforcing Natural Law}
%By Quantumboost

The Inevitables are unusual among ``exemplary'' creatures from Mechanus in that they actually don't have the Outsider (Law) type; they're straight-up Constructs. They're still Extraplanar, and they still have the Law subtype, but they're forged from steel and other metals, or at least the closest Outer Planar equivalents. What they represent among the creatures of Law is the idea of Law as ``universal rules'' -- that there is some code which should be obeyed by all beings, and that if you don't abide by it super fighting robots from another dimension will come and stop you from disobeying if and when they find out.

\subsection{Formians: In Many, One}
%By Quantumboost

\quot{``There is no good for the bee that is bad for the hive.''}

\desc{The Formians are the current Lawful Outsiders, and true to their name are giant ants -- and that's most of what you need to know about them. Formians represent Law as Conformity -- conformity to their society, and removal of anyone who won't conform. They're expansionistic, militaristic, and each hive effectively acts as one organism as they spread across the gears. Like most portrayals of giant ant-people, Formians are telepathic and have a hive-mind. However, they are giant ant \emph{people}, and each individual Formian above Worker level has a level of mental ability at least what you might expect from a human. So interacting with them is a lot like working with the Borg - you can totally }

\desc{The most straightforward way to use the Formians is to have them show up and want to conquer your favorite hangout -- either taking away the people to work in their hive (which is a lot like what the Hobgoblins will do, except that there probably isn't any chance of advancement from the slave thing) or turning it into a new hive, superintelligent mind-enslaving queen ant and all. But you can also use them as more mundane antagonists or even as potential allies. Unlike Modrons, Formians actually have discernable goals and motivations, their motivations just all happen to be ``what is good for the hive''. If you're working towards a goal or offering a reward that they think will help the hive, they'll totally be willing to work with you and might not even bother you about the whole assimilation thing beyond handing out some pamphlets.}



\chapter{Mechanics with Class}

\classname{Gadgeteer}
\vspace{-8pt}
\quot{When my Armageddon Clock is complete, no-one will ever again laugh at me!}

\ability{Alignment:}{Gadgeteers can be of any alignment.}

\ability{Starting Wealth:}{6d4x10 gp (150 Gold)} 

\ability{Starting Age:}{As the Rogue} 

\ability{Hit Die:}{d8} 

\ability{Class Skills:}{Appraise (Int), Bluff (Cha), Concentration (Con), Craft (Wis), Decipher Script (Int), Disable Device (Int), Jump (Str), Knowledge [Any] (Int), Search (Int), Sense Motive (Wis), Spot (Wis), (Ab)Use Magic Device (Cha)}

\ability{Skills/Level:}{6 + Intelligence Bonus}

\begin{table}[htb]
\begin{small}
\begin{tabular}[h]{lp{1.9cm}p{0.7cm}p{0.7cm}p{0.7cm}p{7cm}ll}
Level&Base Attack Bonus&Fort Save&Ref Save&Will Save&Special&Daily&Charged\\
1st&  +0  & +2&  +2&  +0&Gadgetry, Coax Device  & 1& 2 \\
2nd&  +1  & +3&  +3&  +0&Invention  & 2& 2 \\
3rd&  +2  & +3&  +3&  +1&Evasion  & 2& 3 \\
4th&  +3  & +4&  +4&  +1&Good Gadgets  & 3& 4 \\
5th&  +3  & +4&  +4&  +1&Invention, Duct Tape  & 4& 4 \\
6th&  +4  & +5&  +5&  +2&Confuse Magic Device  & 4& 5 \\
7th&  +5  & +5&  +5&  +2&Specialty Field  & 5& 5 \\
8th&  +6  & +6&  +6&  +2&Greater Gadgets  & 6& 6 \\
9th&  +6  & +6&  +6&  +3&Invention  & 6& 7 \\
10th& +7  & +7&  +7&  +3&Crafted Companion  & 7& 7 \\
11th& +8  & +7&  +7&  +3&Rapid Rebuild  & 8& 7 \\
12th& +9  & +8&  +8&  +4&Giga Gadgets  & 9& 8 \\
13th& +9  & +8&  +8&  +4&Invention, Percussive Maintenance  & 9& 9 \\
14th& +10 & +9&  +9&  +4&Specialty Field  & 9& 10\\
15th& +11 & +9&  +9&  +5&Command Magic Device  & 10&10\\
16th& +12 & +10& +10& +5&Galactic Gadgets  & 11&11\\
17th& +12 & +10& +10& +5&Invention, Throw a Spanner at it  & 11&12\\
18th& +13 & +11& +11& +6&`Stronger, Faster, More Expensive'  & 12&12\\
19th& +14 & +11& +11& +6&Cybernetic Soldier  & 13&12\\
20th& +15 & +12& +12& +6&The Grand Contraption  & 13&13\\
\end{tabular}
\end{small}
\end{table}

\smallskip\noindent All of the following are class features of the gadgeteer.

\ability{Weapon and Armor Proficiency:}{Clerics are proficient with all simple weapons, light armor, medium armour, and any gadget they create.}

\ability{Gadgetry (Ex):}{The Gadgeteer is able to create Gadgets. As he gains levels, he learns more blueprints (according to the table). Some are ``Daily'' - the preparation involved means they are set up at the beginning of the day (requiring an hour total) and then once used, are unavailable for the rest of the day. Others are simpler, and can be used essentially at will, as long as they are charged (covered in the item description). Making multiples of the same thing does not allow more uses, for reasons unknown. 

\smallskip\noindent Devices can be handed to other people to use, however they are not automatically considered proficient (anyone with "All Martial Weapons" proficiency should be considered proficient with weaponlike Gadgets), and the Gadgeteer must continue to maintain the items (as part of his "pool" of Gadgets) for them to continue functioning. 

\smallskip\noindent Any save DC is 10 + half level + Int modifier. Anything that scales with level uses the character level.}

\ability{Coax Device (Su:)}{The Gadgeteer can always take 10 on Abuse Magic Device and Disable Device checks, even if threatened, rushed or on fire.}

\ability{Inventions (Ex):}{Any time the Gadgeteer becomes able to make an Invention, he may base it on anything he currently holds a blueprint to - at the time of gaining the ability (so they cannot be stored up for later levels to cash them all in for Galactic Gadgets). An Invention takes twice as long to build, but grants one of the following benefits when equipped, not counting as a magic item for the limit:

\smallskip

\begin{list}{$\bullet$}{\itemspace}
\item +1/3 levels Enhancement bonus to Int (round up) 
\item +1/4 levels Deflection bonus to AC (round up) 
\item +1/3 levels Resistance bonus to saving throws (round up) 
\item +1/4 levels Enhancement bonus to all attack rolls (round up) 
\item Damage Reduction 1/2 levels, overcome by Adamantine (round up)
\end{list}

\smallskip

\smallskip\noindent Additionally it may be used twice per day if a Daily, or truly at will without charging if Charged.}

\ability{Evasion (Ex):}{At third level the Gadgeteer gets Evasion.}

\ability{Good Gadgets:}{At level 4, the Gadgeteer starts to learn blueprints for Good Gadgets.}

\ability{Duct Tape (Sp):}{At level 5, the Gadgeteer gains the ability to perform quick repairs on things. He may cast Mending, Make Whole and Repair Light Damage at will, with a caster level equal to his character level.}

\ability{Confuse Magic Device (Su):}{At level 6, the Gadgeteer gains the ability to confuse magic items with an Abuse Magic Device check. If it possesses charges, adding +10 to the DC can activate the item without using a charge (this does not apply to single-use items). If it casts a spell that deals a type of damage, or grants protection against a type of damage, this can be changed for that use by adding +5 to the DC. It can be Widened or Extended by adding +5 to the DC as well. Finally, traps that use a magic form of detection are not triggered by the Gadgeteer.}

\ability{Specialty Field (Ex):}{At levels 7 and 14, the Gadgeteer gains a specialty field. This is an area of expertise with the type of Gadget in question. The benefits depend on the field chosen.}

\smallskip

\begin{itemize}\itemspace\begin{small}
\item \ability{Energy:} All Fire/Electricity/Light damage dealt by Gadgets is Empowered (multiplied by 1.5)
\item \ability{Locomotion:} Any movement speeds or teleport ranges are doubled 
\item \ability{Durability:} Any Gadget has twice as many HP and +10 to Hardness, which means fuck-all for ray guns but is awesome for steam tanks. 
\item \ability{Efficiency:} Any non-instantaneous duration is multiplied by 1.5 (round up) 
\item \ability{Destruction:} All damage (except for Fire/Electricity/Light) dealt by Gadgets is Empowered. 
\item \ability{Overcharge:} Any areas of effect are doubled in size. 
\item \ability{Homing:} Gadgets ignore soft cover and any Concealment less than total, and may re-roll failed Miss Chances.
\end{small}\end{itemize}

\smallskip

\ability{Greater Gadgets:}{At level 8, the Gadgeteer starts to learn the blueprints for Greater Gadgets.}

\ability{Crafted Companion (Ex):}{At level 10, the Gadgeteer may build a cohort, mount or familiar - a Construct with a CR at least 3 less than his level. As he levels, he may add to it, to keep the creature's CR up - or outright disassemble and rebuild into a new form. It takes 8 hours to construct. If it is destroyed, all he needs to do is repair it or build a new one.}

\ability{Rapid Rebuild (Ex):}{At level 11, the Gadgeteer becomes faster at making things he is familiar with. As long as he has built something once in the past, he may rebuild it in half the time it would take to build normally (including his cohort), and can use a full round action to restore 150 HP to any Gadget or Invention.}

\ability{Giga Gadgets:}{At level 12, the Gadgeteer begins to learn blueprints for Giga Gadgets.}

\ability{Percussive Maintenance (Su):}{At level 13, the Gadgeteer learns how to fix (and damage) objects with a solid kick. With a melee attack that must hit the AC of the target, he may duplicate a Heal or Harm effect that only works on Constructs (ignoring any usual immunities to Heal/Harm or supernatural effects in general).}

\ability{Command Magic Device (Su):}{At level 15. the Gadgeteer knows how to command magic devices that don't even belong to him: no Construct will ever attack him, even sentient ones, unless he attacks them first. Furthermore, with a DC (15 + CR of Construct) Abuse magic Device check, he may control it (as per Dominate Monster, except it affects the Construct despite immunities, and allows no save) for one minute, after which it cannot be controlled by him again that day.

\smallskip\noindent Additionally, as a standard action he may activate a magic item somebody else is using or holding, from up to 50 feet away, as long as he is aware of it and targets it. This adds +10 to the activation DC. He may also make an Abuse Magic Device check as an immediate action (DC = base DC to activate) to prevent it from being used, wasting the attempt. }

\ability{Galactic Gadgets:}{At level 16, the Gadgeteer begins learning blueprints for Galactic Gadgets.}

\ability{Throw a Spanner at it (Ex):}{At level 17, the Gadgeteer learns how to affect Constructs at a distance. The Percussive Maintenance ability can work with any ranged attack, though actually throwing a construction tool gains a +5 Circumstance bonus to the attack roll.}

\ability{Stronger, Faster, More Expensive (Ex):}{At level 18, the Gadgeteer learns how to make impressive golem bodies for people. With 24 hours and some amount of planar currency, he may transfer someone (even himself) into a body of iron, granting them the following benefits:

\smallskip

\begin{list}{$\bullet$}{\itemspace}
\item Type: Construct (Augmented (former type) as Subtype) 
\item Leave BAB, HP, ability scores (even Con) etc. all the same 
\item +6 Enhancement bonus to Strength 
\item Double one movement speed 
\item Damage Reduction 15/Adamantine 
\item Natural weapons overcome DR and Hardness as though made of Adamantine (they are) 
\item Not immune to Mind-Affecting Effects 
\item Always counts as Exhausted (ignore the immunity to such). A full round action spent winding them up (they can do this to themselves) negates this for one minute.
\end{list}}

\smallskip

\ability{Cybernetic Soldier (Ex):}{At level 19, the Gadgeteer adds pieces of gadgetry to himself to become a sort of cyborg. Don't think sleek Shadowrun stuff, he probably has a key in the back of his head. He gains the following benefits:

\smallskip

\begin{list}{$\bullet$}{\itemspace}
\item Immune to Mind-Affecting Effects 
\item Medium Fortification 
\item -3 Circumstance Penalty to all Int-based skills unless a standard action is spent winding up or whatever, which turns it into a +3 Enhancement Bonus for one minute. 
\item Low-Light Vision and Darkvision 120' 
\item Always-on Detect Magic effect
\end{list}}

\smallskip

\ability{The Grand Contraption (Ex):}{At level 20, the Gadgeteer wins the game.}

\subsubsection{Gadgets}

\noindent\underline{Flamethrower} [Daily]
 
\noindent This device must be wielded in both hands to be used, and requires a Standard action to activate. This creates a 30' cone or 60' line of fire, dealing 1d10 Fire damage per level (Ref half). Those who fail the save catch fire. Additionally, the area is filled with thick smoke that causes Concealment and forces a Fort save to all in the area, against being Nauseated for one round due to choking. It dissipates after 1d4 rounds. 

\smallskip\noindent This also ignites any unattended flammable objects in the area. 

\medskip\noindent\underline{Shotgun} [Daily] 

\noindent This device must be wielded in both hands to be used, and requires an attack action to activate. It can be activated twice before being used up for the day. Each activation deals 1d4 bludgeoning damage per level to a 20' cone (Ref half). Additionally, everyone within 30' must make a Fort save or be Deafened for 1 minute. 

\medskip\noindent\underline{Mining Drill} [Daily] 

\noindent This device must be wielded in both hands to be used, and requires a Standard action to activate. It then lasts for one minute, though requires a Standard action to utilise each round. With a Standard action it can destroy an adjacent 5' cube of anything softer than stone, or deal 1d6 piercing damage per 2 levels (round up), ignoring material DR or DR/magic, to a target. 

\smallskip\noindent As a steam-powered device, this is quite noisy and grants a -10 penalty to Move Silently checks when activated. 

\medskip\noindent\underline{Hover Pack} [Daily] 

\noindent This must be worn on the back, and requires a Swift or Immediate action to activate. it then lasts for one hour, causing the wearer to fall slowly (as Featherfall), and hover up to 20' above the ground (though only able to move up to 15' per round in a given direction). 

\medskip\noindent\underline{Dynamite} [Daily] 

\noindent This device is small enough to be carried or stored in a pocket, and requires a Swift action to activate. Three rounds later, hopefully with the user nowhere near it, it explodes in a 50' radius spread. This deals 1d10 Fire/Sonic damage per level to all in the area, Ref half, and anyone who fails the save is knocked prone. Additionally, it specifically affects all objects and structures in the area (including individual 5' cubes of terrain), dealing double damage and ignoring Hardness less than that of Steel. 

\medskip\noindent\underline{Shock Staff} [Charged] 

\noindent This device can be wielded in one hand, and has reach. It requires a Move Equivalent action to charge up by winding, then holds the charge for up to a minute or until used. A melee touch attack activates it, dealing 1d6 Electricity damage per 2 levels (round up) and requiring the target to make a Fortitude save. If they fail the save, they are Dazed for 1 round. 

\medskip\noindent\underline{Glider} [Charged] 

\noindent This device can be folded up to take little space, but must be worn in the Cloak slot to be used. It is effectively charged by any rush of air, so any time you would want to use it, it is going to be charged anyway. This causes the wearer to slowly fall instead of plummeting to their death, at the rate of 10' per round, but they may also move up to 20' per round in any other direction except for up. 

\medskip\noindent\underline{Launch Boots} [Charged] 

\noindent These bulky boots must be worn on the feet (duh), and require the user to move 30' or more in order to charge, at which point they remain charged until used, which requires a Move action. Doing so launches them up to 50' in the air (this can be adjusted on use in 10' increments so as to only go, say, 20' up). The jumper may move as far forward as upward. The fall afterwards is treated as though 10' less for the purpose of working out falling damage. 

\medskip\noindent\underline{Harpoon} [Charged] 

\noindent This weapon must be wielded in both hands to activate. it requires a Full Round action to fully wind up, and can then be activated at any time with a Swift action, firing out to 60' away. It deals 1d6 piercing damage per level, and if it deals any damage (after DR/Hardness), sticks in firmly, requiring a DC 20 Strength check to remove. A cable attached can be wound up as a Move-equivalent action, allowing the user to either pull himself towards the landing point, or drag a target towards him with an opposed Strength check. 

\medskip\noindent\underline{Steam Spray} [Charged]
 
\noindent This weapon must be wielded in both hands to activate. It requires one minute to charge up, but will then boil away happily for up to 8 hours or until used up. Charging it provides 3 activations, each one requiring a Standard action. Doing so creates a 20' cone of steam, dealing 1d4 Fire damage per level (Ref half) but also really cleaning the targets, though they may not appreciate it. Failing the save causes a -2 morale penalty to attacks and a -4 morale penalty to AC for 3 rounds. 

\medskip\noindent\underline{Wind-up Flare Lantern} [Charged] 

\noindent This device may be freely worn on the shoulder or even on top of a helmet, but takes a free hand to activate or adjust the settings. Winding it up for one minute provides a charge to keep it providing light (as per a Light) spell for one hour or a Daylight effect that even reveals Ethereal/Incorporeal creatures for 10 minutes. At any time a Standard action may be used to end this effect, suddenly using up all the remaining power in one flash. This forces all in 50' to attempt a Fortitude save or be rendered Blind for 1 round per level. Those warned may cover their eyes with a Reflex save. 

\medskip\noindent\underline{Proximity Detector} [Charged] 

\noindent This device takes up no slot, whirling around the wearer. If it is moderately windy (enough to potentially knock a lightweight travelling hat off a head or lift a summer dress), this requires no charging at all, otherwise it requires the use of a Swift action every turn to turn a crank that helps the wheels spin. 

\smallskip\noindent Using sonar echoes, it grants Blindsense out to 15' and Trapsense to the wearer. 

\subsubsection{Good Gadgets}

\noindent\underline{Pressure Jetpack} [Daily] 

\noindent This device must be worn on the back, and requires a Standard action to activate. It then fires a blast of steam, lifting the wearer into the sky and allowing them to Fly with a speed of 40' (Average). This lasts for 1 hour before it runs out of water and overheats, used up for the day. It will sputter and slowly lower to the ground when this happens. 

\medskip\noindent\underline{Oil Spray} [Daily] 

\noindent This device can be carried and fired with just one hand, and firing it takes a Standard action. A 40' cone is filled with a fine mist of oil - the ground is affected as though by Incendiary Slime, as are all people in the area, and all in the area must additionally make a Fortitude save or be Nauseated for 1d4 rounds. 

\medskip\noindent\underline{Gatling Gun} [Daily] 

\noindent This device must be carried in both hands when activated, requiring a Full Round action to do so. It then fires for that entire round, and the next two rounds (Standard actions are required to keep control of it for this duration) before finally running out of ammunition. Each round of use, one target of choice within 60' takes 1 hit per 3 levels that each deal 4d6 Piercing damage (Ref negates). For every save passed, the next target behind them (providing they are also in range) is hit and must save, and so on until you run out of targets/range/hits. 

\medskip\noindent\underline{Cannon} [Daily]
 
\noindent This device can be assembled or taken apart with a Full Round action, but must otherwise be rolled or carried about. To activate, it must be set on the ground, braced and lit, requiring a Full Round action. On the following round it fires, Bullrushing the user with a bonus of +10, and deafening all in 30' on a failed Fort save. However even more damage happens at the other end, where the shot lands: one 40' radius blast is designated up to 250' away, dealing 2d6 Fire/Bludgeoning damage per level to all in the area (Ref half). Those who fail the save are knocked prone, and Stunned for 1 round. 

\medskip\noindent\underline{Defibrillator} [Daily]
 
\noindent This device can fit inside a regular backpack (barely), but to activate, one component must be held in each hand (assuming two hands), and a Standard action used on an adjacent ally. This causes an incredible bolt of electricity, burning out the mechanisms. This duplicates a Raise Dead spell, but with no level loss, or can stabilise anyone who is Dying. 

\medskip\noindent\underline{Spy Balloon} [Daily]
 
\noindent This device can easily be stored in a pouch when not in use, then takes only one Full Round action to activate. A small balloon fills with air and starts to drift at the speed of 50' per round, assisted by small nozzles controlled via wire attached to the control box. It has a glass ball connected to a long fibre leading back to a ``screen'' on the control box, as well as a basic speaker system: two horns connected by string. Effectively, it can be controlled out to 1000' and requires a Spot check of DC 10 + class level + Int mod in order to spot. It allows the user to see and hear through it, and pass messages on as well. After two hours of use, however, it runs out of air and must be rewound and stored away, to be refilled and the wires repaired the next day. 

\medskip\noindent\underline{Cog Powered Fist} [Charged]
 
\noindent This device must be worn on one hand, like a bulky gauntlet, and provides a -3 penalty to skill checks involving that hand. It must be wound up to be activated, requiring a Standard action to provide enough power for five rounds. When powered, it can be used to make Slam attacks that deal 2d6+Str for a Medium creature, with a critical value of 20/x3. It also adds a +6 Enhancement bonus to the wearer's Strength, and grants a Constrict attack of automatic Slam damage. 

\medskip\noindent\underline{Pump-Action} [Charged]
 
\noindent This device must be wielded in both hands to be used, and requires an attack action to activate. It can be activated eight times before being used up, needed 10 minutes to make another 8 shells, and one Full Round action to load them all in. Each activation deals 1d4 bludgeoning damage per level to a 20' cone (Ref half). Additionally, everyone within 30' must make a Fort save or be Deafened for 1 minute. 

\medskip\noindent\underline{Frost Blaster} [Charged] 

\noindent This device is worn as a back-pack, but to be used it also has an attachment that must be wielded in one hand. It has three uses per charge, with a minute required to re-fill it. With a Standard action, it unleashes a very cold spray in a 20' cone. This deals 1d4 Cold damage per level (Fort half), and all who fail the Fortitude save become Entangled for one minute or until exposed to a strong source of heat. Anyone already Entangled instead becomes Slowed for one minute or until exposed to a strong source of heat, and anyone already Slowed becomes Paralysed for one minute or until exposed to a strong source of heat. 

\medskip\noindent\underline{Lightning Rod} [Charged]
 
\noindent This device can be carried in one hand quite easily. It can be charged either by winding it up to generate power (requiring one Standard action per charge) or by being hit by a source of electricity that would do at least 5d6 Electricity damage (providing one charge). it can hold one charge at a time, and can be discharged with a Standard action. Doing so unleashes a Lightning Bolt (as per the spell), except the damage is uncapped. Additionally, if it is not charged, it will automatically absorb any electricity-based attacks aimed at the wielder (until it is charged). 

\medskip\noindent\underline{Jetskates} [Charged]
 
\noindent These are worn on the feet like bulky boots. To create a fuel charge requires a minute of work, and they can hold up to three charges at a time. Activating them requires only a Swift action, and allows the user to move at 100' per Move action and skim over water, but they cannot fly. Turning is possible but difficult and stopping is out of the question, as though flying [Clumsy]. Each charge lasts for one round. 

\medskip\noindent\underline{Fuel-Efficient Flamethrower} [Charged]
 
\noindent This device must be wielded in both hands to be used, and requires a Standard action to activate. This creates a 30' cone or 60' line of fire, dealing 1d10 Fire damage per level (Ref half). Those who fail the save catch fire. Additionally, the area is filled with thick smoke that causes Concealment and forces a Fort save to all in the area, against being Nauseated for one round due to choking. It dissipates after 1d4 rounds. 

\smallskip\noindent This also ignites any unattended flammable objects in the area. Refilling the device with fuel takes three full rounds. 

\medskip\noindent\underline{Flashbang Generator} [Charged] 

\noindent This device is carried as a staff, and holds one charge at a time. It takes three rounds to mix up another charge to load in, but only a Move equivalent action to activate. Activating it creates a bright flash of light (as the Flare Lantern, double the duration) and a loud ``bang'', causing those who pass the save to be Deafened for 2d6 rounds, and those who fail to be Stunned for 1 round and Deafened for 2d6 minutes. 

\subsubsection{Greater Gadgets}

\noindent\underline{Cog Staff} [Daily]
 
\noindent This staff must be held in two hands to use properly. It requires a Standard action to activate by twisting, the cogs turning and grinding to produce a charge of energy similar to magic. This burns the cogs out, but creates an (Ex) effect from the following list:

\begin{list}{$\bullet$}{\itemspace}
\item Shadow Evocation 
\item Eyebite (2 or more HD more than inventor: Sickened, within 1 HD of inventor: Panicked and Sickened, 2 or more HD less than inventor: Comatose, Panicked and Sickened) 
\item One 10' radius burst in 60' takes 2d6 Fire damage per level (Fort half), with those who fail the save becoming Nauseated for one round 
\item Maximised Lightning Leap
\end{list}

\medskip\noindent\underline{Teleport-Pump} [Daily] 

\noindent This fold-out device can be carried in one hand or stored in a pack until unfolded (a Swift action). It then takes a Full Round action to pump up until it sparks, frying the fuse and teleporting the user as per Teleport Without Error. 

\medskip\noindent\underline{Fume Engine} [Daily] 

\noindent This device is bulky, but fits in a backpack. It also comes with a gas mask that can be worn on the face to provide immunity to inhaled Poisons and Diseases, as well as gases and smell-based effects. The main device, however, can only be used once per day, with a Swift action to activate. One round later, the engine starts, running for one minute before being empty. For the duration, it creates a 20' radius spread of Poison gas. It obscures vision, granting Concealment to all, but also poisons all who breathe it in (Fort negates). Primary and Secondary damage are both 2d6 Con. 

\medskip\noindent\underline{Boiler-Powered Armour} [Daily]
 
\noindent This armour must be worn all over the body, except the hands, feet and head. It is treated as Adamantine Full Plate normally, but when activated (a Standard action to start the boiling process, then a one minute wait), it provides 4 hours of uninterrupted power. When powered, the user gains a +6 Enhancement bonus to Constitution, and replaces the Armour bonus with a Deflection bonus to AC. Additionally, they gain a +10 bonus to Break checks and their melee attacks ignore Hardness. As long as at least one full hour remains the remainder may be used up in one burst, granting a +10 Enhancement bonus to Strength and treating the wearer as two size categories larger, both for only one round. It is a Swift action to do this. 

\medskip\noindent\underline{Turbine Laser} [Daily] 

\noindent This weapon must be held in two hands, with an additional mount on a shoulder or on the head, consisting of a turbine. It requires some form of breeze to charge, providing one charge per two rounds unless in strong winds where it provides a charge every round, or hurricane force winds, where it provides three charges in one round. It may store up to three charges at a time, and requires an Attack action to fire - however it can only be fired once per day, the power destroying the laser components. 

\smallskip\noindent Once activated, it makes a Ranged Touch Attack with a maximum range of 200'. This deals 1d6 untyped energy damage per level and has a regular critical hit (20/x2). If two charges are stored when it is fired, it deals 2d6 damage per level. If three charges are stored when it is fired, it deals 3d6 damage per level. 

\medskip\noindent\underline{Wand-Gatler} [Daily] 

\noindent This device can be wielded in one hand, though a second hand is needed to actually turn the handle. One to five wands must be slotted in before use. Turning the handle is a Standard action, and activates all wands at the same time, with the same target/targets/area of effect (or closest match). The magical overload burns the engine out. 

\medskip\noindent\underline{Aero-Amplifier} [Charged]
 
\noindent This device can be held in both hands or worn around the shoulders. It resembles a giant brass instrument, with one handle. As long as there is even a light breeze or source of sound, a Standard action can be used to squeeze the handle, opening the valves and amplifying the noise and airflow. This creates a 30' cone that deals 1d6 Sonic damage per level (Fort half). All who fail the save are knocked prone and hurled 50' backward, +/- 10' for every size category smaller/larger than Medium. Those who pass the save are merely shoved back 10'. 

\medskip\noindent\underline{Simple-Ammo Gat} [Charged] 

\noindent This device must be carried in both hands when activated, requiring a Full Round action to do so. It then fires for that entire round, and the next two rounds (Standard actions are required to keep control of it for this duration) before finally running out of ammunition. Each round of use, one target of choice within 60' takes 1 hit per 3 levels that each deal 4d6 Piercing damage (Ref negates). For every save passed, the next target behind them (providing they are also in range) is hit and must save, and so on until you run out of targets/range/hits. 

\smallskip\noindent Once it runs out of ammunition, it takes only 10 minutes to make a new chain of ammunition, and a Standard action to load in. 

\medskip\noindent\underline{Turbo Boost} [Charged] 

\noindent This device is worn like a vest, clamped onto the torso. It has a key which must be wound to power it up. One Full Round action of winding supplies one minute of power, treating the wearer as though under a Haste effect. 

\medskip\noindent\underline{Clockwork Chainsaw} [Charged] 

\noindent This bulky melee weapon must be wielded with two hands, and requires one Full Round action to provide power for three rounds. Up to 9 rounds of power can be charged up at a time. Using it is as simple as making a melee attack, where it has the following profile (treated as a magic weapon): 

\begin{list}{$\bullet$}{\itemspace}
\item +1 Adamantine Chainsaw, 3d6+1 (+ Str*1.5) Magical Slashing, 18-20/x3 
\item Flesh Tearing: anyone hit by the Chainsaw takes 1 point of Ability Damage to each of Strength, Dexterity and \item Constitution. This is tripled on a critical hit. 
\item Romero Effect: this weapon can score critical hits against corporeal Undead. 
\end{list}

\medskip\noindent\underline{Buggy} [Charged] 

\noindent This device cannot be carried (save by a very strong character) but can instead carry four Medium creatures (or any mix that equals the same amount of space). It requires one minute to power up via turning the handle and pouring more basic fuel in, in order to keep it moving for one straight hour. This device has 200 HP and Hardness 15, and provides Full Cover to everyone inside. It trundles along at 100' per round (and cannot run, make double-move actions etc.) 

\smallskip\noindent Spells such as Heat Metal have an especially intense effect on those inside, causing double damage. Being rammed (requiring a ``ranged'' attack by the driver that does not benefit from Rapid Shot or whatever) deals 2d8+10 bludgeoning damage and a Bullrush at +13 (+8 size, +5 Strength) 

\medskip\noindent\underline{Perpetual Pressure Jetpack} [Charged] 

\noindent This device must be worn on the back, and requires a Standard action to activate. It then fires a blast of steam, lifting the wearer into the sky and allowing them to Fly with a speed of 40' (Average). This device somehow recharges itself, using steam power at first, which also turns turbines to store up power, letting it be used at will. 

\subsubsection{Giga Gadgets}

\noindent\underline{Hurricane Device} [Daily]
 
\noindent This device requires ten minutes of assembly (if the user doesn't wish to carry around a 20' long pole with a rotor on top), and must be placed on the ground before use. Activating it simply requires turning the handle for one minute, at which point the storm begins. It creates a hurricane with a 2 mile radius, lasting for 6 hours, with a safety zone in the eye of the storm - a 50' radius spread originating from the device itself. If the device moves, the storm moves, and if the device is disassembled or destroyed, the storm will simply remain in place for the duration. 

\medskip\noindent\underline{Disintegration Torch} [Daily] 

\noindent This device can be carried in one hand, and requires only a Standard action to activate. The fuel ignites, sending out a blast of plasma that is hot enough to effectively annihilate whatever it torches. One creature, object or 15' cube of material must be hit with a melee touch attack and, if hit, must succeed on a Fortitude save that affects objects. If they fail the save, they are reduced to a carbon shadow (this is not a [Death] effect). If they succeed, they still take 2d6 Fire damage per level. Should this be enough to kill them, they are also completely destroyed. The heat is even sufficient to destroy [Force] effects. 

\medskip\noindent\underline{Mechanical Suit} [Daily] 

\noindent This suit is very similar to the Steam-Powered Armour (above), but is powered by a fuel engine and cogs, and more powerful. It is treated as +3 Adamantine Full Plate normally, but when activated (a Standard action to start the engine), it provides 4 hours of uninterrupted power. When powered, the user gains a +10 Enhancement bonus to Strength and Constitution, and replaces the Armour bonus with a Deflection bonus to AC. Additionally, they gain a +10 bonus to Break checks and their melee attacks ignore Hardness. 

\smallskip\noindent The wearer is also treated as two size categories larger, gains a 4d8+2*Str Trample, and pumps out smoke. All creatures adjacent to the wearer must pass a Fort save each round or be rendered Nauseated for the round. Additionally, the furnace can blast excess fire out from large shoulder-mounted pipes, dealing 1d10 Fire damage per level in a 30' cone (Ref half, those who fail the save catch fire) as a Standard action. 

\smallskip\noindent The suit of armour has 300 Hit Points, and only stops functioning when reduced to zero.

\medskip\noindent\underline{Mind Control Helmet} [Daily] 

\noindent Although it isn't subtle, the device is still effective: it consists of a bulky remote that must be held in two hands, and a bulky helmet with an antenna and a key. The helmet, if placed on the head of a creature not immune to [Mind Affecting] effects, must then be activated with a Standard action by turning the key. The wearer is then under the control of whoever has the remote, for up to 4 hours or until the helmet is removed or destroyed, whichever happens first. 

\smallskip\noindent There is no saving throw against this effect initially, but any time the controller attempts to make them do anything particularly against their personal beliefs/ethics/morality/code, they are entitled to a Will save to resist that command. It requires a Standard action to input a command, and commands cannot be received from further than 100' away. 

\medskip\noindent\underline{Shadowcrank} [Daily] 

\noindent This device could fit into a small backpack, and has a large crank attached to a pipe, in turn attached to the device by a cable. Turning the crank for three rounds activates it, sucking darkness in through the plane of shadow and squeezing it out to replicate any of the following effects as an (Ex) ability: 

\begin{list}{$\bullet$}{\itemspace}
\item Mass Invisibility 
\item Shadow Conjuration, Greater 
\item Shadow Evocation, Greater 
\end{list}

\medskip\noindent\underline{Aero-Barrier} [Charged] 

\noindent This device can be worn as a backpack, and has a mouth-piece for activation. Activating it requires a Move-Equivalent action each round, and prevents speaking. Each round that it is activated, it provides a bubble of protection for one full round (thus allowing it to be maintained indefinitely). This bubble extends out to 5' in each direction past the space occupied by the user, and prevents anything from getting in, like a [Force] effect. An opposed check (Strength of intruder against Int of user) can break past this to force entry, though this requires a Standard action. If attacked from outside the Bubble, anything not requiring an Attack roll automatically fails, and against things that do require an attack roll, it is treated as a +10 Deflection bonus to AC. 

\medskip\noindent\underline{Scalding Mist Generator} [Charged] 

\noindent This bulky device must be carried in both hands to be activated, and as such, comes with protective clothing in the form of a thick, water-resistant sheet and goggles. Those who forego the goggles, feeling they do nothing, are not protected from the effects. 

\smallskip\noindent To activate, a mere Swift action is needed to switch it on, however it will take two full rounds for steam to start pouring out. Once this happens, it keeps going for a full minute, creating a 75' radius spread of mist that obscures vision. Additionally, all in the area who are not protected take an automatic 5d4 Fire damage every round (though no ignition occurs). Anyone not wearing protective goggles must also pass a Fort save or be rendered permanently blind as well as suffering a -2 Morale penalty to attacks and -4 Morale penalty to AC for 10 minutes. Those wearing goggles need only save against the penalties. The penalties are a [Pain] effect. 

\medskip\noindent\underline{Anti-Magic Pressure Generator} [Charged] 

\noindent This large pump can be carried in one hand, but needs two hands to use. It can be activated with a Standard action, using incredible pressure to force magic out of the area. This creates a 30' radius Anti-Magic Field, however it will reduce by 15' radius per round until ended. Every round that the device is activated, the radius extends by 25', allowing it to slowly grow in size if kept up. 

\medskip\noindent\underline{Prism Flash Bulb} [Charged] 

\noindent This device can be mounted on a shoulder or on the head and looks quite pretty, though looking pretty is not the main use. Powder must be loaded into it (a Move-Equivalent action) before it can be activated (a Swift action), and creating the powder only requires a minute of work. 

\smallskip\noindent When activated, it duplicates an (Ex) Prismatic Spray effect. I know it can banish people to other planes, that sounds pretty extraordinary to me. 

\medskip\noindent\underline{Airship} [Charged] 

\noindent This device cannot feasibly be carried, and actually requires 6 full hours to build. Once built, however, it is built to last, with 300 HP and Hardness 10 for the main section of it. It has an engine that must be stoked and filled every 10 minutes to retain power, and generally requires somebody to pilot it, spending a Standard action every round. When active, this device flies with a Fly speed of 150' (Average), and provides Full Cover for those inside, although it does have arrow slits. People may elect to climb up onto the roof so as to leap off, throw things etc. however this requires a Balance check to avoid falling off (10 if sailing smoothly, 15 for some turbulence, 20 for clouds and heavy turbulence, 25 for hurricanes or snowstorms, +10 for being in active combat). 

\smallskip\noindent It can carry six Medium creatures inside, or any other combination equating to the same. 

\medskip\noindent\underline{Steam Tank} [Charged] 

\noindent This device requires 4 hours to build, and only carries up to two Medium creatures. It provides Full Cover to those inside, though it has arrow slits that can be closed, and rumbles along at up to 40' per round. it has 300 HP and Hardness 20, and a +8 Stability bonus to avoid being knocked over. Every ten minutes, the boiler must be attended to (a Standard action) or it will run out of steam and stall. Best of all, a large cannon sits on the top, and can be rotated and fired at any angle, treated as the Good Gadget: Cannon. Crafting more shot and powder takes a full minute, and loading it from inside takes another full minute, so it is a slow process but a safe one.

\subsubsection{Galactic Gadgets}

\noindent\underline{Doomsday Cannon} [Daily] 

\noindent This device is similar to the Cannon, except three times as large in each dimension, and thus involves more effort to trundle it around. Firing it requires one round to simply get it aimed in the right general vicinity, then a Standard action to fire it afterwards. It can be fired anywhere out to 1 mile away, however cannot be fired at a point within 200 feet of itself. Everything within 500' of the point of impact has an extremely bad day. Everything - including objects and the ground itself - must make a Fort save or be obliterated. Even those who pass still take 2d6 Fire/Bludgeoning damage per level and are Stunned for 1 round and permanently Deafened. The terrain is also obliterated, effectively becoming a giant, cracked, smoking crater. 

\medskip\noindent\underline{Gate-Gun} [Daily] 

\noindent This device involves a one-handed gun attached to a belt. When fired (a Standard action), the point designated opens up into a Gate, as per the Gate spell. This can be the travel kind or the broken kind. 

\medskip\noindent\underline{Timestop Clock} [Daily] 

\noindent This looks like a simple watch, however a Standard action to halt it shows otherwise. Time is stopped for all but the one who activated it, as per the Timestop spell. If the watch then breaks or is destroyed, the effect ends. Otherwise, in 10 pseudo-rounds, time will resume as the tightly wound springs snap, smashing the watch to pieces. 

\medskip\noindent\underline{20MT Bomb} [Daily] 

\noindent This bomb is as large as a Medium creature, and weighs about 500lbs, so may be difficult to carry, but using it is worth the effort. To activate, one merely needs to strike the end of it (a Standard action, or dropping it from a height). At this point, everything within 500' takes 2d6 Fire damage per level (Ref half) and 2d4 Negative levels (Fort half), and everything out to 1 mile takes 1d6 Fire damage per level (Ref half) and 1 Negative level (Fort negates). The 0-500' radius area is then obscured completely by thick smoke for one minute. 

\medskip\noindent\underline{Dragonmech} [Daily] 

\noindent This Colossal device can carry only one Large or smaller pilot inside, sadly, and requires a minute to activate. Once activated, it can be controlled for a full hour, with a Fly speed of 100' (Poor). It has Hardness 20, an Armour Class of 35 (+0 Dex, -8 Size, +28 Armour, +5 Deflection), and 300 Hit Points. If used to attack, use the BAB of the pilot, plus the Strength (50) of the Dragon. It has one Primary Bite (4d6+Str*1.5), two Secondary Claws (2d6+Str*.5) and one Primary Tail (3d6+Str*1.5), as well as a breath weapon once per 3 rounds: a 100' cone of fire that deals 2d6 Fire damage per level (Ref half). Additionally, the eyes may fire one Lightning Bolt per 3 rounds, and once per minute it may create an Acid Fog effect beneath it with a Standard action. 

\smallskip\noindent It also has a payload of six bombs, any number of which can be dropped as one Standard action. They explode like Dynamite upon hitting the ground. The bombs may instead be replaced with a pair of Gatling Guns, though they may not be used if the dragon is unpowered. 

\smallskip\noindent When unpowered, it still lumbers along with a Fly speed of 50 (Clumsy) and provides the basic protection, but lacks the Deflection bonus, lightning, acid fog and fiery breath, and only has a Strength of 30. 

\medskip\noindent\underline{Storm Treadmill} [Charged] 

\noindent This device is fairly large and bulky, taking up most of a 5' square, but being only 1 foot tall. The user may activate it by setting it down and running in place atop it (a Full Round action). Doing so causes a Whirlwind or Storm of Vengeance effect, though the effect ends on the round the user stops running. 

\medskip\noindent\underline{Gravity Pump} [Charged] 

\noindent This device can be carried in one hand, but requires both hands and a Standard action to activate. Doing so creates a Reverse Gravity and Vertigo Field effect in a 100' spread, though the user can elect to not include his own occupied space (at time of use) in this area. This lasts until undone by a second activation, or for 10 minutes, whichever comes first. 

\medskip\noindent\underline{Boiler of Untouchability} [Charged] 

\noindent This boiler is big and heavy, but can still be carried in both arms. It must be set down to be used, and requires a Swift action to activate, though there is a three round wait as the boiler gets going. It then remains active for one minute, though towards the end of this, a Full Round action can be spent adding more water and fuel to extend it for another minute. 

\smallskip\noindent While active, a force field is created, creating a safety bubble of 30' radius. Nothing is able to pass through it, and it blocks line of effect. 

\medskip\noindent\underline{Binding Box} [Charged] 

\noindent This box is about 1' cubed, and can be carried with little difficulty. Activating it is a Full Round action, twisting the cogs to make them spin. The box opens and whirls, and one designated target within 100' is sucked inside despite the size, and trapped there as though by an Imprisonment effect. 

\smallskip\noindent Only one creature at a time can be stored there, and when released (a Standard action or the destruction of the device), there is a 50\% chance that they will be dead and a 50\% chance that they will be alive. This is only revealed when actually released - until that moment they are in a quantum state and cannot be affected by any effect that targets only the dead/undead (such as Resurrection) or only the living. 

\medskip\noindent\underline{Turbine of Death} [Charged]
 
\noindent This device can barely be carried in both arms, and is quite heavy. To activate, it must be set down and a Swift action spent pulling a lever. It then charges up for three rounds if there is a breeze (two rounds if in strong winds, one round in a hurricane, impossible if there is no breeze). After it has charged, it unleashes on the following round: everyone within 100' except for those actually touching it must pass a Fort save or die instantly. 

\medskip\noindent\underline{Fusion Torch} [Charged] 

\noindent This device works exactly like the Disintegration torch, aside from being Charged, except where noted below: 

\begin{list}{$\bullet$}{\itemspace}
\item It is effectively unlimited in use, always charged. 
\item It can be used with an Attack action. 
\item Passing the save still bestows a Negative level on top of the damage.
\end{list}
\classname{Monster Tamer} \label{class:monstertamer}
\vspace{-8pt}
\quot{"I choose you!"}

\ability{Playing a Monster Tamer:}{Monster Tamers do not make good front line fighters, although their short range thrown weapons can be devastating. They frequently need Fighting characters to soften up powerful Monsters for capture as well as to distract powerful Monsters long enough for a Monster Tamer to capture it. Sometimes a Monster Tamer will be attacked by creatures or adversity that are not Monsters, in such cases the abilities of Wizards and Sorcerers are invaluable - a Monster Tamer's relative dominance over Monsters can allow conventional spellcasters to save their powers for use against non-Monster foes. Monster Tamers can eventually heal their own Monsters fairly effectively - thus limiting their use for Clerics, however they cannot heal themselves. As a result Monster Tamers are sometimes seen to be both cowardly and ungrateful by their non-Monster companions.}

\ability{Alignment:}{Most Monster Tamers have an extreme alignment, although many are kindly masters, others are vicious and cruel. Monster Tamers tend to shy away from neutrality as their constant battles of will with Monsters generally make them quite accustomed to choosing sides.}

\ability{Races:}{Monster Tamers are usually Human, although there is a sizable number of Halfling Monster Tamers as well. Monster Tamers are usually not well thought of in Elven communities, and many turn to the road. In the depths of the Dwarven mountain halls Monster Tamers are seen as a valuable method of removing dangerous Monsters from the caverns but are also frequently shunned if they are seen training their Monsters. Gnomes are more likely to be scholars of Monsters than to attempt to capture any themselves. Amongst the savage humanoids Monster Tamers are usually laughed at and scorned until they can capture something large enough to frighten compliance out of others.}

\ability{Religion:}{Monster Tamers have no special ties to particular deities. However, powerful Monster Tamers have significant dealings with the outer planes - and many become Clerics. Gods of Elemental or Alignment domains are frequent choices - as are Gods of Plant or Animal.}

\ability{Background:}{Most Monster Tamers dedicate their lives to taming Monsters very early in life. Monster Tamers generally come from single parent homes or are orphans. Many Monster Tamers learn their skills because they love Monsters or are simply competitive - while others see Monsters as a relatively easy path to power and dominate their Monsters in order to fuel their lusts for eternal acquisitiveness. Such Monster Tamers may turn to theft or extortion to attempt to steal the Monsters of other Monster Tamers.}

\ability{Adventures:}{The life of a Monster Tamer naturally leads itself to adventure. Most Monster Tamers spend at least some of their time exploring in order to find and capture new Monsters and hone their skills.}

\ability{Starting Equipment:}{3d4x5 gp (37.5 Gold), one soul prism, one CR 1/2 `monster'}

\ability{Starting Age:}{Monster Tamers often begin their adventuring lives earlier than other classes, when determining starting age for a 1st level Monster Tamer simply choose the age at which that race becomes an `adult'.}

\ability{Hit Die:}{d6}

\ability{Class Skills:}{The Monster Tamer's class skills (and the key ability for each skill) are: Alchemy (Int), Animal Empathy (Cha), Bluff (Cha), Concentration (Con), Craft (Int), Handle Animal (Cha), Heal (Wis), Intimidate (Cha), Knowledge (Arcana - Int), Perform (Cha), Profession (Wis), Ride (Dex), Speak Language (special), Survival (Wis).}

\ability{Skill Points at 1st Level:}{(4 + Int modifier) x 4.}

\ability{Skill Points per level:}{4 + Int Modifier.}

\begin{table}[tbh]
\begin{small}
\begin{tabular}{lp{2cm}p{0.7cm}p{0.7cm}p{0.7cm}l}
Level  &Base Attack  Bonus &Fort Save &Ref Save &Will Save &Special\\
1st  &+0          &+0 &+2  &+0 & Control Monster, Caster Levels, Train Monster, Dread Lore \\
2nd  &+1          &+0 &+3  &+0 & Craft Soul Prison, Heal Monster \\
3rd  &+2          &+1 &+3  &+1 & Subtype Specialization \\
4th  &+3          &+1 &+4  &+1 & Increased Awareness, Double Team \\
5th  &+3          &+1 &+4  &+1 & Speak with Monsters \\
6th  &+4          &+2 &+5  &+2 & Craft Greater Soul Prison \\
7th  &+5          &+2 &+5  &+2 & Type Specialization \\
8th  &+6/+1       &+2 &+6  &+2 & Transfer Control \\
9th  &+6/+1       &+3 &+6  &+3 & Advanced Monster Healing \\
10th &+7/+2       &+3 &+7  &+3 & Craft Leaden Seal \\
11th &+8/+3       &+3 &+7  &+3 & Store Monster, Recall Monster \\
12th &+8/+3       &+4 &+8  &+4 & Second Subtype Specialization \\
13th &+9/+4       &+4 &+8  &+4 & \\
14th &+10/+5      &+4 &+9  &+4 & Second Type Specialization \\
15th &+11/+6/+6   &+5 &+9  &+5 & Craft Master Prison \\
16th &+12/+7/+7   &+5 &+10 &+5 & \\
17th &+12/+7/+7   &+5 &+10 &+5 & Fast Recall Monster \\
18th &+13/+8/+8   &+6 &+11 &+6 & Third Subtype Specialization \\
19th &+14/+9/+9   &+6 &+11 &+6 & Third Type Mastery \\
20th &+15/+10/+10 &+6 &+12 &+6 & Subtype Mastery \\
\end{tabular}
\end{small}
\end{table}

\smallskip\noindent All of the following are class features of the Monster Tamer.

\ability{Weapon and Armor Proficiency:}{Monster Tamers are proficient with all simple weapons, nets, bolas, Orcish Shotputs, Halfling Skiprocks, harpoons, shuriken, and whips. Monster Tamers have proficiency only with light armor. Monster Tamers are considered proficient with using any bludgeoning weapon they are normally proficient with for inflicting subdual damage (thus, they do not duffer a -4 to-hit penalty when attempting to inflict subdual damage with any bludgeoning weapon they are proficient with).}

\ability{Caster Levels:}{Even though Monster Tamers do not gain spells per day or have spell levels - Monster Tamers have many caster level dependant abilities. A Monster Tamer gains a Monster Tamer caster level for every Monster Tamer class level. If a Monster Tamer gains a Prestige Class which adds to Caster levels - she may choose to raise Monster Tamer caster levels instead of other caster levels.}

\ability{"Monster":}{A Monster is any Aberration, Beast, Dragon, Elemental, Magical Beast, Ooze, Outsider, Plant, Shapeshifter, or Vermin which advances by "Hit Dice" rather than "By Character Class." Creatures which can advance by hit dice or character class - like Beholders, are Monsters even if they have character class levels. Deity level creatures, including unique dragon types and unique arch-fiends, are not Monsters regardless of creature type. A Monster Tamer can use the Animal Empathy skill on any Monster as a normal Diplomacy attempt to influence NPC attitudes - regardless of whether or not the Monster Tamer shares a language with the Monster or the intelligence of the Monster.}

\ability{Soul Prisons and Monsters:}{When a Monster is caught with a Soul Prison (see Craft Soul Prison below) it is shrunk down and placed in stasis like in Gloves of Storing (DMG: pages 217-218). While in a Soul Prison, Monsters do not to eat, sleep, breathe, etc. A Monster can be returned to its Soul Prison or removed from its Soul Prison as a standard action by the Monster Tamer which owns it - with a range of 25' + 5' per 2 caster levels. If a Soul Prison with a Monster is traded, given, or sold to another person, ownership of the Monster is also transferred. A Monster heals rapidly while in its Soul Prison. Regular damage is converted to subdual damage at the rate that subdual damage normally heals for the creature. Subdual damage heals at the normal rate while in its Soul Prison.}

\ability{Control Monster (Ex):}{A Monster Tamer can have a number of Monsters in Soul Prisons equal to her Charisma Modifier be "Controlled." A Controlled Monster behaves like a summoned monster when released from its Soul Prison, and is essentially under the control of the Monster Tamer. A Monster Tamer cannot control a Monster whose Challenge Rating is equal to or greater than the Monster Tamer's Caster Level. Remember the rubric for increasing challenge rating based on extra hit dice or class levels to determine if the Monster is controlled. An uncontrolled Monster will act as it sees fit , possibly going on a rampage, running away, or simply sleeping until it is returned to its Soul Prison. Furthermore, Dragon type Monsters are harder to control than other Monsters, and use twice their CR (or their own CR + 4, whichever is less) to determine whether they will obey their Monster Tamer. A Controlled Monster cannot use any Summoning ability to summon uncontrolled Monsters.

\smallskip\noindent More than one controlled Monster can be out of their balls at any one time - but only the first one released behaves like a summoned monster - any subsequent released Monster will act normally, usually standing around and watching events transpire, or sleeping (extreme events can cause them to take direct action at DM's option).

\smallskip\noindent Increases to Charisma only affect the number of Monsters which can be controlled if the increase would affect spells per day. As such, effects like Eagle’s Splendor do not increase the number of controllable Monsters, but a Cloak of Charisma would. Once a Monster Tamer has reached the limit of the number of Monsters which can be controlled, the Monster Tamer cannot control any more until one or more of the controlled Monsters are released from control or killed. Releasing a Monster from control takes about 10 minutes. Control can be reasserted, but only if the Monster Tamer has the ability to control that many Monsters.}

\ability{Losing Monsters:}{A Monster Tamer can, at any time, release their Monsters into the wild. This is a process that takes about 10 minutes during which the Monster Tamer says her goodbyes to the Monster. The Monster is then free to do whatever it wishes, its current intellignce, alignment, and abilities do not inherently change from this release. The Monster's Soul Prison is broken in the process, and is no longer attuned to that Monster. Monsters who were treated especially well or poorly by their Monster Tamer will not forget that treatment and may, at the DM's discretion, act accordingly either immediately or at some time in the future.}

\ability{Death and Monsters:}{Sometimes, Monsters die, this causes a great loss to the Monster Tamer, both emotionally and spiritually. A Monster Tamer whose controlled Monster dies immediately loses 200 XP times the CR of the Monster (zero XP for Monsters below CR 1). A Monster Tamer can make a Will save (DC 15) to halve the XP loss. XP lost in this way are recovered if the Monster is raised from the dead by any means (usually Raise Dead or Resurrection). The XP is recovered if the Monster is Reincarnated, but the new body breaks the Monster Tamer to Monster link and the Monster is no longer controlled, and may no longer be a Monster (depending on its new type).}

\ability{Train Monster (Ex):}{A Monster Tamer can train or evolve their Monster with their Handle Animal skill. As an extraordinary ability, a Monster Tamer need not choose specific animals as trainable and can use Handle Animal on any Monster. Training a Monster takes 8 hours and has a DC of 15 + Monster's (new) CR. The effects available from Training Monster are based on the number of Ranks in Handle Animal the Monster Tamer has:

\listthree
	\item \ability{3 ranks - Learn Trick:}{This is just like teaching to an animal companion (see DMG page 46). Note that some Monsters are intelligent enough so that they are able to perform "tricks" without being specifically taught - and all Monsters are able to learn at least 4 tricks even if their intelligence would not normally be high enough.}
	\item \ability{6 ranks - Grow Monster:}{This causes the Monster to advance 1 Hit Die, if it would not cause the Monster to exceed its advancement limit. This may cause the creature to grow in size category, see the monster description. This may also cause the Monster to become uncontrolled, if this raises its CR to past the maximum CR the Monster Tamer can control. You select what skills, if any, a Monster Tamer gains for its level, and if this would cause a Monster to gain a feat you may select the feat.}
	\item \ability{9 ranks - Evolve Monster:}{This causes the Monster to evolve to a more advanced form. The Monster gains a template of your choice. Note that this may cause the Monster to become uncontrolled, if this raises the CR to past the maximum CR the Monster Tamer can control. The Monster remains a Monster even if its type changes to a type which is not normally a Monster. Monsters who become Dragons in this way are not harder to control than natural dragons are. You select what skills, if any, a Monster gains with its template, and if this would cause a Monster to gain one or more feats you may select the feat(s). At the DM's option, a Monster may be evolved into a similar but more powerful form that is normally represented by a separate entry. For example: a DM might allow a Monster Tamer to evolve her Red Slaad into a Green Slaad, or a Fiendish Horse into a Nightmare. A Monsters of type Beast which is evolved into a different type, gains a permanent one-time "Hard to Control" modifier as if its CR was 1 higher than it actually is.}
	\item \ability{12 ranks - Inspire Monster:}{You may be an especially kind or cruel master to your Monster, giving it a permanent +2 Sacred or Profane bonus to any statistic. You may only give this bonus once to each Monster, and you cannot give different bonuses (Sacred or Profane) to different Monsters.}
\end{list}}

\ability{Dread Lore (Ex):}{A Monster Monster Tamer accumulates significant knowledge about the Monsters that they face. The amount of knowledge a Monster Monster Tamer has on an encountered wild monster is linked to the Monster Tamer's Knowledge Arcana or Survival skill - whichever is higher. The abilities granted depend upon how many ranks the Monster Tamer has in the relevant skill:

\listthree
	\item \ability{3 ranks - Identify Monster:}{A Monster Tamer can automatically identify the name, type, and subtype of any Monster encountered.}
	\item \ability{6 ranks - Full Monster Entry:}{A Monster Tamer's player can open the Monster Manual (or other relevant source material) to the appropriate page and read the Monster's entry. If the Monster Tamer's player chooses, she may read the relatively uninformative descriptive text at the beginning of the entry to other players out loud. In addition, a Monster Tamer may note whether a Monster encounterred in the wild has extra advancement hit dice and/or class levels - though not necessarily what kind or how many.}
	\item \ability{12 ranks - Fully Identify Monster:}{The Monster Tamer is able to instantly identify any Monster's advancement hit dice and class level (if any).}
\end{list}}

\ability{Craft Soul Prison (Sp):}{A 2nd level Monster Tamer can craft Soul Prisons. A Soul Prison costs 100 GP and 8 XP to make. Alternately, it costs 200 GP to buy one if it is available. A Soul Prison acts as a thrown weapon, which is used as a ranged touch attack with a range increment of 15'. Using a Soul Prison is considered to be using a spell like ability. If a Soul Prison thrown by a Monster Tamer hits a Monster it inflicts 1 point of subdual damage per caster level - if the Monster is unconscious after being hit by the Soul Prison it is sucked into the Soul Prison and now belongs to the Monster Tamer who threw the Soul Prison - the Soul Prison is now sitting in a square formerly occupied by the captured Monster. If a Soul Prison hits a Monster it is attuned to that Monster and cannot be used on any other Monster - ever.}

\ability{Heal Monster (Sp):}{A Monster Tamer may attempt to accelerate the healing of a Monster in its Soul Prison. By spending a fullround action, a Monster Tamer can attempt a Heal Check (DC 15) to either convert all regular damage suffered by the Monster into subdual damage, or to confer the benefits of 1 day of rest to the Monster (2 Hit Points per hit die, 1 day worth of repaired Ability damage, the recovery of any limited uses/day abilities, and the healing of all subdual damage). This ability may be used on each Monster 3 plus the Monster Tamer's Wisdom bonus (if positive) times per day.}

\ability{Subtype Specialization (Ex):}{A Monster Tamer can choose a subtype which is her specialty. A Monster Tamer gains a +1 bonus on all Bluff, Animal Empathy, Handle Animal, Knowledge, Listen, Sense Motive, Spot, and Survival, checks when using these skills on or about such creatures for every 3 caster levels she has. A Monster Tamer can choose a second Subtype to be equally proficient with at 12th level, and a third at 18th. A Monster Tamer can Control one extra Monster which must be of a subtype that she specializes in. Subtypes include: Air, Aquatic, Chaotic, Cold, Earth, Electricity, Evil, Fire, Good, Lawful, Reptilian, and Water.}

\ability{Increased Awareness (Ex):}{At 4th level and above, a Monster Tamer's Monster become more intelligent and aware. After the Monster Tamer has owned her Monster for at least 1 week, its intelligence changes to the Monster Tamers ranks in Handle Animal if that is more than its normal intelligence.

\smallskip\noindent In addition, a Monster Tamer can make her Monster gradually see things her way - a Monster's alignment shifts one degree towards the Monster Tamer's each week if she can succeed in an Animal Empathy check at a DC of (10 + the Monster's CR). The DM decides whether it moves Law/Chaos or Good/Evil first depending upon circumstances. So if a Lawful Good Monster Tamer captured an Imp (lawful evil Monster), the Imp could become Lawful Neutral after one week, and could be Lawful Good after 2 weeks. Monster subtypes are unaffected - so an Evil Monster such as an Efreet would stay subtype [Evil] even if it subsequently became of Good alignment.}

\ability{Double Team:}{Upon reaching 4th level, the Monster Tamer is able to control two Monsters out of their balls simultaneously, even in battle. This ability only functions so long as both Monsters are more than 2 CR less than the Monster Tamer's caster level. For example, a 5th level Monster Tamer could command a single CR 4 Monster in battle or two CR 2 Monsters, but could not command a CR 1 Monster and a CR 3 Monster simultaneously.}

\ability{Speak with Monsters (Ex):}{At fifth level a Monster Tamer has Tongues - always on, which only effects Monsters. Even though a Gorgon's speech still sounds like "Groarrough" it is perfectly intelligible to the Monster Tamer. Further, the Monster Tamer's speech is understandable by Monster even if they do not normally have a language - even Oozes and other Monsters not normally capable of communicating at all.}

\ability{Craft Greater Soul Prison (Sp):}{A Monster Tamer can craft a Greater Soul Prison, which is a more powerful form of Soul Prison. It behaves just like a Soul Prison except that it costs 1000 Gold and 80 XP to craft - and inflicts d4 subdual damage per caster level.}

\ability{Type Specialization:}{At 7th level, you can choose a single creature type to gain the same skill bonuses as your subtype specialization with a creature type instead. You are not limited to normal Monster types. You may choose a second type to Specialize in at 14th level, and a third at 19th. You may have an additional controlled Monster, which must be of a type you are specialized in. Type and Subtype Specialization bonuses are cumulative.}

\ability{Transfer Control:}{At 8th level a Monster Tamer can choose to change which Monster she controls, up to her regular limit of controlled Monsters. All newly controlled Monsters must be in Soul Prisons possessed and owned by the Monster Tamer. Transfer Control is a full-round action. Normally transferring control takes 10 minutes per Monster so transferred.}

\ability{Advanced Monster Healing (Sp):}{A Monster Tamer can, at 9th level, use Heal as a Spell like ability a number of times a day equal to her wisdom modifier, with a minimum of once a day. A Monster Tamer can only Heal Monsters she controls, but can heal them whether they are in their Soul Prisons or not.}

\ability{Craft Leaden Seal (Sp):}{A Monster Tamer can craft a Leaden Seal. A Leaden Seal is a much more powerful form of Soul Prison. It costs 5000 GP and 400 XP to make. When used, it inflicts d8 points of subdual damage per caster level.}

\ability{Store Monster (Sp):}{Starting at 11th level, as a move equivalent action, a Monster Tamer can send a Soul Prison with a Monster in it to a completely safe extra dimensional space. A Soul Prison must be within Close range (25 feet + 5 feet per 2 caster levels) to be stored. Store Monster cannot be combined with a normal move. Store Monster is a spell-like ability.}

\ability{Recall Monster (Sp):}{Starting at 11th level, as a fullround action, a Monster Tamer can transport a Stored Soul Prison from her extra dimensional space to her hand.}

\ability{Craft Master Prison (Sp):}{A Master Prison is the ultimate expression of the Monster Hunter - it costs a hefty 10000 GP and 800 XP to manufacture, and subdues the first Monster it hits, if that Monster does not have more than 2 hit dice for every caster level of the Monster Tamer who threw it. If a Monster is too strong to be captured automatically it may yet succumb as it still suffers d12 subdual damage per caster level.}

\ability{Fast Recall Monster (Sp):}{As Recall Monster, but Recalling Monster is a free action.}

\ability{Subtype Mastery:}{The Monster Tamer chooses one subtype that she is already specialized in to Master. All her Leaden Seals function like Master Prisons against Monsters of that subtype, there is no limit to the CR of Monsters of that subtype that she can control - and she can control one extra Monster of that subtype, in addition to her bonus controlled Monsters from type and subtype specialization.}

\ability{Monster Tamers and Multiclassing:}{Monster Tamers rarely multiclass, however if they multiclass into another spellcasting class and have access to domains, the Spellcaster levels stack for purposes of controlling Monster of a type or subtype sharing of those domains. So a Monster Tamer 6/ Cleric 5 with the domains of Evil and Fire would control Monsters as Caster level 6, but would control Evil or Fire Monsters as a Caster Level 11 Monster Tamer.}

\chapter{Mechanics with Prestige}

\classname{Golem-Knight of Mechanus} \label{comm:prestige:golemknight}
\vspace*{-8pt}
\quot{``GIGA-KNIGHT! GIGA-KNIGHT!''}

\ability{Requirements:}{}
\listprereq
\itemability{BAB:}{+7}
\itemability{Skills:}{Knowledge (engineering) 10 ranks}
\itemability{Special:}{Must have spent time learning the art of construction in Mechanus. A Knight who is of an appropriate Order need not possess the ranks in Knowledge (engineering) or spend time learning in Mechanus.}
\end{list}\vspace*{8pt}


\ability{Hit Die:}{d12}

\ability{Class Skills:}{Whatever you want, since Koumei doesn't believe in cross-class skills.}

\ability{Skill Points at Each Level:}{6 + Int modifier.}

\begin{table}[tbh]
\begin{small}
\begin{tabular}{lp{1.9cm}p{0.7cm}p{0.7cm}p{0.7cm}l}
Level  &Base Attack  Bonus &Fort Save &Ref Save &Will Save &Special\\
1st &+1 &+2 &+0 &+0 &Craft Mechanus Armor, Amplifiers\\
2nd &+2 &+3 &+0 &+0 &Mechanus Mount, Rocket Lance\\
3rd &+3 &+3 &+1 &+1 &Artificer's Secret Weapon, Fuel Lines\\
4th &+4 &+4 &+1 &+1 &Craft Grenades, Boomstick\\
5th &+5 &+4 &+1 &+1 &Golem Body, Craft Servo-Arm\\
\end{tabular}
\end{small}
\end{table}

\smallskip\noindent All of the following are Class Features of the Golem Knight prestige class.

\ability{Weapon and Armor Proficiency:}{The Golem Knight is considered proficient with any weapon she personally crafts.}

\ability{Craft Mechanus armor:}{The Golem Knight can, given about a day and enough scrap materials, construct a suit of Mechanus Armor. This is extra heavy Adamantine Full Plate that has an Enhancement bonus equal to one quarter of the Knight's character level. However, it only gains this effect when worn by the Knight who crafted it.

Additionally, the wearer gains all Construct traits (but ability scores remain unchanged), along with the ability to add the armor's Enhancement bonus to all Break attempts and damage rolls.}

\ability{Amplifiers:}{The Golem Knight has a booming voice that can carry on for miles. Additionally, they may cast Command at will as an Extraordinary ability.}

\ability{Mechanus Mount:}{The Golem Knight may, given enough scrap material and about a day, craft a mighty steed that looks like a warhorse with too many layers of plate barding. This counts as a Cauchemar Nightmare of the Construct type (ability scores are unchanged) and no actual alignment. It loses its Spell-like abilities (but retains the smoke, breath weapon and fire damage) and Subtypes, and has hit dice equal to the Knight's character level.

It possesses a Natural armor bonus of +0 but an armor bonus equal to its hit dice and an Enhancement bonus to Strength and AC equal to the class level of the Golem Knight. It is completely loyal to the Golem Knight, but when not being ridden, it stands motionless, unable to act.

It can be crafted to any size that the Golem Knight could ride, and the Knight can see through its Smoke.}

\ability{Rocket Lance:}{The Golem Knight may, with an hour of work, modify a lance to be fired as though from a cannon. It can be used as a ranged weapon with range increments of 15' and counts as charging when fired. It still adds the Strength modifier of the Knight to damage, but does not multiply the Strength by one and a half.}

\ability{Artificer's Secret Weapon:}{With a day of hard work, the Golem Knight can craft special weapons. They are treated as though magical (with a Greater Magic Weapon effect), though may trade the bonuses out for special abilities. However, they are in no way magical and thus function normally in an AMF and cannot be disjoined. They still bypass DR/magic and can strike incorporeal targets.}

\ability{Fuel Lines:}{Special fuel pipes can be built into the Mechanus armor with a whole hour of work. This allows the Golem Knight to drink any potion they have as a Swift Action.}

\ability{Craft Grenades:}{With an hour and some random materials, the Golem Knight may craft 1d4+1 special grenades. Sadly, they are unstable and will be unusable after 24 hours. Until then, they may be thrown as grenade weapons (duh), exploding in a 30' radius. The save DC is 10 + half the Knight's HD + their Int modifier. The Golem Knight may choose the effects upon creation:

\listone
    \item Incendiary: all in the area take 1d6 Fire damage per character level of the Golem Knight. A successful Ref save halves this damage, but those who fail the save catch fire.
    \item Frost Cloud: all in the area take 1d6 Cold damage per character level of the Golem Knight and are Slowed for 1 round. A successful Fort save halves the damage and negates the Slow effect.
    \item Flash Bang: all in the area are Blinded and Deafened for 1d4 rounds on a failed Ref save.
    \item Concussion: all in the area are knocked prone automatically, and Dazed for 1 round on a failed Fort save.
    \item Electro Pulse: all in the area take 1d6 Electricity damage per character level of the Golem Knight. A successful Ref save halves this, but those who fail the save count as Entangled for 1 round as their limbs spasm.
    \item Sleep Gas: all in the area become drowsy for 2 rounds on a failed Fort save. Drowsy characters count as Fatigued and take a -6 penalty on saves against Sleep effects.
    \item Corrosive Gas: all in the area take 1d6 Acid damage per character level of the Golem Knight. A Ref save halves this. Those who fail the save take damage equal to the Golem Knight's character level on the following round.
    \item Disruption Pulse: all Constructs and Undead in the area take 1d8 untyped damage per character level of the Golem Knight. A Fort save halves this, and they are not immune to this effect.
\end{list}
\vspace{6pt}

Grenades are not magical.}

\ability{Boomstick:}{With half a day of work, the Golem Knight can craft a double-barrelled shotgun of sorts. Firing this two-handed weapon is an Attack action and it takes a Partial Action to reload (Swift with Rapid Reload or Sleight of Hand 10 ranks, Free with both of the above), but can fire twice before needing to be reloaded. 1d6+1 rounds can be crafted with an hour of work.

\listone
    \item Solid Shot: 40' range increments, 3d6 Bludgeoning damage, critical 20/x4.
    \item Scatter Shot: 40' cone, 3d6 Bludgeoning damage, Ref half (same DC as grenades). Those who fail are knocked Prone.
    \item Incendiary: 40' cone, 2d6 Fire damage, Ref half. Those who fail catch fire.
    \item Flash-Bang Powder: 40' cone, all in the area are Blinded and Deafened for 1 round, Ref negates.
    \item Explosive Shells: 20' range increments, 1d10 bludgeoning damage and 3d6 Fire damage, sets the target on fire. Crit 20/x3 (only multiplies the Bludgeoning damage).
    \item Meltdown Shells: 20' range increments, 1d10 bludgeoning damage and 3d6 Acid damage, with 2d6 Acid damage the following round. Crit 20/x3 (only multiplies the Bludgeoning damage).
    \item Executioner Seeker Rounds: creates a non-magical Magic Missile (try not to think too hard about it) effect with a CL equal to the class level of the Golem Knight.
\end{list}
\vspace{6pt}

This weapon is not magical.}

\ability{Golem Body:}{With a week of work and enough materials, the Golem Knight may construct their very own actual golem body. This can be a giant chunk of baked clay that looks like a few rocks stacked together, or it could be an intricate suit of many plates of armor. Whatever. It takes a whole hour to don this "body", or to escape it, but when worn, the following benefits are gained, in addition to those that would be gained if the Knight was wearing their Mechanus armor (they can't wear armor while wearing this):
\newline Size: 1 category larger (you may wish to construct a bigger horse)
\newline Ability Scores: +8 Strength, -4 Dexterity, +4 Constitution
\newline Speed: Remains the same
\newline Reach: Extends by 5'
\newline Armor: As the Mechanus armor +5
\newline Attacks: May wield weapons, or gains 2 slams (2d6+Str for a Large creature). \newline May also use a Bite attack (3d6+\half Str for a Large creature).
\newline Special:
\listone
    \item Trample (4d6 for a Large creature)
    \item Improved Grab
    \item Spell Resistance equal to 10 + hit dice
    \item Furnace Blast (Ex): 50' breath weapon deals 2d6 Fire damage per hit die, Ref half (Con-based). Can be used once per minute, and those who fail the save catch fire. May instead be focused on a character hit by a Bite attack, as a Swift action. This hits automatically as though they had failed the save, but no-one else is affected.
\end{list}
\smallskip

This ``armor'' is not magical.}

\ability{Craft Servo Arm:}{The Golem Knight may, with a day's work, craft an additional arm to connect to a mount, Mechanus armor, or the Golem Body. This arm reaches out to 10' further than the subject normally could, and can manipulate objects like a Telekinesis spell (CL = HD), or may be used to attack foes like a Bigby's Crushing Fist spell (CL = HD). It also has a form of blow-torch and saw added, which helps it assist the Knight in creating items (halving the time), but also is very unfortunate for those grappled by the arm. They take an additional 2d6 Adamantine Slashing damage and 5d6 Fire damage per round.
\smallskip
The Servo Arm is also not magical.}


\section{High Adventure in the Lower Planes}

The Lower Planes are infinite in size, and this is often taken as meaning that they are somehow filled with infinite power. This is essentially completely false. Remember that the Primes are essentially infinite in scope as well, and while there are ancient dragons and even Xixicals\ldots \textit{somewhere}, the fact is that you could adventure your whole life and never ever meet one. The world is mostly filled with forests, and mountains, and little river valleys, and most of the time the villains you encounter are going to be rabid dire weasels and bugbear junkies who will try to resell your shoes for a hit of mordayn vapor. Gehena is actually just like that, except that instead of you never seeing powerful dragons in your day to day life, you never see Arcanoloths. The bad guys you encounter may well be a \textit{fiendish} dire weasel and a bugbear junkie \textit{petitioner}, but the essential threat level is pretty much the same.

Low level adventuring, thus, is extremely plausible in the lower planes. It's not advisable for low level characters to go running around Tiamat's lair or anything, but the fact that the Elder Brain Pool is somewhere in the Underdark hasn't stopped \textit{any} low level campaigns from tunnel crawling as far as I can recall. What follows is some wilderness adventure seeds from the lower planes for low (1-5), medium (6-10), and high (11-15) level. Players who want to adventure at near epic levels (16+) don't even need adventure seeds of this sort because they actually can just take on The Dark Eight or whatever. For whatever reason, lots of ink has been spilled on near epic adventuring in the lower planes, and I have every confidence in a decent DM's ability to throw a Balor at a party and make a rollicking and dangerous encounter.


\subsection{High Adventure in\ldots Acheron!}

The first thing to realize about Acheron is that it really isn't a bad place to be. It's not even Evilly Aligned, so even campaigns using The Face of Horror have no reason to play up the terror of being here -- the sand of Acheron is not Evil. But it \textbf{is} made out of steel. Characters who are going to go adventuring will do so in Avalas, because that's the part of the plane that doesn't \textit{turn you to stone}.

\subsubsection{Campaign Seed: The Tunnel to Pandemonium}

Here's a little piece of D\&D history for you -- In AD\&D, Orcs were \textit{Lawful Evil}, so the Orcish pantheon lives in Acheron to war eternally with the Goblin pantheon \textit{even though Orcs are Chaotic now}. That means that the cube of Nishrek, where Gruumsh calls his most favored and despised for Gruumshian Justice when they have passed on -- is itself a bubble of Pandemonium found far from its place in the Wheel. There are, therefore, numerous portals to Pandemonium all over Clangor and Nishrek, so characters who wish to fight Orcs and Goblins in the lower planes can do so to an unlimited degree by portal hopping through the Pandemonium and Acheron layers. As an agent of Gruumsh or Maglubiet, characters can fight their way through savage humanoids, savage humanoid armies, savage humanoids with fiendish allies, savage humanoid war machines, and even powerful outsiders aligned with savage humanoids \textit{well into epic}. You can also use this rivalry as the backdrop for any of a number of ''find the artifact before it falls into seriously the wrong hands'' type adventures, with the characters switching sides repeatedly based on who seems to have the artifact now.

\subsubsection{Campaign Seed: You're in the Army Now}

Cities and castles populate the lands of Acheron without number, and all of them are on a war footing at all times. Characters can travel generally without molestation throughout this area and conduct a fairly profitable bit of trading and scavenging if they do things right. But if they do things wrong, they may end up drafted into some local militia or imperial army. Characters can have substantial numbers of adventures as part of a military force, or they can attempt to resist being drafted by any of a number of means. Unfortunately, the laws of Acheron being what they are, once the characters impress their will by force of personality or arms enough to avoid the draft, they'll find themselves as a \textit{side} -- which means that they'll be treated as a hostile army all themselves by other forces. At that point they can try to stick it out alone, or try to get some help, of course almost every empire in Acheron started the same way. So the players can progress smoothly from the ''chased by bad guys'' scenarios to the ''forge an empire in blood'' scenarios to the ''marry the princess, design your castle'' scenarios.

\subsubsection{Ten Low Level Adventures in Acheron}

You pull into the hamlet's bar, and see what they have to offer a stranger. It isn't good. After a brief set of questions to make sure you aren't going to burn the place down, the bartender tells you\ldots

\listone
	\item The town is infested with fiendish rats. Beer just isn't safe until their gone, sorry.
	\item A rival faction as poisoned the well, and someone needs to find a new source of water.
	\item Brigands are holding the pass. I hear one of them is an Ogre.
	\item The man you are looking for\ldots he was taken away by the Scarthian Army.
	\item That signet ring is part of King Imag's royal accoutrements. If someone could get all of them together\ldots it could spell big changes for the County of Yevekh.
	\item Orcs have come through the tunnel, their leader has a silver sword and noone dares to stand against him.
	\item After the Citadel of Zor fell, bodies were piled as high as your arm pit. I hear someone is making them all into zombies now, it's a shame really.
	\item I'd love to give you change, but after Sir Garreth set the taxes to 100\%, I'm afraid I have no coins to give you.
	\item In this town, either you're for Sheriff Braxton, or you're dead. This town, we like to have choices.
	\item It's free drinks here if you can get Clarrissa the hobgoblin matron to allow her daughters to marry.
\end{list}

\subsubsection{Ten Mid Level Adventures in Acheron}

An emissary of hoary Surog, the ice count, contacts you. He has (the ring, the antidote, the code) you need, and he'll give it you, but first\ldots

\listone
	\item One of his lieutenants has betrayed him; since you are random strangers, he can trust you to find out which one.
	\item His daughter has run off with the blue falcon, that accursed do-gooder. Bring her home, do with him as you wish.
	\item His daughter is the blue falcon. Stop her, but don't kill her.
	\item His daughter is the blue falcon, and Surog's rival, Cardinal Valgos, has put-her-in-a-death-trap. Rescue her, without letting on that Surog knows her identity.
	\item Cardinal Valgos has found some route to smuggle forces into Yevekh. Find how they're getting in.
	\item Cardinal Valgos is planning an attack, and Surog is not prepared. Infiltrate his mercenary forces and cause as much delay as possible.
	\item Cardinal Valgos has placed Surog in some kind of suspended animation! You have to lift the curse before one of Surog's underlings makes a play for power.
	\item A blue crossbow bolt with a head shaped like a stylized raptor strikes the emissary from nowhere, killing him before he can deliver your mission! Who is trying to stop you, and why?
	\item Cardinal Valgos has Imag's heir, or so he claims. Prove the heir is false, or steal him away.
	\item Cardinal Valgos has tricked the fox of the mountains, Dagipert, into allying with him. Break up this alliance one way or another.
\end{list}

\subsubsection{Ten High Level Adventures in Acheron}

You stand at the front of your army, triumphant over every foe the Lichking has sent against you, over the next hill you see\ldots

\listone
	\item The Lichking's vampire sister, all alone with a white flag.
	\item A pile of bodies impaled to the top of a 200 foot metal rod.
	\item A stampede of zombie elephants.
	\item A chasms cleaved into the side of the cube burbling with lava.
	\item A portal opening up upon an army of orcs in Pandemonium, easily the equal of your own.
	\item A huge pile of what appears to be gold.
	\item A huge pile of what appears to be skulls on fire.
	\item A wyvern bearing a message in its claws.
	\item The daughter of King Zormmund, tied to an elder earth elemental.
	\item Your grand vizier, who has apparently betrayed you again.
\end{list}


\subsection{High Adventure in\ldots Pandemonium!}

Pandemonium is a victim of the terrible confusion that permeates Law and Chaos in D\&D literature, and its inhabitants are portrayed in a number of improbable lights. Pandemonium is not an Evil plane, but it's fairly wicked and it's inherently Chaotic. How it and the people who live there appear in your game is entirely dependent upon how your game ends up handling Chaos in general. Pandemonium might be extremely disorganized, or inherently deceitful, or starkly unhelpful, or simply a lawless wilderness. But what it almost certainly \textit{isn't} is a source of low comedy where people do ''whacky stuff'' because they are so ''crazy''. That's the kind of thing that makes us cry.

Pandemonium can be a source of classic D\&D adventure at its finest -- the towns of Pandemonium are located right next to twisting tunnels through the stone and loud noises sound continuously through the warrens. So characters can go right from the town to the dungeon crawl without any explanation or overland travel, and those dungeon encounters are inherently episodic because nothing can hear your combats.

Pandemonium is dark and loud, and filled with confused people. At its best, Pandemonium is basically a huge rave. At its worst, Pandemonium is a huge rave. Like every part of the D\&D afterlife, Pandemonium can be a punishment or a reward. And like Acheron, this place isn't inherently Evil. So even if you are using The Face of Horror, the Eternal Rave isn't that bad of a place.

\subsubsection{Campaign Seed: Life in the Big City}

Welcome to The Madhouse. It's one of the largest planar metropolises in D\&D, and unlike places like the City of Brass or Sigil, it really \textit{doesn't} have some group of powerful outsiders ruling it with an iron fist. In fact, The Madhouse has no rulership of any kind. The place is dark, and loud, and the only light comes from naked women with glow sticks. Essentially, you can get away with pretty much anything without interference from opponents significantly outside your level range. You can keep having urban adventures continuously from 1st to 20th without ever getting seriously derailed by concerns of DM self-insertion characters coming over to knock over your house of cards. At the same time, there really \textit{are} Balors in this complex, so if you actually want to \textit{seek out} higher-powered enemies, that's doable.

\subsubsection{Campaign Seed: The Largest Dungeon}

Tunnels crisscross Pandemonium all over the place, and they are completely stable because the way gravity works there actually can't be a cave-in. But the place is dark and windy, and filed with tunnels that move around for no reason. The caverns are filled with monsters, traps, and treasure. It's all there, from shambling zombies to ninja temples, the low level areas cross seamlessly into the higher level ones. Oddly, this is the only place in the entire multiverse of D\&D where the old Gygaxian standby of having deeper and deeper levels of the dungeon filled with nastier and nastier monsters and traps actually makes sense. There's a town nearby, and the map doesn't have to make any sense at all. If you're looking for Nethack style adventuring, Pandemonium delivers.

\subsubsection{Ten Low Level Adventures in Pandemonium}

You lean over the counter to the waitress, not because she's so beautiful, but because you can barely hear her over the din. Honest. You're pretty sure she said\ldots

\listone
	\item WE DON'T SERVE YOUR KIND HERE. THE MILLER ONLY SENDS US BASALT FLOUR NOW.
	\item WE GOT AN ORDER OF APRICOTS IN THIS WEEK, THE CRAZ NAKED MAN CLAIMS TO MAKE IT HIMSELF.
	\item THE TUNNELS ON THE WEST SIDE, NO ONE COMES BACK FROM THOSE. NOT EVEN THOSE NICE MEN FROM LAST MONTH WITH ALL THE WEAPONS.
	\item IF KELLIGAN SEES YOU LEANING ON ME LIKE THIS, HE'LL KILL US BOTH.
	\item THERE WAS A MAN LOOKING FOR YOU. HE SAID HE OWED YOU MONEY.
	\item DO I KNOW YOU? AFTER THE WATER TURNED BLACK, I'VE HAD TO ASK EVERYONE THAT.
	\item I HAVE THE CURSE. YOU SHOULDN'T STAND SO CLOSE.
	\item I CAN'T FEEL MY MIND. STOP TAUNTING ME!
	\item THE BEER IS FREE TODAY. IT'S A LONG STORY.
	\item DON'T UNCOVER THOSE LIGHTS! THERE'S A WIGHT IN THE BUILDING.
\end{list}

\subsubsection{Ten Mid Level Adventures in Pandemonium}

You've found the sage you were looking for, but it looks like he's dead. His corpse is torn apart and lying on a heap against the part of the floor that's the ceiling to you. Droplets of congealing blood rotate slowly in the la grange points between ceiling and floor. He's got a piece of parchment in his cold hands, and it says\ldots

\listone
	\item wights have found me kill me kill me kill me
	\item I think this corpse will fool the howlers. At least for a while. If you wanted some water it's become more dangerous.
	\item NWNENWWS
	\item This man is an example. If Hruggek's Ninja Temple requests taxes, pay them.
	\item It's written in an old Orcish tongue. You'll have to find an Orc slain on the Prime at least a thousand years ago.
	\item The man's name is Gregor.
	\item Orcs! How I hate them! Their scimitars open the way!
	\item This is a ruse. The sage has escaped.
	\item Go back. Erythnul is not to be mocked.
	\item Itchy. Tasty.
\end{list}

\subsubsection{Ten High Level Adventures in Pandemonium}

The gates of the building have been torn asunder, as the characters run in, it seems that they're too late because\ldots

\listone
	\item Wights swarm over the insides, covering every piece of furniture with writhing limbs and moaning incessantly.
	\item Neogi great old masters hang from the ceiling, affixed by strands of hardened mucous.
	\item The pews stand empty as dust sweeps through the ancient church propelled by powerful winds.
	\item Hruggekian throwing stars are imbedded in virtually every wooden surface.
	\item A gaping planar rift hovers in the middle of the room, the winds of Pandemonium hurtling small objects into the void.
	\item The red dragon is already here, the hobgoblin princess is in his grasp.
	\item Black fires lick the insides of the room, the tomes are most likely destroyed!
	\item A tremendous serpent creeps over the tattered carpet.
	\item The winds howl even louder in here. Or maybe\ldots there are air elementals!
	\item A friendly and purring kitten is tossed back and forth by the terrible winds.
\end{list}


\subsection{High Adventure in\ldots Carceri!}

Point of fact: being in Carceri sucks. It's hard to leave, and it's an unpleasant place to be. That's the whole point. But believe it or not, those who please Nerull sufficiently are \textit{rewarded} with an eternity in Carceri. Now some of these people are just sadists -- creatures who enjoy the suffering of others so much that being able to assist in the degradation of others is payment and more for having to live in a hell dimension in the Prison Plane. But for others, life in Carceri is just genuinely pretty good. Some of these prison dimensions are minimum security white people jail -- there's a golf course and your ''guards'' are attractive women. It's still a prison of course, but if someone doesn't \textit{want} to leave, are they really a prisoner?

Anywhere you go in Carceri, it's all Evil, and people normally only go here if they are themselves Evil. That means that the people who are being punished here are being punished for \textit{failure}, not wickedness. The most successfully wicked individuals actually are rewarded here. Carceri can be a great place to introduce horrific elements into your story because by its nature anything that happens in Carceri, \textit{stays} in Carceri. Horrifying and depraved elements you introduce in a Carceri adventures don't have to apply to any subsequent adventures if you don't want them to.

\subsubsection{Campaign Seed: A Ring of Keys}

Carceri is a never ending parade of pocket dimensions filled with punishments and rewards that are both cruel and ironic. Travel between these cells is almost impossible, but there are ways. Most notably, there are maps that can tell you a secret path to get from one prison to the next; and there are adjustable rings that can transport a character directly from one prison to another depending upon how it is adjusted. Either can make for unlimited hours of enjoyment as players hop from one piece of episodic turmoil to the next. The maps work just like the map from \underline{Time Bandits}, and the rings work just like the devices from \underline{Sliders}. Really. Furthermore, those objects are authorized personnel \textit{only}, so if the players have one they are going to be hunted by Demodands with a new wacky scheme to catch them every adventure.

\subsubsection{Campaign Seed: Escape from Tartarus}

Just because you have been placed in a prison plane doesn't mean you deserved this punishment, or even that you committed a crime. The plane itself will punish impersonally, hiding its portals behind elaborate stages designed to elicit suffering.

Fight your way our of Tartarus, and no prison in any plane will every hold you\ldots

\subsubsection{Ten Low Level Adventures in Carceri}

You pass through the portal and find yourself in a new prison dimension. This one is\ldots

\listone
	\item Filled with thick, thorny foliage. Also it smells like boar and the thorns splinter and get into your armor.
	\item A town where the streets are filled with fighting.
	\item An expansive desert. Vultures fly overhead, but the scorpions seem unwilling to wait for you to die.
	\item A foul sewer. The water is waste deep. At least, you hope it's water.
	\item A scrubland with rusted iron spikes jutting out of the ground. Cages filled with starving madmen top some of the spikes, while other cages have long since fallen to the ground.
	\item A banquet hall stacked with delicious looking food. Haggard goblins look at the food with longing, but nothing seems to stand between them\ldots
	\item A windswept glacier. Far beneath you, there is a shadow in the ice. Far in the distance, a wolf howls.
	\item A stark stone room, where light filters oddly through a great number of spider webs and a dusty stained glass window.
	\item An earthy sinkhole. Worms poke through the topsoil everywhere around you, their eyeless heads wriggling like mad.
	\item A garden maze under an overcast sky. Fantastic shapes are cut into the hedges, and some ever seem to watch you.
\end{list}

\subsubsection{Ten Mid Level Adventures in Carceri}

If you could figure out the secret of this prison, you could escape\ldots

\listone
	\item The labyrinth seems to have four spatial dimensions\ldots
	\item The land shakes with earthquakes, but they still try to build houses.
	\item That eagle keeps eating that guy's entrails\ldots hey wait, I have entrails\ldots
	\item Why does that sanitarium seem to be inside-out?
	\item Why does everyone here wear a mask?
	\item Criminals in this put themselves into prison cells?
	\item The ghosts don't die when we kill them, and if we can't kill them we can't leave this building.
	\item It looks like a brothel, but who are the petitioners? The clients or the girls?
	\item The portal has a gold lock on it, and I was sure I saw a glint of gold in one of those oozes.
	\item An endless desert of white sand\ldots Or is it bone dust? 
\end{list}

\subsubsection{Ten High Level Adventures in Carceri}

If you just had it, then you'd be free\ldots

\listone
	\item A ship of chaos passes this way every day at the same time. If I could only make it notice me\ldots
	\item I almost have enough money to bribe the demodands into releasing me.
	\item That demon is a master of planar magic, and its said that his enemies get tossed to other planes.
	\item The fiends involved in the Blood War come from other planes. If I had an army large enough to impress them, they might show me a way out.
	\item If I could remember my home, I could just cast a spell and go home.
	\item The sage knows a way out, but he's so crazy that he'll only tell the secret to someone he considers a peer. What do I have to learn to do that?
	\item I can't believe that she's here. Do you think she'll forgive me?
	\item That war machine that looks like a bug the size of a mountain\ldots I hear its powered by a portal to the Astral Plane.
	\item I could open this portal, but I need the Blessing of Nerull.
	\item A wizard has been traveling Carceri for rare components, and it's said that he has access to plane-hopping effects.
\end{list}


\subsection{High Adventure in\ldots Hell!}

The Infernal Realm of Baator is essentially 9 infinitely large regions that happen to have a big pit that acts as a portal to the other 8 somewhere in them. So while the gods (and official publications) spend a lot of time worrying about that big pit in the middle, the fact is that the vast majority \textit{of Baatorian residents} don't even know it exists. Near epic play will spend an inordinate amount of time worrying about the goings-on around The Pit, and send the heroes off to go siege the fortresses around the ledge and such, but for the rest of your character's life the Nine Hells of Baator are just some inhospitable terrain filled with level-appropriate monsters.

\subsubsection{Campaign Seed: A Kafkaesque Nightmare}

Baator is home to one of the multiverse's most pervasive, efficient, and \textit{evil} bureaucracies. They don't lose your documents, they don't forget to mail things to you when they said they were going to, they simply have a set of rules that is at once awe-inspiringly complex and actually \textit{designed} to cause suffering to those who need to use its services. A campaign set around the backdrop of filling out forms sounds about as entertaining as doing your taxes in Hell, but there's ample opportunity for comedy, horror, and adventure in such a scenario, as well as ample prospect for character growth. The action starts when the characters need to change their registered employment, or want to protest their home getting knocked over to build a throughway, or perform some other completely mundane bureaucratic task. Unfortunately, the form they need to begin this process is clearly on display downstairs in the room marked ''Beware of Leopard''.

Surfing bureaucracy in Baator is about the only place where that makes for exciting D\&D adventures. The challenges to be overcome are social, mental, physical, and magical and efficient bureaucrats will tell you \textit{exactly what you need to do} to get things accomplished. This isn't like a Kafkaesque Nightmare on Earth, where you'll get stonewalled or your papers will just get lost, this is completely efficient and functional -- but designed by super geniuses to make your character uncomfortable. At lower levels there's a fiendish leopard in the room with the papers you need. At higher levels there's a golem that's supposed to stop people from entering the office where you need to convince a Gelugon to stamp your form. As the characters push their way to the top, they will find themselves in the position of being able to create their own red tape\ldots

On a side note: I just want to point out that my spell-checker recognizes ''Kafkaesque'' as a word. Sweet.

\subsubsection{Campaign Seed: Law of the West}

The great cities of Baator are infinitely far away from some of the nether regions of the plane. But the Law (and the Evil) still needs to be maintained. If you get far enough out into the boonies, Pit Fiends and the like just can't be bothered to show up and solve problems. So when Chaos (or Good) comes in to assault a frontier town, it falls to hard boiled individuals like the Player Characters to set things right. There's a new sheriff in town, and he's got levels in a PC class. This is your chance to use all your Western clichéin a fantasy setting, when you can turn Cowboy Movies into Kurosawa flicks.

Once the players beat back the gnolls who have come in at the behest of hyena ranchers trying to drive the gloom farmers off the land, the place is going to be a nicer place and attract Ogre Duelists or dishonest bankers. When it becomes known that the portal nexus is coming through, suddenly all that property is going to shoot up in value. And suddenly the pit fiends \textit{do} care what goes on in your sleepy neck of the woods.

\subsubsection{Ten Low Level Adventures in Hell}

It's a dusty little town, like an infinite number of others just like it both functionally and aesthetically. You don't know what makes this town special, and with the number of horrors you've seen on the plains -- you're not sure you want to. Still, this is a place it doesn't pay to break the rules when it isn't important, so the first thing you to is walk in through the curtain they hung up on the door to the Town Hall. Inside you see\ldots

\listone
	\item A dried out sahuagin sits behind the desk. He's mumbling about how the water is all gone.
	\item An officious imp attempts to shoo you right back out the door.
	\item Five corpses in fancy clothes lay strewn about the entrance hall.
	\item Putrid husks of humans in cages hang from the ceiling while a ghoul repeatedly jumps up trying to get at the rotting morsels.
	\item A mountain of papers covers the desk. From somewhere behind them a voice tells you that it is busy.
	\item A hobgoblin sits with his feet on the desk. As you enter, he stands up smartly and asks your business.
	\item Long lines of petitioners block off any hope of registering an time soon.
	\item Zombies shamble around the insides of the building and an imp is attempting to complete its paperwork while flying around the ceiling.
	\item The floor has collapsed entirely
	\item The front counter has been smashed and the interior smells like hyena urine.
\end{list}

\subsubsection{Ten Mid Level Adventures in Hell}

At last! You stand before the magistrate, it seems like you've been waiting for an eternity. You state your case, and he tells you\ldots

\listone
	\item ''You have the choice of death by platricorn or death by fire. Choose!''
	\item ''I grant you writ of ownership of Gelzugh's Tavern. You have the full backing of Hell in taking control of it from Gelzugh. Way back.''
	\item ''Your circlet is not \textit{jade}, it's \textit{malachite}, which is totally different. You're going to have to go back into the mines and find a \textit{jade} circlet.''
	\item ''Every one of you are sentenced to clean the sewers of Leng of the crawlers or die in the attempt.''
	\item ''It is Tuesday, so you're going to have to travel to Chitterport to have this taken care of.''
	\item ''Actually, this contract looks legitimate to me; Baelphor is legally the child's father.''
	\item ''I find nothing in this documentation to lead me to believe that these passports have been stamped correctly. Deport everyone.''
	\item ''You can't be serious. These swords aren't even magical.''
	\item ''Foolhardy mortals! You have wasted my valuable time and now I shall waste yours!''
	\item ''Raelzella's marriage is now void, the ownership of the larvae will be decided by combat.''
\end{list}

\subsubsection{Ten High Level Adventures in Hell}

Sorting through the ancient paperwork in the forgotten tower, you've found\ldots

\listone
	\item Documentation that proves that you personally are descended from an Erinyes.
	\item A small plush doll of a petrified Pit Fiend. It appears to be a \spell{shrunk item}.
	\item Spellbooks belonging to an evil lich.
	\item A map of a mighty fortress that appears to have stood where the shard spires stand now.
	\item Proof that a powerful Gelugon is not entitled to his position.
	\item A recipe for a dish now famous throughout the plane.
	\item Tongues of an ancient beast in a box. When the box is opened, the tongues speak of a fortress filled with giants.
	\item A portal to a deeper Hell in between the pages of a book.
	\item Poetry thought lost for a thousand years.
	\item Prophecies that mention you by name.
\end{list}


\subsection{High Adventure in\ldots The Abyss!}

The Abyss is well known for being infinitely big and infinitely bad in all directions, and it is. If there is some hellscape in your nightmares, its probably somewhere in the Abyss and there is someone there waiting to hurt you. The only thing it has going for it is that its very unorganized, meaning that the endless evil is only rarely directed enough the threaten other planes and planar oasis tend to places of great turmoil, meaning that small groups can easily blend in and ingratiated themselves amid the variety of beings that call these planes home.

Unlike other planes, there is no ''standard'' Abyssal Plane, aside from the top level called the Plane of Infinite Portals. These planes may be set up like a deck of cards, but they only share the chaos and evil traits, any particular plane can have any elemental or magic traits in the book and have geography ranging from the mundane mountains, forests, and plains to fantastic locations harmful to all but the most exotic forms of life. The only thing one can depend on is that pits and holes in the Abyss are often planar portals, and they only lead to deeper and wilder layers of the Abyss. Climbing back out of the Abyss is a much more difficult task, one that requires knowledge of planar pathways like the River Styx or powerful magic.

\subsubsection{Campaign Seed: We're the Exotic Products Trading Company (Abyssal Branch!)}

''We are here to serve your needs, and we offer a range of services ranging from capture of exotic lifeforms to collection of unique minerals and lore! We even have an on-call Search and Recovery Team available to recover lost individuals, `bargain' with demon governments, or protect important trade shipments! Contact one of our offices in Sigil or our home office on the Plane of Infinite Portals!''

\subsubsection{Campaign Seed: Pirates of the River Styx!}

''Yo ho, me hearties! The River Styx be vast and mysterious and its waters kiss the Abyssal planes like a cheating lover! Why set sail in the other Lower Planes when the Abyss is infinite and lawlessness is a virtue of its people? The good boat The Groping Marilith has room for any brave soul whose handy with steel or spell and has an eye for exotic and demonic beauties in every port and magic and jewels hidden in the nether regions of every fiend. Come ply the Abyss with us, and forget your troubles on the River Styx!''

\subsubsection{Ten Low Level Adventures in The Abyss}

\listone
	\item Food Run! Demon weevils have infected an Abyssal Town on the river Styx, and the first group to bring untainted food for them will earn a valuable ally.
	\item Race! A Nalfeshness ruler of miles-long city straddling the River Styx on the 33th level of the Abyys has decided to host a riverboat race to please his unruly people. There's big money to be made in this no holds barred sailing race through an Abyssal city!
	\item The good ship Lollyjaws is plying the River Styx with its zombie crew, and they've decided that you've hit a big score and you need help ''investing'' it.
	\item Message in a bottle. A map written in Celestial has been found in a blottle on the River Styx. Its this a map to a treasure, some poor soul's hope for rescue, or a clever trap to capture well equipped adventure seekers?
	\item Run aground! A chaos ship containing mysterious spices and drugs and run aground near a port town, and its bedlam as psychotropic clouds spew forth to wreak chaos in the town. Loot the vessel before the helplessness of the townsfolk attracts powerful fiends who'll sweep up the any booty.
	\item A dark, beautiful, and mysterious stranger decides that only your organization can retrieve a packet of information from the 411th plane.
	\item Mapquest! Map a planar route to an exotic locale in the Abyss, and return to collect your reward.
	\item ''There's an emergency! Deliver this call for help to the 911th plane!''
	\item Worm farmer! Travel to Noisesome Vale on the 489th layer of the Abyss and collect samples of the worms that eat sulfer gas and exhale breathable air for a Fiendish Gnome client with ideas for a Styxian submarine.
	\item An erratic portal between the 1st and 239th planar has started functioning properly again, and the Lost of City of Azzabanazanazan has been found (much to the inhabitants surprise). A little clever negotiating between this city and a few of the more popular demon cities could mean big profit.
\end{list}

\subsubsection{Ten Mid Level Adventures in The Abyss}

\listone
	\item Naval vessels of the Nine Hells have made serious incursions along the river Styx, and a clever ''privateer'' can make a little coin by signing up with a demon lord to resist these salty devils.
	\item Smiley Tom, the infamous Incubus captain of the legendary Slippery Cat has been imprisoned in Graz'zt realm for unknown crimes. Rescue him to gain his legendary gratitude, or use this opportunity to steal the Slippery Cat, the greatest ship to ever sail the River Styx.
	\item The Forgetful Fog Technique. Some clever pirate has figured out a way to create fog on the waters of the River Styx, then push these vapors onto towns and cities, looting them silly while the inhabitants are blissfully unaware. Catch these clever thieves to stop their amnesiac attacks, or perhaps gain a monopoly on this tactic yourself.
	\item One of your mates have finally bedded one lass too many\ldots she's been granted a wish by a glabrezu, and ill-luck follows your mate and his friends(which is unfortunately you). Win her affections back or find her a new romance in the Abyss, or else the curse will be the end of you.
	\item Ever hear of the sea elves living in a city hidden under the River Styx on the 356th plane? Their touch steal memories and they sell them on the demonic market and\ldots what was I saying? Hey, who are you? Who am I?
	\item A lazy balor chief running the glorious demon city of Belzasharazar on the 45th layer wants a new pleasure palace constructed, but his succubus consort has other ideas. Burn the construction often enough and he'll lose interest, and you'll earn a powerful patron in the demon city.
	\item The latest fad in Sigil is the practice of keeping glowing dragonflies as party lighting, but these exotic insects are found only on the 232nd layer of the Abyss, a plane suddenly caught in a vicious conflict between two barely-known demon lords. Deliver a shipment of these blinky bugs to Sigil and you'll be invited to all the best parties, opening up other pecuniary possibilities.
	\item You've been approached by a cabal of wizard from the Prime, and they want information on the Black Tower. Infiltrate the Black Tower to steal their secrets, or turn sides and lead a strike force to the Prime to nip these nosy wizards in the bud.
	\item A cargo box shows up on your door with a valuable, but difficult-to-sell and dangerous product (like a shipment of souls), and several parties seem to think that you are the owner. Find a way to sell the cargo to a more powerful individual or else these parties will take it from you with extreme prejudice.
	\item An old associate has deeded you a confectionary in the City of Brass that specializes in demon chocolates and sweets. The Sultan has decreed that if you don't pay back taxes in city of Brass currency that he'll foreclose on the property (and your soul). Go on a whirlwind tour of the Abyss to collect enough stock to make enough quick cash to save the shop (and your hide).
\end{list}

\subsubsection{Ten High Level Adventures in The Abyss}

\listone
	\item Over a dozen pirate ships working the River Styx have been declaring that they are part of an Armada in order to pass along blame, and they are saying that you are the Admiral! Find and smash these lying upstarts or ''gently convince'' them to actually accept your command.
	\item A general in the Blood War has found a way to divert the River Styx and he is using these pathways to strike key demon and devil armies, killing both his enemies and competitors. Both sides are willing to handsomely reward the party capable of ending this maritime terrorism.
	\item Rumors and hints point to a powerful artifact being transported along the River Styx in a vessel of unusual design, and factions vie to be one to seize this powerful item.
	\item An island has appeared in a notoriously wide section of the River Styx, and dragons have been leaving the island to raid vessels. By your estimation, they should have amassed a horde that is fantastically large, even by the standards of dragons.
	\item The Mask of the Captain has resurfaced, a powerful artifact that creates and closes permanent gateways between the River Styx and the Prime Material Plane, and a powerful Prime nation has decided that they will increase the wealth of their people by plundering the cities of the Abyss.
	\item A trading vessel of unusual design flies into the Abyss, avoiding known planar pathways. It is crewed by a race that planar sages have never seen, and they offer trade goods of exotic and powerful design. Is this a simple trade mission, or an incursion from another plane by a new planar power?
	\item Orcus's agents have begun purchasing magic items related to planar travel, hinting at an invasion of enormous proportions.
	\item A demon lord of waning power has declared that his power and command over his layer of the Abyss will pass onto the individual to defeat him in single combat, and contestants have gathered at his fortress. Is this a ruse to gather the equipment and souls of powerful individuals, or is he truly offering a chance at the title of demon lord?
	\item An old friend brings news of the discovery of an empty city found in perfect condition in the Abyss full of trade goods and magic, but without a single living or undead soul. To take control of this city is to learn its secrets, and possibly gain its enemies\ldots enemies unconcerned with wealth or magical power.
	\item Yeenoghu has decided that you are a demon lord in disguise who is pretending at weakness as a ruse, and he is sparing no cost to send agents to test this theory. Convince him that you are a mortal, or strike him so hard that he ceases his attacks.
\end{list}


\subsection{High Adventure in\ldots Gehenna!}

First, it's the home of the Yugoloths. These outsiders are the dealmakers and compromisers of the fiendish world, so they might be involved in any plot or any scheme that makes its way across the planes. The land itself is series of volcanic mountains where sentients have forced their own existence into, jammed between the Hades and Hell and connect to the River Styx, so it is well situated between several of the Lower Planes. The works of mortals and immortals alike are eventually destroyed by tremors in this architect's nightmare of a plane and only the works of the gods last here. That being said, the entire plane has an angle that ranges from inconvenient (45 degrees) to unlivable (straight up), meaning life in Gehenna is far more socially dependant than other Lower Planes due to the fact that the only place to live is in the cubbies, caves, boltholes and settlements that litter this plane. It's not that you can't live in on the slopes and are forced to cooperate and co-oexist and you are forced to compete for space like in Hades, its just that life in Gehenna without a clique \textit{sucks}.

What do all of these things mean? It means that Gehenna is a realm for movers and shakers, a place where ''the deal'' and ''the juice'' matters more than any ideals or hopes. Even the petitioners of this plane are only concerned with power, and only the cruel nature of this plane keeps them chained here. Brinksmanship and counting coup and favors are the symbols of power here, and mere physical might or magical power take a backseat to one's ability to \textit{manipulate people with physical power and magical might}.

\subsubsection{Campaign Seed: The Yugoloths Want You!}

While Tanar'ri generals are known the power and might of their hordes and Baatezu armies are know for their frightening disciple and efficiency, it is the Yugoloth forces that are know for their subtlety and tactical elegance. They don't fight for reputation or honor; they fight to fulfill a contract and make a profit, making them among the deadliest generals in the Lower Planes.

You've joined that organization now, and the Yugoloths have need for elite squads of problem-solvers with a propensity for violence and a capability for discretion.

\subsubsection{Campaign Seed: The Grand Game in the Crawling City}

In the Crawling City, you've got to be useful or you're dead. You attached yourself to a minor Yugoloth noble, and he's begun using you as behind the scenes agents in the Lower Planar courts. With skill and nerve, one day you might earn the fear and respect of the fiends and become a player in your own right.

\subsubsection{Ten Low Level Adventures in Gehenna}

\listone
	\item A famous Yugoloth tactician is taking new students, and he's set a distinctly fiendish entry requirement: interested students publicly apply, and one week later the first to present themselves is accepted. The last time he took new students, no applicant ended the week alive enough to show up\ldots
	\item Small bands of petitioners have been gathering under the banner of a charismatic profit and raiding minor settlements in the night. Eliminate the threat by assassination or counterattack.
	\item Tremors! Minor rumblings and a trusted fiendish seer predict a major lava spout in a small settlement, destroying it, and several interestied parties want to loot it or the refugees in the final hour. Intercept these rogues, or plunder the settlement for yourselves.
	\item A minor Baatezu noble has been spotted in the Crawling City, and it's suspected that he's trying to hire away an elite group of Baatezu mercenaries when their current contract expires. Find and interrogate him, and the Yugoloths will repay this little favor. Whether he returns to his home plane with his life and valuables is your own business.
	\item The Double ''H'' Run. Despite the Blood War, some trade does exist between the Baatezu and the certain Tanar'ri, and the Yugoloths have their hand in it. Escort a package between the Nine Hells and Hades, avoiding agents from both fiendish factions who would use it to discredit their countrymen.
	\item The Masked Ball is next week, and a clever soul capable of learning the identities of several indiscrete parties can earn a few coins with the information brokers of Gehenna.
	\item A tiefling fop of a swordsman has defeated several prominent Yugoloth blademasters in mostly fair duels, despite his obvious lack of skill. Several persons of note would like to know his secret, and would pay even more to have that secret removed at an opportune moment.
	\item A mortal Sorceress of rare skill and infamous carnal desires has come to Crawling City, and entities of power are jostling to be known as one of her clients. Secure her cooperation for a client and win wealth; secure it for yourselves and win power and danger.
	\item A Tanar'ri of an unusually Lawful bent has entered the service of a Yugoloth of middling power. Discover the secret of his service, and that service can be passed on to a more worthy fiend, or kept as secret weapon for yourself.
	\item A Yugoloth of some influence has secured the services of an unusual household staff of famous, though powerless, Prime mortals. Spoil his coup by tempting, tricking, or intimidating these mortals into committing terrible blunders during the next power meeting, and you can harvest some amount of his influence.
\end{list}

\subsubsection{Ten Mid Level Adventures in Gehenna}

\listone
	\item A powerful Tanar'ri fortress has been bidded for destruction, and the Yugoloths will pay well for the group that finds an exploitable weakness.
	\item Several subcommanders have been bickering over the right to extract a powerful dragon of a military bent from Carceri, and rewards will fall upon anyone capable of securing this beast's services for the Yugoloth.
	\item A key planar touchstone in Hades will prove the key to an isolated fraction of the Blood War, insuring victory for one side or the other. Destroy this site, or profits for the Yugoloth in this conflict will fall dramatically. Secure it for yourself and turn it against both armies to secure a stalemate, and some fraction of the increased profits will fall your way.
	\item A powerful Yugoloth well- known for patronizing up-and-coming allies has declared that you are his protege, making you a target for his enemies Punish these enemies, and you might secure his patronage for real.
	\item A small army in the Blood War has wandered into Gehenna and is a threat to the Yugoloths. Destroy its leadership and loot its paymaster, and the Yugoloths will see that you are amply rewarded.
	\item A band of thieves have turned the Crawling City upside-down. Recover and return the valuable objects and win influence. Hold the objects hostage for future favors, and gain power that money can't buy.
	\item An unknown magical effect has stopped the feet of the Crawling City, and the first to discover the cause will win no small amount of gratitude from the ultraloth ruler of the city
	\item A series of businesses across Gehenna have been vandalized, an obvious turf war between two competing interests, and the first group to discover the identity of either player can earn a contract to accelerate or reverse the destruction.
	\item A spellbook of unique magics useful to a courtly mage has been found, and the owner of such magics would pay handsomely to not have his secrets revealed.
	\item A Baatezu diplomat has come to Crawling City, and he has decided that you will become his agent. Avoid a diplomatic incident without betraying the Yugoloths, and the powers that be may reward your ability to resolve such a conflict.
\end{list}

\subsubsection{Ten High Level Adventures in Gehenna}

\listone
	\item A cabal of liches have a sudden need for several rare components, and they are willing to trade battlefield magic for the first party to collect their list.
	\item It has come to your attention that several key subcommanders are plotting a coup over the control of the Crawling City. Shatte this conspiracy, or risj all and become its ringleader.
	\item The Yugoloths are looking to subcontract a dangerous mission on the prime against a noble house of demon-hunters. Get the contract and eliminate the hunters, or accept a greater bribe from the them to hold the contract long enough for them to counterattack.
	\item Key contracts for the Blood War have been stolen, and the first person to recover them will control a Yugoloth army of immense proportions.
	\item A war machine of great size and terrible power has been spotted in Mechanus, and such a device would fetch a king's ransom in the war markets of the Crawling City.
	\item A clique of fiendish spellcasters has set a challenge: the first entity to scour the planes for a specific but almost unique spell will earn a tome of their greatest spells. They expect one of their members to win and then resolve a dispute about claims of leadership of the clique, but an indiscrete servant blabbed the rules of the contract and now several interests seek to win the contest.
	\item A mortal noble of rare talents has entered the Crawling City and is recruiting agents for one goal: recover the contract that dooms his soul to property after death. To help him is to defy Yugoloth tradition, but the rewards might just be right.
	\item For some unknown reason, Inevitables stalk the Crawling City, and a clever stagemen might just be able to divert them towards one's enemies.
	\item The ruler of the Crawling city is missing, and chaos rules as several factions make a bid for power.
	\item Negative energy has begun to permeate the Crawling City and undead powerful enough to challenge of Yugoloth leadership have begun to rise. Is this an attack by a god whose portfolio is death, or some ruse to put the Yugoloth against an enemy they cannot defeat.
\end{list}


\subsection{High Adventure in\ldots Hades!}

One would think that Hades is among the worst Lower Planes to adventure in\ldots and they'd be right. The plane itself has the two nasty qualities: it poisons you with the Grays until you become a depressed Goth, and the Entrapping trait takes your memories and makes you want to never leave like a bad house guest. That being said, adventure is still possible, even for the least powerful adventurer.

It works like this: think of Hades as an unforgiving desert. Travel in this ''desert'' is only done by moving from oasis to oasis. These oases are towns and settlements that are built in such a way to resist the Grays and the Entrapping trait (see the Handbook of the Planes for an example of such a place). The only things that permanently live in the desert are creatures who are both immune to the Entrapping trait (like outsiders) or who have already succumbed to it (which has no other game effect other than ''become an NPC who doesn't want to leave''); these creatures also have some way of dealing with the Grays, and so they are creatures with SR 10 or better or are immune to Wis damage (like undead). This generally means that the ''desert'' that is Hades is filled with wild-eyed hermits and bandits and other forlorn spirits (which might be actual undead) living in the blasted and ruined geography of Hades, or creatures of some special power who skirt the edges of civilizations. Some NPCs you meet might just be Entrapped, but enter an oasis once in a while to recover from the Grays; other such characters might have ways to cure the Wis damage that the Grays cause, thus they are entrapped by Hades, but have no reason to enter an oasis, and some powerful creatures can resist The Grays almost indefinitely due to their high Saves.

Hades also has a few other features of note: It's the ultimate source of Evil of all types, and all of the evil outsiders are equally (un)welcome there. You could easily see a Yugoloth, a Devil, or a Demon without that being part of a plot device. Since Hades is the creation place for larva, the serving-sized petitioner souls of very evil people, the big evils of the multiverse have taken to fighting and brokering for this natural resource full time, and it all starts here. Night Hags and Liches are other players in this economy, but they are the freelancers in the publication of evil.

\subsubsection{Campaign Seed: The End of Oasis}

You've lived in the town all your life, and you know that only madmen and the 'loths live beyond the walls, but now you must travel the wasted plains to find your destiny.

\subsubsection{Campaign Seed: A World At War}

The Blood War wages endlessly and pointlessly across the Gray Wastes, with most territory never held or even claimed. The only things that have value in this whole plane are the occasional portal, oasis, or larva vein. Every other patch of land is a liability and \textit{no one} wants it.

\subsubsection{Ten Low Level Adventures in Hades}

\listone
	\item A Yugoloth has died while on a trading mission to your town, leaving behind a shipment of larva. To prevent your town from falling under the 'Loths gaze, you must take them to the nearest Yugoloth city for sale.
	\item A battle in the Blood War was fought near your town, and the undead fodder from that battle now terrorize the countryside.
	\item The leader of your town wants it to become a waypoint for message delivery, and he hires you to delivery the first messages.
	\item Something has been coming in from the wilderness to stalk the townsfolk. Will you track it back to its lair outside of town?
	\item The well has been poisoned, and you must find a new source of water for the town deep underground, far from the protective influence of your home.
	\item A terrible new disease has been ravaging all the nearby towns, and the Oinoloth has decreed that the town with the best gift will be spared.
	\item Devil agents want to construct a supply depot far from their own infernal realm, and will pay well for the location of new oasis(minus any current inhabitants).
	\item The nearest town has its eye on the riches of your town, and now has agents and a small force scouting for weak points and key personality to kidnap.
	\item Two caravans have entered your town at the same time, and now they have begun attacking and sabotaging each other at night in an effort to be the only one to leave.
	\item It's Election Day! Factions in town work against each other in an effort to become the new Mayor, and everyone knows that the loser will end up exiled to the wastes.
\end{list}

\subsubsection{Ten Mid Level Adventures in Hades}

\listone
	\item For some, mere death is not a real revenge. A powerful leader hires the party to defend a prison built in order to entrap entities in Hades in a spot unprotected from the effects of the plane.
	\item A legion of elemental soldiers have been led through a Gate, and they have succumbed to the effects of the plane. The first town leader to convince them to join him will gain a powerful fighting force.
	\item The Yugoloths have decided to annex your township, and only a show of overwhelming force or a high bribe will convince them to leave your town alone.
	\item Something is destroying oasis after oasis, isolating your town from the trade paths.
	\item A Gate has been opened to Celestia, and celestials have offered asylum to your township. Is this an opportunity to evacuate your town, or is this a fiendish trick to destroy your town?
	\item during a battle in an unfamiliar oasis, your party is trasnported to an unknown location in Hades, far from any oasis. Can you find your way home, or even to a safe location before you succumb to the planes traits.
	\item A series of Gates have opened up to a distant region in Hades, and townships now vie to control the altered landscape.
	\item The river Styx is flooding, and threatens to wipe out several cities built on its waters, including your town's primary trade partner.
	\item A caravan of bioloths has entered your town, beginning a carnival that threatens to enslave everyone.
	\item A powerful Yugoloth has been working against the Oinoloth, and your town is caught in the cross-fire. Will you work against it, or for it?
\end{list}

\subsubsection{Ten High Level Adventures in Hades}

\listone
	\item Rumors hint that your town holds a mystical font that can make anyone bathing in its waters immune to Entrapping and the Grays, and several powerful forces vie to control this wonder.
	\item The Blood War has boiled up in your region, and a clever party could benefit from working with one side or the other, or even both.
	\item A powerful devil decides that he needs more exotic troops, and he is willing to extend his protection to your town if you can capture powerful creatures from several legendary parts of Hades.
	\item Angels have gained a foothold into Hades, and have decided that your town is the first to be ''purified.''
	\item During a particularly brutal battle in the Blood War, a powerful artifact has been lost. The first to regain such an artifact might be a threat to even the Yugoloths.
	\item A cabal of Night Hag Sorcerers have decided to harvest your town, and the only way to catch them is to breach the barrier between your plane and theirs.
	\item A powerful outsider offers his services to your town, saying that he can create planar gates. Such a resource would transform your town into a planar metropolis, but can it survive the attention it will attract?
	\item A powerful Warlord has taken over rulership of several towns, attempting to build an empire in Hades, and your town must either gather the forces of the surrounding towns to fight this menace, or usurp rulership for yourselves.
	\item A dangerous wizard has found a way to concentrate the evil of the plane, and he is using this evil as weapon that can corrupt even the Yugoloths to his person brand of evil.
	\item Strange and terrible diseases are taking their roll on all the inhabitants of Hades, and the only way to stop these plagues is to assume the mantle of the Oinoloth.
\end{list}

\section{High Adventure in the Elemental Planes}

\subsection{High Adventure in\ldots The Plane of Fire!}

More than any other Inner Plane, adventures in the Plane of Fire tend to take place in planar bubbles. If you can breathe water, the majority of the Plane of Water is basically just a lukewarm benthic zone, and it's the kind of place that Sahuagin might live without even realizing that they weren't on the Prime. But the archetypical expanse of the Plane of Fire is just, well, fire. It's like the churning surface of a sun that extends in all directions for eternity. And while it is colder and less destructively melty than the all-consuming plasma of an actual star, it's still basically just an endless expanse of fluid, dangerous, useless fire. Did I say useless? You bet, because heat engines actually work by heat difference, so from the standpoint of residents of the Plane of Fire it is actually cold that you use to run a power plant. The fire in between everything is just like the worthless emptiness of deep space except that it will also catch you on fire. Forget Carceri or the Gray Wastes - the Elemental Plane of Fire is the worst place in the D\&D multiverse.

But just because it's a horrible place, even the worst place, doesn't mean that there isn't stuff you want there. And just because it is the most inhospitable place imaginable, doesn't mean that low level characters can't adventure there. The key is the planar bubbles exist. That is basically the only reason that anyone gives the Plane of Fire the time of day. The most important bubbles are Prime Bubbles. These are areas of land and sea with atmospheres, that happen to be shaped like a Ptolemic world - a circle of land and sea with a hemisphere of atmosphere above. And of course, outside that is endless roiling fire. So the ground gets kind of rocky and parched, what with the sky being a never-ending holocaust without reason or respite - essentially it's like living in a Dragonforce video.

Those Bubbles aren't just the only place your characters can survive, they are the only places that any of the residents give a damn about. Remember that even if you happen to be a fire elemental, you still eat ''flammable" materials if you want to grow any larger, and those only come from the ''cold" spots. So not only is the practically usable terrain in the Plane of Fire very small compared to the plane's total volume, but the space between is inhospitable void. And not just inhospitable void - it's opaque inhospitable void. Standing on one of the floating islands, you can't even see the other islands. When you look into the inferno you have no way of knowing whether the next place of value or substance is a few centimeters or a few parsecs of burning emptiness in any particular direction.

So what does that mean for the low level adventurer? It means that practically speaking, no one expects your character to want to go anywhere that would cause them to actually catch fire. No one else does, not even the planar residents who are actually made out of fire. So it's totally workable as an adventure locale at any level. The Plane of Fire is run by the Efreet Sultans, and that gives the entire place a very fantasy-Arabic feel. Ignan, the approved lingua franca of the universe, is explicitly based on Arabic. That thing where Arabic calligraphy kind of looks like living flame? Yeah, they went there. While the Djinn have a presence in the Plane of Air and the Dao have their own Caliphate in the Plane of Earth, the Sultan of Fire owns the Plane of Fire. Because there is hardly any real estate, and finding or getting to it is in most cases a Wish Economy proposition.

The Plane of Fire is your chance not only to throw out every Arabian Nights cliché you know, it's also a place to throw in 1950s sci-fi left and right. Basically everywhere that anyone lives is one of those bubble colonies or asteroid mining facilities from the Heinlein juveniles. To get from on planetoid to another requires getting into a heat protected shell and then throwing yourself from one to the other. Once you leave a Planar Bubble, there's no gravity or wind, so it's basically exactly like one of those personal space ships that were talked about in the old Republic Serials. Some of them are even saucer shaped.

Campaign Seed: Conquest of the New World: Even beings of pure fire cannot see far into the firmament, and so it is that new places of interest are ''discovered" all the time in the most surprising of places. The iron ships that travel between bubbles need exacting angles of departure, because once they are off course, there's really no measurement you could take to figure that out (and often nothing you could do about it if you did). So a new island might well be just 1 degree off an established trade route. And once a new land is discovered, it's Columbian Conquest all over again. This new world may well have occupants that object to being ''discovered" let alone colonized, but on the other hand they could seriously have fountains of youth or cities of gold.

Exploring a new Planar Bubble in The Plane of Fire is a good way to bring out any kind of D\&D adventure you want. The PCs have literally no idea what they might find there, and there's a very great incentive to keep exploring since even wood and water are hugely valuable resources once you get off this gravity well and back to a more civilized one. You don't just get to loot the temples of stone using pyramids, you also get to confront their heathen demon gods, find relics of fallen ancient civilizations or the secrets of long forgotten wizards. A Planar Bubble that ''no one" knew about on The Plane of Fire is about the safest place in the entire damn multiverse, so anyone who did know about it could have stored or imprisoned, well, anything there.

\subsubsection{Campaign Seed: Janissaries of the Fire Sultan:} The Efreeti sultan is cruel, but he is not stupid. And he is well aware of the limitations of being a guy who is on fire all the time when the only things in the entire universe that have any value are things that are not on fire. And so it is that the Fire Sultan has children of non-flaming races raised in his employ. These children grow up to be Janissaries: creatures who act as agents for the Efreeti and build their empire without incidentally burning it down. There is a lot of room for advancement in the Janissaries, the Sultan genuinely values your skills more than he values the skills of the other Efreeti. First of all, there is basically no chance of you ever actually becoming Sultan (you just don't have the right fire in your blood), and secondly, unlike a real Efreet, you can do stuff that the Sultan cannot. There are a lot of politics that go in court, and the rest of the Efreeti have a tendency to rather resent Janissaries; while at the same time doing their damnedest (literally) to avoid any direct confrontation with something the Sultan considers to be ''his." Do the Sultan proud, and you can have your every wish granted (as long as that wish doesn't include becoming Sultan or leaving the Sultan's employ). Fail him sufficiently, and he may allow the more jealous members of the court to take their frustrations out on you.

\subsubsection{Ten Low Level Adentures in The Plane of Fire:}

You're getting the report from the overseer of the pipeline workers. The Kobold tells you that they aren't getting as much water as expected because\ldots

\listone
	\item A group of Firenewts has claimed that the pipeline runs through their tribal lands and have begun monkey wrenching.
    \item The water reserves aren't as extensive as hoped near the surface, and the pipeline will have to be extended into the caverns.
    \item Superstitious fears have broken out among the workers, they speak of burning snakes.
    \item Drilling has broken through to inferno before expected, this rock isn't as stable as we'd hoped.
    \item The water has some kind of creatures living in it. Creatures that live in water.
    \item Some creatures have been bringing clouds of smoke with them when they crawl over the pipeline.
    \item A rival mining group is siphoning water from our reserves.
    \item Some guy who looked like a Yak has paid more than enough money for the land to get the crew to drill elsewhere.
    \item Everyone who touches the water seems to forget what they were doing.
    \item The water has been draining up to the mountain.
\end{list}


\subsubsection{Ten Mid Level Adentures in The Plane of Fire:}

Laughing, the Efreet relays the news. It's never a good thing when an Efreet is happy to tell you something, and this is no exception because\ldots

\listone
    \item Some group of xorn came in with a load of opals just two days ago. You're going to have to go farther afield if you want to liquidate those gems.
    \item It seems that while you were out, they've made a new appointment of Sheriff.
    \item The land title has been revoked and given to Hakim
    \item Surtyr wants his money back. Now.
    \item Yak Men have taken over the entire city.
    \item A Red Dragon has claimed the water reserves.
    \item The Bubble has begun wobbling, the only way home is by wish.
    \item The princess is in another palace.
    \item The gnomes have themselves a Frost Salamander that they are keeping alive somehow, and mere flammables are virtually worthless here.
    \item The great astrolabe has been shattered.
\end{list}


\subsubsection{Ten High Level Adentures in The Plane of Fire:}

The Iron Flask isn't completely inscrutable, and your research indicates that it contains\ldots

\listone
    \item One of the Sultan's uncles.
    \item A potion of Immortality
    \item A gate to a deep layer of Baator
    \item The heart's blood of Baphomet
    \item The phylactery of a powerful Lich
    \item A decree from the previous Sultan
    \item A heretical Genie who was imprisoned for predictions that appear to have come true.
    \item The crown of Pyriria
    \item The condensed gaseous form of a Chaos Roc. One of several, if the accompanying letter is to be believed.
    \item The laughter of Queen Chandra.
\end{list}


\subsection{High Adventure in\ldots The Plane of Water!}

The Elemental Plane of Water is an endless expanse of relatively static water permeated by a soft ambient light. There is only gravity if you want there to be, and the incompressible medium makes gravitational movement slower than walking. But nonetheless, you can move pretty much anything at the rate of about three and a half miles per hour just by ''falling" or ''rising" with it. Outside of an occasional ''pressure zone" the entire plane is pretty much one giant coastal shallows, with a water pressure at any point about that of being under just a meter of water. The Elemental Plane of Water is also the largest place in all of the D\&D multiverse in real terms.

Sure, it is ''infinite in all spacial dimensions and time" just like all the other Inner Planes, but it is markedly different in that every point in the Plane of Water is also a place. None of it is empty or impassable, it's all just made of water. So you can go and be anywhere, and you won't be ''between" things because the place you will be will be an actually stable location in and of itself that you can put stuff down in or give directions to. Every point. And that means that there are more places to be, and by extension more stuff than in any of the other planes. Indeed, like how on Earth about 70\% of your body is water, and about 70\% of the world's surface is water, about 70\% of the creatures and structures in the Inner Planes are on the Elemental Plane of Water. And like the oceans of every Prime World - the Plane of Water still gets less press than the other planes because it is full of water. In general, things on the Elemental Plane of Water stay where they are put, with little in the way of mobility. This means that when there is an air bubble, people can pretty much run around in it without fear that the air will bubble up away from them. Because there is no up. This also means that disposal of bodily waste is ''gross." There is nowhere to ''bury" anything, so stuff that comes out of you just sits there accusingly. Fortunately, there are a lot of plants and little animals that will come clean that up, but this process is no nicer to watch on the Plane of Water than it is anywhere else. There are areas where, for whatever reason, the ambient water is flowing with some kind of current. Some of these currents are incredibly fast, but as a rule they are not that ''large" and full mixing doesn't happen. The fresh parts of the endless sea stay fresh and the salty parts stay salty. The hot parts stay hot and the frozen parts stay frozen.

The Marids are, individually speaking, the most hard core of the Genies. However, the Great Padisha of the Citadel of Ten Thousand Pearls is basically just the mayor of a town of one thousand occupants. One thousand occupants where one in five of them can grant frickin wishes, but just a thousand all the same. You could seriously move around the plane your whole life and never come within the demesnes of a Marid. Each Marid considers themselves to be royalty and to rule all they survey - which is basically true but functionally meaningless because you normally can only see about 60 feet on the Plane of Water because there's microbes and sand and stuff in the water pretty much everywhere. This contrasts sharply with the Sahuagin empires, some of which are ten thousand miles across (note: this is bigger than the entire Earth, and we're talking volume rather than surface area, so some of these empires have populations that measure in the tens of billions), but which due entirely to the sheer vastness of the plane and the smallness of any visitor's personal experience of the place (60 feet or so around them and movement as fast as they can sink or swim), it is still entirely likely that you've never heard of any of them.

While the visibility on the plane of water is total crap, the audibility is intense. Water is nearly incompressible and it's nothing but water forever and ever. Sound pretty much follows the rule that any noise is four times as quiet when at twice the distance, with no additional dampening from the atmosphere. Any noise ever propagates with such totality and speed that to the human visitor it is nothing but a constant deafening roar. Indeed, since sound travels so much faster in water than in air, any non-aquatic visitor needs 10 ranks of listen to even have a hope of locating any sound. Even sounds that are loud or close enough to be distinctly made out sound like they are from “everywhere.” This is not a problem that natives have, and indeed a Sahuagin can locate you by the sound of the water against your skin.

Secession is constant in the Plane of Water. Anyone can just pick up their house and leave at a bit over 3 miles an hour. Between this tax day and the next, you could have moved your house about 29,000 miles – which is noticeably more than the circumference of the Earth. And when you factor in the fact that there is no guaranty that anyone will find your house if you move it 100 meters, one can see that you can vanish from a government's radar very easily if you are not actively imprisoned. The standard therefore is to be required to pay taxes to the local authorities at the beginning of the year and subsequently be allowed to provide proof of citizenship to receive services for the following year. Surprisingly, much of the civilization in the Plane of Water is actually more recognizable by connoisseurs of modern nationalism than are the kingdoms of other planes of existence. If you want to live in a “country”, you have a citizenship card and rights and social services and stuff. Anyone who doesn't want those things (or doesn't want to pay for them), just leaves and lives elsewhere in the roaring darkness.

\subsubsection{Campaign Seed: Heralds of the Empire:} Sound travels fast under water, but news does not. When a new nation takes hold of a region, it can take a long time to even find everyone who lives there. And so it is that any nation state or empire needs to send out groups to patrol their territory. Not just to keep an eye on the citizens and provide whatever services the empire provides to the hinterlands – but also to keep the maps updated. After all, any part of the empire that hasn't been patrolled in the last month could seriously have had someone move a castle from 4000 kilometers away to there in the meantime. As representatives of the state being sent into areas of water that the state either has not been to yet or has not been to recently, the PCs could encounter pretty much anything at all. And they have a built-in plot hook that encourages them to interact with anything they fine. Whether they face level appropriate wandering monsters, social encounters with dubious locathah, or hostile empires coming the other way, the PCs can plausibly encounter level appropriate opposition at any level.

\subsubsection{Campaign Seed: Tidal Merchants:} The great tidal streams are currents that move with surpassing speed. Those who ride them can get places that are very far away in very short periods of time. And that's saying something in a world where seriously anyone can tie themselves to their cargo and “sink” 80 miles a day just by deciding to. The currents don't just provide fast transport, they also provide a path, a place to go. And so it is no surprise that as one drifts along the tidal stream, one can hear the drums of civilization from all sides just as you can see the glowing lights of fast food joints while driving on a freeway on Earth. Traveling along the tidal streams brings one from one urban development to another with all the vast spaces between literally washed away.

\subsubsection{Ten Low Level Adventures in the Plane of Water:}

The old Locathah is certainly interested in your proposal. But he says he has other problems\ldots

\listone
    \item Sahuagin raiding has hit several nearby kelp farms.
    \item Shark attacks are on the rise.
    \item No one seems to want to buy the sponges he has been growing.
    \item His daughter has the ick.
    \item Food supplies are running low.
    \item The fish are migrating out.
    \item A local hot spot is attributed to Fire leakage.
    \item Those who die seem to come back as zombies.
    \item A siren has been throwing her weight around.
    \item Pirates have seized the oyster bed. 
\end{list}

\subsubsection{Ten Mid Level Adventures in the Plane of Water:}

The sound of drums has called you to the activities like moths to a flame. When it comes into view, it appears to be\ldots

\listone
    \item A brass sphere, with no immediately obvious entrances.
    \item An army of skeletons.
    \item The coral towers of a merfolk city; they look sick.
    \item An ice factory.
    \item Angry tritons.
    \item A giant eel that had been mimicking civilization sounds by slapping rocks together.
    \item A Sahuagin kelp outpost.
    \item A family of scrag wreckers.
    \item A Marid Sattrapi.
    \item Some sort of mechanical vessel shaped like a lobster. 
\end{list}

\subsubsection{Ten High Level Adventures in the Plane of Water:}

You've broken into the massive mechanical manta ship. Inside you find\ldots

\listone
	\item Spongy, organic passageways... this ship is alive.
    \item The crew are long dead and dust.
    \item The captain's log mentions you by name.
    \item Kuo-Toan pirates and their Yugoloth servants.
    \item Sack after sack of dream dust.
    \item These look like dragon eggs.
    \item The spectral pirates who run this thing.
    \item A cargo hold full of wild eyed prisoners.
    \item A cargo hold full of non aquatic and fearful prisoners.
    \item The ship's wizard captain and his crew of blood-indifferent golem pirates.
\end{list}


\subsection{High Adventure in\ldots The Negative Energy Plane!}

If you're even considering running a game in the Negative Energy Plane, it is very probable that you are using Playing With Fire morality for your necromancy. This is in large part because every writeup of the NEP ever made has assumed Playing With Fire, and that indeed it is precisely these descriptions that give people the best scriptural ammunition against Crawling Darkness. But also because if Negative Energy is inherently evil, the plane becomes incredibly boring. We already have the Gray Wastes of Gehenna, so there's no real point in having another gray desert made out of ultimate evil.

The game provides two supposedly different Negative Energy Planes for you to consider. One is made out of Major Negative Energy Dominant with patches that are Minor Negative Energy Dominant, and the other is made out of Minor Negative Energy Dominant with patches of Major Energy Dominant. Well, anyone who has ever looked at a splotchy cow knows that whether you have a black cow with white spots or a white cow with black spots is entirely a matter of perspective. Since the NEP is infinite, both Major and Minor patches are infinite in size and in scope, so it really makes no difference at all which one you are nominally using. From a practical standpoint, either way you're going to be in either a Major or Minor Negative Energy area, the adventure location you are going to next will either be in the same area or a different one, and if you go far enough in any direction you will go from one to the other. And anyway, both Minor and Major Negative Dominant ares are totally fatal to living creatures, and completely harmless to undead and constructs, and the baleful effects are completely negated by negative energy protection or attune plane. So seriously: who cares? Since the only actual difference is the unprotected living creatures crumble to ash in Minor Dominant and are transformed into wraiths in Major Dominant, our suggestion would be to go with Major Dominant most of the time. It's largely academic, because outside the planar bubbles there is no air (so without some sort of magical attunement, every living creature is just going to die of asphyxiation, negative energy or no).

The Negative Energy Plane hates life. It hates the good, and it hates the wicked both the same. It does not condone or aid harm or murder, it simply greedily and expeditiously extinguishes any life exposed to it. But if you're alive that's basically no worse than the vacuum of space, and if you're not alive it's a whole lot better. For those who are undead, non-living, or have the right kind of protections, the Negative Energy Plane is a lot like any other void plane of the D\&D cosmology save that there is no ambient light source. Comparisons can be made to Limbo, the Astral Plane, and of course: the Elemental Plane of Air. The difference is just the fact that it is unlit, and therefore looks like the night sky rather than extending out to a gray fog where the soft glow of the ambient light eventually wipes out anything you could see.

Once you factor in the Planar Bubbles (which as an ironic statement, are called ''doldrums" by Negative Energy inhabitants), the Negative Energy Plane is basically exactly the same as our universe. If you were on a prime bubble, you pretty much would only with difficulty be able to know that you weren't on a Prime. There's a dark hostile, airless void outside your planet, and there's absolutely nothing stopping any light source of any distance from eventually sending its ray to you. So the sky above you is black and full of tiny lights. Well, it wouldn't really be that difficult to figure it out, because absolutely everyone can fly just by thinking about it. And the lights in the sky are just like what ancient people thought about them: some of them are very large and far away (like Elemental Fire bubbles that function as stars), and others are more modest light sources that are more reasonable distances. The intrinsic flight includes not only hovering, but also acceleration that is only relativistically limited. You can accelerate at 1G or more by sheer willpower as long as you want without energy expenditure. So a trip from the Earth to Mars would take less than 5 days even at its most distant point (assuming that they were both on the Negative Material Plane). So titanic, even solar distances are quite reachable. Also of note is that the directions to Neverland (Third star on the left, and straight on 'til morning) are completely reasonable directions, and represent another planar bubble that is about 2 million kilometers away. Like all regions of subjective gravity, going ''towards" a point will automatically have you accelerate continuously to the halfway mark and then have acceleration away from it for the rest of the journey, so you never ram into anything at relativistic speed.

The distances between things in the Negative Elemental Plane are truly vast, but travel is so easy that from a practical standpoint, things in the Negative Energy Plane are actually kind of ''happening." The exception of course, is unlit structures. These are called ''Castles Perilous" by the locals, and making one is pretty much a declaration that you under no circumstances want visitors. After all, without giving off any light, you're basically about as findable as any rock out in deep space is in the real world. The only ways to find one are to happen to see them passing in front of a light source or to shoot one's self off into the void looking for the automatic deceleration that accompanies moving towards a real object - and even knowing that second one is an option requires the kind of math you'd need a Knowledge (Planes or Engineering) DC 25 test to do.

An important thing to consider is the presence of Voidstone. It's a special material that will destroy and absorb any creature (even undead creatures) if they come into contact with it for a few seconds. Truly badass creatures like dragons and gods might be able to hold it for a minute or two before being eradicated from existence, but as you might imagine, that stuff is still in huge demand for making into weaponry. Since it doesn't do anything to other inert elemental material like, say metal tools, it ends up being quite workable and incredibly valuable. Voidstone is planar currency for obvious reasons - but finding it is very difficult because it's not very large, pure black, and forms in the middle of large sections of empty void.

But perhaps the most important point about the Negative Energy Plane is that the parity with the Positive Energy Plane is not complete. Living creatures are natural, so they have no protection from being exposed to ''too much" positive energy - and they can totally explode. Undead creatures are unnatural and only exist at all because they are supported by magic to siphon off a specific and measured quantity of negative energy. So they don't ever ''explode" in Negative Dominant areas, whether they have ''protection" or not. As such, groups of intelligent undead often make homes out of Castles Perilous in the middle of strong Negative Energy Vortices. Because seriously: why not?

\subsubsection{Campaign Seed: Death World:} A Doldrum region in the Negative Energy Plane is a lot like Neverland if it was made by American McGee. Everyone can fly like Peter Pan, and each region fills up with weird crap from all over the planes like tribes of Indians, mermaids, and pirates. However, these places are also constantly under assault by a low level rain of zombies from space. That's not a joke, undead beasts literally float around in the void and choose to fall towards points of light. So if you're running around Pixie Hollow, there is a not insignificant chance that some undead monster is going to fall out of the sky and go on a rampage. This setup allows for very reasonably scaling D\&D adventuring. After all, if the PCs become masters of their surroundings and conquer the Maze of Regrets, you have a totally reasonable excuse to have a level appropriate undead army fall from space and tart causing havoc. In the meantime, even though the levels of Negative Energy aren't high enough to snuff the life out of anything, they are leaking into Doldrums enough to make things subtly creepy and unpleasant. Feel free to use any Ravenloft clichés you want. Or just American McGee it up - people live on a fricking Death World, so have just messed up stuff happen all the time. Have cats croak out ''help\ldots me\ldots" for no reason. Have thorns drip unexplained blood. Have trees inexplicably drain of color. Inhabitants go crazy and start eating pieces of themselves. Go nuts.

\subsubsection{Campaign Seed: Welcome to the Void Heart:} There is a city built into the inside of a one-mile diameter iron Dyson Sphere which is called ''Heart of the Void" or ''Deathheart" depending on who you ask. Some sages built a city there a long time ago and eventually an army of the undead broke in and murdered everyone. Tonight it's a minor necropolis that is broken up into factions that fight each other for domination. And I know what you're thinking: so what? I mean, that's only 3.14 square miles of city, and even though it has the population density of New York, it still only has 70,000 inhabitants, and a lot of them are ghouls. But the really important thing is what the sages used to do, which was to track all the objects in the Negative Energy Plane. All the rocks of Voidstone, all the Castles Perilous, everything. No one knows how they did it, because some vampiric minotaur killed the last of them a few hundred years back and feasted on her heart - but they did leave notes. All over the city, there are books filled with page after page of descriptions of the size, shape, and location of various objects in the void. There are a lot of adventures there: some books are useless without other books in the same series; some books are the possessions of hostile undead gangs that either do or do not know how valuable they are; and many books detail the locations of items and structures that are themselves interesting and valuable adventuring locales.

\subsubsection{Ten Low-Level Adventures in The Negative Energy Plane}

The ghoul chitters and licks his parched lips. Seemingly reluctant to proceed, he whispers\ldots

\listone
	\item ''You may have defeated me, but there are a dozen more on their way\ldots"
    \item ''Fellnax wants his coins. He wants them bad\ldots"
    \item ''You can kill me, I'll never tell you were the diadem is."
    \item ''I knew someone would find me. I didn't know who, but after the Hellmire job, I knew it was only a matter of time\ldots"
    \item ''These bones\ldots these bones are mine\ldots"
    \item ''You traitors! I'll feast on you!"
    \item ''Do you have the scrolls? My master said you would have the scrolls\ldots"
    \item ''You don't look like Fellnax's men."
    \item ''Fellnax sent me to tell you, to tell you that he is going to kill all of you\ldots"
    \item ''We still have the girl, please don't do anything we'd both regret."
\end{list}


\subsubsection{Ten Mid-Level Adventures in The Negative Energy Plane}

It's good to meet another outworlder. But there's something weird about this guy\ldots

\listone
    \item There are faint sobs coming from his backpack.
    \item He casts no reflection.
    \item Everytime he mentions the Castle Perilous he came from, he looks over his shoulder.
    \item There are the scars of bite marks all over his arm.
    \item When he talks about his family getting eaten, it's like he doesn't even care.
    \item When he mentions the golden statues of Kath, it's like he doesn't even care.
    \item He seems genuinely relieved to be here.
    \item He steps right over the ghoul corpses as if that was a normal thing.
    \item He has one of Fellnax's amulets. Or something that looks just like one\ldots
    \item There is a wraith following behind him, one that looks just like he does\ldots
\end{list}


\subsubsection{Ten High-Level Adventures in The Negative Energy Plane}

You've got a fix on the Voidstone you were looking for. Unfortunately it's\ldots

\listone
    \item Suspended inside the chest cavity of a dracolich.
    \item Worshiped by a death cult of Kuo Toa.
    \item Inside a Castle Perilous named ''Doom Watch"
    \item Been made into a sword by a mad Duergar.
    \item Guarded by a Void Shadow.
    \item Guarded by a Shadow Dragon
    \item The Tomb of a fallen god.
    \item Locked in Lethe Ice.
    \item On the far side of an Allip Belt
    \item In the workshop of a Master Skincrafter.
\end{list}

\end{document}

% Traps
\section{Playing the Game}
