Mountaineer [Skill]
Other people get vertigo just by looking at you.
This is a skill feat that scales with your ranks in Climb.

Benefits: You gain a +3 bonus to climb checks. You don't take armor check penalties to climb checks.
4: You may climb one-handed with a -5 penalty. If you don't move during a round, you may use both hands for other tasks if you succeed on a climb check (normal DC for the wall/slope; failure means you fall).
9: You gain a climb speed equal to half your land speed.
14: You may move across a surface of any incline as easily as if it were horizontal (including walking on the bottoms of ceilings and the like). You may substitute your climb check for a balance check if you use one hand to steady yourself.
19: You gain a +4 bonus to tripping or to resist being tripped. You may substitute your climb check for your reflex save.


Super Jump [Skill]
You leap tall buildings in a single bound.
This is a skill feat that scales with your ranks in Jump.

Benefits: You gain a +3 bonus to jump checks. You don't take armor check penalties to jump checks.
4: You suffer no penalties for jumping without a running start. Any penalties to your jump for having a speed less than 30 feet are reduced by half.
9: You suffer no speed penalties for crossing difficult terrain. You may move across water, rotted wood, or other surfaces that do not support your weight with no penalties and no chance of falling through as long as you move at least 20 feet each round. When you attempt a high jump, you may use the DC for a long jump of the same distance, but the vertical distance moved still counts against your speed.
14: As a move action, you may move a distance in feet equal to 10 + your jump modifier in a straight line in any direction (including vertically), even if you have no surface to jump off of. You may choose to hover in mid-air at any time, but anything that grounds a flying creature also works against you.
19: When taking a 5-foot step, you may move 10 feet (in a straight line) with a DC 40 jump check. You may take 5-foot steps while hovering, and may choose to move vertically when you take a 5-foot step.


Mystic [Skill]
The workings of magic are no secret to you.
This is a skill feat that scales with your ranks in Knowledge (Arcana).

Benefits: You receive a +3 bonus on Knowledge (Arcana) checks. Knowledge (Arcana) is always a class skill for you.
4: With a DC 20 Knowledge (Arcana) check, you can cast detect magic as a spell-like ability. You may substitute your Knowledge (Arcana) check for the Spellcraft check to identify a school of magic based on its aura.
9: You gain spell resistance equal to 5 + your ranks in Knowledge (Arcana).
14: With a DC 30 Knowledge (Arcana) check, you can cast dispel magic as a spell-like ability.
19: Any weapon, armor, or shield you use is considered magical, and benefits from a +5 enhancement bonus at all times (unless they already have a better enhancement bonus).


Architect [Skill]
You understand the intricacies of artificial structures.
This is a skill feat that scales with your ranks in Knowledge (Architecture and Engineering).

Benefits: You receive a +3 bonus on Knowledge (Architecture and Engineering) checks. Knowledge (Architecture and Engineering) is always a class skill for you.
4: When within arm's reach of any artificial terrain (such as a worked floor, wall, pillar, etc.), you can take a move action to give yourself a +4 circumstance bonus in resisting bull rush, trip, hold down, and lift maneuvers until you leave your current space.
9: Your attacks ignore hardness less than your Knowledge (Architecture and Engineering) skill modifier.
14: When within arm's reach of any artificial terrain (such as a worked floor, wall, pillar, etc.), you can give yourself cover against all attackers for one round with a DC 30 Knowledge (Architecture and Engineering) check. This is a free action usable once per round.
19: You are expert at using terrain to your advantage. You may substitute your ranks in Knowledge (Architecture and Engineering) for your BAB for purposes of bull rush, grapple, and trip maneuvers.


Delver [Skill]
You spend a lot of time in dungeons. Certain habits inevitably result.
This is a skill feat that scales with your ranks in Knowledge (Dungeoneering).

Benefits: You receive a +3 bonus on Knowledge (Dungeoneering) checks. Knowledge (Dungeoneering) is always a class skill for you.
4: You receive a +3 dodge bonus to AC and saves against traps.
9: The first time you see a cursed item, you immediately know it is cursed if you pass a Knowledge (Dungeoneering) check with a DC of 10 + the item's caster level (this check is rolled secretly, of course). You may suppress the cursed effects of an item for one day with a Knowledge (Dungeoneering) check, DC 10 + item's caster level (if you fail, the curse activates as if you had attempted to use the item), but the curse reasserts itself if anyone but you attempts to use the item.
14: You automatically know if there is an armed trap within 300 feet of you, but do not know its nature or exact location.
19: You unerringly know the shortest navigable route to daylight at all times.


World Traveler [Skill]
The entire world is your home.
This is a skill feat that scales with your ranks in Knowledge (Geography).

Benefits: You receive a +3 bonus on Knowledge (Geography) checks. Knowledge (Geography) is always a class skill for you.
4: You and your traveling companions constantly benefit from an endure elements effect.
9: You receive a +10' competence bonus to your land speed.
14: Anyone attempting to use Survival to track you must beat you in an opposed check against your Knowledge (Geography).
19: With a DC 40 Knowledge (Geography) check, you can cast greater teleport as a spell-like ability.


Historian [Skill]
Your awareness of antiquated minutiae is staggering.
This is a skill feat that scales with your ranks in Knowledge (History).

Benefits: You receive a +3 bonus on Knowledge (History) checks. Knowledge (History) is always a class skill for you.
4: You gain the Bardic Knowledge ability of a Bard, using your character level instead of class level. If you already have Bardic Knowledge, you may roll it once for each source and take the best result.
9: Almost everything figures into history somehow. You may substitute your Knowledge (History) check for any other Knowledge check, but you suffer a -10 penalty when doing so (you do not count as untrained).
14: Once per day as a free action, you may attempt a DC 35 Knowledge (History) check to gain knowledge as if you had used legend lore.
19: Historical context and examples give you insight into many fields. You receive a +2 insight bonus on all skill checks. Yes, all of them. For simplicity, that includes Knowledge (History) checks.


Local Color [Skill]
You get friendly with the locals everywhere you go.
This is a skill feat that scales with your ranks in Knowledge (Local).

Benefits: You receive a +3 bonus on Knowledge (Local) checks. Knowledge (Local) is always a class skill for you. Additiionally, you are considered to have local knowledge for any area you have been in for any length of time. When you arrive in a new land, your ranks in Knowledge (Local) "catch up" at the rate of one per day as long as you are able to spend at least one hour per day listening to gossip.
4: Your understanding of local economy generally allows you to barter all goods for their full price, as if they were trade goods. However, there are still some items that absolutely no one in a region will want.
9: Your local contacts help you find services you need. You can get a local caster (if one exists) to cast a core cleric, druid, sorcerer, or wizard spell for you, but this takes one hour per caster level (plus the casting time of the spell), you must pay any component or XP costs, the caster level must be at least 5 less than your ranks in Knowledge (Local), and the spell must not entail any significant risk to the caster. At your DM's option, more exotic spells or services may be available.
14: With one day and a DC 30 Knowledge (Local) check in an appropriate milieu, you can make contacts with access to wishes or similar abilities and obtain the effects of one wish each day thereafter (while in that area), as long as the wish carries no XP cost and isn't likely to attract the wrath anything with a CR greater than your level - 5. No, you can't use this wish on the spur of the moment while out adventuring.
19: With one day and a DC 35 Knowledge (Local) check in a suitable milieu, you can convert planar currency into any other type of planar currency (your choice). For every 5 full points by which you exceed the DC, the time required is reduced by one step from one day to 4 hours to 2 hours to 1 hour to 30 minutes to 15 minutes.


Environmentalist [Skill]
You embrace the great outdoors.
This is a skill feat that scales with your ranks in Knowledge (Nature).

Benefits: You receive a +3 bonus on Knowledge (Nature) checks. Knowledge (Nature) is always a class skill for you.
4: While in a vegetated area, you can entangle an adjacent foe for one round with a move action and a Knowledge (Nature) check, DC 10 + target's STR + target's level or CR.
9: You can enhance your body's self-healing abilities. With a full-round action, you may perform a trance that restores HP equal to half your ranks in Knowledge (Nature). This action requires concentration and provokes attacks of opportunity.
14: You gain Scent.
19: With a DC 35 Knowledge (Nature) check, you can cast tree stride as a spell-like ability.


Feral Spirit [Skill]
You are at home only when you're not in a home.
This is a skill feat that scales with your ranks in Knowledge (Nature).

Benefits: Your natural armor is increased by 1.
4: You gain a natural bite, a sting, or two claws appropriate to a creature of your size as natural weapons.
9: You gain Scent.
14: You gain a +15' competence bonus to your land speed.
19: You gain DR 5/Wood and Fast Healing 1.


Regal [Skill]
Your studied grace gives you a regal demeanor.
This is a skill feat that scales with your ranks in Knowledge (Nobility).

Benefits: You receive a +3 bonus on Knowledge (Nobility) checks. Knowledge (Nobility) is always a class skill for you.
4: You seem above common conflict. You are protected by a constant sanctuary effect (save DC 8 + INT + half your ranks in Knowledge: Nobility). If this effect is ended for any reason (such as when you attack), you may renew it with a five-minute ritual.
9: Others obey you instinctively. With a DC 20 Knowledge (Nobility) check, you may cast command as a spell-like ability.
14: You never flinch. You are immune to fear effects.
19: With a DC 35 Knowledge (Nobility) check, you may cast greater command as a spell-like ability.


Undead Hunter [Skill]
You kill things that are already dead.
This is a skill feat that scales with your ranks in Knowledge (Religion).

Benefits: You receive a +3 bonus on Knowledge (Religion) checks. Knowledge (Religion) is always a class skill for you.
4: As a swift action, you may emit a flash that inflicts light damage equal to half your ranks in Knowledge (Religion) against undead and creatures specifically vulnerable to light within 10' (undead that are also specifically vulnerable to light, such as vampires, take double damage).
9: Your attacks can score critical hits against undead, and they are not immune to your sneak attacks. If you target an undead creature with an effect that it would be immune to due to the involvement of a Fortitude save, it merely receives a +5 bonus to the save instead. None of these effects apply to undead with a CR greater than your ranks in Knowledge (Religion).
14: You may substitute your Knowledge (Religion) check for Bluff, Hide, or Move Silently checks made against undead.
19: Any undead that you kill are permanently destroyed and cannot regenerate or be reanimated by any means short of DM fiat.


Profound Presence [Skill]
Just being near you is a religious experience.
This is a skill feat that scales with your ranks in Knowledge (Religion).

Benefits: You gain a +1 bonus on all charisma-based skill checks.
4: Adjacent allies gain a +1 morale bonus to attack and damage (you do not).
9: As a swift action, you may emit a searing aura that inflicts 2d6 light damage against all creatures within 10 feet of you (except yourself).
14: Once per round as a free action, you may designate a target within 60 feet. If the targeted creature meets your gaze within the next round, it must make a Will save (DC 8 + CHA + half your ranks in Knowledge (Religion)) or be dazed for one round.
19: You gain heal 1/day as a quickened spell-like ability.


Cosmologist [Skill]
You understand the big picture.
This is a skill feat that scales with your ranks in Knowledge (The Planes).

Benefits: You receive a +3 bonus on Knowledge (The Planes) checks. Knowledge (The Planes) is always a class skill for you.
4: As a swift action, you may grant yourself energy resistance to one energy type equal to half your ranks in Knowledge (The Planes), or damage resistance equal to half your ranks that only applies to attacks with an alignment of your choice, including the attacks of creatures with that alignment subtype. This effect lasts for one round.
9: Your attacks count as having any alignment you wish for purposes of penetrating DR. You may change this alignment as a free action.
14: With a DC 30 Knowledge (The Planes) check, you can cast dismissal as a spell-like ability, with a caster level equal to your character level. The save DC is INT-based.
19: With a DC 40 Knowledge (The Planes) check, you can cast plane shift as a spell-like ability.


Scholar [Skill]
Knowledge is power.
This is a skill feat that scales with your ranks in your highest Knowledge skill.

Benefits: If your knowledge check to identify an opponent gives a result of at least 15 + the foe's CR, you have The Edge against that opponent until its CR increases.
4: You may take 10 on knowledge skill checks.
9: You are adept at identifying monsters' weaknesses. With a swift action and an appropriate knowledge check (DC 15 + creature's CR), for 5 rounds, you gain either a +2 insight bonus to attack and damage against a creature or a +2 insight bonus to the save DC of any save you force it to make (your choice which). You may not take 10 on this check.
14: When you make a successful knowledge check using the 9-rank ability of this feat, your attacks also ignore the target creature's DR and hardness for 5 rounds.
19: All your attacks have a doubled critical threat range.

Elemental Mastery [Skill]
Some people strive for wealth, knowledge, or power. Others just want to see the world burn.
This is a Skill feat that scales with your ranks in Knowledge (Arcana).

Benefits: Choose one energy type (acid, cold, electricity, fire, or sonic). You may treat any spell you cast as if it had Energy Substitution (using that type) applied to it, even if you did not prepare it that way.
4: Whenever you inflict energy damage of your selected type, the affected creature must make a Reflex Save (DC 10 + CHA + half your level) or suffer an appropriate affliction. Acid: sickened for 2 rounds; Cold: slowed for 1 round; Electricity: knocked down; Fire: set on fire for 1 round (as Alchemist's Fire); Sonic: deafened for 1 minute. If the effect that inflicted the damage was a spell or already required a save, you can use your key spellcasting ability or the ability used for the save DC instead of Charisma.
9: You have Resistance 10 against your chosen energy type.
14: When you inflict energy damage of your selected type, you ignore the first 5 points of the target's energy resistance, and the damage cannot be reduced to less than half by energy resistance or immunity.
19: You are immune to your chosen energy type.
Special: This feat may be selected multiple times. Its effects do not stack; instead, each instance of the feat applies to a different energy type.

Cerebral Assassin [Skill: Bluff]
You out-think your foes, aiming to win before battle starts, and fooling your foes into walking into your expertly laid traps.
(You hit people. In the face. With your mind.)
0: As a Swift action, you may call an opponent's attention towards you. This brief distraction grants all allies a +1 bonus to strike them until your next turn, and +1d6 damage.
4: ranks in Bluff: Anyone who misses you in melee by 5 or more points suffers damage equal to 1d6 + your Intelligence modifier.
9: ranks in Bluff: The bonuses improve to +2 and +2d6 respectively. Additionally, if you make a trap, the DC to spot and disarm is equal to the Craft (Trap) check result. Yeah, it's that hard. The trap deals extra damage (if a damage-dealing trap) equal to your Int modifier.
14: ranks in Bluff: The bonuses improve to +3 and +3d6. Anyone who misses you in melee combat takes damage equal to your Int modifier. If they miss by 5 or more, they fall prone in the adjacent square of your choice. Whenever you attack a flat-footed foe, they must make a Will save (DC 10 + half your HD + your Int modifier) or be Dazed or Blinded for one round.
19: ranks in Bluff: As a standard action, you may perform a low blow against a target. As part of this, you may make a feint attempt. If it succeeds, your attack is against their flat-footed AC as normal. Additionally, if the feint succeeds, your attack (if successful) deals additional damage equal to your Bluff modifier and renders them prone. They also are Stunned for one round, and lose access to one limb of your choice.
You may also quickly throw or spit a spray of sand, grit or the like to blind foes. This is a move action, and affects all in a 10' cone (Ref negates, DC 10 + half your HD + your Cha modifier) or one target of a ranged touch attack. If they are hit, they are blinded for 3 rounds.

Deific Attention [Skill: Knowledge (Religion)]
You have a friend in high places.
Prerequisite: Worship a specific deity that you don't tend to piss off.
0: Knowledge (Religion) is always a class skill for you. Additionally, once per day you may cast a single 1st level spell as a spell-like ability. This spell must come from a domain the deity possesses, and if it has any expensive material components or an XP cost, you must still pay it.
4: You gain one Domain power that your deity can grant, using your character level for anything that relies on cleric levels.
9: Your deity can sometimes intervene and save your ass. Once per day, if an attack would reduce you to 0 HP or less, you can elect for your deity to step in and halve the damage. You are still knocked prone, so it looks like you were killed, in case you want to play dead and not attract the wrong kind of attention
14: You now gain an additional spell-like ability once per day. It must also come from a domain list your deity offers, and has the same exception on costs, but it may be of any level up to and including 6th.
19: Once per day, you may smite the enemies of your deity, given some half-assed justification. This grants you a +4 bonus to hit, and deals an additional amount of Holy or Profane damage equal to 3d6 plus your character level plus your Charisma modifier. If this slays the enemy of your deity, they are so pleased that you may, at any point before the end of the day, activate a Heal spell upon yourself.

Ki Shot [Skill: Concentration]
You belong in a series that has a Z on the end of it.
0: You have a supernatural ability to shoot energy out of your empty hand as an attack action that deals 1d6 damage. It has a range increment of 60 feet (1d6/x3, unaffected by your size). It may only travel as far as 3 range increments before it dissipates harmlessly.
4: You may improve one of your ki shot's attributes by one step (damage 1d8, 19-20 threat range, or x4 critical). You also gain a touch attack as a supernatural ability which does 1d8 damage, useable as a standard action. Both your ranged attack and your touch deal additional damage equal to half of your ranks in Concentration.
9: Your ki shot is now treated as though it were a force effect and it deals Force damage. It can now travel out to 10 range increments before dissipating.
14: Your range increments with your ki shot double, and one of its attributes increases by a step (your touch attack likewise increases in damage). Both it and your touch attack now deal 1 point of additional damage for each rank of concentration you have. This replaces the similar benefit gained at rank 4.
19: Your ki shot can strike targets with touch attacks. Two of its attributes improve by a step. Your ki shot now travels until it hits a creature or object, giving you unlimited range increments (You still take distance penalties).

Lifecrafter (Necromantic) (Skill: Craft)
Requirements: Craft 4 ranks (Alchemy, Embalming, Weaponmaking).
Benefit: Thanks to a study of the alchemy of life (I can�t believe I just wrote that) and necromancy, you have begun to cross into realms considered blasphemous by most religions. With the right materials and enough time, you can grow obscenely organic living weapons. Growing a weapon require a framework of some kind�it�s usually broken and re-shaped bone�and a growth pool of common alchemic substances and liquid viscera.

The weapons themselves vary widely in appearance. The most simple ones appear to be a dull bone-like substance, maybe with a slightly irregular shape. The more complex ones are downright obscene collections of muscle and other tissue and can make people retch.

The base growth time and the number of gallons of an approximately fifty-fifty mix of alchemic materials and liquid viscera are needed are as follows:

Diminutive object: 3 days, a half-gallon.

Tiny: 6 days, a gallon

Small: 12 days, four gallons

Medium: 20 days, ten gallons.

Almost all Large or larger living weapons are �stillborn� and useless.

A living weapon is always modeled after a normal weapon, but it�s usually only about half of the weight. Use its base stats.

For the improvements granted by this feat, take the base growth time, then add up all the time increases and multiply (be sure to convert the percentages to decimals, so, for example, 150% becomes 1.50)

+4: Your living weapons are slightly better than the normal things. When you�re growing one, you may add these qualities at the cost of extra time spent growing them.
-Increased critical range by 1 / Increase critical multiplier by 1: +20 percent time (round up to the nearest day) for each one of these, maximum of +80%
-Magic weapon: +100% time.
-Bonus damage: It has ripping spines or fins or whatever bullshit you want. Point it, is starts off doing an additional +1d4 damage at the cost of +20% time, can be stacked up to 3 times.

+9: Your growing skill lets you make more complex living weapons. You have the following options:
-The weapon can flex and squirm and does so constantly, giving it a +2 bonus on Trip and Disarm attempts and +Eleventy Billion on checks to freak people out. +40% time and lots of muscle tissue.
-The weapon generates its own poison and does 1d6 damage to an ability score chosen at the time of the weapon�s birth when it hits someone. It can generate one dose every minute. +40% time, 1 dose of a poison that targets the appropriate ability score.
-The weapon can get its own attack, such as a bite or a claw or something. It does damage appropriate to its size and hits when you hit someone.

+14: By now, you're not making weapons, you're making creatures.
-A flexible weapon can have its own Str and Dex scores which are added to relevant attack/damage rolls like yours are. Start off at 10 and increase one by +2 for each iteration of this, maximum of 14 in each. +40% time, and muscles and nerves.
-A weapon can have mental scores. Start off at 1, and increase one score by 1 for each iteration of this. Each point costs +10% time, and 18 is the limit. You also need a lot of brains to go into the liquid viscera. These weapons have egos just like convention sentient magic weapons do.
-A wapon may have its own sense organs, although it needs mental scores to really work them. Its ranks in a sense skill is half of its score in the relevant ability. +10% per skill (Spot, Listen, Search).
-A weapon may have some way to communicate, either vocally or telepathically. +10% for vocal communication and +20% for telepathic. You need either a jaw and some vocal cords or the brain of a telepathic creature. Non-sentient telepathic weapons obey their wielder's will completely, and will use any of their abilities as an extension of the wielder.
-A weapon may have a spell-like ability, which it can either get from its creator or through having an appropriately-enchanted item placed in it at creation. At its first iteration, it gets the abilities once per day. Second iteration gives it the ability 3/day. Third iteration gives it the ability at-will. +50% for each iteration of each spell.
-The weapon gets a ranged touch attack with a range of 30 feet. This touch attack may inflict temporary ailments such as nausea, blindness, paralysis, sleep, pain (treat like a Symbol of Pain effect on one person), pretty much anything along those lines. It lasts for 2d4 rounds.

+19: Your children, they are powerful.
-A sentient flexible weapon may make its own attacks, including BAB, at a BAB equal to the creator's ranks in Craft-3. +50% time
-All of the +4 or +9 abilities have their caps doubled.
-A weapon's natural attack does damage as if it were one size larger. +20%, can be taken taken twice.


Mental Weaponry [Skill: Concentration]
You can make weapons with your mind. Handy at the Queen�s Ball.
0: As an free or immediate action, you may create any melee or thrown weapon(s) that you are proficient with in your main hand and/or your off hand(s).
This weapon(s) has all the attributes of the original and deal damage as if it were a normal weapon of its type. You may dismiss these weapon(s) as a free action. If any weapon ever leaves your grasp, it dissipates. It has a hardness and hit point value equal to the normal weapon that you are creating.
Concentration also becomes a class skill for you at all levels.
4: You can make ranged attacks with your Mental Weapons. If the weapon you wish to attack with does not normally have a range increment, it now has a range increment of 10 feet. The Mental Weapons never actually leaves your hand, you simply can make ranged attacks with it instead of melee attacks. In addition, your Mental Weapons deal 1 point of damage per 2 ranks you have in the Concentration Skill.
9: Your Mental Weaponry is treated as though it was a Force effect and (surprise) it deals Force damage, which means that it bypasses DR and Hardness, and can affect incorporeal targets.
14: The range increments of your Mental Weaponry double. Additionally, your Mental Weaponry deals extra damage per hit equal to your ranks in Concentration; this replaces the similar benefit gained at rank 4.
19: Attacks made with your Mental Weaponry are resolved as touch attacks.

Special: A Samurai may use one set of Mental Weapons as his ancestral weapon, as long as its form is exalted by the Samurai's warrior culture. The Samurai may select more than one weapon to qualify; however all weapons thus chosen must be weapons that he can wield at the same time. Thus a Daisho (Katana and Wakisashi) is permissible for most creatures that can use two weapons. A Glaive and a Greatsword are not, unless the creature can wield both at the same time.

Painmonger [Skill: Concentration]
Being hurt doesn't really affect you.

0: You gain Concentration as a class skill and can make Concentration checks using one of your mental ability scores [the mental ability score is selected when this feat is selected and cannot be changed without DM approval]

4: Scar Tissue: Scarred tissue is much tougher than normal tissue. Broken bones heal more solidly and stronger than unbroken bones. You gain DR X/-, where X is equal to 1/2 your ranks in Concentration. You may choose to have this ability active or not on any attack that you sustain, but must decide before the damage is determined.

Pain is nothing: Being cut, mangled, bruised, charred or frozen doesn't just harm you, it focuses your mind and actions with terrible resolve
If you do not use your Scar Tissue ability and take damage from a Physical attack or take damage from any non-Physical source, you gain +1 to all of your Attack, Caster level, Damage, Skill check and Saving Throw dice rolls for the next minute. This ability can be used more than once and the bonus increases each time you use it, your maximum bonus is equal to one half your ranks in Concentration.

9: Touch of pain: You've experienced enough pain to know how to harm others without weapons. With just a touch you can jab at nerves, punch organs close to the skin and threaten weak points such as eyes, ears or noses.
You can make touch attacks that injure. As an attack action, you may make a touch attack that deals damage equal to a concentration check. This damage may be lethal or subdual damage and bypasses no Damage Reduction unless your natural attacks can already do so. This ability only affects creatures that can feel pain.

Painful Strikes: You know how strike your enemies to really hurt them with your weapon attacks.
All of your attacks against creatures that can feel pain may deal an additional amount of damage up to your ranks in Concentration. This bonus damage cannot be multiplied or increased in any manner.

14: Internal Scars: The mental damage and traumatic events that you have been victim to, been a part of or witnessed no longer harm you the same as they would affect others.
You can make Concentration checks in place of Saving Throws against effects that would instantly kill you.

The DC of the skill check is equal to the the Saving Throw DC required to survive. If you fail the check, you instead take one point of damage per caster level of the effect that you were trying to save against (or Hit Dice of the creature that was using the ability against you) and 1d4 points of damage to one ability score (chosen by you each time this occurs).

You are immune to fear effects and enchantments.

19: Masochism: You know that pain focuses you. Now you deal it to yourself in order to be at your peak whenever you choose.
You can attack yourself up to your number of attacks per round. Attacking yourself in this manner is a non-action that provokes attacks of opportunity per attack that you perform on yourself. Your bonus to dice rolls granted by your Pain is nothing ability is equal to the damage that you take on any attack, with the same limitations to the size of the bonus.
Note: You may always choose to deal minimal damage (1 damage) on any attack that you do. This rule does not usually get mentioned because usually you want to deal the most damage possible.

Sky Scraper [Skill: Jump]
You jump high. Reeeeeeeeeeal high.
0+: You have a +3 bonus to Jump checks.
4+: When jumping horizontally, you do not gain height as part of the movement; You jump in a straight line.
9+: You are always considered to have made a running start when making a Jump check.
14+: Anytime in the middle of a jump, you may stay still on that spot. Until you wish to fall, or a foe move you (Through a bull rush or trip attempt), you are considered on solid ground.
19+: When jumping, you are considered to be flying with an average maneuverability, at your jump check's distance for speed.

Spirit Pressure [Skill: Concentration]
Your focus and mental sturdiness makes your foes quake in fear, feel dizziness, and is potentially harmful to pregnant women.
0: Any time you would make an Intimidate check, you may instead make a Concentration check and use that result.
4: When you initiate a Duel of Wills (Tome of Battle, page 27) or try to intimidate a foe, the penalty your adversary suffers from a failure is augmented by one per four ranks you have in the Concentration skill.
9: You may now Intimidate all opponents adjacent to you in a single action. You only roll once, and all opponents resist your attempt individually.
You may Intimidate in combat as a move action, and may spend a move action each round after having successfully Intimidated a foe to extend his condition for one more round.
14: You may Intimidate in combat anyone to which you have Line of Sight, in a single action.
You may Intimidate in combat as a swift action, and may spend a swift action each round after having successfully Intimidated a foe to extend his condition for one more round.
19: Any opponent you successfully Intimidate is also Staggered.
