\documentclass[10pt]{report}

\usepackage{appendix}
\usepackage{floatflt}
\usepackage{fancyhdr}
\usepackage{textcomp}
\usepackage[usenames]{color}
\usepackage{isoent}    %\sfrac
%\usepackage{palatino} %font
\usepackage{newcent}   %font
\usepackage{sectsty} %custom section headings

\newcommand{\normalsections}{
\sectionfont{\noindent\rule{\textwidth}{0.015in}\\\nohang}
\subsectionfont{\noindent\rule{\textwidth}{0.005in}\\\nohang}
}

\newcommand{\columnsections}{
\sectionfont{\vspace*{-20pt}\noindent\rule{3.5in}{0.015in}\\\nohang}
\subsectionfont{\vspace*{-20pt}\noindent\rule{3.5in}{0.005in}\\\nohang}
}


\sectionfont{\noindent\rule{\textwidth}{0.015in}\\\nohang}
\subsectionfont{\noindent\rule{3.5in}{0.005in}\\\nohang}

\usepackage[Bjarne]{fncychap} %Canned chapter headings


\usepackage{multicol}
\usepackage[bookmarks=true,colorlinks,linkcolor=cyan,breaklinks]{hyperref}



%%%%Margins%%%
\topmargin 0pt
\advance \topmargin by -\headheight
\advance \topmargin by -\headsep
\textheight 8.9in
\oddsidemargin -0.25in
\evensidemargin \oddsidemargin
\textwidth 7in
\oddsidemargin -0.25in
%\setlength {\parindent} {0pt}



%%%Formatting%%%
\newcommand{\ability}[2]{\smallskip \noindent \textbf{#1} #2}
\newcommand{\shortability}[2]{\noindent\textbf{#1} #2\\}
\newcommand{\bolded}[1]{\noindent\textbf{#1}}
\newcommand{\itemability}[2]{\item \textbf{#1} #2}
\newcommand{\featname}[1]{\vspace*{0.1cm plus 0.2cm minus 0.05cm}\noindent\textbf{#1}\\}
\newcommand{\featnamelist}[1]{\vspace*{0.1cm plus 0.2cm minus 0.05cm}\noindent\textbf{#1}}

\newcommand{\descfeat}[2]{\featname{#1}\emph{#2}\\}

\newcommand{\business}[6]{
	\vspace{2pt}
	\noindent\hspace*{1cm}\textbf{#1}\\
	\noindent\hspace*{1cm}\textbf{Primary Skill:} #2\\
	\noindent\hspace*{1cm}\textbf{Secondary Skills:} #3\\
	\noindent\hspace*{1cm}\textbf{Capital:} #4\\
	\noindent\hspace*{1cm}\textbf{Resources:} #5\\
	\noindent\hspace*{1cm}\textbf{Risk:} #6
	\vspace{2pt}
}
\newcommand{\classname}[1]{\subsection{#1}}
\newcommand{\condition}[1]{\emph{#1}}
\newcommand{\quot}[1]{\emph{#1}\medskip}
\newcommand{\desc}[1]{#1 \medskip}
\newcommand{\example}[1]{\emph{#1}}
\newcommand{\magicitem}[1]{\emph{#1}}
\newcommand{\monster}[1]{\subsection{#1} \label{monster:#1}}
\newcommand{\monsterline}[2]{\textbf{#1:} #2\\}
\newcommand{\monstersizetype}[2]{\textbf{#1 #2}\\}
\newcommand{\spell}[1]{\emph{#1}}
\newcommand{\spelllist}[1]{\smallskip \noindent \underline{\textbf{#1}}}

% For the feats -- moved here from feats.tex because we added another 
% section(s) for feats. -Surgo
\newcommand{\minitabular}[1]{\begin{tabular}{p{0.25in}p{2.9in}} #1\\ \end{tabular}}
\newcommand{\babfeat}[7]{
	\noindent\minitabular{\multicolumn{2}{l}{\parbox{3in}{\textbf{#1}}}}
	\minitabular{\multicolumn{2}{l}{\parbox{3in}{\small #2}}}
	\minitabular{\raggedleft\textbf{\small \textbf{+0:}}& {\small #3}}
	\minitabular{\raggedleft\textbf{\small +1:} & {\small #4}}
	\minitabular{\raggedleft\textbf{\small +6:} & {\small #5}}
	\minitabular{\raggedleft\textbf{\small +11:} & {\small #6}}
	\minitabular{\raggedleft\textbf{\small +16:} &{\small #7}}
}
\newcommand{\skillfeat}[8]{
	\noindent\minitabular{\multicolumn{2}{l}{\parbox{3in}{\textbf{#1}}}}
	\minitabular{\multicolumn{2}{l}{\parbox{3in}{\small #2}}}
	\minitabular{\multicolumn{2}{l}{\parbox{3in}{\small\textbf{#3}}}}
	\minitabular{\raggedleft\textbf{\small \textbf{0:}} &{\small #4}}
	\minitabular{\raggedleft\textbf{\small 4:} &{\small #5}}
	\minitabular{\raggedleft\textbf{\small 9:} &{\small #6}}
	\minitabular{\raggedleft\textbf{\small 14:} &{\small #7}}
	\minitabular{\raggedleft\textbf{\small 19:} &{\small #8}}
}

\newcommand{\boxit}[1]{\frame{\parbox{\textwidth}{#1}}}

% A new box command -- the first argument is the box's header, the second the 
% contents of the box. This looks a lot better than \boxit. -Surgo
\newcommand{\abox}[2]{\vspace{5pt}
	\fbox{\begin{minipage}{0.8\linewidth}
	\setlength{\parindent}{0.15in}
	\textbf{#1}

	#2
\end{minipage}}
\vspace{5pt}}

\newcommand{\itemspace}{\setlength{\itemsep}{-1mm}\setlength{\topsep}{-1mm} }

\newcommand{\listone}{\begin{list}{$\bullet$}{\itemspace}}
\newcommand{\listprereq}{\begin{list}{\vspace*{-2pt}}{\itemspace}}
\newcommand{\listtwo}{\begin{list}{$\triangleright$}{\itemspace}}
\newcommand{\listthree}{\begin{list}{--}{\itemspace}}

% Bold the first argument in the listitem
\newcommand{\bolditem}[2]{\item \textbf{#1} #2}

\newcommand{\slashfrac}[2]{${}^{#1}$\hspace*{-1pt}\makebox[2pt]{$\diagup$}${}_{#2}$}
%\newcommand{\half}[0]{\slashfrac{1}{2}}
\newcommand{\half}[0]{\ensuremath{\sfrac{1}{2}} }
\newcommand{\third}[0]{\ensuremath{\sfrac{1}{3}} }
\newcommand{\fourth}[0]{\ensuremath{\sfrac{1}{4}} }
%\newcommand{\half}[0]{\ensuremath{\scriptscriptstyle 1/2}}
%\newcommand{\threefourths}[0]{\ensuremath{\sfrac{3}{4}}}


%\rule{\textwidth}{0.005in}

%\renewcommand{\sectionmark}[1]{\markright{\thesection\ \boldmath\emph{#1}\unboldmath}\\\rule{\textwidth}{0.005in}\\}

\setcounter{tocdepth}{1}

\begin{document}

\pagestyle{plain}
%Cover Page

\begin{center} \Huge

\textsc{Races of War}\end{center}



\vspace{2cm}
\begin{center}\large By Frank Trollman \& K\end{center}


\newpage

\vspace*{4in}

\noindent Please address all complaints and comments about balance to the authors at\\
{\color{blue} \href{http://tgdmb.com/viewforum.php?f=1}{http://tgdmb.com/viewforum.php?f=1}}

\vspace{0.2in}

%\noindent For a hypertext version of some of this information, especially classes, please look at\\
%{\color{blue} \href{http://www.d20ragon.com/frank/}{http://www.d20ragon.com/frank/}}

%\vspace{0.2in}

\noindent Amateur Typesetting by Joshua Middendorf, updated by Morgon ``Surgo'' Kanter, ``Aktariel'', and Stephen ``Quantumboost'' Smith.

\vspace{0.15in}

\noindent Please address all comments regarding the quality (or lack thereof) of the typesetting (that is, formatting of the pdf) to Joshua Middendorf (\href{mailto:middendorfproject@gmail.com}{middendorfproject@gmail.com}), Morgon Kanter (\href{mailto:morgon.kanter@gmail.com}{morgon.kanter@gmail.com}), Aktariel (\href{mailto:aktariel@gmail.com}{aktariel@gmail.com}), or simply comment in the above forum.


%\vspace{0.2in}

%\emph{Enjoy!}


\vspace{1in}
\noindent Published on \today, version 0.6\\
\noindent You may find the most recent version of this document at:\\
{\color{blue} \href{http://www.tgdmb.com/viewtopic.php?t=36046}{http://www.tgdmb.com/viewtopic.php?t=36046}}

\newpage


\pagestyle{fancy}
%Fix the spacing so it's all reasonable
\linespread{.9}  \small  \normalsize \itemspace \normalsections

\tableofcontents

\chapter{Morality and Fiends}

\section{Alignment}

Alignment has historically been the source of significant disagreement and headaches. It is possible to play games without using it at all, and plenty of people are perfectly happy doing just that. If you're going to use alignment though, it's a good idea to consider it's ramifications and to get the whole group on the same page.

\subsection{Good and Evil: How Black is the Night?}

The central moral question surrounding fiends is not particulary obvious, but in no way unimportant to your game. We \textit{know} that a Gelugon is Evil, he's got a \textit{subtype} that denotes him as being specifically Evil, that's not the question. What we don't know is \textit{how} Evil he is. That's a central question that has to be addressed within the context of each game. Let's face it, a lot of people really aren't comfortable with villainy more pernicious than the antagonists in a Saturday morning cartoon. Other people have a different and equally valid hang-up: they aren't comfortable having their characters stab enemies in the face repeatedly until they bleed to death unless those enemies are \textit{extremely bad people}. As so frequently happens, the rules for Dungeons and Dragons are written to accommodate both play styles, which in reality ends up including \textit{nothing}. Perhaps unfortunately, you \textit{must} come to a table-wide consensus about what your gaming is not doing before you can have your game do anything at all.  Keep in mind that none of these play styles are ``worse" or ``better".\\

\subsubsection{Moral Option 1: A Worthy Opponent}
\vspace*{-8pt}
\quot{``Fools! You have interfered with my plans for the last time!''}

For many games, the fact that the bad guys are bad is pretty much sufficient. Like the villains in Saturday Morning Cartoons, their villainy requires -- and gets -- no explanation. Actual villainy is fairly upsetting to contemplate, and a lot of people don't want to do it. I don't blame them, cannibalism, deliberate infliction of pain, and exploitation of the innocent are \textit{unpleasant}. Talking about secret prisons where torture is conducted night and day without respite or reason is \textit{super depressing}.

\ability{Implications:}{ The biggest implication here is that since Evil and Good are basically just political parties or ethnic hats, it is perfectly OK to have mixed alignment parties \textit{or} to ban mixed alignment parties. You're never going to have a serious discussion about what it is that Evil people do, so it's actually not important how you handle them. You can even switch how you're handling it in the middle for no reason. One day, the Atomic Skull can just chip in to save the world from Darkseid. Another day you can go kill the Atomic Skull without feeling bad. It's very liberating, because you can tell a lot of stories -- so long as none of those stories involve actual evil actions happening on camera.}

\ability{Pit Falls:}{ While it is certainly a load off the mind to not be constantly reminded of child abuse, torture, and sexual misconduct, bear in mind that this is Dungeons and Dragons -- your foes are more than likely going to be killed with extreme stabination. Possibly in the face. Possibly more than once. If the villains \textit{aren't} doing anything overwhelmingly bad, it's entirely possible that it won't seem like they \textit{deserve} being killed. If subjected to enough analysis, one might even find that your own ``heroes" appear to be the villains in your cooperative storytelling adventure. Certainly, He-Man never took that sword and chopped Skeletor into chunks. Star Wars: Episode One was such an unsatisfying movie in no small part because the villains never did anything bad.}

Glossing over the villainous activities of the bad guys should go hand in hand with all of the players acknowledging and understanding what you are doing and why you are doing it. As long as everyone is making the active and informed choice to not deal with the heavy moral questions -- it's all good.

\subsubsection{Moral Option 2: The Banality of Evil}
\vspace*{-8pt}
\quot{``It's 9 o'clock, time to get back to some Evil.''}

Many DMs will want to play their fiends pretty much like Nazis -- their agenda is hateful, but in their off time they go hang out at the pub just like everyone else. You could even sit there with them and drink together unless you happen to be a Jew. This is the default assumption of a lot of Planescape literature, for example. An Evil creature is Evil because it \textit{ever} does Evil things, not because it's necessarily doing any Evil \textit{right now}. Darkness and light are, in this model, pretty ephemeral concepts -- characters who wish to save their own sanity will end up either paying perhaps too much attention or ignoring them completely often as not.

\ability{Implications:}{ Since bad guys (and presumably good guys as well) spend most of their time being regular guys and only infrequently perform acts worthy of praise or scorn, it's extremely easy for heroes to fall to Evil and extremely easy for villains to be redeemed for full value. People on both sides of the Good/Evil axis are doing pretty unexceptional stuff most of the time, so the allegiance that even Evil Clerics have to darkness is pretty tenuous.}

This way of handling things is so much better at handling mysteries than are other morality systems that it may as well be a requirement if you ever want to play a ``who-done-it" adventure. Since the good guys and bad guys spend most of their day being actually indistinguishable one from another, it makes distinguishing them actually difficult -- and that has to happen if there is to be any question of who the PCs are supposed to stab.

\ability{Pit Falls:}{ Be wary of over-humanizing the villains. In many stories, the bad guys are a lot more interesting than the white hats; and that can seriously derail a campaign if it happens in a role playing scenario. Beware also of the fact that if the Evil Overlord is mostly chillin' like a villain with his family and having brews with his bros, it's going to be pretty hard to justify it when you inevitably stab him right in the face. Also remember that while The Banality of Evil is great for mysteries, it's actually so good for mysteries that the game can bog down. Players can get caught up in the minor goings-on of characters you don't even care about. Paranoia can be paralyzing when any scullery maid could really just go Evil at any time and poison your food to try to get your wallet. It can be realistic, but realism takes place in real time. That's not good if you're trying to raise hippogriffs as steeds.}

\subsubsection{Moral Option 3: The Face of Horror.}
\vspace*{-8pt}
\quot{``I think I have Evil sand. In my pants.''}

Many DMs will want to make their Evil as \textbf{Evil} as possible. That can get\ldots\  pretty Evil. It can actually get so Evil that people who \textit{overhear} you playing the game will get a very bad impression about your group and the things you talk about. The starker the contrast between Good and Evil, the more righteous the acts of heroism the players commit. Tales of monstrous action are fascinating and the horrid and disgusting can hold people's interest indefinitely. By having the forces of Evil disembowel people in loving detail you can capture the imaginations of your players with actually relatively little creative work on the part of the DM. There have been over 10 Jason movies because those things practically write themselves, and people keep watching them because they genuinely are as intriguing as the are revolting.

\ability{Implications:}{ With the forces of Evil running around doing actual stomach churning crime, having Evil and Good ``team up" is essentially implausible. In fact, having Good and Evil characters in the same party is pretty much a non-starter. When playing with The Face of Horror the universe is essentially a cosmic battle between Good and Evil, the forces of Law and Chaos have some fights too, but essentially that's just crime compared to the world shaking conflict of darkness and light.}

Further, while Good and Evil being as immiscible as Rubidium and Water makes for a well defined party demographic, it also has other far reaching consequences. When you go to the Abyss, the sand itself is Evil. Once you've made the determination that this means more than that Paladins can find every grain -- you've bought yourself into the determination that beaches in the Abyss are themselves morally reprobate somehow.

\ability{Pit Falls:}{ While The Face of Horror ends up making Good and Evil a much more important distinction than Law vs. Chaos, that's not really a \textit{problem}. Sure, it's not reciprocal or equivalent and that's a breach of the Great Wheel tirade, but that's not really important to the game. Let's face it, when was the last time you saw a statted up enemy prepared to cast \spell{dictum}? No, the problem is that if you make Evil as nasty as it can be made, it's really nasty. It makes other people in the game uncomfortable, and it disturbs people who hear portions of your game out of context. People like talking about stabbing their sword into an evil monster, that's all heroic and crap, but actually looking at sword wounds is icky. People don't want to do it.}

Evil, if defined as ``things we don't like," is pretty much exclusively composed of things we \textit{don't like}. That means that the more we focus our attention on the details of what's going on, the more we'll want to clean our eyes out with soap. And while skirting that line can make a story grimly compelling, remember always that different people have different tolerances for this sort of thing. Just because something is gross enough to catch your prurient interest without wrecking your lunch doesn't mean that it isn't so nasty as to drive other people away. Tolerance for discussing child murder in the context of a story is not a virtue, and there is absolutely nothing wrong with the people who don't enjoy watching movies in the splatter horror genre.

\subsubsection{Moral Option 4: Perfection in Balance}
\vspace*{-8pt}
\quot{``What use is the light that casts not a shadow?''}

In this model, evil is a force that sits diametrically opposed to good. In order for one to exist, the other must exist as well. Evil is what gives good its meaning, and in fact one can simply define one by the other: to be good is not-evil, and to be evil is to be not-good. When playing with this option, evil plays a vital role in society and cannot be eliminated without dire consequences. For example, when the Jedi eliminate the Sith Lords, they set themselves up for an even more powerful Sith Lord to rise and kill them all, ushering in a new order of Evil, which is in turn later demolished by the calling out of a powerful Jedi who can defeat it. Neutrality is the rule of the day in this model, in the sense that evil and good will always be in the midst of trumping each other in an effort to ``win'', a goal that is as meaningless as it is impossible.

What does that mean for your game? In this model, evil will always be the fly in your ointment and the piss in your cheerios, and good will always be the silver lining in the stormcloud and the complementary bag of nuts in your red-eye flight. Even the most powerful and good organization of clerics in your world will have a cruel inquisitor, and even the most death-hungry cabal of necromancers will have a guy who is kind to puppies and little children. Organizations and people will be ``mostly'' one thing or the other, but not all of anything, and people will be OK with that. Kind kings will be mostly good, but will have no problem massacring an entire generation of goblinkind in an effort to keep the roads safe, and liches who eat souls will defend the land from rampaging chimera without reward in an effort to keep the peace.

\ability{Implications:}{ In a sense, this is the easiest of moral options, as you won't need to really keep track of what's going on with alignments. People will occasionally do things out of character, and that's fine. Society will be quite tolerant, as they completely think its OK for there to be a Temple Street with a shrine for Orcus worshippers competing for space with a hospital sponsored by the clergy of Pelor. When one organization for good or evil gets stomped down, another one will pop up to replace it in an endless game of cosmic whack-a-mole.}

For character with alignment related class features, \spell{atonement} is a far easier process. Occasional deeds that violate your alignment are tolerated, as long as attempts at acts of \spell{atonement} are made in a reasonable time frame. The Paladin that kills an innocent to defeat a powerful demon may have to visit the innocent's family and make restitution after the battle, and the Cleric of Murder who defends the king from an assassin may have to seek out several of the King's loved one's in order to rededicate himself to his dark god.

\ability{Pit Falls:}{ It can be pretty cool to have a party that has an assassin, a druid, and a champion of light in it -- there's a lot of early D\&D that has that as virtually the iconic party -- but if the great game between Good and Evil is an inherently \textit{pointless} game, that can make the story of your characters seem pretty banal. It's a line that can be hard to walk. It's just plain difficult to simultaneously have any individual attempt to destroy the world be important while having it be built into the contract that there will be another one tomorrow.}



\subsection{Law and Chaos: Your Rules or Mine?}

Let's get this out in the open: Law and Chaos do not have any meaning under the standard D\&D rules.\\

We are aware that especially if you've been playing this game for a long time, you personally probably have an understanding of what you \textit{think} Law and Chaos are supposed to mean. You possibly even believe that the rest of your group thinks that Law and Chaos mean the same thing you do. But you're probably wrong. The nature of Law and Chaos is the source of more arguments among D\&D players (veteran and novice alike) than any other facet of the game. More than attacks of opportunities, more than weapon sizing, more even than spell effect inheritance. And the reason is because the ``definition" of Law and Chaos in the Player's Handbook is written so confusingly that the terms are not even mutually exclusive. Look it up, this is a written document, so it's perfectly acceptable for you to stop reading at this time, flip open the Player's Handbook, and start reading the alignment descriptions. The Tome of Fiends will still be here when you get back.

There you go! Now that we're all on the same page (page XX), the reason why you've gotten into so many arguments with people as to whether their character was Lawful or Chaotic is because absolutely every action that any character ever takes could logically be argued to be \textbf{both}. A character who is honorable, adaptable, trustworthy, flexible, reliable, and loves freedom is a basically stand-up fellow, and meets the check marks for being ``ultimate Law" \textit{and} ``ultimate Chaos". There aren't any contradictory adjectives there. While Law and Chaos are \textit{supposed} to be opposed forces, there's nothing antithetical about the descriptions in the book.


\subsubsection{Ethics Option 1: A level of Organization.}
Optimal span of control is 3 to 5 people. Maybe Chaotic characters demand to personally control more units than that themselves and their lack of delegation ends up with a quagmire of incomprehensible proportions. Maybe Chaotic characters refuse to bow to authority at all and end up in units of one. Whatever the case, some DMs will have Law be well organized and Chaos be poorly organized. In this case, Law is objectively a virtue and Chaos is objectively a flaw.

Being disorganized doesn't mean that you're more creative or interesting, it just means that you accomplish less with the same inputs. In this model pure Chaos is a destructive, but more importantly \textit{incompetent} force.

\subsubsection{Ethics Option 2: A Question of Sanity.}
Some DMs will want Law and Chaos to mean essentially ``Sane" and ``\textit{In}sane". That's fine, but it doesn't mean that Chaos is \textit{funny}. In fact, insanity is generally about the least funny thing you could possibly imagine. An insane person reacts inappropriately to their surroundings. That doesn't mean that they perform \textit{unexpected} actions, that's just surrealist. And Paladins are totally permitted to enjoy non sequitur based humor and art. See, insanity is when you perform the same action over and over again and expect different results.

In this model we get a coherent explanation for why, when all the forces of Evil are composed of a multitude of strange nightmarish creatures, and the forces of Good have everything from a glowing patch of light to a winged snake tailed woman, every single soldier in the army of Chaos is a giant frog. This is because in this model Limbo is a place that is \textit{totally insane}. It's a place where the answer to every question \textit{really is} ``Giant Frog". Creatures of Chaos then proceed to go to non Chaotically-aligned planes and are disappointed and confused when doors have to be pushed and pulled to open and entrance cannot be achieved by ``Giant Frog".

If Chaos is madness, it's not ``spontaneous", it's ``non-functional". Actual adaptability is \textbf{sane}. Adapting responses to stimuli is what people are supposed to do. For reactions to be sufficiently inappropriate to qualify as insanity, one has to go pretty far into one's own preconceptions. Actual mental illness is very sad and traumatic just to watch as an outside observer. Actually living that way is even worse. It is strongly suggested therefore, that you don't go this route at all. It's not that you can't make D\&D work with sanity and insanity as the core difference between Law and Chaos, it's that in doing so you're essentially making the Law vs. Chaos choice into the choice between good and bad. That and there is a certain segment of the roleplaying community that cannot differentiate absurdist humor from insanity and will insist on doing annoying things in the name of humor. And we hate those people.

\subsubsection{Ethics Option 3: The Laws of the Land.}
Any region that has writing will have an actual code of laws. Even oral traditions will have, well, \textit{traditions}. In some campaigns, following these laws makes you Lawful, and not following these laws makes you Chaotic. This doesn't mean that Lawful characters necessarily have to follow the laws of Kyuss when you invade his secret Worm Fort, but it does mean that they need to be an ``invading force" when they run around in Kyuss' Worm Fort. Honestly, I'm not sure what it even \textit{means} to have a Chaotic society if Lawful means ``following your own rules". This whole schema is workable, but only with extreme effort. It helps if there's some sort of divinely agreed upon laws somewhere that nations and individuals can follow to a greater or lesser degree. But even so, there's a lot of hermits and warfare in the world such that whether people are following actual laws can be just plain hard to evaluate.

I'd like to endorse this more highly, since any time you have characters living up to a specific arbitrary code (or not) it becomes a lot easier to get things evaluated. Unfortunately, it's really hard to even imagine an entire nation fighting for not following their own laws. That's just\ldots\  really weird. But if you take Law to mean law, then you're going to have to come to terms with that.

\subsubsection{Ethics Option 4: My Word is My Bond.}
Some DMs are going to want Law to essentially equate to following through on things. A Lawful character will keep their word and do things that they said they were going to. In this model, a Lawful character has an arbitrary code of conduct and a Chaotic character does not. That's pretty easy to adjudicate, you just announce what you're going to do and if you \textit{do it}, you're Lawful and if you \textit{don't} you're not.

Here's where it gets weird though: That means that Lawful characters have a \textit{harder time} working together than do non-lawful characters. Sure, once they agree to work together there's some Trust there that we can capitalize, but it means that there are arbitrary things that Lawful characters won't do. Essentially this means that Chaotic parties order one mini-pizza each while Lawful parties have to get one extra large pizza for the whole group -- and we know how difficult that can be to arrange. A good example of this in action is the Paladin's code: they won't work with Evil characters, which restricts the possibilities of other party members.

In the world, this means that if you attack a Chaotic city, various other chaotic characters will trickle in to defend it. But if you attack a Lawful city, chances are that it's going to have to stand on its own.

\subsubsection{Adherence to Self: Not a Rubric for Law}
Sometimes Lawfulness is defined by people as adhering to one's personal self. That may \textit{sound} very ``Lawful'', but there's no way that makes any sense. Whatever impulses you happen to have, those are going to be the ones that you act upon, \textit{by definition}. If it is in your nature to do random crap that doesn't make any sense to anyone else -- then your actions will be contrary and perplexing, but they will still be completely consistent with your nature. Indeed, there is literally nothing you can do that isn't what you would do. It's circular.

\subsubsection{Rigidity: Not a Rubric for Law}
Sometimes Lawfulness is defined by people as being more ``rigid'' as opposed to ``spontaneous'' in your action. That's crap. Time generally only goes in one direction, and it generally carries a one to one correspondence with itself. That means that as a result of a unique set of stimuli, you are \textit{only going to do one thing}. In D\&D, the fact that other people weren't sure what the one thing you were going to do is handled by a Bluff check, not by being Chaotic.



\subsection{To Triumph Over Evil}

Equally important to the place of ultimate Evil in your game is the activities of Good in your game. Like Evil, the designers have tried to run the spectrum of possible interpretations of righteousness\ldots\  and the results are that the overlap of actions depicted as Good with those described as Evil is almost total. Ultimately, your campaign is going to have to come to a consensus over what you are going to accept as Good. Most importantly, the inverse of Evil \textit{is not Good}. It really takes a lot less harm to be Evil than it takes aid to be Good. If you fix twenty people's roofs, you're Jimmy the Helpful Thatcher. But if you eat your neighbor's daughter, you're Jimmy the Cannibal -- and no additional carpentry assistance will change that. This is why the \underline{Book of Exalted Deeds} is such an unsatisfying read\ldots\  you can't just take the material in the \underline{Book of Vile Darkness} and multiply by negative one to get Good.


\subsubsection{The Importance of Consequentialism}
Every action has motivations, expectable results, and actual results. In addition, every action can be described with a verb. In the history of moral theory (a history substantively longer than \textit{human} history) it has at times been contested by otherwise bright individuals that any of those (singly or collectively) could be used as a rubric to determine the rightness of an action. D\&D authors agreed. With all of those extremely incompatible ideas. And the result has been an unmitigated catastrophe. No one knows what makes an action Good in D\&D, so your group is ultimately going to have to decide for yourselves. Is your action Good because your intentions are Good? Is your action Good because the most likely result of your action is Good? Is your action Good because the actual end result of that action is Good? Is your action Good because the verb that bests describes your action is in general Good? There are actually some very good arguments for all of these written by people like Jeremy Bentham, Immanuel Kant, and David Wasserman -- but there are many other essays that are so astoundingly contradictory and ill-reasoned that they are of less help than reading nothing. Unfortunately for the hobby, some of the essays of the second type were written by Gary Gygax.

This is not an easy question to answer. The rulebooks, for example, are no help at all. D\&D at its heart is about breaking into other peoples' homes, stabbing them in the face, and taking all their money. That's very hard to rationalize as a Good thing to do, and the authors of D\&D have historically not tried terribly hard.

\subsubsection{Godliness isn't Goodliness}
Whatever religion you personally have, the religion in D\&D revolves around a set of gods both Good and Evil of equal strength and importance. Most modern day religions have however many gods they worship be of sufficient goodness that they are at least worthy of respect -- so it can be hard to remember that in D\&D the gods as a whole are precisely zero sum on any issue. Being ``divine" doesn't make you Good in D\&D, it just makes you more. If you're Good it makes you more Good, but if you're Evil it makes you more Evil. Clerics detect strongly of whatever alignment they have, but there's nothing Good about priests as a whole. Turning your back on the gods isn't a bad thing in D\&D, it's a perfectly valid \textit{and neutral} choice. If Ur Priests are to have any alignment restriction at all, it should be the same as Druids -- stealing from the gods is a profoundly neutral act, not Good and not Evil.

\subsubsection{There is no Salvation or Redemption in D\&D}
All of the major religions of our world that utilize the concepts of Ultimate Good and Ultimate Evil use the concept of Redemption (that people have a state of innocence that they can lose and perhaps regain through \spell{atonement}) or Salvation (that people have a state of inherent unworthiness that they can overcome). D\&D, despite having a spell called \spell{atonement} actually has neither of those concepts. The \spell{atonement} spell actually dedicates (or rededicates) a character to any alignment, Good or Evil, Law or Chaos. Baby kobolds are not born into original sin and baby elves are not born in a state of grace, D\&D doesn't even have those concepts. Creatures with an alignment subtype (most Fiends, for example) are born into that alignment and are only going to stray from it if subjected to powerful magic or arguments. Everyone else is born neutral.

In D\&D, creatures do not ``fall" into Evil. Being Evil is a valid choice that is fully supported by half the gods just as Good is. Those who follow the tenets of Evil throughout their lives are judged by \textit{Evil Gods} when they die, and can gain rewards at least as enticing as those offered to those who follow the path of Good (who, after all, are judged by \textit{Good Gods} after they die). So when sahuagin run around on land snatching children to use as slaves or sacrifices to Baatorians, they aren't putting their soul in danger. They are actually keeping their soul \textit{safe}. Once you step down the path of villainy, you get a \textit{better deal} in the afterlife by being \textit{more evil}.

The only people who get screwed in the D\&D afterlife are traitors and failures. A traitor gets a bad deal in the afterlife because whichever side of the fence they ended up on is going to remember their deeds on the other side of the fence. A failure gets a bad deal because they end up judged by gods who wanted them to succeed. As such, it is \textit{really hard} to get people to change alignment in D\&D. Unless you can otherwise assure that someone will die as a failure to their alignment, there's absolutely no incentive you could possibly give them that would entice them to betray it.

\subsubsection{Code of Conduct: Paladins}
Nothing causes more arguments in-game than Paladins. Can Paladins kill baby kobolds? What about baby mind flayers? Honestly, while these questions have generated a lot of ink and a lot of bad feelings, they aren't important. Paladins are Lawful Good, but they aren't ``champions of Law and Good" -- that's an Archon. A Paladin doesn't get Smite \textit{Chaos}, they aren't forced to abandon team members who behave in a Chaotic fashion (whatever that means, see below). Paladins are Champions of Good\texttrademark\ \textit{and} they are required to be Lawful. Whether or not that makes any sense depends on how you're handling Law and Chaos.

Paladins are as Good as any character can be, and they are required to follow a code of conduct. However, following this code is no what makes them Good, we know this because Clerics of Good (who detect as being just as Good as Paladins) don't have to follow that code. The code is completely arbitrary, and has no bearing on the relative Goodness of a character. Paladins also lose their powers if they don't drink for a few days, but that doesn't put Blackguards in danger of losing their alignment when they quaff a glass of water.

The Paladin's code is uncompromising, but it is also exhaustive about what it won't allow:

\listone
    \item  \ability{The Use of Poison:}{If a park ranger hits a bear with a tranq dart, that's not an Evil act. Poison isn't any more or less Evil than a blade. Paladins can't use poison because they agreed not to -- not because there's anything wrong with poison. Maybe Paladins only get to keep their magically enhanced immune system so long as they don't take it for granted by using things that would tax it on purpose. Maybe their concern for public safety is so great that they are only willing to use weapons that \textit{look} like weapons. Whatever. The point is that Paladins have to be Good \textit{and} they can't use Poison, and these are separate restrictions.}
    \item  \ability{Lies:}{A Paladin can't lie. Whether telling a lie is a good or evil act depends on what you're saying and who you are saying it to. But a Paladin won't do it. That means that if the Nazis come to the door and demand to know if the Paladin is hiding any Jews (she is), she can't glibly say ``No." That does not mean that she has to say ``Yes, they're right under the stairs!" -- it means that she has to tell the Nazis point blank ``I'm not going to participate in your genocidal campaign, it's wrong." This will start a fight, and may get everyone killed, so the Paladin is well within her code to eliminate the middle man and just stab the Gestapo right there before answering. That's harsh, but the Paladin's code isn't about doing what's easy, or even what's \textit{best}. It's about doing what you said you were going to.}
    \item  \ability{Cheating:}{Paladin's don't cheat. They don't have to keep playing if they figure out that someone else is cheating, but they aren't allowed to cheat at dice to rescue slaves or whatever. Again, there's nothing Good about not cheating, it's just something they have to do \textit{in addition} to being Good all the time.}
    \item  \ability{Association Restrictions:}{Paladins are not allowed to team up with Evil people. They aren't allowed to offer assistance to Evil people and they aren't allowed to receive assistance \textit{from} Evil people. Intolerance of this sort isn't Evil, but it isn't Good either. It's just another thing that Paladins have to do.}
\end{list}



\subsection{I Fought the Law}

Regardless of what your group ends up meaning when they use the word ``Law'', the fact is that some of your enemies are probably going to end up being Lawful. That doesn't mean that Lawful characters can't stab them in their area, whatever it is that you have alignments mean it's still entirely acceptable for Good characters to stab other Good characters and Lawful characters to stab other Lawful characters (oddly, no one even asks if it's a violation of Chaotic Evil to kill another Chaotic Evil character, but it isn't). There are lots of reasons to kill a man, and alignment disagreements don't occupy that list exclusively.


\subsubsection{Code of Conduct: \hyperref[class:barbarian]{Barbarian}}
A Barbarian who becomes Lawful cannot Rage. Why not? There's no decent answer for that. Rage doesn't seem to require that you not tell people in advance that you're going to do it, nor does it actually force you to break promises once you're enraged. It doesn't force you to behave in any particular fashion, and no one knows why it is restricted.

\subsubsection{Code of Conduct: Bard}
If \textit{anyone} can tell me why a concert pianist can't be Lawful I will personally put one thing of their choice into my mouth. Music is expressionistic, but it is also mathematical. Already there are computers that can write music that is indistinguishable from the boring parts of Mozart in which he's just going up and down scales in order to mark time.

\subsection{Beating Back Chaos}
Long ago ``Law'' and ``Chaos'' were used euphemisms by Pohl Anderson for Good and Evil, and that got taken up by other fantasy and science fiction authors and ultimately snow-balled into having a Chaos alignment for D\&D. If you go back far enough, ``Chaos" actually \textit{means} ``The Villains", and when it comes down to it there's no logical meaning for it to have other than that -- so the forces of Chaos really are going to show up at your door to take a number for a whuppin at some point. Depending upon what your group ends up deciding to mean by Chaos, this may seem pretty senselessly cruel. If the forces of Chaos are simply unorganized then you are essentially chasing down hobos and beating down the ones too drunk to get away. If Chaos is insanity than the Chaos Hunters in your game are essentially going door to door to beat up the retarded kids.

The key is essentially to not overthink it. Chaos was originally put into the fantasy genre in order to have bad guys without having to have black hatted madmen trying to destroy the world. So if Team Chaos is coming around your door, just roll with it. The whole point is to have villains that you can stab without feeling guilty while still having villains to whom your characters can \textit{lose} without necessarily losing the whole campaign world.

\subsubsection{Code of Conduct: \hyperref[class:knight]{Knight}}
Sigh. The Knight's code of conduct doesn't represent Lawful activity no matter what your group means by that term. They \textit{can't} strike an opponent standing in a grease effect, but they \textit{can} attack that same person \textit{after they fall down in the grease!} They also are not allowed to win a duel or stake vampires (assuming, for the moment that you were using some of the house rules presented in The Tome of Necromancy that allow vampires to be staked by \textit{anyone}). So the Knight's code is not an example of Lawfulness in practice, it's just a double fistful of stupid written by someone who obviously doesn't understand D\&D combat mechanics. If you wanted to make a Knight's Code that represented something like ``fighting fair'', you'd do it like this:

\listone
    \item May not accept benefit from Aid Another actions.
    \item May not activate Spell Storing items (unless the Knight cast the spell into the item in the first place).
    \item May not use poison or disease contaminated weapons.
\end{list}

But remember: such a code of fair play is no more Lawful than \textit{not} having a code of fair play. Formians are the embodiment of Law, and they practically wrote the book on cooperation. So while a Knight considers getting help from others to be ``cheating", that's not because he's Lawful. He considers getting such aid to be cheating \textit{and} he's Lawful. What type of Lawful a Knight represents is determined by your interpretation of Law as a whole. Maybe a Knight has to uphold the law of the land (right or wrong). Maybe a Knight has to keep his own word. Whatever, the important part is that the arbitrary code that the Knight lives under is just that -- \textit{arbitrary}. The actual contents of the code are a separate and irrelevant concern to their alignment restriction.

\subsubsection{Code of Conduct: \hyperref[class:monk]{Monk}}
Again with the sighing. No one can explain why Monks are required to be Lawful, least of all the Player's Handbook. Ember is Lawful because she ``follows her discipline", while Mialee is not Lawful because she is ``devoted to her art". FTW?! That's the same thing, given sequentially as an example of being Lawful and not being Lawful. Monk's training requires strict discipline, but that has nothing to do with Lawfulness no matter what setup for Law and Chaos you are using. If Lawfulness is about organization, you are perfectly capable of being a complete maverick who talks to no one and drifts from place to place training constantly like the main character in Kung Fu -- total lack of organization, total ``Chaotic" -- total disciplined Monk. If Law is about Loyalty, you're totally capable of being treacherous spies. In fact, that's even an example in the PHB ``Evil monks make ideal spies, infiltrators, and assassins." And well, that sentence pretty much sinks any idea of monks having to follow the law of the land or keeping their own word, doesn't it? The only way monk lawfulness would make any sense is if you were using ``adherence to an arbitrary self" as the basis of Law, and we already know that can't hold.

\subsubsection{Code of Conduct: Paladin Again}
This has to be repeated: Paladins don't get Smite Chaos. They are not champions of Law and Good, they are Champions of Good who are required to be Lawful. If your game \underline{is not} using Word is Bond Ethics, Paladins have no reason to be Lawful. Paladins are only encouraged to follow the laws of the country they live in if those laws are Good. They are actually forbidden by their code of conduct from following the precepts of Evil nations. The Paladin shtick works equally well as a loner or a leader, and it is by definition distinctly disloyal. A Paladin must abandon compatriots.


\chapter{Fiends with Class}

\section{Base Classes}

What follows are four base classes. The Summoner is a magician class available to any player character race, while the other three are intended only for Fiendish use and are of primary utility to construct and advance Fiends for use as villains and cohorts.


\classname{True Fiend [Fiend]} \label{class:truefiend}
\vspace*{-8pt}


\quot{"I am a lord of the realms infernal, if I wanted your opinion I would beat it out of you."}

\desc{Tanar'ri, Baatezu, Yugoloth, Demodand\ldots\  these are the
names that inspire terror throughout the planes, and with good
reason. These True Fiends are far more powerful than the other
denizens of the Dark Realms. These fiends stand above others and are
destined to a life of greatness, to be legendary in the annals of
Evil.}

\desc{These true fiends are good at everything they do, but this doesn't make them more powerful at any particular level than any other fiend. Indeed, level is a measure of power. The most powerful fiends are True Fiends and higher level than Fiendish Brutes. The True Fiend advances in everything all at once, and thus gains new abilities relatively slowly compared to what other, more focused Fiendish progressions are capable of.}

\ability{Alignment:}{Nothing \textit{requires} a True Fiend to be Evil, but
it's\ldots\  highly recommended that they be Evil.}

\ability{Races:}{The True Fiend is \emph{only} available to
Outsiders with a plane of origin in the Lower Planes. Creatures from
the prime material plane whose ancestors were from a Lower Plane may
take this class, but they must have the Outsider type.}

\ability{Starting Gold:}{6d6x10 gp (210 gold)}

\ability{Starting Age:}{Since a True Fiend is immortal and never
ages, a character may claim any starting age she wishes.}

\ability{Hit Die:}{d8}

\ability{Class Skills:}{The True Fiend's class skills (and the key
ability for each skill) are Bluff (Cha), Climb (Str), Concentration
(Con), Diplomacy (Cha), Disguise (Cha), Escape Artist (Dex), Hide
(Dex), Intimidate (Cha), Knowledge (all skills taken individually)
(Int), Listen (Wis), Move Silently (Dex), Search (Int), Sense Motive
(Wis), Sleight of Hand (Dex), Spellcraft (Int), Spot (Wis), Survival
(Wis), Use Magic Device (Cha), and Use Rope (Dex).}

\ability{Skills/Level:}{8 + Intelligence Bonus}

\begin{table}[tbh]
\begin{small}
\begin{tabular}{lp{3.5cm}p{0.7cm}p{0.7cm}p{0.7cm}l}
Level  &Base Attack  Bonus &Fort Save &Ref Save &Will Save &Special\\
1st &+1 &+2 &+2 &+2 &Immortality, Fiendish Traits\\
2nd &+2 &+3 &+3 &+3 &Telepathy\\
3rd &+3 &+3 &+3 &+3 &Fiendish Damage Reduction\\
4th &+4 &+4 &+4 &+4 &Sphere\\
5th &+5 &+4 &+4 &+4 &Greater Fiendish Traits\\
6th &+6/+1 &+5 &+5 &+5 &\\
7th &+7/+2 &+5 &+5 &+5 &Greater Fiendish Damage Reduction\\
8th &+8/+3 &+6 &+6 &+6 &Sphere\\
9th &+9/+4 &+6 &+6 &+6 &Summon\\
10th &+10/+5 &+7 &+7 &+7 &\\
11th &+11/+6/+6 &+7 &+7 &+7 &Bonus Feat\\
12th &+12/+7/+7 &+8 &+8 &+8 &Sphere\\
13th &+13/+8/+8 &+8 &+8 &+8 &Greater Fiendish Damage Reduction\\
14th &+14/+9/+9 &+9 &+9 &+9 &\\
15th &+15/+10/+10 &+9 &+9 &+9 &Greater Summoning\\
16th &+16/+11/+11/+11 &+10 &+10 &+10 &Sphere\\
17th &+17/+12/+12/+12 &+10 &+10 &+10 &Dark Power\\
18th &+18/+13/+13/+13 &+11 &+11 &+11 &\\
19th &+19/+14/+14/+14 &+11 &+11 &+11 &Epic Damage Reduction\\
20th &+20/+15/+15/+15 &+12 &+12 &+12 &Sphere\\
\end{tabular}
\end{small}
\end{table}

\smallskip\noindent All of the following are Class Features of the True Fiend class.

\ability{Weapon and Armor Proficiency:}{True Fiends are proficient with all simple and martial weapons, as well as the whip, the scourge, and the dire flail. True Fiends are proficient with light armor but not with shields of any kind.}

\ability{Immortality (Ex):}{Like a pizza on the counter, the True
Fiend only gets worse with age.}

\ability{Fiendish Traits:}{A True Fiend is a member of one of the
iconic aristocracies evil. Starting at first level she has gains
access to the distinctive abilities of her race, as befits her plane
of origin:}

\listone
    \itemability{Baator:}{Baatezu Traits:}
    \listtwo
      \itemability{See in Darkness (Ex):}{A Baatezu can see in normal or magical darkness as if it was fully illuminated.}
      \itemability{Immunity to Fire:}{A Baatezu takes no damage from fire of any kind.}
    \end{list}
    \itemability{Gehenna:}{Yugoloth Traits:}
    \listtwo
      \itemability{Magic Resistance (Ex):}{Yugoloths are inherently resistant to magic, and have a Spell Resistance of \mbox{10 + their character level}. If a Yugoloth has SR from any other source, this ability increases that SR by +2 (if that would be more beneficial than simply replacing the other SR).}
      \itemability{Immunity to Acid:}{A Yugoloth takes no damage from Acid of any kind.}
    \end{list}
    \itemability{The Abyss:}{Tanar'ri Traits:}
    \listtwo
      \itemability{Bonus Feat:}{Forged with unbridled Chaos, every Tanar'ri is unique. Upon gaining access to its Fiendish Traits, a Tanar'ri gains one bonus feat of any [Fiend] feat that it qualifies for.}
      \itemability{Immunity to Electricity:}{A Tanar'ri takes no damage from electricity of any kind.}
    \end{list}
    \itemability{Carceri:}{Demodand Traits:}
    \listtwo
    \itemability{Freedom of Movement (Ex):}{A Demodand benefits from the effects of a freedom of movement spell at all times.}
      \itemability{Immunity to Poison (Ex):}{A Demodand suffers no harmful effect from poisons of any kind.}
    \end{list}
\end{list}


\ability{Telepathy (Su):}{At 2nd level, a True Fiend gains the
ability to communicate telepathically with any creature that speaks
a language within 100 feet.}

\ability{Fiendish Damage Reduction (Su):}{At 3rd level, the True
Fiend gains damage reduction that stops their class level in damage
and that can be penetrated by Good weapons or weapons made out of a
material that is baneful to the Fiend's race (Silver for Baatezu,
Wood for Yugoloths, Iron for Tanar'ri, and Stone for Demodands). At
7th level, the Damage Reduction can be penetrated only by Good
weapons. At 13th level the Damage Reduction is only penetrated by
weapons which are both Good and made of a baneful substance. At 19th
level, the True Fiend's Damage Reduction can only be penetrated by
Epic weapons.}

\ability{Sphere:}{The True Fiend gains basic access to a sphere at
4th level, and gains a new sphere at every fourth level afterwards.
If the True Fiend selects a sphere that she already has basic access
to, she upgrades it to advanced access. If she already had advanced
access, she gains expert access.}

\ability{Greater Fiendish Traits:}{A True Fiend of 5th level or more
gains access to more of the distinctive abilities of her race, as
befits her plane of origin:}

\listone
    \itemability{Baator:}{Baatezu Traits:}
    \listtwo
      \itemability{Mundane Poison Immunity (Ex):}{A Baatezu is immune to all non-magical poisons.}
      \itemability{Resistances:}{A Baatezu has Cold and Acid Resistance 10.}
    \end{list}
    \itemability{Gehenna:}{}{Yugoloth Traits:}
    \listtwo
      \itemability{Mundane Poison Immunity (Ex):}{A Yugoloth is immune to all non-magical poisons.}
      \itemability{Resistances:}{A Yugoloth has Cold, Fire, and Electricity Resistance 10.}
    \end{list}
    \itemability{The Abyss:}{}{Tanar'ri Traits:}
    \listtwo
      \itemability{Mundane Poison Immunity (Ex):}{A Tanar'ri is immune to all non-magical poisons.}
      \itemability{Resistances:}{A Tanar'ri has Cold, Fire, and Acid Resistance 10.}
    \end{list}
    \itemability{Carceri:}{Demodand Traits:}
    \listtwo
      \itemability{Immunities:}{The worlds of Carceri are varied beyond belief and each is filled with a new torture that defies comprehension. Upon reaching 5th level, a Demodand gains immunity to 2 Energy Types. Once chosen, the energy types cannot be changed.}
    \end{list}
\end{list}

\ability{Summon (Sp):}{At 9th level, a True Fiend can attempt to
summon others of its kind \example{(for example: a Yugoloth could
summon other Yugoloths)}. Summoning another Fiend of the same
character level has a 40\% chance of success, and summoning a Fiend
of a lower level increases the chances of success by 10\% for every
level the summoner's level exceeds the CR of the target.}

\ability{Bonus Feat:}{At 11th level, a True Fiend gains a bonus
feat. This feat may be any [Fiend] feat for which she meets the
prerequisites.}

\ability{Greater Summoning:}{A True Fiend of 15th level may attempt
to use her summon power to summon a fiend of a level higher than her
own, though doing so carries only a 30\% chance of success.}

\ability{Dark Power:}{The powers of the lower planes are awesome to
behold. At 17th level, the True Fiend gains a +10 bonus to defeating
Spell Resistance with the spell-like abilities granted by her
spheres.}


\classname{Fiendish Brute [Fiend]}
\vspace*{-8pt}
\quot{"Rowwr!"}



\desc{The power of a fiend goes beyond mere magical power: it is a transformation into a form that most suits one's evil. For some fiends, physical power is the route by which they work their will on the world, and fangs and claws are just a few of the weapons that fiends develop in order to rend the helpless flesh of their prey. Physical transformations like stingers, wings, poisons, and vile diseases all find their way into the armaments of fiends, but do not think that these are the limits of their evil; fiends are nothing if not creative in the pursuit of their own particular brand of evil.}

\ability{Alignment:}{A character must be non-good to take any levels in Fiendish Brute. Nothing happens to a Fiendish Brute if he becomes Good, save that he must look elsewhere for class advancing.}

\ability{Races:}{The Fiendish Brute is \emph{only} available to Outsiders with a plane of origin in the Lower Planes. Creatures from the prime material plane whose ancestors were from a Lower Plane may take this class, but they must have the Outsider type.}

\ability{Starting Gold:}{4d4x10 gp (100 gold)}

\ability{Starting Age:}{As Warforged Rogue.}

\ability{Hit Die:}{d10}

\ability{Class Skills:}{The Fiendish Brute's class skills (and the key ability for each skill) are Balance (Dex), Climb (Str), Hide (Dex), Jump (Str), Listen (Wis), Move Silently (Dex), Sense Motive (Wis), Spot (Wis), Survival (Wis), and Swim (Str).}

\ability{Skills/Level:}{4 + Intelligence Bonus}



\begin{table}[tbh]
\begin{small}
\begin{tabular}{lp{1.9cm}p{0.7cm}p{0.7cm}p{0.7cm}p{8cm}}
Level  &Base Attack Bonus &Fort Save &Ref Save &Will Save &Special\\
1st &+0 &+2 &+2 &+0 &Natural Weapons, Natural Armor, Attribute Boost\\
2nd &+1 &+3 &+3 &+0 &Bonus Feat\\
3rd &+1 &+3 &+3 &+1 &Attribute Boost\\
4th &+2 &+4 &+4 &+1 &Bonus Feat\\
5th &+2 &+4 &+4 &+1 &Attribute Boost\\
6th &+3 &+5 &+5 &+2 &Bonus Feat\\
7th &+3 &+5 &+5 &+2 &Attribute Boost\\
8th &+4 &+6 &+6 &+2 &Bonus Feat\\
9th &+4 &+6 &+6 &+3 &Attribute Boost\\
10th &+5 &+7 &+7 &+3 &Bonus Feat\\
\end{tabular}
\end{small}
\end{table}

\smallskip\noindent All of the following are Class Features of the Fiendish Brute class.

\ability{Weapon and Armor Proficiency:}{Fiendish Brutes are proficient only with armor spikes, and aren't inherently proficient with any armor.}

\ability{Natural Weapons:}{A Fiendish Brute has 2 claw attacks, one attached to each arm (or its two front legs if it is a quadruped, if it has no limbs at all it grows two vestigial arms that have claws at the end). These claws are natural weapons and inflict damage normal for the creature's size.}

\ability{Natural Armor:}{A Fiendish Brute has a natural armor bonus of 3 plus its class level. So a 7th level Fiendish Brute has a natural armor bonus of +10.}

\ability{Attribute Boost:}{At 1st level and every odd numbered level afterwards, the Fiendish Brute's physical attributes improve, as if it had gained several character levels. Every time the Fiendish Brute gains an attribute boost, two of his physical attributes irrevocably increase by 1.}

\ability{Bonus Feat:}{At every even numbered level, the Fiendish Brute gains a bonus feat. This feat may be any [Fiend], [Monstrous], or [General] feat, and the Fiendish Brute must meet the prerequisites.}


\classname{Conduit of the Lower Planes [Fiend]} \label{class:conduit}
\vspace*{-8pt}
\quot{``My powers are more than enough to deal with the likes of you!''}


\desc{A fiend is more than an individual; he is a representative of a particular brand of evil, and that role carries power with it. To become a Conduit of the Lower Planes is to embrace that role and become a living pathway by which the energies of the Lower Planes can be made manifest. Raw magical power is the result of this process, and a fiend that walks this route refines his mastery of his innate mystical arts to a terrible degree.}

\ability{Alignment:}{A character must be non-good to take any levels in Conduit of the Lower Planes. Nothing happens to a Conduit of the Lower Planes if he becomes Good, save that he must look elsewhere for class advancing.}

\ability{Races:}{The Conduit of the Lower Planes is \emph{only} available to creatures with a plane of origin in the Lower Planes. Creatures from the prime material plane whose ancestors were from a Lower Plane may take this class.}

\ability{Starting Gold:}{4d4x10 gp (100 gold)}

\ability{Starting Age:}{As Wizard.}

\ability{Hit Die:}{d6}

\ability{Class Skills:}{The Conduit of the Lower Planes's class skills (and the key ability for each skill) are Concentration (Con), Craft (Int), Diplomacy (Cha), Handle Animal(Cha), Knowledge (arcana) (Int), Knowledge (history) (Int), Knowledge (religion) (Int), Knowledge (the planes) (Int), Profession (Wis), and Spellcraft (Int).}

\ability{Skills/Level:}{2 + Intelligence Bonus}


\begin{table}[tbh]
\begin{small}
\begin{tabular}{lp{1.9cm}p{0.7cm}p{0.7cm}p{0.7cm}p{8cm}}
Level  &Base Attack Bonus &Fort Save &Ref Save &Will Save &Special\\
1st &+0 &+0 &+0 &+2 &Sphere, Petitioner Immunities\\
2nd &+1 &+0 &+0 &+3 &Enhanced Sphere Access\\
3rd &+2 &+1 &+1 &+3 &Sphere\\
4th &+3 &+1 &+1 &+4 &Bonus Feat\\
5th &+3 &+1 &+1 &+4 &Sphere\\
6th &+4 &+2 &+2 &+5 &Petitioner Skills\\
7th &+5 &+2 &+2 &+5 &Sphere\\
8th &+6/+1 &+2 &+2 &+6 &Bonus Feat\\
9th &+6/+1 &+3 &+3 &+6 &Sphere\\
10th &+7/+2 &+3 &+3 &+7 &Magical Training\\
\end{tabular}
\end{small}
\end{table}


\smallskip\noindent All of the following are Class Features of the Conduit of the Lower Planes class.

\ability{Weapon and Armor Proficiency:}{Conduits of the Lower Planes are proficient with all simple and martial weapons, as well as the whip, the scourge, and the dire flail. Conduits of the Lower Planes are proficient with light armor but not with shields of any kind.}

\ability{Sphere:}{The Conduit of the Lower Planes gains basic access to a sphere at every odd numbered level. If the Conduit of the Lower Planes selects a sphere that he already has basic access to, he upgrades it to advanced access. If he already had advanced access, he gains expert access.}

\ability{Petitioner Immunities:}{A Conduit of the Lower Planes gains his power from a specific lower plane, and is protected by the nature of that plane:}
\listone
    \item \ability{Pandemonium:}{Sonic Immunity}
    \item \ability{The Abyss:}{Immunity to Electricity}
    \item \ability{Carceri:}{Immunity to Cold}
    \item \ability{Hades:}{Immunity to Fear and Morale Effects.}
    \item \ability{Gehenna:}{Immunity to Acid}
    \item \ability{Baator:}{Immunity to Fire}
    \item \ability{Acheron:}{Immunity to [Compulsion] effects.}
\end{list} \vspace*{8pt}

\ability{Enhanced Sphere Access:}{At 2nd level, the Conduit of the Lower Planes gains extra uses of the spell-like abilities that he gains from his Spheres. The Conduit of the Lower Planes gains a number of extra uses of any spell-like ability equal to half the number his character level exceeds the minimum character level to use the spell-like ability (rounded up). So if the Conduit of the Lower Planes has a character level of 4, he would gain 1 extra use of a spell-like ability that is granted by one of his spheres at character level 3 and 2 extra uses of any spell-like from one of his spheres with a minimum level of 1. Upon gaining this ability, the Conduit of the Lower Planes immediately gains a number of extra feats that must all have the [Fiend] tag equal to the number of spheres he has expert access to. If he ever gains expert access to another sphere, he also gains an extra [Fiend] feat.}

\ability{Bonus Feat:}{At 4th level, the Conduit of the Lower Planes gains a bonus feat. This feat may be any [Fiend], [Monstrous], or [Item Creation] feat, and the Conduit of the Lower Planes must meet the prerequisites. He gains another such feat at level 8.}

\ability{Petitioner Skills:}{A Conduit of the Lower Planes gains his power from a specific lower plane, and at 6th level gains abilities from the nature of that plane (this must be the same plane as was chosen at 1st level):}
\listone
    \item \ability{Pandemonium:}{+10 to Listen checks.}
    \item \ability{The Abyss:}{+10 bonus to Survival checks.}
    \item \ability{Carceri:}{+10 bonus to Bluff checks.}
    \item \ability{Hades:}{+10 bonus to Hide checks.}
    \item \ability{Gehenna:}{+10 bonus to Climb checks, if he doesn't already have a climb speed, he gains one equal to half his normal ground speed (bonuses to Climb from having a Climb speed gained in this way would be included in the +10 bonus)}
    \item \ability{Baator:}{+10 bonus to Disguise checks.}
    \item \ability{Acheron:}{+10 bonus to Intimidate checks.}
\end{list}\vspace*{8pt}

\ability{Magical Training:}{At 10th level, the Conduit of the Lower Planes is able to cast magic in a more traditional fashion. He has the spells known and spells per day of a 6th level Sorcerer. The Conduit of the Lower Planes has a caster level of 10, and can take classes that improve spellcasting in order to gain additional spellcasting ability.}

\classname{Summoner} \label{class:summoner}
\vspace{-8pt}
\quot{"Watch me pull a rhinoceros out of my hat."}

\ability{Alignment:}{A Summoner may be of any alignment.}

\ability{Races:}{Summoners appear in every race. Halflings in particular are very likely to adopt the way of the Summoner.}

\ability{Starting Gold:}{4d4x10 gp (100 gold)}

\ability{Starting Age:}{As Rogue.}

\ability{Hit Die:}{d6}

\ability{Class Skills:}{The Summoner's class skills (and the key ability for each skill) are Concentration (Con), Craft (Int), Decipher Script (Int), Diplomacy (Cha), Escape Artist (Dex), Handle Animal (Cha), Intimidate (Cha), Knowledge (arcana) (Int), Knowledge (nature) (Int), Knowledge (the planes) (Int), Profession (Wis), Ride (Dex), Sense Motive (Cha), and Spellcraft (Int).}

\ability{Skills/Level:}{4 + Intelligence Bonus}


\begin{table}[tbh]
\begin{small}
\begin{tabular}{lp{1.2cm}p{0.7cm}p{0.7cm}p{0.7cm}p{4.7cm}llllllllll}
Level  &Base Attack  Bonus &Fort Save &Ref Save &Will Save &Special & \multicolumn{10}{c}{Spells Per Day}\\
&&&&&& 0 &1 &2 &3 &4 &5 &6 &7 &8 &9\\
1st &+0 &+0 &+2 &+2 &Rapid Summoning, Summoned Cohort, Armored Casting, Aura &5 &3 &- &- &- &- &- &- &- &-\\
2nd &+1 &+0 &+3 &+3 &Sudden Extend &6 &4 &- &- &- &- &- &- &- &-\\
3rd &+1 &+1 &+3 &+3 &Advanced Learning &6 &5 &- &- &- &- &- &- &- &-\\
4th &+2 &+1 &+4 &+4 &Extended Summoning: 1st level &6 &6 &3 &- &- &- &- &- &- &-\\
5th &+2 &+1 &+4 &+4 &Advanced Learning &6 &6 &4 &- &- &- &- &- &- &-\\
6th &+3 &+2 &+5 &+5 &Master Tactician &6 &6 &5 &3 &- &- &- &- &- &-\\
7th &+3 &+2 &+5 &+5 &Advanced Learning, Summon &6 &6 &6 &4 &- &- &- &- &- &-\\
8th &+4 &+2 &+6 &+6 &  &6 &6 &6 &5 &3 &- &- &- &- &-\\
9th &+4 &+3 &+6 &+6 &Advanced Learning, Extended Summoning: 2nd level &6 &6 &6 &6 &4 &- &- &- &- &-\\
10th &+5 &+3 &+7 &+7 &Medium Armor Proficiency &6 &6 &6 &6 &5 &3 &- &- &- &-\\
11th &+5 &+3 &+7 &+7 &Advanced Learning, Improved Summoning &6 &6 &6 &6 &6 &4 &- &- &- &-\\
12th &+6/+1 &+4 &+8 &+8 &Advanced Aid &6 &6 &6 &6 &6 &5 &3 &- &- &-\\
13th &+6/+1 &+4 &+8 &+8 &Advanced Learning &6 &6 &6 &6 &6 &6 &4 &- &- &-\\
14th &+7/+2 &+4 &+9 &+9 &Extended Summoning: 3rd level &6 &6 &6 &6 &6 &6 &5 &3 &- &-\\
15th &+7/+2 &+5 &+9 &+9 &Advanced Learning, Improved Summoning &6 &6 &6 &6 &6 &6 &6 &4 &- &-\\
16th &+8/+3 &+5 &+10 &+10 &Shield Proficiency &6 &6 &6 &6 &6 &6 &6 &5 &3 &-\\
17th &+8/+3 &+5 &+10 &+10 &Advanced Learning &6 &6 &6 &6 &6 &6 &6 &6 &4 &-\\
18th &+9/+4 &+6 &+11 &+11 &  &6 &6 &6 &6 &6 &6 &6 &6 &5 &3\\
19th &+9/+4 &+6 &+11 &+11 &Advanced Learning, Extended Summoning: 4th level &6 &6 &6 &6 &6 &6 &6 &6 &6 &4\\
20th &+10/+5 &+6 &+12 &+12 &Perfect Summon &6 &6 &6 &6 &6 &6 &6 &6 &6 &6\\
\end{tabular}
\end{small}
\end{table}

\smallskip\noindent All of the following are Class Features of the Summoner class.

\ability{Weapon and Armor Proficiency:}{Summoners are proficient with all simple weapons, as well as the bola, the whip, the net, and the harpoon. Summoners are proficient with light armor but not with shields of any kind. At 10th level, a Summoner gains proficiency with Medium Armor. At 16th level, the Summoner gains proficiency with shields.}

\ability{Spellcasting:}{The Summoner is an Arcane Spellcaster with the same spells per day progression as a Sorcerer. A Summoner casts spells from the Summoner Spell List (below). A Summoner automatically knows every spell on her spell list. She can cast any spell she knows without preparing them ahead of time, provided that spell slots of an appropriate level are still available.To cast a Summoner spell, she must have a Charisma at least equal to 10 + the Spell level. The DC and bonus spells of the Summoner's spells is Charisma based.}

\ability{Rapid Summoning:}{A Summoner can cast summoning spells in less time than most other casters. Any Conjuration spell of the [Summoning] subschool that a Summoner casts that would have a casting time of one full round have a casting time of 1 standard action instead. This ability has no effect on spells which already require less than one full round to cast, nor does it affect spells with a casting time greater than one full round.}

\ability{Summoned Cohort:}{Once per day when a Summoner casts a Conjuration spell of the [Summoning] subschool, she may extend its duration to 24 hours. This ability is not cumulative with other effects that increase the duration of a spell.}

\ability{Armored Casting:}{A Summoner casts arcane spells, but she is not affected by the arcane spell failure of any armor or shield she is proficient with. This ability only applies to her Summoner spells, if she is able to cast any other arcane spells, they are affected by arcane spell failure normally.}

\ability{Aura:}{A Summoner's strong connection to the outer planes causes them to be detected very strongly of whatever alignment she has. For purposes of spells like \spell{detect chaos}, levels in Summoner count as Outsider hit dice.}

\ability{Sudden Extend:}{At 2nd level, the Summoner gains Sudden Extend as a bonus feat. If she already has Sudden Extend, she may gain any metamagic feat that she qualifies for instead.}

\ability{Advanced Learning:}{At 3rd level and every two levels afterwards, the Summoner may permanently add one spell to her spell list. This spell must be of a level she can already cast, and it must be of the Conjuration school. Only spells from the Cleric, Druid, or Wizard spell list may be added in this way.}

\ability{Extend Summoning:}{Conjuration spells of the [Summoning] subschool that the Summoner casts can be affected by the Extend Spell metamagic for free. At 4th level, the Summoner may apply this free spell extending to spells of 1st level or lower. At 9th level, she may apply it to 2nd level spells, at 14th level she may apply it to 3rd level spells, and at 19th level she may apply Extend Spell for free to 4th level spells.}

\ability{Master Tactician:}{A Summoner learns to fight with many allies. At 6th level, any ally within 30 feet of her may gain a +4 bonus for flanking instead of the normal +2 bonus as long as they can perceive the Summoner.}

\ability{\spell{Summon (Sp)}:}{At 7th level, a Summoner can attempt to summon outsiders with an alignment identical to her own. Summoning another creature of the same character level has a 40\% chance of success, and summoning a creature of a lower level increases the chances of success by 10\% for every level the Summoner's level exceeds the CR of the target.}

\ability{Advanced Aid (Ex):}{A Summoner of 12th level may take the Aid Another action as a free action once per round.}

\ability{Improved \spell{Summoning}:}{At 11th level, the Summoner's chances to \spell{summon} an outsider increase by 10\%. This chance increases by another 10\% at 15th level.}

\ability{Perfect \spell{Summon}:}{At 20th level, the Summoner's \spell{summon} power has a 100\% chance of success, even when summoning a creature of equal level to herself.}


\spelllist{Summoner Spell List:}
\small

\ability{0th level:}{\emph{Anticipate Teleportation, Caltrops, Darkness, Detect Magic, Detect Poison, Light, Protection from Alignment, Read Magic, Resist Planar Alignment}}

\ability{1st level:}{\emph{Avoid Planar Effects, Comprehend Languages, Grease, Portal Beacon, Summon Frost Beast I, Summon Monster I, Summon Nature's Ally I, Summon Undead I, Wall of Smoke}}

\ability{2nd level:}{\emph{Analyze Portal, Baleful Transposition, Dimension Hop, Daylight, Deeper Darkness, Entangle, Phantom Steed, Planar Tolerance, Portal Alarm, Regroup, Sleet Storm, Summon Frost Beast II, Summon Monster II, Summon Nature's Ally II, Summon Undead II, Tongues, Web, Wind Wall}}

\ability{3rd level:}{\emph{Clairaudience/Clairvoyance, Dimension Step, Dimensional anchor, Greater Anticipate Teleportation, Magic Circle Against Alignment, Plant Growth, Stinking Cloud, Summon Frost Beast III, Summon Monster III, Summon Nature's Ally III, Summon Undead III, Vipergout, Wall of Ice}}

\ability{4th level:}{\emph{Dimension Door, Dismissal, Drawmij's Instant Summons, Lesser planar ally, lesser planar binding, Summon Frost Beast IV, Summon Monster IV, Summon Nature's Ally IV, Summon Undead IV, Wall of Fire}}

\ability{5th level:}{\emph{Dimension Shuffle, Dimensional Lock, Evard's Black Tentacles, Greater Dimension Door, Planeshift, Summon Frost Beast V, Summon Monster V, Summon Nature's Ally V, Summon Undead V, Teleport, Wall of Iron, Wall of Stone, Word of Recall}}

\ability{6th level:}{\emph{Antipathy, Greater Planeshift, Planar Ally, Planar Binding, Planar Bubble, Summon Frost Beast VI, Summon Monster VI, Summon Nature's Ally VI, Sympathy, Wall of Thorns}}

\ability{7th level:}{\emph{Binding, Forcecage, Greater Teleport, Maze, Shadowwalk, Summon Frost Beast VII, Summon Monster VII, Summon Nature's Ally VII, Teleport Object}}

\ability{8th level:}{\emph{Freedom, Greater Planar Ally, Greater Planar Binding, Shades, Summon Frost Beast VIII, Summon Golem, Summon Monster VIII, Summon Nature's Ally VIII}}

\ability{9th level:}{\emph{Elemental Swarm, Gate, Imprisonment, Refuge, Summon Elemental Monolith, Summon Frost Beast IX, Summon Monster IX, Summon Nature's Ally IX, Teleportation Circle, Unbinding}}
\normalsize


\section{Prestige Classes}

For something that has received so much ink, the world of fiend related prestige classes is remarkably non-functional. Fiendish Cultists, Dark Summoners, and Fiend-blooded Sorcerers are classic D\&D fodder. But unfortunately, the previously published classes for these archetypes are generally� not good. And that makes us sad. Here are some classes designed to fill those perceived holes:

\classname{Hellwalker}
\vspace*{-8pt}
\quot{"What terrors do you think I have not already seen?"}

Life in the Lower Planes is generally nasty, brutish, and short, and it is only the strongest survive long enough to learn the mysteries and unique properties of these locales. The Hellwalker is one such individual. Trained in the Lower Planes and weaned on the taint of evil, these solitary hunters roam the Lower Planes, traveling along planar byways and through networks of portals too dangerous or erratic for regular use.

\ability{Requirements}{}
\listprereq
%To qualify to become a Hellwalker, a character must fulfill all the following criteria.
\itemability{Skills:}{Knowledge (Planes) 4 ranks, Survival 8 ranks.}
\itemability{Feats:}{Track}
\itemability{Special:}{Must have visited every Lower Plane, and lived at least one year in the Lower Planes.}
\end{list}\vspace*{8pt}

\ability{Hit Die:}{d8}

\ability{Class Skills:}{The Hellwalker's class skills (and the key ability for each skill) are Balance (Dex), Climb (Str), Concentration (Con), Diplomacy (Cha), Handle Animal (Cha), Hide (Dex), Knowledge (geography) (Int), Knowledge (planes) (Int), Listen (Wis), Move Silently (Dex), Profession (Wis), Ride (Dex), Speak Language, Spellcraft(Int), Spot (Wis), and Survival (Wis).}

\ability{Skill Points at Each Level:}{4 + Int modifier.}

\begin{table}[tbh]
\begin{small}
\begin{tabular}{lp{1.9cm}p{0.7cm}p{0.7cm}p{0.7cm}p{6cm}l}
Level  &Base Attack Bonus &Fort Save &Ref Save &Will Save &Special &Spellcasting\\
1st &+0 &+2 &+2 &+0 &Members Only &+1 spellcaster level\\
2nd &+1 &+3 &+3 &+0 &Craft of the Lower Planes & --\\
3rd &+2 &+3 &+3 &+1 &Free Traveller's Checks &+1 spellcaster level\\
4th &+3 &+4 &+4 &+1 &Craft of the Lower Planes & --\\
5th &+3 &+4 &+4 &+1 &Membership has Privileges, Skill of War &+1 spellcaster level\\
\end{tabular}
\end{small}
\end{table}


\smallskip\noindent All of the following are Class Features of the Hellwaker prestige class.

\ability{Weapon and Armor Proficiency:}{Hellwalkers gain no proficiency with any weapon or armor.}

\ability{Spellcasting:}{At levels 1, 3, and 5, the Hellwalker casts spells (including gaining any new spell slots and spell knowledge) as if he had also gained a level in a spellcasting class he had previous to gaining that level. If the character does not have any levels in any spellcasting classes when he takes his first level of Hellwalker, this class feature gives him levels in Sorcerer spellcasting.}

\ability{Members Only(Ex):}{At 1st level, a Hellwalker attunes himself to energies of the Lower Planes. While on the Lower Planes, he counts as if under the effect of a \spell{Attune Form} spell. If not currently on a Lower Plane, he may cast \spell{plane shift} at will, with his target destination being a location on a Lower Plane.}

\ability{Craft of the Lower Planes:}{At 2nd and 4th level, the Hellwalker learns the martial or mystical arts of the Lower Planes. He can choose to gain a +1 to spellcasting level, or a +1d6 to sneak attack dice (but only if he has sneak attack dice from another class), or a [Combat] feat that he qualifies for. Once chosen, this ability cannot be changed.}

\ability{Free Traveller's Checks(Sp):}{At 3rd level, the Hellwalker has so attuned himself to the energies of the Lower Planes that he can call this energy to himself, ferrying himself with it when the energy is strong and drawing it to him when it does not exist. While on a Lower Plane, the Hellwalker can cast \spell{greater teleport} at will as a spell-like ability, with a limit of 50 pounds over his own weight. While not on a Lower Plane, a Hellwalker can cast \spell{planar bubble} on himself, at will, and he can choose to be a native of any one Lower Plane (chosen each time this effect is used) for the purposes of this spell.}

\ability{Membership has Privileges(Sp):}{A Hellwalker of 5th level has tapped into a rarely used and poorly maintained network of portals dotting the Lower Planes, and he can manipulate those portals for his own purposes. The Hellwalker may cast \spell{gate} (travel version only) as a spell-like ability once per day. Unlike a normal \spell{gate}, only living creatures and attended objects may pass through this gate. This ability can only be used if at least one side of the gate is on a Lower Plane.}

\ability{Skill of War:}{At 5th level, the Hellwalker's Base Attack Bonus increases by 1 permanently.}

\classname{Seer of the Tempest}
\vspace*{-8pt}
\quot{``Aaaah! AAAAAAAH! AAAAAAH!"}

The Windswept Depths of Pandemonium breed a special kind of magician. Driven half-mad by the eternal darkness and thunderous din of the eternal windstorm, they walk the battlefields of the multiverse bringing the terrors they know to those who do not.

Most characters who become Seers of the Tempest are Warmages, though sometimes this path appeals to Wizards of a particularly martial bent.

\ability{Requirements:}{}
\listprereq
%To qualify to become a Seer of the Tempest, a character must fulfill all the following criteria:
\itemability{Skills:}{Intimidate 9 ranks.}
\itemability{Alignment:}{Non-Lawful}
\itemability{Spells:}{Must be able to cast 3rd level Arcane spells, and must be able to cast Evocations of every level she can cast.}
\end{list}\vspace*{8pt}

\ability{Hit Die:}{d4}

\ability{Class Skills:}{The Seer of the Tempest's class skills (and the key ability for each skill) are Concentration (Con), Craft (Int), Intimidate (Cha), Knowledge (all skills taken individually) (Int), Profession (Wis), and Spellcraft (Int).}

\ability{Skills/Level:}{2 + Intelligence Bonus}

\begin{table}[tbh]
\begin{small}
\begin{tabular}{lp{1.9cm}p{0.7cm}p{0.7cm}p{0.7cm}p{6cm}l}
Level  &Base Attack Bonus &Fort Save &Ref Save &Will Save &Special &Spellcasting\\
1st &+0 &+0 &+0 &+2 &Increased Edge, Sonic Spells, Suffer the Firmament &+1 spellcaster level\\
2nd &+1 &+0 &+0 &+3 &Scary Noises &+1 spellcaster level\\
3rd &+2 &+1 &+1 &+3 &Winds &+1 spellcaster level\\
4th &+3 &+1 &+1 &+4 &Attune Domain: Storm &+1 spellcaster level\\
5th &+3 &+1 &+1 &+4 &Swift Winds &+1 spellcaster level\\
6th &+4 &+2 &+2 &+5 &Uproar &+1 spellcaster level\\
7th &+5 &+2 &+2 &+5 &Sense the Winds &+1 spellcaster level\\
8th &+6 &+2 &+2 &+6 &Attune Domain: Darkness &+1 spellcaster level\\
\end{tabular}
\end{small}
\end{table}


\smallskip\noindent All of the following are Class Features of the Seer of the Tempest class.

\ability{Weapon and Armor Proficiency:}{The Seer of the Tempest gains no proficiency with armor or weapons.}

\ability{Spellcasting:}{Every level, the Seer of the Tempest casts spells (including gaining any new spell slots and spell knowledge) as if she had also gained a level in a spellcasting class she had previous to gaining that level.}

\ability{Increased Edge:}{If the Seer of the Tempest has the Edge class feature, she may add her class level to the Edge bonus. This has no effect if she does not have the Edge class feature.}

\ability{Sonic Spells:}{Every spell that the Seer of the Tempest casts that inflicts any damage may inflict Sonic damage instead at her option when the spell is cast. No other effects of the spell are changed, nor is the casting time of the spell. Spells that are made to inflict Sonic damage in this way gain the [Sonic] descriptor.}

\ability{Suffer the Firmament (Ex):}{A Seer of the Tempest is at home in a tornado as a calm day. She is unaffected by winds of any strength, and suffers no penalties to her listen checks in even the stiffest gale.}

\ability{Scary Noises (Su):}{At 2nd level, the booming destruction of the Seer's magical attacks channels the haunting terrors of Pandemonium. Any creature damaged by a [Sonic] spell cast by the Seer must make a Willpower Save or become shaken. If a creature becomes shaken twice, she becomes frightened. The Save DC is Charisma based, and this is considered a [Fear] effect.}

\ability{Winds (Sp):}{A Seer of the Tempest who has reached 3rd level may use \spell{gust of wind} as a spell-like ability at will. At 5th level, using this ability becomes an Immediate action.}

\ability{Attune Domain:}{At 4th level, the Seer of the Tempest gains Attune Domain (Storm Domain) as a bonus feat. At 8th level she gains Attune Domain (Darkness Domain) as a bonus feat.}

\ability{Uproar (Su):}{As a free action, a Seer of the Tempest of 6th level can attempt to Intimidate all creatures she can target within line of effect that are within an area afflicted by Severe (or worse) winds.}

\ability{Sense the Winds (Su):}{At 7th level, a Seer of the Tempest can feel the disturbances caused by creatures and objects within the wind. She can perceive and target any creature or object within 120 feet of herself so long as there is at least a Light Breeze.}

\classname{Initiate of the Black Tower}
\vspace*{-8pt}
\quot{``Bow before your new Master!"}

Fiends and mortals have played the slavery game for eons, each tempting the other with promises of greater power and threats of greater tortures. In this game, there are favored playing pieces, and they are The Initiates of the Black Tower. These fiend-trained conjurers are weapons in the Blood War, aimed at rival fiends and upstart mortals who have little respect for the scions of the Lower Planes. In exchange for otherworldly secrets, these casters pay a terrible price in service to the fiends of the Black Tower.

\ability{Requirements:}{}
\listprereq %To qualify to become an Initiate of the Black Tower, a character must fulfill all the following criteria.
\itemability{Skills:}{Knowledge (planes) 10 ranks, Spellcraft 10 ranks, Diplomacy 4 ranks, Intimidate 4 ranks.}
\itemability{Feats:}{Broker of the Infernal}
\itemability{Spells:}{Ability to cast 5th-level arcane spells, and the spell \spell{lesser planar binding}}
\itemability{Special:}{Must be a native of the Prime Material Plane, and must either specialize in the conjuration school or have one [summoning] or [calling] spell known at each spell level available.}
\end{list}\vspace*{8pt}

\ability{Hit Die:}{d4}

\ability{Class Skills:}{The Initiate of the Black Tower's class skills (and the key ability for each skill) are Bluff (Cha), Concentration (Con), Craft (Int), Diplomacy(Cha), Intimidate(Cha), Knowledge (all skills taken individually) (Int), Profession (Wis), Search (Int), and Spellcraft (Int).}

\ability{Skill Points at Each Level:}{2 + Int modifier.}


\begin{table}[tbh]
\begin{small}
\begin{tabular}{lp{1.9cm}p{0.7cm}p{0.7cm}p{0.7cm}p{6cm}l}Level  &Base Attack Bonus &Fort Save &Ref Save &Will Save &Special &Spellcasting\\
1st &+0 &+0 &+0 &+2 &Access to the Registrar, Binding Sense &+1 spellcaster level\\
2nd &+1 &+0 &+0 &+3 &Dispel the Chains of Service &+1 spellcaster level\\
3rd &+1 &+1 &+1 &+3 &Marks of the Black Tower &+1 spellcaster level\\
4th &+2 &+1 &+1 &+4 &True Scrying &+1 spellcaster level\\
5th &+2 &+1 &+1 &+4 &Usurp Services &+1 spellcaster level\\
6th &+3 &+2 &+2 &+5 &Share Name &+1 spellcaster level\\
7th &+3 &+2 &+2 &+5 &Calling Ruse &+1 spellcaster level\\
8th &+4 &+2 &+2 &+6 &Unnamed &+1 spellcaster level\\
9th &+4 &+3 &+3 &+6 &Recall the Servant &+1 spellcaster level\\
10th &+5 &+3 &+3 &+7 &Master of the Black Tower &+1 spellcaster level\\
\end{tabular}
\end{small}
\end{table}


\smallskip \noindent All of the following are Class Features of the Initiate of the Black Tower prestige class.

\ability{Weapon and Armor Proficiency:}{Initiates of the Black Tower gain no proficiency with any weapon or armor.}

\ability{Spellcasting:}{Every level, the Initiate casts spells (including gaining any new spell slots and spell knowledge) as if he had also gained a level in a spellcasting class he had previous to gaining that level.}

\ability{Access to the Registrar:}{At 1st level, the Initiate gains access to The Registrar, a library of True Names of extraplanar beings. For each level in this class, he learns one True Name of a creature of a CR equal to his character level at the time this ability is gained, minus two.}

\ability{Binding Sense(Su):}{At 1st level, an Initiate of the Black Tower may make a Spellcraft check to identify a [calling] or [summoning] effect as if it was a spell being cast whenever he come within of 100' of such an effect. Also, he is alerted whenever a creature whose True Name he knows is called or summoned.}

\ability{\spell{Dispel the Chains of Service}:}{at 2nd level, this spell is added to the Initiate of the Black Tower's list of spells known as a 3rd level spell. This spell functions in all ways as \spell{Dispel Evil}, but it affects [calling] and [summoning] effects instead of evil effects.}

\ability{Marks of the Black Tower(Ex):}{By scratching runes and magical diagrams into the walls, ceilings, and floors of a building, an Initiate of the Black Tower of 3rd level can ward an area from the effects of [calling] and [summoning] spells used by other casters. No creature can be called from the area unless a True Name is used, and no other caster can use [summoning] spells in the area. To create these marks, chalks or paints must be used to inscribe the marks, and a Spellcraft check must be made (DC equals 10 + 5 for every 20' radius). Up to one 10' by 10' area in each Marked area can be free of chalk or paint and still have the Marks be effective. If any markings are disturbed, the effect ends. It takes 10 minutes per 10' radius to scribe these Marks.

If the character uses 100 GP in magical reagents and silver per 10' radius, these markings can be made more difficult to disturb. A Disable Device check is needed to disturb these Marks (DC 25).}

\ability{True Scrying (Sp):}{At 4th level, the Initiate gains the ability to scry on creatures whose True Name he knows as a spell-like ability usable at-will. By casting this effect on any reflective service, the Initiate of the Black Tower can \spell{Scry} on any creature whose True Name he knows with no chance of failure. This also gives enough information to cast \spell{teleport} or other transportation magic as if the location of creature was a well known location.}

\ability{Usurp Services (Sp):}{At 5th level, an Initiate of the Black Tower may make a caster level check as swift action to Usurp a [calling] or [summoning] effect. If this check is successful, the Initiate becomes the caster of the spell for that effect, gaining services, control over summoned monsters, etc.}

\ability{Share Name (Sp):}{At 6th level, the Initiate of the Black Tower may Share Spells with a creature whose True Name he knows as if the creature was a Familiar of the Initiate.}

\ability{Calling Ruse:}{At 7th level, the Initiate of the Black Tower may substitute himself whenever a creature whose True Name he knows is called by a summoning or calling effect. Instead of the creature, he appears and he has three rounds to act as he pleases before the spell affects him as if he were the original creature called.}

\ability{Unnamed (Sp):}{At 8th level, the Initiate of the Black Tower's knowledge of True Names has become so great that he has learned a way to remove True Names from creatures without removing them from existence. He may use this ability once per month, and for all effects, they no longer have a True Name, and can no longer be a target for [summoning] or [calling] spells.}

\ability{Recall the Servant (Sp):}{At 9th level, the Initiate of the Black Tower gains the ability to cast a special version of \spell{gate} to call any creature whose True Name he knows. This may be used three times per day, and it ends any preexisting [calling] or [summoning] effects.}

\ability{Master of the Black Tower:}{At 10th level, the Initiate of the Black Tower has mastered the secrets of the Black Tower. He may now use his Unnamed ability once per day, except that when he uses this ability he can instead takes their True Names. Taking a True Name means that these creatures are now immune to any [calling] or [summoning] effects, except for effects cast by the Initiate.}

\classname{Barrister of the Nine}
\vspace*{-8pt}
\quot{``Perhaps unfortunately, your contract actually does not allow that."}

The rules governing planar contracts are inviolate, but they are not written down in their entirety in any book of laws or treatise on the planes. Only the most agile, creative, and logical of minds can grasp the entirety of the mystical laws governing contracts between the planes, and these individual find service among power brokers and demon princes of the Nine Hells. If they survive this term of service, they eventually learn all the ways to abuse and exploit the unclear language and loopholes in the contract agreements of the planes.

\ability{Requirements:}{}
\listprereq
%To qualify to become an Barrister of the Nine, a character must fulfill all the following criteria.
\itemability{Skills:}{Knowledge (planes) 10 ranks, Profession (Infernal Lawyer) 10 ranks, Diplomacy 4 ranks, Intimidate 4 ranks.}
\itemability{Feats:}{Broker of the Infernal, Apprentice(Devil)}
\itemability{Spells:}{Ability to cast 5th-level arcane spells, and the spell \spell{lesser planar binding}}
\itemability{Special:}{Must be a native of the Prime Material Plane}
\itemability{Special:}{Must be lawful and cannot be good.}
\end{list} \vspace*{8pt}

\ability{Hit Die:}{d4}

\ability{Class Skills:}{The Barrister of the Nine class skills (and the key ability for each skill) are Bluff(Cha), Concentration (Con), Craft (Int), Diplomacy (Cha), Intimidate(Cha), Knowledge (all) (Int), Profession (Wis), and Spellcraft (Int).}

\ability{Skills/Level:}{4 + Intelligence Bonus}

\begin{table}[tbh]
\begin{small}
\begin{tabular}{lp{1.9cm}p{0.7cm}p{0.7cm}p{0.7cm}p{6cm}l}
Level  &Base Attack Bonus &Fort Save &Ref Save &Will Save &Special &Spellcasting\\
1st &+0 &+0 &+0 &+2 &Seal the Contract, Corner Office, Immunity Deal &+1 spellcaster level\\
2nd &+1 &+0 &+0 &+3 &Penalized &+1 spellcaster level\\
3rd &+1 &+1 &+1 &+3 &Pacts: Stability Pact &+1 spellcaster level\\
4th &+2 &+1 &+1 &+4 &Contractual Obligations &+1 spellcaster level\\
5th &+2 &+1 &+1 &+4 &Pacts: Renewal Pact &+1 spellcaster level\\
6th &+3 &+2 &+2 &+5 &Proof of Payment &+1 spellcaster level\\
7th &+3 &+2 &+2 &+5 &Pacts: Death Pact &+1 spellcaster level\\
8th &+4 &+2 &+2 &+6 &Inheritance Clause &+1 spellcaster level\\
9th &+4 &+3 &+3 &+6 &Pacts: Contingency &+1 spellcaster level\\
10th &+5 &+3 &+3 &+7 &Loophole &+1 spellcaster level\\
\end{tabular}
\end{small}
\end{table}


\smallskip\noindent All of the following are Class Features of the Barrister of the Nine prestige class. Any use of the Barrister of the Nine's class abilities (except Corner Office) creates a written contract. If this written contract is destroyed, the spell modified by the Barrister's class ability ends.

\ability{Weapon and Armor Proficiency:}{Barristers of the Nine gain no proficiency with any weapon or armor.}

\ability{Spellcasting:}{Every level, the Barrister casts spells (including gaining any new spell slots and spell knowledge) as if he had also gained a level in a spellcasting class he had previous to gaining that level.}

\ability{Seal the Contract(Su):}{At first level, a Barrister of the Nine may use any of the planar binding spells to seal an agreement between any two individuals. While the Barrister must cast the spell, one of the two individuals becomes the caster for the purposes of services owed and payments.}

\ability{Corner Office (Sp):}{At 1st level, the Barrister gains an extraplanar office for his dealings. This effects may be used at will, and is in effect a magnificent mansion, except that it is the same location between castings, meaning that when the duration ends this extradimensional space is inaccessible, but items and creatures still be there when this ability is used again (effectively trapped unless planar travel magic is used) . This location counts as the Prime for calling spells, and has a permanent calling circle and an unchanging layout.}

\ability{Immunity Deal (Ex):}{A Barrister of the Nine is immune to energy drain and wisdom drain.}

\ability{Penalized:}{At 2nd level, the Barrister of the Nine has learned to incorporate double-talk and legal trickery in his calling spells, meaning that if called creatures survive to complete their service(s), they must return half the GP value treasure used to buy their services. Note: If a creature accepted the caster's services in return for their service, they must instead pay half the value of such services in GP or magic items.}

\ability{Pacts (Sp):}{At 3rd level, the Barrister of the Nine can cast \spell{Stability Pact} as a spell-like ability one per day.
At 5th level, he may cast \spell{Renewal Pact} once per day as a spell-like ability.
At 7th level, he may cast \spell{Death Pact} once per day as a spell-like ability.
At 9th level, he may cast \spell{Contingency} once per day as a spell-like ability.}

\ability{Contractual Obligations:}{When making a bargain with a extraplanar being, the Barrister can gain one additional service for every +5 added to the DC of bargaining check.}

\ability{Proof of Payment:}{At 6th level, the Barrister of the Nine has learned to add in clauses to his planar binding spells that delays payment for services until those services are completed. If the called creature is killed before the service is complete, the Barrister does not need to pay for services.}

\ability{Inheritance Clause:}{A Barrister of the Nine of 8th level can transfer services owed him by creatures he has bound to other creatures, even if those creatures are not present at the time a calling spell is cast.}

\ability{Loophole:}{At 10th level, the Barrister of the Nine has learned to word his agreements is such a way as to avoid payment. He no longer needs to pay for any agreements made with his calling spells.}

\classname{Celestial Beacon}\label{class:celestialbeacon}

The Lower Planes are realms of evil of such magnitude that the good are oppressed in such a place. Those whose goodness is so strong that it radiates from them are beacons to the denizens of these realms, and some have learned to harness this light to burn away the unclean presence of the fiends.

\ability{Requirements}{}
\listprereq
%To qualify to become a Celestial Beacon, a character must fulfill all the following criteria.
\itemability{Skills:}{Knowledge (Planes) 4 ranks, Knowledge (Religion) 9 ranks.}
\itemability{BAB:}{+5}
\itemability{Special:}{Must radiate moderate good.}
\itemability{Special:}{Must be proficient in heavy armor.}
\end{list}\vspace*{8pt}

\ability{Hit Die:}{d8}

\ability{Class Skills:}{The Celestial Beacon's class skills (and the key ability for each skill) are Concentration (Con), Craft (Int), Diplomacy (Cha), Intimidate(Cha), Heal(Wis), Knowledge (Religion) (Int), Profession (Wis), and Spellcraft (Int).}

\ability{Skills/Level:}{2 + Intelligence Bonus}

\begin{table}[tbh]
\begin{small}
\begin{tabular}{lp{1.9cm}p{0.7cm}p{0.7cm}p{0.7cm}p{6cm}l}
Level  &Base Attack Bonus &Fort Save &Ref Save &Will Save &Special &Spellcasting\\
1st &+1 &+2 &+0 &+2 &Aura of Good, Smite Evil &+1 spellcaster level\\
2nd &+2 &+3 &+0 &+3 &Halo of the Righteous &+1 spellcaster level\\
3rd &+3 &+3 &+1 &+3 &Arms of the Holy &+1 spellcaster level\\
4th &+4 &+4 &+1 &+4 &Armament of the Holy &+1 spellcaster level\\
5th &+5 &+4 &+1 &+4 &Death Ward &+1 spellcaster level\\
6th &+6 &+5 &+2 &+5 &Flare of Goodness &+1 spellcaster level\\
7th &+7 &+5 &+2 &+5 &Light of Peace &+1 spellcaster level\\
8th &+8 &+6 &+2 &+6 &Holyfire Shield &+1 spellcaster level\\
9th &+9 &+6 &+3 &+6 &Celestial Aspect &+1 spellcaster level\\
10th &+10 &+7 &+3 &+7 &Ward Against Evil Magic &+1 spellcaster level\\
\end{tabular}
\end{small}
\end{table}


\smallskip\noindent All of the following are Class Features of the Celestial Beacon prestige class.

\ability{Weapon and Armor Proficiency:}{Celestial Beacons gain no proficiency with any weapon or armor, but do gain proficiency in the Tower Shield.}

\ability{Aura of Good (Ex):}{The power of a Celestial Beacon's aura of good (see the \spell{detect good} spell) is equal to her Celestial Beacon level. This stacks with any other Aura of Good Ability gained from other sources.}

\ability{Smite Evil (Su):}{As a free action, a Celestial Beacon may attempt to smite evil with one normal melee attack. She adds her Charisma bonus (if any) to her attack roll and deals 1 extra point of damage per Celestial Beacon level. If the Celestial Beacon accidentally smites a creature that is not evil, the smite has no effect, but a use of the ability is expended for that day. The Celestial Beacon may use this ability once for every level of Celestial Beacon, and uses per day and bonuses of this effect stacks with any Smite Evil gained from other classes.}

\ability{Halo of the Righteous(Su):}{At 2nd level, the Celestial Beacon emanates a \spell{Magic Circle Against Evil} effect, as the spell.}

\ability{Arms of the Holy:}{At 3rd level, any melee attack performed by a Celestial Beacon counts as good-aligned for the purposes by bypassing damage reduction.}

\ability{Armament of the Holy (Su):}{At 4th level, any armor worn by the Celestial Beacon takes on a silvery of golden shine, and it is one category lighter than normal for purposes of movement and other limitations. Heavy armors are treated as medium, and medium armors are treated as light, but light armors are still treated as light. Spell failure chances for armors and shields worn by a Celestial Beacon are decreased by 10\%, maximum Dexterity bonus is increased by 2, and armor check penalties are lessened by 3 (to a minimum of 0). The Celestial Beacon gains DR 5/evil}

\ability{Death Ward(Su) :}{At 5th level, the Celestial Beacon is continuously under the effects of a \spell{death ward}, as the spell. The irises of his eyes become gold or silver.}

\ability{Flare of Goodness (Su):}{At 6th level, the Celestial Beacon may perform a Flare of Goodness as an immediate action. This has the effects of a \spell{Sunburst} spell, and any evil magic in the radius is automatically dispelled. After using this ability, all class features gained from Celestial Beacon levels cease functioning for 1d4+1 rounds.}

\ability{Light of Peace:}{Upon reaching 7th level, the Celestial Beacon is continuously under the effects of a \spell{sanctuary} spell. If the Celestial Beacon attacks during a round, this effect ends for 1d4 rounds, then renews itself.

The Celestial Beacons also gains the ability to shine as brightly as a torch. He may suppress or renew this ability as a swift action.}

\ability{Holyfire Shield (Sp):}{As a swift action, an 8th level Celestial Beacon can cast use a spell-like ability called Holyfire Shield at will. This effect is like a golden-colored \spell{fire shield}, but the damage it inflicts is holy damage and it grants immunity to unholy damage.}

\ability{Celestial Aspect:}{At 9th level, the Celestial Beacon becomes an Outsider, and gains a +2 to Str, +2 to Wis, and +2 to Cha. He may be restored to life according to his previous type.}

\ability{Ward Against Evil Magic:}{At 10th level, the Celestial Beacon gains SR of 15 + character level, but only against [evil] spells and spell-like abilities and the spells and spell-like abilities cast by evil-aligned creatures.}

\classname{Boatman of Styx}

The River Styx is a planar path winding its way along the Lower Planes servicing the myriad planes of evil, and its waters rob mortals and immortals of their memories. Ferrymen ply these waters, offering safe passage and travel between the planes, but most are con-men looking to make a quick silver, predators in disguise, or well-meaning fools. Few beings know the truly safe routes along the River Styx, thus earning the title of boatmen, and even fewer have accepted its nature into their very body.

\ability{Requirements:}{}
%To qualify to become a Boatman of Styx, a character must fulfill all the following criteria.
\listprereq
\itemability{Skills:}{Knowledge (Planes) 8 ranks, Survival 4 ranks, Profession (boatman) 4 ranks.}
\itemability{Feats:}{Quickdraw}
\itemability{Special:}{Must have visited every Lower Plane via the River Styx, and lived at least one year on a boat on the River Styx.}
\itemability{Special:}{+2d6 sneak attack, skirmish, or sudden strike}
\end{list}\vspace*{8pt}


\ability{Hit Die:}{d8}

\ability{Class Skills:}{The Boatman of the Styx's class skills (and the key ability for each skill) are Balance (Dex), Bluff (Cha), Climb (Str), Craft (Int), Decipher Script (Int), Diplomacy (Cha), Disable Device (Int), Disguise (Cha), Escape Artist (Dex), Forgery (Int), Gather Information (Cha), Hide (Dex), Intimidate (Cha), Jump (Str), Knowledge (planes) (Int), Listen (Wis), Move Silently (Dex), Open Lock (Dex), Search (Int), Sense Motive (Wis), Sleight of Hand (Dex), Spot (Wis), Swim (Str), Tumble (Dex), Use Magic Device (Cha), and Use Rope (Dex).}

\ability{Skill Points at Each Level:}{6 + Int modifier.}

\begin{table}[tbh]
\begin{small}
\begin{tabular}{lp{1.9cm}p{0.7cm}p{0.7cm}p{0.7cm}l}
Level  &Base Attack  Bonus &Fort Save &Ref Save &Will Save &Special\\
1st &+0 &+2 &+2 &+0 &Safe Passage, Stygian Blood\\
2nd &+1 &+3 &+3 &+0 &River-born, +1d6 sneak attack dice\\
3rd &+2 &+3 &+3 &+1 &Stygian Tears\\
4th &+3 &+4 &+4 &+1 &River-marked, +2d6 sneak attack dice\\
5th &+3 &+4 &+4 &+1 &Kiss of Lethe\\
6th &+4 &+5 &+5 &+2 &Eternal Boatman, +3d6 sneak attack dice\\
7th &+5 &+5 &+5 &+2 &Bonus Feat\\
8th &+6 &+6 &+6 &+2 &Knowledgeable, +4d6 sneak attack dice\\
9th &+6 &+6 &+6 &+3 &Stygian Tributary\\
10th &+7 &+7 &+7 &+3 &Xirfilstyx Dance, +5d6 sneak attack dice\\
\end{tabular}
\end{small}
\end{table}



\smallskip \noindent All of the following are Class Features of the Boatman of Styx prestige class.

\ability{Weapon and Armor Proficiency:}{A Boatman of Styx gains proficiency in the Long Staff, but does not gain any other weapon or armor proficiency.}

\ability{Safe Passage (Su):}{At 1st level, the Boatman of Styx learns to navigate the River Styx. When leading a boat or ship, the Boatman can lead a craft to any other planar location that contains some portion of the River Styx. This journey takes 1d6 hours, and is uneventful.}

\ability{Stygian Blood(Ex):}{At 1st level, the Boatman of Styx has absorbed small amounts of the River Styx into his blood, and from this point forward he is immune to mind-affecting effects and the effects of the River Styx.}

\ability{River-born:}{At 2nd level, the Boatman of Styx gains a swim speed of 60', and he may take 10 on any Swim check. He also does not need to breathe while submersed in water.}

\ability{Sneak Attack:}{A Boatman of Styx gains an additional die of Sneak Attack at every even numbered level.}

\ability{Stygian Tears:}{At 3rd level, the Boatman of Styx's tears gain the effects of the River Styx. He may collect up to one vial of tears per day. Being struck by a vial of these tears has the same effects as an exposure to the waters of the River Styx (DM's option). If kept stoppered, these vials of tears last indefinitely.}

\ability{River-marked:}{At 4th level, the Boatman of Styx's continues exposure to the River Styx grants a +8 to Swim checks, a +6 to Disguise checks, and a +4 to Bluff checks.}

\ability{Kiss of Lethe(Su):}{At 5th level, the Boatman of Styx may kiss an enemy on an opposed Grapple check. This kiss has the same effects as an exposure to the River Styx (DMs option).}

\ability{Eternal Boatman(Sp):}{At 6th level, the Boatman of Styx may summon a normal wooden boat once per day. This boat is large enough to carry the Boatman and up to eight Medium sized passengers. This boat lasts 24 hours, then falls apart into pieces of rotted wood.}

\ability{Bonus Feat:}{At 7th level, the Boatman of Styx gains a bonus feat. He must meet any prerequisites of this feat to choose it.}

\ability{Knowledgeable:}{At 8th level, the Boatman of Styx's has absorbed a critical mass of memories from swimming in the River Styx. He always has at least 10 ranks in all Knowledges, and may expend skill points to raise these Knowledges as if they were class skills. The Boatman also gains the ability to speak, read, and write any language.}

\ability{Stygian Tributary(Sp):}{At 9th level, the Boatman of Styx learns a secret about the River Styx: it sometimes touches rivers on planes other than the Lower Planes. Once per day, a Boatman of Styx may cast a \spell{gate} (travel version only) as a spell-like ability, but only while standing on a boat in a river. This portal leads to any other river on any plane.}

\ability{Xirfilstyx Dance(Su):}{At 10th level, the Boatman of Styx has studied these fiends of the River Styx and learned a secret fighting art. Any creature struck a Boatman of Styx's sneak attack must make a Will save or be dazed for 1d4 rounds. Any affected creature does not remember any events while dazed.}


\chapter{Fiends With Style}

This section contains new feats and spheres as well as an optional rule to make people assuming the mantle of fiendish power seem a little more� fiendish.

\section{The Feats}

%\newcommand{\babfeat}[7]{\noindent\textbf{#1} \\ \emph{#2} \\ \textbf{Benefit:}#3 \\ \textbf{+1:}#4  \\ \textbf{+6:}#5 \\ \textbf{+11:}#6 \\ \textbf{+16:}#7 \bigskip }
%\newcommand{\skillfeat}[7]{\noindent\textbf{#1} \\ \emph{#2} \\ \textbf{Benefit:} #3 \\ \textbf{4:} #4  \\ \textbf{9:} #5 \\ \textbf{14:} #6 \\ \textbf{19:} #7 \bigskip }

\section{The Failure of Feats}
\vspace*{-10pt}
\quot{``How about instead of being able to travel anywhere in the multiverse, transform yourself into anything you can think of, stop time, and slay everyone you can see, we just give a nice +1 to hit with your secondary weapon? Deal?''}

\noindent\desc{Feats were an interesting idea when they were ported to 3rd edition D\&D. But let's face it; they don't go nearly far enough. Feats were made extremely conservative in their effects on the game because the authors didn't want to offend people with too radical a change. Well, now we've had third edition for 6 years, and we're offended. Feats are an interesting and tangible way to get unique abilities onto a character, but they have fallen prey to two key fallacies that has ended up turning the entire concept to ashes in our mouths. The first is the idea that if you think of something kind of cool for a character to do, you should make it a feat. That sounds compelling, but you only get 7 feats in your whole life. If you have to spend a feat for every cool thing you ever do, you're not going to do very many cool things in the approximately 260 encounters you'll have on your way from 1st to 20th level. The second is the idea that a feat should be equivalent to a cantrip or two. This one is even less excusable, and just makes us cry. A +1 bonus is something that you seriously might forget that you even have. Having one more +1 bonus doesn't make your character unique, it makes you a sucker for spending one of the half dozen feats you'll ever see on a bonus the other players won't even mention when discussing your character.}

\noindent\desc{We all understand this problem, what do we do about it? Well, for starters, Feats have to do more things. Many characters are 5th level or so and they only have 2 feats. Those feats should describe their character in a much more salient way than ``I'm no worse shooting into melee than I am shooting at people with cover that isn't my friends.'' This was begun with the tactical feats, but it didn't go far enough. It's not enough to add additional feats that do something halfway interesting for high level characters to have -- we actually have to replace the stupid one dimensional feats in the PHB with feats that rational people would care about in any way. Spending a single feat should be enough to make you a ``sniper character'' because for a substantial portion of your life you only get one feat. Secondly, we have to clear away feats that don't provide numeric bonuses large enough to care about. The minimum bonus you'll ever notice is +3, because that's actually larger than the difference between having rolled well and having rolled poorly on your starting stats. Numeric bonuses smaller than that are actually insulting and need to be removed from the feats altogether. 3.5 Skill Focus was a nice start, but that's all it was -- a start.}

\noindent{Furthermore, the fundamental structure of feats has been a disaster. The system of prerequisites often ensures that characters won't get an ability before it would be level appropriate for them to do so, but actually does nothing to ensure that such characters are in fact getting level appropriate abilities. Indeed, if a 12th level character decides that they want to pursue a career in shooting people in the face, they have to start all over gaining an ability that is supposed to be level appropriate for a 1st level character. Meanwhile, when a wizard of 12th level decides to pursue some new direction in spellcasting -- he learns a new 6th level spell right off -- and gets an ability that's level appropriate for a 12th level character.}

\subsection{Exploits}

\noindent\desc{Getting proficiency with a weapon isn't worth a feat. They hand that crap out with your character class for free. Seriously, even exotic weapon proficiencies aren't a big deal. Therefore, we're instituting Exploits as something that can be acquired in-game. These are for any of the binary abilities that simply don't have a massive impact on your character's performance at any level.}

\noindent\desc{If you have Martial Weapon Proficiency, it's really unreasonable for it to be that hard to learn how to use a new weapon, whether it's exotic or not. If you spend a week training with a weapon, you can make an Int check (DC 10) to simply gain the Exploit of Exotic Weapon Proficiency. And no, you can't take 10 on that.}

\noindent{If you don't have Martial Weapon Proficiency and you want to use a new weapon, that's touchier. But if you have a weapon for an entire level, you should just gain proficiency in it when you gain your next level whatever level you happen to select.}

\subsection{The New Feat System}

\noindent{So where are we going with this? First of all, feat chains are gone. That seemed like a good idea, but it wasn't. Secondly, the vast majority of feats don't have prerequisites at all, they scale. A [Combat] feat scales to your Base Attack Bonus, a [Skill] feat scales to your ranks in a skill, and a [Metamagic] feat scales to the highest level spell you can cast. And that's because those are the only things in the game that actually have anything to do with the level your character is in any way that we feel good about.}

\section{The New Combat Ready Feats} \label{feats:combat}

\begin{multicols}{2}
\hypertarget{feat:blindfighting}{}\babfeat{Blind Fighting [Combat]}{
You don't have to see to kill.}{
You may reroll your miss chances caused by concealment.}{
While in darkness, you may move your normal speed without difficulty.}{
You have Blindsense out to 60', this allows you to know the location of all creatures within 60'.}{
You have Tremorsense out to 120', this allows you to ``see'' anything within 120' that is touching the earth.}{
You cannot be caught flat footed.}

\hypertarget{feat:blitz}{}\babfeat{Blitz [Combat]}{
You go all out and try to achieve goals in a proactive manner.}{
While charging, you may opt to lose your Dexterity Bonus to AC for one round, but inflicting an extra d6 of damage if you hit.}{
You may go all out when attacking, adding your Base Attack Bonus to your damage, but provoking an Attack of Opportunity.}{
Bonus attacks made in a Full Attack for having a high BAB are made with a -2 penalty instead of a -5 penalty.}{
Every time you inflict damage upon an opponent with your melee attacks, you may immediately use an Intimidate attempt against that opponent as a bonus action.}{
You may make a Full Attack action as a Standard Action.}

\hypertarget{feat:combatlooting}{}\babfeat{Combat Looting [Combat]}{
You can put things into your pants in the middle of combat.}{
You may sheathe or store an object as a free action.}{
You get a +3 bonus to  \hyperlink{combat:disarm}{Disarm} attempts. Picking up objects off the ground does not provoke an attack of opportunity.}{
As a Swift action, you may take a ring, amulet/necklace, headband, bracer, or belt from an opponent you have successfully \hyperlink{combat:grapple}{grappled}. You may pick up an item off the ground in the middle of a move action.}{
If you are grappling with an opponent, you may activate or deactivate their magic items with a successful Use Magic Device check. You may make Appraise checks as a free action.}{
You can take 10 on Use Magic Device and Sleight of Hand checks.}

\vspace{100pt}
\hypertarget{feat:combatschool}{}\babfeat{Combat School [Combat]}{
You are a member of a completely arbitrary fighting school that has a number of recognizable signature fighting moves.}{
First, name your fighting style (such as ``Hammer and Anvil Technique'' or ``Crescent Moon Style'', or ``Way of the Lightning Mace''). This fighting style only works with a small list of melee weapons that you have to describe the connectedness to the DM in a half-way believable way. Now, whenever you are using that technique in melee combat, you gain a +2 bonus on attack rolls.}{
Your immersion in your technique gives you great martial prowess, you gain a +2 to damage rolls in melee combat.}{
When you strike your opponent with the signature moves of your fighting school in melee, they must make a Fortitude Save (DC 10 + 1/2 your level + your Strength bonus) or become dazed for one round.}{
You may take 10 on attack rolls while using your special techniques. The DC to disarm you of a school-appropriate weapon is increased by 4.}{
You may add +5 to-hit on any one attack you make after the first each turn. If you hit an opponent twice in one round, all further attacks this round against that opponent are made with The Edge.}

\hypertarget{feat:command}{}\babfeat{Command [Combat] [Leadership]}{
You lead tiny men.}{
You have a Command Rating equal to your Base Attack Bonus divided by five (round up).}{
You can muster a group of followers. Your leadership score is your Base Attack Bonus plus your Charisma Modifier.}{
You are able to delegate command to a loyal cohort. A cohort is an intelligent and loyal creature with a CR at least 2 less than your character level. Cohorts gain levels when you do.}{
With a Swift Action you may rally troops, allowing all allies within medium range of yourself to reroll their saves vs. Fear and gain a +2 Morale Bonus to attack and damage rolls for 1 minute. This is a language-dependent ability that may be used an unlimited number of times.}{
Your allies gain a +2 morale bonus to all saving throws if they can see you and you are within medium range.}


\hypertarget{feat:dangersense}{}\babfeat{Danger Sense [Combat]}{
Maybe Spiders tell you what's up. You certainly react to danger with uncanny effectiveness.}{
You get a +3 bonus on Initiative checks.}{
For the purpose of Search, Spot, and Listen, you are always considered to be ``actively searching''. You also get Uncanny Dodge.}{
You may take 10 on Listen, Spot, and Search checks.}{
You may make a Sense Motive check (opposed by your opponent's Bluff check) immediately whenever any creature approaches within 60' of you with harmful intent. If you succeed, you know the location of the creature even if you cannot see it.}{
You are never surprised and always act on the first round of any combat.}


\hypertarget{feat:elusivetarget}{}\babfeat{Elusive Target [Combat]}{
You are very hard to hit when you want to be.}{
You gain a +2 Dodge bonus to AC.}{
Your opponents do not gain flanking or higher ground bonuses against you.}{
Your opponents do not inflict extra damage from the \hyperlink{combat:powerattack}{Power Attack} option.}{
Diverting Defense -- As an immediate action, you may redirect an attack against you to any creature in your threatened range, friend or foe. You may not redirect an attack to the creature making the attack.}{
As an immediate action, you may make an attack that would normally hit you miss instead.}


\hypertarget{feat:experttactician}{}\babfeat{Expert Tactician [Combat]}{
You benefit your allies so good they remember you long time.}{
You gain a +4 bonus when flanking instead of the normal +2 bonus. Your allies who flank with you gain the same advantage.}{
You may \hyperlink{combat:feint}{Feint} as an Immediate action.}{
As a move action, you may make any 5' square adjacent to yourself into difficult ground.}{
For determining flanking with your allies, you may count your location as being 5' in any direction from your real location.}{
You ignore Cover bonuses less than full cover.}


\hypertarget{feat:ghosthunter}{}\babfeat{Ghost Hunter [Combat]}{
You smack around those folks in the spirit world.}{
Your attacks have a 50\% chance of striking incorporeal opponents even if they are not magical.}{
You can hear incorporeal and ethereal creatures as if they lacked those traits (note that shadows and the like rarely bother to actively move silently).}{
You can see invisible and ethereal creatures as if they lacked those traits.}{
Your attacks count as if you had the Ghost Touch property on your weapons.}{
Any Armor or shield you use benefits from the Ghost Touch quality.}

\vspace{100pt}
\hypertarget{feat:giantslayer}{}\babfeat{Giant Slayer [Combat]}{
Everyone has a specialty. Yours is miraculously finding ways to stab creatures in the face when it seems improbable that you would be able to reach that high.}{
When you perform a \hyperlink{combat:grabon}{grab on} Grapple maneuver, you do not provoke an attack of opportunity.}{
You gain a +4 Dodge bonus to your AC and Reflex Saves against attacks from any creature with a longer natural reach than your own.}{
You have The Edge against any creature you attack that is larger than you. Also, an opponent using the Improved Grab ability on you provokes an attack of opportunity from you. You may take this attack even if you do not threaten a square occupied by your opponent.}{
When you attempt to trip an opponent, you may choose whether your opponent resists with Strength or Dexterity.}{
When involved in an opposed bull rush, grapple, or trip check as the attacker or defender, you may negate the size modifier of both participants. You may not choose to negate the size modifier of only one character.}


\hypertarget{feat:greatfortitude}{}\babfeat{Great Fortitude [Combat]}{
You are so tough. Your belly is like a prism.}{
You gain a +3 bonus to your Fortitude Saves.}{
You die at -20 instead of -10.}{
You gain 1 hit point per level.}{
You gain DR of 5/-.}{
You are immune to the fatigued and exhausted conditions. If you are already immune to these conditions, you gain 1 hit point per level for each condition you were already immune to.}


\hypertarget{feat:hordebreaker}{}\babfeat{Horde Breaker[Combat]}{
You kill really large numbers of people.}{
You gain a number of extra attacks of opportunity each round equal to your Dexterity Bonus (if positive).}{
Whenever you drop an opponent with a melee attack, you are entitled to a bonus ``cleave'' attack against another opponent you threaten. You may not take a 5' step or otherwise move before taking this bonus attack. This Cleave attack is considered an attack of opportunity.}{
You may take a bonus 5' step every time you are entitled to a Cleave attack, which you may take either before or after the attack.}{
You may generate an aura of fear on any opponents within 10' of yourself whenever you drop an opponent in melee. The save DC is 10 + the Hit Dice of the dropped creature.}{
Opponents you have the Edge against provoke an attack of opportunity from you by moving into your threatened area or attacking you.}

\vspace{30pt}
\hypertarget{feat:hunter}{}\babfeat{Hunter [Combat]}{
You can move around and shoot things with surprising effectiveness.}{
The penalties for using a ranged weapon from an unstable platform (such as a ship or a moving horse) are halved.}{
Shot on the Run -- you may take a standard action to attack with a ranged weapon in the middle of a move action, taking some of your movement before and some of your movement after your attack. That still counts as your standard and move action for the round.}{
You suffer no penalties for firing from unstable ground, a running steed, or any of that.}{
You may take a full round action to take a double move and make a single ranged attack from any point during your movement.}{
You may take a full round action to run a full four times your speed and make a single ranged attack from any point during your movement. You retain your Dexterity modifier to AC while running.}


\hypertarget{feat:insightfulstrike}{}\babfeat{Insightful Strike [Combat]}{
You Hack people down with inherent awesomeness.}{
You may use your Wisdom Modifier in place of your Strength Modifier for your melee attack rolls.}{
Your attacks have The Edge against an opponent who has a lower Wisdom and Dexterity than your own Wisdom, regardless of relative BAB.}{
Your melee attacks have a doubled critical threat range.}{
You make horribly telling blows. The extra critical multiplier of your melee attacks is doubled (x2 becomes x3, x3 becomes x5, and x4 becomes x7).}{
Any Melee attack you make is considered to be made with a magic weapon that has an enhancement bonus equal to your Wisdom Modifier (if positive).}


\hypertarget{feat:ironwill}{}\babfeat{Iron Will [Combat]}{
You are able to grit your teeth and shake off mental influences.}{
You gain a +3 bonus to your Willpower saves.}{
You gain the slippery mind ability of a Rogue.}{
If you are stunned, you are dazed instead.}{
You do not suffer penalties from pain and fear.}{
You are immune to compulsion effects.}

\vspace{300pt}
\hypertarget{feat:juggernaut}{}\babfeat{Juggernaut [Combat]}{
You are an unstoppable Juggernaut.}{
You may be considered one size category larger for the purposes of any size dependant roll you make (such as a Bull Rush, Overrun, or Lift action).}{
You do not provoke an attack of opportunity for entering an opponent's square.}{
You gain a +4 bonus to attack and damage rolls to destroy objects. You may shatter a Force Effect by inflicting 30 damage on it.}{
When you successfully \hyperlink{combat:bullrush}{bullrush} or overrun an opponent, you automatically Trample them, inflicting damage equal to a natural slam attack for a creature of your size.}{
You gain the Rock Throwing ability of any standard Giant with a strength equal to or less than yourself.}


\hypertarget{feat:lightningreflexes}{}\babfeat{Lightning Reflexes [Combat]}{
You are fasty McFastFast. It helps keep you alive.}{
You gain a +3 bonus to your Reflex saves.}{
You gain Evasion, if you already have Evasion, that stacks to Improved Evasion.}{
You may make a Balance Check in place of your Reflex save.}{
You gain a +3 bonus to your Initiative.}{
When you take the Full Defense Action, add your level to your AC.}


\hypertarget{feat:mageslayer}{}\babfeat{Mage Slayer [Combat]}{
You have trained long and hard to kill magic users. Maybe you hate them, maybe you just noticed that most of the really dangerous creatures in the world use magic.}{
You gain Spell Resistance of 5 + Character Level.}{
Damage you inflict is considered ``ongoing damage'' for the purposes of concentration checks made before the beginning of your next round. All your attacks in a round are considered the same source of continuing damage.}{
Creatures cannot cast defensively within your threat range.}{
Your attacks ignore Deflection bonuses to AC.}{
When a creature uses a [Teleportation] effect within medium range of yourself, you may choose to be transported as well. This is not an action.}


\hypertarget{feat:mountedcombat}{}\skillfeat{Mounted Combat [Skill]}{
You are at your best when fighting with an ally that you are sitting on.}{
Ride Ranks:}{
Once per turn, you may attempt to negate an attack that hits your mount by making a Ride skill check with a DC equal to the AC that the attack hit. Attacks that do not require an attack roll cannot be negated in this way.}{
While Mounted, you may take a charge attack at any point along your mount's movement, so long as your mount is moving in a straight line up to the point of your attack.}{
You suffer no penalty to your ride or handle animal skill checks when training or riding unusual mounts such as magical beasts or dragons.}{
You may use your Ride Check in place of your mount's Balance, Jump, Climb, or Reflex Saving Throws.}{
Any time a spell effect would target your mount, you may elect to have it target you instead. Any time a spell effect would target you, you may elect to have it affect your Mount instead.}


\hypertarget{feat:murderousintent}{}\babfeat{Murderous Intent [Combat]}{
You stab people in the face.}{
You may make a \hyperlink{combat:coupdegrace}{Coup de Grace} as a standard action.}{
When you kill an opponent, you gain a +2 Morale Bonus to your attack and damage rolls for 1 minute.}{
Once per round, you may take an attack of opportunity against an opponent who is denied their Dexterity bonus to AC.}{
You may take a \hyperlink{combat:coupdegrace}{Coup de Grace} action against opponents who are stunned.}{
You may take a \hyperlink{combat:coupdegrace}{Coup de Grace} action against opponents who are dazed.}


\hypertarget{feat:phalanxfighter}{}\babfeat{Phalanx Fighter[Combat]}{
You fight well in a group.}{
You may take attacks of opportunity even while flat footed.}{
Any Dodge bonus to AC you gain is also granted to any adjacent allies for as long as you benefit from the bonus and your ally remains adjacent.}{
Charging is an action that provokes an attack of opportunity from you. This attack is considered to be a ``readied attack'' if it matters for purposes like setting against a charge.}{
You may attack with a reach weapon as if it was not a reach weapon. Thus, a medium creature would normally threaten at 5' and 10' with a reach weapon.}{
You may take an Aid Another action once per round as a free action. You provide double normal bonuses from this effect.}

\vspace{300pt}
\hypertarget{feat:pointblankshot}{}\hypertarget{feat:pbshot}{}\babfeat{Point Blank Shot [Combat]}{
You are crazy good using a ranged weapon in close quarters.}{
When you are within 30' of your target, your attacks with a ranged weapon gain a +3 bonus to-hit.}{
You add your base attack bonus to damage with any ranged attack within the first range increment.}{
You do not provoke an attack of opportunity when you make a ranged attack.}{
When armed with a Ranged Weapon, you may make attacks of opportunity against opponents who provoke them within 30' of you. Movement within this area does not provoke an attack of opportunity.}{
With a Full Attack action, you may fire a ranged weapon once at every single opponent within the first range increment of your weapon. You gain no additional attacks for having a high BAB. Make a single attack roll for the entire round, and compare to the armor class of each opponent within range.}


\hypertarget{feat:sniper}{}\babfeat{Sniper [Combat]}{
Your shooting is precise and dangerous.}{
Your range increments are 50\% longer than they would ordinarily be. Any benefit of being within 30' of an opponent is retained out to 60'.}{
Precise Shot -- You do not suffer a -4 penalty when firing a ranged weapon into melee and never hit an unintended target in close combats or grapples.}{
Sharp Shooting -- Your ranged attacks ignore Cover Bonuses (total cover still bones you).}{
Opponents struck by your ranged attacks do not automatically know what square your attack came from, and must attempt to find you normally.}{
Any time you hit an opponent with a ranged weapon, it is counted as a critical threat. If your weapon already had a 19-20 threat range, increase its critical multiplier by 1.}


\hypertarget{feat:subtlecut}{}\babfeat{Subtle Cut [Combat]}{
You cut people so bad they have to ask you about it later.}{
Any time you damage an opponent, that damage is increased by 1.}{
As a standard action, you can make a weapon attack that also reduces a creature's movement rate. For every 5 points of damage this attack does, reduce the creature's movement by 5'. This penalties lasts until the damage is healed.}{
As a standard action, you may make a weapon attack that also does 2d4 points of Dexterity damage.}{
Any weapon attack that you make at this level acts as if the weapon had the wounding property.}{
As a standard action, you may make an attack that dazes your opponent. This effect lasts one round, and has a DC of 10 + half your level + your Intelligence bonus.}

\vspace{10pt}
\hypertarget{feat:twoweaponfighting}{}\babfeat{Two Weapon Fighting [Combat]}{
When armed with two weapons, you fight with two weapons rather than picking and choosing and fighting with only one. Kind of obvious in retrospect.}{
You suffer no penalty for doing things with your off-hand. When you make an attack or full-attack action, you may make a number of attacks with your off-hand weapon equal to the number of attacks you are afforded with your primary weapon.}{
While armed with two weapons, you gain an extra Attack of Opportunity each round for each attack you would be allowed for your BAB, these extra attacks of opportunity must be made with your off-hand.}{
You gain a +2 Shield Bonus to your armor class when fighting with two weapons and not flat footed.}{
You may Feint as a Swift action.}{
While fighting with two weapons and not flat footed you may add the enhancement bonus of either your primary or your off-hand weapon to your Shield Bonus to AC.}


\hypertarget{feat:weaponfinesse}{}\babfeat{Weapon Finesse [Combat]}{
You are incredibly deft with a sword.}{
You may use your Dexterity Modifier instead of your Strength modifier for calculating your melee attack bonus.}{
Your special attacks are considered to have the Edge when you attack an opponent with a Dexterity modifier smaller than yours, even if your Base Attack Bonus is not larger.}{
You may use your Dexterity modifier in place of your Strength modifier when attempting to trip an opponent.}{
You may use your Dexterity modifier in place of your Strength modifier for calculating your melee damage.}{
Opportunist -- Once per turn, when an opponent is struck, you may take an attack of opportunity on that opponent.}

\vspace{300pt}
\hypertarget{feat:whirlwind}{}\hypertarget{feat:whirlwindattack}{}\babfeat{Whirlwind [Combat]}{
You are just as dangerous to everyone around you as to anyone around you.}{
As a full round action, you may make a single attack against each opponent you can reach. Roll one attack roll and compare to each available opponent's AC individually.}{
You gain a +3 bonus to Balance checks.}{
As a full round action, you may take a regular move action and make a single attack against each opponent you can reach at any point during your movement. Roll one attack roll and compare to each available opponent's AC individually.}{
Until your next round after making a whirlwind attack, you may take an attack of opportunity against any opponent that enters your threatened area.}{
As a full round action, you take a charge action, overrunning any creature in your path, and may make a single attack against each opponent you can reach at any point during your movement. Roll one attack roll and compare to each available opponent's AC individually.}


\hypertarget{feat:zenarchery}{}\babfeat{Zen Archery [Combat]}{
You are very calm about shooting people in the face. That's a good place to be.}{
You may use your Wisdom Modifier in place of your Dexterity Modifier on ranged attack rolls.}{
Any opponent you can hear is considered an opponent you can see for purposes of targeting them with ranged attacks.}{
If you cast a Touch Spell, you can deliver it with a ranged weapon (though you must hit with a normal attack to deliver the spell).}{
As a Full Round Action, you may make one ranged attack with a +20 Insight bonus to hit.}{
As a Full Round Action, you may make one ranged attack with a +20 Insight bonus to hit. If this attack hits, your attack is automatically upgraded to a critical threat. If the threat range of your weapon is 19-20, your critical multiplier is increased by one.}

\end{multicols}

\subsection{The Spheres}

Fiends (and some of their minions and associates) cast magic primarily through spell-like abilities. While many signature fiends have arbitrary lists of spell-like abilities, the Tome of Fiends offers a method to advance Fiends into thematically appropriate spell-like abilities when they advance. When a fiend has access to a sphere, she is able to use all of the abilities within that sphere up to her character level. If she gains more levels, more powers of the sphere become available. In this way the spell-like abilities of fiends created with the rules in this tome should always be \ae sthetically and level appropriate.

\ability{Basic Sphere Access:}{When a creature has basic access to a sphere, she can use any of the spells listed in the sphere may be used once per day (each) as spell-like abilities, provided that their listed level is equal or lower to the creature's character level.}

\ability{Advanced Sphere Access:}{When a creature has advanced access to a sphere, she can use any of the spells listed in the sphere may be used 3 times per day (each) as spell-like abilities, provided that their listed level is equal or lower to the creature's character level.}

\ability{Expert Sphere Access:}{When a creature has expert access to a sphere, any spells listed in the sphere may be used at will as spell-like abilities, provided that their listed level is equal or lower to the creature's character level.}

\ability{Creating new spheres:}{The following list of spheres isn't intended to be comprehensive, and we fully expect that some players and DMs will want many more spheres than we have scribed. All new spheres must be approved of by the DM, and should represent some actual (indifferent or evil) trait like ``intoxication" or ``badgers" rather than a game mechanical notion like ``kicking ass and being totally sweet" or something praiseworthy like ``generosity". A good place to start is actually Domains, as these are already a source by which a character gain a spell at every odd-numbered level.}

\ability{Spheres and Spell Levels:}{Spell-like abilities used out of spheres are considered to be cast as a spell level equal to half the minimum needed character level to use the ability (rounded up). The save DC of a spell-like ability granted through Sphere access is Charisma-based. Thus, the save DC for a spell-like ability which becomes available at character level 5 is 13 + Charisma bonus.}



\newcommand{\sphere}[9]{\begin{list}{}{\itemspace}\item \textbf{1:} \spell{#1} \item \textbf{3:} \spell{#2} \item \textbf{5:} \spell{#3} \item \textbf{7:} \spell{#4} \item \textbf{9:} \spell{#5} \item \textbf{11:} \spell{#6} \item \textbf{13:} \spell{#7} \item \textbf{15:} \spell{#8} \item \textbf{17:} \spell{#9}}
\newcommand{\spherecont}[1]{\item\textbf{19:} \spell{#1} \end{list}\medskip}

\begin{multicols}{2}
\section{Spheres} \hypertarget{spheres}{}

\newcommand{\sphere}[9]{\begin{list}{}{\itemspace}\item \textbf{1:} \spell{#1} \item \textbf{3:} \spell{#2} \item \textbf{5:} \spell{#3} \item \textbf{7:} \spell{#4} \item \textbf{9:} \spell{#5} \item \textbf{11:} \spell{#6} \item \textbf{13:} \spell{#7} \item \textbf{15:} \spell{#8} \item \textbf{17:} \spell{#9}}
\newcommand{\spherecont}[1]{\item\textbf{19:} \spell{#1} \end{list}\medskip}

Fiends, celestials, and some characters cast magic primarily through spell-like abilities. While many monsters and characters get arbitrary spell lists, spheres present a way to advance spellcasting in a thematic way. When a creature has access to a sphere, she is able to use all of the abilities within that sphere up to her character level. If they gains more levels, more powers of the sphere become available. In this way the spell-like abilities of fiends created with the rules in this tome should always be \ae sthetically and level appropriate.

\ability{Basic Sphere Access:}{When a creature has basic access to a sphere, she can use any of the spells listed in the sphere may be used once per day (each) as spell-like abilities, provided that their listed level is equal or lower to the creature's character level.}

\ability{Advanced Sphere Access:}{When a creature has advanced access to a sphere, she can use any of the spells listed in the sphere may be used 3 times per day (each) as spell-like abilities, provided that their listed level is equal or lower to the creature's character level.}

\ability{Expert Sphere Access:}{When a creature has expert access to a sphere, any spells listed in the sphere may be used at will as spell-like abilities, provided that their listed level is equal or lower to the creature's character level.}

\ability{Creating new spheres:}{The following list of spheres isn't intended to be comprehensive, and we fully expect that some players and DMs will want many more spheres than we have scribed. All new spheres must be approved of by the DM, and should represent some actual (indifferent or evil) trait like ``intoxication" or ``badgers" rather than a game mechanical notion like ``kicking ass and being totally sweet" or something praiseworthy like ``generosity". A good place to start is actually Domains, as these are already a source by which a character gain a spell at every odd-numbered level.}

\ability{Spheres and Spell Levels:}{Spell-like abilities used out of spheres are considered to be cast as a spell level equal to half the minimum needed character level to use the ability (rounded up). The save DC of a spell-like ability granted through Sphere access is Charisma-based. Thus, the save DC for a spell-like ability which becomes available at character level 5 is 13 + Charisma bonus.}
\end{multicols}


\section{Optional Rules for Fiends}

There are a number of places in the rules that governing Fiends that just don't work at all, or don't work in a way that is good. This is an attempt to fix them.

\subsection{No Wishing for More Wishes!}
The 3.5 wish spell is very explicit in what it can do, and extremely vague about what it can't do. It has a big list of things it is capable of, and then tells the DM to ad hoc things if anyone wishes for anything that isn't on that list. Unfortunately, wishing for a Staff of 50 wishes is on the list of things you can wish for. The XP cost is considerable (512,180 XP), but if you get your wishes from a magic item (like a Staff of 50 Wishes) or a spell-like ability (like an Efreet), you don't have to pay that XP cost, so the fact that it is stupidly large doesn't even matter. Needless to say, the game completely breaks down as soon as that happens. So in that spirit, we suggest an alternate list of things wish can do, coupled with some things wish actually can't do:

\listone
    \item Free Wishes -- the following wishes have no XP cost:
    \listtwo
      \item Wealth: A character can wish for mundane wealth whose total value is 25,000 gp or less.
      \item Magic Item: A character can wish for a magic item that costs 15,000 gp or less.
      \item Power: A character can wish to increase an inherent bonus to any attribute by 1 (to a maximum of +5)
      \item Spell: A character can wish for the effects of any spell that lacks an XP cost that is lower level than the highest level spell in its spell list (a wizard spell of 8th level or less, or a paladin spell of 3rd level or less, for example).
      \item Transport: A character can wish herself and 1 other willing creature per caster level to any location on any plane.
    \end{list}
    \item Wishes that aren't Free -- the following wishes cost XP or gp or both:
    \listtwo
      \item Add to the Powers of a Magic Item: A character can increase the powers of a magic item to anything she could enhance it to with her own item creation feats. This requires 1 XP for every 10 gp increase in magic item value.
      \item Raise the Dead: A character can bring the dead back to "life", even if they were an undead, construct, or other creature that cannot normally be brought back to life. This may even be able to bring back a creature who has been devoured by a Barghest (50\% chance of success). This costs 3,000 XP, which can be paid in any combination by the caster or the target. The spent XP for this wish can reduce a character's level, but coming back to life in this manner otherwise won't do so.
      \item Undo Misfortune: A character can wish back the sands of time in order to force events of the last round to be replayed. Time can be reset to any point back to the character's previous initiative pass. This use costs 1000 XP. While the action spent to cast wish in this case is restored, the character still loses the spell slot and XP used to power it.
      \item Turn Back Time: A poorly fated adventure can be averted entirely with a wish. The character expends the slot and pays 5,000 xp, and none of it ever happened.
    \end{list}
    \item Wishes that are Rituals -- some wishes have much greater costs, at the whim of the DM. Here is an example:
    \listtwo
      \item Become a new Creature: A character can wish themselves into being a new creature. This must be done when a character is eligible to gain a new level, and the character makes the wish and takes a level of the new racial class (or racial paragon class) and is now the new race.
    \end{list}
\end{list}


Any use of wish causes the wisher to become fatigued (and yes, there are ways to get around that).

Creatures with spell-like abilities that grant wishes may only grant wishes that have no XP cost. So an Efreet can give you as many +2 swords as it wants, but an Efreet can't give in to your request to have a +3 sword. Also, you'll notice that we categorize the inherent bonuses as something that's free and therefore going to be rapidly available to all the player characters somewhere between 11th and 15th levels. That's because we seriously believe that it is more balanced for characters to all gain +5 inherent bonuses than it is for some characters to figure out how to manipulate XP gains and thought bottles to get inherent bonuses while the other players don't. Inherent bonuses need to be available or not available to everyone or they break the game.

Magic items with wish on them can be used to cast wishes with an XP cost of at most 5,000 XP, and are produced as items using spells with a cost of 5,000 XP. As a result, you can't wish for an item that has wish on it.

\subsection{Damage Reduction and Special Materials}

The 3.5 rules were rather\ldots\  overzealous with splitting up material DR, and the result has been that high level characters actually just curl up and cry. Here are some guidelines to streamline things a bit:

\listone
\item Any steel weapon counts as ``cold iron" for the purposes of beating DR. Cold Iron being a special kind of iron mined deep underground is, well, insultingly stupid. Cold Iron is an actual word, it's the first mass-produced type of iron in history, and in song and story is effective against fairies and chaos demons because it symbolizes order and industrialism. Cold iron is cheap, that's the whole point. If it wasn't cheap, it wouldn't be available in industrial quantities, and then it wouldn't have any symbolic effect against savage fey and demons of disorder.
\item Alchemical Silver has no damage penalty. The fact that Silver has a damage penalty is sort of justifiable, except that in D\&D weapons made out of wood don't have a damage penalty. The game simply doesn't have a fine enough grain to keep track of the ways in which you'd rather have a sword made out of steel than a silver plated one. Also the thing where DR 1/Silver is in fact impossible to beat is incredibly dumb.
\item Material DR beats Material DR. Alignment DR beats Alignment DR. Creatures with DR can hurt other creatures with DR as if they had natural weapons made out of whatever punches through their DR. And creatures with alignment subtypes penetrate DR with their manufactured weapons as if they had the alignment of their subtype. So when a Balor punches a Pit Fiend (needs Silver and Good), his fist counts as Good and Iron. When a Balor swings a Silver Sword at the Pit Fiend, his weapon counts as Silver and Evil -- he has got all the needed adjectives, he just can't get them all at the same time. And that is really dumb. What should happen is the fact that the Balor needs an aligned weapon made out of a special material to be hurt should be sufficient to hurt the Pit Fiend with his natural weapons.
\item There can only be five! An unfortunate and unintended result of the 3.5 DR rules is that as more materials and monsters get written, the chances of you having whatever material your target's DR is penetrated by drops to a number pretty close to zero. In order to keep that from happening, we propose that for the purposes of DR, there are only 5 materials, and absolutely everything counts as one of those five. So if your weapon isn't made out of: Adamantine, Iron, Silver, Stone, or Wood, it counts as being made of one of those materials. Here is a suggested weapon equivalency chart:
\listtwo
    \item Adamantine: \listthree\small
       \item Alchemical Gold
       \item Black Steel
       \item Orichalcum
       \item N Metal
       \item Thinaun
       \item Urdrukar\end{list}
    \item Iron:\listthree\small
       \item Blood Steel
       \item Green Steel
       \item Morghuth Iron
       \item Trusteel\end{list}
    \item Silver:\listthree\small
       \item Pandemonic Silver
       \item Astral Driftmetal
       \item Entropium
       \item Nerra Mirrorblade
       \item Ysgardian Heartwire
       \item Mithril\end{list}
    \item Stone:\listthree\small
       \item Tainted Obsidian
       \item Blended Quartz
       \item Elukian Clay
       \item Kaorti Resin\end{list}
    \item Wood:\listthree\small
       \item Bronzewood
       \item Chitin
       \item Darkwood
       \item Iron wood
       \item Boneblade
       \item Dragon Bone\end{list}
\end{list}\end{list}

\subsection{Putting the Prime Back in Prime Material Plane (Alternate Prime Material Plane Rules)}

Many classic fiend stories involve demons or devils doing their best to get into the Prime. The real question is: why? The Lower Planes, while often inhospitable to natives of the Primes, is often perfectly suited to fiends since these planes are each individually infinite in size and fiends are well suited to their environment (they speak the native tongue and are immune to the average environmental threats, and natives don't freak out when they see them). It can't be an issue of new lands to conquer, or even new innocents to torture, as the Lower Planes are filled with both, and in infinite abundance. So why do powerful nasties want into the Prime? The following rules are changes to the D\&D cosmology, and they clear up the role of outsiders in the affairs in the realms so that more logical and fun adventuring can be had for players.

\subsubsection{The Prime is Better Than Cancun}
Prime Material Planes have one unique trait in all the universe: once in a Prime, you can't be summoned or called. For fiends, this means that they are no longer subject to the hierarchies of whatever place they hang their hat. For a fiend whose True Name is being passed around like a trading card, this is a huge thing: the Prime becomes a place where he can finally determine his own destiny, and no longer be a (potential) slave to the whims of mortals or his fiendish superiors. Fiends who are plotting coups in their own realm want to be able to get to Prime so that they are outside of the authority structure of their own race, and can lay low and build up their forces for a triumphant return to their particular Lower Plane. In this way, their superiors can't summon them and put them to the question in order to catch wind of their plans.

\subsubsection{You Can Get Room Service}
The second most important aspect of the Primes is that calling spells only work from the Prime. While regular summoning spells can call certain individuals, Conjuration magic of the [calling] subschool only works while in the Prime Material Plane. This means that beings that want to abuse calling magic to build armies can only do so while in a Prime. This particular rule clears up silliness like demons binding angels and forcing them to fight in Hell, or otherwise serve, which the current rules allow.

\subsubsection{Better Service for VIPs}
Natives of the Primes also hold a special place in the universe: they can't be summoned or called. This is actually a pretty big deal, since this means that natives of the Primes are the premier agents in the politics of the planes. Not only can be they summon or call natives of the planes while on the Prime, but they alone are free from the Conjuration spells that enslave and bind together the Lower Planes. In addition, the basic spells of \spell{raise dead} and \spell{resurrection} only function on creatures native to the Prime. Other creatures can be restored to life (with revive outsider, for example), but it's comforting to know that absolutely any Cleric can restore one's life if she wants to -- and Prime Natives live with that comfort every day of their lives.

\subsection{Practical Demonology: Additional Rules for Summoning}
One of the most contentious parts of the D\&D ruleset involves the summoning and binding of Extraplanar beings. We all agree that we want demon summoning, but we can't agree on what we want it to do. Should they be mindless slaves, or should they be tricksy tricksters who will eat your face if given the slightest chance? How exactly do \spell{planar ally} and \spell{planar binding} work? Can you just intimidate an outsider, or do you need to bargain with them with fair trade? Below are some additional rules to flesh out the experience:

\subsubsection{The Deal}
Making a deal with a fiend is usually a DM's call. He decides just how much interference he wants a summoning spell to do with his adventure, then he lets the party offer trade or threats until they get what they want up to the limits he has set. For DMs who don't want to stop-rule this each time, here is a list of tasks you can ask of a creature called by summoning spells:

\subsubsection{Part 1: Differences between Summoning and Calling}
First, we must reiterate the difference between summoning and calling.

\listone
    \item Summoning brings a creature to your location that follows both the intent and letter of your orders, has no free will, and will not act willingly act against your interests. When this creature dies, it and any effect it created vanishes (unless that effect was an instant effect). This creature has knowledge, but no personality or history. In effect, it only exists while the spell lasts.
    \item Calling spells bring an actual creature to your location, ripped from whatever place in the universe it existed. If you know a creature's name (not its True Name, which we will discuss later, but a use-name that it answers to), you can call that individual, along with any equipment or treasure it is carrying, but otherwise you get a random individual of that race. It has a personality and feelings, and when the spell ends it is returned to its original location. In effect, this creature has a life, and if treated badly enough, it may seek out its summoner for revenge.
\end{list}

\subsubsection{Part 2: Choosing a Pawn}
D\&D rules are silent on the issue of the limits of calling magic. While spells with the [summoning] subtype have specific lists of creatures that they call, [calling] spells usually have no such limits (except for the planar ally spells that force the DM to choose a creature). A simple way to limit creatures called is to only allow a summoner to call creatures that he could reasonably know about, and this means a Knowledge check.

Force the player to make a Knowledge check each time he wants to summon a particular race of creature for the first time (in effect, the base creature in the Monster Manual or other source). If he fails that check, he may not attempt another check for that base creature until he gains at least one rank in the relevant skill. Once he can summon a base creature, he may summon a templated version of that creature with an additional Knowledge check (and if he fails that check, he may not attempt another check for that templated creature until he gains at least one rank in the relevant skill).

This check uses the same Knowledge skill that would be required to identify that creature. The following modifiers also apply to the DC of the check.

\begin{quote}
+15 A normal creature, but with the extraplanar subtype\\
+5 Per CR of racial templates applied to base creature\\
-5 Spent one day studying the dead body a creature of the same race and racial templates.\\
-10 Spent one week studying a living member of that race and racial templates\\
+10 Never seen an example of the creature.\\
-10 Detailed written description of appearance and powers (must be 100\% complete)\\
\end{quote}

*Creatures with class levels or versions of monsters advanced by HD count as unique creatures, and they cannot be called without their use-name.

A player is responsible for recording each monster that he can call, and the ones he has failed to call. Once he has made a check for a particular combination of race and templates, he does not need to do so again.

Here is an example:


      Morgothazan the Dark casts \spell{lesser planar binding}, and he would like to call a Small Fire Elemental. To identify such a creature, he would need a Knowledge (the planes) check of 10 + the HD of a Small Fire Elemental creature, which is 2, meaning he needs a 12 to identify and call a Small Fire Elemental. As a 9th level wizard with a +14 modifier in Knowledge(the Planes), he automatically succeeds.\\
      The next day, he decides that he wants to call a Half Fiend Small Fire Elemental. He has never seen such a creature, but he knows that it must exist somewhere in the planes. His base DC is 12, plus another +10 for never having seen this oddity, and another +5 for the additional CR added to it, bringing his DC to 27. His modifier is +14, and he rolls a 12, meaning his gets a 26. Until he raises his Knowledge planes skill, he can't call a Small Fire Elemental modified by the Half-Fiend template.\\
      Several weeks later, Morgothazan the Dark wants to conjure a Half-fiendish Earth elemental. He already knows how to conjure a Small Earth Elemental, and he has actually fought and defeated a dead Half-fiendish Small Earth elemental. His base DC is 12, plus another +5 for the template, bringing his DC to 17. He rolls a 5, and he can call this monster.\\
      Emboldened by his success, he wants to be able to call a Half-Celestial Half-fiend Small Fire Elemental, but he remembers that he cannot (he can't conjure a Half-Fiendish Small Fire Elemental, so a Half-Celestial Half-fiend Small Fire Elemental is not possible). Instead, he tries the same templates on a Small Earth Elemental, as he has a detailed description of such a creature and he has had success with Half-fiend Small Earth Elementals. His base DC is 12, and his detailed description (-10) offsets the fact that he has never seen this creature(+10). Then an additional +10 is added for the CR increase from the two templates, making his final DC 22. He rolls a 10 and succeeds!

\subsubsection{Part 3: Services!}
When you cast a calling spell, you are bargaining for a single service. While normal bargaining could get you more complex arrangements, conjuring magic that calls real creatures can only force compliance to single services. For example, while a greater planar binding spell can bring a Pit Fiend to the Prime, the spell can only force to creature to obey the agreement set for a single service. Any additional services would not be guaranteed by the magic of the spell, and the Pit Fiend would keep or break any agreements as normal for that creature.

Within the limits of the single service, a called creature can do whatever it wants. A genie ordered to guard a room is under no compulsion to use its create food and water ability for allowed occupants of that room, and it may choose whether to converse, sit or stand, eat, or do any other act that does not interfere with its task. Clever conjurers often set tasks with exceptions in them like ``kill my enemies in Redstone Castle'', knowing that if they didn't define ``enemies'' and instead said ``kill everyone in the Redstone Castle,'' the called creature would be free to attack the conjurer if he entered Redstone Castle.

Called creatures will not agree to any services that are suicidal, self-destructive (like submitting to mind-control magic), or involve permanent self-sacrifice (like expending XP). They will also not agree to tasks that are impossible, or tasks that are so open-ended that could easily result in the creature's destruction.\\

\featnamelist{Things you can ask a creature to do:}
\listone
    \item Participate in a single battle
    \item Use a single use of one of its own abilities.
    \item Seek out an individual and either kill them or bring them to the summoner.
    \item Guard a spot for as long as the summoning spell lasts.
    \item Use a magic item
    \item Provide the results of one skill check
    \item Perform one task that does not involve any danger (like delivering a message by a safe route, survey a safe land, or dig a hole in an uncontested piece of land, etc)
    \item Offer their use-name$^*$
    \item Surrender personal treasure$^*$
\end{list}
\vspace*{8pt}
{$^*$ \small Requires a successful Intimidate check.}\\

\featnamelist{Things that creatures will do for free (not services):}
\listone
    \item Wait in a safe place in order to perform a service.
    \item Discuss the services they are willing to perform, and payment for those services.
    \item Exclude individuals from services (``kill anyone who enters except me'', ``tell me about everyone you saw in the tunnel except the sorceress'')
\end{list}
\vspace*{8pt}

\featnamelist{Some things demons won't do, even under pain of death or destruction:}
\listone
    \item Surrender their true name
    \item Voluntarily fail a save vs. an effect that would enslave or kill the demon
    \item Agree to unlimited service for a time period (for example, ``Do my bidding for a week.'')
    \item Guard an individual for a time period.
    \item Agree to not act in a situation (for example, they will not agree to not act while someone builds a prison around them).
    \item Wait in an obviously dangerous place (``just wait in front of that army of archons, and shoot the first one'').
    \item Perform an act that would violate its alignment or code of conduct.
\end{list}

\subsubsection{Part 4: Closing the Deal}
Once you have agreed on services to be performed, it is necessary to convince a creature so serve. Many spells simply bring a creature and enforce any agreement, they do not actually create an agreement.\\

To make an agreement, there are some things that must happen first:

\begin{enumerate}\itemspace
\item The Conjurer must be able to communicate with the creature. This means that the creature must be capable of communication (Int 3 or better) and they must have a form of communication (shared language, telepathy, \spell{tongues}, etc).

\item The Conjurer must successfully convince the creature.

\item The Conjurer must pay for services (if necessary).
\end{enumerate}

The initial attitude of a creature is Indifferent, unless the conjurer has an opposed alignment (good and evil, law vs. chaos) in which case they are Hostile.

To make an agreement, a successful Diplomacy check is required, and the attitude of the creature must be raised to at least Friendly. Once raised to Friendly, the creature performs its task as agreed and leaves when the task is completed, or when the spell's duration ends. A bribe of treasure equal to the amount of treasure an encounter equal to the creature's CR would earn is necessary to pay for these services. A Helpful check halfs this amount of treasure. A failed check means that the creature is not convinced, and a new check can be made the next day. A treasure of four time normal value automatically secures the creatures trust (Friendly result with no check). Note: \spell{planar ally} spells call a Friendly creature, and only the treasure need be paid.

Intimidate can be used as well, and this use of the skill can negate the need to pay for services, but earns the enmity of the creature. When the task is ended, but while the spell's duration lasts, the creature may return home, ending the spell, but also has the option of seeking out the conjurer and attempting to harm him or foil his plans. When the duration of the spell ends, the creature is not returned home. This creature may choose at some later date to seek revenge on the conjurer.

Bluff can also be used in place of Diplomacy in order to make the creature believe that items being offered are real treasure (when they might be worth less, or actually worthless). A successful result means that the creature accepts the offering and performs as if Diplomacy had made the creature Friendly. If the creature discovers during the course of the service that the treasure is not real, the binding magic fails and the creature is no longer forced to perform the service, and its attitude becomes Hostile. While the spell lasts, the creature may return home once as a free action, ending the spell.

\subsubsection{Part 5: The Business of Serving:}

One bound and a deal is made, the creature obeys according to pact made. Should the spell be ended, the creature is under no compulsion to obey the agreement (though some will out of fear or duty). Also, should the creature be put into a situation where the service cannot be competed (the person to be captured is killed by someone else, or the creature is forced to return to its home plane, for example), the service ends, and the creatures stays or returns as normal.

If the summoner betrays the creature by attacking it, stealing its treasure, or doing some other harm, the spell ends and the creature may return home or stay to seek its revenge.

\subsubsection{Part 6: Appendix: True Names and Use-names}
True Names are names of special power, and most creatures don't even know their True Name, or even how to get it. Special skills and some spells and effects can unravel a True Name, but the most common way to learn a True Name is for a powerful spellcaster to trade that knowledge to another creature for some treasure, favor, or True Name of equal power. Merely knowing a True Name is enough to grant power, since speaking the extraordinarily difficult word is a magical process that is unnecessary for most summonings (True Naming magic is a separate art from divine and arcane spellcasting, and is frankly not powerful enough for most would-be summoners). The feat Broker of the Infernal is one way of using True Names without learning the True Name skill or brand of magic.)

Use-Names, on the other hand, are far simpler. If you have seen a creature's true form and you know a name that it answers to, you can use calling magic to summon it.

\subsection{Weapon Proficiencies? You've got to be kidding me!}
The thing where being an Outsider automatically gives you proficiencies in all martial weapons is extremely dumb. There are substantial limits to the ``types as classes'' rules, and when we come to weapon proficiencies, we know that's it. An Erinyes should be proficient with a longbow or a whip, but a Howler should not. Honestly, the outsider type is so extremely varied that any rules you applied to the entire Outsider type would certainly cause more problems than they could fix. You are better off using no rules at all than the listed rules in the Monster Manual for weapon and armor proficiencies.

When you \emph{ad hoc} things and attempt to play by common sense rather than the wording in a book, you leave yourself open for horrible arguments because I am pretty sure my gut tells me different things than your gut tells you. But that's still better than getting into the arguments about how high level alienists and yeth hounds can use glaives. Without rewriting the entirety of every single monster book, this is a problem that actually has no resolution -- but it's also a problem that can usually be ignored. Don't give players any special weapon proficiencies for changing their type and generally assume that monsters are proficient with whatever weapons that they happen to be holding. It's not fair, it's not consistent, but at least it's not stupid.


\section{High Adventure in the Lower Planes}

The Lower Planes are infinite in size, and this is often taken as meaning that they are somehow filled with infinite power. This is essentially completely false. Remember that the Primes are essentially infinite in scope as well, and while there are ancient dragons and even Xixicals\ldots \textit{somewhere}, the fact is that you could adventure your whole life and never ever meet one. The world is mostly filled with forests, and mountains, and little river valleys, and most of the time the villains you encounter are going to be rabid dire weasels and bugbear junkies who will try to resell your shoes for a hit of mordayn vapor. Gehena is actually just like that, except that instead of you never seeing powerful dragons in your day to day life, you never see Arcanoloths. The bad guys you encounter may well be a \textit{fiendish} dire weasel and a bugbear junkie \textit{petitioner}, but the essential threat level is pretty much the same.

Low level adventuring, thus, is extremely plausible in the lower planes. It's not advisable for low level characters to go running around Tiamat's lair or anything, but the fact that the Elder Brain Pool is somewhere in the Underdark hasn't stopped \textit{any} low level campaigns from tunnel crawling as far as I can recall. What follows is some wilderness adventure seeds from the lower planes for low (1-5), medium (6-10), and high (11-15) level. Players who want to adventure at near epic levels (16+) don't even need adventure seeds of this sort because they actually can just take on The Dark Eight or whatever. For whatever reason, lots of ink has been spilled on near epic adventuring in the lower planes, and I have every confidence in a decent DM's ability to throw a Balor at a party and make a rollicking and dangerous encounter.


\subsection{High Adventure in\ldots Acheron!}

The first thing to realize about Acheron is that it really isn't a bad place to be. It's not even Evilly Aligned, so even campaigns using The Face of Horror have no reason to play up the terror of being here -- the sand of Acheron is not Evil. But it \textbf{is} made out of steel. Characters who are going to go adventuring will do so in Avalas, because that's the part of the plane that doesn't \textit{turn you to stone}.

\subsubsection{Campaign Seed: The Tunnel to Pandemonium}

Here's a little piece of D\&D history for you -- In AD\&D, Orcs were \textit{Lawful Evil}, so the Orcish pantheon lives in Acheron to war eternally with the Goblin pantheon \textit{even though Orcs are Chaotic now}. That means that the cube of Nishrek, where Gruumsh calls his most favored and despised for Gruumshian Justice when they have passed on -- is itself a bubble of Pandemonium found far from its place in the Wheel. There are, therefore, numerous portals to Pandemonium all over Clangor and Nishrek, so characters who wish to fight Orcs and Goblins in the lower planes can do so to an unlimited degree by portal hopping through the Pandemonium and Acheron layers. As an agent of Gruumsh or Maglubiet, characters can fight their way through savage humanoids, savage humanoid armies, savage humanoids with fiendish allies, savage humanoid war machines, and even powerful outsiders aligned with savage humanoids \textit{well into epic}. You can also use this rivalry as the backdrop for any of a number of ''find the artifact before it falls into seriously the wrong hands'' type adventures, with the characters switching sides repeatedly based on who seems to have the artifact now.

\subsubsection{Campaign Seed: You're in the Army Now}

Cities and castles populate the lands of Acheron without number, and all of them are on a war footing at all times. Characters can travel generally without molestation throughout this area and conduct a fairly profitable bit of trading and scavenging if they do things right. But if they do things wrong, they may end up drafted into some local militia or imperial army. Characters can have substantial numbers of adventures as part of a military force, or they can attempt to resist being drafted by any of a number of means. Unfortunately, the laws of Acheron being what they are, once the characters impress their will by force of personality or arms enough to avoid the draft, they'll find themselves as a \textit{side} -- which means that they'll be treated as a hostile army all themselves by other forces. At that point they can try to stick it out alone, or try to get some help, of course almost every empire in Acheron started the same way. So the players can progress smoothly from the ''chased by bad guys'' scenarios to the ''forge an empire in blood'' scenarios to the ''marry the princess, design your castle'' scenarios.

\subsubsection{Ten Low Level Adventures in Acheron}

You pull into the hamlet's bar, and see what they have to offer a stranger. It isn't good. After a brief set of questions to make sure you aren't going to burn the place down, the bartender tells you\ldots

\listone
	\item The town is infested with fiendish rats. Beer just isn't safe until their gone, sorry.
	\item A rival faction as poisoned the well, and someone needs to find a new source of water.
	\item Brigands are holding the pass. I hear one of them is an Ogre.
	\item The man you are looking for\ldots he was taken away by the Scarthian Army.
	\item That signet ring is part of King Imag's royal accoutrements. If someone could get all of them together\ldots it could spell big changes for the County of Yevekh.
	\item Orcs have come through the tunnel, their leader has a silver sword and noone dares to stand against him.
	\item After the Citadel of Zor fell, bodies were piled as high as your arm pit. I hear someone is making them all into zombies now, it's a shame really.
	\item I'd love to give you change, but after Sir Garreth set the taxes to 100\%, I'm afraid I have no coins to give you.
	\item In this town, either you're for Sheriff Braxton, or you're dead. This town, we like to have choices.
	\item It's free drinks here if you can get Clarrissa the hobgoblin matron to allow her daughters to marry.
\end{list}

\subsubsection{Ten Mid Level Adventures in Acheron}

An emissary of hoary Surog, the ice count, contacts you. He has (the ring, the antidote, the code) you need, and he'll give it you, but first\ldots

\listone
	\item One of his lieutenants has betrayed him; since you are random strangers, he can trust you to find out which one.
	\item His daughter has run off with the blue falcon, that accursed do-gooder. Bring her home, do with him as you wish.
	\item His daughter is the blue falcon. Stop her, but don't kill her.
	\item His daughter is the blue falcon, and Surog's rival, Cardinal Valgos, has put-her-in-a-death-trap. Rescue her, without letting on that Surog knows her identity.
	\item Cardinal Valgos has found some route to smuggle forces into Yevekh. Find how they're getting in.
	\item Cardinal Valgos is planning an attack, and Surog is not prepared. Infiltrate his mercenary forces and cause as much delay as possible.
	\item Cardinal Valgos has placed Surog in some kind of suspended animation! You have to lift the curse before one of Surog's underlings makes a play for power.
	\item A blue crossbow bolt with a head shaped like a stylized raptor strikes the emissary from nowhere, killing him before he can deliver your mission! Who is trying to stop you, and why?
	\item Cardinal Valgos has Imag's heir, or so he claims. Prove the heir is false, or steal him away.
	\item Cardinal Valgos has tricked the fox of the mountains, Dagipert, into allying with him. Break up this alliance one way or another.
\end{list}

\subsubsection{Ten High Level Adventures in Acheron}

You stand at the front of your army, triumphant over every foe the Lichking has sent against you, over the next hill you see\ldots

\listone
	\item The Lichking's vampire sister, all alone with a white flag.
	\item A pile of bodies impaled to the top of a 200 foot metal rod.
	\item A stampede of zombie elephants.
	\item A chasms cleaved into the side of the cube burbling with lava.
	\item A portal opening up upon an army of orcs in Pandemonium, easily the equal of your own.
	\item A huge pile of what appears to be gold.
	\item A huge pile of what appears to be skulls on fire.
	\item A wyvern bearing a message in its claws.
	\item The daughter of King Zormmund, tied to an elder earth elemental.
	\item Your grand vizier, who has apparently betrayed you again.
\end{list}


\subsection{High Adventure in\ldots Pandemonium!}

Pandemonium is a victim of the terrible confusion that permeates Law and Chaos in D\&D literature, and its inhabitants are portrayed in a number of improbable lights. Pandemonium is not an Evil plane, but it's fairly wicked and it's inherently Chaotic. How it and the people who live there appear in your game is entirely dependent upon how your game ends up handling Chaos in general. Pandemonium might be extremely disorganized, or inherently deceitful, or starkly unhelpful, or simply a lawless wilderness. But what it almost certainly \textit{isn't} is a source of low comedy where people do ''whacky stuff'' because they are so ''crazy''. That's the kind of thing that makes us cry.

Pandemonium can be a source of classic D\&D adventure at its finest -- the towns of Pandemonium are located right next to twisting tunnels through the stone and loud noises sound continuously through the warrens. So characters can go right from the town to the dungeon crawl without any explanation or overland travel, and those dungeon encounters are inherently episodic because nothing can hear your combats.

Pandemonium is dark and loud, and filled with confused people. At its best, Pandemonium is basically a huge rave. At its worst, Pandemonium is a huge rave. Like every part of the D\&D afterlife, Pandemonium can be a punishment or a reward. And like Acheron, this place isn't inherently Evil. So even if you are using The Face of Horror, the Eternal Rave isn't that bad of a place.

\subsubsection{Campaign Seed: Life in the Big City}

Welcome to The Madhouse. It's one of the largest planar metropolises in D\&D, and unlike places like the City of Brass or Sigil, it really \textit{doesn't} have some group of powerful outsiders ruling it with an iron fist. In fact, The Madhouse has no rulership of any kind. The place is dark, and loud, and the only light comes from naked women with glow sticks. Essentially, you can get away with pretty much anything without interference from opponents significantly outside your level range. You can keep having urban adventures continuously from 1st to 20th without ever getting seriously derailed by concerns of DM self-insertion characters coming over to knock over your house of cards. At the same time, there really \textit{are} Balors in this complex, so if you actually want to \textit{seek out} higher-powered enemies, that's doable.

\subsubsection{Campaign Seed: The Largest Dungeon}

Tunnels crisscross Pandemonium all over the place, and they are completely stable because the way gravity works there actually can't be a cave-in. But the place is dark and windy, and filed with tunnels that move around for no reason. The caverns are filled with monsters, traps, and treasure. It's all there, from shambling zombies to ninja temples, the low level areas cross seamlessly into the higher level ones. Oddly, this is the only place in the entire multiverse of D\&D where the old Gygaxian standby of having deeper and deeper levels of the dungeon filled with nastier and nastier monsters and traps actually makes sense. There's a town nearby, and the map doesn't have to make any sense at all. If you're looking for Nethack style adventuring, Pandemonium delivers.

\subsubsection{Ten Low Level Adventures in Pandemonium}

You lean over the counter to the waitress, not because she's so beautiful, but because you can barely hear her over the din. Honest. You're pretty sure she said\ldots

\listone
	\item WE DON'T SERVE YOUR KIND HERE. THE MILLER ONLY SENDS US BASALT FLOUR NOW.
	\item WE GOT AN ORDER OF APRICOTS IN THIS WEEK, THE CRAZ NAKED MAN CLAIMS TO MAKE IT HIMSELF.
	\item THE TUNNELS ON THE WEST SIDE, NO ONE COMES BACK FROM THOSE. NOT EVEN THOSE NICE MEN FROM LAST MONTH WITH ALL THE WEAPONS.
	\item IF KELLIGAN SEES YOU LEANING ON ME LIKE THIS, HE'LL KILL US BOTH.
	\item THERE WAS A MAN LOOKING FOR YOU. HE SAID HE OWED YOU MONEY.
	\item DO I KNOW YOU? AFTER THE WATER TURNED BLACK, I'VE HAD TO ASK EVERYONE THAT.
	\item I HAVE THE CURSE. YOU SHOULDN'T STAND SO CLOSE.
	\item I CAN'T FEEL MY MIND. STOP TAUNTING ME!
	\item THE BEER IS FREE TODAY. IT'S A LONG STORY.
	\item DON'T UNCOVER THOSE LIGHTS! THERE'S A WIGHT IN THE BUILDING.
\end{list}

\subsubsection{Ten Mid Level Adventures in Pandemonium}

You've found the sage you were looking for, but it looks like he's dead. His corpse is torn apart and lying on a heap against the part of the floor that's the ceiling to you. Droplets of congealing blood rotate slowly in the la grange points between ceiling and floor. He's got a piece of parchment in his cold hands, and it says\ldots

\listone
	\item wights have found me kill me kill me kill me
	\item I think this corpse will fool the howlers. At least for a while. If you wanted some water it's become more dangerous.
	\item NWNENWWS
	\item This man is an example. If Hruggek's Ninja Temple requests taxes, pay them.
	\item It's written in an old Orcish tongue. You'll have to find an Orc slain on the Prime at least a thousand years ago.
	\item The man's name is Gregor.
	\item Orcs! How I hate them! Their scimitars open the way!
	\item This is a ruse. The sage has escaped.
	\item Go back. Erythnul is not to be mocked.
	\item Itchy. Tasty.
\end{list}

\subsubsection{Ten High Level Adventures in Pandemonium}

The gates of the building have been torn asunder, as the characters run in, it seems that they're too late because\ldots

\listone
	\item Wights swarm over the insides, covering every piece of furniture with writhing limbs and moaning incessantly.
	\item Neogi great old masters hang from the ceiling, affixed by strands of hardened mucous.
	\item The pews stand empty as dust sweeps through the ancient church propelled by powerful winds.
	\item Hruggekian throwing stars are imbedded in virtually every wooden surface.
	\item A gaping planar rift hovers in the middle of the room, the winds of Pandemonium hurtling small objects into the void.
	\item The red dragon is already here, the hobgoblin princess is in his grasp.
	\item Black fires lick the insides of the room, the tomes are most likely destroyed!
	\item A tremendous serpent creeps over the tattered carpet.
	\item The winds howl even louder in here. Or maybe\ldots there are air elementals!
	\item A friendly and purring kitten is tossed back and forth by the terrible winds.
\end{list}


\subsection{High Adventure in\ldots Carceri!}

Point of fact: being in Carceri sucks. It's hard to leave, and it's an unpleasant place to be. That's the whole point. But believe it or not, those who please Nerull sufficiently are \textit{rewarded} with an eternity in Carceri. Now some of these people are just sadists -- creatures who enjoy the suffering of others so much that being able to assist in the degradation of others is payment and more for having to live in a hell dimension in the Prison Plane. But for others, life in Carceri is just genuinely pretty good. Some of these prison dimensions are minimum security white people jail -- there's a golf course and your ''guards'' are attractive women. It's still a prison of course, but if someone doesn't \textit{want} to leave, are they really a prisoner?

Anywhere you go in Carceri, it's all Evil, and people normally only go here if they are themselves Evil. That means that the people who are being punished here are being punished for \textit{failure}, not wickedness. The most successfully wicked individuals actually are rewarded here. Carceri can be a great place to introduce horrific elements into your story because by its nature anything that happens in Carceri, \textit{stays} in Carceri. Horrifying and depraved elements you introduce in a Carceri adventures don't have to apply to any subsequent adventures if you don't want them to.

\subsubsection{Campaign Seed: A Ring of Keys}

Carceri is a never ending parade of pocket dimensions filled with punishments and rewards that are both cruel and ironic. Travel between these cells is almost impossible, but there are ways. Most notably, there are maps that can tell you a secret path to get from one prison to the next; and there are adjustable rings that can transport a character directly from one prison to another depending upon how it is adjusted. Either can make for unlimited hours of enjoyment as players hop from one piece of episodic turmoil to the next. The maps work just like the map from \underline{Time Bandits}, and the rings work just like the devices from \underline{Sliders}. Really. Furthermore, those objects are authorized personnel \textit{only}, so if the players have one they are going to be hunted by Demodands with a new wacky scheme to catch them every adventure.

\subsubsection{Campaign Seed: Escape from Tartarus}

Just because you have been placed in a prison plane doesn't mean you deserved this punishment, or even that you committed a crime. The plane itself will punish impersonally, hiding its portals behind elaborate stages designed to elicit suffering.

Fight your way our of Tartarus, and no prison in any plane will every hold you\ldots

\subsubsection{Ten Low Level Adventures in Carceri}

You pass through the portal and find yourself in a new prison dimension. This one is\ldots

\listone
	\item Filled with thick, thorny foliage. Also it smells like boar and the thorns splinter and get into your armor.
	\item A town where the streets are filled with fighting.
	\item An expansive desert. Vultures fly overhead, but the scorpions seem unwilling to wait for you to die.
	\item A foul sewer. The water is waste deep. At least, you hope it's water.
	\item A scrubland with rusted iron spikes jutting out of the ground. Cages filled with starving madmen top some of the spikes, while other cages have long since fallen to the ground.
	\item A banquet hall stacked with delicious looking food. Haggard goblins look at the food with longing, but nothing seems to stand between them\ldots
	\item A windswept glacier. Far beneath you, there is a shadow in the ice. Far in the distance, a wolf howls.
	\item A stark stone room, where light filters oddly through a great number of spider webs and a dusty stained glass window.
	\item An earthy sinkhole. Worms poke through the topsoil everywhere around you, their eyeless heads wriggling like mad.
	\item A garden maze under an overcast sky. Fantastic shapes are cut into the hedges, and some ever seem to watch you.
\end{list}

\subsubsection{Ten Mid Level Adventures in Carceri}

If you could figure out the secret of this prison, you could escape\ldots

\listone
	\item The labyrinth seems to have four spatial dimensions\ldots
	\item The land shakes with earthquakes, but they still try to build houses.
	\item That eagle keeps eating that guy's entrails\ldots hey wait, I have entrails\ldots
	\item Why does that sanitarium seem to be inside-out?
	\item Why does everyone here wear a mask?
	\item Criminals in this put themselves into prison cells?
	\item The ghosts don't die when we kill them, and if we can't kill them we can't leave this building.
	\item It looks like a brothel, but who are the petitioners? The clients or the girls?
	\item The portal has a gold lock on it, and I was sure I saw a glint of gold in one of those oozes.
	\item An endless desert of white sand\ldots Or is it bone dust? 
\end{list}

\subsubsection{Ten High Level Adventures in Carceri}

If you just had it, then you'd be free\ldots

\listone
	\item A ship of chaos passes this way every day at the same time. If I could only make it notice me\ldots
	\item I almost have enough money to bribe the demodands into releasing me.
	\item That demon is a master of planar magic, and its said that his enemies get tossed to other planes.
	\item The fiends involved in the Blood War come from other planes. If I had an army large enough to impress them, they might show me a way out.
	\item If I could remember my home, I could just cast a spell and go home.
	\item The sage knows a way out, but he's so crazy that he'll only tell the secret to someone he considers a peer. What do I have to learn to do that?
	\item I can't believe that she's here. Do you think she'll forgive me?
	\item That war machine that looks like a bug the size of a mountain\ldots I hear its powered by a portal to the Astral Plane.
	\item I could open this portal, but I need the Blessing of Nerull.
	\item A wizard has been traveling Carceri for rare components, and it's said that he has access to plane-hopping effects.
\end{list}


\subsection{High Adventure in\ldots Hell!}

The Infernal Realm of Baator is essentially 9 infinitely large regions that happen to have a big pit that acts as a portal to the other 8 somewhere in them. So while the gods (and official publications) spend a lot of time worrying about that big pit in the middle, the fact is that the vast majority \textit{of Baatorian residents} don't even know it exists. Near epic play will spend an inordinate amount of time worrying about the goings-on around The Pit, and send the heroes off to go siege the fortresses around the ledge and such, but for the rest of your character's life the Nine Hells of Baator are just some inhospitable terrain filled with level-appropriate monsters.

\subsubsection{Campaign Seed: A Kafkaesque Nightmare}

Baator is home to one of the multiverse's most pervasive, efficient, and \textit{evil} bureaucracies. They don't lose your documents, they don't forget to mail things to you when they said they were going to, they simply have a set of rules that is at once awe-inspiringly complex and actually \textit{designed} to cause suffering to those who need to use its services. A campaign set around the backdrop of filling out forms sounds about as entertaining as doing your taxes in Hell, but there's ample opportunity for comedy, horror, and adventure in such a scenario, as well as ample prospect for character growth. The action starts when the characters need to change their registered employment, or want to protest their home getting knocked over to build a throughway, or perform some other completely mundane bureaucratic task. Unfortunately, the form they need to begin this process is clearly on display downstairs in the room marked ''Beware of Leopard''.

Surfing bureaucracy in Baator is about the only place where that makes for exciting D\&D adventures. The challenges to be overcome are social, mental, physical, and magical and efficient bureaucrats will tell you \textit{exactly what you need to do} to get things accomplished. This isn't like a Kafkaesque Nightmare on Earth, where you'll get stonewalled or your papers will just get lost, this is completely efficient and functional -- but designed by super geniuses to make your character uncomfortable. At lower levels there's a fiendish leopard in the room with the papers you need. At higher levels there's a golem that's supposed to stop people from entering the office where you need to convince a Gelugon to stamp your form. As the characters push their way to the top, they will find themselves in the position of being able to create their own red tape\ldots

On a side note: I just want to point out that my spell-checker recognizes ''Kafkaesque'' as a word. Sweet.

\subsubsection{Campaign Seed: Law of the West}

The great cities of Baator are infinitely far away from some of the nether regions of the plane. But the Law (and the Evil) still needs to be maintained. If you get far enough out into the boonies, Pit Fiends and the like just can't be bothered to show up and solve problems. So when Chaos (or Good) comes in to assault a frontier town, it falls to hard boiled individuals like the Player Characters to set things right. There's a new sheriff in town, and he's got levels in a PC class. This is your chance to use all your Western clichéin a fantasy setting, when you can turn Cowboy Movies into Kurosawa flicks.

Once the players beat back the gnolls who have come in at the behest of hyena ranchers trying to drive the gloom farmers off the land, the place is going to be a nicer place and attract Ogre Duelists or dishonest bankers. When it becomes known that the portal nexus is coming through, suddenly all that property is going to shoot up in value. And suddenly the pit fiends \textit{do} care what goes on in your sleepy neck of the woods.

\subsubsection{Ten Low Level Adventures in Hell}

It's a dusty little town, like an infinite number of others just like it both functionally and aesthetically. You don't know what makes this town special, and with the number of horrors you've seen on the plains -- you're not sure you want to. Still, this is a place it doesn't pay to break the rules when it isn't important, so the first thing you to is walk in through the curtain they hung up on the door to the Town Hall. Inside you see\ldots

\listone
	\item A dried out sahuagin sits behind the desk. He's mumbling about how the water is all gone.
	\item An officious imp attempts to shoo you right back out the door.
	\item Five corpses in fancy clothes lay strewn about the entrance hall.
	\item Putrid husks of humans in cages hang from the ceiling while a ghoul repeatedly jumps up trying to get at the rotting morsels.
	\item A mountain of papers covers the desk. From somewhere behind them a voice tells you that it is busy.
	\item A hobgoblin sits with his feet on the desk. As you enter, he stands up smartly and asks your business.
	\item Long lines of petitioners block off any hope of registering an time soon.
	\item Zombies shamble around the insides of the building and an imp is attempting to complete its paperwork while flying around the ceiling.
	\item The floor has collapsed entirely
	\item The front counter has been smashed and the interior smells like hyena urine.
\end{list}

\subsubsection{Ten Mid Level Adventures in Hell}

At last! You stand before the magistrate, it seems like you've been waiting for an eternity. You state your case, and he tells you\ldots

\listone
	\item ''You have the choice of death by platricorn or death by fire. Choose!''
	\item ''I grant you writ of ownership of Gelzugh's Tavern. You have the full backing of Hell in taking control of it from Gelzugh. Way back.''
	\item ''Your circlet is not \textit{jade}, it's \textit{malachite}, which is totally different. You're going to have to go back into the mines and find a \textit{jade} circlet.''
	\item ''Every one of you are sentenced to clean the sewers of Leng of the crawlers or die in the attempt.''
	\item ''It is Tuesday, so you're going to have to travel to Chitterport to have this taken care of.''
	\item ''Actually, this contract looks legitimate to me; Baelphor is legally the child's father.''
	\item ''I find nothing in this documentation to lead me to believe that these passports have been stamped correctly. Deport everyone.''
	\item ''You can't be serious. These swords aren't even magical.''
	\item ''Foolhardy mortals! You have wasted my valuable time and now I shall waste yours!''
	\item ''Raelzella's marriage is now void, the ownership of the larvae will be decided by combat.''
\end{list}

\subsubsection{Ten High Level Adventures in Hell}

Sorting through the ancient paperwork in the forgotten tower, you've found\ldots

\listone
	\item Documentation that proves that you personally are descended from an Erinyes.
	\item A small plush doll of a petrified Pit Fiend. It appears to be a \spell{shrunk item}.
	\item Spellbooks belonging to an evil lich.
	\item A map of a mighty fortress that appears to have stood where the shard spires stand now.
	\item Proof that a powerful Gelugon is not entitled to his position.
	\item A recipe for a dish now famous throughout the plane.
	\item Tongues of an ancient beast in a box. When the box is opened, the tongues speak of a fortress filled with giants.
	\item A portal to a deeper Hell in between the pages of a book.
	\item Poetry thought lost for a thousand years.
	\item Prophecies that mention you by name.
\end{list}


\subsection{High Adventure in\ldots The Abyss!}

The Abyss is well known for being infinitely big and infinitely bad in all directions, and it is. If there is some hellscape in your nightmares, its probably somewhere in the Abyss and there is someone there waiting to hurt you. The only thing it has going for it is that its very unorganized, meaning that the endless evil is only rarely directed enough the threaten other planes and planar oasis tend to places of great turmoil, meaning that small groups can easily blend in and ingratiated themselves amid the variety of beings that call these planes home.

Unlike other planes, there is no ''standard'' Abyssal Plane, aside from the top level called the Plane of Infinite Portals. These planes may be set up like a deck of cards, but they only share the chaos and evil traits, any particular plane can have any elemental or magic traits in the book and have geography ranging from the mundane mountains, forests, and plains to fantastic locations harmful to all but the most exotic forms of life. The only thing one can depend on is that pits and holes in the Abyss are often planar portals, and they only lead to deeper and wilder layers of the Abyss. Climbing back out of the Abyss is a much more difficult task, one that requires knowledge of planar pathways like the River Styx or powerful magic.

\subsubsection{Campaign Seed: We're the Exotic Products Trading Company (Abyssal Branch!)}

''We are here to serve your needs, and we offer a range of services ranging from capture of exotic lifeforms to collection of unique minerals and lore! We even have an on-call Search and Recovery Team available to recover lost individuals, `bargain' with demon governments, or protect important trade shipments! Contact one of our offices in Sigil or our home office on the Plane of Infinite Portals!''

\subsubsection{Campaign Seed: Pirates of the River Styx!}

''Yo ho, me hearties! The River Styx be vast and mysterious and its waters kiss the Abyssal planes like a cheating lover! Why set sail in the other Lower Planes when the Abyss is infinite and lawlessness is a virtue of its people? The good boat The Groping Marilith has room for any brave soul whose handy with steel or spell and has an eye for exotic and demonic beauties in every port and magic and jewels hidden in the nether regions of every fiend. Come ply the Abyss with us, and forget your troubles on the River Styx!''

\subsubsection{Ten Low Level Adventures in The Abyss}

\listone
	\item Food Run! Demon weevils have infected an Abyssal Town on the river Styx, and the first group to bring untainted food for them will earn a valuable ally.
	\item Race! A Nalfeshness ruler of miles-long city straddling the River Styx on the 33th level of the Abyys has decided to host a riverboat race to please his unruly people. There's big money to be made in this no holds barred sailing race through an Abyssal city!
	\item The good ship Lollyjaws is plying the River Styx with its zombie crew, and they've decided that you've hit a big score and you need help ''investing'' it.
	\item Message in a bottle. A map written in Celestial has been found in a blottle on the River Styx. Its this a map to a treasure, some poor soul's hope for rescue, or a clever trap to capture well equipped adventure seekers?
	\item Run aground! A chaos ship containing mysterious spices and drugs and run aground near a port town, and its bedlam as psychotropic clouds spew forth to wreak chaos in the town. Loot the vessel before the helplessness of the townsfolk attracts powerful fiends who'll sweep up the any booty.
	\item A dark, beautiful, and mysterious stranger decides that only your organization can retrieve a packet of information from the 411th plane.
	\item Mapquest! Map a planar route to an exotic locale in the Abyss, and return to collect your reward.
	\item ''There's an emergency! Deliver this call for help to the 911th plane!''
	\item Worm farmer! Travel to Noisesome Vale on the 489th layer of the Abyss and collect samples of the worms that eat sulfer gas and exhale breathable air for a Fiendish Gnome client with ideas for a Styxian submarine.
	\item An erratic portal between the 1st and 239th planar has started functioning properly again, and the Lost of City of Azzabanazanazan has been found (much to the inhabitants surprise). A little clever negotiating between this city and a few of the more popular demon cities could mean big profit.
\end{list}

\subsubsection{Ten Mid Level Adventures in The Abyss}

\listone
	\item Naval vessels of the Nine Hells have made serious incursions along the river Styx, and a clever ''privateer'' can make a little coin by signing up with a demon lord to resist these salty devils.
	\item Smiley Tom, the infamous Incubus captain of the legendary Slippery Cat has been imprisoned in Graz'zt realm for unknown crimes. Rescue him to gain his legendary gratitude, or use this opportunity to steal the Slippery Cat, the greatest ship to ever sail the River Styx.
	\item The Forgetful Fog Technique. Some clever pirate has figured out a way to create fog on the waters of the River Styx, then push these vapors onto towns and cities, looting them silly while the inhabitants are blissfully unaware. Catch these clever thieves to stop their amnesiac attacks, or perhaps gain a monopoly on this tactic yourself.
	\item One of your mates have finally bedded one lass too many\ldots she's been granted a wish by a glabrezu, and ill-luck follows your mate and his friends(which is unfortunately you). Win her affections back or find her a new romance in the Abyss, or else the curse will be the end of you.
	\item Ever hear of the sea elves living in a city hidden under the River Styx on the 356th plane? Their touch steal memories and they sell them on the demonic market and\ldots what was I saying? Hey, who are you? Who am I?
	\item A lazy balor chief running the glorious demon city of Belzasharazar on the 45th layer wants a new pleasure palace constructed, but his succubus consort has other ideas. Burn the construction often enough and he'll lose interest, and you'll earn a powerful patron in the demon city.
	\item The latest fad in Sigil is the practice of keeping glowing dragonflies as party lighting, but these exotic insects are found only on the 232nd layer of the Abyss, a plane suddenly caught in a vicious conflict between two barely-known demon lords. Deliver a shipment of these blinky bugs to Sigil and you'll be invited to all the best parties, opening up other pecuniary possibilities.
	\item You've been approached by a cabal of wizard from the Prime, and they want information on the Black Tower. Infiltrate the Black Tower to steal their secrets, or turn sides and lead a strike force to the Prime to nip these nosy wizards in the bud.
	\item A cargo box shows up on your door with a valuable, but difficult-to-sell and dangerous product (like a shipment of souls), and several parties seem to think that you are the owner. Find a way to sell the cargo to a more powerful individual or else these parties will take it from you with extreme prejudice.
	\item An old associate has deeded you a confectionary in the City of Brass that specializes in demon chocolates and sweets. The Sultan has decreed that if you don't pay back taxes in city of Brass currency that he'll foreclose on the property (and your soul). Go on a whirlwind tour of the Abyss to collect enough stock to make enough quick cash to save the shop (and your hide).
\end{list}

\subsubsection{Ten High Level Adventures in The Abyss}

\listone
	\item Over a dozen pirate ships working the River Styx have been declaring that they are part of an Armada in order to pass along blame, and they are saying that you are the Admiral! Find and smash these lying upstarts or ''gently convince'' them to actually accept your command.
	\item A general in the Blood War has found a way to divert the River Styx and he is using these pathways to strike key demon and devil armies, killing both his enemies and competitors. Both sides are willing to handsomely reward the party capable of ending this maritime terrorism.
	\item Rumors and hints point to a powerful artifact being transported along the River Styx in a vessel of unusual design, and factions vie to be one to seize this powerful item.
	\item An island has appeared in a notoriously wide section of the River Styx, and dragons have been leaving the island to raid vessels. By your estimation, they should have amassed a horde that is fantastically large, even by the standards of dragons.
	\item The Mask of the Captain has resurfaced, a powerful artifact that creates and closes permanent gateways between the River Styx and the Prime Material Plane, and a powerful Prime nation has decided that they will increase the wealth of their people by plundering the cities of the Abyss.
	\item A trading vessel of unusual design flies into the Abyss, avoiding known planar pathways. It is crewed by a race that planar sages have never seen, and they offer trade goods of exotic and powerful design. Is this a simple trade mission, or an incursion from another plane by a new planar power?
	\item Orcus's agents have begun purchasing magic items related to planar travel, hinting at an invasion of enormous proportions.
	\item A demon lord of waning power has declared that his power and command over his layer of the Abyss will pass onto the individual to defeat him in single combat, and contestants have gathered at his fortress. Is this a ruse to gather the equipment and souls of powerful individuals, or is he truly offering a chance at the title of demon lord?
	\item An old friend brings news of the discovery of an empty city found in perfect condition in the Abyss full of trade goods and magic, but without a single living or undead soul. To take control of this city is to learn its secrets, and possibly gain its enemies\ldots enemies unconcerned with wealth or magical power.
	\item Yeenoghu has decided that you are a demon lord in disguise who is pretending at weakness as a ruse, and he is sparing no cost to send agents to test this theory. Convince him that you are a mortal, or strike him so hard that he ceases his attacks.
\end{list}


\subsection{High Adventure in\ldots Gehenna!}

First, it's the home of the Yugoloths. These outsiders are the dealmakers and compromisers of the fiendish world, so they might be involved in any plot or any scheme that makes its way across the planes. The land itself is series of volcanic mountains where sentients have forced their own existence into, jammed between the Hades and Hell and connect to the River Styx, so it is well situated between several of the Lower Planes. The works of mortals and immortals alike are eventually destroyed by tremors in this architect's nightmare of a plane and only the works of the gods last here. That being said, the entire plane has an angle that ranges from inconvenient (45 degrees) to unlivable (straight up), meaning life in Gehenna is far more socially dependant than other Lower Planes due to the fact that the only place to live is in the cubbies, caves, boltholes and settlements that litter this plane. It's not that you can't live in on the slopes and are forced to cooperate and co-oexist and you are forced to compete for space like in Hades, its just that life in Gehenna without a clique \textit{sucks}.

What do all of these things mean? It means that Gehenna is a realm for movers and shakers, a place where ''the deal'' and ''the juice'' matters more than any ideals or hopes. Even the petitioners of this plane are only concerned with power, and only the cruel nature of this plane keeps them chained here. Brinksmanship and counting coup and favors are the symbols of power here, and mere physical might or magical power take a backseat to one's ability to \textit{manipulate people with physical power and magical might}.

\subsubsection{Campaign Seed: The Yugoloths Want You!}

While Tanar'ri generals are known the power and might of their hordes and Baatezu armies are know for their frightening disciple and efficiency, it is the Yugoloth forces that are know for their subtlety and tactical elegance. They don't fight for reputation or honor; they fight to fulfill a contract and make a profit, making them among the deadliest generals in the Lower Planes.

You've joined that organization now, and the Yugoloths have need for elite squads of problem-solvers with a propensity for violence and a capability for discretion.

\subsubsection{Campaign Seed: The Grand Game in the Crawling City}

In the Crawling City, you've got to be useful or you're dead. You attached yourself to a minor Yugoloth noble, and he's begun using you as behind the scenes agents in the Lower Planar courts. With skill and nerve, one day you might earn the fear and respect of the fiends and become a player in your own right.

\subsubsection{Ten Low Level Adventures in Gehenna}

\listone
	\item A famous Yugoloth tactician is taking new students, and he's set a distinctly fiendish entry requirement: interested students publicly apply, and one week later the first to present themselves is accepted. The last time he took new students, no applicant ended the week alive enough to show up\ldots
	\item Small bands of petitioners have been gathering under the banner of a charismatic profit and raiding minor settlements in the night. Eliminate the threat by assassination or counterattack.
	\item Tremors! Minor rumblings and a trusted fiendish seer predict a major lava spout in a small settlement, destroying it, and several interestied parties want to loot it or the refugees in the final hour. Intercept these rogues, or plunder the settlement for yourselves.
	\item A minor Baatezu noble has been spotted in the Crawling City, and it's suspected that he's trying to hire away an elite group of Baatezu mercenaries when their current contract expires. Find and interrogate him, and the Yugoloths will repay this little favor. Whether he returns to his home plane with his life and valuables is your own business.
	\item The Double ''H'' Run. Despite the Blood War, some trade does exist between the Baatezu and the certain Tanar'ri, and the Yugoloths have their hand in it. Escort a package between the Nine Hells and Hades, avoiding agents from both fiendish factions who would use it to discredit their countrymen.
	\item The Masked Ball is next week, and a clever soul capable of learning the identities of several indiscrete parties can earn a few coins with the information brokers of Gehenna.
	\item A tiefling fop of a swordsman has defeated several prominent Yugoloth blademasters in mostly fair duels, despite his obvious lack of skill. Several persons of note would like to know his secret, and would pay even more to have that secret removed at an opportune moment.
	\item A mortal Sorceress of rare skill and infamous carnal desires has come to Crawling City, and entities of power are jostling to be known as one of her clients. Secure her cooperation for a client and win wealth; secure it for yourselves and win power and danger.
	\item A Tanar'ri of an unusually Lawful bent has entered the service of a Yugoloth of middling power. Discover the secret of his service, and that service can be passed on to a more worthy fiend, or kept as secret weapon for yourself.
	\item A Yugoloth of some influence has secured the services of an unusual household staff of famous, though powerless, Prime mortals. Spoil his coup by tempting, tricking, or intimidating these mortals into committing terrible blunders during the next power meeting, and you can harvest some amount of his influence.
\end{list}

\subsubsection{Ten Mid Level Adventures in Gehenna}

\listone
	\item A powerful Tanar'ri fortress has been bidded for destruction, and the Yugoloths will pay well for the group that finds an exploitable weakness.
	\item Several subcommanders have been bickering over the right to extract a powerful dragon of a military bent from Carceri, and rewards will fall upon anyone capable of securing this beast's services for the Yugoloth.
	\item A key planar touchstone in Hades will prove the key to an isolated fraction of the Blood War, insuring victory for one side or the other. Destroy this site, or profits for the Yugoloth in this conflict will fall dramatically. Secure it for yourself and turn it against both armies to secure a stalemate, and some fraction of the increased profits will fall your way.
	\item A powerful Yugoloth well- known for patronizing up-and-coming allies has declared that you are his protege, making you a target for his enemies Punish these enemies, and you might secure his patronage for real.
	\item A small army in the Blood War has wandered into Gehenna and is a threat to the Yugoloths. Destroy its leadership and loot its paymaster, and the Yugoloths will see that you are amply rewarded.
	\item A band of thieves have turned the Crawling City upside-down. Recover and return the valuable objects and win influence. Hold the objects hostage for future favors, and gain power that money can't buy.
	\item An unknown magical effect has stopped the feet of the Crawling City, and the first to discover the cause will win no small amount of gratitude from the ultraloth ruler of the city
	\item A series of businesses across Gehenna have been vandalized, an obvious turf war between two competing interests, and the first group to discover the identity of either player can earn a contract to accelerate or reverse the destruction.
	\item A spellbook of unique magics useful to a courtly mage has been found, and the owner of such magics would pay handsomely to not have his secrets revealed.
	\item A Baatezu diplomat has come to Crawling City, and he has decided that you will become his agent. Avoid a diplomatic incident without betraying the Yugoloths, and the powers that be may reward your ability to resolve such a conflict.
\end{list}

\subsubsection{Ten High Level Adventures in Gehenna}

\listone
	\item A cabal of liches have a sudden need for several rare components, and they are willing to trade battlefield magic for the first party to collect their list.
	\item It has come to your attention that several key subcommanders are plotting a coup over the control of the Crawling City. Shatte this conspiracy, or risj all and become its ringleader.
	\item The Yugoloths are looking to subcontract a dangerous mission on the prime against a noble house of demon-hunters. Get the contract and eliminate the hunters, or accept a greater bribe from the them to hold the contract long enough for them to counterattack.
	\item Key contracts for the Blood War have been stolen, and the first person to recover them will control a Yugoloth army of immense proportions.
	\item A war machine of great size and terrible power has been spotted in Mechanus, and such a device would fetch a king's ransom in the war markets of the Crawling City.
	\item A clique of fiendish spellcasters has set a challenge: the first entity to scour the planes for a specific but almost unique spell will earn a tome of their greatest spells. They expect one of their members to win and then resolve a dispute about claims of leadership of the clique, but an indiscrete servant blabbed the rules of the contract and now several interests seek to win the contest.
	\item A mortal noble of rare talents has entered the Crawling City and is recruiting agents for one goal: recover the contract that dooms his soul to property after death. To help him is to defy Yugoloth tradition, but the rewards might just be right.
	\item For some unknown reason, Inevitables stalk the Crawling City, and a clever stagemen might just be able to divert them towards one's enemies.
	\item The ruler of the Crawling city is missing, and chaos rules as several factions make a bid for power.
	\item Negative energy has begun to permeate the Crawling City and undead powerful enough to challenge of Yugoloth leadership have begun to rise. Is this an attack by a god whose portfolio is death, or some ruse to put the Yugoloth against an enemy they cannot defeat.
\end{list}


\subsection{High Adventure in\ldots Hades!}

One would think that Hades is among the worst Lower Planes to adventure in\ldots and they'd be right. The plane itself has the two nasty qualities: it poisons you with the Grays until you become a depressed Goth, and the Entrapping trait takes your memories and makes you want to never leave like a bad house guest. That being said, adventure is still possible, even for the least powerful adventurer.

It works like this: think of Hades as an unforgiving desert. Travel in this ''desert'' is only done by moving from oasis to oasis. These oases are towns and settlements that are built in such a way to resist the Grays and the Entrapping trait (see the Handbook of the Planes for an example of such a place). The only things that permanently live in the desert are creatures who are both immune to the Entrapping trait (like outsiders) or who have already succumbed to it (which has no other game effect other than ''become an NPC who doesn't want to leave''); these creatures also have some way of dealing with the Grays, and so they are creatures with SR 10 or better or are immune to Wis damage (like undead). This generally means that the ''desert'' that is Hades is filled with wild-eyed hermits and bandits and other forlorn spirits (which might be actual undead) living in the blasted and ruined geography of Hades, or creatures of some special power who skirt the edges of civilizations. Some NPCs you meet might just be Entrapped, but enter an oasis once in a while to recover from the Grays; other such characters might have ways to cure the Wis damage that the Grays cause, thus they are entrapped by Hades, but have no reason to enter an oasis, and some powerful creatures can resist The Grays almost indefinitely due to their high Saves.

Hades also has a few other features of note: It's the ultimate source of Evil of all types, and all of the evil outsiders are equally (un)welcome there. You could easily see a Yugoloth, a Devil, or a Demon without that being part of a plot device. Since Hades is the creation place for larva, the serving-sized petitioner souls of very evil people, the big evils of the multiverse have taken to fighting and brokering for this natural resource full time, and it all starts here. Night Hags and Liches are other players in this economy, but they are the freelancers in the publication of evil.

\subsubsection{Campaign Seed: The End of Oasis}

You've lived in the town all your life, and you know that only madmen and the 'loths live beyond the walls, but now you must travel the wasted plains to find your destiny.

\subsubsection{Campaign Seed: A World At War}

The Blood War wages endlessly and pointlessly across the Gray Wastes, with most territory never held or even claimed. The only things that have value in this whole plane are the occasional portal, oasis, or larva vein. Every other patch of land is a liability and \textit{no one} wants it.

\subsubsection{Ten Low Level Adventures in Hades}

\listone
	\item A Yugoloth has died while on a trading mission to your town, leaving behind a shipment of larva. To prevent your town from falling under the 'Loths gaze, you must take them to the nearest Yugoloth city for sale.
	\item A battle in the Blood War was fought near your town, and the undead fodder from that battle now terrorize the countryside.
	\item The leader of your town wants it to become a waypoint for message delivery, and he hires you to delivery the first messages.
	\item Something has been coming in from the wilderness to stalk the townsfolk. Will you track it back to its lair outside of town?
	\item The well has been poisoned, and you must find a new source of water for the town deep underground, far from the protective influence of your home.
	\item A terrible new disease has been ravaging all the nearby towns, and the Oinoloth has decreed that the town with the best gift will be spared.
	\item Devil agents want to construct a supply depot far from their own infernal realm, and will pay well for the location of new oasis(minus any current inhabitants).
	\item The nearest town has its eye on the riches of your town, and now has agents and a small force scouting for weak points and key personality to kidnap.
	\item Two caravans have entered your town at the same time, and now they have begun attacking and sabotaging each other at night in an effort to be the only one to leave.
	\item It's Election Day! Factions in town work against each other in an effort to become the new Mayor, and everyone knows that the loser will end up exiled to the wastes.
\end{list}

\subsubsection{Ten Mid Level Adventures in Hades}

\listone
	\item For some, mere death is not a real revenge. A powerful leader hires the party to defend a prison built in order to entrap entities in Hades in a spot unprotected from the effects of the plane.
	\item A legion of elemental soldiers have been led through a Gate, and they have succumbed to the effects of the plane. The first town leader to convince them to join him will gain a powerful fighting force.
	\item The Yugoloths have decided to annex your township, and only a show of overwhelming force or a high bribe will convince them to leave your town alone.
	\item Something is destroying oasis after oasis, isolating your town from the trade paths.
	\item A Gate has been opened to Celestia, and celestials have offered asylum to your township. Is this an opportunity to evacuate your town, or is this a fiendish trick to destroy your town?
	\item during a battle in an unfamiliar oasis, your party is trasnported to an unknown location in Hades, far from any oasis. Can you find your way home, or even to a safe location before you succumb to the planes traits.
	\item A series of Gates have opened up to a distant region in Hades, and townships now vie to control the altered landscape.
	\item The river Styx is flooding, and threatens to wipe out several cities built on its waters, including your town's primary trade partner.
	\item A caravan of bioloths has entered your town, beginning a carnival that threatens to enslave everyone.
	\item A powerful Yugoloth has been working against the Oinoloth, and your town is caught in the cross-fire. Will you work against it, or for it?
\end{list}

\subsubsection{Ten High Level Adventures in Hades}

\listone
	\item Rumors hint that your town holds a mystical font that can make anyone bathing in its waters immune to Entrapping and the Grays, and several powerful forces vie to control this wonder.
	\item The Blood War has boiled up in your region, and a clever party could benefit from working with one side or the other, or even both.
	\item A powerful devil decides that he needs more exotic troops, and he is willing to extend his protection to your town if you can capture powerful creatures from several legendary parts of Hades.
	\item Angels have gained a foothold into Hades, and have decided that your town is the first to be ''purified.''
	\item During a particularly brutal battle in the Blood War, a powerful artifact has been lost. The first to regain such an artifact might be a threat to even the Yugoloths.
	\item A cabal of Night Hag Sorcerers have decided to harvest your town, and the only way to catch them is to breach the barrier between your plane and theirs.
	\item A powerful outsider offers his services to your town, saying that he can create planar gates. Such a resource would transform your town into a planar metropolis, but can it survive the attention it will attract?
	\item A powerful Warlord has taken over rulership of several towns, attempting to build an empire in Hades, and your town must either gather the forces of the surrounding towns to fight this menace, or usurp rulership for yourselves.
	\item A dangerous wizard has found a way to concentrate the evil of the plane, and he is using this evil as weapon that can corrupt even the Yugoloths to his person brand of evil.
	\item Strange and terrible diseases are taking their roll on all the inhabitants of Hades, and the only way to stop these plagues is to assume the mantle of the Oinoloth.
\end{list}

\section{High Adventure in the Elemental Planes}

\subsection{High Adventure in\ldots The Plane of Fire!}

More than any other Inner Plane, adventures in the Plane of Fire tend to take place in planar bubbles. If you can breathe water, the majority of the Plane of Water is basically just a lukewarm benthic zone, and it's the kind of place that Sahuagin might live without even realizing that they weren't on the Prime. But the archetypical expanse of the Plane of Fire is just, well, fire. It's like the churning surface of a sun that extends in all directions for eternity. And while it is colder and less destructively melty than the all-consuming plasma of an actual star, it's still basically just an endless expanse of fluid, dangerous, useless fire. Did I say useless? You bet, because heat engines actually work by heat difference, so from the standpoint of residents of the Plane of Fire it is actually cold that you use to run a power plant. The fire in between everything is just like the worthless emptiness of deep space except that it will also catch you on fire. Forget Carceri or the Gray Wastes - the Elemental Plane of Fire is the worst place in the D\&D multiverse.

But just because it's a horrible place, even the worst place, doesn't mean that there isn't stuff you want there. And just because it is the most inhospitable place imaginable, doesn't mean that low level characters can't adventure there. The key is the planar bubbles exist. That is basically the only reason that anyone gives the Plane of Fire the time of day. The most important bubbles are Prime Bubbles. These are areas of land and sea with atmospheres, that happen to be shaped like a Ptolemic world - a circle of land and sea with a hemisphere of atmosphere above. And of course, outside that is endless roiling fire. So the ground gets kind of rocky and parched, what with the sky being a never-ending holocaust without reason or respite - essentially it's like living in a Dragonforce video.

Those Bubbles aren't just the only place your characters can survive, they are the only places that any of the residents give a damn about. Remember that even if you happen to be a fire elemental, you still eat ''flammable" materials if you want to grow any larger, and those only come from the ''cold" spots. So not only is the practically usable terrain in the Plane of Fire very small compared to the plane's total volume, but the space between is inhospitable void. And not just inhospitable void - it's opaque inhospitable void. Standing on one of the floating islands, you can't even see the other islands. When you look into the inferno you have no way of knowing whether the next place of value or substance is a few centimeters or a few parsecs of burning emptiness in any particular direction.

So what does that mean for the low level adventurer? It means that practically speaking, no one expects your character to want to go anywhere that would cause them to actually catch fire. No one else does, not even the planar residents who are actually made out of fire. So it's totally workable as an adventure locale at any level. The Plane of Fire is run by the Efreet Sultans, and that gives the entire place a very fantasy-Arabic feel. Ignan, the approved lingua franca of the universe, is explicitly based on Arabic. That thing where Arabic calligraphy kind of looks like living flame? Yeah, they went there. While the Djinn have a presence in the Plane of Air and the Dao have their own Caliphate in the Plane of Earth, the Sultan of Fire owns the Plane of Fire. Because there is hardly any real estate, and finding or getting to it is in most cases a Wish Economy proposition.

The Plane of Fire is your chance not only to throw out every Arabian Nights cliché you know, it's also a place to throw in 1950s sci-fi left and right. Basically everywhere that anyone lives is one of those bubble colonies or asteroid mining facilities from the Heinlein juveniles. To get from on planetoid to another requires getting into a heat protected shell and then throwing yourself from one to the other. Once you leave a Planar Bubble, there's no gravity or wind, so it's basically exactly like one of those personal space ships that were talked about in the old Republic Serials. Some of them are even saucer shaped.

Campaign Seed: Conquest of the New World: Even beings of pure fire cannot see far into the firmament, and so it is that new places of interest are ''discovered" all the time in the most surprising of places. The iron ships that travel between bubbles need exacting angles of departure, because once they are off course, there's really no measurement you could take to figure that out (and often nothing you could do about it if you did). So a new island might well be just 1 degree off an established trade route. And once a new land is discovered, it's Columbian Conquest all over again. This new world may well have occupants that object to being ''discovered" let alone colonized, but on the other hand they could seriously have fountains of youth or cities of gold.

Exploring a new Planar Bubble in The Plane of Fire is a good way to bring out any kind of D\&D adventure you want. The PCs have literally no idea what they might find there, and there's a very great incentive to keep exploring since even wood and water are hugely valuable resources once you get off this gravity well and back to a more civilized one. You don't just get to loot the temples of stone using pyramids, you also get to confront their heathen demon gods, find relics of fallen ancient civilizations or the secrets of long forgotten wizards. A Planar Bubble that ''no one" knew about on The Plane of Fire is about the safest place in the entire damn multiverse, so anyone who did know about it could have stored or imprisoned, well, anything there.

\subsubsection{Campaign Seed: Janissaries of the Fire Sultan:} The Efreeti sultan is cruel, but he is not stupid. And he is well aware of the limitations of being a guy who is on fire all the time when the only things in the entire universe that have any value are things that are not on fire. And so it is that the Fire Sultan has children of non-flaming races raised in his employ. These children grow up to be Janissaries: creatures who act as agents for the Efreeti and build their empire without incidentally burning it down. There is a lot of room for advancement in the Janissaries, the Sultan genuinely values your skills more than he values the skills of the other Efreeti. First of all, there is basically no chance of you ever actually becoming Sultan (you just don't have the right fire in your blood), and secondly, unlike a real Efreet, you can do stuff that the Sultan cannot. There are a lot of politics that go in court, and the rest of the Efreeti have a tendency to rather resent Janissaries; while at the same time doing their damnedest (literally) to avoid any direct confrontation with something the Sultan considers to be ''his." Do the Sultan proud, and you can have your every wish granted (as long as that wish doesn't include becoming Sultan or leaving the Sultan's employ). Fail him sufficiently, and he may allow the more jealous members of the court to take their frustrations out on you.

\subsubsection{Ten Low Level Adentures in The Plane of Fire:}

You're getting the report from the overseer of the pipeline workers. The Kobold tells you that they aren't getting as much water as expected because\ldots

\listone
	\item A group of Firenewts has claimed that the pipeline runs through their tribal lands and have begun monkey wrenching.
    \item The water reserves aren't as extensive as hoped near the surface, and the pipeline will have to be extended into the caverns.
    \item Superstitious fears have broken out among the workers, they speak of burning snakes.
    \item Drilling has broken through to inferno before expected, this rock isn't as stable as we'd hoped.
    \item The water has some kind of creatures living in it. Creatures that live in water.
    \item Some creatures have been bringing clouds of smoke with them when they crawl over the pipeline.
    \item A rival mining group is siphoning water from our reserves.
    \item Some guy who looked like a Yak has paid more than enough money for the land to get the crew to drill elsewhere.
    \item Everyone who touches the water seems to forget what they were doing.
    \item The water has been draining up to the mountain.
\end{list}


\subsubsection{Ten Mid Level Adentures in The Plane of Fire:}

Laughing, the Efreet relays the news. It's never a good thing when an Efreet is happy to tell you something, and this is no exception because\ldots

\listone
    \item Some group of xorn came in with a load of opals just two days ago. You're going to have to go farther afield if you want to liquidate those gems.
    \item It seems that while you were out, they've made a new appointment of Sheriff.
    \item The land title has been revoked and given to Hakim
    \item Surtyr wants his money back. Now.
    \item Yak Men have taken over the entire city.
    \item A Red Dragon has claimed the water reserves.
    \item The Bubble has begun wobbling, the only way home is by wish.
    \item The princess is in another palace.
    \item The gnomes have themselves a Frost Salamander that they are keeping alive somehow, and mere flammables are virtually worthless here.
    \item The great astrolabe has been shattered.
\end{list}


\subsubsection{Ten High Level Adentures in The Plane of Fire:}

The Iron Flask isn't completely inscrutable, and your research indicates that it contains\ldots

\listone
    \item One of the Sultan's uncles.
    \item A potion of Immortality
    \item A gate to a deep layer of Baator
    \item The heart's blood of Baphomet
    \item The phylactery of a powerful Lich
    \item A decree from the previous Sultan
    \item A heretical Genie who was imprisoned for predictions that appear to have come true.
    \item The crown of Pyriria
    \item The condensed gaseous form of a Chaos Roc. One of several, if the accompanying letter is to be believed.
    \item The laughter of Queen Chandra.
\end{list}


\subsection{High Adventure in\ldots The Plane of Water!}

The Elemental Plane of Water is an endless expanse of relatively static water permeated by a soft ambient light. There is only gravity if you want there to be, and the incompressible medium makes gravitational movement slower than walking. But nonetheless, you can move pretty much anything at the rate of about three and a half miles per hour just by ''falling" or ''rising" with it. Outside of an occasional ''pressure zone" the entire plane is pretty much one giant coastal shallows, with a water pressure at any point about that of being under just a meter of water. The Elemental Plane of Water is also the largest place in all of the D\&D multiverse in real terms.

Sure, it is ''infinite in all spacial dimensions and time" just like all the other Inner Planes, but it is markedly different in that every point in the Plane of Water is also a place. None of it is empty or impassable, it's all just made of water. So you can go and be anywhere, and you won't be ''between" things because the place you will be will be an actually stable location in and of itself that you can put stuff down in or give directions to. Every point. And that means that there are more places to be, and by extension more stuff than in any of the other planes. Indeed, like how on Earth about 70\% of your body is water, and about 70\% of the world's surface is water, about 70\% of the creatures and structures in the Inner Planes are on the Elemental Plane of Water. And like the oceans of every Prime World - the Plane of Water still gets less press than the other planes because it is full of water. In general, things on the Elemental Plane of Water stay where they are put, with little in the way of mobility. This means that when there is an air bubble, people can pretty much run around in it without fear that the air will bubble up away from them. Because there is no up. This also means that disposal of bodily waste is ''gross." There is nowhere to ''bury" anything, so stuff that comes out of you just sits there accusingly. Fortunately, there are a lot of plants and little animals that will come clean that up, but this process is no nicer to watch on the Plane of Water than it is anywhere else. There are areas where, for whatever reason, the ambient water is flowing with some kind of current. Some of these currents are incredibly fast, but as a rule they are not that ''large" and full mixing doesn't happen. The fresh parts of the endless sea stay fresh and the salty parts stay salty. The hot parts stay hot and the frozen parts stay frozen.

The Marids are, individually speaking, the most hard core of the Genies. However, the Great Padisha of the Citadel of Ten Thousand Pearls is basically just the mayor of a town of one thousand occupants. One thousand occupants where one in five of them can grant frickin wishes, but just a thousand all the same. You could seriously move around the plane your whole life and never come within the demesnes of a Marid. Each Marid considers themselves to be royalty and to rule all they survey - which is basically true but functionally meaningless because you normally can only see about 60 feet on the Plane of Water because there's microbes and sand and stuff in the water pretty much everywhere. This contrasts sharply with the Sahuagin empires, some of which are ten thousand miles across (note: this is bigger than the entire Earth, and we're talking volume rather than surface area, so some of these empires have populations that measure in the tens of billions), but which due entirely to the sheer vastness of the plane and the smallness of any visitor's personal experience of the place (60 feet or so around them and movement as fast as they can sink or swim), it is still entirely likely that you've never heard of any of them.

While the visibility on the plane of water is total crap, the audibility is intense. Water is nearly incompressible and it's nothing but water forever and ever. Sound pretty much follows the rule that any noise is four times as quiet when at twice the distance, with no additional dampening from the atmosphere. Any noise ever propagates with such totality and speed that to the human visitor it is nothing but a constant deafening roar. Indeed, since sound travels so much faster in water than in air, any non-aquatic visitor needs 10 ranks of listen to even have a hope of locating any sound. Even sounds that are loud or close enough to be distinctly made out sound like they are from “everywhere.” This is not a problem that natives have, and indeed a Sahuagin can locate you by the sound of the water against your skin.

Secession is constant in the Plane of Water. Anyone can just pick up their house and leave at a bit over 3 miles an hour. Between this tax day and the next, you could have moved your house about 29,000 miles – which is noticeably more than the circumference of the Earth. And when you factor in the fact that there is no guaranty that anyone will find your house if you move it 100 meters, one can see that you can vanish from a government's radar very easily if you are not actively imprisoned. The standard therefore is to be required to pay taxes to the local authorities at the beginning of the year and subsequently be allowed to provide proof of citizenship to receive services for the following year. Surprisingly, much of the civilization in the Plane of Water is actually more recognizable by connoisseurs of modern nationalism than are the kingdoms of other planes of existence. If you want to live in a “country”, you have a citizenship card and rights and social services and stuff. Anyone who doesn't want those things (or doesn't want to pay for them), just leaves and lives elsewhere in the roaring darkness.

\subsubsection{Campaign Seed: Heralds of the Empire:} Sound travels fast under water, but news does not. When a new nation takes hold of a region, it can take a long time to even find everyone who lives there. And so it is that any nation state or empire needs to send out groups to patrol their territory. Not just to keep an eye on the citizens and provide whatever services the empire provides to the hinterlands – but also to keep the maps updated. After all, any part of the empire that hasn't been patrolled in the last month could seriously have had someone move a castle from 4000 kilometers away to there in the meantime. As representatives of the state being sent into areas of water that the state either has not been to yet or has not been to recently, the PCs could encounter pretty much anything at all. And they have a built-in plot hook that encourages them to interact with anything they fine. Whether they face level appropriate wandering monsters, social encounters with dubious locathah, or hostile empires coming the other way, the PCs can plausibly encounter level appropriate opposition at any level.

\subsubsection{Campaign Seed: Tidal Merchants:} The great tidal streams are currents that move with surpassing speed. Those who ride them can get places that are very far away in very short periods of time. And that's saying something in a world where seriously anyone can tie themselves to their cargo and “sink” 80 miles a day just by deciding to. The currents don't just provide fast transport, they also provide a path, a place to go. And so it is no surprise that as one drifts along the tidal stream, one can hear the drums of civilization from all sides just as you can see the glowing lights of fast food joints while driving on a freeway on Earth. Traveling along the tidal streams brings one from one urban development to another with all the vast spaces between literally washed away.

\subsubsection{Ten Low Level Adventures in the Plane of Water:}

The old Locathah is certainly interested in your proposal. But he says he has other problems\ldots

\listone
    \item Sahuagin raiding has hit several nearby kelp farms.
    \item Shark attacks are on the rise.
    \item No one seems to want to buy the sponges he has been growing.
    \item His daughter has the ick.
    \item Food supplies are running low.
    \item The fish are migrating out.
    \item A local hot spot is attributed to Fire leakage.
    \item Those who die seem to come back as zombies.
    \item A siren has been throwing her weight around.
    \item Pirates have seized the oyster bed. 
\end{list}

\subsubsection{Ten Mid Level Adventures in the Plane of Water:}

The sound of drums has called you to the activities like moths to a flame. When it comes into view, it appears to be\ldots

\listone
    \item A brass sphere, with no immediately obvious entrances.
    \item An army of skeletons.
    \item The coral towers of a merfolk city; they look sick.
    \item An ice factory.
    \item Angry tritons.
    \item A giant eel that had been mimicking civilization sounds by slapping rocks together.
    \item A Sahuagin kelp outpost.
    \item A family of scrag wreckers.
    \item A Marid Sattrapi.
    \item Some sort of mechanical vessel shaped like a lobster. 
\end{list}

\subsubsection{Ten High Level Adventures in the Plane of Water:}

You've broken into the massive mechanical manta ship. Inside you find\ldots

\listone
	\item Spongy, organic passageways... this ship is alive.
    \item The crew are long dead and dust.
    \item The captain's log mentions you by name.
    \item Kuo-Toan pirates and their Yugoloth servants.
    \item Sack after sack of dream dust.
    \item These look like dragon eggs.
    \item The spectral pirates who run this thing.
    \item A cargo hold full of wild eyed prisoners.
    \item A cargo hold full of non aquatic and fearful prisoners.
    \item The ship's wizard captain and his crew of blood-indifferent golem pirates.
\end{list}


\subsection{High Adventure in\ldots The Negative Energy Plane!}

If you're even considering running a game in the Negative Energy Plane, it is very probable that you are using Playing With Fire morality for your necromancy. This is in large part because every writeup of the NEP ever made has assumed Playing With Fire, and that indeed it is precisely these descriptions that give people the best scriptural ammunition against Crawling Darkness. But also because if Negative Energy is inherently evil, the plane becomes incredibly boring. We already have the Gray Wastes of Gehenna, so there's no real point in having another gray desert made out of ultimate evil.

The game provides two supposedly different Negative Energy Planes for you to consider. One is made out of Major Negative Energy Dominant with patches that are Minor Negative Energy Dominant, and the other is made out of Minor Negative Energy Dominant with patches of Major Energy Dominant. Well, anyone who has ever looked at a splotchy cow knows that whether you have a black cow with white spots or a white cow with black spots is entirely a matter of perspective. Since the NEP is infinite, both Major and Minor patches are infinite in size and in scope, so it really makes no difference at all which one you are nominally using. From a practical standpoint, either way you're going to be in either a Major or Minor Negative Energy area, the adventure location you are going to next will either be in the same area or a different one, and if you go far enough in any direction you will go from one to the other. And anyway, both Minor and Major Negative Dominant ares are totally fatal to living creatures, and completely harmless to undead and constructs, and the baleful effects are completely negated by negative energy protection or attune plane. So seriously: who cares? Since the only actual difference is the unprotected living creatures crumble to ash in Minor Dominant and are transformed into wraiths in Major Dominant, our suggestion would be to go with Major Dominant most of the time. It's largely academic, because outside the planar bubbles there is no air (so without some sort of magical attunement, every living creature is just going to die of asphyxiation, negative energy or no).

The Negative Energy Plane hates life. It hates the good, and it hates the wicked both the same. It does not condone or aid harm or murder, it simply greedily and expeditiously extinguishes any life exposed to it. But if you're alive that's basically no worse than the vacuum of space, and if you're not alive it's a whole lot better. For those who are undead, non-living, or have the right kind of protections, the Negative Energy Plane is a lot like any other void plane of the D\&D cosmology save that there is no ambient light source. Comparisons can be made to Limbo, the Astral Plane, and of course: the Elemental Plane of Air. The difference is just the fact that it is unlit, and therefore looks like the night sky rather than extending out to a gray fog where the soft glow of the ambient light eventually wipes out anything you could see.

Once you factor in the Planar Bubbles (which as an ironic statement, are called ''doldrums" by Negative Energy inhabitants), the Negative Energy Plane is basically exactly the same as our universe. If you were on a prime bubble, you pretty much would only with difficulty be able to know that you weren't on a Prime. There's a dark hostile, airless void outside your planet, and there's absolutely nothing stopping any light source of any distance from eventually sending its ray to you. So the sky above you is black and full of tiny lights. Well, it wouldn't really be that difficult to figure it out, because absolutely everyone can fly just by thinking about it. And the lights in the sky are just like what ancient people thought about them: some of them are very large and far away (like Elemental Fire bubbles that function as stars), and others are more modest light sources that are more reasonable distances. The intrinsic flight includes not only hovering, but also acceleration that is only relativistically limited. You can accelerate at 1G or more by sheer willpower as long as you want without energy expenditure. So a trip from the Earth to Mars would take less than 5 days even at its most distant point (assuming that they were both on the Negative Material Plane). So titanic, even solar distances are quite reachable. Also of note is that the directions to Neverland (Third star on the left, and straight on 'til morning) are completely reasonable directions, and represent another planar bubble that is about 2 million kilometers away. Like all regions of subjective gravity, going ''towards" a point will automatically have you accelerate continuously to the halfway mark and then have acceleration away from it for the rest of the journey, so you never ram into anything at relativistic speed.

The distances between things in the Negative Elemental Plane are truly vast, but travel is so easy that from a practical standpoint, things in the Negative Energy Plane are actually kind of ''happening." The exception of course, is unlit structures. These are called ''Castles Perilous" by the locals, and making one is pretty much a declaration that you under no circumstances want visitors. After all, without giving off any light, you're basically about as findable as any rock out in deep space is in the real world. The only ways to find one are to happen to see them passing in front of a light source or to shoot one's self off into the void looking for the automatic deceleration that accompanies moving towards a real object - and even knowing that second one is an option requires the kind of math you'd need a Knowledge (Planes or Engineering) DC 25 test to do.

An important thing to consider is the presence of Voidstone. It's a special material that will destroy and absorb any creature (even undead creatures) if they come into contact with it for a few seconds. Truly badass creatures like dragons and gods might be able to hold it for a minute or two before being eradicated from existence, but as you might imagine, that stuff is still in huge demand for making into weaponry. Since it doesn't do anything to other inert elemental material like, say metal tools, it ends up being quite workable and incredibly valuable. Voidstone is planar currency for obvious reasons - but finding it is very difficult because it's not very large, pure black, and forms in the middle of large sections of empty void.

But perhaps the most important point about the Negative Energy Plane is that the parity with the Positive Energy Plane is not complete. Living creatures are natural, so they have no protection from being exposed to ''too much" positive energy - and they can totally explode. Undead creatures are unnatural and only exist at all because they are supported by magic to siphon off a specific and measured quantity of negative energy. So they don't ever ''explode" in Negative Dominant areas, whether they have ''protection" or not. As such, groups of intelligent undead often make homes out of Castles Perilous in the middle of strong Negative Energy Vortices. Because seriously: why not?

\subsubsection{Campaign Seed: Death World:} A Doldrum region in the Negative Energy Plane is a lot like Neverland if it was made by American McGee. Everyone can fly like Peter Pan, and each region fills up with weird crap from all over the planes like tribes of Indians, mermaids, and pirates. However, these places are also constantly under assault by a low level rain of zombies from space. That's not a joke, undead beasts literally float around in the void and choose to fall towards points of light. So if you're running around Pixie Hollow, there is a not insignificant chance that some undead monster is going to fall out of the sky and go on a rampage. This setup allows for very reasonably scaling D\&D adventuring. After all, if the PCs become masters of their surroundings and conquer the Maze of Regrets, you have a totally reasonable excuse to have a level appropriate undead army fall from space and tart causing havoc. In the meantime, even though the levels of Negative Energy aren't high enough to snuff the life out of anything, they are leaking into Doldrums enough to make things subtly creepy and unpleasant. Feel free to use any Ravenloft clichés you want. Or just American McGee it up - people live on a fricking Death World, so have just messed up stuff happen all the time. Have cats croak out ''help\ldots me\ldots" for no reason. Have thorns drip unexplained blood. Have trees inexplicably drain of color. Inhabitants go crazy and start eating pieces of themselves. Go nuts.

\subsubsection{Campaign Seed: Welcome to the Void Heart:} There is a city built into the inside of a one-mile diameter iron Dyson Sphere which is called ''Heart of the Void" or ''Deathheart" depending on who you ask. Some sages built a city there a long time ago and eventually an army of the undead broke in and murdered everyone. Tonight it's a minor necropolis that is broken up into factions that fight each other for domination. And I know what you're thinking: so what? I mean, that's only 3.14 square miles of city, and even though it has the population density of New York, it still only has 70,000 inhabitants, and a lot of them are ghouls. But the really important thing is what the sages used to do, which was to track all the objects in the Negative Energy Plane. All the rocks of Voidstone, all the Castles Perilous, everything. No one knows how they did it, because some vampiric minotaur killed the last of them a few hundred years back and feasted on her heart - but they did leave notes. All over the city, there are books filled with page after page of descriptions of the size, shape, and location of various objects in the void. There are a lot of adventures there: some books are useless without other books in the same series; some books are the possessions of hostile undead gangs that either do or do not know how valuable they are; and many books detail the locations of items and structures that are themselves interesting and valuable adventuring locales.

\subsubsection{Ten Low-Level Adventures in The Negative Energy Plane}

The ghoul chitters and licks his parched lips. Seemingly reluctant to proceed, he whispers\ldots

\listone
	\item ''You may have defeated me, but there are a dozen more on their way\ldots"
    \item ''Fellnax wants his coins. He wants them bad\ldots"
    \item ''You can kill me, I'll never tell you were the diadem is."
    \item ''I knew someone would find me. I didn't know who, but after the Hellmire job, I knew it was only a matter of time\ldots"
    \item ''These bones\ldots these bones are mine\ldots"
    \item ''You traitors! I'll feast on you!"
    \item ''Do you have the scrolls? My master said you would have the scrolls\ldots"
    \item ''You don't look like Fellnax's men."
    \item ''Fellnax sent me to tell you, to tell you that he is going to kill all of you\ldots"
    \item ''We still have the girl, please don't do anything we'd both regret."
\end{list}


\subsubsection{Ten Mid-Level Adventures in The Negative Energy Plane}

It's good to meet another outworlder. But there's something weird about this guy\ldots

\listone
    \item There are faint sobs coming from his backpack.
    \item He casts no reflection.
    \item Everytime he mentions the Castle Perilous he came from, he looks over his shoulder.
    \item There are the scars of bite marks all over his arm.
    \item When he talks about his family getting eaten, it's like he doesn't even care.
    \item When he mentions the golden statues of Kath, it's like he doesn't even care.
    \item He seems genuinely relieved to be here.
    \item He steps right over the ghoul corpses as if that was a normal thing.
    \item He has one of Fellnax's amulets. Or something that looks just like one\ldots
    \item There is a wraith following behind him, one that looks just like he does\ldots
\end{list}


\subsubsection{Ten High-Level Adventures in The Negative Energy Plane}

You've got a fix on the Voidstone you were looking for. Unfortunately it's\ldots

\listone
    \item Suspended inside the chest cavity of a dracolich.
    \item Worshiped by a death cult of Kuo Toa.
    \item Inside a Castle Perilous named ''Doom Watch"
    \item Been made into a sword by a mad Duergar.
    \item Guarded by a Void Shadow.
    \item Guarded by a Shadow Dragon
    \item The Tomb of a fallen god.
    \item Locked in Lethe Ice.
    \item On the far side of an Allip Belt
    \item In the workshop of a Master Skincrafter.
\end{list}

\section{To Rule in Hell\ldots}

Some of our favorite bad guys are the Arch Devils and the Demon Lords. More people can name Jubilex than can name Erythnul, and that's no accident. Jubilex is just a little bit more awesome than any of the official Chaotic Evil Gods. Unfortunately, like many things related to the Lower Planes, there is substantial discrepancy available as to what exactly it is that these Dark Lords \textit{are}. We suggest that you make up your mind and distribute this decision to your player characters:

\subsection{Ascended to Godhood}

In this option, demon princes are gods, which means that they have all the rights, privileges, and pitfalls that status implies. Killing a god is, under this model, the same as trying to kill something as intrinsic to the universe as something like "water" or "love". This doesn't mean that you can't fight these guys: they have worshipers that can be killed, temples that can be burned, and avatars that can be trapped under mountains. Since avatars are like deities you can stab (exactly like that in fact), it doesn't really matter if Mephistopheles is an unkillable ideal that serves as a focus for the prayers and power of his worshippers since you can kick his avatar's can and loot his palace. The real question is this: are demon princes part of your pantheon, or are they merely aspects of existing gods?

If demon princes are gods in their own right, I can't imagine Nerull sitting comfortably next to Yeenoghu at the divine dining table, as those two have a certain amount of overlap in their portfolios (and no one likes a copy-cat). This could lead to interesting adventures as the more accepted evil deities compete for cosmic real estate with the demon princes leading to worshipper on worshipper violence and games of deception where good organizations are manipulated into fighting the followers of the other god.

The same kinds of conflicts could occur if the Demon Princes are just hardcore aspects of more accepted evil gods. Heresy inside a faith can be ripe opportunity for adventuring, as wars have destroyed entire continents over this kind of thing. If Orcus is just an aspect of Wee Jas, you could have the fuel for a truly righteous schism.

\subsubsection{Slaying the Gods}

Let's face it: in the original writeup, Lolth had \textbf{66 hit points}. Not an avatar or anything, the dark goddess of the Drow had less hit points than a 7th level Fighter who had just rolled well (everything had less hit points back then, and 7th level was more impressive, but you get the idea). D\&D has a long history of characters stabbing gods directly in the face. Nevertheless, the gods presented in Deities and Demigods are extremely unsatisfying for that purpose (for one thing, they aren't even epic characters, and for another thing they arbitrarily have infinite power in weird, poorly defined ways and that makes any rules adjudication into a game of Cops and Robbers). Orcus, on the other hand, can be knocked down and beaten like a King until he stops moving and a good time is had by all. By making the Dark Lords into "real" gods and keeping their essential nature as stabbable entities, you can achieve some classic D\&D moments that are largely missing otherwise from 3rd edition.

\subsection{President of the Corporation}
\vspace*{-8pt}
\quot{"Orcus is dead! Long live the new Orcus!"}

We all know Orcus, he's one of our favorite guys. He's been portrayed as a fat bat-winged dude with a skull-stick, and as a gaunt skeletal guy with a sword. But somehow he's always Orcus, the Demon Lord of the Undead. In the CEO model of Arch Fiends, Orcus is a \textit{title}. Once someone takes out the current Orcus, then that opens up the possibility for another person or monster to become Orcus. Being Orcus gets you Orcus' desk, the accumulated debts and assets of the Orcus estate, and the authority to molest members of Orcus' numerous cults across the planes. It also gets you a posse of paladins from distant planes who have already dedicated themselves to your destruction.

\subsection{Powerful Adventurers}

Alternately of course, the Demon Lords and Arch Devils can just be adventurers. Like the Player Characters, they've been the protagonists of their own stories and they aren't going to be permanently killed by anything. No matter what takes them down, they'll definitely come back. In this model, Yeenoghu is "just" an Epic Character, and that means that the only reason you know his name is that he's just a little bit cooler than Valishar Goldeneyes or any of the other epic characters that D\&D gets cluttered up with.

In this model, the archfiends can be defeated by powerful adventurers. But like the powerful adventurers that they face off against, they have the interest of other powerful party members and assistants who will faithfully cast powerful magic or go on dangerous quests to bring them back from death. D\&D characters of sufficient stature can be \textit{defeated}, but permanently killing them is generally not on the table. If enough people know your name, eventually someone is going to \spell{wish} you back.


\section{Rulership of the Lower Planes}

Player characters can gain Influence and Rule in the Lower Planes just as they can in the Prime Material plane. In fact, it's a little \textit{easier} to assume rule on the planes because the planes themselves are conditioned to accept themselves as the property of powerful beings. The Prime Material is just a bunch of rocks in space, but Carceri is \textit{divinely morphic}. This means that to a certain extent, belief shapes the planes in a way that the prime has no real equivalent to. With enough patience and sufficiently good propaganda you can change the weather, populate your region with unlikely creatures, and encourage the region to conform to your expectations in a myriad of subtle ways. Being the unquestioned ruler of a region of any lower plane increases your Influence by +1 per year, in addition to any other benefits you might accrue. Long established archfiends, therefore, have tremendous effective Leadership scores.

Rulership of an infinite plane is, of course, impossible. Even the gods themselves have boundaries to their personal dominions. The realms of the deities are tremendous in scope, but in the context of the plane they reside within they are grains of sand in the wind. The dominions that can be carved out by lesser beings are smaller still, though again they can extend across regions that in any more familiar world would be called vast.

Most of the evil planes are themselves Evilly aligned. That means that any region you happen to take control of is essentially like a huge Film Noir movie. Evil people are just smarter, more perceptive, and \textit{cooler} in these places than Good creatures are. Strongly evil regions are even worse: even neutral creatures are penalized (so badly in fact, that animals brought into Hades just curl up and die because the plane itself is dissing them so badly).

\subsection{Location, Location, Location}

So you're looking at some Infernal real estate, and you \textit{don't} want some Gray Waste that's 8 days hike to a planar oasis, and you don't want something across the street from a Nalfeshnee Spawning Pit. But what \textit{do} you want? The Lower Planes have a lot of locations that are financially, strategically, magically, or socially important -- in addition to those places that have a property value somewhere between "no" and "\textit{Hell} no!"

\subsubsection{Finality}

Finality is a planar metropolis in the Infernal Battlefield of Acheron near a minor tributary of the River Styx. Its harsh laws are kept in rigid and uncompromising order by the will of powerful pit fiends, and the city serves as a marketplace for the lucrative trade in souls. Magic items can be bought or sold here, but the currency is always souls (as a planar metropolis, Finality has a gp limit of 600,000 gp). Souls are valued at their CR squared, multiplied by 100 gp. Many items purchased from this location radiate evil, buyer beware. Lodging may be purchased at flat rate of one soul per day per person. The section of Acheron that Finality rests in has the Timeless trait and is mildly Lawfully aligned. The population of The City is about 100,000 people (with uncounted millions of souls), most of whom are Baatezu.

The rules in Finality are uncompromising and bizarre, and the punishments for breaking them are vindictively carried out to the letter by powerful devils. But there is no warfare allowed in the city, and even Celestials and Tanar'ri come to participate in the great mercantile dance of soul collection. Characters must make a Knowledge (Local: Finality) check everyday with a DC of 10 + 2 per day they've been in the city or unknowingly break one of the city's many inscrutable laws (knowingly breaking the law by starting fights or stealing goods is a whole different thing). Punishments range from perplexing to fatal. Characters who stay away from Finality for more than a month are no longer subject to the baroque residency rules and their DCs are returned to 10 the next time they visit.

The city itself is a collection of gothic stone architecture twisted by infernal and magical power into fantastic and improbable spires and towers and nestled amid squat merchant houses, the very stones tainted by the trade that has given birth to the city. Gray stone is unknown, as the city is build from exotic dark-colored stones dragged from dozens of lower planes, and it is amusing to note that the city itself is not built of souls, unlike several other planar locales. Undying fires light the orderly and aesthetically similar city streets, and unsleeping outsiders conduct traffic and trade continuously throughout the day. Food and drink is only available at exorbitant rates, as few residents have a need for such unseemly mortal concerns.

As a city based on a trade practice that is undeniably evil, one would think that heroes and celestials assault the city on a daily basis; but nothing could be further from the truth. Two factors keep the city free from harassment: Finality serves as a way to recover powerful souls of heroes and angels, and the city itself has a glut of powerful spellcasters whose stock and trade requires the existence of the city. Even the most idealistic champion of good knows that to destroy the city is to spread the soul trade far and wide across the planes so that greater depravities would be necessary for these merchants to stay in business. As long as they are bound by the laws of the pit fiends ruling the city, the worst excesses of the soul trade are curbed and bargains can be made with the most untrustworthy fiend for the recovery of a valuable soul.

Though the city is based in a plane with more than a few gods, these powerful entities do not attempt to sack the city for its riches in souls and magic. It is rumored that far back in the city's past, a moderately powerful deity attempted such a feat (or perhaps was trying to recover some follower's souls), and that the spellcasters of the city banded together to capture this bold immortal's soul and bind it into Finality, and have been slowly burning this being's undying essence to fuel their own magics. Whether this story is true or not remains to be seen, but divine spellcasters who enter the city can feel a terrible sucking sensation emanating from the very ground beneath their feet.

\subsubsection{Soul Veins}

No one, not even in Baator, tries to keep an accurate census of the residents of the lower planes. That's because new residents are popping up all the time. People who were particularly awesome show up in death pretty much as they did in life. They get special attention from the gods who punish or reward them according to how closely their lives matched the expectations of the gods in charge of judging them (which means that most high-level adventurers end up living the sweet life in whatever outer plane they end up in). Furthermore, these guys are often whisked back to life by powerful magics that have no time cap at all. Famous heroes can be brought back to a living state hundreds of years after they die, and retain their sense of self continuously throughout. But people with small, unimportant lives get a much worse deal in the D\&D afterlife. The forgotten multitudes of the lower level mooks and farmers of the worlds get overlooked by the great judging. Their souls are used as building materials -- even in the "Good" planes (where, for example, Celestials take the souls of those not deemed important enough to warrant special attention to power street lights and provide illumination for cafes).

Souls of the forgotten are used to create outsiders. Every Hound Archon or Glabrezu you see was at one time one of the forgotten. But others are used to power magic, or to make equipment, furniture, or even walls. Souls are a valuable resource, and sight unseen sell for 100 gp (more interesting or valuable souls can sell for much more depending upon who's buying). The souls of the forgotten bubble up all over the place on the lower planes, and many of those places are pretty inhospitable or even \textit{under ground}. When the souls of the dead appear in a confined space they are crushed into a taffy-like substance called Soulstone. Veins of Soulstone are all over the lower planes, though ones near planar oases, fortresses, and other settlements are generally already tapped to one degree or another. A single soul's worth of Soulstone weighs 1/50th of a pound.

\subsubsection{Portals}

Portals connect every plane, and many of the portals on the Lower Planes are in areas sufficiently dangerous, that few creatures know of them. Still, so many of the planar denizens have the ability to use \spell{greater teleport} with a limit of 50 pounds of carried items that items of less than 50 pounds from all over Pandemonium can be found for sale in The Mad House at quite reasonable prices. The money in shipment isn't in moving small or fungible things within a plane (any Glabrezu can transport a tonne of rice to anywhere in 9 minutes), the money comes from transporting things \textit{between planes} or transporting objects that weigh \textit{over fifty pounds}. The mark-up there is \textit{intense}, and beings of the Lower Planes are willing to accept price gouging on interplanar and high-mass transport because they understand that the normal "teleport tag" model of goods transference doesn't work for those kinds of transports. Controlling a portal from anywhere in a Lower Plane to anywhere in any other Plane of existence can get a prospective merchant lord the benefits of being a Monopolist (+5 to Profit checks), but only if the portal is opened up to commercial use.

The easiest way to open up a portal to commercial use is to get the word out that you control a Portal and are willing to allow fiends to use it for a fee. That works, as there is enough rapid transport available on the serial teleportation circuit that goods will make their way to your portal as soon as the existence of your Portal becomes common knowledge amongst merchant fiends. There is, after all, a market for \textit{something} in every plane. You may not want to \textit{look} in the baskets that the Gelugons are hauling, but they'll pay in gold or souls, so the money is right. Unfortunately, having your Portal be well known is its own punishment -- fiends have a marked tendency towards greed, so if your Portal becomes profitable enough you may have to contend with hordes from the hells coming to take your stuff and not give it back.

With more difficulty, one could attempt to find and manipulate markets on both sides of the portal yourself. The rewards of doing so are even greater (no middle-men means more profits, a +10 bonus to Profit checks in fact). But you'll have to find a source of goods or services, a demand for those services, and transport those goods or services yourself. And while your operation is initially not under scrutiny, if enemies find out about it they'll be even more interested in knocking you over to take it away.

Portals that go between centers of economic activity can be valuable even if they are on the same plane of existence. Moving mid-sized and large objects around the planes is extremely difficult because there's generally no infrastructure for it. The very ease of moving bricks one at a time across the Wastes of Hades has led there to be almost nothing in the way of \textit{carts} or \textit{roads}. Transporting even a mid-sized stone is all but impossible across any meaningful distance. As a result, if a portal is capable of moving heavier objects and connects two places that host reasonable amounts of economic activity on both sides, the controller of that Portal is considered a monopolist (+5 to Profit checks). Portals connecting planar metropolises on different planes are even more valuable, and provide the bonus (and the potential hostile interest) of both a Portal between planes \textit{and} a Portal between centers of economic activity. And yes, you \textit{can} get the bonuses for being a monopolist twice.

\subsubsection{Planar Oases}

The planar traits on many of the planes (especially Hades) are\ldots not good. But there are places on any plane that lack those planar traits or have the planar traits of other planes. Those places are \textit{extremely valuable}, as they are pretty much the only place that most planar denizens can live, work, or play on many of the planes. Planar oases in places like Pandemonium or Acheron are fairly valuable because being Lawfully aligned in Pandemonium is unpleasant, but a Planar Oasis in the Gray Wastes is \textit{extremely} valuable because life in the Wastes is almost impossible for most extraplanar creatures. Owning a stronghold in a planar oasis draws planar denizens to your banner, causing your Influence modifier to increase (PoF).

\listone
	\item Every month that you hold a Planar Oasis on Pandemonium, Acheron, or Carceri increases your Influence by +1.
	\item Every month that you hold a Planar Oasis on Baator, Gehenna, or the Abyss increases your Influence by +2.
	\item Every month that you hold a Planar Oasis in the Gray Wastes of Hades increases your Influence by +5.
\end{list}

Note that holding a planar oasis isn't easy. Fiends and even Celestials from all over the planes will come to take your stuff for use as a military beachhead or planar resort. Although the bonuses to your influence are cumulative for holding a Planar Oasis for a long time, you lose them all if someone else takes your control away. When you take control of an oasis from another creature you can either allow the current tenants to stay (keeping the entire Influence onus of the previous owner for yourself and inheriting whatever problems the previous owner had allowed in), or attempting to clear out the old residents and start over (resetting the Influence bonus to zero as soon as you're done, but allowing you to do things "right").

\subsection{Wondrous Architecture of the Lower Planes}

The planes are well known for fantastic locations, and the fiendish constructions of the Lower Planes are no exception: demon cities of unusual construction compete with the infernal strongholds of powerful fiends in both grandeur and designed atrocities. Magic often goes into the construction of these locations so that these places become conduits for the energies of the Lower Planes. Such places must be built from scratch to create these effects; no existing city can be modified to gain this wondrous architecture.

Any Prerequisites for these kinds of cities are flavor considerations. It is a DM's option to allow such a city to be built, and he will determine any costs or prerequisites needed to create such a place. This is not because we are trying to keep these effects out of the hands of the player characters -- quite the opposite. In fact, it is because we \textit{want} players to use these effects now and again that we make them uncosted. Within the context of the D\&D metaeconomy (where one is specifically allowed to \textit{purchase} epic magic items with tonnes of gold), there is just no possible fair cost for an entire city covered with magical effects. That kind of thing is really awesome flavor-wise, but giving it a cost unfortunately leaves it transferable into magical equipment that can destabilize the game. See the Book of Gears in this document for information on getting around this issue.

\subsubsection{Necromantic City}

Built with materials associated with necromancy and populated by the undead, these cities have features like obsidian walls buildings, bone dust in the street instead of dirt, and images of death on every surface. Pale ghost light illuminates the streets and the living slowly die within its walls.

\ability{Prerequisites:}{Over 75\% of the population must be undead.}

\ability{Effects:}{All Necromancy effects are at a +4 caster level, and every day spent in the city inflicts a negative level (this heals undead of 5 HPs).}

\subsubsection{Serpentine Labyrinth}

The streets of this city are twisting mazes and the angles formed by buildings and walkways are designed to confuse and inspire disorientation.

\ability{Prerequisites:}{City must have been designed by someone with at least 30 ranks in Knowledge(architecture).}

\ability{Effects:}{Any non-native entering this city halves his movement rate while in the city.}

\subsubsection{Redstone City}

The stones of the city absorb blood, and the city itself inspires violence and hatred.

\ability{Prerequisites:}{The city must be build from stone pulled from sites of great violence.}

\ability{Effects:}{All starting reactions of NPCs are one level more hostile.}

\subsubsection{Spired}

Magic has been used in the construction of the city to enable tall spires of surpassing delicacy. Only the most agile fliers can enter such a city, but defenders can fire down upon invaders with surprising ease.

\ability{Prerequisites:}{City must have been designed by someone with at least 10 ranks in Knowledge(architecture) and the ability to cast 5th level effects.}

\ability{Effects:}{Only flyers with a Maneuverability Rating of Good or better can fly in the city. Attackers in the city suffer from archer attacks every 10 minutes as natives exploit the unique construction of their city.}

\subsubsection{Basalt}

Magically heated stone forms the basis of every building in this place, and the weather in this city is always equal to high summer in the dessert.

\ability{Prerequisites:}{These cities must be built by natives of Fire-aligned planes or with the ability to cast 5th level effects.}

\ability{Effects:}{The weather in the area is always at least as hot as "Severe Heat". Unearthly and burning heat happen wih surprising regularity.}

\subsubsection{Plague Town}

A Plague Town is known for its poor sanitation, lax attitude about corpse and garbage collection, and vile culinary practices. As a result, living within such a place means that disease is a constant companion, and few visitors travel to such a hellhole.

\ability{Prerequisites:}{Such cities are always in remote locations.}

\ability{Effects:}{Non-natives must make a DC 15 Fort save for every day spent in the city or else catch a random disease. Each native is a carrier of 1d4 diseases, but is immune to their effects.}

\subsubsection{Forgotten}

Some locales in mist covered mountains and secludeed valleys seem to slip from mortal memory, and travelers can seldom find these places after they have left them.

\ability{Prerequisites:}{Such cities are always in remote locations.}

\ability{Effects:}{This city is invisible to anyone not within 100' of it, and pathways leading to it are concealed by illusions to appear impassable. Any non-native leaving the city must make a DC 15 Will save or forget which pathways lead to its location and details of its interior (meaning that you cannot use teleportation or travel magic to return). Such a city can be found again if the general area is searched again (such as the entire plain, mountain range, swamp, or ocean).}

\subsubsection{Forsaken}

Forsaken cities are cursed and empty, the sites of great betrayals or massacres. Only the strong-willed can enter and remain in such a place.

\ability{Prerequisites:}{}

\ability{Effects:}{When entering the city, and every day afterwards, a DC 15 Will Save must be made or else the subject cannot bring themselves to willingly enter the city for a month.}

\subsubsection{Rune-built}

Streets and buildings in this city form runes when viewed from a great distance, granting the effects of a spell upon the city or its people. Some notable examples of this kind of city include:

\listone
	\bolditem{The Palace of the Maskers:}{A city known for secret meetings and negotiations where every person in this city is affected by \spell{alter self}.}
	\bolditem{Wide Sky:}{A floating city on the Elemental Plane of Air where everyone can \spell{fly}.}
	\bolditem{The Free Nation:}{A town in Limbo protected by a \spell{magic circle against law}.}
	\bolditem{The City of Secret Things:}{A major trading post in the Astral plane where \spell{obscure object} has been cast on every object.}
\end{list}

\ability{Prerequisites:}{}

\ability{Effects:}{A spell of up to 3rd level can affect every person, object, or area in the city, and this effect cannot be dispelled. This effect does not last beyond the borders of the city.}

\subsubsection{Hungry}

Some cities seem to have a life of their own, and they consume the weak and the foolish. People entering such a place vanish without a trace when they leave sight of their friends, and only the strong last long in such a place. Such a city might protect group of predators with magic effects, might have a high crime rate due to magically enforced disrespect for laws, or it might simply animate buildings or statues and devour the unwary; no two Hungry cities are the same. Such cities are noted for having small police forces as troublemakers are either protected or devoured by such places.

\ability{Prerequisites:}{Varies.}

\ability{Effects:}{For every day spent within this city, an individual must defeat a CR 5 encounter. Should an individual be defeated by this encounter, his body vanishes. Natives of this city are immune to its effects.}

\subsubsection{Magic Dead}

Some beings only trust the power of muscle and steel, and have carefully crafted their city in order to scour clean the flow of magic. Such a city is note for being well constructed and sturdy and its people hard-working, but unimaginative.

\ability{Prerequisites:}{Such a city is never ruled by a magicracy or race with racial spell-like or supernatural abilities and never sells magic items in its shops or markets.}

\ability{Effects:}{Every area of this city is inside an \spell{antimagic field}. \spell{Scrying} cannot pierce this place and travel magic cannot bring one closer than the gates of this city.}

\subsubsection{Wicked}

Terrible acts are performed in this place, and evidence to this fact is written in both the construction of the city and faces of the natives. Vile statuary and murals cover every available surface, and natives of this unholy place do little to hide their depraved desires or acts.

\ability{Prerequisites:}{City must be in a planar area that is evil aligned.}

\ability{Effects:}{When entering this city, and every day afterward, make a Will Save (DC 15). On a failed save, the victim is Shaken for the duration of his stay.}

\subsubsection{Infernal Fortress}

This city is designed to hinder the spell-like abilities of attacking fiends with features like shifting geography to foil \spell{greater teleport}, runes that block summonings, and mists that negate the advantage of being able to see in darkness.

\ability{Prerequisites:}{City must be designed by someone with knowledge in Architecture, History, and the Planes of at least 10 ranks.}

\ability{Effects:}{Other than the formidable defenses of the metropolis, there are no effects.}

\subsection{Business as Usual}

There is profit to be made on the lower planes for the unethical, and that means that almost everyone has a scheme to get rich quick or swindle the other man. In the lower planes, as everywhere else in the multiverse, everyone thinks that they're smarter than average, have a good sense of humor, and are good in bed. If someone in Gehenna tells you that they don't have a scheme, \textit{that's part of their scheme}.

\subsubsection{Orchards of Larvae}

Larvae appear whenever a particularly evil creature dies on the Prime Material, and none of the gods care enough to do anything specific with them. They appear all over the Lower Planes, but they appear in some places more than others. Those areas where larvae appear with more frequency are called \textit{orchards}, and those of them as have been discovered are generally heavily built out. An exception is those that are in the Gray Wastes of Hades, which are at best occasionally looted by Night Hags or Blood War Soldiers. Larvae burrow themselves into the ground and huddle in the dirt soaking up evil until they metamorphose into a Fiend or are eaten by infernal wildlife.

Gaining control of a Larvae Orchard is like gaining a business, save that its relative location isn't important. Even large Larvae are less than fifty pounds and quite portable by \spell{teleportation}. A Larvae Orchard, thus, has a +2 Profit Modifier regardless of whether it is in a Planar Metropolis or the middle of a hoary wilderness. Larvae sold on the open market are used for everything from a luxury food to a source of powerful servants.

\business{Larvae Orchard}{Search}{Handle Animal, Sense Motive}{Special (4,000 gp)}{Medium}{High}

Running a Larvae Orchard is a highly eventful proceeding. Every season, roll a d6: on a 1-3 subtract 10 from your Business Events roll each month, on a 4-6 add 5 to your Business Events roll each month. These modifiers are replaced each season.

\subsubsection{Pain Stills}

Liquid Pain can be harvested from any sentient creature tortured to near death over a long period of time. It is also a powerfully addictive drug and a source of intense magical power. The creation of Liquid Pain is quite Evil, but that in no way discourages anyone in the Lower Planes. Liquid Pain can be used to create magic items or empower spells, and no one even notices that it turns items created with it [Evil] because the environment is doing that anyway.

To make a Pain Still, one merely needs a relatively stable area to keep a tremendously expensive alchemical apparatus and a dungeon full of prisoners with a diverse assortment of torturing gear to agonize them. Some Pain Stills torture victims to near death and allow them to go, others simply kill victims who are no longer capable of being juiced.

\business{Pain Still}{Craft (Alchemy)}{Heal, Profession (Torturer)}{Low}{High}{High}

\subsubsection{Zombie Factories}

Fiends abound who are able to create undead with their spell-like abilities, and this allows them to create undead without using valuable Onyx. In a zombie factory, these powers are used as part of a service to create uncontrolled undead by the score. These services are often employed by wizards, clerics, and dread necromancers who have the ability to control uncontrolled unintelligent undead by any of a number of means.

Corpses and skeletons are brought to the zombie factory, where they are modified with the addition of metal plates and the like, and then animated with fiendish powers. It is then up to the customer to take command of their new toys and take them away. High priced zombie factories exist that procure specific requested bodies to be animated.

\business{Zombie Factory}{Craft (Armorer)}{Knowledge (Nature), Appraise}{Medium}{Low}{High}

\subsubsection{Fossil Storage}

Powers exist in the lower planes that can petrify creatures, leaving them as calcified statues sleeping away the eons in a blanket of stone. With the known relative ease of raising the dead, the ability to remove an opponent without literally killing them is in high demand.

Some creatures even hire these services, not as jailers, but as hiding places. Keeping themselves in storage out of the reach of vengeful arms long enough to be forgotten. The Petrification Guild guarantees that you'll be revived at the appropriate time specified in your contract.

A Petrifying Prison is different from ordinary businesses in that it is in all ways advantageous for it to be far away from any civilization. The profit check for a Petrifying Prison gains bonuses that are inverted for its location: the Wilderness grants a Profit Modifier of +10, Rural +4, Town +2, City +0, Metropolis -4, Planar Metropolis -10.

\business{Petrifying Storage}{Knowledge (Dungeoneering)}{Knowledge (Architecture), Listen}{Low}{High}{Medium}

\subsubsection{Arbitrage and Skullduggery}

Of course, lots of mortal businesses have counterparts in the lower planes. Mercenaries, Ferrymen, Merchants, and Farmers all exist aplenty in the dark planes of existence. A business run on the planes functions just like the normal businesses listed in the DMG2, save that the Risk is increased by one level (to a maximum of High).


\end{document}