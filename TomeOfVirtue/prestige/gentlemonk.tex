\classname{Gentle Monk} \label{comm:prestige:gentlemonk}
\vspace*{-8pt}
\quot{``He hit me five times, one hand after another stabbing at my shoulders, my stomach, my chest. It hurt, but not...physically. And then, I felt the tingle of magic in my body, but, rather than the warm strength it normally is, it was an angry fizz or buzz. He looked at me with his pale eyes, and said that my magic would return with rest, that I should stay out of the rest of the battle, and then ran off to go help his ally in the heavy armor.''}

\desc{The Gentle Monks are a feared order. The irony of their name freezes the spines of martial artists and spellcasters who know of them. You see, Gentle Monks do not fight by breaking bones or bruising muscle, but by attacking the key points of the body itself. Their Gentle Strike style involves injecting their body's energy with each precise strike, disrupting the life force of their opponent. This technique is very flexible, letting them choose whether to slay or not. Furthermore, creatures with spellcasting or natural magic are susceptible to this technique, as it can be adapted to shut off their access to the magic forces -- temporarily. }

\ability{Prerequisites:}{}
\listprereq
\itemability{BAB:}{+6}
\itemability{Skills:}{Concentration 9 ranks, Heal 9 ranks}
\itemability{Feats:}{Two-Weapon Fighting, Insightful Strike}
\itemability{Special:}{Fatal Strike, Armored in Life, Must have a Fighting Style which inflicts Constitution Damage.}
\end{list}\vspace*{8pt}

\ability{Hit Die:}{d8}

\ability{Class Skills:}{The Gentle Monk's class skills (and the key ability for each skill) are Bluff (Cha), Balance (Dex), Climb (Str), Concentration (Con), Diplomacy (Cha), Heal (Wis), Hide (Dex), Intimidate (Cha), Jump (Str), Knowledge (all) (Int), Listen (Wis), Move Silently (Dex), Sense Motive (Wis), Spot (Wis), and Swim (Str)}

\ability{Skills/Level:}{4 + Intelligence Bonus}


\begin{table}[tbh]
\begin{small}
\begin{tabular}{lp{1.9cm}p{0.7cm}p{0.7cm}p{0.7cm}p{7cm}l}
Level&Base Attack Bonus&Fort Save&Ref Save&Will Save&Special\\
1&+1&+2&+2&+2&Monk Training, The White Eye, Gentle Strike\\
2&+2&+3&+3&+3&Chakra Disruption\\
3&+3&+3&+3&+3&Heavenly Sphere\\
4&+4&+4&+4&+4&Chakra Shock\\
5&+5&+4&+4&+4&Closing The Soul\\
\end{tabular}
\end{small}
\end{table}

\smallskip\noindent All of the following are Class Features of the Gentle Monk class.

\ability{Weapon and Armor Proficiency:}{A Gentle Monk gains proficiency with no new weapons or armor.}

\ability{Monk Training:}{The Gentle Monk continues honing his skills at evading attacks. Levels of Gentle Monk stack with levels of Monk (DnD Class) for determining the AC value of Armored in Life. }

\ability{The White Eye (Ex):}{After much meditation and concentration, the Gentle Monk has earned the ability to see the very flows of energy that surround him -- literally. When this is activated (a standard action, as is deactivating), he gains the following benefits:he benefits of this are manyfold:}
\begin{itemize}
    \item{He gains Lifesight with a range of 30 feet per character level.}
    \item{He gains 360-degree vision when the White Eye is activated, preventing him from being flanked.}
    \item{He gets Detect Magic within the range of Lifesight.}
\end{itemize}

\ability{Gentle Strike (Su):}{The Gentle Monk does not crudely beat his opponents to death -- his attacks target the inner workings of their bodies.

First, when using his Constitution-reducing Fighting Style, his Slam or unarmed attacks may, at his option, reduce Strength or Dexterity instead.

Second, he may forego damage to inflict a 10\% magic failure chance per successful hit. This affects both arcane and divine spellcasting, and also interferes with the activation of spell-like and supernatural abilities.

Gentle Strike may only be used when White Eye is active, and does not affect creatures with no Constitution score.}

\ability{Chakra Disruption (Su):}{At 2nd level, the Gentle Monk learns how to create discord in the ki of his opponent's body. As a standard action, he may make a touch attack which nauseates or exhausts his opponent (his choice) if they fail a Fortitude save (DC 10 + \half character level + Wis bonus). If the target passes the Fortitude save, they are instead sickened or fatigued (as appropriate). The condition lasts for a number of rounds equal to the Gentle Monk's Wisdom modifier. Chakra Disruption may only be used while the White Eye is active.}

\ability{Heavenly Sphere (Su):}{By emitting his ki from all over his body and spinning rapidly, the Gentle Monk may create a 10-foot radius sphere, centered on him, as a swift or immediate action.

Anyone (apart from the Gentle Monk) within the radius takes Xd8 Force damage (X equal to Wisdom modifier), and is pushed out to the edge of the sphere. This is useful when surrounded, but the more important function is that the force of the spinning ki disrupts magical effects which come in contact with it.

It stops any targeted or area-of-effect spell from affecting the Gentle Monk, and blocks Line effects. He may do this a number of times a day equal to his Wisdom Modifier. Because of the strain involved in doing it, it only lasts for an instant -- long enough to negate one spell or attack.}

\ability{Chakra Shock (Su):}{At 4th level, the Gentle Monk can prepare for a hit with a readied action. Any creature which strikes him in melee is affected as if they had been hit by by the Gentle Strike or the Chakra Disruption effect.}

\ability{Closing the Soul (Su):}{If the Gentle Monk performs a full attack with the Gentle Strike style and successfully hits at least four times, the target must make three Fortitude Saves (usual DC), to avoid losing spellcasting, spell-like abilities, and supernatural abilities, respectively. A day's rest will restore these abilities.}
