\chapter{The Morality of Necromancy: Black and Gray}

The rules of D\&D attempt to be all things to all people, and unfortunately that just isn't possible if you're trying to make a system of objective morality. By trying to cater to two very different play styles as regards to the moral quandaries of the use of negative energy, the game ends up catering to neither -- and this has been the cause of a great many arguments for which there actually are no possible resolutions. Ultimately therefore, it falls to every DM to determine whether in their game the powers of Necromancy are inherently evil, or merely extremely dangerous. That's a choice which must be made, and has far reaching implications throughout the game. That's an awful lot of work, and most DMs honestly just don't care enough to be bothered with it, and I understand. Fortunately, we have collated those changes for you right here:

\section{Moral Option 1: The Crawling Darkness}

Many DMs will choose to have Negative Energy in general, and undead in particular, be inherently Evil. So much so that we can capitalize it: Evil. And say it again for emphasis: Evil. That means that when you cast a negative energy wave you are physically unleashing Evil onto the world. When you animate a corpse, you are creating a being whose singular purpose is to make moral choices which are objectionable on every level.

That's a big commitment. It means that anyone using Inflict Wounds is an awful person, at least while they are doing it. The Plane of Negative Energy is in this model the source of all Evil, more so than the Abyss or Hell. It's Evil without an opinion, immorality in its purest most undiluted form.

\section{Moral Option 2: Playing with Fire}

Many DMs will choose to have Negative Energy be a base physical property of the magical universe that the D\&D characters live in -- like extremes of Cold or Fire it is inimical to life, and it is ultimately no more mysterious than that. An animate skeleton is more disgusting and frightening to the average man than is a stone golem, but it's actually a less despicable act in the grand scheme of things because a golem requires the enslavement of an elemental spirit and a skeleton has no spirit at all.

The Plane of Negative Energy in this model is precisely the same as all the other elemental planes: a dangerous environment that an unprotected human has no business going to.

\subsection{Implications}

It's not actually enough to simply make a sweeping generalization about the morality of Negative Energy and leave it at that. Like a butterfly flapping its wings, such changes will eventually cause Godzilla to destroy Tokyo. Or something like that, I stopped math at Calculus.

\section{Creatures}

Some monsters have been written up with the (incorrect) assumption that either ``The Crawling Darkness'' or ``Playing With Fire'' was the general rule. Others have been written in such a fashion that is actually incompatible with any possible interpretation of morality in D\&D.

\subsection{Revenants} If Negative Energy is inherently Evil, Revenants are Lawful Evil. They are undead who live only to kill and survive on hatred and the desire for vengeance. While they are victims and their actions are understandable, the Justice of their actions makes them Lawful, but they are still Evil and can be treated accordingly.

With the Playing with Fire option, there is no change to the Revenant. All is fair in avenging your own death, and they are the unliving emissaries of the Balance in its pure form.

\subsection{Skeletons} If Negative Energy is inherently Evil, Skeletons must be as well. That means that they actually do Evil things. An uncontrolled skeleton will find the nearest source of life and start ripping it to pieces. A skeleton does not need to be commanded to attack, but to stop tearing up your vegetable garden (assuming even that it had not already found a more vigorous source of life such as the family dog). A commanded skeleton is a vicious, unthinking killer on a chain -- not an inert construct awaiting commands.

If Negative Energy isn't Evil by itself, neither are skeletons. As described they aren't moral agents. That means that they don't have an alignment other than Neutral. Like a viper or a scorpion, though they do things that a paladin wouldn't necessarily condone (such as use poison for the snake or move around after death for the skeleton), they aren't gifted with the ability to make moral choices and default to the same Neutrality of the animated cabinet. Ordering a skeleton around could be Good, Evil, or Neutral depending on whether you are telling it to save children from a burning house, throw bloated corpses into the town well, or just carry your swag out of your basement.

\subsection{Vampires} Vampires are the rockstars of the undead world, but also the most affected by the gulf between Playing With Fire and Crawling Darkness Necromancy. Either vampires are tragically cursed Euro-trash with nice outfits or they are blood hungry princes of death \ldots\ heck, sometimes they are depicted as both, as in the case of the patron saint of D\&D vampires, Strahd Von Zarovich.

Unlike most undead, vampires are morally affected by negative energy in a perversely contrary fashion; Zombies are evil if (and only if) negative energy makes zombies evil, but the opposite is true of the vampire. If Negative energy is a hungry and malevolent force that hungers for the light of the living, the vampire is a tragic figure compelled by dark desires he cannot control. He can even just be Good, but that's not going to stop him from taking a nip from the farmer's daughter. If negative energy is an objective force, then being a vampire is actually an evil act since you don't have to eat babies for eternal life \ldots\  you're just a jerk.

\subsection{Zombies} Like Skeletons, Zombies must hunger for the flesh of the living or have no moral indictments. Either they sit and wait for their chance to devour your liver or they are Neutral. The Monster Manual version cannot stand. A zombie in the fields is either a figure of horror or comedy.

\subsection{Energons} Eregons are actually made out of energy. So if Positive and Negative Energy have an alignment, so do they. If using the Crawling Darkness option, the Xag-Ya is Neutral Good, and the Xeg-Yi is Neutral Evil. If using the Playing With Fire option, both remain as printed -- they are Neutral.

\section{Spells}

\subsection{Animate Dead} If Negative Energy isn't Evil, this spell isn't either. Zombies and Skeletons are the only possible creations of this spell, so the alignment tag is contingent on Negative Energy itself being a moral choice. Interestingly, create undead and create greater undead stay [Evil] even if animate dead doesn't. Regardless of the moral inclinations of negative energy in general, Ghouls and shadows are just not nice people -- they are a disease that exists for no purpose but to consume the living. So those [Evil] tags are on no matter what skeletons do with their free time.

\subsection{Deathwatch} This spell doesn't even use Negative Energy, it allows you to see positive energy. There's no reason for this spell to be evil no matter what version you use -- this is just a typographical error as far as we can tell. Maybe this evil tag was supposed to be on death knell.

\subsection{Create Undead} While animate dead may or may not be evil depending upon your setup, create undead and create greater undead is an [Evil] spell regardless of the morality version you use. It creates evil creatures that unlive for nothing but to slay innocents, so it gets the Evil tag for the same reason that planar binding gets the [Evil] tag if it is used to call a Demon -- it's bringing irredeemable evil into the world -- the moral implications of the negative energy used are irrelevant.
