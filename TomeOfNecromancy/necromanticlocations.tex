\section{Locations of Necromantic Importance}

Rules for locations that have interesting effects upon the dead are scattered throughout various published sourcebooks. Of primary interest is Black Sand (from Sandstorm) which heals undead and can be grown by killing creatures on it (no necromancer should be without a portable hole bottomed with this stuff), and Black Water (from Stormwrack) that acts as a desecration effect and is available with a Wizard spell (thereby breaking the stranglehold monopoly of Clerics on getting bonus hit points for their skeletons). But none of those locations are necromantically important. They are essentially quasi-mobile magic items that necromancers like to put in their pants. What follows are some locations that Necromancers will care about for more than a single mining session.

\subsection{Necromantic Intelligence}

Great and terrible crimes are often committed, sometimes causing the dead to rise. When enough dead rise in a single place, or a single act of murder or slaughter is so great as to create dozens of the undead, a Necromantic Intelligence can be born in a location central to the event. In such a place, trees and undergrowth wither and animals die, the sun no longer shines as brightly as mists obscure the sky and evil descends on the land. In such a place, all of those who die become undead and lose their free will. Ghoul or shadow infestations, vampire massacres, sites of great battles or disasters, or even the combined works of cabals of necromancers can create Necromantic Intelligences.

In the area of a Necromantic Intelligence, the land is either shadowy during the day as dark clouds obscure the sky, or misty (treat as an obscuring mist, even though it may be composed of dust, ashes, or some other substance). Any living creature killed in the area becomes an undead creature with a CR equal to its former CR (DM's choice, unless the Necromantic Intelligence is Aspected) when the sun next sets.

The most terrifying facet of a Necromantic Intelligence is that it has a purpose and a will, and it coordinates the undead that compose it. Assume that it is a creature that can see anything that any of its undead can see. Often it will coordinate fiendishly clever tactics using masses of undead to fulfill its purpose. Like a ghost, if it should ever attain its purpose, it will be destroyed. Knowing this, some clever heroes have helped Necromantic Intelligences in an effort to destroy them. A \spell{legend lore} or bardic knowledge check is often needed to discover an Intelligence's purpose.

The Necromantic Intelligence commands the activities of a great number of undead of varying powers. As a rough guide, the Necromantic Intelligence controls undead with CRs equal to the levels of followers attractable by a character with a Leadership score of 35 or more (using the Epic Leadership rules). Challenging the entire Necromantic Intelligence is an EL 11 adventure.

\subsubsection{Aspected Necromantic Intelligence} While most Intelligences are random manifestations of negative energy, creating many different kinds of undead, some places are Aspected. These places only create one kind of undead. For example, a Necromantic Intelligence created in a ghoul warren may only create ghouls, while an Intelligence created during a plague may only create plague zombies. Decrease the EL of such an Intelligence by at least 1 as players will prepare tactics suited to that specific kind of undead.

\subsubsection{Cleansing the Focus} Every Necromantic Intelligence has a Focus. This is an area that is the symbolic center of the undead infestation. If anyone can perform a hallow spell at the site of the Focus, the Necromantic Intelligence will be destroyed; however, once the ritual is started, the Necromantic Intelligence will be alerted and it will send all available undead to destroy the caster.

\subsection{Tombs}

While most tombs are merely places of rest for remains, some tombs become focal points for Negative Energy as hundred of years pass in the presence of the dead. Also, years of habitation by undead creatures in an enclosed space can also wear at the boundaries of the Negative Energy Plane. Some Necromancy effects can create or exploit this property. The game effect of a Tomb is that all undead inside it gain fast healing 1 and cannot be Turned or Rebuked, and spells with the [Tomb] subtype can be cast within it. Undead cannot be created within the confines of a Tomb, and creatures slain by undead do not become spawn.

Tombs are always enclosed places, and if they should ever be exposed to sunlight (by smashing in the roof, for example), they lose all special properties and no longer confer effects to undead creatures.


\subsection{Forsaken Graveyards}

The number of deaths is one per person even without the intercession of powerful magic. And once spells like raise dead are taken into account, it is clear that in Dungeons and Dragons there are significantly more deaths than people. So the locations where the most deaths occurred are simply the locations where the most living people live. The sites with the greatest death count are aspected to life and trade, not to death and destruction. But there are places that are inexorably linked with death, where the dead rise to slay the living. Creating a land of horror such as this requires more than killing a bunch of people (although that certainly helps), the deaths themselves must be meaningless and cruel, the ends coming about through betrayal.

A Forsaken Graveyard is a dangerous place, even for a necromancer. Creatures within a Forsaken Graveyard have Turning Resistance of +3. This makes both turning and rebuking more difficult, and throws salt in the game of both the necromancer and the hunter of the dead. Corpses left within a forsaken graveyard have a tendency to rise up and slay the living from time to time. Every sunset, a number of undead creatures are created and go on a rampage. These undead creatures fall back to death when the sun rises. A body left within a Forsaken Graveyard for more than an hour can be turned into an undead creature even if it had previously been an undead creature and been destroyed. Undead creatures created within a Forsaken Graveyard have an extra 2 hit points per hit die.
A Forsaken Graveyard can be cleansed with four castings of consecrate or desecrate (one at each corner of the area), or a single casting of Tasha's tomb tainting (in the middle). Unfortunately, these spells can only affect it during the nighttime (as during the day, there is literally nothing to cleanse). Once cleansed, any undead created by the Forsaken Graveyard lose their bonuses, but are also not recalled at sunrise. Such undead creatures will continue their rampage until slain. Unlike a necromantic intelligence, the Forsaken Graveyard has no ability to direct the undead against specific targets.

Cleansing a Forsaken Graveyard is normally an appropriate adventure for a 6th level party, and the location itself spawns one CR 7 creature, one CR 6 creature, two CR 5 creatures, and six CR 4 creatures every night. These creatures are undirected in their assaults on the living, and travel individually or in groups of two. A Forsaken Graveyard adventure can be scaled up or down for adventurers of differing power by changing the power levels of the creatures within it, or simply changing the parameters of the encounter. If a standard graveyard is itself small enough that every creature is encountered simultaneously, that would be an EL 11 encounter.

\subsection{Pools of Deep Shadow}

Veteran players of Dungeons and Dragons often ask ``Why don't Shadows just take over the whole world?" Certainly, there are very few residents of the worlds of D\&D that can fight against a Shadow at all, and their victims rise from the dead as Shadow Spawn, so it doesn't take a lot of imagination to see where this is going. However, there are a few things limiting the growth of Shadow armies that are not mentioned in the core books at all.

The first is that only intelligent creatures slain by Shadows turn into spawn. That's important, as it means that Shadows cannot simply hunt frogs in the swamp until they number in the tens of thousands before they roll over cities and dragon caves like a fog of Death Incarnate.

But perhaps even more importantly is that almost any time you see a Shadow, or for that matter any incorporeal undead creature, you are looking at a summoned creature. When the Shadow's summoning ends, all of its spawn vanish. Most of the time, an incorporeal undead is summoned forth from the Negative Energy Plane by an object that looks much like a puddle of very oily water, called a Pool of Deep Shadow. Whenever light falls directly upon the pool, or the sun rises high enough in the sky that there are no shadows (about half an hour before and after noon), the summoning effect ends and the Shadow vanishes. When the shadows grow long and darkness has fallen upon the pool, a Shadow is again summoned.

This means that an individual Shadow or Wraith has a very difficult time destroying the whole world, as there is no particular way for them to get more than a day's float from their pool. It also means, however, that areas inhabited by Shadows are extremely dangerous -- for even if such a creature is destroyed it will return again the following day. And on every day it will return until those charged with exterminating it are caught unlucky or unaware. In order to permanently destroy such a pool, a flask of Holy Water (or Unholy Water) need simply be poured into it, causing the blackness to depart and the water to become quite clear and drinkable.

\subsection{Finality}

Finality is a planar metropolis in the Infernal Battlefield of Acheron. Its harsh laws are kept in rigid and uncompromising order by the will of powerful pit fiends, and the city serves as a marketplace for the lucrative trade in souls. Magic items can be bought or sold here, but the currency is always souls (as a planar metropolis, Finality has a gp limit of 600,000 gp). Souls are valued at their CR squared, multiplied by 100 gp. Many items purchased from this location radiate evil, buyer beware. Lodging may be purchased at flat rate of one soul per day per person. The section of Acheron that Finality rests in has the Timeless trait and is mildly Lawfully aligned. The population of The City is about 100,000 people (with uncounted millions of souls), most of whom are Baatezu.

The rules in Finality are uncompromising and bizarre, and the punishments for breaking them are vindictively carried out to the letter by powerful devils. But there is no warfare allowed in the city, and even Celestials and Tanar'ri come to participate in the great mercantile dance of soul collection. Characters must make a Knowledge (Local: Finality) check everyday with a DC of 10 + 2 per day they've been in the city or unknowingly break one of the city's many inscrutable laws (knowingly breaking the law by starting fights or stealing goods is a whole different thing). Punishments range from perplexing to fatal. Characters who stay away from Finality for more than a month are no longer subject to the baroque residency rules and their DCs are returned to 10 the next time they visit.
