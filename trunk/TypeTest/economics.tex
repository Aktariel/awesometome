
\section{The Economicon} %: Making Sense of the Gold Standard
\vspace*{-10pt}
\quot{``100 pounds of gold for a house? How does anyone make rent without a wheelbarrow?"}

Since time immemorial, D\&D has used the ``gold piece" as its primary currency. It is apparently a chunk of reasonably pure gold of vaguely standardized weight that people use fairly interchangeably in different cities populated by different species. In the bad old days, each gold coin was a tenth of a pound, which was hilarious and inane. In the current edition, each gold piece is a fiftieth of a pound. That's 3.43 gp to the Troy Ounce, which means that in the modern economy, each gp is about \$171 worth of gold. Obviously, gold is significantly more common in D\&D than it is on Earth, gold is also undervalued because its status as a currency standard drives it out of industrial uses and causes inflation. Further, populations in D\&D are orders of magnitude smaller than they are in the real world, so the gold per person is higher even with the same amount of gold. So the gold piece is massively less valuable in D\&D economies than it would be in Earth's economies.

Nonetheless, things are really expensive in D\&D, and the high price in gold means that there's a distinct limitation of how much wealth can be transported by any means available. The economies of currency transaction are actually so unfavorable that currency as we understand the term does not exist. Things don't have prices or costs -- all transactions are conducted in barter and a common medium of exchange is heavy lumps of precious metal.

\subsection{Wish and the Economy}

An Efreet can provide a wish for any magical item of 15,000 gp or less. A Balor can greater teleport at will, but can only carry 30 pounds of currency while doing so. Even in platinum pieces, that's 15,000 gp worth of metal. The long and the short of it is that at the upper end of the economy, currency has no particular purchasing power, and magic items of 15,000 gp value or less are viewed as wooden nickels at best. You can spend 15,000 gp and get magic items, but people in the know won't sell you a magic item worth 15,001 gp for money. That kind of item can only be bought for love. Or human souls. Or some other planar currency that is not replicable by chain binding a room full of Efreet to make in bulk.

Powerful characters actually can have bat caves that have sword racks literally covered in 15,000 gp magic items. It's not even a deal because they could just go home and slap some Efreet around and get some more. But even a single major magic item -- that's heavy stuff that such characters will notice. Those things don't come free with hope alone, and every archmage knows that.

\subsection{Wartime Economies Make for Shortages}

Many people wonder why a masterwork dagger goes for more than its weight in gold. That's a pretty valid question to ask; certainly I'm not going to attempt to justify the 600 gp price tag on a masterwork walking stick -- that's just an example of simplistic game mechanics run amok. But to an extent the crazy prices can be justified by the fact that every settlement in every D\&D world is on a war footing all the time. The idea that Peace is somehow a natural state is a fairly recent one, and based on the frequency of wars all over the world -- it's obviously just wishful thinking anyway. War is the default position of every major economy in the world, and that means that weapons have an immediate, and desperate, clientele. Iron is still relatively cheap, because you can't kill people with it right now, but actual weapons and armor are crazy expensive.

That doesn't explain the fact that the PHB charges you over a quarter Oz. of gold just to get a backpack, and it doesn't explain the fact that the markup on masterworking a buckler is the same as the markup on masterworking a breastplate -- that's just a game simplification that makes no real-world sense. But it's a start.

\subsection{Coins are Big and Heavy}
\vspace*{-8pt}
\quot{``How many boards could the Mongols hoard if the Mongol hordes got bored?"}

From the standpoint of the adventurer, the primary difficulty of the D\&D currency system is that the lack of a coherent banking and paper currency system means that there are profound limits to what you could possibly purchase even with platinum. But the currency system hurts on the other end as well. Untrained labor gets a silverpiece a week. That's 500 copper coins a year, which means that no matter how cheap things are they can only make one purchase a day most of the time. That's pretty stifling to the economy, in that however much gets produced, no one can buy it. Demand, from the economics standpoint, is strangled to the point where large production outputs don't even matter (remember that in economics Demand doesn't mean ``what people want," it means ``what people are willing and able to pay for," so if the average person only has 500 discreet pieces of currency per year, that puts an absolute cap on economic demand, even though the people are of course both needy and greedy enough to want anything you happen to produce).

What's worse, those coins are heavy. For our next demonstration, reach into your change drawer and fish out nine pennies. That's a decent lump in your pocket, neh? That's about one copper piece. Gold pieces are smaller (less than half the size, actually), but weigh the same. D\&D currency, therefore, is more like a Monopoly playing piece than it is like a modern or ancient coin. There's no reason to even believe these things are round, people are seriously marching around gold hats and silver dogs as the basic medium of exchange.

Now, you may ask yourself why these coins are so titanic compared to real coins. The answer is because having piles of coins is awesome. Dragons are supposed to sleep on that stuff, and that requires big piles of coins. Consider my own mattress, which is a ``twin-size" (pretty reasonable for a single medium-size creature) and nearly .2 cubic meters. If it was made out of gold, it would be about 3.9 tonnes. That's about eighty-six hundred pounds, and even with the ginormous coins in D\&D, that's four hundred and thirty thousand gold pieces. In previous editions, that sort of thing was simply accepted and very powerful dragons really did have the millions of gold pieces -- which was actually fine. Since third edition, they've been trying to make gold actually equal character power, and the result has been that dragon hoards are\ldots\  really small. None of this ``We need to get a wagon team to haul it all away", no. In 3rd edition, hoard sizes have become manageable, even ridiculously tiny. When a 6th level party defeats a powerful and wealthy monster, they can expect to find\ldots\  nearly a liter of gold. That is, the treasure ``hoard" of that evil dragon you defeated will actually fit into an Evian bottle.

There are two ways to handle this:
\begin{enumerate}\itemspace
   \item Live with the fact that treasures are small and unexciting in modern D\&D.
   \item Live with the fact that characters who grab a realistic dragon's hoard become filthy stinking rich and this fundamentally changes the way they interact with society.
\end{enumerate}


But once you accept that the realities of the wish based economy, you actually don't have to live with characters unbalancing the game once they find a real mattress filled with gold. That's not even a problem once characters are no longer excited by a +2 enhancement bonus to a stat or a +3 enhancement bonus to Armor. Which means somewhere between 9th and 13th level it's perfectly fine for players to find actual money without unbalancing the game. Really, you can stop worrying about it.

\subsection{Bad Money Drives Out Good: The Penalties of Paper}

People from the modern world are generally pretty perplexed by this idea of handing back and forth actual metal as a medium of exchange. It is an undeniable truth in our lives that the idea of currency is just that: an idea. As long as whatever I'm trading for goods and services can be traded for goods and services, it doesn't actually matter if the exchange commodity has any ascribed intrinsic worth. Paper descriptions of value or even ephemeral electronic representations are not only adequate, they're convenient. But more than that, using valuable commodities as a medium of exchange inhibits the growth of the economy. As long as a certain portion of the wealth is locked up in currency, the economy is strangled coming and going: not only is there a completely arbitrary limit on how many goods and services can be exchanged (the gold supply), but there is also a limit on the kinds of industry and artistic expression that can occur (in that if you use gold for anything but currency you're actually shrinking the money supply and producing negative GDP).

So\ldots\  you're going to solve that by instituting a paper-based exchange system where initially the paper is exchangeable for gold and that eventually gets phased out when the Plebes realize that handing actual gold back and forth is inconvenient and dumb, right? Wrong. Remember that this is the Iron Age, and people haven't invented Nationalism yet. The cornerstone of the Greenback currency is a belief in the nation that prints it -- and nations simply don't exist. You've got empires, and you've got kingdoms, and you've got tribes, and you've got unincorporated villages\ldots\  and that's it as far as civilization goes. When you look at a map in D\&D and a colored region has a name on it, that's the name of the region. Possibly it's even the name of some guy in the region. The point is, that it's not a country in the modern sense of the word, so if some new guy walks in who's bad enough the next cartographer will put his name on the region instead.

And that means that ``The Full Faith and Credit of the Kingdom of Daxall" is worth precisely nothing. And while King Daxall can, through force of arms, take all the gold away from all the peasants and get them to trade pieces of paper for goods and services in its place -- no one will actually believe that the paper is currency. They're literally trading promises by King Daxall that he'll let them have their money back if they leave town. And since the serfs can't even leave town, even that promise is meaningless to them. A serf accepts paper for goods and services only because he'll be beheaded if he doesn't. The black market value of these pieces of paper is pretty close to zero. Worse, nearby governments will see this as a blatant attempt to sequester all the gold in King Daxall's pants and will probably declare war (in addition to the fact that no one outside the reach of King Daxall's pikemen will accept Daxall Dollars).

\subsection{Powerful Creatures Have a Powerful Economy}

The amount of gold it takes to get anywhere as a land lord is very large. The question that arises then, is why awesome architecture exists at all. It's a valid question, the listed costs to put things like pit traps and thrones made of bone into your dungeon are stupendously large and actual magical swag can be made available for much less than that. The answer is that:

\begin{enumerate} \itemspace
   \item People don't actually pay all that gold to have their homes remodeled (see the peonomicon below).
   \item Powerful artificers and adventurers don't even want your gold. If something has a value of 100,000 gold pieces, it can't be purchased with gold pieces at all -- because that's an actual ton of gold that you'd have to plop over the counter and the merchant you're dealing with won't take your money even if you have it.
\end{enumerate}

Here we're going to be focusing in on

\listone
    \item Gems
    \item Souls
    \item Concentration
    \item Hope
    \item Raw Chaos
\end{list}

\subsubsection{Gems: Truth or Dare}

Gems are, to the vast majority of participants in the economy, pretty much worthless. A 500 gp diamond is pretty much the same as a gold piece to someone who intends to purchase things with a value of 1 gp or less. And of course, there are a lot more individuals out there who will stab a peasant in the face for a diamond than a gold piece. So why does anyone care?

Well, two reasons: the first is the obvious one that gold is extremely limited in what it can possibly purchase. A +2 sword is worth your weight in gold. Not its weight in gold, your weight in gold. It seriously costs over 166 pounds of gold, and that's just not reasonable for most people to put into their pockets. So people interacting with even the shallow end of the magic trade need there to be some crazily expensive items that have no purpose save to look pretty and be exchangeable for other stuff. But unlike our world gems actually have real value as well: as the fuel for powerful magics.

On Earth, the only reason that a diamond is expensive is because there's an international organization called DeBeers that seriously has actual assassins that will shoot you in the face if you try to sell diamonds for less than the price they've determined that they're supposed to be sold for. D\&D doesn't have that kind of armed monopoly to maintain gem prices, but it does have the fact that people continuously use up gems for spells like raise dead and item creation and the like. So the fact that you can use ruby dust to make continual flames that you can turn around and sell as Everburning Torches means that ruby dust will continue to have value as long as people value light.

The D\&D rules actually only go into the spell component uses of a handful of gems, but rest assured that all the rest are similarly useful when we get into the ephemerals of item creation. A lot of those ``components" that cost piles of thousands of gold pieces are actually just piles of gems. Onyx keeps its value based on the needs of necromancers, but amethyst is just as needed to bind illusion magic into a cloak. The exchange rate between gems and magic items is in no danger of going anywhere. Minor magic items and gems are traded avidly by shopkeepers, adventurers, and even powerful outsiders and wizards.

But even so, gems can be simply acquired by the very powerful. The realities of the wish based economy ensure that gems can simply be obtained in large numbers by anyone who really cares enough to dedicate a conjured earth elemental to collecting them. Magical items that cannot be created with the application of spells (that is, magic items valued at more than 15,000 gp) cannot be purchased on the open market with mundane currency, not even gems. That isn't to say that you can't cheat a goblin out of a staff of power with some shiny rocks, you totally can (heck, you could also stab the goblin in the face and take that staff of power), but doing so is not considered a ``fair trade" and requires a bluff check on your part.

In addition, many D\&D worlds posit the existence of magic gems, which can be used to make magic items, increase personal power, make a snazzy grill with the bottom row made of gold, and all kinds of stuff. In addition to getting hot women to ask you to smile, these magical gems are magical and are actually considered fair exchange in the near-epic economy. You can't wish for Eberron Dragonshards or Planescape Planar Pearls, so those things have real value to Efreet and other creatures participating in the Big Pond. Rules for using magic gems appear in the Tome of Tiamat.

\subsubsection{Magical Currency}

\listone

\itemability{Souls:}{The souls of powerful creatures are trapped in gems and the trade in them is brisk on the outer planes, especially in the planar metropolis of Finality on Acheron. Once a soul is in a gem, the gem itself is of little or no value, but the soul goes for 100 gp times the square of the CR of the creature whose soul is trapped (see Tome of Fiends for more information on the use of souls).}

\itemability{Concentration:}{Ideas take form on the outer planes, and really pernicious or stellar ideas can be so powerful that they take a while to form. In the before-time, they can be found as an amber-like substance that is extremely valued on Mechanus, and by extension every single other outer plane as well. Concentration is actually made out of ideas, and while it looks like a solid object it is actually a liquid that flows so slowly that you could watch it for a year and only a Modron could tell you have far the flow had taken it. A pound of concentration goes for 50,000 gp to an interested party, and can be used in magical crafting by those with the patience to learn its secrets (see Book of Gears for more information on the use of Concentration).}

\itemability{Hope:}{Hope is funny stuff, it has lots of inertia, but those who carry it are not weighed down in the least. It has mass, but not weight. Even the smallest piece of Hope sheds light like a daylight spell (the effective spell level for this effect is 7, and Hope can overcome almost any darkness). Hope is measured in kilograms rather than pounds, and a kilo of Hope goes for 100,000 gp to those who want it, and it can be used in magical crafting (see Tome of Virtue for more information on the use of Hope).}

\itemability{Raw Chaos:}{The plane of Limbo is filled with possibility and change. Usually this manifests as a continuous creation and destruction that is awe inspiring and terrifying at the same time. Sometimes, for whatever reason this possibility doesn't become anything, and just stays as Raw Chaos. Raw Chaos can have any dimensions and any amount of mass, but from a practical standpoint you either have it or you don't. If you have Raw Chaos and someone else doesn't you can give it to them, and it is generally considered good form for them to give you magical items or planar currency worth 200,000 gp in exchange. Raw Chaos can be transformed into magical items by those with the correct skills (See Tome of Tiamat for more information on the use of Raw Chaos).}
\end{list}

\subsection{The Service Economy: The Profession Rules Don't Work}

The profession rules make us sad. Very sad. Which is unfortunate, because almost everyone in the entire world who isn't an adventurer apparently lives and dies by these things. While the powerful adventurers go off into the planes and exchange Raw Chaos and the Souls of Champions for powerful magical items and favors, your average orc is running around delivering halfling food or joining the army of a powerful warlord for little bits of metal. When the players begin their adventuring careers, they'll be caught up in this economy as well. And even if they eventually become powerful enough to purchase mighty rods with planar currencies they might still be intimately involved in it -- as one of those mighty warlords who throws out tiny pieces of metal to orcish warriors and starting adventurers.

Here's the deal: if your character is a Sailor, that's character flavor. It's not a major portion of your character's power and we really are willing to just give it to you. Having a profession is like knowing a language: sometimes it will come up and sometimes it won't. In that spirit, we suggest that Profession cease being a ranked skill altogether. Just like people don't make ``speak dwarvish" checks to have words come out of their mouth, characters should not have to make ``Profession: Barkeep" checks to successfully sit behind a bench and hand people ale.

People who have a profession don't make checks to make money, they get a wage if they happen to have a job. The wage will depend on what kind of work they are doing (so no, you can't put 10 ranks into Profession: Janitor and be better paid than the magistrate). Characters are assumed to make a wage approximately similar to the one in the table below if they are working and have an appropriate professional skill. DMs may allow a character to put two ranks into a single Profession skill and be a ``master whatever".  Such characters may be able to boast about their skills or perhaps even make more money. The important part is that this means that you can find really good scullery maids who don't have a +5 BAB. Young children can often be drafted to do grown-up jobs, and need only be paid 1/10th the normal rate for whatever it is that you have them doing. Child labor is cheap, but in some ways you get what you pay for and children may become distracted or sick before completing important or dangerous jobs.

\subsubsection{Professions and their Pay Scale}


\featnamelist{Profession Wage/Week}

\begin{multicols}{2}
\begin{small}
\listone
    \item Acolyte 5 GP$^\dagger$
    \item Alchemist 10 GP$^\dagger$
    \item Artisan 5 GP
    \item Bartender/Innkeeper 15 SP
    \item Barrister 8 GP
    \item Butler 2 GP
    \item Clerk 3 GP$^\dagger$ (includes more influential administrators)
    \item Cook 1 GP
    \item Courtesan 5 SP$^\dagger$
    \item Farmer 5 CP (Farmers also feed themselves)
    \item Fisherman 3 SP
    \item Groom 1 GP
    \item Guard 15 SP$^\dagger$
    \item Laborer 1 SP (note: this means no profession at all)
    \item Laborer, Skilled 2 GP
    \item Librarian 3 GP
    \item Janitor/Maid 8 SP
    \item Military Officer 5 GP$^\dagger$
    \item Miner 2 GP
    \item Porter 6 SP
    \item Runner 1 GP
    \item Sage 10 GP$^{\dagger \star}$
    \item Sailor 2 GP
    \item Scribe 2 GP
    \item Servant 8 SP
    \item Shepherd 2 SP
    \item Smith 15 GP
    \item Smith, Master 150 GP
    \item Soldier 15 SP$^\dagger$
    \item Tailor 1 GP
    \item Teamster 2 GP
    \item Torturer 2 GP
    \item Valet 15 SP
    \item Wage Mage 10 GP$^\dagger$
\end{list}

\noindent $^\dagger$: Some professons are actually dependent upon class level and abilities. A 1st level Wage Mage commands a wage of 10 GP a week to sit around and cast 1st level spells and cantrips from time to time, but a 12th level Wizard would command an earnings per week so large that most kingdoms find it more expedient to simply make such magicians part of the government.

\noindent $^\star$: Any skilled profession that is based on one of the ten Knowledge skills in D\&D is a Sage, and is not handled with the Profession skill at all. An Architect does not have ``Profession: Sage", he has Knowledge: Architecture and Engineering. The pay scale of a Sage of any kind is extremely dependent upon his skill results. A character with four or five ranks in a couple of knowledges might pull down 10 GP per week, but a character who can regularly make a DC 30 check in any subject no matter how arcane can pull down the big bucks. Assuming of course that he can find someone that actually needs his services.
\end{small}
\end{multicols}


Just because you selected a profession that makes a lot of money doesn't mean that anyone will hire you. Generally only relatively organized areas actually have economies that even can hire Butlers and Clerks. But just because there is work available in an area doesn't mean that there's work available for you. Even in major cities there aren't a whole lot of jobs for a clerk or a barrister, so the competition for those jobs is pretty stiff. Prospective employers are fairly choosy about who they select for such employment, and they'll usually go to guilds (whose reputation is on the line every time they vouch for someone) or their own aristocratic family members rather than hire some random Half-Orc who claims to have the requisite skills.

\subsection{Running a Business}

The rules presented in the DMG2 for running a business make us very sad. Apparently the best way to make money is to run a shop out of a shack in the woods and pour money into it until noble djinni are teleporting to your door to hand over large gems for whatever the heck it is that you're selling. That doesn't make any kind of sense at all. We propose instead that the costs and benefits of running a business should be kind of comparable to those of working for a wage -- since it is essentially exactly the same thing. What we're looking for is rules for running a business that aren't so obviously abusable over time, and which reward various business models rather than finding the killer app that makes the most money (the Shop as it happens) and just using that over and over again.

\subsubsection{Capitalization}

First off: the thing where in the basic DMG2 rules you can capitalize over and over again forever and have the profits go off towards infinity is as abusable as it is dumb. So the very first change that needs to be made is the divine decree that you can't do that. In fact, the concept of recapitalizing just wasn't handled well there at all. It takes money to make money, but investment is not a ladder where you set money on fire until the pyre lights the heavens ablaze and gets you epic items in parcels like clockwork. Instead, starting a business venture costs money -- we call that initial capitalization. That's a one-time cost and the only way you can spend it again is if you start up a second business. After that, you have to supply one-third of the business' expected earnings for each month up front, we call that operational capital. If your business is still running at the end of the month, you get that money back (in addition to the earnings themselves), but if the business venture folds or you get driven off by rampaging monsters, or business events cause the venture to make no money for a month -- that operational capital is gone and you're out a pocket full of shells.

Initial capitalization isn't any cheaper in the wilderness than it is in a big city. Actually, it's more expensive because you have to get goods shipped out into the wilderness to get the whole thing off the ground -- and the wilderness in D\&D is dangerous and teamsters make 2 GP a week each in compensation for that fact. Operational capitalization is cheaper in the wilderness, because expected earnings are less and therefore 1/3 of those earnings is also less. Yes, this means that business owners normally go to the city to conduct business, where there is a whole governmental apparatus to facilitate business dealings and a steady parade of caravans and ships to bring your product or service to the world. The only reasons that anyone does their business outside of major cities is because some particularly risky ventures can only be done far from town (for example: a Larvae Orchard is a high-risk, and therefore high profit enterprise, but it can only be located in the Wastes of Hades).

\subsubsection{Risk}

Risky business ventures make more money. But they also suffer catastrophic mishaps more often. That's what makes them risky. They are not simply an increase to the multiplier on the profit check, because that just makes you more money because player characters don't start businesses that aren't going to have positive profit checks. Maintaining a Risky venture involves you having more challenges to maintain your business -- which in a roleplaying game like D\&D means essentially that you spend more adventures maintaining your business and therefore spend less adventures looting other peoples' dungeons. The extra profit you make from the risky business is offset by the extra challenges you need to overcome. Essentially, taking on a risky business is just like getting the gold from your encounters before you go adventuring.

Risky businesses have a CR and a frequency. The DM is encouraged to send additional problems your way at roughly the frequency of the risk factor, and the ELs of the problems thrown your way should be roughly the same as the CR of the risk factor. Risky businesses also make a lot more money -- roughly the value of an ``average" treasure of an encounter of a CR equal to the risk factor every interval of time equal to the risk factor (see the DMG, p. 51). So an onyx mine that had a risk factor of 5/4 months would generate an extra 400gp per month (1600 gp/4) and be plagued with an EL 5 encounter roughly 3 times a year. It's just that easy.

Not all shops are the same. If you're selling burlap clothing, the profits are going to be small and ogre bandits won't even try to take all your stuff. If you're selling weapons of war or magical materials, then you can bet that those ogre mercenaries are going to be a little bit more interested. If you're running a more valuable business (that is, one which makes more money), the villains of the D\&D world will come to take it from you -- the risk factors adjust themselves pretty much automatically when your business improves, making this approximation amazingly accurate in addition to simple.

\subsubsection{Resources}

Resources are like Capitalization that you get to keep. While the presentation in the DMG2 is essentially ``something that makes it harder to turn a profit on your business", the fact is that what they actually are is your own private dungeon. While the full rules for actually building your dungeon are going to have to wait until Book of Gears and the advanced crafting rules, for now we're going to assume that the prices in the Stronghold Builder's Guidebook hold up (and yes, we know how silly that is, but we haven't written anything better yet). Essentially, this means that your business needs to be housed in a building, or ship, or cart, or dungeon of some kind. Bigger, more high-scale business ventures are going to need to be housed in more expensive surroundings. That sounds bad, but remember that when business events and risk factors happen to your business, they happen to your business, which means that if you have a ship or a tower to hold your stuff in, you actually get to use it when it gets attacked by gnoll pirates.

Keep in mind that if a business is booming, it may require more resources to house. A shack is all well and good if you plan to sell a couple of pots a month, but if you want to move inventory you've got to have inventory. And that means you need a place to show that inventory. Practically, that means that your projected profits (before calculating Risk-based Profits), can't ever exceed 1/10th the value of your business' resources. Of course, some businesses can only exist with large amounts of resources backing them up. And that's fine, since you really only get the benefits of large resources in large urban areas, this means that in general there are a lot of services that can be found in the big city that can't be found in smaller towns. Which is exactly what you'd want, right?

\subsubsection{Growing the Business}

Characters may outgrow collecting melloweed from the Bane Mires. The occasional hydra they have to defeat to get the goods just doesn't challenge them anymore, and the gold the whole thing takes in every month just doesn't seem worth the hassle. When this happens there are two options: franchise the operation, or grow the business up. A business can be expanded to a larger operation by investing in the next level of resources (causing it to be eligible to make more profits), or by taking on higher value/risk goods and clients (causing the risk factor to increase and profits to increase as well).

Franchising a business simply involves starting up a second (or third) business in another location. Resolve it as a whole new business.

\subsubsection{Profits}

So how much money do these things make? Well, in addition to Resource Limitations, there are demand limitations. That is, the amount of money that people can spend on your goods and services is proportional to how much money they have -- larger communities can spend more money than can smaller communities. The maximum profits per month of any venture are based on the total population that business serves. If you compete with other businesses providing the same goods and services, simply divide the region's population according to market share before you determine maximum profits.

\featnamelist{Population Size / Gold per Month}
\listone
    \item 20-80 -- 4 GP/month
    \item 81-400 -- 10 GP per Month
    \item 401-900 -- 20 GP per Month
    \item 901-2000 -- 80 GP per Month
    \item 2001-5000 -- 300 GP per Month
    \item 5001-12,000 -- 1,500 GP per Month
    \item 12,001-25,000 -- 4,000 GP per Month
    \item 25,001-100,000 -- 10,000 GP per Month
    \item 100,001+ -- 60,000 GP per Month
\end{list}

\vspace*{8pt}

Remember that while this determines the maximum profits, there's no guaranty that your business will actually do as hoped. Things don't always work out as planned, and many business plans aren't good. In order to make your business succeed, you'll have to make a Profit Check. Actually making the projected Profits is a DC 20 check. Every point you fail that DC, reduce your income by 5\%. For every point you exceed 20 on your Profit Check, add 5\% (essentially this just means that you make a 5\% return for every point of Profit Check you make).

The Profit Check itself is simply a straight ability check, using your choice of your Intelligence, Wisdom, or Charisma. Some of the modifiers to Profit Checks from the DMG2 are appropriate, others are not. For your convenience, we're replicating the entire chart with all the needed modifications:

\listone
    \item Owner has appropriate Profession Skill +1
    \item Owner has two appropriate Profession skills +2
    \item Owner is a member of an associated guild +1
    \item Owner spends less than 8 hours per week assisting business operations -8
    \item Owner spends more than 40 hours per week assisting business +1
    \item Business is considered a Monopoly +10
    \item Business is an Oligarchy +4
    \item A Business Partner aids during the term +2
    \item Specialists are on staff +2
    \item Previous Profit Checks ``Failed" -1 per consecutive check below 15.
\end{list}


\subsubsection{Command Economies}

Sometimes your ``business" is actually just that you run a country, or a guild, or a church, or a criminal organization, or a mercenary command. Or whatever. The point is that your job is to run things, and people pay taxes (or tithes, or protection money, or whatever the kids are calling it these days) to you to make sure that you keep running things in a manner that doesn't involve them being stabbed in the face. The amount of lucre you can squeeze out of these situations has nothing to do with your skill checks or capitalization -- you're essentially stealing from these people so the amount of money you can crank out of them depends largely on how much you're willing to squeeze them and how many people you are squeezing. Taxing a group of people can generate as much money as running a business serving them would. Your ``business" in this case is ``not stabbing them in the face".

You can be senselessly wicked and punitive on a population and make twice as much gold, but your subjects will hate you. You can also simply sack a region, making ten times as much gold, but driving the remaining population away as refugees. Lawful creatures (such as Hobgoblins and Dwarves) are more likely to pay taxes or save money and taxing or looting them is worth twice as much. Especially impoverished regions (such as one which has labored under a cruel governor for a long time) are worth half as much or less.

\subsection{Bringing the World out of the Dark Ages}

It is historical fact that you can take a ridiculous and crumbling imperium with serfs and horse-drawn carts managed by a tyrannical and squabbling aristocracy and boot strap it into being a technologically sophisticated global power that can win the space race and such in a single generation even while being invaded by an evil and genocidal empire. The people at the top don't even need to be nice or sane, they just have to understand that economics is an entirely voodoo science, and the limits of production can be broken by thousands of percentage points by getting everyone to buy on credit, work on projects that people looking at the big picture tell them to work on, continuously invest in productive capital, and believe in the future.

Right. That's called Communism, and it ends the dark ages immediately even if it isn't run well. Presumably if it was being run by Paladins who actually radiate goodness and Wizards who are inhumanly intelligent and can cast powerful divinations to determine projected needs and goods could be distributed to the masses with teleportals -- it would work substantially better. That sort of thing is not outside the capabilities of your characters in D\&D. It's not outside the capabilities of the people in the village your characters are saving from gnollish invasion. It's not even technically complicated. But it isn't done.

Partly it isn't done because we're playing Dungeons \& Dragons, not Logistics \& Dragons. While it is true that you can fix the world's ills in a much more tangible fashion by industrializing the production of grain and arranging a non-gold based distribution system such that staple food stuffs are available to all, thereby freeing up potential productive labor for use in blah blah blah\ldots\  the fact is that to a very real degree we play this game because telling stories about slaying evil necromancers and swinging on chandeliers is awesome. But the other reason is that the society in D\&D really isn't ready for a modern or futuristic social setup. No one is going to understand how they are supposed to interact with Socialism, Capitalism, or Fascism, things are Feudal and people understand that. Wealth is exchanged for goods and services on the grounds that people on both sides of the exchange aren't sure that they would win the resulting combat if they tried to take the goods or wealth by force of arms.

Rome had steam engines. Actual difference engines that propelled a metal device with the power of a combustion reaction through the medium of the expansion of heated water. Really. They never built rail roads because slaves were cheaper than donkeys and the concept of investing in labor saving devices was preposterous. In D\&D, the idea of having an economy based around trust in the government and labor/wealth equivalencies is similarly preposterous. It's not that the idea wouldn't work, it's that every man, woman, and child in society would simply laugh you out of the room if you tried to explain it to them.
