
\section{Optional Rules for Fiends}

There are a number of places in the rules that governing Fiends that just don't work at all, or don't work in a way that is good. This is an attempt to fix them.

\subsection{No Wishing for More Wishes!}
The 3.5 wish spell is very explicit in what it can do, and extremely vague about what it can't do. It has a big list of things it is capable of, and then tells the DM to ad hoc things if anyone wishes for anything that isn't on that list. Unfortunately, wishing for a Staff of 50 wishes is on the list of things you can wish for. The XP cost is considerable (512,180 XP), but if you get your wishes from a magic item (like a Staff of 50 Wishes) or a spell-like ability (like an Efreet), you don't have to pay that XP cost, so the fact that it is stupidly large doesn't even matter. Needless to say, the game completely breaks down as soon as that happens. So in that spirit, we suggest an alternate list of things wish can do, coupled with some things wish actually can't do:

\listone
    \item Free Wishes -- the following wishes have no XP cost:
    \listtwo
      \item Wealth: A character can wish for mundane wealth whose total value is 25,000 gp or less.
      \item Magic Item: A character can wish for a magic item that costs 15,000 gp or less.
      \item Power: A character can wish to increase an inherent bonus to any attribute by 1 (to a maximum of +5)
      \item Spell: A character can wish for the effects of any spell that lacks an XP cost that is lower level than the highest level spell in its spell list (a wizard spell of 8th level or less, or a paladin spell of 3rd level or less, for example).
      \item Transport: A character can wish herself and 1 other willing creature per caster level to any location on any plane.
    \end{list}
    \item Wishes that aren't Free -- the following wishes cost XP or gp or both:
    \listtwo
      \item Add to the Powers of a Magic Item: A character can increase the powers of a magic item to anything she could enhance it to with her own item creation feats. This requires 1 XP for every 10 gp increase in magic item value.
      \item Raise the Dead: A character can bring the dead back to ''life", even if they were an undead, construct, or other creature that cannot normally be brought back to life. This may even be able to bring back a creature who has been devoured by a Barghest (50\% chance of success). This costs 3,000 XP, which can be paid in any combination by the caster or the target. The spent XP for this wish can reduce a character's level, but coming back to life in this manner otherwise won't do so.
      \item Undo Misfortune: A character can wish back the sands of time in order to force events of the last round to be replayed. Time can be reset to any point back to the character's previous initiative pass. This use costs 1000 XP. While the action spent to cast wish in this case is restored, the character still loses the spell slot and XP used to power it.
      \item Turn Back Time: A poorly fated adventure can be averted entirely with a wish. The character expends the slot and pays 5,000 xp, and none of it ever happened.
    \end{list}
    \item Wishes that are Rituals -- some wishes have much greater costs, at the whim of the DM. Here is an example:
    \listtwo
      \item Become a new Creature: A character can wish themselves into being a new creature. This must be done when a character is eligible to gain a new level, and the character makes the wish and takes a level of the new racial class (or racial paragon class) and is now the new race.
    \end{list}
\end{list}


Any use of wish causes the wisher to become fatigued (and yes, there are ways to get around that).

Creatures with spell-like abilities that grant wishes may only grant wishes that have no XP cost. So an Efreet can give you as many +2 swords as it wants, but an Efreet can't give in to your request to have a +3 sword. Also, you'll notice that we categorize the inherent bonuses as something that's free and therefore going to be rapidly available to all the player characters somewhere between 11th and 15th levels. That's because we seriously believe that it is more balanced for characters to all gain +5 inherent bonuses than it is for some characters to figure out how to manipulate XP gains and thought bottles to get inherent bonuses while the other players don't. Inherent bonuses need to be available or not available to everyone or they break the game.

Magic items with wish on them can be used to cast wishes with an XP cost of at most 5,000 XP, and are produced as items using spells with a cost of 5,000 XP. As a result, you can't wish for an item that has wish on it.

\subsection{Damage Reduction and Special Materials}

The 3.5 rules were rather\ldots\  overzealous with splitting up material DR, and the result has been that high level characters actually just curl up and cry. Here are some guidelines to streamline things a bit:

\listone
\item Any steel weapon counts as ''cold iron" for the purposes of beating DR. Cold Iron being a special kind of iron mined deep underground is, well, insultingly stupid. Cold Iron is an actual word, it's the first mass-produced type of iron in history, and in song and story is effective against fairies and chaos demons because it symbolizes order and industrialism. Cold iron is cheap, that's the whole point. If it wasn't cheap, it wouldn't be available in industrial quantities, and then it wouldn't have any symbolic effect against savage fey and demons of disorder.
\item Alchemical Silver has no damage penalty. The fact that Silver has a damage penalty is sort of justifiable, except that in D\&D weapons made out of wood don't have a damage penalty. The game simply doesn't have a fine enough grain to keep track of the ways in which you'd rather have a sword made out of steel than a silver plated one. Also the thing where DR 1/Silver is in fact impossible to beat is incredibly dumb.
\item Material DR beats Material DR. Alignment DR beats Alignment DR. Creatures with DR can hurt other creatures with DR as if they had natural weapons made out of whatever punches through their DR. And creatures with alignment subtypes penetrate DR with their manufactured weapons as if they had the alignment of their subtype. So when a Balor punches a Pit Fiend (needs Silver and Good), his fist counts as Good and Iron. When a Balor swings a Silver Sword at the Pit Fiend, his weapon counts as Silver and Evil -- he has got all the needed adjectives, he just can't get them all at the same time. And that is really dumb. What should happen is the fact that the Balor needs an aligned weapon made out of a special material to be hurt should be sufficient to hurt the Pit Fiend with his natural weapons.
\item There can only be five! An unfortunate and unintended result of the 3.5 DR rules is that as more materials and monsters get written, the chances of you having whatever material your target's DR is penetrated by drops to a number pretty close to zero. In order to keep that from happening, we propose that for the purposes of DR, there are only 5 materials, and absolutely everything counts as one of those five. So if your weapon isn't made out of: Adamantine, Iron, Silver, Stone, or Wood, it counts as being made of one of those materials. Here is a suggested weapon equivalency chart:
\listtwo
    \item Adamantine: \listthree\small
       \item Alchemical Gold
       \item Black Steel
       \item Orichalcum
       \item N Metal
       \item Thinaun
       \item Urdrukar\end{list}
    \item Iron:\listthree\small
       \item Blood Steel
       \item Green Steel
       \item Morghuth Iron
       \item Truesteel\end{list}
    \item Silver:\listthree\small
       \item Pandemonic Silver
       \item Astral Driftmetal
       \item Entropium
       \item Nerra Mirrorblade
       \item Ysgardian Heartwire
       \item Mithril\end{list}
    \item Stone:\listthree\small
       \item Tainted Obsidian
       \item Blended Quartz
       \item Elukian Clay
       \item Kaorti Resin\end{list}
    \item Wood:\listthree\small
       \item Bronzewood
       \item Chitin
       \item Darkwood
       \item Iron wood
       \item Boneblade
       \item Dragon Bone\end{list}
\end{list}\end{list}

\subsection{Putting the Prime Back in Prime Material Plane (Alternate Prime Material Plane Rules)}
Many classic fiend stories involve demons or devils doing their best to get into the Prime. The real question is: why? The Lower Planes, while often inhospitable to natives of the Primes, is often perfectly suited to fiends since these planes are each individually infinite in size and fiends are well suited to their environment (they speak the native tongue and are immune to the average environmental threats, and natives don't freak out when they see them). It can't be an issue of new lands to conquer, or even new innocents to torture, as the Lower Planes are filled with both, and in infinite abundance. So why do powerful nasties want into the Prime? The following rules are changes to the D\&D cosmology, and they clear up the role of outsiders in the affairs in the realms so that more logical and fun adventuring can be had for players.

\subsubsection{The Prime is Better Than Cancun}
Prime Material Planes have one unique trait in all the universe: once in a Prime, you can't be summoned or called. For fiends, this means that they are no longer subject to the hierarchies of whatever place they hang their hat. For a fiend whose True Name is being passed around like a trading card, this is a huge thing: the Prime becomes a place where he can finally determine his own destiny, and no longer be a (potential) slave to the whims of mortals or his fiendish superiors. Fiends who are plotting coups in their own realm want to be able to get to Prime so that they are outside of the authority structure of their own race, and can lay low and build up their forces for a triumphant return to their particular Lower Plane. In this way, their superiors can't summon them and put them to the question in order to catch wind of their plans.

\subsubsection{You Can Get Room Service}
The second most important aspect of the Primes is that calling spells only work from the Prime. While regular summoning spells can call certain individuals, Conjuration magic of the [calling] subschool only works while in the Prime Material Plane. This means that beings that want to abuse calling magic to build armies can only do so while in a Prime. This particular rule clears up silliness like demons binding angels and forcing them to fight in Hell, or otherwise serve, which the current rules allow.

\subsubsection{Better Service for VIPs}
Natives of the Primes also hold a special place in the universe: they can't be summoned or called. This is actually a pretty big deal, since this means that natives of the Primes are the premier agents in the politics of the planes. Not only can be they summon or call natives of the planes while on the Prime, but they alone are free from the Conjuration spells that enslave and bind together the Lower Planes. In addition, the basic spells of \spell{raise dead} and \spell{resurrection} only function on creatures native to the Prime. Other creatures can be restored to life (with revive outsider, for example), but it's comforting to know that absolutely any Cleric can restore one's life if she wants to -- and Prime Natives live with that comfort every day of their lives.

\subsection{Practical Demonology: Additional Rules for Summoning}
One of the most contentious parts of the D\&D ruleset involves the summoning and binding of Extraplanar beings. We all agree that we want demon summoning, but we can't agree on what we want it to do. Should they be mindless slaves, or should they be tricksy tricksters who will eat your face if given the slightest chance? How exactly do \spell{planar ally} and \spell{planar binding} work? Can you just intimidate an outsider, or do you need to bargain with them with fair trade? Below are some additional rules to flesh out the experience:

\subsubsection{The Deal}
Making a deal with a fiend is usually a DM's call. He decides just how much interference he wants a summoning spell to do with his adventure, then he lets the party offer trade or threats until they get what they want up to the limits he has set. For DMs who don't want to stop-rule this each time, here is a list of tasks you can ask of a creature called by summoning spells:

\subsubsection{Part 1: Differences between Summoning and Calling}
First, we must reiterate the difference between summoning and calling.

\listone
    \item Summoning brings a creature to your location that follows both the intent and letter of your orders, has no free will, and will not act willingly act against your interests. When this creature dies, it and any effect it created vanishes (unless that effect was an instant effect). This creature has knowledge, but no personality or history. In effect, it only exists while the spell lasts.
    \item Calling spells bring an actual creature to your location, ripped from whatever place in the universe it existed. If you know a creature's name (not its True Name, which we will discuss later, but a use-name that it answers to), you can call that individual, along with any equipment or treasure it is carrying, but otherwise you get a random individual of that race. It has a personality and feelings, and when the spell ends it is returned to its original location. In effect, this creature has a life, and if treated badly enough, it may seek out its summoner for revenge.
\end{list}

\subsubsection{Part 2: Choosing a Pawn}
D\&D rules are silent on the issue of the limits of calling magic. While spells with the [summoning] subtype have specific lists of creatures that they call, [calling] spells usually have no such limits (except for the planar ally spells that force the DM to choose a creature). A simple way to limit creatures called is to only allow a summoner to call creatures that he could reasonably know about, and this means a Knowledge check.

Force the player to make a Knowledge check each time he wants to summon a particular race of creature for the first time (in effect, the base creature in the Monster Manual or other source). If he fails that check, he may not attempt another check for that base creature until he gains at least one rank in the relevant skill. Once he can summon a base creature, he may summon a templated version of that creature with an additional Knowledge check (and if he fails that check, he may not attempt another check for that templated creature until he gains at least one rank in the relevant skill).

This check uses the same Knowledge skill that would be required to identify that creature. The following modifiers also apply to the DC of the check.

\begin{quote}
+15 A normal creature, but with the extraplanar subtype\\
+5 Per CR of racial templates applied to base creature\\
-5 Spent one day studying the dead body a creature of the same race and racial templates.\\
-10 Spent one week studying a living member of that race and racial templates\\
+10 Never seen an example of the creature.\\
-10 Detailed written description of appearance and powers (must be 100\% complete)\\
\end{quote}

*Creatures with class levels or versions of monsters advanced by HD count as unique creatures, and they cannot be called without their use-name.

A player is responsible for recording each monster that he can call, and the ones he has failed to call. Once he has made a check for a particular combination of race and templates, he does not need to do so again.

Here is an example:


      Morgothazan the Dark casts \spell{lesser planar binding}, and he would like to call a Small Fire Elemental. To identify such a creature, he would need a Knowledge (the planes) check of 10 + the HD of a Small Fire Elemental creature, which is 2, meaning he needs a 12 to identify and call a Small Fire Elemental. As a 9th level wizard with a +14 modifier in Knowledge(the Planes), he automatically succeeds.\\
      The next day, he decides that he wants to call a Half Fiend Small Fire Elemental. He has never seen such a creature, but he knows that it must exist somewhere in the planes. His base DC is 12, plus another +10 for never having seen this oddity, and another +5 for the additional CR added to it, bringing his DC to 27. His modifier is +14, and he rolls a 12, meaning his gets a 26. Until he raises his Knowledge planes skill, he can't call a Small Fire Elemental modified by the Half-Fiend template.\\
      Several weeks later, Morgothazan the Dark wants to conjure a Half-fiendish Earth elemental. He already knows how to conjure a Small Earth Elemental, and he has actually fought and defeated a dead Half-fiendish Small Earth elemental. His base DC is 12, plus another +5 for the template, bringing his DC to 17. He rolls a 5, and he can call this monster.\\
      Emboldened by his success, he wants to be able to call a Half-Celestial Half-fiend Small Fire Elemental, but he remembers that he cannot (he can't conjure a Half-Fiendish Small Fire Elemental, so a Half-Celestial Half-fiend Small Fire Elemental is not possible). Instead, he tries the same templates on a Small Earth Elemental, as he has a detailed description of such a creature and he has had success with Half-fiend Small Earth Elementals. His base DC is 12, and his detailed description (-10) offsets the fact that he has never seen this creature(+10). Then an additional +10 is added for the CR increase from the two templates, making his final DC 22. He rolls a 10 and succeeds!

\subsubsection{Part 3: Services!}
When you cast a calling spell, you are bargaining for a single service. While normal bargaining could get you more complex arrangements, conjuring magic that calls real creatures can only force compliance to single services. For example, while a greater planar binding spell can bring a Pit Fiend to the Prime, the spell can only force to creature to obey the agreement set for a single service. Any additional services would not be guaranteed by the magic of the spell, and the Pit Fiend would keep or break any agreements as normal for that creature.

Within the limits of the single service, a called creature can do whatever it wants. A genie ordered to guard a room is under no compulsion to use its create food and water ability for allowed occupants of that room, and it may choose whether to converse, sit or stand, eat, or do any other act that does not interfere with its task. Clever conjurers often set tasks with exceptions in them like ''kill my enemies in Redstone Castle'', knowing that if they didn't define ''enemies'' and instead said ''kill everyone in the Redstone Castle,'' the called creature would be free to attack the conjurer if he entered Redstone Castle.

Called creatures will not agree to any services that are suicidal, self-destructive (like submitting to mind-control magic), or involve permanent self-sacrifice (like expending XP). They will also not agree to tasks that are impossible, or tasks that are so open-ended that could easily result in the creature's destruction.\\

\featnamelist{Things you can ask a creature to do:}
\listone
    \item Participate in a single battle
    \item Use a single use of one of its own abilities.
    \item Seek out an individual and either kill them or bring them to the summoner.
    \item Guard a spot for as long as the summoning spell lasts.
    \item Use a magic item
    \item Provide the results of one skill check
    \item Perform one task that does not involve any danger (like delivering a message by a safe route, survey a safe land, or dig a hole in an uncontested piece of land, etc)
    \item Offer their use-name$^*$
    \item Surrender personal treasure$^*$
\end{list}
\vspace*{8pt}
{$^*$ \small Requires a successful Intimidate check.}\\

\featnamelist{Things that creatures will do for free (not services):}
\listone
    \item Wait in a safe place in order to perform a service.
    \item Discuss the services they are willing to perform, and payment for those services.
    \item Exclude individuals from services (''kill anyone who enters except me'', ''tell me about everyone you saw in the tunnel except the sorceress'')
\end{list}
\vspace*{8pt}

\featnamelist{Some things demons won't do, even under pain of death or destruction:}
\listone
    \item Surrender their true name
    \item Voluntarily fail a save vs. an effect that would enslave or kill the demon
    \item Agree to unlimited service for a time period (for example, ''Do my bidding for a week.'')
    \item Guard an individual for a time period.
    \item Agree to not act in a situation (for example, they will not agree to not act while someone builds a prison around them).
    \item Wait in an obviously dangerous place (''just wait in front of that army of archons, and shoot the first one'').
    \item Perform an act that would violate its alignment or code of conduct.
\end{list}

\subsubsection{Part 4: Closing the Deal}
Once you have agreed on services to be performed, it is necessary to convince a creature so serve. Many spells simply bring a creature and enforce any agreement, they do not actually create an agreement.\\

To make an agreement, there are some things that must happen first:

\begin{enumerate}\itemspace
\item The Conjurer must be able to communicate with the creature. This means that the creature must be capable of communication (Int 3 or better) and they must have a form of communication (shared language, telepathy, \spell{tongues}, etc).

\item The Conjurer must successfully convince the creature.

\item The Conjurer must pay for services (if necessary).
\end{enumerate}

The initial attitude of a creature is Indifferent, unless the conjurer has an opposed alignment (good and evil, law vs. chaos) in which case they are Hostile.

To make an agreement, a successful Diplomacy check is required, and the attitude of the creature must be raised to at least Friendly. Once raised to Friendly, the creature performs its task as agreed and leaves when the task is completed, or when the spell's duration ends. A bribe of treasure equal to the amount of treasure an encounter equal to the creature's CR would earn is necessary to pay for these services. A Helpful check halfs this amount of treasure. A failed check means that the creature is not convinced, and a new check can be made the next day. A treasure of four time normal value automatically secures the creatures trust (Friendly result with no check). Note: \spell{planar ally} spells call a Friendly creature, and only the treasure need be paid.

Intimidate can be used as well, and this use of the skill can negate the need to pay for services, but earns the enmity of the creature. When the task is ended, but while the spell's duration lasts, the creature may return home, ending the spell, but also has the option of seeking out the conjurer and attempting to harm him or foil his plans. When the duration of the spell ends, the creature is not returned home. This creature may choose at some later date to seek revenge on the conjurer.

Bluff can also be used in place of Diplomacy in order to make the creature believe that items being offered are real treasure (when they might be worth less, or actually worthless). A successful result means that the creature accepts the offering and performs as if Diplomacy had made the creature Friendly. If the creature discovers during the course of the service that the treasure is not real, the binding magic fails and the creature is no longer forced to perform the service, and its attitude becomes Hostile. While the spell lasts, the creature may return home once as a free action, ending the spell.

\subsubsection{Part 5: The Business of Serving:}

One bound and a deal is made, the creature obeys according to pact made. Should the spell be ended, the creature is under no compulsion to obey the agreement (though some will out of fear or duty). Also, should the creature be put into a situation where the service cannot be competed (the person to be captured is killed by someone else, or the creature is forced to return to its home plane, for example), the service ends, and the creatures stays or returns as normal.

If the summoner betrays the creature by attacking it, stealing its treasure, or doing some other harm, the spell ends and the creature may return home or stay to seek its revenge.

\subsubsection{Part 6: Appendix: True Names and Use-names}
True Names are names of special power, and most creatures don't even know their True Name, or even how to get it. Special skills and some spells and effects can unravel a True Name, but the most common way to learn a True Name is for a powerful spellcaster to trade that knowledge to another creature for some treasure, favor, or True Name of equal power. Merely knowing a True Name is enough to grant power, since speaking the extraordinarily difficult word is a magical process that is unnecessary for most summonings (True Naming magic is a separate art from divine and arcane spellcasting, and is frankly not powerful enough for most would-be summoners). The feat Broker of the Infernal is one way of using True Names without learning the True Name skill or brand of magic.)

Use-Names, on the other hand, are far simpler. If you have seen a creature's true form and you know a name that it answers to, you can use calling magic to summon it.

\subsection{Weapon Proficiencies? You've got to be kidding me!}
The thing where being an Outsider automatically gives you proficiencies in all martial weapons is extremely dumb. There are substantial limits to the ''types as classes'' rules, and when we come to weapon proficiencies, we know that's it. An Erinyes should be proficient with a longbow or a whip, but a Howler should not. Honestly, the outsider type is so extremely varied that any rules you applied to the entire Outsider type would certainly cause more problems than they could fix. You are better off using no rules at all than the listed rules in the Monster Manual for weapon and armor proficiencies.

When you \emph{ad hoc} things and attempt to play by common sense rather than the wording in a book, you leave yourself open for horrible arguments because I am pretty sure my gut tells me different things than your gut tells you. But that's still better than getting into the arguments about how high level alienists and yeth hounds can use glaives. Without rewriting the entirety of every single monster book, this is a problem that actually has no resolution -- but it's also a problem that can usually be ignored. Don't give players any special weapon proficiencies for changing their type and generally assume that monsters are proficient with whatever weapons that they happen to be holding. It's not fair, it's not consistent, but at least it's not stupid.
