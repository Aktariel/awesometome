
\section{Advanced Combat}

Advanced combat is a somewhat disingenuous name, as it implies that these rules are more complex than the ''basic" rules found in the PHB. In truth, some of them are and some are not. Mostly, we look at these rules as a revision of the existing rules to make them more useful to players and DMs. In part that means taking incomprehensible portions of the combat system (grapple, for example), and cutting them down into discreet actions people can actually use without a half-hour argument. Partly this means taking basic combat actions and making sure that they have a valid purpose at all levels of play.

\subsection{Base Attack Bonus and Combat Maneuvers}

If you looked at the classes in the PHB, you'd think that BAB actually meant something. Classes with good BABs are severely restricted in other areas, and they only get 1 or 2 more BAB every four levels as compared to full spellcasters. Clearly, having even a slight bulge in BAB is supposed to be a major advantage. But in the basic rules, it really isn't. The bonus that a Fighter gets to his BAB over a Wizard is actually smaller than the variance of having rolled well and having rolled poorly on one's attributes. There is no guaranty that an Elven Fighter is better with a bow than an Elven Wizard is at 1st or even 4th level. Even when the BAB starts to pull ahead, it does so very slowly. A net +1 to-hit is something that you seriously might never even notice if you rolled your dice in secret. A +1 to-hit means that out of 20 attacks, one attack that would have missed would hit instead. Which, compared to the difference in numbers of attacks that land between someone who rolls well and someone who rolls poorly during an adventure is vanishingly small.

So what we're doing is actually making BAB mean something. It's supposed to represent the amount of combat skill you have, so let's work with that. From now on, if you have more BAB than the target of your attacks, you are considered to ''Have the Edge'' on that attack. Combat Maneuvers will perform better when used by someone with the Edge. So while anyone can attempt to Disarm an opponent (provoking an attack of opportunity and dropping the weapon on the ground on a successful opposed attack roll), a character with the Edge can disarm better (provoking no attack of opportunity and sending the weapon flying in a direction of his choice). In this manner, a character who takes full BAB classes always has a fundamental advantage in combat over characters who do other things.

\subsection{Attacks of Opportunity}

As you may have noticed, we have put in a lot more mechanics that interact with Attacks of Opportunity. That's because we're also instituting the following change to the mechanics of AoOs:

  \listone \item If you have a Base Attack Bonus high enough to warrant gaining additional attacks, you also get additional Attacks of Opportunity.  So a character with a BAB of +6 can make 2 AoOs each round. A character with a BAB of +11 can make 3, and a character with +16 can make 4.\end{list}


\subsection{Bonus Attacks and BAB}

The bonus attacks that characters get for hitting a BAB of +11 or +16 are not good. I don't know what that was about, but I can only assume that it had to do with a fundamental lack of playtesting past level 10. Anyway, the penalty for taking a bonus attack in a Full Attack action should never rise above -5. So if you have a BAB of +17, your attack routine should look like this: +17/+12/+12/+12. Really.

\subsection{Attack Options}

Characters have a number of options when they attack their opponents.

\listone\hypertarget{combat:expertise}{}
\item \textbf{{Expertise}}\\
\emph{You leverage your combat skill into defense rather than offense.}\\
\shortability{Requirement:}{You must make an attack action and have a BAB of at least +1. You need not specifically attack an enemy.}
\shortability{Effect:}{Before making an attack roll, you may take an attack penalty of up to your BAB on this attack and all further attacks until your next turn, and gain an equal Dodge Bonus to AC. You may only use this option once per turn.}\\

\hypertarget{combat:powerattack}{}
\item\textbf{{Power Attack}}\\
\emph{You leverage your combat skill into devastating attacks at the expense of accuracy.}\\
\shortability{Requirement:}{You must make an attack action and have a BAB of at least +1.}
\shortability{Effect:}{Before making an attack roll, you may voluntarily take an attack penalty of up to your BAB, and inflict two times that amount in extra damage with that attack. You may take this option on any or all of your attacks if you wish.}
\end{list}

\subsection{Special Attack Actions}

All of the following maneuvers may be made in place of an attack. Any time a character is permitted an attack for any reason (including an attack of opportunity or the attack at the conclusion of a charge), they may make a special attack action instead.\\

\listone\hypertarget{combat:bullrush}{}
\item\textbf{{Bullrush}}\\ \small
If you have not moved your entire allotted distance this turn, you may attempt to push your opponent back as a melee attack. First, you move into your opponent's square (which probably provokes an attack of opportunity, see movement). Then you make an opposed size-modified strength check against a DC of 10 + the target's Strength modifier + the target's size modifier (you do not have to roll to hit). If you succeed, you push your opponent back 5 feet. If you succeed by more than 1, you may move your opponent back a single 5' square for every 2 points your check exceeds the DC.

\ability{Modifiers:} The Size Modifier to both the Bullrush check and the DC is +4 for every size larger than medium and -4 for every size smaller than medium.

\ability{Special:} The movement used during a Bullrush counts against your movement this turn. If you do not take a move or charge action this turn, you will normally be limited to five feet of movement. This movement does not provoke an attack of opportunity from you or the target, but is quite likely to provoke an attack of opportunity from any other creature standing nearby. During a bullrush, both characters provide cover for each other.

\smallskip\emph{\underline{Edge Option:} If you have the edge on your target, you do not provide cover for your opponent even if they are the same size as you. Further, you may move your opponent in a direction up to 45 degrees off from your initial approach, altering your own course to push them more than 5 feet if necessary. If you fail the initial strength check, you may choose which adjacent square you are pushed into.}\\

\hypertarget{combat:coupdegrace}{}
\normalsize\item\textbf{{Coup de Grace}}\\\small
You may attempt to slay an opponent outright if they are helpless. As a full-round action, you may automatically hit a helpless opponent in melee range. This attack is automatically a critical hit. This action provokes an attack of opportunity.

\ability{Interrupting a Coup de Grace:}{A character who suffers damage during the Coup de Grace must make a Concentration Check (DC 10 + Damage Inflicted) or the action is resolved as a normal attack.}

\smallskip\emph{\underline{Edge Option:} If you have the Edge on an opponent who threatens you during a Coup de Grace, you do not provoke an attack of opportunity from them.}\\

\hypertarget{combat:coveringfire}{}
\normalsize\item\textbf{{Covering Fire}}\\\small
You may use your ranged attacks to provide cover for your allies. Take an attack with your ranged weapon and roll a normal attack roll. Until the beginning of your next turn one of your allies may use the result of your attack roll as their Armor Class against one attack of opportunity.

\smallskip\emph{\underline{Edge Option:} If you have The Edge against an opponent whose attack of opportunity was negated by Covering Fire, your ranged weapon may hit them. Simply compare the attack roll to their armor class as if it was also a normal attack.}\\

\hypertarget{combat:disarm}{}
\normalsize\item\textbf{{Disarm}}\\\small
You may attempt to disarm your opponent with a melee attack. Disarm is a special attack action. Make an attack roll against an ''armor class" of 10 + the target's melee attack bonuses with the item in question. If you succeed, one weapon or held item is snatched out of your opponent's grasp. Failing a Disarm attempt provokes an attack of opportunity from the target. A disarmed item lands in a randomly determined square adjacent to the target.

\ability{Defending against a Disarm:}{An item held in two hands is harder to disarm, increasing the DC by +4. An item tied to one's body with a sword-wrap or locked gauntlet is much harder to disarm, increasing the DC by +8.}

\ability{Special:}{A Disarm may be used to attempt to remove a weapon that is presently being used in an attack against the disarmer even if the creature using the weapon is out of range or otherwise not threatened by the character. A Disarm (or any attack) is normally only usable during an attack against such creatures as an Attack of Opportunity or a Readied Action.}

\smallskip\emph{\underline{Edge Option:} If you have the Edge on your target, your Disarm attempt does not provoke an attack of opportunity, and you may choose which adjacent square your opponent's weapon or held item lands in. If you have a free hand, the item may end up in your possession instead.}\\

\hypertarget{combat:feint}{}
\normalsize\item\textbf{{Feint}}\\\small
By performing a distracting maneuver or fencing your opponent into a poor position, you may make an attack against them at their worst. You take an attack action to make a Bluff check with a DC of 10 + your opponent's Wisdom modifier + the higher of your opponent's BAB or ranks in Sense Motive. If you succeed, your opponent does not get their Dexterity Bonus to AC against the next attack you make against them (if it is within the next round).

\smallskip\emph{\underline{Edge Option:} If you have the Edge on your target and you successfully Feint, you may make an attack against that opponent this round as a Swift action.}\\

\hypertarget{combat:grapple}{}
\normalsize\item\textbf{{Grapple}}\\\small
Grapple is collectively 3 separate maneuvers that all fall under the super-heading of ''grappling". Any grapple attempt provokes an attack of opportunity unless your attack has the edge.

\listtwo\hypertarget{combat:grabon}{}
      \normalsize\item\textbf{{Grab On}}\\\small
      Sometimes, you want to attach yourself to a larger creature, getting inside their reach and then repeatedly stabbing them or simply weighing them down.  As an attack action you may attempt to grab on to an opponent.

      Grabbing on to an opponent provokes an attack of opportunity and requires a check with the same bonuses as a melee attack. The DC to grab on to an opponent is their Touch AC plus their BAB. If you have 5 ranks of Climb or Ride, you get a +2 synergy bonus on this maneuver for each skill.

      \ability{Holding on:}{Once you've attached yourself to your opponent, you go wherever they go. Move in to their space, and move where they do automatically (this movement does not provoke attacks of opportunity or count against your movement in any way). You may attack with any light or one handed weapon, and your opponent is denied his Dexterity bonus against you.}

      \ability{Being Held on to:}{If another creature has grabbed on to your character, their weight counts against your carrying capacity. If you're overloaded, you may be unable to move or even collapse until you shake your opponent off. You can attempt to attack a creature holding on to you, but your strength modifier is halved for such attacks and your attacks are at -4. You may attempt to shake your opponent off as an attack action by making a check with a bonus equal to your melee attack or Escape Artist and a DC of 10 + the greatest of your opponent's BAB, Climb Ranks, or Ride Ranks.}

      \smallskip\emph{\underline{Edge Options:} If you have the edge on an opponent when you grab them, they may not attack you at all once you have grabbed on to them. Further, grabbing on to an opponent does not provoke an attack of opportunity.}\\

     \hypertarget{combat:holddown}{}\hypertarget{combat:pin}{}
      \normalsize\item\textbf{{Hold Down}}\\\small
      Sometimes you want to pin an opponent to the ground. First, make a touch attack. Then, make a Grapple Check (BAB + Strength Modifier + Special Size Modifier) with a DC of 10 + Defender's Grapple Check Modifier. If you succeed, your opponent is pinned for one round. They can't move, and you may put ropes or manacles on them if you wish with an attack action. At the end of any turn you are pinning your opponent, you may inflict unarmed or constriction damage. With subsequent attack actions, you may attack with natural weapons or light weapons with no penalty.

      \ability{Escaping a Pin:}{If you're pinned you can attempt to fight back, but you're prone and you suffer an additional -4 penalty to attack the creature pinning you (generally a -8 total penalty to attack your attacker). You can get out with an attack action by making a Grapple or Escape Artist check with a DC of 10 + your opponent's Grapple Modifier.}

      \smallskip\emph{\underline{Edge Options:} If you're pinning an opponent and your attacks have the edge, your opponent cannot attack you or anyone else until they get free. Furthermore, if anyone else attacks them, they are considered helpless.}\\

     \hypertarget{combat:lift}{}
      \normalsize\item\textbf{{Lift}}\\\small
      Sometimes you want to put an opponent in your mouth or carry away a struggling princess. Make a touch attack and then make a Grapple Check with a DC equal to 10 + your opponent's Grapple modifier. If you succeed, your opponent is hefted into the air. You may move around freely while carrying your opponent (their weight counts against your limits of course). You may perform a coup de grace or swallow whole action on a character you have lifted, but doing so ends the lift whether it succeeds or fails.

      \ability{Escaping a Lift:}{When you've been lifted, you cannot move under your own power, but you can continue to attack. Attacks against the creature which has lifted you are at a -4 penalty. You can also attempt to escape with an attack action by making a Grapple or Escape Artist check with a DC of 10 + your opponent's Grapple Modifier.}

      \smallskip\emph{\underline{Edge Options:} If you have the edge on an opponent you have lifted, they may not attack you or anyone else until they escape.}\\
\end{list}

\hypertarget{combat:trip}{}
\normalsize\item\textbf{{Trip}}\\\small
As an attack action, you may attempt to knock an opponent prone. Make a touch attack, and if you succeed make a Strength + BAB check against a DC of 10 + your opponent's Strength + BAB or Balance modifier (whichever is greater). Success leaves your opponent prone. Failure provokes an attack of opportunity.

\ability{Modifiers:}{The DC to trip an opponent who has four legs or is otherwise inherently stabile is increased by 4. Radially symmetrical creatures like Oozes cannot be tripped at all.}

\smallskip\emph{\underline{Edge Option:} If you have the edge on your target, you do not provoke an attack of opportunity if your trip attempt fails, but your target provokes an attack of opportunity from you if your trip succeeds.}

\end{list}
