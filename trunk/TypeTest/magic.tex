\section{The Maginomicon}
\vspace*{-10pt}
\quot{''With great powers come laser eyebeams."}

\subsection{Easter Egg Class Features: Artifact Swords and Powergloves}

Here's a secret: some characters really can't even play the game at high level. But they do anyway, all the time. Sometimes the players never even realize that their character has no intrinsic capability to play the game at the level he's competing at. And that's because of two things: DMs control the Monsters, and DMs control the Treasure. It is our hope that the Monks and Assassins in this document will be able to hold their own without needing to get Power Gloves that act as magic weapons for their natural weapons or anything else really cheesy like that. That being said, we still haven't covered everything:

\listone
    \item Rogues still need a magical object that allows them to use the Hide skill by about level 9.
    \item Fighters still need their artifact swords at level 10.
    \item Bards still need some completely arbitrary magic item that summons a monster or something so that they can contribute at all past level 12.
    \item Mounted characters need a magical beast or dragon to ride around on by level 7.
\end{list}


And so on. As this series continues, we will attempt to solve some of these outstanding issues.

\subsection{Your Money is No Good Here}

As described in the Economicon, you can't just throw a walrus' weight in gold on the table and get powerful artifacts in return. You can get powerful magical items in exchange for rare planar currency, but you can only do that in a few planar locations. From the standpoint of the DM this is very convenient, because it means that you can hand out all the opal statues you want without worrying that the players are going to pool it all and get some totally hardcore magic items that will undermine everything. At the same time, it means that you can hand out planar currency and know for a fact that it's going to be used for powerful magical items.

\subsection{It's Not Stupid, It's Advanced!}

The 15,000 gp limit for purchasing equipment can be pretty limiting, but the game works much better once you realize that it's there. Still, while characters can't go out and buy a +4 sword with pieces of gold (all 647 pounds of it), they can purchase a +1 flaming or ghost touch sword with chunks of non-magic metal. You can also pump those up with greater magic weapon to be something level appropriate. This offends some people, but it really is part of the way the D\&D magic item economy is supposed to work. People are supposed to be fighting with weapons that are level appropriate, and people are supposed to be purchasing new weapons for different occasions, and there are not supposed to be stores with racks of powerful swords that would be level appropriate for 12th level characters stacked up in various setups on shelves.

\listone
    \itemability{Bonus Rule:}{The game actually works better if every character of 6th level or higher simply has \spell{greater magic weapon} 1/day as a spell-like ability. Caster level is equal to character level. Try it, it's amazing how many problems are solved by this relatively simple change.}
\end{list}


\subsection{Material Components: A Joke Gone Way Out of Hand}

Material components are a joke. I'm not saying that they are metaphorically a joke in that they don't act as a consistent or adequate limiting factor to spellcasting, I mean that they are actually a joke. Material components are supposed to be ''ha ha" funny. The fact that even after having this brought to your attention, you still aren't laughing, indicates that this is a failed attempt at humor. Most material components are based on technological gags, when you cast scrying you are literally supposed to grab yourself a ''specially treated" mirror, some wire, and some lemons -- which is to say that you make a TV set to watch your target on and then power it with an archaic battery. When you cast see invisibility you literally blow talc all over the place -- which of course reveals invisible foes. Casting lightning bolt requires you to generate a static charge with an amber rod and some fur, tongues requires that you build a little Tower of Babel, and of course fireball requires that you whip up some actual gunpowder. Get it? You're making the effects MacGuyver style and then claiming that it's ''magic" after the fact. Are you laughing yet?

Of course not, because that joke is incredibly lame and there's no way for it to hold your attention for several months of a continuous campaign.

\subsection{Some Spells Don't Work}

Many spells are underwhelming for their level or have mechanics that are hard to explain. But first and foremost of all the spells that are bad for the game is Polymorph. That spell is integral to any fantasy setting, but people haven't made it work in 3rd edition. Mostly, this is because people keep writing it long instead of short. Remember, if you can't explain an effect in 2 minutes, everyone else is already confused.

\subsubsection{Polymorph Version 1: Character Replacement}

If you take part of your character -- any part of your character -- and part of a monster from one of the many monster books in D\&D, and you put them together into a single Voltron-like body, you have broken D\&D. That should be obvious, but since we are over six years into the ridiculous circus that is polymorph in 3rd edition Dungeons and Dragons, apparently it isn't. If it is important to you that you be allowed to dumpster dive through the monster books and find an appropriate to transform into, it is important to D\&D that absolutely no part of your character be mixed and matched during that period. If you want to truly become a monster, you have to actually become that monster. Not ''the monster with all my spell effects running," not ''the monster with my formidable mental attributes. No. You need to become the monster exactly as it appears in the monster book or there's no chance of you getting a balanced result. Some people are going to end up as mediocre monsters with carry-over abilities that happen to synergize well and become tremendously powerful while other people are just as unbalanced in the other direction when they find that drawbacks of their character are carried over and overwrite the abilities of a monster that are supposed to make them any good at all.

And this isn't just hyperbole or doomsday predictions, this is established fact. We've all played with some of the multitude of different versions of Polymorph errata and ''fixes," and the abject horror caused by every single iteration. The idea doesn't work. If you're going to replace any part of the character, you have to replace it all. So here's a version of polymorph that won't make us cry. This ain't rocket science, it just takes a little bit of discipline:

\begin{quote}
\featname{Polymorph Self}
\begin{small}
\shortability{Transmutation}{}
\shortability{Level:}{Sor/Wiz 4}
\shortability{Components:}{V, S}
\shortability{Casting Time:}{1 Standard Action}
\shortability{Range:}{Self}
\shortability{Duration:}{10 minutes/level (D)}
\shortability{Saving Throw:}{Fortitude Negates (Harmless)}
\shortability{Spell Resistance:}{No}
\end{small}
\quot{''A Turtle am I? Let's see how Turtlike I\ldots\  CAN\ldots\  BE!'' And with that, the mage was a giant turtle.}

You vanish and a monster of your choice appears in your place. The creature shares your alignment, personality and goals, and will continue to act as you would within the limits of its intelligence and abilities. The creature must be at least 3 CR less than your character level, may not have the incorporeal or swarm subtype, and is unexceptional for its type. If the monster is killed, the spell is ended. When the spell ends, the monster vanishes and you appear where the monster was with an amount of lethal, nonlethal, and ability damage on you equal to the amount the monster had suffered when the spell ended (this means that if the spell ended because the monster was slain and the monster had an equal or greater number of hit points as you, you may well be dead when you appear).
\end{quote}

\begin{quote}
\featname{Polymorph Other}
\begin{small}
\shortability{Transmutation}{}
\shortability{Level:}{Sor/Wiz 4}
\shortability{Components:}{V, S}
\shortability{Casting Time:}{1 Standard Action}
\shortability{Range:}{Medium}
\shortability{Target:}{One Creature}
\shortability{Duration:}{Permanent (D)}
\shortability{Saving Throw:}{Fortitude Negates}
\shortability{Spell Resistance:}{Yes}
\end{small}
\quot{The witch snarled at the trespasser and pointed her wand vindictively at him. A short incantation later left nothing but a pig in his place.}

Your target vanishes and a creature of your choice appears in its place. The creature shares the alignment, personality and goals of the target, and will continue to act as it would within the limits of its intelligence and abilities. The creature must be at least 5 CR less than your character level, may not have the incorporeal or swarm subtype, and is unexceptional for its type. If the creature is killed, the spell is ended. When the spell ends, the creature vanishes and the target appears where the creature was with an amount of lethal, nonlethal, and ability damage on it equal to the amount the creature had suffered when the spell ended (this means that if the spell ended because the creature was slain and the creature had an equal or greater number of hit points as the original target, it may well be dead when it appears).

\end{quote}


\begin{quote}
\featname{Mass Polymorph}
\begin{small}
\shortability{Transmutation}{}
\shortability{Level:}{Sor/Wiz 7}
\shortability{Components:}{V, S}
\shortability{Casting Time:}{1 Standard Action}
\shortability{Range:}{Medium}
\shortability{Target:}{Any number of creatures within a 20' radius}
\shortability{Duration:}{Permanent (D)}
\shortability{Saving Throw:}{Fortitude Negates}
\shortability{Spell Resistance:}{Yes}
\end{small}
\quot{The crowd looked uncomfortable. They had weapons and were brandishing them in a fashion quite menacing. But the magician was laughing, and that really put a damper on the mood of the entire event. They started to regain their composure and again advance upon him. He snorted and muttered an incantation, and something about swine \ldots}

Each target vanishes and creature of your choice appears in its place. The creatures share the alignment, personality and goals of the targets, and will continue to act as they would within the limits of their intelligence and abilities. The creatures must be at least 7 CR less than your character level, need not be the same for all targets, none may have the incorporeal or swarm subtype, and all are unexceptional for their type. If a creature is killed, the spell is ended for that target only. When the spell ends, the creatures vanish and the targets appear where the creatures were with an amount of lethal, nonlethal, and ability damage on it equal to the amount the creature had suffered when the spell ended (this means that if the spell ended for a target because the creature was slain and the creature had an equal or greater number of hit points as the original target, it may well be dead when it appears).

\end{quote}

\subsubsection{Polymorph Version 2: Fixed Forms}

The other version is one where transforming leaves you essentially yourself, only with a new hairdo and possibly some bonuses. In this case, you keep everything about yourself and simply get a disguise and some advantages consistent with a buff spell. All of the ''Whatever-Form" spells don't stack with multiple castings or even with each other, because they are considered to be ''one spell makes another spell irrelevant" for purposes of spell stacking.


\begin{quote}\featname{Human Form}
\begin{small}
\shortability{Transmutation}{}
\shortability{Level:}{Brd 1; Sor/Wiz 2}
\shortability{Components:}{V, S}
\shortability{Casting Time:}{1 Standard Action}
\shortability{Range:}{Touch}
\shortability{Target:}{One Willing Creature}
\shortability{Duration:}{10 minutes/level}
\shortability{Saving Throw:}{Fortitude Negates (Harmless)}
\shortability{Spell Resistance:}{Yes}
\end{small}
\quot{The man looked at the fallen prince and smiled. He whispered some eldritch words, and then there were two princes. One living, and one dead. The living prince smiled.}

The target assumes the appearance of a specific individual of medium size or smaller, or of a generic member of a humanoid race. The target is effectively disguised, and gains a +10 bonus on Disguise checks made to impersonate the genuine article. The target suffers no penalties to Disguise for assuming the visage of a different race or sex.
\end{quote}


\begin{quote}
\featname{Lycanthropy}
\begin{small}
\shortability{Transmutation}{}
\shortability{Level:}{Sor/Wiz 3}
\shortability{Components:}{V, S}
\shortability{Casting Time:}{1 Standard Action}
\shortability{Range:}{Touch}
\shortability{Target:}{One Willing Creature}
\shortability{Duration:}{10 minutes/level}
\shortability{Saving Throw:}{Fortitude Negates (Harmless)}
\shortability{Spell Resistance:}{Yes}
\end{small}
\quot{The shaman howled in rage and transformed into a wolverine.}

The target assumes the appearance of a specific or generic animal or magical beast of small, medium, or large size. The target is effectively disguised, and gains a +10 bonus on Disguise checks made to impersonate the genuine article. The target suffers no penalties to Disguise for assuming the visage of a different race or sex. The new form is unable to use normal equipment (all carried or worn items meld into the new form when the spell takes effect), and has whatever natural weapons the caster desires (to a maximum of 1 natural weapon per four levels). These natural weapons inflict an amount of damage appropriate for a magical beast of the new form's size. Any equipment the character had is subsumed into their new form.
\listone
    \item Small, Flying:90' flight speed (good), +4 Dex, -4 strength
    \item Small, Land:+2 Dex
    \item Small, Swimming:60' swim speed
    \item Medium, Flying:60' flight speed (good), +2 Dex
    \item Medium, Land:40' land speed, +2 Strength, +2 Natural Armor
    \item Medium, Swimming: 60' swim speed, +2 Strength, +2 Natural Armor
    \item Large, Flying: 90' flight speed (average), +2 Dex, +4 strength, +1 Natural Armor
    \item Large, Land: +6 Strength, +5 Natural Armor
    \item Large, Swimming: 60' swim speed, +6 Strength, +4 Natural Armor
\end{list}
\end{quote}


\begin{quote}
\featname{Monstrous Form}
\begin{small}
\shortability{Transmutation}{}
\shortability{Level:}{Sor/Wiz 4}
\shortability{Components:}{V, S}
\shortability{Casting Time:}{1 Standard Action}
\shortability{Range:}{Touch}
\shortability{Target:}{One Willing Creature}
\shortability{Duration:}{10 minutes/level (D)}
\shortability{Saving Throw:}{Fortitude Negates (Harmless)}
\shortability{Spell Resistance:}{Yes}
\end{small}
\quot{With a sweep of your cloak you become a creature of nightmare.}

The target assumes a horrific and monstrous countenance of a monster of Medium, Large, or Huge Size. The basic structure can look like pretty much anything, and the descriptions are just guidelines. All of the character's equipment melds into his new form. The character no longer has the ability to use equipment, but has a number of natural weapons appropriate to the new form:

\listone
    \item Yeth Hound (Medium):50' speed, +4 Str, +4 Dex, Bite, Improved Trip
    \item Displacer Beast (Large): +8 Str, +2 Dex, +5 Natural Armor, 1 Primary Bite and 2 secondary Tentacle Whips, Concealment.
    \item Monstrous Spider (Large): 30' Climb Speed, +8 Str, +8 Natural Armor Bonus, 1 natural weapon Bite, Poison (1d6 Con/ 1d6 Con)
    \item Chuul (Large): 60' Swim Speed, +8 Str, +6 Natural Armor, 2 Primary Pinchers, character gains the [Aquatic] Subtype.
    \item Bulette (Large): 20' Burrow Speed, +8 Str, +10 Natural Armor
    \item Manticore (Large): 60' Fly Speed (Average) +8 Str, +6 Natural Armor, 2 natural weapon Claws, 2 natural weapon ranged spikes attacks (1d8 + Str, 19-20 crit, 20' range increment)
    \item Giant Serpent (Huge): +14 Str, +10 Natural Armor Bonus, 1 natural weapon bite, Poison (1d6 Dex damage/1d6 Dex damage), Improved Grab.
\end{list}
\end{quote}


\begin{quote}\featname{Fiend Form}
\begin{small}
\shortability{Transmutation}{}
\shortability{Level:}{Sor/Wiz 5}
\shortability{Components:}{V, S}
\shortability{Casting Time:}{1 Standard Action}
\shortability{Range:}{Touch}
\shortability{Target:}{One Willing Creature}
\shortability{Duration:}{10 minutes/level (D)}
\shortability{Saving Throw:}{Fortitude Negates (Harmless)}
\shortability{Spell Resistance:}{Yes}
\end{small}
\quot{With a foul guttural utterance and a rude gesture, the wizard transforms into a fiend from the lower planes.}

The target assumes the appearance of a specific individual of medium size or smaller, or of a generic member of a fiendish race. The target is effectively disguised, and gains a +10 bonus on Disguise checks made to impersonate the genuine article. The target suffers no penalties to Disguise for assuming the visage of a different race or sex. While in Fiendish form, the target gains two bonus [Fiend] feats of your choice that it would meet the requirements for if it was actually a member of a fiendish race, and gains access to a sphere of your choice. In order to use a spell-like ability from the sphere, the target must expend one spell-slot or prepared spell of an equal or greater spell-level, but there is no other limit to how many times the spell-like abilities can be used. Rules for [Fiend] feats and spheres may be found in the Tome of Fiends.
\end{quote}


\begin{quote}\featname{Dragon Form}
\begin{small}
\shortability{Transmutation}{}
\shortability{Level:}{Sor/Wiz 6}
\shortability{Components:}{V, S}
\shortability{Casting Time:}{1 Standard Action}
\shortability{Range:}{Touch}
\shortability{Target:}{One Willing Creature}
\shortability{Duration:}{10 minutes/level (D)}
\shortability{Saving Throw:}{Fortitude Negates (Harmless)}
\shortability{Spell Resistance:}{Yes}
\end{small}
\quot{The final incantations are completed and you transform into a dragon.}

The target character assumes the form of a huge dragon. The character gets a +14 Strength bonus and a -4 Dexterity penalty. The character gains a +18 Natural Armor Bonus. The character gains immunity to one energy type (which must be Acid, Cold, Electricity, Fire, or Poison), and a breath weapon that inflicts 1d6 per level of the same type of damage. Using the breath weapon is a supernatural ability that requires a standard action and may only be used at most once every 1d4+1 rounds. The character has a flight speed of 120' with poor maneuverability. A character in Dragon Form has three natural attacks: a primary Bite and two secondary claws. Worn equipment is subsumed into the new draconic form.
\end{quote}

%\end{small}
%\end{multicols}

\subsection{Some Effects Don't work}

\subsubsection{Stacking Spell Resistance}

Spell Resistance does stack, but it does so in a really weird way that the authors have never actually taken the time to explain. SR is a DC for a level check, and that means that it is actually calculated as the number people are supposed to roll to penetrate your SR plus your CR. A SR of 15 on a CR 1 monster is awesome (it means that people of your level are supposed to roll a 14 to penetrate your SR), while a SR of 15 for a CR 13 monster is a joke (it means that enemies fail to penetrate your SR only on a natural 1). When you have Spell Resistance, and your CR goes up, your Spell Resistance also goes up. An Imp with a CR of 2 and a SR of 6 who takes enough levels of Wizard to gain a CR (2 levels as it happens) gains 1 SR as well and is SR 7.

But what happens if more than one source gives you SR? Well, it still stacks, it just does so in the aforementioned really weird way. First, you take your highest SR, then you start adding very small numbers to it based on what your other sources of SR would give you. If a secondary source of SR is less than 6 + your CR, having it increases your SR by +1. If a secondary source of SR is 6 + your CR or more, but less than 11 + our CR, your primary SR increases by +2. If a secondary source of SR is between 11 + CR and 15 + CR, it increases the primary SR by +3. And finally, a secondary source of SR that is 16 + CR or more adds +4 to the primary SR.

It would be nice if the basic rules ever explained that, but they don't. It doesn't come up all that often, but Drow Monks, for example, don't end up with SR in the high 30s at mid level. Their SR is actually just moderately impressive.

\subsubsection{Hiding in 3.5 D\&D is Dumb}

OK, we all know that it makes us feel kind of bad when the Rogue sneaks up on people and stabs them in the face without them ever seeing who did it. But you know what? People totally do that crap all the time. It's not even an uncommon occurrence, and there's really no cause to get excited about. The 3.5 rules for hiding, where you need cover or concealment to hide, are retarded. That makes Rogues run around with tower shields so that they can hide themselves and their equipment behind the cover of the tower shield (including the tower shield itself, which makes my brain hurt). Yes, you can totally hide when there are no intervening objects between you and the victim. It's called ''sneaking up behind people" and in a game with no facing it's handled with a hide check opposed by spot.\\

If you attempt to hide in a combat setting, you are under a number of restrictions:
\listone
    \item A character who has been attacked automatically can guess what square you are in. You may retain your invisibility, but that's just Full Concealment, and they could very plausibly hit you.
    \item There is a -20 penalty to Hide for attempting to fight while hidden. The distance penalties on Spot are pretty amazing, but most people can't hide at a -20 penalty.
    \item Once they see you, they see you. If an opponent successfully spots you even once (and they get to try every round while in combat), they just plain see you until you manage to get all the way out of their field of view (generally requiring you to leave the scene or make bluff checks or something).
    \item Spot Bonuses can get quite large. A spotter who knows what he's looking for gets a +4 bonus, and a spotter who is extremely familiar with the target gets a +10 bonus -- these bonuses are weirdly listed under the Disguise skill, but they still apply (so if someone says ''There's a halfling Ninja over there!" every other Guard gets a +4 bonus).
\end{list}

\vspace*{8pt}

But you can do it. Hiding in combat is hard, but it's a thing that powerful characters may be able to do against some opponents. Some of the D\&D authors have an outdated idea that Rogues should be forced to ''hide in shadows" or something. But this is D\&D, and most enemies have Darkvision. There are no shadows. Attempting to force Rogues to hide only in areas that they could plausibly hide in if a suspicious person was looking right at them and knew what they were looking for is incredibly cruel. In any kind of stressful situation that isn't an accurate picture of what is going on.

\subsubsection{Clerics and Druids get Broken with Supplements}

Sometimes it seems that WotC authors can't even write a supplement without writing a new Cleric spell. Unfortunately, that drives Clerics straight into crazy town because they actually know every spell on their list. So if someone writes 5 new cleric spells for a minor adventure, that's five new options that every Cleric player has for no reason. That has to stop.

Characters like Clerics and Druids are, with few exceptions (*cough*divine power*cough*) pretty much OK with the spells in the Player's Handbook. It's only when we mix in all the crazy options in additional sources that they go over-the-top. It is our contention, then, that such characters continue to be allowed access to all spells in the PHB -- and to only get one bonus known spell from other sourcebooks each level (choose wisely). In this manner, the Clerics and druids of the world will end up having a couple of specific gimmicks, and they won't all just be cookie-cutter copies of each other with an answer for every occasion. Thereafter, such characters could potentially find magical writings with new spells in their discipline that they could learn and use in the same manner as a Wizard. I have nothing against a Druid finding a copy of some ancient text that allows her to call upon the legendary bloodsnow, but it's pretty ridiculous the way in the current rules every Druid can get up one day and decide to have an explosion of bloodsnow.
