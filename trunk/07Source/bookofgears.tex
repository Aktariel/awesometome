\section{Introduction}

Cooperative Storytelling is essentially all about artifice. The stories we create are \textit{created}, the shared narrative is an illusion which fills our mind and pushes us forward. So it is no surprise that creating things within that narrative is so very contentious. Building a house in the game or creating an illusion in the story is a

An illusion is something that isn't real inside a story that isn't real. Forging a sword is creating something within a tale that is being created around it. These actions, while very integral to the source material upon which our cooperative storytelling games are based, are yet one more step removed from reality when contrasted with the old standards of pretending to be a knight who kills imaginary dragons to save fictitious princesses.

So it seems not at all surprising in retrospect that the rules we have used to represent the creation of stuff within the game world have historically been extremely unsatisfactory. Creating things takes time, which is a problematic concern in a game where time passages narratively. That means that the time a character spends nailing boards together for his dream house may be spent in a montage that ends in subtitles reading ``six months later" and it may happen interspersed with a rollicking adventure where seconds count and the hammering essentially never gets done.

The result has been that previous editions have attempted to put additional or alternate costs on crafting of all sorts. From Constitution points to years off your life to XP, D\&D has experimented with about a dozen different rubrics by which characters could trade one part of their character for more magic items. In almost all cases this allowed players to trade things they weren't using anyway for powerful artifacts that allowed them to conquer worlds, although in a few cases the flip side showed up made item creation so crappy that people seriously didn't do it at all. Needless to say, this has been unsatisfying, and it is our intention to help remedy these problems.

The rules presented here present a different take entirely. Creating magic items is something that takes only time, and adventures can be expected to be completed without ever doing it at all.

% Character Advancement
% FIXME: Maybe move "Playing the Game" to here.
\section{Character Advancement: Power and Wealth}
\vspace*{-8pt}
\quot{``Assuming that I make the use of most of our spells, I should be able to advance a circle of magic every week or so, which essentially means that the optimum solution to this difficulty to simply scare up minor tangential difficulties in the woods for two months so that I can go back in time and solve this problem retroactively.''}

While we're talking about magical items, we really have to talk about XP at the same time. And that's not just because the DMG asks us to pay small amounts of XP to create them. D\&D is based on two kinds of advancement: XP and GP. Both of them have failed, because we're actually playing a cooperative storytelling game and not Diablo multiplayer. We know that a high level guy can whack low level stuff again and again at virtually no risk, and that this can be repeated endlessly for levels. We know that people can take off downtime to just plain farm to get GP endlessly. Seriously, ``XP Grind" is extremely boring and players should not be exposed to it under any circumstances.

No one wants to hear about the time you threw a cloud kill into a Satyr tavern and then teleported home so that you could try out the new spells that appeared in your book because you just dinged to 10th level. That's a story that is dumb, and the current rules pretty much expect you to do it over and over again. If we're going to have a rational system for magic items, we can't have it work that way.

\subsection{XP: Beer Me}
\vspace*{-8pt}
\quot{``Boil an Anthill: Go Up One Level.''}

The rubrics for challenge and advancement as depicted in the DMG have to go. We've looked at them from every direction, and they don't work. At all. And no, I'm not talking about the classic problems like the variable difficulty inherent in fighting a giant scorpion (an interesting intellectual exercise for a 4th level horse archer or a brutal melee slugfest for a 14th level swordsman). That's a real problem, but we are talking about the basic structure of fighting monsters of increasing CR, getting increased piles of XP, and moving on with your life. That's got to end.

Here's why: according to the DMG you are supposed to face about 4 equal-level challenges per day of adventuring. Further, going by the XP chart, your 4-person party will go up a level every time you defeat 13.3 of those encounters -- which is less than 4 days worth of encounters according to the first idea. So if you adventure ``like you're supposed to" -- you'll go up 2 levels a week. And of course, if you encounter less than 4 enemies a day, spell-slot characters like Wizards and Druids are \textit{crazy} good. Essentially, this means that D\&D characters go from 1st level to 20th level in half the time as it takes to bring a pregnancy to term.

Indeed, D\&D society is essentially impossible. Not because Wizards are producing expensive items with their minds or because high level Clerics can raise the dead -- but because the character advancement posited in the DMG is \textit{so fast} that it is literally impossible for anyone to keep tabs on what the society even is. High level characters are the military, economic, and social powerbases of the world. And they apparently rise from \textit{nothing} in about 2\half months. That means that if a peasant goes home to plant his crops, then when he gets back to the city with his harvest in the fall the city will have seen the rise of a group of hearty adventurers who attempt to conquer the world and achieve godhood \textit{four times} while he's gone. The city will have been conquered by a horde of Dao and sucked into the Elemental Plane of Earth and then returned to the prime material as a group of escaped Dao slaves achieved their freedom and themselves became powerful plane hopping adventurers who graduated to the Epic landscape. Then a team of renegade soldiers from the Dao army will have run off into the countryside and survived in the Spider Woods long enough to return with the Spear of Ankhut to return the city to the Dao Sultan in exchange for a gravy train of concubines and \spell{wishes}. Then a squad of frustrated concubines will have turned on their masters and engaged in a web of intrigue culminating in the poisoning of the Dao Sultan with Barghest Bile and ultimately turned the city into a matriarchal magocracy run by ex-concubine sorceresses. So when the peasant returns with his harvest of wheat, he returns to a black edifice of magical stone done up in Arabian styles and bedecked with weaponry from Olympus that is all controlled by epically subtle and powerful wizards who are themselves the masters of a setting created from the fallout of the destruction of a setting that is itself the fallout of the destruction of a setting that was \textit{in turn} created out of the destruction of the setting that our peasant walked away from with a bag of grain come planting time last year.

And while purely intellectual exercises in a universe that is essentially a giant lava lamp of crazy can be interesting, satisfying storytelling is impossible. If the players can't make lasting impact, the game has no meaning. And if players are seriously going from 1st to 20th in a single season, lasting impact of any kind is absurd to even contemplate. It behooves players and DMs to come to a consensus about how they want their campaign to be structured. There is no single best way to handle character advancement in a cooperative storytelling game, and there are a lot of ways to really piss off the other players at the table if you aren't all on the same page to begin with.

\subsection{Reach for the Stars: Character Advancement}

All classic fantasy adventures take place in D\&D terms somewhere between 1st and 10th level. Seriously. Conan is like 3rd level, Theseus is about 3rd level too. Adventures for 13th level in literature of any kind are hard to come by and generally involve wearing capes or being a god. However, D\&D is not a game about modeling tales of legendary knights, skilled samurai, or barbarian chiefs -- it is a game about adventuring in the world of D\&D. And in D\&D, characters do become 20th level, at which point they either become honorary Olympians or join the Justice League. Within that context, character advancement should follow a few basic principles:

\listone
	\bolditem{Stagnant Characters are frustrating.}{That is, in a game which offers so much potential for advancement, it is frustrating to be in the position where you don't actually get to do any of it. Sure, in a game like Shadowrun there's no disappointment to be had from not being able to achieve godhood and in a game like Champions you don't need to advance your character at all to have a good time. But D\&D is a leveled system and not getting those levels makes us sad.}
	\bolditem{Advancement of Characters shouldn't destroy the setting.}{If you're playing a ``pirate game" then you shouldn't get to the point where there is no longer a purpose served in piracy as long as you still play that game. Furthermore, you shouldn't be adverse to downtime on the grounds that waiting a month or two for a storm to go by will leave your enemies driving air cars powered by t-rexes on bicycles.}
	\bolditem{Players should be able to play with their toys.}{Too often, a character will get a shiny new trick only to go up in level and have no further use for it long before he has had a chance to actually \textit{use} it. And that defeats the entire purpose of leveling up in the first place.}
	\bolditem{Characters should not be rewarded for doing stupid crap.}{Seriously. Your goal is to rescue the princess, so what should you do? \textit{Rescue the princess}, or run around the compound she's being held in punching out the baron's attack dogs? An army is heading for your city, should you sneak in and kill the enemy general or should you try to wrestle the army's horses one at a time?}	
\end{list}

This leads us to several conclusions of varying palatability:

\subsubsection{Wealth By Level Has Got to Go.}

This hurts a lot of people, but it's true. If you can turn a pile of silver into increases to your natural armor bonus, the setting is going to be destroyed. Quite literally, and with crowbars. Fantasy settings are filled with bridges made of opal and castles faced with blue ice that sty forever cold and stuff. This fantastic scenery is awesome, and it contributes to the feel of fantasy that should permeate the cooperative stories we tell within a D\&D game. If player character power is determined by ``wealth" in any directly measurable fashion, you can bank on PCs ripping all the expensive facing off the castles they conquer -- and then we all lose.

See, it's pragmatic and even sort of reasonable to rip the marble off the Great Pyramid at Giza and use it to build fancy houses in Cairo. But for all the future generations, it sucks. There really is a correlation here: if we don't allow people to trade blocks of marble for extra spells per day and more powerfully magical swords, then people will leave our pyramids alone. Otherwise, future generations will look at another unfaced ziggurat and wonder what wonders the ancient battlefields possessed before vandals came and destroyed our fantasy world.

\subsubsection{Encounter XP Has Got to Go.}

XP rewards are a form of incentive towards heroic behavior. The problem is that individual challenges don't make things more heroic, they just make things more time consuming. By parting out XP per \textit{encounter} rather than per \textit{quest} the game is actually discouraging intelligent play. Avoiding difficulties is supposed to get you XP according to the DMG but we all know that doesn't actually happen in any game or published module.

Adventurers respond very rapidly to incentives. If you give incentives for painstakingly stabbing minotaur after minotaur in the face the players will do that. If you incentivize running past the horde of minotaurs and rescuing the princess the players will do that instead. So if the XP comes from quest completion, players will \textit{complete quests}. If XP comes from Final Fantasy style XP dancing in the woods -- the players will do that instead. Since one of them makes for awesome stories, and the other is a rote repetition of the worst kind of World of Warcraft nonsense, we know what has to be done.

\abox{A Little Note on XP Costs}{I know that you're probably thinking ``If XP rewards are handed out in a less per-diem manner, doesn't that mean that XP costs would be more noticeable and even actually have meaning?" And of course the answer is ``yes." Sort of.

The problem with XP costs isn't just that they don't really cost anything ``in the long run" (which they don't), the problem is that they are bad for the game. Like Age increases before them, an XP cost is essentially running up a credit card bill. You get whatever it is that you were buying with the XP cost \textit{now}, and you pay \textit{later} (by death from old age or not going up in level when you otherwise would). That's never balanced, because there's no guaranty that the character in question will still be being played when that credit card comes due.

So even though staggering XP gains out longer as suggested in this book \textit{would} make XP costs more meaningful than the hoax they are in the basic rules, we still strongly aadvise you to do away with them in your home games as we have in ours.}


\subsection{Strategies of Advancement.}

Having determined the core problems with advancement in the manner described in the DMG, let's talk about some of the ways you could do it that might be satisfying. Like the handling of alignment and necromancy that we're talked of in the past, there really is no right answer -- it really depends upon what your group wants to do.

\subsubsection{Steady State 1: Serial Heroism}
\vspace*{-8pt}
\quot{``We have another mission for you\ldots''}

Let's face it: in a lot of fictional source material, the characters don't really change between their adventures at all. In fact, that's kind of the \textit{point} of a lot of stories. The hero is the one fixed point and the story is just the fixed character reacting to different situations. You read about Conan or Hercules fighting the Moon Men or the Ice Jarls, but you don't really read the story set after Hercules got a laser gun and grew wings. Even the books where Conan is an old man rarely reference specific events from previous books.

In the serial heroism campaign, characters begin play at the level that depicts their abilities appropriately. Characters have signature equipment and a collection of levels and skills that are integral to their character. Over the course of the adventure, the characters may well find new equipment and learn special crap and be blessed by Nymph Pools and whatever -- but at the beginning of the next adventure they will be back to exactly the same place they were last time. Even characters getting married or having limbs whacked off doesn't have any effect on the next episode.

There are a lot of ways to explain this. Adventurers spend money profligately and put equipment into bat caves and bequeath magic swords to temples and favored wenches. Major wounds can be healed, and we all know how rarely things work out between men and women -- especially when one is a halfling rogue and the other is a giant iguana. You can either begin each episode by coming up with an amusing off-the-cuff answer to why you begin the next adventure just like you began the last one or you can just ignore it the way Saturday morning cartoons do. It's not a big problem.

There are a lot of advantages to this sort of thing. If the characters already do what they are supposed to (generally about level 6 or so with a couple of standard magic items and an artifact), then advancement of any kind just makes the character less like himself. He Man didn't become a better show when Prince Adam got a plane -- it just lost focus. But there are pitfalls as well. Certainly it is the case that games like World of Warcraft or Everquest can be remarkably unsatisfying \textit{precisely} because no real accomplishment can occur. It is a fine line between a character not changing and a character's actions not mattering -- walking that line is sometimes quite difficult. Certainly, before such a sweeping change is implemented, very frank discussions must be had between players and the DM. The game is essentially now a series of once off adventures that happen to have the same characters in them.

In the Serial Heroism game the character's core abilities are the same in every tale. That can be mythic. Like the Robin Hood songs. But it can also be retarded. Like the Smurfs.

\subsubsection{Steady State 2: Trophy Hunting}
\vspace*{-8pt}
\quot{``So\ldots\ where \emph{are} we putting the giant penny?''}

Characters like Conan and He-Man are pretty much the same between issues or episodes. But what of characters like Angel and Buffy who really do pick up and use equipment found in previous episodes? This is also a very plausible setup of ``nearly steady state" storytelling with limited character development. The character stays relatively recognizable one adventure to another. Chapter after chapter goes by without the player ever growing wings, learning to fly, shooting laser eye beams or in any other way having obviously gained a level of Bard. Important plot points and devices are referenced in later installments, allowing the characters to use the Doom Glaive after they took it off the cooling body of Bruc Avec Piti\`{e} immediately in that adventure and subsequently in later adventures as well. While in the true Serial the characters would have destroyed the Doom Glaive at the end of the adventure, in the Trophy Hunting model it stays in the Bat Cave only until it is needed for a later adventure.

In this model of steady state dynamics, the players gradually increase in power -- though they do so in an asymmetric fashion that is not level dependent. This means that the amount of Ogres that the party can successfully dispatch \textbf{will} increase considerably over time. But it won't increase \textit{dramatically} and the players may never be able to take on a really hardcore monster like a Cranium Rat Swarm or a Pit Fiend.

In this model then, it is expected that even the \textit{idea} of ``Wealth by Level" be tossed in the trash. The players are literally gaining as many as infinity magic items per level because by and large they aren't going up levels \textit{at all}, while magic items are accumulating slowly. Characters can bathe in magic puddles that increase their stats or find statues that transform into giant frogs; but this can also happen pretty slowly and still be fine because players aren't being forced into situations where they necessarily face higher leveled opposition all the time.

\subsubsection{Rapid Advancement: Level a Session}
\vspace*{-8pt}
\quot{``That was last week. This week I am a master of fire.''}

It is entirely plausible to play a game where the characters go up a level every adventure or even every session. While this sort of rapid advancement scenario is often dismissed as ``munchkin", it actually does capture the feel of many stand-alone books and movies quite well. There are a lot of stories like The Wheel of Time or The Matrix which are actually ruined by having sequels at all -- they are much better as a single progression where the characters begin as youngsters who don't even know about the major Evil that threatens the world and progress briskly into becoming world straddling badasses who control reality with willpower alone.

In this set up it is highly recommended that the DM hand out magic items like candy. After all, while the players are fighting hill giants today, they'll be up against a swarm of bloodfiend locusts next week and a rogue deva the week after that. The players will need new swag to face their new enemies just as they'll need new class abilities.

Many players feel that this sort of play environment is simple minded, but really nothing could be further from the truth. In fact, players have no chance to get acquainted with their new abilities before they are laden with even newer abilities. With only a single adventure to make use of each new level of powers it is entirely possible that the Wizard will \textit{never} get a chance to use one or even both of the shiny new spells he picks up each level. Indeed, since both the characters and the opposition is coming up with more power and options each week, the game is actually \textit{really hard}.

And that ironically, is the most major drawback of this gaming style. Some of the players who gravitate the most towards this advancement system are actually the least able to successfully juggle a new class level and two new magic items every week. Sure, there are difficulties to be had in this scenario when players miss a session or three (nothing says ``suck" like finding out that Fighter's girlfriend the sorceress cohort is actually a more powerful magician than you are). But that can be worked around in a number of ways: the DMG suggests giving out experience bonuses to people who fall behind until they catch up in level and that works well enough. Of course, to actually make use of that you'd have to chuck the idea of not being able to level more than once per session -- which makes characters even more confusing -- but there you go.

\subsubsection{Attenuating Advancement: Diminished Returns.}
\vspace*{-8pt}
\quot{``You youngsters have no concept of how difficult it was to get the Doom Glaive.''}

If one considers advancement at face value: a direct method to prevent adventuring from becoming ``stale", then it is entirely reasonable to question its inclusion in the game at all. After all, a sixth level party could very plausibly encounter a manticore, a summoning ooze, a dragon, a war party of ogres, a troll, an evil wizard, a dinosaur, a nymph, a mud salad, a nerra facechanger, a medusa, a circle of myconid, a cathedral protected by a stained glass golem, a cadre of yak folk, an infestation of ash rats, a room full of hammerers, a spawn of Kyuss, or a dreadful cleric with some orcish minions. Or whatever. The point is, you could very plausibly face different opposition every week until half the players move out of town before you ever run out of monsters to fight. The staleness then, comes not at the hands of the players in any case, but for the DM. After all, once the DM has thrown the adventure where an ancient cathedral of Pelor has been taken over by an evil group of Yak Folk who have bound a Janni and forced her to tell them the secret password that allows them to break into the inner cloister without having the stained glass tear itself out of the wall and attack them in order to conduct a foul ritual to transform the daughter of the old king into a medusa and set up some zombie ogres to protect themselves while the mighty ritual commences -- that leaves some of the DM's favorite monsters used up out of that level. More importantly however, the players are presenting essentially the same skill set so long as their skill set doesn't change -- meaning that the DM can become bored finding challenges for the PCs unless the PCs demonstrably change over time.

Be that as it may, the fact is that higher level characters with more magical swag have more abilities than do lower level characters and quite definitely present a face to team monster with more attachments on their Swiss Army knives.


% Crafting rules
% FIXME: Move the "magic item" things from "Magic and Magic Items" to here maybe?
\section{Crafting}

\subsection{Why a Revision to the Crafting Rules?}

An overhaul to the Craft rules may sound fairly unbalancing, as the current Craft rules were created to prevent characters from making a lot of money and potentially destabilizing their games with an influx of magic items. Unfortunately, like Level Allowance, the heavy nerfing to Crafting resulted in a lot of characters simply becoming unviable, a lot of very dumb things happening all around, and it still doesn't actually stop characters from breaking the game if they really want to. If the party is made out of Elves, they can simply set a single skill rank on fire and announce that they're going to spend 100 years farming, making trained Profession (Farmer) checks every week. That'll get them about 6 gp a week for the next 5,200 weeks -- for a total of 31,200 gp at first level before they even start adventuring. And as elves, they can honestly just spend 200 years farming or spend some real skill ranks on that to get even more money.

If the DM is willing to simply let players roll dice, have downtime, and purchase magic items of unlimited power, the game is already broken on first principles at first level using the PHB alone. If the DM wants to keep sanity going at all, then something in that equation is going to have to go. Probably everything in that equation should go. As discussed in the Dungeonomicon, there is an inherent limit to what players could reasonably be expected to be able to purchase with pieces of gold, so to a very real extent crafting for money is simply multiplying the amount of low-level equipment you have -- it doesn't particularly get you more powerful equipment. And of course there's no reason for players to be able to do all of this 9 to 5 working without having on-camera adventures. An adventure where you are running a silk factory and will make a bunch of money as soon as you can stop the goblin syndicate from extorting all your profits is pretty much the same as the adventure where you run off into a dungeon, fight the goblins, and take the money they stole from the silk merchants home in a sack.

So the nerfs on Crafting just aren't necessary. But what actually needs to change?

\listone
	\bolditem{Valuable Raw Materials Aren't Valuable:}{This is a part of the rules that makes me cry. Since the amount of value you make each day is based on the \textit{difficulty} of working the material and not on the \textit{value} of said material, there is no way for a goldsmith to stay in business. Gold is very easy to work and therefore the DC to work it is very low, and therefore a goldsmith makes very little in the way of finished product each week. A five pound gold candle holder is roughly four ounces and fits into the palm of your hand, but it'll take a master goldsmith (+10 Craft Bonus) almost a year to finish one (500 gp value, at DC 5 = 50 weeks).}
	\bolditem{The Costs of Materials are WHAT?}{Remember that five pound gold candle holder? It's worth 500 gp and therefore requires 167 gp worth of materials to make it. But it's worth 250 gp just as a lump of gold. So you can buy things as raw materials and sell them as trade goods and make \textit{lots of money}. The reverse happens when you make complex or finely worked items. A masterwork sword is made out of pretty much the same materials as a normal sword and is much more expensive because it's better made. But because the higher quality crafting will make it sell for more down the line, the cost of the materials goes up by a 100 gp. Where does that money go? What are you getting for 2 pounds of gold? Sure, maybe you get some better coal or something, but really, that doesn't even begin to cover it.}
	\bolditem{Field Fortifications Cannot Happen:}{Even the simplest of traps (such as a bucket with some acid in it balanced on a partially open door) has a cost that is very high -- in the hundreds of gp. That means even the most gifted craftsman is going to take weeks to boobytrap a room or lay down some field fortifications. When longbowmen want to hammer some stakes into the ground to protect themselves from the knight stampede that's going to come when the battle starts, the Craft rules essentially tell them that they can't do it. Which for those of us who have seen Henry V, seems unlikely.}
	\bolditem{Risky and Illegal Trades are Pointless:}{Some products are expensive because producing them is risky (poison, flower arrangements from the Bane Mires). Some products are expensive because their production and sale is in some manner restricted by the authorities (shrunken dwarf heads, disrespectful puppets of the king). In the real world, people produce these things because they can charge inflated prices because of the risk. It's a gamble, where sometimes you make big money and sometimes you get killed by hydras or agents of King Ronard. But with craft times directly dependent upon resale value, these crafts are gambles where sometimes you make the same amount of money you would have making night stands, and sometimes you get killed by your own poison or Clerics of Torm.}
\end{list}


% Traps
\section{Dangerous Locations: When the Floor has a CR}

It is an undeniable truth that hunting goblins in a dank warren filled with dead falls and snares is both more exciting and more dangerous than hunting the same goblins in an open field. However, it must be stressed that the way 3rd edition D\&D has traditionally dealt with this -- to give CRs to individual traps as if they were enemy monsters in their own right -- is both unsatisfying and unplayable. The fact is that you probably are \textit{never} going to tell a story about the time your party was shot at by an arrow trap, it just isn't interesting in the same way that overcoming an evil necromancer or slaying a greedy dragon is.

And why is that? It's essentially because an arrow trap is not an encounter, it's an \textit{attack}. Just a single salvo in an ongoing battle between you and the dungeon, not a battle in and of itself. And when looked at in that manner, the problem becomes obvious: a \spell{glyph of warding} is a single spell. Overcoming it is like making your saving throw against the same Cleric's \spell{hold person}, it's simply wildly inappropriate to stop the action and play the battle complete music at that point.

So what do we do about it? Well, just as one does not stop and record a victory every time you bypass a summoned monster or overcome an opponent's thrown javelin, we shouldn't be worrying about the CR of individual traps. No, we should be concerned only with the CR of \textit{areas} that have traps in them. For one thing, this means that we don't have to have endless arguments over whether people should get the XP for bypassing the \spell{symbol of pain} on the door if they came in through the floor. For another, the act of avoiding \textit{that} stupid argument really helps to encourage characters to play things a bit smarter and not simply run through the ``Hallway of Leveling" where they go up a level every thirteen traps and get to 20th level in less than 150 doors.

\subsection{Location CRs: Quality and Quantity}
\vspace*{-8pt}
\quot{``Why check the door? Maybe because there was a trap on \textbf{every single other door in this entire complex?!}''}

To a limited extent an area can become more dangerous by making traps more ubiquitous. We say a ``limited" extent because there is a profound sense of diminishing returns when the chance of encountering a trap equals one. Our classic example is the Citadel of Fire, the castle that is the home of the Efreeti King. It's \textit{on fire}. Every square is \textit{on fire}. Every door is \textit{on fire}. And if you go there, \textit{you'll be on fire}. To an extent, that means that the kind of dangerous area that you might have seen in the Lizard Temple when you were 4th level is now every square on the battlemat. That's bad. But it's not unconquerably bad. It doesn't take a whole lot of Fire Resistance to survive in that kind of environment, and you don't have to be amazingly high level to get your grubby mitts on that kind of fire resistance. The fact that every single doorknob and chair is on fire in the Citadel essentially just means ``Only Adventurers with Fire Resistance can Adventure here" or even ``You must be at least as tall as this sign to attack the Citadel."

And the effect would be pretty much the same if you just had to wade through a \textit{moat} of Fire. There are literally dozens of rooms in the Citadel of Fire that are on fire without this increasing the difficulty of your assault in any way. And that's OK. In fact, people would be slightly offended if large amounts of the Citadel of Fire were not in fact on fire, which would be the logical way to do it if you were handing out XP or construction costs on a per flaming room basis. It adds to the immersion to have some relatively homogenous fantasy environments.

Practically speaking, this means that by the time you have put in enough of a single type of difficulty that the players will not plausibly be able to complete their quest without taking appropriate precautions, the CR of the location shouldn't rise any more by adding more of the same difficulty. And that goes for more than just places being on fire. If there are enough pressure plates linked to arrows that the PCs aren't going to get through alive without the Rogue taking 20 on her Search checks, throwing in some more arrow traps (or tripwires, or anything else that the Rogue can find and bypass by just taking the time to search thoroughly) doesn't make the area any more difficult. A cave at the bottom of the sea isn't any more difficult when it's \textit{completely} full of water than when it's \textit{mostly} full of water -- you still need \spell{water breathing} just to get there.

\subsection{WWMD? Disabling Traps.}
\vspace*{-8pt}
\quot{``A paperclip can be a wondrous thing. More times than I can remember, one of these has gotten me out of a tight spot.''}

The Disable Device Skill is extremely powerful and amazingly bizarre. You don't need it to bypass a trap, there are dungeons full of Kuo Toans who have no more Disable Device than you do who bypass traps every day. What Disable Device \textit{does} do is allow you to interfere with the mechanisms of mechanical and magical devices such that they don't get in your stuff when you \textit{don't} have access to the special catch or magic word or whatever it is that you're supposed to have. In short, any fool can press an off switch or simply not step on an on-switch; Disable Device allows you to shut things down \textit{without} access to those things.

Once you have found a trap with the Spot skill, it requires no skill roll at all to simply walk around it. If you discover a pressure plate, you can normally expect to simply step or jump over it without even making a Disable Device check. What Disable Device let's you do is set the plate to not trigger if you do walk on it. Often that's pretty pointless, but sometimes it's pretty useful, especially if you're up against a ``trap" that is a siege defense or hostile spell (such that its normal deactivation trigger is far away). Remember however, that you can still activate traps by any of a number of means without actually being in harm's way. Summoned monsters, tossed barrels and the ubiquitous 10' pole have been used by generations of adventurers to activate traps from 10' or more away. Again, that totally works and requires \textit{zero} ranks in disable device. However, sometimes you don't want a trap to go off at all or a trap can go off virtually limitless numbers of times -- that's where disable device comes in.

So what counts as a device? Well\ldots\ \textit{everything}. Every mechanical or magical effect is a device. A \spell{Wall of Force} is a device as is a giant stone block that is set to fall down on a foolish intruder who breaks a trip wire. A character with sufficient Disable Device can successfully turn off any magical effect or prevent virtually any cause and effect chain from occurring. You can stop an avalanche (DC 15) even after it has begun (DC 35). You can remove any permanent magic effect, even curses like \spell{Cause Blindness} (DC 32). What you \textit{can't} do is disable instantaneous effects. \spell{Flesh to Stone}, therefore, is out of bounds for disabling, as is \spell{Wall of Stone}. Sorry, once an instantaneous effect has gone off, there's nothing left to disable.

How does that work? I have no frickin idea. Rogues, Thief Acrobats, Ninjas, and Gadgeteers are capable of simply turning off \spell{Geas} and there's no physical explanation for how it is that they do it. The fact is that most of the devices in D\&D are beyond my understanding. I don't know how a \spell{symbol of death} works, I don't know how the magical energies stay in place for weeks or years until activated, so I don't know how a Ninja goes about making those magical energies dissipate harmlessly without entering the kill zone. I do know that he can do it, and if required I can make something up that sounds cool. That's a DM's job, after all.

\abox{Item Spotlight: Bag of Flour}{The bag of flour can be used to disable any rune or sigil without meaningful risk. A magical rune can only detonate if it is uncovered. So if you throw some flour on it, the symbol can't ever explode and is now completely safe. You may want to put the flour on the end of a pole because moving your hand \textit{close} to a rune may trigger it before the flour lands.}

\subsection{I \textit{live} here: Setting off Traps}
\vspace*{-8pt}
\quot{``How did those gnolls run through that hallway if the whole thing collapses when people are in it?''}

The common conceit of trap placement is that they automatically go off against player characters who don't find them and automatically don't go off against Team Monster. Needless to say, that's ridiculous, and it actually harms the game when you implement it. While there are magical traps that are virtually guaranteed to go off against certain kinds of creatures and are nonetheless bypassable with something as simple as a command word, those are not PC/NPC selective. A command word bypassed Symbol will go off against any creature that doesn't say the magic word. That means that creatures without language capabilities like bears holding sharks or remorhazz will set those traps exactly as PCs who don't know any better would. It also means that any player character in the correct position can simply \textit{listen} for the command words that Goblins use when safely passing over the danger zone and use it themselves. The base DC is only 15 so the challenge here is actually getting into position to observe enemies bypassing magical traps rather than the replication of the technique itself. The bypass words on magical symbols are pretty forgiving, they can be spoken by blink dogs, Sahuagin and Xorn without serious risk of misunderstanding.

But what of other traps? Mechanical traps go off mechanically, which means that to make them go off you have to \textit{do} something to make it go off. And that means that there is a chance that even someone who doesn't have a clue what they are doing might simply happen to not set off the trap. Life is filled with Mr. McGoos and if there is \textit{any} path to walk across an area without setting off a pressure plate there is a chance that people will happen to do so. And yet, if there isn't a way to move past a trap, there's a whole area that the residents of an area have to avoid altogether (or just be immune to the effect of the trap). Here are some common trap triggers:

\listone
	\bolditem{Opening a Door:}{This is a common and fun one because unless someone decides to go through the wall (and sometimes even then) the trap will go off any time the door is opened. This can either be placed on ``fake" doors that the occupants have no intention of ever opening, or it can be put on doors that are used frequently if there is a separate switch to deactivate the trap (be sure to get buzzed in). The important part about this is that an opening trigger will go off any time the door is opened normally. If you cut a hole in the middle of the door and squeeze through it, you're probably safe. After all, the door itself is acting as a switch in this case, methods of entrance that don't literally involve turning that hinge often don't involve pulling the switch.}
	\bolditem{Tripping a Wire:}{Strings and wires can be strung in walkways at anything from ground to eye level. A trip wire sets off a trap when it is broken or pulled upon, and thus won't go off at all if creatures shorter than the wire run underneath it (barring polearms and the like). A tripwire lower to the ground is more likely to be randomly stepped over than is a higher tripwire, but less likely to be seen. Several trip wires can be run in tandem across a walkway to virtually guaranty that a passerby will sever them, but in doing so they become a lot more visible. In general, a trip wire can go off 25\% of the time when someone moves through its space and have a spot DC of 20, go off 50\% of the time and have a spot DC of 15, or go off 100\% of the time and have a spot DC of only 10. A trip wire can be severed without triggering the trap by holding both ends of the wire and slicing out the middle -- but this requires a Disable Device check (DC 20). Failure triggers the trap. A tripwire can be triggered from range by throwing a chair at the problem, or with an arrow (against projectile weapons a tripwire has an AC of 13, against a larger object such as a barrel or a couple of cabbages tied together the AC is negligible).}
	\bolditem{Pressing a Plate:}{Bizarrely complex mechanisms can be hidden inside of walls and a pressure plate is as good a manner as any to get those mechanisms up and working. I seriously don't have any idea what the mechanical pieces under the floor look like, and neither do you. And that's generally OK. Mostly players won't respond to pressure plates by breaking the floor or walls open to get at the clockwork (though that is a viable option), mostly players will gamely accept whatever fate the pressure plate has in store for them. Without tearing up the scenery, characters can disable a pressure plate with a Disable Device check (generally DC 20, though more awesome plates exist). Pressure plates can be disguised as regular floor and are often quite difficult to spot (DC 16-30). A pressure plate can be as small as a single out of place brick or floorboard and may go off quite rarely (1-5 times out of 20 when someone moves through the space), this has the advantage that characters ``in the know" can step over it (though enemies are presented with the same option). Alternately, pressure plates can cover entire squares, being triggered automatically if any creature heavier than a specific cutoff enters the square. In any case, characters can fly over a pressure plate or climb along the wall and simply never activate it.}
	\bolditem{Getting Stabbed:}{The old ones are the good ones, and many a trap has been simply to put pointy bits on areas that a character might step on, touch, or fall into. One can with exaggerated care simply step over such things, but in the heat of battle this may be pretty difficult. A single caltrop or blade is rather unlikely for someone to step on (a 1 on a d20 unless the character is crawling or otherwise stepping on more of the square than one might expect), and can be quite difficult to find unless one is specifically looking for it (DC 18 to spot). An area covered with spikes, caltrops, or blades is generally pretty obvious (DC 5 to spot), but it is generally assumed that anyone who moves into a covered square will step on one unless they take some sort of precautions. Caltrop covered terrain is difficult terrain, and characters who move through it at faster than a ½ speed walk are going to step on something they'd rather not unless they make a Reflex Save (DC 20). Characters standing in an area covered with caltrops or the like are denied their Dex bonus to AC unless they have 5 ranks in Balance or allow themselves to step on something every time they are attacked.}
	\bolditem{Offending a Glyph:}{Magical runes have at times been implied to have the power to determine a character's alignment, their level, their class, even what they've eaten recently. That's not good for anyone, and we cannot suggest that it be allowed. So here's what Runes do: first, they are constantly taking 20 on a Listen check. That means that you need to make a Stealth check DC 21 to sneak past one. It also means that they will, generally speaking, hear a command word to turn off or turn on. A Magic Rune can also have a detection spell imbedded in them, which last until the rune triggers. So a rune might be set to go off as soon as a source of ``Good" was brought to within 10 feet of the Rune. A Rune might also simply be set to go off whenever any creature moves through its area while it is active (being activated and deactivated with command words set when the rune is). The parameters of a rune can be determined with a DC 20 + Spell Level Knowledge Arcana check.}
\end{list}

\subsection{Facing the Architect: The CR of Locations}

When you adventure in a dangerous or exotic location you are essentially encountering the architect of that location. Each trap, obstacle, and danger of the region can be looked at as the contingent spells and attacks of the force that put that together. Sometimes a devious maze is engineered by a mad architect or fabricated by an elusive wizard and this is in fact literally true. Other times the Forest of Dread is just really dangerous on its own lookout and the only ``architect" involved is just the DM.

The importance here is that an individual \spell{fire trap} isn't really an encounter. It's a single attack, and a pretty ineffective one at that. When the wizard tries to soften you up with his \spell{explosive runes}, that's a lot like the same wizard softening you up by conjuring some celestial badgers and sending them around the corner to engage your forces.

So while we definitely do not suggest doing something dumb like giving out XP for each trap bypassed, we do encourage you to consider the traps in an area to collectively be an opponent. An opponent that spends a lot of time hiding and taking opportunistic attacks. The Kobold Warrens, for example, have a number of trip wires set to launch crossbow bolts at anyone tall enough to pass through them. In an ideal world, the trip wires would be fairly visible, but in the heat of battle characters may feel compelled to chase after kobolds through the strings.

\subsubsection{Structuring Encounters in a Day}

Challenge Ratings have a real utility as a DM, but do not substitute for having a decent idea of what your party is capable of. We're going to go back to the Giant Scorpion a few times, because it's a very poignant example, but we could just as easily be talking about Fairies or Elementals. The Monstrous Scorpion comes in a variety of CRs based on its size and overall awesomeness. Don't be fooled: in reality a monstrous scorpion is essentially of identical difficulty regardless of size based entirely upon what the players are capable of tactically. The Monstrous Scorpion has no intelligence, no ranged attacks, and no interesting abilities -- it's just a biological construct that happens to be exceptionally tough in its one-dimensional way. If you can simply get to longish range (or \textit{fly}) and use ranged attacks, you win. It'll take a while, but you will win. It doesn't really matter what level you are, or how strong your ranged attacks are, victory will be yours. On the other hand, if the Scorpion is presented as a closet troll, it'll mess you right up.

What the CR grants you as DM then is a basic idea of how much ``resources" an encounter is liable to use up. The Scorpion, for example, will use up a lot of arrows and not a small amount of time. It probably won't cause any damage if the players play it smart, but it will drag things out for a bit. Higher CRs will take a bite out of the arrows of higher level parties and so on. Still, the fact is that in no way will facing an appropriately CRed monster use up the 20\% of your resources specified by the DMG. Not at any level. What kinds of resources will be used up will depend upon the types of opposition:

\listone
	\bolditem{Traps:}{Trapped locations of an appropriate CR are generally speaking time sinks more than anything else. At levels 1-6, the characters will normally Search regions that are known to contain traps, which reduces the character's speed through the area to 5' per 6 seconds (about \half MPH or 0.9 KPH).

	So even though we're looking to completely toss the idea that players should actually \textit{get} anything for necessarily killing ``Ogre Thug \#2" that doesn't mean that he shouldn't be there.

	As player characters become higher level they can take on more opposition. This does not necessarily mean they should be confronted with \textit{more powerful} opposition, but they should certainly encounter more of it. A Lunar Ravager and a Sand Giant are basically two large sized men with funny colored skin and a bad attitude. The fact that one is massively more powerful than the other is a staple of the D\&D system, but doesn't make an extremely exciting story. Having just looked up the stats of a Lunar Ravager and a Sand Giant I am confident that defeating a Sand Giant is a more difficult feat -- though of course it is not a more \textit{impressive} feat since as previously described both opponents are just 3 meter tall dudes with funny colored skin and a sword. Taking on 45 bug bears, which is something the stronger party could easily accomplish is however much more impressive than defeating 15 gnolls, as would be a light romp for the party who might otherwise face the Lunar Ravager.

	It is therefore important to note that parties should generally speaking not run into level appropriate opposition until quite late in an adventure. It's fine for a boss to be a True Fiend, Wizard, or Androsphix who is 2 or 3 CRs higher than the average character level in the party, but the vast majority of opposition should be several levels lower and a crap tonne more numerous than the PCs. This isn't just because this sort of thing keeps cleaving and \spell{fireballs} as reasonably viable tactics, but because high level combats really do involve lots of participants on both sides of the combat kicked out of the battle from time to time and if there's only one enemy it gets really anticlimactic.}
\end{list}



% Playing the game
% FIXME: Maybe move this to "Character Advancement"
\section{Playing the Game}

\subsection{What's that Noise?! Playing at Low Level}

There is a reason that the XP charts in the DMG completely fudge character levels 1-3. That is because those levels genuinely don't have a good consistent rubric for how powerful things are. There are damn few first level PCs that wouldn't go down if they took a lucky crit from a kobold's small light crossbow, and a first level Wizard has a pretty reasonable chance of taking down an orcish warrior by hitting him with a club. At first through third level, combat really is anyone's game and it is strongly advisable that the PCs outnumber their foes in the majority of confrontations at this level of conflict.

The TPK (Total Party Kill) is a very real concern for 2nd level characters, because the success or failure of actions is so very random. A run of bad luck can quite plausibly wipe out even a well-played low level team of adventurers quite easily and it is recommended that DMs use discrete encounters at these low levels in order to minimize the effects of having characters getting dropped by allowing the remaining characters to consistently revive fallen comrades.

\subsection{The Rigors of Command: Playing at High Levels}

A high level party isn't really ``adventuring" in the traditional sense any more, or at least they probably shouldn't be. Instead, they are playing a whole different game -- a \textit{strategic} game. Characters who make it into the Epic landscape can in fact become gods according to long standing D\&D tradition. Along the way it behooves you to conquer and administer stuff in order to propel yourself to victory.

More detail will be gone into in the Tome of Virtue, as the high level world is a really strange place. Almost all the source material from Arthur and Beowulf to Theseus and Ulysses involves characters who are somewhere between 1st and 6th level in D\&D terminology. Stories which involve a 10th level adventure are extremely rare. Perseus killed Medusa (CR 7), and Bellerophon killed Chimera (also CR 7), but they both pulled some fancy equipment and cheesy tactics to pull it off (Bellerophon seriously had a flying mount that was faster than Chimera and shot arrows at the beast until it died).

If one insists upon continuing with powerful characters in an adventuring role, there is a primary conceit which must be embraced: all adventures must be timed adventures. A 14th level Wizard can, with sufficient preparation, kill any challenge in D\&D without exception. And while sitting around planning the perfect murder of a red dragon or the perfect heist of a major artifact is interesting as an intellectual exercise, there is no way that represents an ``adventure" in the way we use that word to describe 4th level characters breaking into pantries and stabbing people in the face for money.


\section{Magic}

\subsection{Illusion Magic: I Don't Believe This Crap}

Illusion magic has the distinguishing characteristic of being either the most powerful school of magic, or the least -- entirely at the whims of your playgroup. Illusions can be used as distractions, threats, enticements, concealment, modes of communication, prisons, attacks, disguises, false targets, entertainments, misdirections, religious inspiration, incitements to riot, madness provokers, commercial fraud, redecoration, time wasters, limited-use ability wasters (like prepared spells, scroll spells, or use-per-day spell-like abilities), or traps (in conjunction with dangerous terrain, monsters, substances, events, or magical effects). And that's just using the 1st level spell \spell{silent image}.

People just don't expect their senses to lead them wrong, even in a world where people know that illusions exist. I mean, if a wall of fire suddenly pops up out of nowhere, it's actually more likely to actually be a real damaging wall made out of magical fire than it is to be an illusion of the same thing. And truthfully, who wants to pop a hand in to check? Not me either.

What this means is that illusions are incredibly powerful because they allow such perfect forgeries of the real world. The downside of this is that lots of DMs try to counter the efforts of creative players by using a particularly harsh interpretation of the Disbelief rules in order to nerf illusions out of existence. It works like this: by the rules, you get a Will save vs. an illusion if you ``interact" with it. DMs looking to throw salt in an illusionist's game usually allow that to mean ``in the same square as an illusion" or ``looking at it." You also automatically make a save if you have ``proof that an illusion isn't real." What that means is anyone's guess, because in D\&D even the most unlikely circumstances could quite plausibly occur without illusionary influence. A silent orc moving through the grass might be a \spell{silent image} of an orc, an orc in a \spell{silence} effect, an incorporeal orc, or just an orc who happens to be really sneaky. Once you disbelieved the illusion, you suddenly got to see through its like it was transparent.

Usually, DMs looking to punish illusionists will give multiple saves per turn, and then at some point just say that the target has automatically disbelieved the illusion, and this is possible only because the rules regarding illusions were written in the style of previous editions of D\&D called ``Rule 0" where playing a pick-up game of D\&D involved a few hours of discussion about how the DM handled most effects. The current edition of D\&D (3.X) mostly did away with this because it sucks up valuable game time to have arguments about D\&D rules and it was the worst part of playing the game; however, illusions were never fully overhauled, so we are still stuck with this noise.

Potential effects of illusions are also hotly debated. Some genius at WotC has laid down the law and said that the various \spell{image} and \spell{illusion} spells don't cause darkness, but that doesn't stop them from creating opaque mist or smoke or dust, obscuring objects, or even autumn leaves that drift around a person's head and float away from his touch, effectively blinding a person from dangers as well as complete darkness. Additionally, there are DM vs Player wars where DMs try to interpret the ``single object, creature, or force" line to mean ``no more than one person or a monster in the illusion" and players respond with things like ``its an illusion of a single force that summoned many monsters like the spell \spell{summon monster} or \spell{gate}" or ``its one object connected by many invisible threads." Other DMs and players are convinced that you control all visual information in the Area of Effect, while others agree but say things like ``you can't trap a creature in a bubble with visual information on the inside that mimics the world except for some key creatures/object/terrain/effects, but people outside see him as normal because his image is on the outside of bubble."

In the end, it's a mess because the current rules can be made to do amazing things by creative people, but those amazing things break the level system and that means that DMs are forced to punish players for their creativity, thus hurting everyone. That being said, here are some playable rules regarding illusions that won't cause you to stab out your own eyes.

% Magic Items
% FIXME: Possibly move this to Crafting? Probably not.
\section{Magic Items}
\vspace*{-8pt}
\quot{``No\ldots\ \emph{This} is a knife.''}

Any man on the street with a few nasty scars and good tale or two can call himself an adventurer, but there are a few true tests that can determine the difference between a talented liar and the kind of person who considers fighting dragons a slow day at the office. It's not demonstrable skills, or nerve, or even a history of past accomplishments. It's magic items.

I know that this sounds counterintuitive, but work it out for a second. Put a fighting guy with just better than average stats, some class features, and some HD out on the front line, and what do you have? Basically, you have a giant, which means ``NPC." Without a magic weapon to bypass DR, good armor to avoid being clobbered, healing magic to recover for the next fight, and crazy extra effects to surprise an enemy like \spell{dimension door}ing with the \magicitem{Cloak of the Montebank} or reflecting a spell with a \magicitem{Ring of Spell Turning}, you just don't have enough mojo to call yourself a PC. Monsters have bigger raw stats and better recharge times on their abilities, so if you don't have something extra you aren't going to be able to compete.

Magic items are the true test of the adventurer because they say ``I'm trying to grow my power asymmetrically and I'm willing to do it by stealing it from other people who are also growing their power asymmetrically." Anyone can fire a bow at a manticore in flight, but only an adventurer is so concerned with power that he'll track that manticore to its lair and risk getting boxed in by a family of manticores just for the opportunity to root through its droppings on the off chance some would-be hero got eaten by the thing and a magical trinket or two survived passing through its innards.

Some would called that ``greedy," but in fact that's ``hardcore." Real adventurers are willing and able to risk their life on just the hope that their efforts will bring magical loot, and it's worth it. The more magical loot one gains, the more able an adventurer is to survive the next terrible risk that might offer magical loot. Heck, just holding onto any reasonable-sized pile of magical loot means that one is to be reckoned with. A real adventurer won't sell his \magicitem{Helm of Brilliance} for a pile of ``magic beans" and he's not going to put himself in a situation where a common cutpurse is going to walk away with his \magicitem{Pearl of the Sirines}, and he's going to set himself up in society in a role where the local king or warlord can't just send a few longbowmen and an apprentice with \spell{detect magic} to confiscate his magical loot ``for the good of the state." Magic items are not just a reflection of your power, but a reflection of your character and your ability to choose your own destiny.

That being said, magic item creation and ownership is a big deal, and should not be the abbreviated (and broken) process that you see in the DMG. Here are some rules to make it sane and easy.

\subsection{The Core of Magic Item Design: Don't Do It Like Diablo}

Diablo II is a great game, but literally every single thing it does with magic items is bad for a table top role playing game.

\subsection{Magic Items with Class(ifications)}

It's all well and good to talk about ``Magic Items" as a whole, but there really is a very big difference between piles of scrolls (which have a modest effect on a single adventure) and a flaming sword (which has a modest effect on all your adventures. Not as much as the writers of the DMG seem to think -- but it's certainly there. An item with ``unlimited" charges is actually \textit{going} to be used a specific and finite number of times before the character stops adventuring, the item is destroyed, or the character starts using something else. While there is no specific limits to how many times you \textit{can} swing a sword, fundamentally there is a limit to how many times you are going to swing that sword.

\subsubsection{Activation vs. Constant}

Walking around in a suit of magic plate assumes that as long as it's worn properly, then without any prompting on the part of the character the suit is providing an enhancement bonus to AC. It's the same with a Ring of Fire Resistance, an Amulet of Natural Armor, and a host of other items. Similarly there are items such as magical swords that can be used round after round generating their effects time and time again without rest or recharge. It's the same with wands, most rods, the vast majority of rings, and collapsible animated ice swans. In either case these items are Constant items. Item providing a Constant effect (or usable in a Constant fashion) must be specifically targeted by a \spell{dispel magic} to be affected.

Other items need to be activated before they work. Scrolls and potions are classic examples, but a good percentage of magic items fall into that category. These are Activation items. Activation items have to be in some way prepped up before they are used. A scroll must be read and deciphered; a potion must be shaken up and opened. Any Activation effect can be dispelled in an Area \spell{dispel magic} or person-targeted effect (as appropriate).

\subsubsection{Ownership is a Privilege, Not a Right}

Several systems of magic item ownership have been attempted in the past. The current system is a pseudo chakra-based BS where magic power is limited by one's body parts where some items are dedicated to a specific body part (magic helmets like a Helm of Telepathy) and others are supposed to be put on the body but get to ignore this system (Ioun Stones are a classic example, as they float around your head and just give you some magic powers but you can have a dozen or a hundred doing that job and it's no problem). Other magic items generally sit in your pocket until you use them, and its assumed that your backpack is stuffed with them (staffs, wands, rods, most rings, scroll, potions, special-use weapons like \spell{ghost touch} swords, and about half of the wondrous magic items).

One of the dumber parts of D\&D has been the tally sheets of items where determining the effects and bonuses on a single character starts to look like doing your taxes. That's lame and slows down the game, and together that's unacceptable. Since we have removed the GP and XP rules from magic items, which were previously the only limiting factor on magic item abuse (which we did because they didn't really work), we have instead have these new rules for magic item ownership:

\begin{enumerate}
	\bolditem{Eight Item Limit:}{Adventurers can have up to eight Constant effect magic items operating on their body at one time. Any items past that limit (8), and the most recent items won't work. This can be any combination of items, but available space on the body is a limiting factor, meaning that you definitely can't wear two sets of chain armor at the same time (no way to get two torsos), but you can wear several amulets (assuming you have a neck, which most oozes don't) or even two helmets (assuming you have two heads like an ettin).

	Carrying around Activation magic items is no problem though. You can have bandoleers of potions across your chest, a scrollcase full of scrolls, or a magic arrow hidden up every seam in your clothes and every body cavity, but only eight items can currently be providing Constant benefits.

	A constant item must be worn/used and working properly for it to count against the Eight Item limit, and activation items can only be used one at a time. For example, Tommy of the Twelve Magic Daggers can wear a constant effect magic armor, a constant effect magic cloak, and five constant effect magic rings and still throw/activate his daggers one at a time in a round (assuming he can throw more than one each round), but if he tried to use two at a time with Two-Weapon Fight (for example: to benefit from qualities like Defending), then one of those daggers is not working and is basically a non-magical dagger. Some situations may arise where it is difficult to decide if a character is exceeding his limit; and in those cases, use your best judgment (meaning that if you are a DM, be consistent). For example, Tommy might be holding a magic longsword by the blade in his hand, so it's not ``active" since he can't take AoOs with it and get its bonus and its not providing him with any Constant benefit.}
	\bolditem{True Ownership:}{A person has to willingly put on a magic item and intend to activate it for it to count as active. That means that clever people can't trick you into putting on weak magic items so that your good magic items won't work when you try to use them. Unconscious or helpless characters can have items activated on their behalf (remember that in D\&D unconscious creatures are always ``willing"), so you can put a Ring of Regeneration on an unconscious buddy or put Dimensional Shackles on a sleeping wizard. Command word and spell completion items cannot be activated on someone's behalf (though you are welcome to use them on another character by dint of pointing the wand at your opponent and shooting lasers at them as normal).

	The flip side of this is that when you put an item down, it still counts as being one of your items for a period of time. This means that when you throw your magic spear it retains any benefits that are dependent upon your level while it is arcing trough the air into the dragon's chest; and it also means that it is not practical to pull a magic skirt off in the middle of combat and replace it with some really cute bike shorts. That's actually a good thing, because while if you're specifically playing Final Fantasy X 2 D20 it is setting appropriate to change your clothes in the middle of combat, in all other settings that sort of thing is just really dumb. Once it leaves your person, a constant magic item generally stops being one of your eight in a d4 minutes. If you're actually dead, your magic items stop counting as being yours the next round.

	Cursed items are the same. You have to try to use a cursed item before it can affect you. Otherwise, you can just keep it in a box labeled ``Cursed sword: Do not use for stab-ination."}
	\bolditem{An End to Bonuses:}{Andy Collins talks a lot about the ``big items" that players need to get in the door at high levels. Mostly swords and shields with bonuses on them. And while he is correct that people \textit{do} need them, I personally think that constantly taking up time worrying about getting another uninteresting ``slightly more magical sword" is bad for the game. The solution is truly that for magic items to fulfill their duty within the game without being really annoying, they just have to scale by level. So the ``+2 Sword" is dead. Now there's just a ``Magic Sword." If you happen to be 6th level when you use that sword, it'll be +2.}
	\bolditem{Artifacts have a Level:}{What makes Artifacts special? Mostly it's that they are a source of power that is completely asymmetric and well outside what the user could be ``expected" to have. This is represented by an artifact simply being a magic item that has a level on its own time. That means that the first level farmer's daughter who picks up Excalibur (an artifact with an inherent level of 15) gets all the benefits that she would had she actually been 15th level herself (a +5 enhancement bonus, being king of England, the whole deal). A character who holds an artifact of a lower level than herself still treats it as a magic item of her level -- the Artifact's level is a minimum, not a maximum.}
\end{enumerate}

\subsubsection{Wanna Take Some Body Slots?}

The slot system of traditional D\&D is more than a little bit insulting and carrying it over into this document would be a tragic failure of our design goal to make things not be like Diablo II. So yes, if you want to have every single one of your eight items be a ring, or an ioun stone, that's fine. Heck, you really could plausibly wear eight rings on one hand, there are people who do that sort of thing. If it's really important that you use three different magic crowns, we welcome you to run around calling yourself \textit{The Thrice Crowned King}. Nevertheless, items do have classes that they fit into fairly neatly:

\begin{enumerate}
	\bolditem{Wielded Items -}{These are held in a hand and brandished, swung, or otherwise triggered to activate their power.}
	\bolditem{Worn Items -}{These are placed somewhere on the body in order to unleash their power. While it is possible for someone to wear multiple sets of clothes, or armor over clothes, or even armor over other armor, only the heaviest armor counts as the armor you are wearing for purposes of AC, special abilities, etc.}
	\bolditem{Miscellaneous Items -}{These are items that are used in some other arbitrary way. Their power continues even when not held or worn, which is good because a lot of these items are things like thrones, golems, or crystal spheres that simply cannot be placed on the body at all.}
\end{enumerate}

% Magic Item Creation
\subsection{Magic Item Creation}

Building a magic item is a big deal. It is a way to expand one's power and a way to transfer power to your lessers, and in many ways the life of an adventurer revolves around the acquisition of magical loot. If magic item creation is too easy, adventuring is less fun, and if its too hard then people won't do it and resent the system and DM.

We know that the current rules don't work. GP and XP costs are things that have little meaning in even a low-level game, and players are notorious for finding ways around them by taking metagame classes like the Artificer, by having cohorts pay those costs, or even by giving morals the finger and having mindcontrolled captured spellcasters do it. That's before we even get to \spell{wishes}, powerful outsiders, or craziness like the Dark craft and soul rules.

There is one thing that hurts characters: time. Adventures and stories happen along a timeline, and players may or may not be able to stop during an adventure to build just the right item for an adventure. Even ``downtime," the time between adventures, is limited because powerful characters attract powerful enemies and predators. Heroes that say ``we'll just take a year off and make magic cloaks for everyone" are basically saying ``we'll sit in the open and let our potential and actual enemies pick the time and place for any battles." DMs can throw enough intrigues in someone's way during that time that before the first cloak is built that the campaign is over.

Creating magic items just requires time. There's work that goes into enchanting a sword, forging a blade, smelting the steel, mining the ore, and all that just takes time. If a character is really dedicated, he really seriously can wander off into the hills collecting reddish stones and then heating them up until iron comes out and then hammering the molten metal into a blade and then enchanting it with his power and walking out of the hills with a magic sword. Various portions of this can be expedited by, for example, \textit{hiring other people} to do a lot of that ? so a character can reasonably expect to throw down gold and buy himself a lot of that time back. But if you just have time; time will suffice. Exactly what magical goods are needed or helpful in magic item creation is highly variable campaign to campaign.

\listone
	\item \textbf{Questing for Reagents}\\It is a classic story for those fantasy settings that have on-camera magic item creation that characters must go quest for magical ingredients they need to make whatever the hell it is that they want to make
\end{list}

\subsubsection{Building a Better Magic Item: the Minor Magic Item}

A Minor Magic Item is one which can be produced in quantity by NPC apprentice factories and can thus be in some sense standardized or expected to exist in major city bazaars. Most minor magic items just provide some sort of unimpressive numeric bonus. The magnitude of that bonus is dependent upon the level of the character who is using that magic item. The rate at which the bonus scales to level varies depending upon what the item is giving a bonus to, and when magic items would provide a fractional bonus always round that fraction up. There are no caps on any of these bonuses. If you're a 19th level guy your sword simply provides a +7 enhancement bonus and that's fine. You're 19th level and you don't even really care.

\listone
	\bolditem{Enhancement Bonus to Weapons ::}{+1/3 per character level.}
	\bolditem{Enhancement Bonus to Armor ::}{+1/3 per character level.}
	\bolditem{Enhancement Bonus to Attributes ::}{+1/3 per character level.}
	\bolditem{Resistance Bonus to Saving Throws ::}{+1/3 per character level.}
	\bolditem{Competence Bonus to Skills ::}{+1 per character level.}
	\bolditem{Energy Resistance to any Energy Type ::}{+1 per character level.}
	\bolditem{Deflection Bonus to AC ::}{\fourth per character level.}
	\bolditem{Enhancement Bonus to Some Other Thing (Natural Armor, DR, SR, whatever) ::}{+1/3 per character level.}
\end{list}

Non-standard bonus types, or as we like to call them around the office: \textit{bullshit} bonus types do not exist. No, you can't have a Sacred Bonus to your AC or an Insight Bonus to your skills. That stuff is straight up broken and will push characters right off the random number generator. If all of your eight magic items are providing a bonus of some sort, they most definitely should not be providing a bonus to the same number -- that sort of thing really does make the d20 system fall apart.

Minor Magic Items which do not provide a numeric benefit usually reproduce the effect of a spell, and are caster level 5 or less. A Minor Magic Item may potentially be traded in the turnip economy. It is conceivable that a man might trade a wand of \spell{cure light wounds} for food or shelter directly. Nonetheless, these items are much more commonly traded for gold, and anything more powerful than a Minor Magic Item is actually less than worthless in the turnip market -- a \magicitem{Frost Brand} or \magicitem{Stone Sphere of Shaz} is really going to draw more fire for a peasant than it's worth.

More powerful magic items begin with a Minor Magic Item base and layer additional abilities on top. In this way a Sword of Sharpness always provides the basic level appropriate enhancement bonus to attack and damage even while it is chopping the heads off of dudes.

\subsubsection{Building a Better Magic Item: the Magic Weapon}

Generally speaking, magic weapons start with the basic minor magic item chassis: ``Weapon with an Enhancement Bonus" and items more powerful weapons also have an ability. There are two kinds of magical weapon ability: Spell-Like Abilities and Supernatural Abilities. An example of the first type is a Rod of Fire and an example of the second type is a Vorpal Sword.

A Spell-like ability is just a spell that having that weapon allows you to use. Using this spell-like ability is a Standard Action, so Quickened Spells aren't particularly interesting.

\abox{Behind the Scenes: What Spells Can Rods and Swords use?}{D\&D has literally thousands of \textit{pages} devoted to spells and it is entirely impractical to go through the list and find all the spells that would be appropriate from an activation ability for a magical rod or staff. Instead, here are some basic ideas of things which are \textit{not} a good idea to put into weapons:

	\listone
		\bolditem{Long Casting Times:}{Spells like \spell{major creation} can make stuff like Adamantine Boxes, which is all fine and dandy until you start making them in combat time by having them be used as a spell-like ability. Then it's suddenly battlefield control with no save allowed and that's just messed up. So while having a Rod of Summoning that allows you to throw down a Fullround spell like \spell{summon monster} whenever you feel the urge is fine -- sources of spells like \spell{move earth} and \spell{planar binding} are deeply problematic.}
		\bolditem{Swift or Quickened Spells:}{\spell{swift fly} is a really crappy spell except for the part where it's castable with a swift action. Even then it's not that great. When used as a Standard Action, it's just crap.}
		\bolditem{Juggling Spells:}{This last category is by far the hardest to nail down, because it isn't precisely quantifiable. But a spell effect that delays an opponent is really a crap tonne more effective if it is repeatable time and time again by autofiring the go button on a magic rod. Spells like \spell{frost breath} and \spell{color spray} are amazingly effective anyway, and if you can just throw them every round you go to straight up unfair territory.}
	\end{list}}

Supernatural abilities, on the other hand, are just things that your weapon does. Like a monster ability, your weapon simply has some effect going all the time. In many cases, this involves inflicting a status effect on enemies struck with the weapon. Status effects will be inflicted on the following circumstances:

\listone
	\item A target is struck at least once during a round (so figuring out some way to scam lots of attacks doesn't give extra statuses).
	\item The target fails a saving throw. The DC is 10 + \half the wielder's character level + the wielder's Charisma Modifier.
\end{list}

Here are some supernatural weapon qualities:

\listone
	\item \textbf{Lesser Qualities:}
	\listtwo
		\bolditem{Defender -}{A defending weapon moves itself to intercept attacks made on the wielder. While wielding a defending weapon, the character has an armor bonus of 5, enhanced by the enhancement bonus of the weapon.}
		\bolditem{Dispelling -}{A weapon of dispelling destroys magic. Anything struck that fails a Willpower Save is targeted by a targeted \spell{dispel magic}, with a dispelling check of d20 + wielder's character level (no cap).}
		\bolditem{Flame -}{A flaming weapon sets things touched by it \textit{on fire}. A victim who fails a Reflex Save is on fire and will suffer a d6 of fire damage every round until they extinguish themselves.}
		\bolditem{Ghost Touch -}{A ghost touch weapon spans the material and ethereal planes. It can be wielded by any standard, incorporeal, or ethereal being and can be used to attack any standard, incorporeal, or ethereal being with no miss chances due to the difference (if any).}
		\bolditem{Terror -}{A weapon of terror strikes fear into the hearts of its foes. A victim who fails a Willpower Save becomes \condition{shaken} for an hour. This is a [Fear] effect.}
		\bolditem{Thunder -}{A thundering weapon makes a whole lot of noise. A victim who fails a Fortitude Save becomes \condition{deafened}, and an object struck by a thundering weapon has its hardness ignored.}
		\bolditem{Time Distortion, Lesser -}{A weapon of lesser time distortion cuts time away from the target. A victim who fails a Willpower Save becomes \spell{slowed} for 5 rounds.}
		\bolditem{Berserking -}{A berserking weapon causes the wielder to go into a red rage of mindless fury. Whenever the user makes an attack with the weapon, the user is immune to mind affecting and fear effects for three rounds. However, during this period the character also cannot cast spells or activate magic items.}
	\end{list}
	\item \textbf{Moderate Qualities}
	\listtwo
		\bolditem{Cursed -}{A cursed weapon cannot be gotten rid of. A character who uses a cursed item will find that it continues to count against her 8 item limit for some time after being set aside, and it can be willed into her hand as a free action regardless of distance. Even if it is destroyed, the cursed weapon reforges itself once every day and continues to count against the wielder's item limit until a successful \spell{remove curse} is used to sever the connection.}
		\bolditem{Disruption -}{A disrupting weapon damages the necromantic animating force of the undead. An undead victim who fails a Fortitude Save is instantly destroyed.}
		\bolditem{Frost -}{A frost weapon freezes things quite severely. A victim who fails a Fortitude Save becomes \condition{fatigued}, normal fires are extinguished, and liquid objects freeze out to a 5' radius. Within 5 feet of an unsheathed frost weapon, the temperature cannot rise above \emph{cold}.}
		\bolditem{Lifestealing -}{A lifestealing weapon damages the souls of the living. A victim who fails a Willpower Save gains a negative level.}
		\bolditem{Planar -}{A planar weapon ironically is infused with the power of the Prime Material and is named thus because it's a good thing to have when traveling the planes. When a victim who is not a native of the Prime or whatever plane you happen to be on fails a Willpower Save it is instantly banished to its home plane, from which it may not leave for 24 hours (treat as a \spell{dimensional anchor}). In addition, an outsider victim must make a Fortitude Save or be \condition{dazed} for 1 round, regardless of what their native plane happens to be.}
		\bolditem{Sharpness -}{A weapon of sharpness cuts stuff into pieces. A victim who fails a Fortitude Save loses a limb (chosen at random), and an object struck by a weapon of sharpness has its hardness ignored. This enhancement is only available for sharp weapons, other weapons should use Withering instead.}
		\bolditem{Sun -}{A sun weapon sheds tremendous amounts of light. A victim who fails a Reflex Save becomes \condition{blind} for 1 round. Such a weapon sheds more light than normal, and is surrounded by a \spell{daylight} effect when in use.}
		\bolditem{Withering -}{This quality is exactly like ``sharpness" except that the special effect is that limbs wither and objects crumble. It is used for blunt weapons.}
		\bolditem{Wounding -}{A wounding weapon causes brutal and horribly bleeding wounds. Damage caused by a wounding weapon is vile physical damage even if the victim has Regeneration. A living victim who fails a Fortitude Save becomes \condition{staggered} for one round.}
	\end{list}
	\item \textbf{Greater Qualities}
	\listtwo
		\bolditem{Petrification -}{A petrifying weapon causes living tissue to transform into stone. A living victim who fails a Fortitude Save is \condition{petrified}.}
		\bolditem{Ruin -}{A ruinous weapon destroys pretty much anything. A ruin weapon ignores all hardness, DR, and resistance to critical hits of any target it strikes.}
		\bolditem{Soul Prison -}{A soul prison weapon absorbs the soul of any enemy slain with it. A victim who is dropped by a soul prison weapon has their soul immediately drawn into the weapon, where it remains until used. A soul prison weapon can hold up to nine such souls at a time, and not even a \spell{wish} can restore the life of a foe whose soul is therein contained. Nominally there is a Willpower save is involved, but since a dropped foe is considered ``willing" that doesn't normally come up. This is a necromantic effect.}
		\bolditem{Time Distortion, Greater -}{A weapon of greater time distortion cuts time away from the target. A victim who fails a Willpower Save becomes affected by \spell{temporal stasis} for ever.}
		\bolditem{Vorpal -}{A vorpal weapon kills things outright with a ``snicker-snack" noise. A victim who fails a Fortitude Save is killed, this is a death effect.}
	\end{list}
\end{list}

\subsubsection{Magic Ammunition}

A magic arrow is indeed a special thing. The only kind of magic arrow that doesn't make us feel really bad about ourselves is the Spell Arrow, so that's the only one that exists. Every magical arrow (or crossbow bolt, or whatever) has one spell in it which is chosen when it is made and which will be cast when it is fired. A spell arrow is not recoverable after the fact because the spell only goes off once. In order to actually get a magical arrow to ``go off", you have to spend a standard action firing it. Otherwise it's just an incredibly expensive arrow.

Magical arrows have the spell go off in whatever way would be most awesome looking. So if you fire an arrow which contains a touch ranged spell like \spell{cure serious wounds} or \spell{incite love} then the spell takes effect on whoever gets hit by the arrow. On the other hand, if you have a spell with a cone or line area of effect like \spell{lightning bolt} or \spell{color spray} it starts the line right in front of the bow. Bursts or Emanations come from wherever the arrow lands, and Personal or 0-range spells can't be made into Spell Arrows at all. In any case, the arrow itself is completely consumed by this process and doesn't do any actual damage (so curative arrows aren't as stupid as they might sound). Hitting a specific target with a Spell Arrow is a ranged touch attack.

\subsubsection{Magic Armor, Clothing, and Accessories}
\vspace*{-8pt}
\quot{``He's the man with the magic pants.''}

Heavy plate armor, racks upon racks of Mr. T style gold chains, shiny pants, and magic belts, these are a small set of examples of the crazy crap that people wear in the D\&D universe. The only difference between wearing, for example, a bunch of gold chains and a sleek set of leather armor is that the leather armor counts as \textit{armor} and has a tendency to provide some sort of Armor Bonus, Armor Check Penalty, and level appropriate bonuses (see Races of War). The gold chains just make you look like a circa-1986 rap star. But basic bonuses aside, all such items are simply a minor magic item unless they have some special ability above and beyond the standard level appropriate effect.

Special abilities on such items can be spell-like or supernatural, exactly as per weapons. The activated spells on a cloak or a belt function exactly like the activated spell-likes provided by

Here are some supernatural Worn-item qualities:

\listone
	\item \textbf{Lesser Qualities:}
	\listtwo
		\bolditem{All Around Vision -}{Iconically placed upon helmets and Spot-bonus items, this enhancement gives the user the ability to see in all directions, preventing enemies from flanking.}
		\bolditem{Aquatic -}{Iconically placed upon any worn item, this enhancement gives the user the aquatic subtype, allowing them to breath water and swim easily.}
		\bolditem{Dark Vision -}{Iconically placed upon helmets and Spot-bonus items, this enhancement gives the user the ability to see without light, as darkvision out to 60'.}
		\bolditem{Tremorsense -}{Iconically placed upon boots and Listen-bonus items, this enhancement gives the user the ability to detect things within 30' who are touching the ground as with tremor sense.}
	\end{list}
	\item \textbf{Moderate Qualities:}
	\listtwo
		\bolditem{Blindsense -}{Iconically placed upon helmets and Listen-bonus items, this enhancement gives the user the ability to detect things within 60' as with blind sense.}
		\bolditem{Madness -}{This enhancement surrounds the user the with maddening trills and whispers, causing all sane creatures within 10 feet of the user to have to save vs. a \spell{hypnotism} effect each round that the item is active and uncovered.}
		\bolditem{Spell Resistance -}{Iconically placed upon protective items and cloaks, this enhancement gives the user Spell Resistance of 10 + Character Level. Spell Resistance from multiple items with this enhancement do not stack.}
		\bolditem{Telepathy -}{Iconically placed upon helmets and Sense Motive-bonus items, this enhancement gives the user the ability to silently communicate with any creature which has a language out to 100' regardless of line of effect.}
	\end{list}
\end{list}

\abox{Armor Bonuses and Natural Armor Bonuses}{Yes, Armor Bonuses and Natural Armor Bonuses stack, but they don't 100\% stack. If you have both an Armor Bonus and a Natural Armor Bonus, you only benefit from half of the smaller bonus (round up). So if you have a +8 Armor bonus and a +5 Natural Armor Bonus, you are getting a total of +11 from Armor and Natty Armor, not +13. The reason for this is because Natural Armor gets \textit{very large} on a number of creatures. Originally this was because writing in a big natural armor bonus is really easy and gives level appropriate overall bonuses for the stuff in the Monster Manual, but when you mix in regular armor it pushes defenses straight off the random number generator.}

\subsubsection{Magic Rings}
\vspace*{-8pt}
\quot{``In brightest day, in darkest night\ldots''}

There is nothing special about Rings. At this point there is enough fantasy material available that there are people deeply immersed in the genre who have never read the Nibelungenlied or Lord of the Rings. When Arneson and Gygax made D\&D back in the day, LotR really was primary inspiration and the natural result was to put rings on some sort of whacky pedestal. Well, nowadays we have people for whom the iconic Magic Item of Vast Power is a lamp (Aladdin), a gem (Dark Crystal), an orb (Castle of Llyr), or whatever. So a Ring is just like any other piece of clothing, save that it rarely provides an enhancement bonus to armor.

\subsubsection{Constant Miscellaneous Magic Objects}

There are a number of objects in D\&D land that are neither worn nor wielded and yet count as constant items. Crystal Balls, Handy Haversacks, and Braziers that call fire elementals are all powerful items that do count against a character's eight item limit. What they don't do is actually stay connected to the user in a physical sense between uses. In order to use one of these items, one must \emph{attune} it, at which point that item is connected to the character who did so. Attuning such an item takes 15 minutes, and it takes that long for it to stop being attuned as well. It takes an act of will to make a magic item of this sort stop working for you, and this act of will can be taken either by you or someone who holds the actual object. So if someone snags your decanter of endless water,

\abox{Behind the Scenes: Attuning Crystal Balls.}{When you draw a flame tongue it bursts into flame immediately upon leaving its sheath -- granting a level appropriate bonus to attack and damage while setting stuff on fire. However, the same does not happen when you uncover a crystal ball. And the reason for this is honestly that items like collapsible bridges, bags of holding, and iron flasks are almost never used in combat time and yet they \textit{do} have a serious impact on your success or failure in an adventure.

It takes longer to swap these objects into and out of your bat cave simply because it is assumed that when you would be doing this you actually have more time to swap things in and out. In fact, it might be pointed out that it takes precisely as long to attune such an item as it does to fill an open spell slot on the fly -- that's not an accident.}

\subsubsection{Building a Better Magic Item: Intelligent Items}
\vspace*{-8pt}
\quot{``Hello computer!''}

In every edition of D\&D, the intelligent item has been listed as something that happened quite frequently. Seriously, even in the 3.5 DMG it says that fully 1\% of all Amulets of Health and Rings of Featherfall have intelligence. Were you to actually roll that up for every item you found it seems a virtual lock that every campaign would have one or more Intelligent Items in them. Since the vast majority of campaigns include \textit{zero} talking swords rather than the 1-5 expected by purely random chance, it seems extremely clear that something is wrong with the way Intelligent Items have been handled in the last 40 years of D\&D.

An Intelligent Item is like having a cohort, and if it is the same level as you are that's really unbalancing to the game. While previous editions have tried to keep track of ego points, we're going to try to make this as simple as possible: An Intelligent Item is a Sorcerer who happens to be a dagger or a pair of shoes. Like any Sorcerer, an Intelligent Item has a character level, and if that character level is more than 2 less than your character's level, it \textit{will not be your cohort}.

And that's it. An Intelligent Item is ``just" a magic item that happens to have one or more levels of Sorcerer, and an Int, Wis, and Cha. If it is within one step of alignment of your character, and is at least two levels lower than your character, and it is attuned as one of your eight items, it will work with you -- casting its spells on your behalf. An Intelligent Item never needs to worry about somatic components, which is just as well because a lot of them don't have moving parts.

That being said, an Intelligent Item is still an extremely powerful, game altering item. An extra spiderweb cloak that is throwing down \spell{web} in pitched battles can make the difference between life and death even at very high levels.

\subsection{The Appearance of Magic Items}
\vspace*{-8pt}
\quot{``Don't touch that sword.''\\
``Why? Because it's on fire? Because it has glowing runes?''\\
``Because the glowing runes say `Don't touch this sword.' ''}

Magic items do not normally require a casting of \spell{detect magic} to uncover. The DC of an appraise check to determine that something is in fact magical is 20 \textit{minus} the object's caster level. A powerful item bends space around it and glows with unearthly soulflame and such and really can be noted as magical by the untrained eye. But what exactly a magic item \textit{looks like} is contingent upon who made it and what they made it out of. Broadly speaking, the magic items made by the Drow really are black and covered with spider motifs; the magic items made by the Hobazad Khanate are generally lacquered in red and black with decorative leafing of gold and brass; the artificer mages of Bladereach make their magic items by etching them with hydra saliva so they look all melty and marbled.

Minor Magic Items of any sort can usually be identified by regular people who are familiar with the culture which produced them. If you're a Drow you've \textit{seen} the cloaks of resistance that the tailors in your society make. You might even own one. It's really not any kind of mystery to you.

Artifacts of course, follow their own set of rules. Some artifacts are instantly identifiable as powerful magical objects by people remotely in the vicinity (good examples of this are the Rod of Orcus and the Machine of Lum the Mad), while others really do adequately disguise themselves as mundane, even commonplace items (good examples of this second type are Aladdin's Lamp and the Pillowcase of Storms).

\subsubsection{Iconic Forms}

Let's face it, magic items are more fun when they come in recognizable forms. See a wizard waving around a stick and knowing that its a wand is more fun than trying to guess the effects of a glowing stone in his hand. That being said, here are additional rules to bridge the gap between our creation system and 3.X D\&D.

Iconic Form bonus: Any item made in both its iconic form (ring, wand, sroll, etc) and enchantment as shown in the DMG or other published source recieves can be created as if it was -2 its normal caster level after creation. This means that if you make a Cloak of the Manta Ray rather than a Ring of the Manta Ray, it takes you the amount of time it would take for a 7th level item instead of its normal 9th level, and it counts as a 7th level item for item creation limits. Note that this does mean that casters can create iconic items by using a lower caster level (so a 7th level caster can create a Cloak of Manta Ray, but not a Ring of the Manta Ray), assuming they can cast (or have cast) the required spells.

% This subsection was moved from later on in the book, because it definitely
% belongs under the "Magic Items" section.
\subsection{Disposing of Magic Items}
\vspace*{-8pt}
\quot{``You're going to have to throw The Ring into Mount Doom. Probably those pants as well.''}

Magic items are \textit{really dangerous}. Not just to use, but also to leave lying around. Or to destroy. Really anything you happen to do or not do with magic items carries significant consequences down the line.

The Bat Cave or Sword Rack is a relatively simple storage system for magical objects, and works fairly well.

---


% Treasure and the World
\section{Treasure and the World}

\subsection{Finding Treasure}
\vspace*{-8pt}
\quot{``There's nothing here but worthless gold!''}

It is an absolutely necessary step in the entire process of dragon slaying that one cart off the pile of gold. Indeed, previous editions oft as not required that one employ literal carts to carry off the fantastic wealth that a single Dragon might hoard. This was made possible by the letters G and H and by the number nine \textit{thousand}. And while it is true that the old alphabetical treasure types may have been a \textit{bit} overboard with the tremendous piles of loot that they handed out, the reverse trend of giving characters piles of gold that fit in one's pocket is fairly unfortunate.

It is absolutely the case that any dragon worth its salt should be worth enough in gold that it can actually \textit{sleep} on said gold. For reference, that's about 760 pounds of gold for a minimal medium-sized dragon, and about three tonnes of gold for a large dragon. But it is equally the case that when you encounter a group of gnolls or bugbears or even hill giants they generally don't have a big pile of gold and more often than not they don't have any magical items. Even more importantly, owlbears don't have any treasure \textit{at all}. Their digestive systems really will destroy all of the valuables they eat, and most of the time they won't even eat valuables because \textit{they're owlbears}! They live in the woods and they kill things and they don't participate in any economic activity at all.

\subsubsection{Books}

One of the most important and interesting things one can find in a cooperative storytelling game is a book. It's a story within a story, a source of potentially needed information and it's not really game breaking for your character to have it. It is for this very reason that virtually every Dungeon Magazine includes at least one book that the characters can find.

So what do books actually do? Well, the obvious thing is that if there are any spells in them, you can copy those down into your spellbook (or your \spell{secret page} manifest pad, if you're a modern wizard). But even if they are completely mundane they can still be useful. If you have enough of them on a subject you can have a \textit{Library}, which allows you to take 20 on Knowledge Checks. And a book about a specific subject can allow a character to spend an hour in study to make a knowledge check as if you had an appropriate Area of Expertise. So if you are confronted with a hobgoblin wartabard, and you can't make a sufficient Knowledge Geography or Knowledge History check to figure out where it comes from -- you can bust out a copy of \underline{Bastions of the Goblin Khanate} and try to find a match -- then you can make another knowledge check with the much lower DC.\\

\textbf{One Hundred Books that you can find in a Fantasy Setting:}
\begin{enumerate}
	\topsep=0pt
	\itemsep=-2pt
	\item The Ascendancy of Fire
	\item Abjuring Minor Demons
	\item Alterationism and Revisionism
	\item August and Winter
	\item Anatomie d'Ghoule
	\item Book of the Wars of Pelor
	\item Bastions of the Goblin Khanate
	\item Bees: Keeping and Secrets
	\item Benevolence and the Duchess
	\item Blzht's Personal Notes on the Badger Kingdoms
	\item Birthrights
	\item Crumbling Shadow
	\item The Cruelty of Healing Magic
	\item Carbuncles in an Elvish Context
	\item Cyclopean Constructions of the Vanished Ones
	\item Cults of the Maggot God
	\item De Vermis Mysteris
	\item The Draconomicon
	\item The Diary of Jakkar the Mad
	\item The Diary of Prince Olaf
	\item Dangerous Plants of the Bane Mires
	\item Djinn Fermentation Techniques
	\item Donjon Menagerie
	\item 101 Secrets of Devilcraft
	\item Eternal Subjects: Stasis and Crystal
	\item Evil
	\item Etherealness, Property, and Government
	\item Extreme Cold: A Goblin's Tale
	\item Ettercap: The Terrible Secret Reality
	\item The Crawling Darkness: Practical Necromancies
	\item Fabled Lands and Mythic Locales
	\item Fairy Courts: Sun and Shadow
	\item Fear in Hoburg
	\item Remnant Cities and Constructions of the Ancients
	\item The Bestiary Arcane
	\item Giant Crab!
	\item The Giant Kingdom: A Traveler's Perspective
	\item Gargoyle Physiology
	\item Grafting Flesh and Lead
	\item Gold: Providence and Necessity
	\item Gnome Lore
	\item Horror and Birds
	\item Harpy Statecraft
	\item The Harvest of Sorrows
	\item Blood of the Innocent
	\item The Asmodeus Gambit
	\item Blood and Silk: Danger Rises as the Sun
	\item The Cutting Edge: A Warrior's Tale
	\item The Dark History of Bladereach: A shocking and true revelation
	\item Fantastic Economie
	\item The Fly and the Serpent: Against the Giant Frog
	\item Cults: Demons: Dark Miracles
	\item Five Beans You Can Eat
	\item The Broken Mask: A Practical Guide to Hunting Shapeshifters
	\item The Book of Odamma
	\item Children of a Dark Star
	\item The Horrible Reality: The Devouring Darkness Unavoidable
	\item Last of its Kind: Twelve Dying Races
	\item Dark Revelations V through IX
	\item The House of Fiery Justice
	\item Aboleth Memories
	\item Industrial Uses of Slimes, Molds and Jellies
	\item Balance and Leverage: Druidic Construction and the Natural Order
	\item The Path of Blood
	\item Plains of Dust on the Planes of Water
	\item The Properties of 120 Magical Plants available anywhere on the planes
	\item Political Maneuvers of the Depraved
	\item Playing with Fire: The Dangers of the Vilest Necromantic Arts
	\item The Planar Political Primer
	\item The Physiology of Pain
	\item Prophecies of Profleggathron the Ever Burning
	\item Ash on the Wind: the Conquest of Valdrana
	\item Stone Unyielding, Impressions and Sand
	\item Servants of Leaf and Branch: Dryads, Nixies, and Nymphes.
	\item Secrets of Life and Death
	\item Slaves to the Black Tower
	\item The Complete Dwarven Histories: Volume XVIII: The Seventh Bugbear Confrontation
	\item A Transcript of the Trial of Harakhdar the Forsaken
	\item Taxidermy for Profiteers
	\item A Treatise on the Efficacy of Fungal Remedies
	\item Unaussprechlichen Kulten
	\item Ur Priest: Eating the Gods
	\item The Void and the Flame: The Story of Elothar
	\item Tactica Implacable: A Primer for Dwarven Strategists
	\item Land Grants of the Wendish Borderlands
	\item Surprisingly Delicious Things
	\item Tracking the Wily Displacer Beast
	\item Potion Miscibility
	\item The Worst the Banemires Can Do
	\item Wanderers of the Void: Giant Frog
	\item Six Problems of Classic Philosophy
	\item Twelve Ninja Clans
	\item The Xorn and the Unicorn: Root and Stone
	\item Path of the Mud Sorcerer:
	\item The Wish Economy and the Brass Sultan
	\item Metallurgical Properties of Mithril and its Common Alloys
	\item The Precepts of Hruggek
	\item Your Word Against Mine: The Kobold Problem
	\item Anathema
	\item Zone Agents
\end{enumerate}

\subsection{The Three (or so) Economies}
\vspace*{-8pt}
\quot{``I'll give you five pounds of gold, the soul of Karlack the Dread King, and three onions for your boat, the Sword of the Setting Sun, and that cabbage\ldots''}

Life in D\&D land is not like life in a capitalist meritocracy with expense accounts and credit cards. There is no unified monetary system and there are no marked prices. \textit{All} transactions are essentially barter, and you can only trade things for goods and services if people genuinely believe that the things you are trading have intrinsic value and the people you are trading to actually want those specific things. Gold can be traded to people only because people in the world genuinely think that gold is intrinsically valuable and that they want to own piles of gold.

That means that in places where people don't want gold -- such as the halfling farming collective of Feddledown, you can't buy anything with it. It's just a heavy, soft metal. But for most people in the fantasy universe, gold has a certain mystique that causes people to want it. That means that they'll trade things they don't need for gold. But no matter what they are giving up they aren't ``selling" things because money as we understand the concept doesn't really exist. They are \textit{trading} some goods or services directly for a physical object -- an actual lump of gold. Not a unit of value equivalency, not a promise of future gold, not a state guaranty of an amount of labor and productive work -- but an actual physical object that is being literally traded. And yeah, that's totally inefficient, but that's what you get when John Locke hasn't been born yet, let alone modern economic theorists like Adam Smith, Karl Marx, or Benito Mussolini. If you really want to get into the \textit{progressive} economic theories that people are throwing around with a straight face, go ahead and check out theoreticians like Martin Luther, Thomas Aquinas, Sir Thomas Moore, or Zheng He. If you want to see what \textit{conservative} opinions look like in D\&D land, go ahead and read up on your Draconis, Li Ssu, Aristotle, or Tamerlain.

\subsubsection{The Turnip Economy}
\vspace*{-8pt}
\quot{``We got rats! Rats on sticks!''}

Most settlements in a D\&D setting are really small and completely unable to sustain any barter for such frivolities as gold or magical goods. The blacksmith of a hamlet does not trade his wares for silver, he trades them for \textit{food}. He does this because the people around him are farmers and they don't make enough surplus to hoard valuable metals. So if he took gold for his services, he would get something he couldn't spend, and then he wouldn't be able to eat. So even though people in the tiny villages you fly over when you get your first gryphon will freely acknowledge that your handful of silver is worth very much more than their radishes, or their tin cups, or whatever it is that they produce for the market, they still won't trade for your metal because they know that by doing so they run the risk of starving to death as rich men.

The economy of your average gnomish village is so depressed by modern standards that even the \textit{idea} of wealth accumulation and currency is incomprehensible. But the idea of \textit{slacking off} is universal. There is a static amount of work that needs to be done on the farm each year and the peasants are perfectly willing to put you up if you do some of their chores. Seriously, they won't let you stay in their house for a copper pfennig or a silver ducat, but they \textit{will} give you food and shelter if you cleanout the pig trough. They have no use for your ``money", but they do need the poop out of the pig pen and they don't want to do it. On the other hand, they also don't want to be eaten by a manticore, so if you publicly slay one that has been terrorizing the village the people will feed you for free pretty much as long as you live. That's why people pay money to bards. Bards spend a lot of time in cities and actually will take payment in copper and gold. And if they sing songs about you, your fame increases. And fame really is something that you can use to buy yourself food and shelter from people in the turnip economy.

``Costs" in the turnip economy are extremely variable. In lean times, the buying power of a carrot is relatively high and in fat times the buying power of a cabbage is very low. It is in this way that the people in tiny hamlets get so very screwed. No matter how much they produce or don't produce, they are pretty much going to get just enough nails and ladders and such to continue the operations of their farms. However, such as there is a unit of currency in the barter economy of the turnip exchange -- it's a unit of 1000 Calories. That's enough food to keep one peasant alive for one day. It's not enough to feed them well, and it's not enough to make them grow big and strong, but it's enough so that they don't actually die (for reference, a specialist eats 2000 Calories a day to stay sharp and an actual adventurer eats 5000 Calories a day to maintain fighting shape). In Rokugan, that's called a Koku, and in much of Faerun it is called a ``ration". It works out to about 2 cups of dry rice (435 mL), or a 12 oz. steak (340 g), or 5 cups of black beans (1.133 kg), or 4.4 ounces of cooking oil (125 g).

Higher Calorie foods like meat and oil are more valuable and lower calorie foods like celery or spinach are less valuable because a lot of people exist on the razor's edge of starvation. The really fatty cuts of meat are the most valuable of all (it's like you're in Japan or Africa in that way). The practical effect of all of this is that people who have a skilled position such as blacksmith or scribe get enough food to grow up big, healthy, and intelligent. The peasants actually are weak and stupid because they only get 1000 Calories a day -- they won't die on that but they don't grow as people. This also means that the blacksmith's son becomes the next blacksmith -- he's the guy in the village who gets enough food to get the muscles you need to actually be a blacksmith.

When you start a party of adventurers, note the really tremendous expenditures that were required to make your characters. A 16 year old first level character didn't just get a longsword from somewhere, he's also been fed a non-starvation diet for 5844 days. That means that at some point your newly trained Fighter or Rogue seriously had someone invest thousands of Koku into him to allow him to get to that point. If your character is a street rat or a war orphan, consider where this food may have come from. Perhaps when the orcs destroyed your village leaving your character alone in the world the granary survived and your character had a huge supply of millet to sustain himself until he could hunt and kill deer to augment his diet.

\abox{A Note on Peasant Uprisings}{Peasants may seem like they get a crap deal out of life. That's because they do. And regardless of whatever happy peasant propaganda you may have seen, peasants aren't really happy with their life even under Good or Lawful rulership. That's because they work hard hours all year and get nothing to show for it. So the fact that they don't get \textit{beaten} by Good regimes or \textit{stolen from} by Lawful regimes doesn't really make them particularly rich or pleased.

In Earth's history, peasant uprisings happened about every other generation in every single county from Europe all the way to China all the way through the entire feudal era (all 1500 years of it). It is not unreasonable to expect that feudal regions in D\&D land would have even more peasant uprisings because the visible wealth discrepancies between Rakshasa overlords and halfling dirt farmers is that much more intense. Sure, as in the real world's history these uprisings would rarely win, and even more rarely actually hold territory (if lords can agree on nothing else, it is that the peasants should not be allowed to rise up and kill the lords). The lords are all powerful adventurers, or the family and friends of powerful adventurers, so the frequent peasant revolts are usually put down with \spell{fireball}s and even \spell{cloudkill}s.

Students of modern economic thought may notice that cutting the remote regions in on a portion of the central government's wealth in order to buy actual loyalty from the hinterlands could quite easily pay itself off in greater stability and the ability to invest in the production of the hinterlands causing the central government's coffers to swell with the enhanced overall economy and making the entire region safer and stronger in times of war ? but as noted elsewhere such talk is considered laughable even by Lawfully minded theorists in the D\&D world. After all, since abstract currency doesn't see use and the villagers don't have any \textit{gold}, it is ``well known" that it is \textit{impossible} to make a profit on investment in the villages. The only possible choices involve taking more or less of their food as taxes/loot as that is all they produce.}

\subsubsection{The Gold Economy}
\vspace*{-8pt}
\quot{``What pleasures can I get for a diamond?''\\
``We'll\ldots\ have to get the book.''}

People who live in cities mostly trade in gold. This is not just because living so far away from the dirt farmers makes the hoarding of turnips as a trade commodity a dangerous undertaking -- but because people living in cities are surrounded by a lot of \textit{people} who provide a wide variety of goods and services they are willing and able to trade for substances generally acknowledged to be valuable rather than trading directly for the goods and services that they actually want. These valuable substances range from precious metals (copper, silver, gold, platinum) to gems (pearls, rubies, onyx, diamond) to spices (salt, myconid spores, hellcandy flowers). In any case, these trade goods are traded back and forth many times before they are ever used for anything

When someone sells an item or a service for trade goods they are doing it for one of two reasons. The first is that they want \textit{something} that the buyer doesn't have. For example, a man might want a barrel of lard or a bolt of silk -- but they'll accept silver coins or something else that they are reasonably certain they can trade to a third party for whatever it is that they are actually interested in. Whoever is using the trade goods is at a disadvantage in the bargaining therefore, because while they are getting something they actually want, the other trader is essentially getting the \textit{potential} to purchase something they want once they walk around and find someone who will take the silver for their goods. It is for this reason that the purchasing power of gold is shockingly low in rural areas: a prospective trader would have to walk for days to get to another place he might actually spend a gold coin -- so all negotiation essentially starts with buying several days of the man's labor and attention. The second reason for accepting a trade good is the belief that the trade good may itself become more valuable. Indeed, when were crocodiles take over a nearby village all the silver becomes a lot more interesting. This sort of speculation happens all the time and is incredibly bad for the economy. People and dragons take enormous amounts of currency out of circulation and the resulting economic downturns are part of what makes the dark ages so\ldots\ \textit{dark}.

Gold and jewels \textit{can} be used to purchase magic items that aren't amazingly impressive. No wizard is ever going to make a masterpiece just to sell it for slips of silver. However, there are more than a few magicians who would be willing to invest some time in order to get a handful of gold that they can use to live their lives easier with. Making even Minor magic items is hard work, and wizards demand piles of gold to be heaped on them for producing even magical trinkets. And because these demands actually work, there's really no chance to purchase anything that would take a Magician a long time to make. That means that Major magic items cannot be purchased with standard trade goods \textit{at all}. There's literally no artificer anywhere who is going to sit down and make a Ring of Spellstoring or a Helm of Brilliance in order to sell it for gold -- because the same artificer can acquire as much gold as he can carry just by making Rings of \spell{Featherfall} or \magicitem{Cloaks of Resistance}.

\subsubsection{The Wish Economy}
\vspace*{-8pt}
\quot{``They scour the land searching for relics of the age of legends. Scant remnants they believe will grant them the powers of the Vanished Ones. I do not. The Age of Legends lives in me.''}

Magicians can only produce a relatively small number of truly powerful magic items. While a magician can produce any number of magic items that hold requirements at least 4 levels below their own -- a wizard is permitted only one masterpiece at each level of their progression. It is no surprise, therefore, that characters would be vastly interested in acquiring magic items produced by others that are even of near equivalence to the mightiest items that a character could produce. A character could plausibly bind 8 magic items, and yet they can only create one which is of their highest level of effect. Gaining powerful magic items from other sources is a virtual requirement of the powerful adventurer.

So it is of no surprise that there is a brisk -- if insanely risky -- trade in magical equipment amongst the mighty. All the ingredients are there: characters are often left holding onto items that they can't use (for example: a third fire scimitar) and they are totally willing to exchange them for other items that they might want (magical teapots that change the weather or helmets that allow a man to see in all directions). And while the mutual benefit of such trades is not to be downplayed, it is similarly obvious that the benefits of betrayal in such arrangements are amazingly amazing. Killing people and taking their magical stuff is what adventurers do, so handing magic items back and forth in a seedy bar in a planar metropolis is an obviously dangerous undertaking.

\subsubsection{Tamerlain's Economy: The Murderocracy}
\vspace*{-8pt}
\quot{``The soldier may die, but he must receive his pay.''}

Let's say that you don't want to exchange goods and services for other goods and services at all. Well, it's medieval times baby, there's totally another option. See, if you \textit{kill} people by stabbing them in the face when they want to be paid for things, you \textit{don't have to pay for things}. Indeed, if you have a big enough pack of gnolls at your back, you don't have to pay anything to anyone except your own personal posse of gnolls.

The disadvantages of this plan are obvious -- people get super pissed when they find out that you murdered their daughter because it was that or pay for a handful of radishes. But let's face it: if that old man can't do anything about it because you've \textit{got a pack of gnolls} -- then seriously what's he going to do? And while this sort of thing is often as not the source for an adventure hook (some guy comes to you and whines about how his whole family was killed by orcs/gnolls/your mom/ ogres/demons/or whatever and suddenly you have to strike a blow for great justice), it is also a cold harsh reality that everyone in D\&D land has to live with. Remember: noone has written \underline{The Rights of Man}. Heck, no one has even written \underline{Leviathan}. The fact that survivors of an attack may appeal to the better nature of adventurers is pretty much the only recompense that our gnoll posse might fear should they simply forcibly dispossess everyone in your village.

So people who have something that the \textit{really powerful} people want are in a lot of danger. If a dirt farmer who does all of his bargaining in and around the turnip economy suddenly finds himself with a pile of rubies that's \textit{bad news}. It's not that there aren't people who would be willing to trade that farmer fine clothing, good food, and even minor magic items for those rubies -- there totally are. But a pile of rubies is just big enough that a Marilith might take time out of her busy schedule to teleport in and murder his whole family for them. And he's a dirt farmer -- there's no way he has the force needed to even \textit{pretend} to have the force needed to stop her from doing it. So if you have planar currencies or powerful artifacts, you can't trade them to innkeepers and prostitutes. You can't even give them away save to other powerful people and organizations.

That doesn't mean that there isn't a peasant who runs around with a ring that casts \spell{charm person} once a day or there isn't a minor bandit chief who happens to have a magic sword. Those guys totally exist and they may well wander the lands trying to parlay their tiny piece of asymmetric power into something more. But the vast majority of these guys don't go on to become famous adventurers or dark lords -- they get their stuff taken away from them the first time they go head to head with someone with real power. Good or Evil, Lawful or Chaotic, \textit{noone} wants some idiot to be running around with a ring that \spell{charms} people -- because frankly that's the kind of dangerous and an accident waiting to happen. If you happen to be powerful and see some small fry running around with some magic -- your natural inclination is to take it from them. It doesn't matter what your alignment is, it doesn't matter if the guy with the wand of \spell{lightning bolt} is currently ``abusing" it, the fact is that if you don't take magic items away from little fish one of your enemies will. There is no right to private property. Noone owns anything, they just hold on to it until someone takes it from them.

\subsubsection{Beelzebub's Economy: The Trade in Favors}
\vspace*{-8pt}
\quot{``I'm certain that there's something we can do to help you\ldots\ but eventually you'll have to help us.''}

Every transaction in D\&D land is essentially barter. People trade a cloth sack for a handful of peas, people trade an embroidered silken sack for a handful of silver, and people trade a powerful magical sack for a handful of raw power. But in any of these cases, the exchange is a one-time swap of goods that one person wants more for goods the other person desires. But there is no reason it has to work like that. Modern economies abstract all of the exchanges by creating ``money" that is an arbitrary tally of how much goods and services one can expect society to deliver -- thereby allowing everyone to ``trade" for whatever they want regardless of what they happen to produce. Nothing nearly that awesome exists anywhere in the myriad worlds of Dungeons and Dragons.

What one \textit{can} see in heavy use is the trade in \textit{favors}. This is just like getting paid in money except that your money is only good with the guy who paid it to you. So you can see why people might be reluctant to sell you things for it. And yet despite the extremely obvious disadvantages of this system, it is in extremely wide use at every level of every economy. And the reason is because it's really convenient. There is no guaranty that a King will have anything you want right now when he needs you to kill the dragon that is plaguing his lands. In fact, with a dragon plaguing his lands, the King is probably in the worst possible position to pay you anything. But once the lands aren't on fire and taxes start rolling in, he can probably pay you quite handsomely. Heck, in two years or so his daughter will be marrying age and since she's just going to end up as an aristocrat unless she becomes the apprentice and cohort of a real adventurer\ldots

Failing to pay one's debts can have disastrous consequences in D\&D land. We're talking ``sold to hobgoblin slavers" levels of bad. Heck, this is a world in which you can seriously go into a court of law and present ``He needed killing" as an excuse for premeditated homicide, so people who renege on their favors owed are in actual mortal danger. Of course, everyone is in mortal danger all the time because in D\&D land you actually can have land shark attacks in your home town -- so it isn't like there are any less people who flake on duties and favors. Of course, if people know you let favors slide they might be less likely to pull you out of the way of oncoming land sharks. Even in Chaotic areas, pissing off your neighbors is rarely a great plan.


% Monsters
\section{Created Monsters: Forged and Bred}
\vspace*{-8pt}
\quot{``It's alive! Or at least animate\ldots\ it's not an object anymore, that's my point.''}

There are three entire types of creatures in D\&D that are to one degree or another created. The obvious one of course is the Construct. It's a creature which was never alive and created by sorcery. Well, most of them are like that. The Flesh Golem is kind of hard to explain actually, but whatever. The point is that every Construct is \textit{constructed}. That's the whole point. And of course there are a lot of Undead that are pretty much the same thing except that they are animated with Negative Energy channeled into them. There's a lot you can say about those guys, and we actually \textit{did} that in the Tome of Necromancy. So we aren't touching that one here. The other one of course, is the Vermin. In D\&D land, ``Vermin" doesn't mean anything vaguely approaching its meaning in natural English. Rats and cockroaches are vermin because they live in your pantry and poop on your food stores, but they aren't \textit{Vermin} because they aren't enormous biological constructs that mindlessly follow the programming planted in them by an ancient race of long departed mage kings. Really. That's what the Vermin type means in D\&D land. Actual giant insects are just animals in the same way that dire toads and weasels are animals.

\subsection{Vermin: Remnants of a Fallen Empire}
\vspace*{-8pt}
\quot{``Great holes secret are digged where earth's pores ought to suffice. Things have learnt to walk which ought to crawl\ldots''}

Ants track by smell and follow trails left by other ants and bees see deep into the ultraviolet spectrum and perceive a beautiful tapestry of gorgeous colors that escape the eye of the man and the mouse. And when dealing with Vermin type creatures that all means precisely \textit{nothing}, because Vermin in D\&D don't do any of that. It's not because the scent ability was ``left off" the Monstrous Ant description, it's because the Ant described in the Monster Manual genuinely doesn't have a good sense of smell. It does have Darkvision out to 60 feet like an outsider or a Construct, and that's not an accident either despite the fact that Earthly ants really demonstrably don't do that regardless of size.

The Monstrous Scorpion isn't a super sized scorpion \textit{at all}. It has a set of abilities which are on the face of it completely bizarre from the context of what actual scorpions do, because it's actually a living construct created by a long fallen empire for use in war. That's why it's immune to hallucinatory poisons and can see in perfect darkness. It's actually created from biomass by powerful magic and not by the interaction of natural and magical mutation across a thousand generations and a harsh selection process hastened by unpredictable climate and predation by manticores. The Vermin have a couple of neat things going for them which is why they were created as war machines in the first place:

\listone
	\bolditem{Mindless -}{Unlike actual or even giant spiders, the \textit{monstrous} spider has no mind at all. It cannot be influenced with magic or confused with poisons. It can't even be detected with \spell{detect thoughts}.}
	\bolditem{Brainless -}{Vermin are subject to critical hits because they have segments and organs, but they don't have any \textit{brains}. That means that they can be blinded, but not killed, by decapitation.}
	\bolditem{Darkvision -}{Vermin can see even in complete darkness, making them quite useful in cave fighting.}
	\bolditem{Aggressive -}{The vast majority of predators will retreat from battles where they are presented with even a chance at serious injury. Yet Vermin fight until they are dead. That's a really bad plan for an individual or even a species, but it's \textit{great} for a battle platform.}
\end{list}

\subsubsection{Who made the Vermin?}
\vspace*{-8pt}
\quot{``They did not \emph{know} that steel marks flesh, and they did not \emph{know} that flesh does not mark steel. In their ignorance they continued to do one task after another in the old ways. They did not \emph{know} what we \emph{know}.''}

Vermin come from the before time. The time when metals were not made and words were not written down. It's quite a feat of construction talent and a testimony to the power and ingenuity of these ancient flesh crafters that these devices are still running, still attempting to fulfill their programming to this day. The answer is not known in the days that D\&D is normally set. They are a product of a bygone age and their origin is a mystery to all but the Aboleth and the memory fish are being \textit{extremely quiet} on this subject. And yet, their conspicuous silence is probably more telling than anything they could possibly say. The Vermin were constructed during the days when aberrations ruled the world, and they were quite obviously designed to fight against aberrations.

\abox{Getting the Program}{All Vermin have a program that they follow at all times, usually involving a spiral search pattern in groups of one to six until they encounter a creature, at which point they will attack it until it is dead. If confronted with more than one type of creature, they will target them in the following order:
\listone
	\item Any Aberration (except their specific non-targeted group)
	\item Any Humanoid
	\item Any other moving creature they can detect
	\item Anything especially edible
\end{list}\vspace{2pt}

This behavior is entirely comprehensible from the standpoint of the wars in the before time -- the Ilithid and Aboleth both sent slave troopers to their death by the millions in their quest for world domination.}

Individual groups of Vermin will usually have one type of aberration that they will not ever attack. It may be an entire race of aberrations (such as Kopru or Neogi), or it may be a specific clan of aberrations (such as the Aboleth spawn of the Great Mother of the Howling Wells, or the Ilithid of the Tallow Halls). In any case, determining the type of Aberration that is completely safe from any group of Vermin can be done by observing the markings on the beast. Extracting that information is a DC 30 Knowledge Nature check.

Vermin eggs persist apparently indefinitely and are produced by the hundred score. A starving Vermin cocoons itself and goes into a state of hibernation so deep that it is essentially mummified. When in the presence of magical auras, the eggs of Vermin progress steadily towards hatching, and the cocoons burst forth their contents. Thus it is not weird or unexplainable for areas that recently have been subjected to incursion by adventurers or mind flayers to spontaneously develop invasions by tiny monstrous centipedes or giant cocooned spiders.

\subsubsection{Vermin Alchemy}
\vspace*{-8pt}
\quot{``The old ways are the good ways.''}

Vermin cannot think for themselves, nor would they have been better at their job given that ability. So it is not surprising that one can severely adjust the behavior of Vermin through the use of chemicals and sounds. Identifying the sounds and smells that a particular group of Vermin will respond to is difficult (requiring a DC 30 Knowledge Nature check), but actually producing them is not particularly. Here is a list of possible behavior modifications one can achieve and the Perform or Craft (Alchemy) check required:

\listone
	\bolditem{DC10 - Rampage}{It's a very simple behavior modification to cause a rampage. The spiral search pattern ends entirely and all affected Vermin take off in a random direction and move at full speed or until their path is blocked by a creature.}
	\bolditem{DC15 - Ignore}{There are chemicals that cause Vermin creatures to simply ignore}
	\bolditem{DC20 - Attack}{}
	\bolditem{DC20 - Shut Down}{}
	\bolditem{DC35 - Command}{}
\end{list}

\subsection{Constructs: Durability at a Price}

Like the Undead, the Constructs suffer tremendously from the fact that they have been over generalized. It is of course thematically appropriate for a Golem to be tireless and work day and night on whatever its last command was for as long as day follows night and night follows day. But it is also thematically appropriate for a clockwork beast to wind down and ``pass out" as it continues to work long or strenuous schedules. Similarly, while it is fine and more than fine for an implacable lump of animated steel to be immune to critical hits, the very idea that there aren't key locations on a geared robot or a colossus given life by a mystical forehead rune is patently ridiculous. The construct type, therefore is filled to the brim with stuff that has no business being there, and this harms the game. The immersiveness of the story is depleted when players cannot rationally deduce what effects a being is resistant or vulnerable to, and anyone who's ever slapped washing machine or tripped over a playstation knows that there's no excuse for a machine to be immune to stunning.

So here it is, the Construct Type. Pared down to the things it should actually do. Remembering of course that the Type itself should contain only those effects that one would want to be a universal law for all constructed beings, rather than rules one could imagine being situationally appropriate for one construct or another:

\listone
	\bolditem{Low Light Vision:}{Sees twice as far in limited illumination.}
	\bolditem{Dark Vision:}{60'}
	\bolditem{Poor Healing:}{Constructs can be healed by any of a number of means but do not heal for periods of rest. A construct's daily healing rate is 0 hp (though of course a construct with Fast Healing has a healing rate \textit{per round} and likely doesn't care).}
	\bolditem{Mindless:}{Even an intelligent construct has a synthetic mind that is unreachable by sorcery. A construct is not affected by [Mind Affecting] effects and cannot be detected with \spell{detect thoughts}.}
	\bolditem{Never Alive:}{A construct cannot be raised or resurrected. A construct is likewise immune to energy drain.}
	\bolditem{Repairable:}{A construct does not become staggered at 0 hit points, nor does it die at -10. If for some reason you are using the ``Death by Massive Damage Rule", constructs aren't affected by it. As soon as a Construct hits zero hit points it becomes inert, and any abilities it may have cease to function (including fast healing abilities). However, a construct in this state can still be brought to working order again with a Craft check with a DC equal to the DC to make it in the first place with a base amount of time of one hour per hit point below 1 the construct was left at.}
	\bolditem{Nonbiological:}{Constructs do not eat or breathe, and do not age.}
	\bolditem{Lacks Squishy Bits:}{A construct is not affected by any effect that allows a Fort save unless that effect affects objects or is a (Harmless) effect. For example, a clockwork horror is not going to catch red fever or become nauseated by a stinking cloud. But it is not outside the realm of possibility for an eidolon to be afflicted with a totally magical disease that functions off of Willpower saves.}
\end{list}
\smallskip

All the stuff about constructs being ``immune to necromancy" is out the window (because we all know that you can use \spell{magic jar} to put your soul into a statue); all the stuff about constructs being immune to ability damage is out the window (because we all know that you can slow down a lumber construct); and of course the immunity to critical hits is \textit{totally} out the window (if you have the name of Pelor on your forehead there is at least one critical location that probably won't go well for you if it is hit).

\subsubsection{Controlling Constructs: Robot Armies and Statuary Servants}

Time and time again adventurers report finding constructs that have been left attending temples and castles long after those buildings have fallen into ruin. The reason for this is twofold: First, constructs don't age; and Second, constructs don't count as one of your eight constant magical items if they are set to guarding a location. This means that powerful wizards are actually encouraged to leave their golems places with patrol or sentry orders and then of course these sentry golems will have a tendency to outlive the wizards, and even the buildings that they guard.

Of course, it's entirely possible to make your constructs follow you around. If you do, they count against your 8 item limit.

\abox{Behind the Curtain: Why the Lower CRs?}{A cohort, or a planarly bound outsider, or a necromantically crafted monster could all plausibly be of a CR that is just 2 less than your character level without particularly disrupting play. So it may seem pretty weird that the constructs one can order around are weaker than that. The reasoning is ironically because the tactical role of a construct is so different from that of a Ghoul or Jarilith. While many potential servant creatures are simply weaker versions of normal characters or dangerous and fragile glass cannons -- in almost all cases a construct is an offensively anemic unit with a highly powerful defense. For those of you who have played tactical games or MMOs, that makes the average construct an ideal ``pet". A strong defense is disproportionately useful for secondary characters expected to travel in front, and the fact that characters aren't allowed to fill their magic item cap with cohort level constructs is no accident.}

\subsection{Specific Constructs Under the New Rules}

% BIG HONKIN' NOTE:
% This uses \subsubsection instead of \monster. This is because we put 
% The Book of Gears in its entirety in its own appendix instead of moving 
% stuff around (I did it this way because the book is unfinished and it would 
% be awfully weird and illogical to move stuff around until it is finished).
% 
% So, when we finally move stuff around and put the simulacrum in the "Monsters" 
% chapter, we're gonna want to use \monster (which creates a subsection and a label 
% for hyperlinking) instead of \subsubsection.
% -Surgo
\subsubsection{Simulacrum}
Whether created by an Effigy Master, a mystic location or some other powerful source of illusion magic, a simulacrum is a construct made of ice and snow which appears to be a normal living creature through the power of illusion. Some 

\noindent\monstersizetype{Medium}{Construct}
\monsterline{Hit Dice}{6d10+6 (39 hit points)}
\monsterline{Initiative}{+1}
\monsterline{Speed}{30'}
\monsterline{AC}{11 (+1 Dex); Flat-footed 10; Touch 11}
\monsterline{BAB/Grapple}{+4/+5}
\monsterline{Attack}{Glamersword +5 melee (1d8+1)}
\monsterline{Full Attack}{Glamersword +5 melee (1d8+1)}
\monsterline{Space/Reach}{5'/5'}
\monsterline{Special Abilities}{Glamered, Imprinting}
\monsterline{Ability Scores}{Str 13; Dex 13; Con 13; Int 15; Wis 15; Cha 15}
\monsterline{Saves}{Fort +3; Reflex +3; Will +4}
\monsterline{Skills}{Bluff +11; Disguise +13 (+23 when Imprinted); Gather Information +11; Sense Motive +11}
\monsterline{Feats}{Impersonation}
\monsterline{Alignment}{As creator}
\monsterline{Organization}{Thrall}
\monsterline{Challenge Rating}{3}

It is important to note, however, that simulacra are entirely capable of using equipment, and usually will do so. Like most constructs, a simulacrum's true power comes to the fore when gifted with some basic mundane and magical equipment. Here is a sample simulacrum which has been given a magic shield, a magic breastplate, and a Frost Sword -- all equipment which is well within the capabilities of an Effigy Master to acquire or produce. While the simulacrum is still a ``CR 3 Creature" -- once it has been armed and equipped it is \textit{much} more formidable.

\noindent\textbf{Simulacrum with Equipment}\\
\monstersizetype{Medium}{Construct}
\monsterline{Hit Dice}{6d10+6 (39 hit points)}
\monsterline{Initiative}{+1}
\monsterline{Speed}{30'}
\monsterline{AC}{22 (+1 Dex, +7 Armor (Magic Breastplate), +4 Shield (Magic Shield)); Flat-footed 21; Touch 11}
\monsterline{BAB/Grapple}{+4/+5}
\monsterline{Attack}{Frost Sword +7 melee (1d8+3, +5 Cold Damage)}
\monsterline{Full Attack}{Frost Sword +7 melee (1d8+3, +5 Cold Damage)}
\monsterline{Space/Reach}{5'/5'}
\monsterline{Special Abilities}{Glamered, Imprinting, Ignore first 5 points of nonlethal damage (from armor), +2 bonus on bull rush attempts (from shield)}
\monsterline{Ability Scores}{Str 13; Dex 13; Con 13; Int 15; Wis 15; Cha 15}
\monsterline{Saves}{Fort +3; Reflex +3; Will +4}
\monsterline{Skills}{Bluff +11; Disguise +13 (+23 when Imprinted); Gather Information +11; Sense Motive +11}


\subsection{Denizens of the Planes of Law}

When you think avatars of Evil in D\&D it is no trouble at all to conjure up images of spiteful devils and destructive demons; but when you talk about a being of \textit{Law} the image that comes up is simply not the same from one person to another. Part of that is because Law doesn't really mean anything consistent in D\&D nomenclature. And part of that is because the actual description of the inhabitants of Mechanus has changed wildly through the generations and editions.

\subsubsection{Modrons: Singularity of Purpose}

For those of you who don't remember: Modrons are the original creatures of Law from the old days of AD\&D. They haven't been seen very often because they were originally written as a joke. Their very existence is as offensive to many players as the fact that they were essentially retconned out of existence is to others. And what's that all about? It's because the Modrons were originally written up as giant dice. Yes, really. The different types of basic Modron are shaped like four sided dice, six sided dice, 8 sided dice, the whole thing.

So if your DM jumps on the ``let's forget this \textit{ever} happened" bandwagon, we understand. The original write up of the Modrons was actually pretty insulting. But since then there have been a number of variously successful attempts to rehabilitate them and make them independently awesome. Different Modron art has been made by Tony DiTerlizzi and Eric Campanella that looks pretty darned awesome -- and not like your DM put a 6 sided die on the battle mat at all. Instead each Modron looks like a ghastly hybrid of metal and flesh covered with cogs and wheels where spindly appendages emerge from a solid (though not rollable) core.

So assuming that you use some of the reform Modrons from late in 2nd Edition, the Modrons are actually pretty cool. They represent the idea of Law as an implacable and incomprehensible force. They are at their best when portrayed as being so single mindedly focused on some long term goal that they actually don't even care about you. Sometimes they destroy your village, sometimes they don't, and there's really no predicting that sort of thing unless you're knowledgeable about the Big Plan. Now I know what you're thinking\ldots\ that having a plan so convoluted and far ranging that mortal minds cannot grasp it or predict its unfolding is actually indistinguishable from not having a plan at all and just performing actions at random. And yeah\ldots\ that's true. That's D\&D alignment for you.

The Modrons come from a city in the Clockwork Nirvana called Regulus and have a rigid caste system where more powerful Modrons are told more of ``the plan" than less powerful Modrons and each Modron is told exactly as much as it needs to know to complete its assigned tasks. And in the face of a long term plan of this magnitude, that pretty much means that every Modron is kept entirely in the dark about just what the heck it is doing or why it is doing it.
