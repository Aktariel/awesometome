\section{Alignment}

\subsection{Morality: How Black is the Night?}

Those readers who have been following this series will remember the basic moral question regarding Necromancy: namely, the fundamental decision for each game as to whether to treat Negative Energy as an objective force or an ultimate moral indictment. The central question surrounding fiends is less obvious, but in no way less important to your game. We \textit{know} that a Gelugon is Evil, he's got a \textit{subtype} that denotes him as being specifically Evil, that's not the question. What we don't know is \textit{how} Evil he is. That's a central question that has to be addressed within the context of each game. Let's face it, a lot of people really aren't comfortable with villainy more pernicious than the antagonists in a Saturday morning cartoon. Other people have a different and equally valid hang-up: they aren't comfortable having their characters stab enemies in the face repeatedly until they bleed to death unless those enemies are \textit{extremely bad people}. As so frequently happens, the rules for Dungeons and Dragons are written to accommodate both play styles, which in reality ends up including \textit{nothing}. Perhaps unfortunately, you \textit{must} come to a table-wide consensus about what your gaming is not doing before you can have your game do anything at all.  Keep in mind that none of these play styles are ``worse" or ``better".\\

\subsubsection{Moral Option 1: A Worthy Opponent}
\vspace*{-8pt}
\quot{``Fools! You have interfered with my plans for the last time!''}

For many games, the fact that the bad guys are bad is pretty much sufficient. Like the villains in Saturday Morning Cartoons, their villainy requires -- and gets -- no explanation. Actual villainy is fairly upsetting to contemplate, and a lot of people don't want to do it. I don't blame them, cannibalism, deliberate infliction of pain, and exploitation of the innocent are \textit{unpleasant}. Talking about secret prisons where torture is conducted night and day without respite or reason is \textit{super depressing}.

\ability{Implications:}{ The biggest implication here is that since Evil and Good are basically just political parties or ethnic hats, it is perfectly OK to have mixed alignment parties \textit{or} to ban mixed alignment parties. You're never going to have a serious discussion about what it is that Evil people do, so it's actually not important how you handle them. You can even switch how you're handling it in the middle for no reason. One day, the Atomic Skull can just chip in to save the world from Darkseid. Another day you can go kill the Atomic Skull without feeling bad. It's very liberating, because you can tell a lot of stories -- so long as none of those stories involve actual evil actions happening on camera.}

\ability{Pit Falls:}{ While it is certainly a load off the mind to not be constantly reminded of child abuse, torture, and sexual misconduct, bear in mind that this is Dungeons and Dragons -- your foes are more than likely going to be killed with extreme stabination. Possibly in the face. Possibly more than once. If the villains \textit{aren't} doing anything overwhelmingly bad, it's entirely possible that it won't seem like they \textit{deserve} being killed. If subjected to enough analysis, one might even find that your own ``heroes" appear to be the villains in your cooperative storytelling adventure. Certainly, He-Man never took that sword and chopped Skeletor into chunks. Star Wars: Episode One was such an unsatisfying movie in no small part because the villains never did anything bad.}

Glossing over the villainous activities of the bad guys should go hand in hand with all of the players acknowledging and understanding what you are doing and why you are doing it. As long as everyone is making the active and informed choice to not deal with the heavy moral questions -- it's all good.

\subsubsection{Moral Option 2: The Banality of Evil}
\vspace*{-8pt}
\quot{``It's 9 o'clock, time to get back to some Evil.''}

Many DMs will want to play their fiends pretty much like Nazis -- their agenda is hateful, but in their off time they go hang out at the pub just like everyone else. You could even sit there with them and drink together unless you happen to be a Jew. This is the default assumption of a lot of Planescape literature, for example. An Evil creature is Evil because it \textit{ever} does Evil things, not because it's necessarily doing any Evil \textit{right now}. Darkness and light are, in this model, pretty ephemeral concepts -- characters who wish to save their own sanity will end up either paying perhaps too much attention or ignoring them completely often as not.

\ability{Implications:}{ Since bad guys (and presumably good guys as well) spend most of their time being regular guys and only infrequently perform acts worthy of praise or scorn, it's extremely easy for heroes to fall to Evil and extremely easy for villains to be redeemed for full value. People on both sides of the Good/Evil axis are doing pretty unexceptional stuff most of the time, so the allegiance that even Evil Clerics have to darkness is pretty tenuous.}

This way of handling things is so much better at handling mysteries than are other morality systems that it may as well be a requirement if you ever want to play a ``who-done-it" adventure. Since the good guys and bad guys spend most of their day being actually indistinguishable one from another, it makes distinguishing them actually difficult -- and that has to happen if there is to be any question of who the PCs are supposed to stab.

\ability{Pit Falls:}{ Be wary of over-humanizing the villains. In many stories, the bad guys are a lot more interesting than the white hats; and that can seriously derail a campaign if it happens in a role playing scenario. Beware also of the fact that if the Evil Overlord is mostly chillin' like a villain with his family and having brews with his bros, it's going to be pretty hard to justify it when you inevitably stab him right in the face. Also remember that while The Banality of Evil is great for mysteries, it's actually so good for mysteries that the game can bog down. Players can get caught up in the minor goings-on of characters you don't even care about. Paranoia can be paralyzing when any scullery maid could really just go Evil at any time and poison your food to try to get your wallet. It can be realistic, but realism takes place in real time. That's not good if you're trying to raise hippogriffs as steeds.}

\subsubsection{Moral Option 3: The Face of Horror.}
\vspace*{-8pt}
\quot{``I think I have Evil sand. In my pants.''}

Many DMs will want to make their Evil as \textbf{Evil} as possible. That can get\ldots\  pretty Evil. It can actually get so Evil that people who \textit{overhear} you playing the game will get a very bad impression about your group and the things you talk about. The starker the contrast between Good and Evil, the more righteous the acts of heroism the players commit. Tales of monstrous action are fascinating and the horrid and disgusting can hold people's interest indefinitely. By having the forces of Evil disembowel people in loving detail you can capture the imaginations of your players with actually relatively little creative work on the part of the DM. There have been over 10 Jason movies because those things practically write themselves, and people keep watching them because they genuinely are as intriguing as the are revolting.

\ability{Implications:}{ With the forces of Evil running around doing actual stomach churning crime, having Evil and Good ``team up" is essentially implausible. In fact, having Good and Evil characters in the same party is pretty much a non-starter. When playing with The Face of Horror the universe is essentially a cosmic battle between Good and Evil, the forces of Law and Chaos have some fights too, but essentially that's just crime compared to the world shaking conflict of darkness and light.}

Further, while Good and Evil being as immiscible as Rubidium and Water makes for a well defined party demographic, it also has other far reaching consequences. When you go to the Abyss, the sand itself is Evil. Once you've made the determination that this means more than that Paladins can find every grain -- you've bought yourself into the determination that beaches in the Abyss are themselves morally reprobate somehow.

\ability{Pit Falls:}{ While The Face of Horror ends up making Good and Evil a much more important distinction than Law vs. Chaos, that's not really a \textit{problem}. Sure, it's not reciprocal or equivalent and that's a breach of the Great Wheel tirade, but that's not really important to the game. Let's face it, when was the last time you saw a statted up enemy prepared to cast \spell{dictum}? No, the problem is that if you make Evil as nasty as it can be made, it's really nasty. It makes other people in the game uncomfortable, and it disturbs people who hear portions of your game out of context. People like talking about stabbing their sword into an evil monster, that's all heroic and crap, but actually looking at sword wounds is icky. People don't want to do it.}

Evil, if defined as ``things we don't like," is pretty much exclusively composed of things we \textit{don't like}. That means that the more we focus our attention on the details of what's going on, the more we'll want to clean our eyes out with soap. And while skirting that line can make a story grimly compelling, remember always that different people have different tolerances for this sort of thing. Just because something is gross enough to catch your prurient interest without wrecking your lunch doesn't mean that it isn't so nasty as to drive other people away. Tolerance for discussing child murder in the context of a story is not a virtue, and there is absolutely nothing wrong with the people who don't enjoy watching movies in the splatter horror genre.

\subsubsection{Moral Option 4: Perfection in Balance}
\vspace*{-8pt}
\quot{``What use is the light that casts not a shadow?''}

In this model, evil is a force that sits diametrically opposed to good. In order for one to exist, the other must exist as well. Evil is what gives good its meaning, and in fact one can simply define one by the other: to be good is not-evil, and to be evil is to be not-good. When playing with this option, evil plays a vital role in society and cannot be eliminated without dire consequences. For example, when the Jedi eliminate the Sith Lords, they set themselves up for an even more powerful Sith Lord to rise and kill them all, ushering in a new order of Evil, which is in turn later demolished by the calling out of a powerful Jedi who can defeat it. Neutrality is the rule of the day in this model, in the sense that evil and good will always be in the midst of trumping each other in an effort to ``win'', a goal that is as meaningless as it is impossible.

What does that mean for your game? In this model, evil will always be the fly in your ointment and the piss in your cheerios, and good will always be the silver lining in the stormcloud and the complementary bag of nuts in your red-eye flight. Even the most powerful and good organization of clerics in your world will have a cruel inquisitor, and even the most death-hungry cabal of necromancers will have a guy who is kind to puppies and little children. Organizations and people will be ``mostly'' one thing or the other, but not all of anything, and people will be OK with that. Kind kings will be mostly good, but will have no problem massacring an entire generation of goblinkind in an effort to keep the roads safe, and liches who eat souls will defend the land from rampaging chimera without reward in an effort to keep the peace.

\ability{Implications:}{ In a sense, this is the easiest of moral options, as you won't need to really keep track of what's going on with alignments. People will occasionally do things out of character, and that's fine. Society will be quite tolerant, as they completely think its OK for there to be a Temple Street with a shrine for Orcus worshippers competing for space with a hospital sponsored by the clergy of Pelor. When one organization for good or evil gets stomped down, another one will pop up to replace it in an endless game of cosmic whack-a-mole.}

For character with alignment related class features, \spell{atonement} is a far easier process. Occasional deeds that violate your alignment are tolerated, as long as attempts at acts of \spell{atonement} are made in a reasonable time frame. The Paladin that kills an innocent to defeat a powerful demon may have to visit the innocent's family and make restitution after the battle, and the Cleric of Murder who defends the king from an assassin may have to seek out several of the King's loved one's in order to rededicate himself to his dark god.

\ability{Pit Falls:}{ It can be pretty cool to have a party that has an assassin, a druid, and a champion of light in it -- there's a lot of early D\&D that has that as virtually the iconic party -- but if the great game between Good and Evil is an inherently \textit{pointless} game, that can make the story of your characters seem pretty banal. It's a line that can be hard to walk. It's just plain difficult to simultaneously have any individual attempt to destroy the world be important while having it be built into the contract that there will be another one tomorrow.}



\subsection{To Triumph Over Evil}

Equally important to the place of ultimate Evil in your game is the activities of Good in your game. Like Evil, the designers have tried to run the spectrum of possible interpretations of righteousness\ldots\  and the results are that the overlap of actions depicted as Good with those described as Evil is almost total. Ultimately, your campaign is going to have to come to a consensus over what you are going to accept as Good. Most importantly, the inverse of Evil \textit{is not Good}. It really takes a lot less harm to be Evil than it takes aid to be Good. If you fix twenty people's roofs, you're Jimmy the Helpful Thatcher. But if you eat your neighbor's daughter, you're Jimmy the Cannibal -- and no additional carpentry assistance will change that. This is why the \underline{Book of Exalted Deeds} is such an unsatisfying read\ldots\  you can't just take the material in the \underline{Book of Vile Darkness} and multiply by negative one to get Good.


\subsubsection{The Importance of Consequentialism}
Every action has motivations, expectable results, and actual results. In addition, every action can be described with a verb. In the history of moral theory (a history substantively longer than \textit{human} history) it has at times been contested by otherwise bright individuals that any of those (singly or collectively) could be used as a rubric to determine the rightness of an action. D\&D authors agreed. With all of those extremely incompatible ideas. And the result has been an unmitigated catastrophe. No one knows what makes an action Good in D\&D, so your group is ultimately going to have to decide for yourselves. Is your action Good because your intentions are Good? Is your action Good because the most likely result of your action is Good? Is your action Good because the actual end result of that action is Good? Is your action Good because the verb that bests describes your action is in general Good? There are actually some very good arguments for all of these written by people like Jeremy Bentham, Immanuel Kant, and David Wasserman -- but there are many other essays that are so astoundingly contradictory and ill-reasoned that they are of less help than reading nothing. Unfortunately for the hobby, some of the essays of the second type were written by Gary Gygax.

This is not an easy question to answer. The rulebooks, for example, are no help at all. D\&D at its heart is about breaking into other peoples' homes, stabbing them in the face, and taking all their money. That's very hard to rationalize as a Good thing to do, and the authors of D\&D have historically not tried terribly hard.

\subsubsection{Godliness isn't Goodliness}
Whatever religion you personally have, the religion in D\&D revolves around a set of gods both Good and Evil of equal strength and importance. Most modern day religions have however many gods they worship be of sufficient goodness that they are at least worthy of respect -- so it can be hard to remember that in D\&D the gods as a whole are precisely zero sum on any issue. Being ``divine" doesn't make you Good in D\&D, it just makes you more. If you're Good it makes you more Good, but if you're Evil it makes you more Evil. Clerics detect strongly of whatever alignment they have, but there's nothing Good about priests as a whole. Turning your back on the gods isn't a bad thing in D\&D, it's a perfectly valid \textit{and neutral} choice. If Ur Priests are to have any alignment restriction at all, it should be the same as Druids -- stealing from the gods is a profoundly neutral act, not Good and not Evil.

\subsubsection{There is no Salvation or Redemption in D\&D}
All of the major religions of our world that utilize the concepts of Ultimate Good and Ultimate Evil use the concept of Redemption (that people have a state of innocence that they can lose and perhaps regain through \spell{atonement}) or Salvation (that people have a state of inherent unworthiness that they can overcome). D\&D, despite having a spell called \spell{atonement} actually has neither of those concepts. The \spell{atonement} spell actually dedicates (or rededicates) a character to any alignment, Good or Evil, Law or Chaos. Baby kobolds are not born into original sin and baby elves are not born in a state of grace, D\&D doesn't even have those concepts. Creatures with an alignment subtype (most Fiends, for example) are born into that alignment and are only going to stray from it if subjected to powerful magic or arguments. Everyone else is born neutral.

In D\&D, creatures do not ``fall" into Evil. Being Evil is a valid choice that is fully supported by half the gods just as Good is. Those who follow the tenets of Evil throughout their lives are judged by \textit{Evil Gods} when they die, and can gain rewards at least as enticing as those offered to those who follow the path of Good (who, after all, are judged by \textit{Good Gods} after they die). So when sahuagin run around on land snatching children to use as slaves or sacrifices to Baatorians, they aren't putting their soul in danger. They are actually keeping their soul \textit{safe}. Once you step down the path of villainy, you get a \textit{better deal} in the afterlife by being \textit{more evil}.

The only people who get screwed in the D\&D afterlife are traitors and failures. A traitor gets a bad deal in the afterlife because whichever side of the fence they ended up on is going to remember their deeds on the other side of the fence. A failure gets a bad deal because they end up judged by gods who wanted them to succeed. As such, it is \textit{really hard} to get people to change alignment in D\&D. Unless you can otherwise assure that someone will die as a failure to their alignment, there's absolutely no incentive you could possibly give them that would entice them to betray it.

\subsubsection{Code of Conduct: Paladins}
Nothing causes more arguments in-game than Paladins. Can Paladins kill baby kobolds? What about baby mind flayers? Honestly, while these questions have generated a lot of ink and a lot of bad feelings, they aren't important. Paladins are Lawful Good, but they aren't ``champions of Law and Good" -- that's an Archon. A Paladin doesn't get Smite \textit{Chaos}, they aren't forced to abandon team members who behave in a Chaotic fashion (whatever that means, see below). Paladins are Champions of Good\texttrademark\ \textit{and} they are required to be Lawful. Whether or not that makes any sense depends on how you're handling Law and Chaos.

Paladins are as Good as any character can be, and they are required to follow a code of conduct. However, following this code is no what makes them Good, we know this because Clerics of Good (who detect as being just as Good as Paladins) don't have to follow that code. The code is completely arbitrary, and has no bearing on the relative Goodness of a character. Paladins also lose their powers if they don't drink for a few days, but that doesn't put Blackguards in danger of losing their alignment when they quaff a glass of water.

The Paladin's code is uncompromising, but it is also exhaustive about what it won't allow:

\listone
    \item  \ability{The Use of Poison:}{If a park ranger hits a bear with a tranq dart, that's not an Evil act. Poison isn't any more or less Evil than a blade. Paladins can't use poison because they agreed not to -- not because there's anything wrong with poison. Maybe Paladins only get to keep their magically enhanced immune system so long as they don't take it for granted by using things that would tax it on purpose. Maybe their concern for public safety is so great that they are only willing to use weapons that \textit{look} like weapons. Whatever. The point is that Paladins have to be Good \textit{and} they can't use Poison, and these are separate restrictions.}
    \item  \ability{Lies:}{A Paladin can't lie. Whether telling a lie is a good or evil act depends on what you're saying and who you are saying it to. But a Paladin won't do it. That means that if the Nazis come to the door and demand to know if the Paladin is hiding any Jews (she is), she can't glibly say ``No." That does not mean that she has to say ``Yes, they're right under the stairs!" -- it means that she has to tell the Nazis point blank ``I'm not going to participate in your genocidal campaign, it's wrong." This will start a fight, and may get everyone killed, so the Paladin is well within her code to eliminate the middle man and just stab the Gestapo right there before answering. That's harsh, but the Paladin's code isn't about doing what's easy, or even what's \textit{best}. It's about doing what you said you were going to.}
    \item  \ability{Cheating:}{Paladin's don't cheat. They don't have to keep playing if they figure out that someone else is cheating, but they aren't allowed to cheat at dice to rescue slaves or whatever. Again, there's nothing Good about not cheating, it's just something they have to do \textit{in addition} to being Good all the time.}
    \item  \ability{Association Restrictions:}{Paladins are not allowed to team up with Evil people. They aren't allowed to offer assistance to Evil people and they aren't allowed to receive assistance \textit{from} Evil people. Intolerance of this sort isn't Evil, but it isn't Good either. It's just another thing that Paladins have to do.}
\end{list}



\subsection{Law and Chaos: Your Rules or Mine?}

Let's get this out in the open: Law and Chaos do not have any meaning under the standard D\&D rules.\\

We are aware that especially if you've been playing this game for a long time, you personally probably have an understanding of what you \textit{think} Law and Chaos are supposed to mean. You possibly even believe that the rest of your group thinks that Law and Chaos mean the same thing you do. But you're probably wrong. The nature of Law and Chaos is the source of more arguments among D\&D players (veteran and novice alike) than any other facet of the game. More than attacks of opportunities, more than weapon sizing, more even than spell effect inheritance. And the reason is because the ``definition" of Law and Chaos in the Player's Handbook is written so confusingly that the terms are not even mutually exclusive. Look it up, this is a written document, so it's perfectly acceptable for you to stop reading at this time, flip open the Player's Handbook, and start reading the alignment descriptions. The Tome of Fiends will still be here when you get back.

There you go! Now that we're all on the same page (page XX), the reason why you've gotten into so many arguments with people as to whether their character was Lawful or Chaotic is because absolutely every action that any character ever takes could logically be argued to be \textbf{both}. A character who is honorable, adaptable, trustworthy, flexible, reliable, and loves freedom is a basically stand-up fellow, and meets the check marks for being ``ultimate Law" \textit{and} ``ultimate Chaos". There aren't any contradictory adjectives there. While Law and Chaos are \textit{supposed} to be opposed forces, there's nothing antithetical about the descriptions in the book.


\subsubsection{Ethics Option 1: A level of Organization.}
Optimal span of control is 3 to 5 people. Maybe Chaotic characters demand to personally control more units than that themselves and their lack of delegation ends up with a quagmire of incomprehensible proportions. Maybe Chaotic characters refuse to bow to authority at all and end up in units of one. Whatever the case, some DMs will have Law be well organized and Chaos be poorly organized. In this case, Law is objectively a virtue and Chaos is objectively a flaw.

Being disorganized doesn't mean that you're more creative or interesting, it just means that you accomplish less with the same inputs. In this model pure Chaos is a destructive, but more importantly \textit{incompetent} force.

\subsubsection{Ethics Option 2: A Question of Sanity.}
Some DMs will want Law and Chaos to mean essentially ``Sane" and ``\textit{In}sane". That's fine, but it doesn't mean that Chaos is \textit{funny}. In fact, insanity is generally about the least funny thing you could possibly imagine. An insane person reacts inappropriately to their surroundings. That doesn't mean that they perform \textit{unexpected} actions, that's just surrealist. And Paladins are totally permitted to enjoy non sequitur based humor and art. See, insanity is when you perform the same action over and over again and expect different results.

In this model we get a coherent explanation for why, when all the forces of Evil are composed of a multitude of strange nightmarish creatures, and the forces of Good have everything from a glowing patch of light to a winged snake tailed woman, every single soldier in the army of Chaos is a giant frog. This is because in this model Limbo is a place that is \textit{totally insane}. It's a place where the answer to every question \textit{really is} ``Giant Frog". Creatures of Chaos then proceed to go to non Chaotically-aligned planes and are disappointed and confused when doors have to be pushed and pulled to open and entrance cannot be achieved by ``Giant Frog".

If Chaos is madness, it's not ``spontaneous", it's ``non-functional". Actual adaptability is \textbf{sane}. Adapting responses to stimuli is what people are supposed to do. For reactions to be sufficiently inappropriate to qualify as insanity, one has to go pretty far into one's own preconceptions. Actual mental illness is very sad and traumatic just to watch as an outside observer. Actually living that way is even worse. It is strongly suggested therefore, that you don't go this route at all. It's not that you can't make D\&D work with sanity and insanity as the core difference between Law and Chaos, it's that in doing so you're essentially making the Law vs. Chaos choice into the choice between good and bad. That and there is a certain segment of the roleplaying community that cannot differentiate absurdist humor from insanity and will insist on doing annoying things in the name of humor. And we hate those people.

\subsubsection{Ethics Option 3: The Laws of the Land.}
Any region that has writing will have an actual code of laws. Even oral traditions will have, well, \textit{traditions}. In some campaigns, following these laws makes you Lawful, and not following these laws makes you Chaotic. This doesn't mean that Lawful characters necessarily have to follow the laws of Kyuss when you invade his secret Worm Fort, but it does mean that they need to be an ``invading force" when they run around in Kyuss' Worm Fort. Honestly, I'm not sure what it even \textit{means} to have a Chaotic society if Lawful means ``following your own rules". This whole schema is workable, but only with extreme effort. It helps if there's some sort of divinely agreed upon laws somewhere that nations and individuals can follow to a greater or lesser degree. But even so, there's a lot of hermits and warfare in the world such that whether people are following actual laws can be just plain hard to evaluate.

I'd like to endorse this more highly, since any time you have characters living up to a specific arbitrary code (or not) it becomes a lot easier to get things evaluated. Unfortunately, it's really hard to even imagine an entire nation fighting for not following their own laws. That's just\ldots\  really weird. But if you take Law to mean law, then you're going to have to come to terms with that.

\subsubsection{Ethics Option 4: My Word is My Bond.}
Some DMs are going to want Law to essentially equate to following through on things. A Lawful character will keep their word and do things that they said they were going to. In this model, a Lawful character has an arbitrary code of conduct and a Chaotic character does not. That's pretty easy to adjudicate, you just announce what you're going to do and if you \textit{do it}, you're Lawful and if you \textit{don't} you're not.

Here's where it gets weird though: That means that Lawful characters have a \textit{harder time} working together than do non-lawful characters. Sure, once they agree to work together there's some Trust there that we can capitalize, but it means that there are arbitrary things that Lawful characters won't do. Essentially this means that Chaotic parties order one mini-pizza each while Lawful parties have to get one extra large pizza for the whole group -- and we know how difficult that can be to arrange. A good example of this in action is the Paladin's code: they won't work with Evil characters, which restricts the possibilities of other party members.

In the world, this means that if you attack a Chaotic city, various other chaotic characters will trickle in to defend it. But if you attack a Lawful city, chances are that it's going to have to stand on its own.

\subsubsection{Adherence to Self: Not a Rubric for Law}
Sometimes Lawfulness is defined by people as adhering to one's personal self. That may \textit{sound} very ``Lawful'', but there's no way that makes any sense. Whatever impulses you happen to have, those are going to be the ones that you act upon, \textit{by definition}. If it is in your nature to do random crap that doesn't make any sense to anyone else -- then your actions will be contrary and perplexing, but they will still be completely consistent with your nature. Indeed, there is literally nothing you can do that isn't what you would do. It's circular.

\subsubsection{Rigidity: Not a Rubric for Law}
Sometimes Lawfulness is defined by people as being more ``rigid'' as opposed to ``spontaneous'' in your action. That's crap. Time generally only goes in one direction, and it generally carries a one to one correspondence with itself. That means that as a result of a unique set of stimuli, you are \textit{only going to do one thing}. In D\&D, the fact that other people weren't sure what the one thing you were going to do is handled by a Bluff check, not by being Chaotic.


\subsection{I Fought the Law}

Regardless of what your group ends up meaning when they use the word ``Law'', the fact is that some of your enemies are probably going to end up being Lawful. That doesn't mean that Lawful characters can't stab them in their area, whatever it is that you have alignments mean it's still entirely acceptable for Good characters to stab other Good characters and Lawful characters to stab other Lawful characters (oddly, no one even asks if it's a violation of Chaotic Evil to kill another Chaotic Evil character, but it isn't). There are lots of reasons to kill a man, and alignment disagreements don't occupy that list exclusively.


\subsubsection{Code of Conduct: \hyperref[class:barbarian]{Barbarian}}
A Barbarian who becomes Lawful cannot Rage. Why not? There's no decent answer for that. Rage doesn't seem to require that you not tell people in advance that you're going to do it, nor does it actually force you to break promises once you're enraged. It doesn't force you to behave in any particular fashion, and no one knows why it is restricted.

\subsubsection{Code of Conduct: Bard}
If \textit{anyone} can tell me why a concert pianist can't be Lawful I will personally put one thing of their choice into my mouth. Music is expressionistic, but it is also mathematical. Already there are computers that can write music that is indistinguishable from the boring parts of Mozart in which he's just going up and down scales in order to mark time.

\subsection{Beating Back Chaos}
Long ago ``Law'' and ``Chaos'' were used euphemisms by Pohl Anderson for Good and Evil, and that got taken up by other fantasy and science fiction authors and ultimately snow-balled into having a Chaos alignment for D\&D. If you go back far enough, ``Chaos" actually \textit{means} ``The Villains", and when it comes down to it there's no logical meaning for it to have other than that -- so the forces of Chaos really are going to show up at your door to take a number for a whuppin at some point. Depending upon what your group ends up deciding to mean by Chaos, this may seem pretty senselessly cruel. If the forces of Chaos are simply unorganized then you are essentially chasing down hobos and beating down the ones too drunk to get away. If Chaos is insanity than the Chaos Hunters in your game are essentially going door to door to beat up the retarded kids.

The key is essentially to not overthink it. Chaos was originally put into the fantasy genre in order to have bad guys without having to have black hatted madmen trying to destroy the world. So if Team Chaos is coming around your door, just roll with it. The whole point is to have villains that you can stab without feeling guilty while still having villains to whom your characters can \textit{lose} without necessarily losing the whole campaign world.

\subsubsection{Code of Conduct: \hyperref[class:knight]{Knight}}
Sigh. The Knight's code of conduct doesn't represent Lawful activity no matter what your group means by that term. They \textit{can't} strike an opponent standing in a grease effect, but they \textit{can} attack that same person \textit{after they fall down in the grease!} They also are not allowed to win a duel or stake vampires (assuming, for the moment that you were using some of the house rules presented in The Tome of Necromancy that allow vampires to be staked by \textit{anyone}). So the Knight's code is not an example of Lawfulness in practice, it's just a double fistful of stupid written by someone who obviously doesn't understand D\&D combat mechanics. If you wanted to make a Knight's Code that represented something like ``fighting fair'', you'd do it like this:

\listone
    \item May not accept benefit from Aid Another actions.
    \item May not activate Spell Storing items (unless the Knight cast the spell into the item in the first place).
    \item May not use poison or disease contaminated weapons.
\end{list}

But remember: such a code of fair play is no more Lawful than \textit{not} having a code of fair play. Formians are the embodiment of Law, and they practically wrote the book on cooperation. So while a Knight considers getting help from others to be ``cheating", that's not because he's Lawful. He considers getting such aid to be cheating \textit{and} he's Lawful. What type of Lawful a Knight represents is determined by your interpretation of Law as a whole. Maybe a Knight has to uphold the law of the land (right or wrong). Maybe a Knight has to keep his own word. Whatever, the important part is that the arbitrary code that the Knight lives under is just that -- \textit{arbitrary}. The actual contents of the code are a separate and irrelevant concern to their alignment restriction.

\subsubsection{Code of Conduct: \hyperref[class:monk]{Monk}}
Again with the sighing. No one can explain why Monks are required to be Lawful, least of all the Player's Handbook. Ember is Lawful because she ``follows her discipline", while Mialee is not Lawful because she is ``devoted to her art". FTW?! That's the same thing, given sequentially as an example of being Lawful and not being Lawful. Monk's training requires strict discipline, but that has nothing to do with Lawfulness no matter what setup for Law and Chaos you are using. If Lawfulness is about organization, you are perfectly capable of being a complete maverick who talks to no one and drifts from place to place training constantly like the main character in Kung Fu -- total lack of organization, total ``Chaotic" -- total disciplined Monk. If Law is about Loyalty, you're totally capable of being treacherous spies. In fact, that's even an example in the PHB ``Evil monks make ideal spies, infiltrators, and assassins." And well, that sentence pretty much sinks any idea of monks having to follow the law of the land or keeping their own word, doesn't it? The only way monk lawfulness would make any sense is if you were using ``adherence to an arbitrary self" as the basis of Law, and we already know that can't hold.

\subsubsection{Code of Conduct: Paladin Again}
This has to be repeated: Paladins don't get Smite Chaos. They are not champions of Law and Good, they are Champions of Good who are required to be Lawful. If your game \underline{is not} using Word is Bond Ethics, Paladins have no reason to be Lawful. Paladins are only encouraged to follow the laws of the country they live in if those laws are Good. They are actually forbidden by their code of conduct from following the precepts of Evil nations. The Paladin shtick works equally well as a loner or a leader, and it is by definition distinctly disloyal. A Paladin must abandon compatriots.
