\classname{Ranger} \label{comm:class:ranger}
\vspace*{-8pt}
\quot{``..."}
The Ranger excels at two things; combat on his own terms, and making combat happen on his own terms. Though lacking in social graces, they are champions of nature, and the wilderness itself follows them into combat whenever necessary.

Note: The Ranger featured here is a forest-themed Ranger. If your Ranger makes his home somewhere else, feel free to change forest/tree-themed abilities to represent some other kind of environment (I dunno, sand dunes for desert-themed Rangers, or big rocks for mountain-themed Rangers, or Richard Simmons for... nevermind).

\ability{Alignment:}{Any, though the impartialness of most Rangers towards the concerns of �civilised� creatures causes most to lean towards neutrality.}

\ability{Starting Gold:}{4d4*10 gp (100 gold)}

\ability{Starting Age:}{As Ranger (Moderate).}

\ability{Hit Die:}{d8}

\ability{Class Skills:}{The Ranger's class skills (and the key ability for each skill) are Acrobatics (Dex), Athletics (Str), Awareness (Wis), Concentration (Con), Craft (Int), Handle Animal (Cha), Knowledge (arcana), Knowledge (dungeon), Knowledge (geography), Knowledge (nature), Knowledge (the planes) (Int), Profession (Wis), Stealth (Dex), Survival (Wis).}

\ability{Skills/Level:}{6 + Intelligence Bonus}

BAB: Good
Saves: Fort: Poor; Ref: Good; Will: Good 

1. Hunter�s Knowledge, Terrain Mastery, Track, Weapon Finesse
2. Hunter�s Knowledge +1d6, Rapid Shot, Two-Weapon Fighting, Terrain Mastery
3. Nature�s Ally, Terrain Mastery
4. Hunter�s Knowledge +2d6, Entangle, Terrain Mastery
5. Giant Slayer, Terrain Mastery
6. Hunter�s Knowledge +3d6, Nature�s Fortitude, Terrain Mastery
7. Improved Rapid Shot, Terrain Mastery
8. Hunter�s Knowledge +4d6, Mother�s Embrace, Terrain Mastery
9. Ghost Hunter, Terrain Mastery
10. Deep Strike, Hunter�s Knowledge +5d6, Terrain Mastery
11. Planer Terrain Mastery, Tree Stride
12. Greater Rapid Shot, Hunter�s Knowledge +6d6, Planer Terrain Mastery
13. Improved Nature�s Fortitude, Planer Terrain Mastery
14. Make Like A Tree, Hunter�s Knowledge +7d6, Planer Terrain Mastery
15. Evergreen Glade, Planer Terrain Mastery
16. Arrow Storm, Hunter�s Knowledge +8d6, Planer Terrain Mastery
17. Planer Terrain Mastery, Superior Rapid Shot
18. Hunter�s Knowledge +9d6, Paralysis Strike, Planer Terrain Mastery
19. Champion Of The Wild, Planer Terrain Mastery
20. Hunter�s Knowledge +10d6, Planer Terrain Mastery

The saving throw DC for any of a Ranger�s abilities is 10+ \half class level + Wisdom modifier. If a Ranger gains a bonus feat that he already has, he may instead choose to gain any [Combat] or [Skill] feat for which he meets the prerequisites.

\ability{Weapon and Armour Proficiency:}{A Ranger is proficient with all simple and martial weapons, the Greatbow and Composite Greatbow, light armour, and the buckler, but not other shields.}

\ability{Hunter�s Knowledge (Ex):}{A Ranger can spend a swift action to attempt to identify a creature currently in combat with him. Make a monster identification check as normal. If successful, the Ranger gains a +2 insight bonus to hit and damage against the creature, as well as a bonus 1d6 damage per 2 levels (rounded down). If a Ranger is facing a creature of a kind (but not necessarily type) that he has previously identified (for example, if a Ranger encounters a Hill Giant and has previously identified a Hill Giant, but not if he encounters a Fire Giant), the DC of this roll is reduced by 10.
A Ranger may use Knowledge (geography) to identify humanoids and Knowledge (the planes) to identify undead with this ability, with a DC of 10 + (creature's CR), plus or minus any normal identification modifiers.}

\ability{Terrain Mastery (Ex):}{At each level up to 10th, a Ranger gains an ability from the following list.}

\begin{itemize}
\item{\spell{Forest:}{You gain a +3 bonus to Stealth checks. In addition, as long as you are in a natural setting, you gain the Hide in Plain Sight ability.}

\item{\spell{Plains:}{You gain a +3 bonus to Awareness checks. In addition, your natural land speed is increased 50\%.}

\item{\spell{Mountain:}{You gain a +3 bonus to Knowledge (arcana) checks. In addition, you gain a Climb speed equal to ? your natural land speed (if you already have a Climb speed, you gain a +8 to Athletics checks to perform a special action or avoid a hazard while climbing; this is in addition to the normal +8 gained from having a Climb speed).}

\item{\spell{River:}{You gain a +3 bonus to Athletics checks. In addition, you gain a Swim speed equal to your natural land speed (if you already have a Swim speed, you gain a +8 to Athletics checks to perform a special action or avoid a hazard while swimming; this is in addition to the normal +8 gained from having a Swim speed).}

\item{\spell{Caverns:}{You gain a +3 bonus on Knowledge (dungeon) checks. In addition, you gain Darkvision out to 60ft, or increase your existing Darkvision by 60ft.}

\item{\spell{Jungle:}{You gain a +3 bonus to Knowledge (nature) checks. In addition, you may move at full normal speed without taking a penalty to Hide or Move Silently checks, may double-move at a penalty -5 instead of -10, and may run at a penalty of -10 instead of -20.}

\item{\spell{Hills:}{You gain a +3 bonus to Knowledge (geography) checks. In addition, you become immune to fatigue, and any effect that would cause you to become exhausted instead causes you to become fatigued. If you are already immune to fatigue or exhaustion, you instead gain 1 hit point per character level per effect you are immune to.}

\item{\spell{Swamp:}{You gain a +3 bonus to Acrobatics checks. In addition, you take no penalty for moving over rough terrain.}

\item{\spell{Ocean:}{You gain a +3 bonus to Knowledge (the planes) checks. In addition, you gain water-breathing and become immune to fear effects. If you are already immune to fear effects (or mind-affecting effects), you instead gain a +3 bonus to Will saves.}

\item{\spell{Desert:}{You gain a +3 bonus to Survival checks. In addition, you gain a Burrow speed equal to half your natural land speed.}

\end{itemize}

Terrain Mastery abilities are always active, not just when the Ranger is in that particular environment. If a Ranger is in an unfamiliar area (hard to define, I leave it to various DM�s and players to discuss), he loses access to his Terrain Mastery abilities unless he succeeds on a Knowledge (geography) check, with a DC equal to 10 + his class level. This check may be repeated once each day if necessary.

\ability{Track (Ex):}{A Ranger can track creatures using a Survival check. See the Survival skill in the PHB 3.5.}

\ability{Weapon Finesse:}{A Ranger gains Weapon Finesse as a bonus feat.}

\ability{Rapid Shot (Ex): Starting at 2nd level, a Ranger using a full-attack action with a bow may make an additional ranged attack. All attacks made by the Ranger that round suffer a -2 penalty.}

\ability{Two-Weapon Fighting:}{A 2nd level Ranger gains Two-Weapon Fighting as a bonus feat.}

\ability{Nature�s Ally (Ex):}{Starting at 3rd level, a Ranger gains a cohort. This cohort may be any animal, dragon, elemental, fey, magical beast, ooze, plant, or vermin with a CR equal to the Ranger�s level -2. Creatures with an Intelligence score of 2 or lower (including mindless creatures) have their Intelligence score changed to 3 and are considered to be able to understand one language the Ranger can speak; otherwise, the creature remains the same. A cohort gained through Nature�s Ally may be dismissed at any time, and a Ranger can gain a new cohort by spending 24 hours in meditation. A Ranger may only have one cohort from Nature's Ally at any time.
In addition, a Ranger may use his Wisdom modifier in place of his Charisma modifier for Handle Animal checks. A Ranger may use a Handle Animal check as a Diplomacy check against any creature with an Intelligence score of 1 or 2.}

\ability{Entangle (Sp):}{Starting at 4th level, a Ranger can use Entangle as a spell-like ability once every 5 rounds, with a caster level equal to his character level.}

\ability{Giant Slayer:}{A 5th level Ranger gains Giant Slayer as a bonus feat.}

\ability{Nature�s Fortitude (Ex):}{Starting at 6th level, a Ranger may use his Will save in place of his Fortitude save against poisons and diseases (natural or otherwise).}

\ability{Improved Rapid Shot (Ex):}{Starting at 7th level, a Ranger using a full-attack action with a bow may make two additional ranged attacks instead of one, all at his highest BAB. All attacks made by the Ranger that round suffer a -2 penalty.}

\ability{Mother�s Embrace (Ex):}{Starting at 8th level, a Ranger who sleeps in a natural setting automatically recovers all wounds upon waking, so long as he has slept for at least four hours (Elven trance counts towards this, as well as any other form of rest).}

\ability{Ghost Hunter:}{A 9th level Ranger gains Ghost Hunter as a bonus feat.}

Deep Strike (Ex): Starting at 10th level, all attacks a Ranger makes with a bow are treated as if they had the Distance special quality. In addition, all melee attacks made by a Ranger have their reach increased by 5 feet.

Planar Terrain Mastery (Ex): At 11th level and each level thereafter, a Ranger gains one of the following abilities.

Fire: The Ranger gains ER: Fire equal to twice his character level, and may choose to have any attack deal half its damage as Fire damage.

Water: The Ranger gains ER: Cold equal to twice his character level, and may choose to have any attack deal half its damage as Cold damage.

Earth: The Ranger gains DR/- equal to ? his character level, and may choose to have any attack he makes treated as the material of his choice (iron, silver, adamantine, or wood).

Air: The Ranger gains ER: Electricity equal to twice his character level. In addition, the Ranger gains a fly speed equal to his natural land speed, with perfect manoeuverability.

Shadow: The Ranger can see perfectly in any darkness, including magical darkness, as far as he could see normally. In addition, the Ranger gains the Darkstalker ability (creatures attempting to locate you must make a Spot check; extrasensory detection abilities such as blindsight or tremorsense fail to automatically locate you).

Ethereal: The Ranger can become corporeal, incorporeal, or ethereal as he chooses as a full-round action.

Limbo: The Ranger can use Dimension Door as a swift action once per round, with a caster level equal to his character level. In addition, the Ranger can choose to have any attack he makes deal chaotic-aligned damage.

Mechanus: The Ranger can spend an immediate action to cause any extradimensional movement methods (including teleport and plane-shift) being used within line-of-sight to fail automatically (this can counter a creature attempting to enter the Ranger's line-of-sight through such a method). In addition, the Ranger can choose to have any attack he makes deal lawful-aligned damage.

Infernal: The Ranger can spend a standard action to create a terrifying scream. This scream is a burst-effect emitting from the Ranger, and has a radius of 5 feet per character level. All creatures within the affected area must make a Will wave or become panicked for one hour; creatures that succeed on their save become shaken instead. This is a sound-dependent, mind-affecting, fear effect. In addition, the Ranger can choose to have any attack he makes deal evil-aligned damage.

Celestial: The Ranger can spend a standard action to create a blinding flash of light. This flash is a burst-effect emitting from the Ranger, and has a radius of 5 feet per character level. All creatures within the affected area (except the Ranger) must make a Reflex save or be blinded; creatures that succeed on their save are dazed for one round. This is a sight-dependent effect. In addition, the Ranger can choose to have any attack he makes deal good-aligned damage.

A Ranger with a Planar Terrain Mastery ability that grants alignment damage is treated as being that alignment at any time it would be beneficial for him (For example, a Lawful-aligned Ranger with the Limbo Planar Terrain Mastery ability would be treated as Chaotic-aligned for the purposes of interacting with a Chaotic-aligned plane or item, and an Evil-aligned Ranger with the Celestial Planar Terrain Mastery ability would be treated as Good-aligned if subject to a Detect Evil or Holy Word spell).
Planar Terrain Mastery abilities are always active unless the Ranger chooses not to have them so (lowering or raising a Planar Terrain Mastery ability is a non-action).
Other Planar Terrain Mastery abilities may be available; talk to your DM about the other planes they may be using in their campaign to find something suitable.

Tree Stride (Ex): Starting at 11th level, a Ranger may move instantly to anywhere he wishes, so long as the place he is going and the place he is departing from both have a tree and he is at least vaguely familiar with the destination (has been there once, has seen a map, etc). The tree can be of any size, but it can�t be a flower or a bush. It has to be a tree. The Ranger can take a number of people with him equal to ? his class level (rounded up).

Greater Rapid Shot (Ex): Starting at 12th level, a Ranger using a full-attack action with a bow may make three additional shots rather than two, all at his highest BAB. All attacks made by the Ranger that round suffer a -2 penalty.

Improved Nature�s Fortitude (Ex): Starting at 13th level, a Ranger is immune to all poisons and diseases. In addition, a Ranger may use his Will save in place of his Fortitude save against the extraordinary, spell-like, and supernatural abilities of any creature he has successfully identified using Hunter�s Knowledge.

Make Like A Tree (Ex): Starting a 14th level, a Ranger has the ability to grow a tree anywhere he wishes in an instant. To do this, he must have a piece of wood (it can be any piece of wood, such as a twig or a staff, so long as it is all wood) and spend a move action to place it into the ground. The tree created through this ability remains so long as the Ranger that created it is within line-of-sight; afterwards, it returns to being the former piece of wood.

Evergreen Glade: A 15th level Ranger is given a sanctuary where he may rest from the trials of his lifestyle. The Evergreen Glade is a demi-plane that may only be accessed by the Ranger's Tree Stride ability; only the Ranger belonging to the glade may reach it, unless he chooses to bring others with him. The Evergreen Glade is considered natural terrain for any Ranger abilities that rely on such, but only for the Ranger it was made for. A Ranger can choose to banish (as the Banishment spell) any creature from his glade as a non-action, with no save. A Ranger who dies in his glade is brought back to life one day later, as per True Resurrection.

Arrow Storm (Ex): Starting at 16th level, a Ranger can spend a full-round action to attack every enemy within his first range increment, up to a maximum of his character level. A Ranger makes a single attack roll, and then compares it to the AC of each target. Any successful hit carries all relevant bonuses, including the Hunter�s Knowledge bonus.

Superior Rapid Shot (Ex): Starting at 17th level, a Ranger using a full-attack action with a bow may make four additional ranged attacks instead of three, all at his highest BAB. All attacks made by the Ranger that round suffer a -2 penalty.

Paralysis Strike (Ex): Starting at 18th level, a Ranger can choose to have a foe damaged by his Hunter�s Knowledge ability be forced to make a Fortitude save or become paralysed for 5 rounds. A Ranger may use this ability only once per round.

Champion Of The Wild (Ex): A 19th level Ranger is nature's greatest warrior, and his form changes to reflect this. His skin grows tough like bark, and his body becomes inhuman. A Ranger with this ability gains immunity to mind-effects, immunity to sleep, paralysis, polymorph, stunning, ability damage or drain, and energy drain, and is no longer subject to critical hits or precision damage. The Ranger gets a natural armour bonus equal to 1/2 his character level (rounded down), and immunity to bludgeoning damage.



For those who prefer to use 'traditional' skills, here is the traditional skill list for this Ranger, and the relevant Terrain Mastery abilities.

Class Skills: Balance (Dex), Climb (Str), Concentration (Con), Craft (Int), Escape Artist (Dex), Handle Animal (Cha), Heal (Wis), Hide (Dex), Jump (Str), Knowledge (arcana), Knowledge (dungeoneering), Knowledge (geography), Knowledge (nature), Knowledge (the planes) (Int), Knowledge (religion) (Int), Listen (Wis), Move Silently (Dex), Profession (Wis), Ride (Dex), Search (Int), Spot (Wis), Survival (Wis), Swim (Str), Tumble (Dex).
Skills/Level: 6 + Intelligence Bonus

Terrain Mastery (Ex): At each level up to 10th, a Ranger gains an ability from the following list.

Forest: You gain a +3 to Hide checks. In addition, as long as you are in a natural setting (a place that�s not a city, town, or dungeon, usually), you gain the Hide in Plain Sight ability.

Plains: You gain a +3 to Spot checks. In addition, your natural land speed is increased by 50%.

Mountain: You gain a +3 bonus to Climb checks. In addition, you gain a Climb speed equal to ? your natural land speed (if you already have a Climb speed, you gain a +8 to Climb checks to perform a special action or avoid a hazard; this is in addition to the normal +8 gained from having a Climb speed).

River: You gain a +3 to Swim checks. In addition, you gain a Swim speed equal to your natural land speed (if you already have a Swim speed, you gain a +8 to Swim checks to perform a special action or avoid a hazard; this is in addition to the normal +8 gained from having a Swim speed).

Caverns: You gain a +3 bonus on Tumble checks. In addition, you gain Darkvision out to 60ft, or increase your existing Darkvision by 60ft.

Jungle: You gain a +3 bonus to Move Silently checks. In addition, you may move at full normal speed without taking a penalty to Hide or Move Silently checks, may double-move at a penalty -5 instead of -10, and may run at a penalty of -10 instead of -20.

Hills: You gain a +3 bonus to Listen checks. In addition, you become immune to fatigue, and any effect that would cause you to become exhausted instead causes you to become fatigued. If you are already immune to fatigue or exhaustion, you instead gain 1 hit point per character level per effect you are immune to.

Swamp: You gain a +3 bonus to Escape Artist checks. In addition, you take no penalty for moving over rough terrain.

Ocean: You gain a +3 to Balance checks. In addition, you gain water-breathing and become immune to fear effects. If you are already immune to fear effects (or mind-affecting effects), you instead gain a +3 bonus to Will saves.

Desert: You gain a +3 to Survival checks. In addition, you gain a Burrow speed equal to half your natural land speed.

Terrain Mastery abilities are always active, not just when the Ranger is in that particular environment. If a Ranger is in an unfamiliar area (hard to define, I leave it to various DM�s and players to discuss), he loses access to his Terrain Mastery abilities unless he succeeds on a Knowledge (geography) check, with a DC equal to 10 + his class level. This check may be repeated once each day if necessary. 