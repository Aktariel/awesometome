\classname{Monster Tamer} \label{class:monstertamer}
\vspace{-8pt}
\quot{"I choose you!"}

\ability{Playing a Monster Tamer:}{Monster Tamers do not make good front line fighters, although their short range thrown weapons can be devastating. They frequently need Fighting characters to soften up powerful Monsters for capture as well as to distract powerful Monsters long enough for a Monster Tamer to capture it. Sometimes a Monster Tamer will be attacked by creatures or adversity that are not Monsters, in such cases the abilities of Wizards and Sorcerers are invaluable - a Monster Tamer's relative dominance over Monsters can allow conventional spellcasters to save their powers for use against non-Monster foes. Monster Tamers can eventually heal their own Monsters fairly effectively - thus limiting their use for Clerics, however they cannot heal themselves. As a result Monster Tamers are sometimes seen to be both cowardly and ungrateful by their non-Monster companions.}

\ability{Alignment:}{Most Monster Tamers have an extreme alignment, although many are kindly masters, others are vicious and cruel. Monster Tamers tend to shy away from neutrality as their constant battles of will with Monsters generally make them quite accustomed to choosing sides.}

\ability{Races:}{Monster Tamers are usually Human, although there is a sizable number of Halfling Monster Tamers as well. Monster Tamers are usually not well thought of in Elven communities, and many turn to the road. In the depths of the Dwarven mountain halls Monster Tamers are seen as a valuable method of removing dangerous Monsters from the caverns but are also frequently shunned if they are seen training their Monsters. Gnomes are more likely to be scholars of Monsters than to attempt to capture any themselves. Amongst the savage humanoids Monster Tamers are usually laughed at and scorned until they can capture something large enough to frighten compliance out of others.}

\ability{Religion:}{Monster Tamers have no special ties to particular deities. However, powerful Monster Tamers have significant dealings with the outer planes - and many become Clerics. Gods of Elemental or Alignment domains are frequent choices - as are Gods of Plant or Animal.}

\ability{Background:}{Most Monster Tamers dedicate their lives to taming Monsters very early in life. Monster Tamers generally come from single parent homes or are orphans. Many Monster Tamers learn their skills because they love Monsters or are simply competitive - while others see Monsters as a relatively easy path to power and dominate their Monsters in order to fuel their lusts for eternal acquisitiveness. Such Monster Tamers may turn to theft or extortion to attempt to steal the Monsters of other Monster Tamers.}

\ability{Adventures:}{The life of a Monster Tamer naturally leads itself to adventure. Most Monster Tamers spend at least some of their time exploring in order to find and capture new Monsters and hone their skills.}

\ability{Starting Equipment:}{3d4x5 gp (37.5 Gold), one soul prism, one CR 1/2 `monster'}

\ability{Starting Age:}{Monster Tamers often begin their adventuring lives earlier than other classes, when determining starting age for a 1st level Monster Tamer simply choose the age at which that race becomes an `adult'.}

\ability{Hit Die:}{d6}

\ability{Class Skills:}{The Monster Tamer's class skills (and the key ability for each skill) are: Alchemy (Int), Animal Empathy (Cha), Bluff (Cha), Concentration (Con), Craft (Int), Handle Animal (Cha), Heal (Wis), Intimidate (Cha), Knowledge (Arcana - Int), Perform (Cha), Profession (Wis), Ride (Dex), Speak Language (special), Survival (Wis).}

\ability{Skill Points at 1st Level:}{(4 + Int modifier) x 4.}

\ability{Skill Points per level:}{4 + Int Modifier.}

\begin{table}[tbh]
\begin{small}
\begin{tabular}{lp{2cm}p{0.7cm}p{0.7cm}p{0.7cm}l}
Level  &Base Attack  Bonus &Fort Save &Ref Save &Will Save &Special\\
1st  &+0          &+0 &+2  &+0 & Control Monster, Caster Levels, Train Monster, Dread Lore \\
2nd  &+1          &+0 &+3  &+0 & Craft Soul Prison, Heal Monster \\
3rd  &+2          &+1 &+3  &+1 & Subtype Specialization \\
4th  &+3          &+1 &+4  &+1 & Increased Awareness, Double Team \\
5th  &+3          &+1 &+4  &+1 & Speak with Monsters \\
6th  &+4          &+2 &+5  &+2 & Craft Greater Soul Prison \\
7th  &+5          &+2 &+5  &+2 & Type Specialization \\
8th  &+6/+1       &+2 &+6  &+2 & Transfer Control \\
9th  &+6/+1       &+3 &+6  &+3 & Advanced Monster Healing \\
10th &+7/+2       &+3 &+7  &+3 & Craft Leaden Seal \\
11th &+8/+3       &+3 &+7  &+3 & Store Monster, Recall Monster \\
12th &+8/+3       &+4 &+8  &+4 & Second Subtype Specialization \\
13th &+9/+4       &+4 &+8  &+4 & \\
14th &+10/+5      &+4 &+9  &+4 & Second Type Specialization \\
15th &+11/+6/+6   &+5 &+9  &+5 & Craft Master Prison \\
16th &+12/+7/+7   &+5 &+10 &+5 & \\
17th &+12/+7/+7   &+5 &+10 &+5 & Fast Recall Monster \\
18th &+13/+8/+8   &+6 &+11 &+6 & Third Subtype Specialization \\
19th &+14/+9/+9   &+6 &+11 &+6 & Third Type Mastery \\
20th &+15/+10/+10 &+6 &+12 &+6 & Subtype Mastery \\
\end{tabular}
\end{small}
\end{table}

\smallskip\noindent All of the following are class features of the Monster Tamer.

\ability{Weapon and Armor Proficiency:}{Monster Tamers are proficient with all simple weapons, nets, bolas, Orcish Shotputs, Halfling Skiprocks, harpoons, shuriken, and whips. Monster Tamers have proficiency only with light armor. Monster Tamers are considered proficient with using any bludgeoning weapon they are normally proficient with for inflicting subdual damage (thus, they do not duffer a -4 to-hit penalty when attempting to inflict subdual damage with any bludgeoning weapon they are proficient with).}

\ability{Caster Levels:}{Even though Monster Tamers do not gain spells per day or have spell levels - Monster Tamers have many caster level dependant abilities. A Monster Tamer gains a Monster Tamer caster level for every Monster Tamer class level. If a Monster Tamer gains a Prestige Class which adds to Caster levels - she may choose to raise Monster Tamer caster levels instead of other caster levels.}

\ability{"Monster":}{A Monster is any Aberration, Beast, Dragon, Elemental, Magical Beast, Ooze, Outsider, Plant, Shapeshifter, or Vermin which advances by "Hit Dice" rather than "By Character Class." Creatures which can advance by hit dice or character class - like Beholders, are Monsters even if they have character class levels. Deity level creatures, including unique dragon types and unique arch-fiends, are not Monsters regardless of creature type. A Monster Tamer can use the Animal Empathy skill on any Monster as a normal Diplomacy attempt to influence NPC attitudes - regardless of whether or not the Monster Tamer shares a language with the Monster or the intelligence of the Monster.}

\ability{Soul Prisons and Monsters:}{When a Monster is caught with a Soul Prison (see Craft Soul Prison below) it is shrunk down and placed in stasis like in Gloves of Storing (DMG: pages 217-218). While in a Soul Prison, Monsters do not to eat, sleep, breathe, etc. A Monster can be returned to its Soul Prison or removed from its Soul Prison as a standard action by the Monster Tamer which owns it - with a range of 25' + 5' per 2 caster levels. If a Soul Prison with a Monster is traded, given, or sold to another person, ownership of the Monster is also transferred. A Monster heals rapidly while in its Soul Prison. Regular damage is converted to subdual damage at the rate that subdual damage normally heals for the creature. Subdual damage heals at the normal rate while in its Soul Prison.}

\ability{Control Monster (Ex):}{A Monster Tamer can have a number of Monsters in Soul Prisons equal to her Charisma Modifier be "Controlled." A Controlled Monster behaves like a summoned monster when released from its Soul Prison, and is essentially under the control of the Monster Tamer. A Monster Tamer cannot control a Monster whose Challenge Rating is equal to or greater than the Monster Tamer's Caster Level. Remember the rubric for increasing challenge rating based on extra hit dice or class levels to determine if the Monster is controlled. An uncontrolled Monster will act as it sees fit , possibly going on a rampage, running away, or simply sleeping until it is returned to its Soul Prison. Furthermore, Dragon type Monsters are harder to control than other Monsters, and use twice their CR (or their own CR + 4, whichever is less) to determine whether they will obey their Monster Tamer. A Controlled Monster cannot use any Summoning ability to summon uncontrolled Monsters.

\smallskip\noindent More than one controlled Monster can be out of their balls at any one time - but only the first one released behaves like a summoned monster - any subsequent released Monster will act normally, usually standing around and watching events transpire, or sleeping (extreme events can cause them to take direct action at DM's option).

\smallskip\noindent Increases to Charisma only affect the number of Monsters which can be controlled if the increase would affect spells per day. As such, effects like Eagle’s Splendor do not increase the number of controllable Monsters, but a Cloak of Charisma would. Once a Monster Tamer has reached the limit of the number of Monsters which can be controlled, the Monster Tamer cannot control any more until one or more of the controlled Monsters are released from control or killed. Releasing a Monster from control takes about 10 minutes. Control can be reasserted, but only if the Monster Tamer has the ability to control that many Monsters.}

\ability{Losing Monsters:}{A Monster Tamer can, at any time, release their Monsters into the wild. This is a process that takes about 10 minutes during which the Monster Tamer says her goodbyes to the Monster. The Monster is then free to do whatever it wishes, its current intellignce, alignment, and abilities do not inherently change from this release. The Monster's Soul Prison is broken in the process, and is no longer attuned to that Monster. Monsters who were treated especially well or poorly by their Monster Tamer will not forget that treatment and may, at the DM's discretion, act accordingly either immediately or at some time in the future.}

\ability{Death and Monsters:}{Sometimes, Monsters die, this causes a great loss to the Monster Tamer, both emotionally and spiritually. A Monster Tamer whose controlled Monster dies immediately loses 200 XP times the CR of the Monster (zero XP for Monsters below CR 1). A Monster Tamer can make a Will save (DC 15) to halve the XP loss. XP lost in this way are recovered if the Monster is raised from the dead by any means (usually Raise Dead or Resurrection). The XP is recovered if the Monster is Reincarnated, but the new body breaks the Monster Tamer to Monster link and the Monster is no longer controlled, and may no longer be a Monster (depending on its new type).}

\ability{Train Monster (Ex):}{A Monster Tamer can train or evolve their Monster with their Handle Animal skill. As an extraordinary ability, a Monster Tamer need not choose specific animals as trainable and can use Handle Animal on any Monster. Training a Monster takes 8 hours and has a DC of 15 + Monster's (new) CR. The effects available from Training Monster are based on the number of Ranks in Handle Animal the Monster Tamer has:

\listthree
	\item \ability{3 ranks - Learn Trick:}{This is just like teaching to an animal companion (see DMG page 46). Note that some Monsters are intelligent enough so that they are able to perform "tricks" without being specifically taught - and all Monsters are able to learn at least 4 tricks even if their intelligence would not normally be high enough.}
	\item \ability{6 ranks - Grow Monster:}{This causes the Monster to advance 1 Hit Die, if it would not cause the Monster to exceed its advancement limit. This may cause the creature to grow in size category, see the monster description. This may also cause the Monster to become uncontrolled, if this raises its CR to past the maximum CR the Monster Tamer can control. You select what skills, if any, a Monster Tamer gains for its level, and if this would cause a Monster to gain a feat you may select the feat.}
	\item \ability{9 ranks - Evolve Monster:}{This causes the Monster to evolve to a more advanced form. The Monster gains a template of your choice. Note that this may cause the Monster to become uncontrolled, if this raises the CR to past the maximum CR the Monster Tamer can control. The Monster remains a Monster even if its type changes to a type which is not normally a Monster. Monsters who become Dragons in this way are not harder to control than natural dragons are. You select what skills, if any, a Monster gains with its template, and if this would cause a Monster to gain one or more feats you may select the feat(s). At the DM's option, a Monster may be evolved into a similar but more powerful form that is normally represented by a separate entry. For example: a DM might allow a Monster Tamer to evolve her Red Slaad into a Green Slaad, or a Fiendish Horse into a Nightmare. A Monsters of type Beast which is evolved into a different type, gains a permanent one-time "Hard to Control" modifier as if its CR was 1 higher than it actually is.}
	\item \ability{12 ranks - Inspire Monster:}{You may be an especially kind or cruel master to your Monster, giving it a permanent +2 Sacred or Profane bonus to any statistic. You may only give this bonus once to each Monster, and you cannot give different bonuses (Sacred or Profane) to different Monsters.}
\end{list}}

\ability{Dread Lore (Ex):}{A Monster Monster Tamer accumulates significant knowledge about the Monsters that they face. The amount of knowledge a Monster Monster Tamer has on an encountered wild monster is linked to the Monster Tamer's Knowledge Arcana or Survival skill - whichever is higher. The abilities granted depend upon how many ranks the Monster Tamer has in the relevant skill:

\listthree
	\item \ability{3 ranks - Identify Monster:}{A Monster Tamer can automatically identify the name, type, and subtype of any Monster encountered.}
	\item \ability{6 ranks - Full Monster Entry:}{A Monster Tamer's player can open the Monster Manual (or other relevant source material) to the appropriate page and read the Monster's entry. If the Monster Tamer's player chooses, she may read the relatively uninformative descriptive text at the beginning of the entry to other players out loud. In addition, a Monster Tamer may note whether a Monster encounterred in the wild has extra advancement hit dice and/or class levels - though not necessarily what kind or how many.}
	\item \ability{12 ranks - Fully Identify Monster:}{The Monster Tamer is able to instantly identify any Monster's advancement hit dice and class level (if any).}
\end{list}}

\ability{Craft Soul Prison (Sp):}{A 2nd level Monster Tamer can craft Soul Prisons. A Soul Prison costs 100 GP and 8 XP to make. Alternately, it costs 200 GP to buy one if it is available. A Soul Prison acts as a thrown weapon, which is used as a ranged touch attack with a range increment of 15'. Using a Soul Prison is considered to be using a spell like ability. If a Soul Prison thrown by a Monster Tamer hits a Monster it inflicts 1 point of subdual damage per caster level - if the Monster is unconscious after being hit by the Soul Prison it is sucked into the Soul Prison and now belongs to the Monster Tamer who threw the Soul Prison - the Soul Prison is now sitting in a square formerly occupied by the captured Monster. If a Soul Prison hits a Monster it is attuned to that Monster and cannot be used on any other Monster - ever.}

\ability{Heal Monster (Sp):}{A Monster Tamer may attempt to accelerate the healing of a Monster in its Soul Prison. By spending a fullround action, a Monster Tamer can attempt a Heal Check (DC 15) to either convert all regular damage suffered by the Monster into subdual damage, or to confer the benefits of 1 day of rest to the Monster (2 Hit Points per hit die, 1 day worth of repaired Ability damage, the recovery of any limited uses/day abilities, and the healing of all subdual damage). This ability may be used on each Monster 3 plus the Monster Tamer's Wisdom bonus (if positive) times per day.}

\ability{Subtype Specialization (Ex):}{A Monster Tamer can choose a subtype which is her specialty. A Monster Tamer gains a +1 bonus on all Bluff, Animal Empathy, Handle Animal, Knowledge, Listen, Sense Motive, Spot, and Survival, checks when using these skills on or about such creatures for every 3 caster levels she has. A Monster Tamer can choose a second Subtype to be equally proficient with at 12th level, and a third at 18th. A Monster Tamer can Control one extra Monster which must be of a subtype that she specializes in. Subtypes include: Air, Aquatic, Chaotic, Cold, Earth, Electricity, Evil, Fire, Good, Lawful, Reptilian, and Water.}

\ability{Increased Awareness (Ex):}{At 4th level and above, a Monster Tamer's Monster become more intelligent and aware. After the Monster Tamer has owned her Monster for at least 1 week, its intelligence changes to the Monster Tamers ranks in Handle Animal if that is more than its normal intelligence.

\smallskip\noindent In addition, a Monster Tamer can make her Monster gradually see things her way - a Monster's alignment shifts one degree towards the Monster Tamer's each week if she can succeed in an Animal Empathy check at a DC of (10 + the Monster's CR). The DM decides whether it moves Law/Chaos or Good/Evil first depending upon circumstances. So if a Lawful Good Monster Tamer captured an Imp (lawful evil Monster), the Imp could become Lawful Neutral after one week, and could be Lawful Good after 2 weeks. Monster subtypes are unaffected - so an Evil Monster such as an Efreet would stay subtype [Evil] even if it subsequently became of Good alignment.}

\ability{Double Team:}{Upon reaching 4th level, the Monster Tamer is able to control two Monsters out of their balls simultaneously, even in battle. This ability only functions so long as both Monsters are more than 2 CR less than the Monster Tamer's caster level. For example, a 5th level Monster Tamer could command a single CR 4 Monster in battle or two CR 2 Monsters, but could not command a CR 1 Monster and a CR 3 Monster simultaneously.}

\ability{Speak with Monsters (Ex):}{At fifth level a Monster Tamer has Tongues - always on, which only effects Monsters. Even though a Gorgon's speech still sounds like "Groarrough" it is perfectly intelligible to the Monster Tamer. Further, the Monster Tamer's speech is understandable by Monster even if they do not normally have a language - even Oozes and other Monsters not normally capable of communicating at all.}

\ability{Craft Greater Soul Prison (Sp):}{A Monster Tamer can craft a Greater Soul Prison, which is a more powerful form of Soul Prison. It behaves just like a Soul Prison except that it costs 1000 Gold and 80 XP to craft - and inflicts d4 subdual damage per caster level.}

\ability{Type Specialization:}{At 7th level, you can choose a single creature type to gain the same skill bonuses as your subtype specialization with a creature type instead. You are not limited to normal Monster types. You may choose a second type to Specialize in at 14th level, and a third at 19th. You may have an additional controlled Monster, which must be of a type you are specialized in. Type and Subtype Specialization bonuses are cumulative.}

\ability{Transfer Control:}{At 8th level a Monster Tamer can choose to change which Monster she controls, up to her regular limit of controlled Monsters. All newly controlled Monsters must be in Soul Prisons possessed and owned by the Monster Tamer. Transfer Control is a full-round action. Normally transferring control takes 10 minutes per Monster so transferred.}

\ability{Advanced Monster Healing (Sp):}{A Monster Tamer can, at 9th level, use Heal as a Spell like ability a number of times a day equal to her wisdom modifier, with a minimum of once a day. A Monster Tamer can only Heal Monsters she controls, but can heal them whether they are in their Soul Prisons or not.}

\ability{Craft Leaden Seal (Sp):}{A Monster Tamer can craft a Leaden Seal. A Leaden Seal is a much more powerful form of Soul Prison. It costs 5000 GP and 400 XP to make. When used, it inflicts d8 points of subdual damage per caster level.}

\ability{Store Monster (Sp):}{Starting at 11th level, as a move equivalent action, a Monster Tamer can send a Soul Prison with a Monster in it to a completely safe extra dimensional space. A Soul Prison must be within Close range (25 feet + 5 feet per 2 caster levels) to be stored. Store Monster cannot be combined with a normal move. Store Monster is a spell-like ability.}

\ability{Recall Monster (Sp):}{Starting at 11th level, as a fullround action, a Monster Tamer can transport a Stored Soul Prison from her extra dimensional space to her hand.}

\ability{Craft Master Prison (Sp):}{A Master Prison is the ultimate expression of the Monster Hunter - it costs a hefty 10000 GP and 800 XP to manufacture, and subdues the first Monster it hits, if that Monster does not have more than 2 hit dice for every caster level of the Monster Tamer who threw it. If a Monster is too strong to be captured automatically it may yet succumb as it still suffers d12 subdual damage per caster level.}

\ability{Fast Recall Monster (Sp):}{As Recall Monster, but Recalling Monster is a free action.}

\ability{Subtype Mastery:}{The Monster Tamer chooses one subtype that she is already specialized in to Master. All her Leaden Seals function like Master Prisons against Monsters of that subtype, there is no limit to the CR of Monsters of that subtype that she can control - and she can control one extra Monster of that subtype, in addition to her bonus controlled Monsters from type and subtype specialization.}

\ability{Monster Tamers and Multiclassing:}{Monster Tamers rarely multiclass, however if they multiclass into another spellcasting class and have access to domains, the Spellcaster levels stack for purposes of controlling Monster of a type or subtype sharing of those domains. So a Monster Tamer 6/ Cleric 5 with the domains of Evil and Fire would control Monsters as Caster level 6, but would control Evil or Fire Monsters as a Caster Level 11 Monster Tamer.}