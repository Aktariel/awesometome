\subsection{Bard}

It is said that music has a special magic, and the bard proves that saying true. Wandering across the land, gathering lore, telling stories, working magic with his music, and living on the gratitude of his audience: such is the life of a bard. When chance or opportunity draws them into a conflict, bards serve as diplomats, negotiators, scouts, and spies.

A bard's magic comes from the heart. If his heart is good, a bard bring brings hope and courage to the downtrodden and uses his tricks, music, and magic to thwart the schemes of evildoers. If the nobles of the land are corrupt, the good bard is an enemy of the state, cunningly evading capture and raising the spirits of the oppressed. But music can spring from an evil heart as well. Evil bards eschew blatant violence in favor of manipulation, holding sway over the hearts and minds of others and taking what enraptured audiences ''willingly'' provide.

\ability{Adventurers:}{Bards see adventures as opportunities to learn. They practice their many skills and abilities, and they especially relish the opportunity to enter a long forgotten tomb, to discover ancient works of magic, to decipher old tomes, to travel to strange places, to encounter exotic creatures, and most importantly � to learn new songs and stories. Bards love to accompany heroes (and villains), joining their entourage to witness their deeds firsthand � a bard who can tell a marvelous story from personal experience earns renown among his fellows. Indeed, after telling so many stories about heroes doing mighty deeds, many bards take these themes to heart and assume the role of hero themselves.}

\ability{Characteristics:}{A bard brings forth magic from his soul, not from a book. He can cast only a small number of spells, but he can cast them without selecting or preparing them in advance. Even more than the wizard, bards are adept at learning new magic. A bard's magic emphasizes charms and illusions over the more dramatic evocations that wizards and sorcerers often use.

In addition to spells, a bard works magic with his music and poetry, which can be seamlessly blended with their spellcasting to produce a number of effects.

Bards have some skills that rogues have, though they are not as focused on skill mastery as rogues are. Bards listen to stories as well as tell them, so they have a vast knowledge of local events and noteworthy items.}

\ability{Alignment:}{Bards are driven first by music, which can be a powerful spontaneous force as easily as it can be a regimented and exacting science. Bards may be of any alignment.}

\ability{Religion:}{Many bards revere Fharlanghn, god of roads. They sometimes camp near his wayside shrines, hoping to earn some coin from the travelers who stop to leave offerings for the god. Some bards, even those without a drop of elvish blood, worship Corellon Larethian, highest god of elves and patron of poetry and music. Good bards are also partial to Pelor, the sun god, believing that he watches over them in their travels. Bards given to chaos and larceny favor Oliadammara, god of thieves. Those who have turned to evil ways are known to worship Erythnul, the god of slaughter, though few will admit to it.}

\ability{Background:}{An apprentice bard learns his skills and songs from a single experienced bard, whom he follows and serves until he is ready to strike out on his own. Many bards were once young runaways or orphans, befriended by wandering bards who became their mentors. Since bards occasionally congregate in formal or informal ''colleges,'' the apprentice bard may meet many of the more prominent bards in the area. Still, a bard rarely has any special allegiance to bards as a whole, and many actively avoid contact with other bards for fear of them learning his songs. Some bards take this to an extreme, and become highly competitive with other bards, jealous of their reputation and defensive of their territory.}

\ability{Races:}{Bards are commonly human, gnomes, elven, or half-elven. Humans take well to the wandering life and adapt well to new lands and customs. Elves are talented in music and magic, so the career of the bard comes naturally to them. Gnomes are inherently gifted with magic of the bardic sort, and the learning of the songs is a natural step for the young gnome. A bard's wandering ways suit many half-elves, who often feel like strangers even when at home. Half-orcs, even those raised among humans, find themselves ill-suited to the demands of a bardic career. There are no bardic traditions among the dwarves or Halflings, though occasional individuals of those races find teachers to train them in the ways of the bard � especially if they grow up near human settlements.

Bards are exceedingly rare among the savage humanoids except among centaurs and kobolds. Centaur bards sometimes train the children of humans or other humanoids. Kobold bardic traditions are ancient and possessed of a bitter rivalry with those of the gnomes.}

\ability{Other Classes:}{A bard works well with the companionship of other classes. He often serves as the spokesman of the party, using his social skills for the party's benefit. In a party without a wizard or a sorcerer, the bard relies on his magic. In a party without a rogue, he uses his skills. A bard is curious about the ways of more focused or dedicated adventurers, often trying to pick up pointers from fighters, sorcerers, and rogues.}

\subsubsection{Game Rule Information}

\ability{Alignment:}{Any.}

\ability{Starting Gold:}{4d4x10 gp (100 Gold)}

\ability{Starting Age:}{As Bard}

\ability{Hit Die:}{d6}

\ability{Class Skills:}{The Bard�s class skills (and the key ability for each skill) are Appraise (Int), Balance (Dex), Bluff (Cha), Climb (Str), Concentration (Con), Craft (Int), Decipher Script (Int), Diplomacy (Cha), Disguise (Cha), Escape Artist (Dex), Gather Information (Cha), Hide (Dex), Jump (Str), Knowledge (all skills, taken individually) (Int), Listen (Wis), Move Silently (Dex), Perform (Cha), Profession (Wis), Sense Motive (Wis), Sleight of Hand (Dex), Speak Language (n/a), Spellcraft (Int), Swim (Str), Tumble (Dex), and Use Magic Device (Cha).}

\ability{Skills/Level:}{(6 + Intelligence Bonus)}

\begin{table}[htb]
\begin{small}
\begin{tabular}{lp{3cm}p{0.7cm}p{0.7cm}p{0.7cm}l}
Level  &Base Attack Bonus &Fort Save &Ref Save &Will Save &Special\\
1st &+0 &+0 &+2 &+0 &Bardic Knowledge, Performance Trick, Blended Casting\\
2nd &+1 &+0 &+3 &+0 &Bonus Feat\\
3rd &+2 &+1 &+3 &+1 &Performance Trick\\
4th &+3 &+1 &+4 &+1 &Bonus Feat, Still Spells\\
5th &+3 &+1 &+4 &+1 &Performance Trick\\
6th &+4 &+2 &+5 &+2 &Bonus Feat, Silent Spells\\
7th &+5 &+2 &+5 &+2 &Performance Trick\\
8th &+6/+1 &+2 &+6 &+2 &Bonus Feat, Spell Focus: Enchantment\\
9th &+6/+1 &+3 &+6 &+3 &Performance Trick\\
10th &+7/+2 &+3 &+7 &+3 &Bonus Feat, Focused Skill Mastery\\
11th &+8/+3 &+3 &+7 &+3 &Performance Trick\\
12th &+9/+4 &+4 &+8 &+4 &Bonus Feat, Spell Focus: Illusion\\
13th &+9/+4 &+4 &+8 &+4 &Performance Trick\\
14th &+10/+5/+9 &+4 &+9 &+4 &Bonus Feat, Spell Penetration\\
15th &+11/+6/+1 &+5 &+9 &+5 &Performance Trick\\
16th &+12/+7/+2 &+5 &+10 &+5 &Bonus Feat, Special Ability\\
17th &+12/+7/+2 &+5 &+10 &+5 &Performance Trick\\
18th &+13/+8/+3 &+6 &+11 &+6 &Bonus Feat\\
19th &+14/+9/+4 &+6 &+11 &+6 &Performance Trick\\
20th &+15/+10/+5 &+6 &+12 &+6 &Bonus Feat, Special Ability\\
\end{tabular}
\end{small}
\end{table}

\begin{floatingfigure}{5.0in}
\begin{small}
\noindent\begin{tabular}{lllllllllll}
 & \multicolumn{10}{c}{Bard Spells Per Day}\\
 &0 &1 &2 &3 &4 &5 &6 &7 &8 &9\\
1 &1 &- &- &- &- &- &- &- &- &-\\
2 &2 &- &- &- &- &- &- &- &- &-\\
3 &2 &0 &- &- &- &- &- &- &- &-\\
4 &2 &1 &- &- &- &- &- &- &- &-\\
5 &2 &1 &0 &- &- &- &- &- &- &-\\
6 &3 &1 &1 &- &- &- &- &- &- &-\\
7 &3 &2 &1 &0 &- &- &- &- &- &-\\
8 &3 &2 &1 &1 &- &- &- &- &- &-\\
9 &3 &2 &2 &1 &0 &- &- &- &- &-\\
10 &3 &2 &2 &1 &1 &- &- &- &- &-\\
11 &4 &3 &2 &2 &1 &0 &- &- &- &-\\
12 &4 &3 &2 &2 &1 &1 &- &- &- &-\\
13 &4 &3 &3 &2 &2 &1 &0 &- &- &-\\
14 &4 &3 &3 &2 &2 &1 &1 &- &- &-\\
15 &4 &3 &3 &3 &2 &2 &1 &0 &- &-\\
16 &4 &4 &3 &3 &2 &2 &1 &1 &- &-\\
17 &5 &4 &3 &3 &3 &2 &2 &1 &0 &-\\
18 &5 &4 &4 &3 &3 &2 &2 &1 &1 &-\\
19 &5 &4 &4 &3 &3 &3 &2 &2 &1 &0\\
20 &5 &4 &4 &4 &3 &3 &2 &2 &1 &1\\
\end{tabular}
\end{small}
\end{floatingfigure}

\smallskip\noindent All of the following are Class Features of the Bard class.

\ability{Weapon and Armor Proficiencies:}{A bard is proficient with all simple weapons, plus any five martial weapons of their choice, and the bladed fan, bolas, elven lightblade, and whip. Bards are proficient with light armor and shields (except tower shields). A bard can cast bard spells while wearing any armor he is proficient with without incurring the normal arcane spell failure chance. However, like any other arcane spellcaster, a bard wearing armor he is not proficient with shield incurs a chance of arcane spell failure if the spell in question has a somatic component (most do). A multiclass bard still incurs the normal arcane spell failure chance for arcane spells received from other classes.}

\ability{Spells:}{A bard casts arcane spells, which are drawn from the bard spell list. He can cast any spell he knows without preparing it ahead of time. Every bard spell has a verbal component (singing, reciting, or music), or a somatic component (dancing, puppetry, or mime), and must make a Perform check of an appropriate type with a DC of 5 + Double the Spell Level or the spell fails. To learn or cast a spell, a bard must have a Charisma score equal to at least 10 + the spell. The Difficulty Class for a saving throw against a bard�s spell is 10 + the spell level + the bard�s Charisma modifier. Unlike a sorcerer, a bard may know any number of spells, but while he will probably know many spells, he can only cast a small number of them each day.}

\ability{Learning Songs:}{A bard begins play knowing 12 0-level bard spells. For each point of Charisma bonus the bard has, he begins playing knowing one additional 0-level spell of your choice. At each new bard level, he gains two new spells of any spell level or levels that he can cast (based on his new bard level). At any time, a bard can also learn spells found in written form or which he perceives being cast by other bards.

If a bard sees another bard casting a bard spell with somatic components, he can make a spot check with a DC of the casting bard's Perform check to immediately learn the spell. If a bard hears another bard casting a bard spell with verbal components, he can make a Listen check with a DC equal to the casting bard's Perform check result to immediately learn the spell. If a bard deciphers arcane magical writing containing a bard spell in written form, he may learn that spell.}

\ability{Bardic Knowledge:}{A Bard gains Skill Focus as a bonus feat for any knowledge skill he has at least one rank in. A bard with Knowledge (local) is considered to have local knowledge for any area he has been in for any length of time. When he arrives in a new land, his ranks in Knowledge (local) ''catch up'' at the rate of one per day as long as he is able to spend at least one hour per day telling stories and listening to gossip.}

\ability{Performance Trick:}{At every odd numbered level, a bard gains a specific magical ability that they can use with a type of performance. All of a bard's Performance Tricks can be used at will and are a Supernatural Ability. Activating or maintaining a Performance Trick requires only a Swift Action each round, but the performance always lasts at least an entire round. The Save DC, if any, of a trick is 10 + � of the bard's character level + the bard's charisma modifier, or the bard's performance skill check result, whichever is \emph{less}. Each trick may only be used with one category of performance from the following list (the types of components used are listed in parenthesis): Acting (S, V), Comedy (S, V), Dance (S), Keyboard Instruments (S, F), Oratory (V), Percussion Instruments (S, F), String Instruments (S, F), Wind Instruments (S, V, F), Singing (V). The list here is not intended to be exclusive, and DMs are encouraged to introduce more unique performance powers into his campaign.}

\ability{Blended Casting:}{A bard is able to cast and maintain concentration on spells simultaneously with using his bardic performance tricks, provided that the spells in question do not use the same components (verbal or somatic) as the performance type being used requires.}

\ability{Bonus Feats:}{At every even numbered level, a bard gains Skill Focus as a bonus feat that may be applied to any skill he has at least one rank in.}

\ability{Still Spells (Ex):}{A bard of 4th level or higher may cast his spells without somatic components at will as if using the Still Spell metamagic, though without using up a higher level spell slot. All of a bard's spells must still have Somatic \emph{or} Verbal components though, so a spell that already has no Verbal components cannot be made Still in this way.}

\ability{Silent Spells (Ex):}{A bard of 6th level or higher may cast his spells without verbal components at will as if using the Silent Spell metamagic, though without using up a higher level spell slot. All of a bard's spells must still have Somatic \emph{or} Verbal components though, so a spell that already has no Somatic components cannot be made Silent in this way.}

\ability{Spell Focus:}{A bard of 8th level gains Spell Focus: Enchantment as a bonus feat. If he already has Spell Focus: Enchantment, he gains Greater Spell Focus: Enchantment instead. A bard of 12th level gains Spell Focus: Illusion as a bonus feat. If he already has Spell Focus: Illusion, he gains Greater Spell Focus: Illusion instead. A bard of 14th level gains Spell Penetration as a bonus feat. If he already has Spell Penetration, he gains Greater Spell Penetration instead.}

\ability{Focused Skill Mastery:}{At 10th level, a bard is able to perform any tasks he really sets his mind towards. He can take 10 on any skill check for which he has the Skill Focus Feat.}

\ability{Special Ability:}{At 16th and 20th level, a bard may select a Rogue Special Ability.}

\spelllist{Bard Spell List:}

\small\ability{0th level:}{\emph{Addiction, Bless, Create Water, Cure Light Wounds, Dancing Lights, Darkness, Daze, Death Grimace, Detect Magic, Detect Poison, Drug Resistance, Flare, Ghost Harp, Ghost Sound, Hypnotism, Insightful Feint, Light, Mage Hand, Mending, Open/Close, Prestidigitation, Purify Food and Drink, Read Magic, Resistance, Restful Slumber, Silent Image, Silent Portal, Songbird, Stick, Ventriloquism}}

      \ability{1st level:}{\emph{Accelerated Movement, Alarm, Amplify, Appraising Touch, Blur, Charm Person or Animal, Cheat, Critical Strike, Cure Moderate Wounds, Detect Secret Doors, Disguise Self, Expeditious Retreat, Feather Fall, Grease, Greater Dispelling, Hypnotic Pattern, Healthful Rest, Identify, Improvisation, Insidious Rhythm, Joyful Noise, Mage Armor, Magic Weapon, Minor Image, Mirror Image, Phantom Threat, Scare, Serene Visage, Silence, Sticky Fingers, Summon Monster II, Tasha's Hideous Laughter, Unseen Servant}}

      \ability{2nd level:}{\emph{Animal Trance, Battle Hymn, Bear's Endurance, Blindness/Deafness, Bull's Strength, Cat's Grace, Circle Dance, Cure Serious Wounds, Daylight, Deeper Darkness, Delay Poison, Detect Thoughts, Eagle's Splendor, Entice Gift, Fox's Cunning, Grace, Greater Alarm, Harmonic Chorus, Haunting Tune, Invisibility, Levitate, Locate Object, Major Image, Mindless Rage, Misdirection, Owl's Wisdom, Prayer, Pyrotechnics, See Invisibility, Shatter, Sound Burst, Suggestion, Summon Monster III, Summon Swarm, Suspended Silence, Tactical Precision, Tongues, Undetectable Alignment, War Cry, Wave of Grief, Whispering Wind, Wraithstrike}}

      \ability{3rd level:}{\emph{Bestow Curse, Blink, Charm Monster, Confusion, Cure Critical Wounds, Dirge of Discord, Dream Walk, Fear, Greater Magic Weapon, Haste, Hymn of Praise, Infernal Threnody, Love's Lament, Magic Circle, Phantom Steed, Puppeteer, Rainbow Pattern, Recitation, Remove Curse, Stunning Screech, Summon Monster IV, Tiny Hut, Voice of the Dragon}}

      \ability{4th level:}{\emph{Break Enchantment, Cacophonic shield, Celebration, Detect Scrying, Dimension Door, Dismissal, Dominate Person, Dream, Greater Resistance, Hallucinatory Terrain, Hold Monster, Improved Invisibility, Lay of the Land, Legend Lore, Scry, Secure Shelter, Locate Creature, Manifest Desire, Manifest Nightmare, Mirror Sending, Modify Memory, Neutralize Poison, Persistent Image, Pronouncement of Fate, Ruin Delver's Fortune, Sensory Deprivation, Shout, Stop Heart, Sirine's Grace, Summon Monster V}}

      \ability{5th level:}{\emph{Atonement, Body Harmonic, Cacophonic Burst, Contact Other Planes, Control Water, Dreaming Puppet, False Vision, Hide from Dragons, Illusory Feast, Mind Fog, Mirage Arcana, Mass Reflective Disguise, Mass Suggestion, Mislead, Morality Undone, Nightmare, Permanent Image, Programmed Image, Shadow Form, Summon Monster VI, Wail of Doom}}

      \ability{6th level:}{\emph{Control Weather, Eyebite, Dirge, Dream Casting, Geas, Greater Scrying, Hindsight, Hiss of Sleep, Illusory Pit, Insanity, Nixie's Grace, Oath of Blood, Plane shift, Project Image, Repulsion, Summon Monster VII, Superior Resistance, Symphonic Nightmare, Veil}}

      \ability{7th level:}{\emph{Antipathy, Dream Sight, Familial Geas, Greater Planeshift, Greater Teleport, Irresistible Dance, Mass Charm, Mass Modify Memory, Powerword: Stun, Shadow walk, Solipsism, Summon Monster VIII, Sympathy, Transfix}}

      \ability{8th level:}{\emph{Demand, Dominate Monster, Maddening Whispers, Plague of Nightmares, Powerword: Blind, Shifting Paths, Summon Monster IX, Superior Invisibility, Wrathful Castigation}}

      \ability{9th level:}{\emph{Gate, Powerword: Kill, Programmed Amnesia, shades, Teleportation, Temporal Stasis, Weird}}
\normalsize

\newcommand{\performtrick}[3]{\bolded{#1:}

#2

\ability{Allowed Performance Styles:}{#3}}

\subsubsection{Performance Tricks:}

\performtrick{Countersong}{While the performance is ongoing, the bard can attempt to suppress magical effects with the [Sonic] descriptor and spells with a verbal component. Anyone within medium range of the bard attempting to use or maintain such a spell or ability must make a Willpower save or their spell or ability fizzles.}{Acting (S, V), Comedy (S, V), Keyboard Instruments (S, F), Oratory (V), Percussion Instruments (S, F), String Instruments (S, F), Wind Instruments (S, V, F), Singing (V)}

\performtrick{Inspire Courage}{All allies within close range of the bard who see or hear the performance gain a morale bonus to attack and damage rolls equal to 1/4th of the bard's class level (round up) for as long as the performance continues and they remain in range, and for 1 round per level after they are no longer able to perceive the performance or are no longer in range.}{Acting (S, V), Comedy (S, V), Dance (S), Keyboard Instruments (S, F), Oratory (V), Percussion Instruments (S, F), String Instruments (S, F), Wind Instruments (S, V, F), Singing (V)}

\performtrick{Destructive Cacophony}{Each round that the cacophony is maintained, one additional target within medium range suffers 1d6 of Sonic damage per turn for as long as it remains within range. The bard must select an additional target each round or end the cacophony. The bard may only maintain this effect for a number of subsequent rounds equal to his bard level.}{ Keyboard Instruments (S, F), Percussion Instruments (S, F), String Instruments (S, F), Wind Instruments (S, V, F), Singing (V)}

\performtrick{Requiem}{The bard can produce a haunting melody that renders undead very docile, possibly even friendly. For the first 10 rounds an undead creature hears the song (and is within close range), the song acts as a \emph{halt undead} effect, undead which fail their Will saves are stuck fast for 10 rounds (or until the performance stops or the undead creatures are no longer in range or take damage). If the effect is still going at the end of 10 rounds, the bard may make a Perform check as if it were a diplomacy check to improve the undead creature's disposition. Regardless of the outcome of that check, the undead creatures can again move normally.}{ Keyboard Instruments (S, F), Percussion Instruments (S, F), String Instruments (S, F), Wind Instruments (S, V, F)}

\performtrick{Bewilder}{The bard makes a performance so avante guarde, so surprising, that onlookers are left unsure of what to do or say. Audiences within medium range must make a Will save each round or become \emph{dazed} for one round.}{Acting (S, V), Comedy (S, V), Dance (S), Keyboard Instruments (S, F), Oratory (V), Percussion Instruments (S, F), String Instruments (S, F), Wind Instruments (S, V, F), Singing (V)}

\performtrick{Song of the Weak Mind}{While the music plays, the audience suffers a -4 penalty on Will saves vs. spells of the Illusion and Enchantment schools for as long as they remain within medium range. The duration of any stunned or dazed conditions are extended y 1 round if they are applied within medium range of the source of the song.}{Keyboard Instruments, String Instruments, Wind Instruments, Singing}

\performtrick{Dance of the Seven Swords}{As long as the bard continues to dance, he is considered to have a BAB equal to his character level. The bard may make an extra attack of opportunity each round (as per the Combat Reflexes feat) for every 10 full points his Perform check exceeds 10. The bard gains a +1 bonus to attack and damage rolls for every full 20 points his perform check exceeds 10.}{Dance}

\performtrick{Torrent of Anger}{The bard attempts to inflame passions against a target. All audience members find the performance fascinating and are compelled to watch for its duration (or until they are snapped out of it by violent action) if they are within medium range and fail a Willpower save. Creatures who watch for at least 10 minutes become unfriendly to a target of the bard's choosing (unless they were alread hostile), and one step more friendly towards any person or group known to be enemies of the new source of antipathy.}{Acting, Comedy, Oratory}