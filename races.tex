
\section{Races}

\subsection{Aasimar}
\vspace*{-8pt}
\quot{``My ancestors were more beautiful than you can imagine."}

\listone
    \item Medium Size
    \item 30' movement.
    \item Outsider Type (Native and Human subtype)
    \item Darkvision: Aasimar can see in the dark up to 60 feet.
    \item +2 Charisma, +2 Wisdom
    \item Aasimar with a Charisma of at least 10 may cast light as a spell-like ability with a caster level equal to their character level once per day.
    \item +2 bonus to Spot, and Listen checks.
    \item Favored Classes: Paladin and Sorcerer
    \item Automatic Languages: Common
    \item Bonus Languages: Abyssal, Aquan, Auran, Celestial, Formian, Ignan, Slaad, Sylvan, Terran.
\end{list}



\subsection{Drow}
\vspace*{-8pt}
\quot{``Time to die for the Spider Queen."}

\listone
    \item Medium Size
    \item 30' movement.
    \item Humanoid Type (Elf subtype)
    \item Darkvision: Drow can see in the dark up to 120 feet.
    \item +2 Dexterity, -2 Constitution
    \item Daylight Sensitivity: While in brightly lit surroundings (such as a daylight spell), a Drow suffers a -2 penalty to attack rolls and precision-based skill checks.
    \item Drow with a Charisma of at least 10 may cast deeper darkness (duration 4 hours), and fairie fire as spell-like abilities with a caster level equal to their character level once per day each.
    \item +2 bonus to saving throws against spells and spell-like abilities.
    \item +2 bonus to Spot, and Listen checks.
    \item Drow never sleep and are immune to sleep effects. Drow must still perform their 4 hour daily trance to stay coherent and rested.
    \item Drow live an exceedingly interesting life and every Drow has proficiency with the rapier and an exotic ranged weapon of their choice.
    \item Favored Classes: Cleric and Wizard
    \item Automatic Languages: Elvish
    \item Bonus Languages: Abyssal, Beholder, Common, Draconic, Drow Sign Language, Dwarvish, Gnome, Kuo-Toa, Terran, Undercommon.
\end{list}

\subsection{Dwarf}
\vspace*{-8pt}
\quot{``I remember that...''}

\listone
		\item Medium Size
		\item 20' movement
		\item Humanoid Type (Dwarf Subtype)
		\item +2 Constitution, -2 Charisma
		\item Dwarves can move up to their full speed even when wearing medium or heavy armor or when carrying a medium or heavy load
		\item Darkvision: Dwarves can see up to 60 feet in the dark.
		\item Stonecunning: This ability grants a dwarf a +2 racial bonus on Search checks to notice unusual stonework, such as sliding walls, stonework traps, new construction (even when built to match the old), unsafe stone surfaces, shaky stone ceilings, and the like. Something that isn�t stone but that is disguised as stone also counts as unusual stonework. A dwarf who merely comes within 10 feet of unusual stonework can make a Search check as if he were actively searching, and a dwarf can use the Search skill to find stonework traps as a rogue can. A dwarf can also intuit depth, sensing his approximate depth underground as naturally as a human can sense which way is up.
		\item Weapon Familiarity: Dwarves may treat dwarven waraxes and dwarven urgroshes as martial weapons, rather than exotic weapons.
		\item Stability: A dwarf gains a +4 bonus on ability checks made to resist being bull rushed or tripped when standing on the ground (but not when climbing, flying, riding, or otherwise not standing firmly on the ground).
		\item +2 racial bonus on saving throws against poison.
		\item +2 racial bonus on saving throws against spells and spell-like effects.
		\item +1 racial bonus on attack rolls against orcs and goblinoids.
		\item +4 dodge bonus to Armor Class against monsters of the giant type. Any time a creature loses its Dexterity bonus (if any) to Armor Class, such as when it�s caught flat-footed, it loses its dodge bonus, too.
		\item +2 racial bonus on Appraise checks that are related to stone or metal items.
		\item +2 racial bonus on Craft checks that are related to stone or metal.
		\item Favored class: Fighter
		\item Automatic Languages: Common and Dwarven.
		\item Bonus Languages: Giant, Gnome, Goblin, Orc, Terran, and Undercommon.
\end{list}

\subsection{Elf}
\vspace*{-8pt}
\quot{``You shall never harm my beautiful trees!''}

\listone
		\item Medium Size
		\item Humanoid Type (Elf Subtype)
		\item 30' movement
		\item +2 Dexterity, -2 Constitution
		\item Low Light Vision: Elves can see twice as far as a human in poor lighting.
		\item Weapon Proficiency: Elves are proficient with the longsword, rapier, longbow (including composite longbow), and shortbow (including composite shortbow).
		\item +2 racial bonus on Listen, Search, and Spot checks. An elf who merely passes within 5 feet of a secret or concealed door is entitled to a Search check to notice it as if she were actively looking for it.
		\item Favored Class: Wizard.
		\item Automatic Languages: Common and Elven. 
		\item Bonus Languages: Draconic, Gnoll, Gnome, Goblin, Orc, and Sylvan.
\end{list}

\subsection{Feytouched}
\vspace*{-8pt}
\quot{``All my life, I have never fit in. Not in town, not in the forest. In some integral fashion I am unlike those around me, and I believe it is my fate to live and die alone."}

\listone
		\item Medium Size
    \item Fey Type
    \item 30' movement
    \item Low-Light Vision: Feytouched can twice as far as a human in poor lighting.
    \item +2 Dexterity, +2 Charisma, -2 Constitution. Feytouched are graceful and those which are not beautiful are terrifying, but they are fragile like flowers.
    \item Immunity to [Compulsion] Effects
    \item Magic Affinity: Every Feytouched is different, and marked by the signature magics of the fey in a different manner. Every Feytouched has one spell that can be used once per day as a spell-like ability. This spell is chosen at 1st level and cannot be changed. Any 1st level Illusion or Enchantment spell from the Sorcerer/Wizard list is fair game, and the save DC is Charisma-based.
    \item Favored Class: Bard
    \item Feytouched speak Common and Sylvan. Bonus Languages may be selected from the following list:
      Aquan, Auran, Elvish, Draconic, Dwarvish, Druidic, Goblin, Gnoll, Gnome, Halfling.
\end{list}

\subsection{Gnome}
\vspace*{-8pt}
\quot{``What's that you say little mole? Kobolds in the well!?''}

\listone
		\item Small Size
		\item 20' movement
		\item Humanoid Type {Gnome subtype}
		\item +2 Constitution, -2 Strength
		\item Low-Light Vision: Gnomes can see twice as far as a human in poor lighting.
		\item Weapon Familiarity: Gnomes may treat gnome hooked hammers as martial weapons rather than exotic weapons.
		\item +2 racial bonus on saving throws against illusions.
		\item Add +1 to the Difficulty Class for all saving throws against illusion spells cast by gnomes. This adjustment stacks with those from similar effects.
		\item +1 racial bonus on attack rolls against kobolds and goblinoids.
		\item +4 dodge bonus to Armor Class against monsters of the giant type. Any time a creature loses its Dexterity bonus (if any) to Armor Class, such as when it�s caught flat-footed, it loses its dodge bonus, too.
		\item +2 racial bonus on Listen checks.
		\item +2 racial bonus on Craft (alchemy) checks.
		\item Spell-Like Abilities: 1/day�speak with animals (burrowing mammal only, duration 1 minute). A gnome with a Charisma score of at least 10 also has the following spell-like abilities: 1/day�dancing lights, ghost sound, prestidigitation. Caster level 1st; save DC 10 + gnome�s Cha modifier + spell level.
		\item Favored Class: Bard
		\item Automatic Languages: Common and Gnome. 
		\item Bonus Languages: Draconic, Dwarven, Elven, Giant, Goblin, and Orc. In addition, a gnome can speak with a burrowing mammal (a badger, fox, rabbit, or the like). This ability is innate to gnomes. See the speak with animals spell description.
\end{list}
	

\subsection{Goblin}
\vspace*{-8pt}
\quot{``You weren't hired to think. You were hired because you have opposable thumbs."}

\listone
    \item Small Size
    \item 30' movement (despite small size).
    \item Humanoid Type (Goblinoid subtype)
    \item Darkvision: Goblins can see up to 60 feet in the dark.
    \item +2 Dexterity, -2 Strength, -2 Charisma
    \item +4 bonus to Move Silently and Ride checks.
    \item Bonus Feat: Mounted Combat
    \item Goblins benefit from an ancient pact with the Worgs, and every Goblin receives a +2 bonus to any Bluff, Diplomacy, Handle Animal, Sense Motive, or Survival check made with respect to a Worg.
    \item Favored Classes: Rogue and Wizard
    \item Automatic Languages: Common, Goblin
    \item Bonus Languages: Draconic, Elvish, Dwarvish, Giant, Gnoll, Infernal, Orcish, Undercommon, and Worg.
\end{list}

\subsection{Half-Elf}
\vspace*{-8pt}
\quot{``I don't fit in anywhere, please, listen to me cry.''}

\listone
		\item Medium Size
		\item 30' Movement
		\item Humanoid Type
		\item Low-Light Vision: Half-Elves can see twice as humans in poor lighting.
		\item Immunity to sleep spells and similar magical effects, and a +2 racial bonus on saving throws against enchantment spells or effects.
		\item +1 racial bonus on Listen, Search, and Spot checks.
		\item +2 racial bonus on Diplomacy and Gather Information checks.
		\item Elven Blood: For all effects related to race, a half-elf is considered an elf.
		\item Favored Class: Any
		\item Automatic Languages: Common and Elven.
		\item Bonus Languages: Any (other than secret languages, such as Druidic).
\end{list}

\subsection{Half-Orc}
\vspace*{-8pt}
\quot{``I don't fit in anywhere, but you may be surprised to know that this dagger fits all kinds of places."}

\listone
    \item Medium Size
    \item 30' movement
    \item Humanoid Type (Orc and Human subtype)
    \item Darkvision: Half-Orcs can see up to 60 feet in the dark.
    \item +2 Strength
    \item +2 to Intimidate, Gather Information, and Survival checks.
    \item Favored Classes: Assassin and Barbarian
    \item Automatic Languages: Orc, Common
    \item Bonus Languages: Any.
\end{list}

\subsection{Halfling}
\vspace*{-8pt}
\quot{``Where are we going Mr. Frodo?''}

\listone
		\item Small Size
		\item 20' movement
		\item +2 Dexterity, -2 Strength
		\item +2 racial bonus on Climb, Jump, Listen, and Move Silently checks.
		\item +1 racial bonus on all saving throws.
		\item +2 morale bonus on saving throws against fear: This bonus stacks with the halfling�s +1 bonus on saving throws in general.
		\item +1 racial bonus on attack rolls with thrown weapons and slings.
		\item Favored Class: Rogue
		\item Automatic Languages: Common and Halfling.
		\item Bonus Languages: Dwarven, Elven, Gnome, Goblin, and Orc.
\end{list}		

\subsection{Hobgoblin}
\vspace*{-8pt}
\quot{``That's some tough talk from a man who wears a basket on his head."}

\listone
    \item Medium Size
    \item 30' movement.
    \item Humanoid Type (Goblinoid subtype)
    \item Darkvision: Hobgoblins can see up to 60 feet in the dark.
    \item +2 Dexterity, +2 Constitution
    \item +4 bonus to Move Silently checks.
    \item Favored Classes: Fighter and Samurai
    \item Automatic Languages: Common, Goblin
    \item Bonus Languages: Draconic, Elvish, Dwarvish, Giant, Gnoll, Ignan, Infernal, Orcish.
\end{list}

\subsection{Human}
\vspace*{-8pt}
\quot{``Yeah, I'm pretty normal.''}

\listone
	\item Medium Size
	\item 30' movement.
	\item Humanoid Type (Human subtype)
	\item 1 extra feat at 1st level.
	\item 4 extra skill points at 1st level and 1 extra skill point at each additional level.
	\item Favored Class: Any. When determining whether a multiclass human takes an experience point penalty, his or her highest-level class does not count.
	\item Automatic Language: Common. 
	\item Bonus Languages: Any (other than secret languages, such as Druidic). See the Speak Language skill.
\end{list}

\subsection{Kobold}
\vspace*{-8pt}
\quot{``Aieeeeeeeee!''}

\listone
		\item Small Size
		\item 30' movement (despite small size)
		\item Humanoid Type (Reptilian subtype)
		\item Darkvision: Kobolds can see up to 60 feet in the dark.
		\item -4 Strength, +2 Dexterity, -2 Constitution
		\item Racial Skills: A kobold character has a +2 racial bonus on Craft (trapmaking), Profession (miner), and Search checks.
		\item +1 natural armor bonus.
		\item Light sensitivity: Kobolds are dazzled in bright sunlight or within the radius of a daylight spell. 
		\item Favored Class: Sorcerer.
		\item Automatic Languages: Draconic.
		\item Bonus Languages: Common, Undercommon.
\end{list}

\subsection{Orc}
\vspace*{-8pt}
\quot{``Waaarrrggghhhh!"}

\listone
    \item Medium Size
    \item 30' movement.
    \item Humanoid Type (Orc subtype)
    \item Darkvision: Orcs can see up to 60 feet in the dark.
    \item +4 Strength, -2 Intelligence, -2 Charisma, -2 Wisdom
    \item Daylight Sensitivity: While in brightly lit surroundings (such as a daylight spell), an Orc suffers the dazzled condition and is thus at a -1 penalty to attack rolls and precision-based skill checks.
    \item +2 bonus to saving throws vs. Poison and Disease.
    \item Immunity to ingested poisons.
    \item +2 to Jump and Survival checks.
    \item Favored Classes: Barbarian and Cleric
    \item Automatic Languages: Orc, Common
    \item Bonus Languages: Dwarvish, Elvish, Giant, Gnoll, Goblin, Sylvan, Undercommon.
\end{list}

\subsection{Tiefling}
\vspace*{-8pt}
\quot{``My ancestors were more evil than you will ever know, but let's see how I compare.''}

\listone
    \item Medium Size
    \item 30' movement.
    \item Outsider Type (Native and Human subtype)
    \item Darkvision: Tieflings can see up to 60 feet in the dark.
    \item +2 Dexterity, +2 Intelligence, -2 Charisma
    \item Tieflings with a Charisma of at least 10 may cast darkness as a spell-like ability with a caster level equal to their character level once per day.
    \item +2 bonus to Bluff, Hide, and Move Silently checks.
    \item Favored Classes: Rogue and True Fiend
    \item Automatic Languages: Common
    \item Bonus Languages: Abyssal, Aquan, Auran, Celestial, Formian, Ignan, Slaad, Sylvan, Terran.
\end{list}

\section{Converting Monsters Into Characters}

\subsection{Method 1: The Easy Way}
Assume that a monster is a character of its CR+1(modified if it is a monster with the [Awesome] tag), and that its stat modifiers are derived from the assumption that the base monster was built using the Elite Array (highest monster stat -- highest elite stat, then repeat for next lowest, etc). For level-dependant effects like skill point maxes, feat prereqs, etc, use the monster's CR+1. Round ability stat mods down to nearest multiple of 2(negative mods up to multiple of 2), and CRs down to nearest whole number.

The nice part of this method is that it is easy, fast, and you can get to playing a monster immediately without as lot of DM intervention or paperwork. The downside is that you might get an underpowered or overpowered monster character if you are not careful (like you forgot that Dragons are actually CRed two less than they should be, or that Sprites are unplayable).

Here's two examples:

\begin{itemize}\itemspace
   \item \textbf{Minotaur:} Its Base CR is 4, and add +1 for being a PC. Its stat mods are (monster-elite array) Str 19-15=+4, Con 15-14=+0(rounded down) Dex 10-13=-2 (rounded) Wis 10-12= -2 Int 10, Cha 8-10= -2 Int 8-7=+0, for a total of +4 Str, -2 Dex, -2 Int, -2, Cha -2 Wis, which is perfectly reasonable. It's a level 5 PC with skill rank maxes of 8 and 6 monster HD.
      Frankly, it's a warrior class with a little bit of punch from natural armor, small stat mods from its size, and some fun but not good noncombat abilities. It's nothing to write home about as a 5th level character, and that's much more reasonable than the ECL 8 the MM would have you play it at.
   \item \textbf{Succubus:} CR 7, +1 for being a PC. Stat mods equal Cha 26-15=+10(rounded), Int 16-14=+2, Wis 14-13=+0(rounded), Str 13- 12=+2, Con 12- 10= +2, Dex 12-8=+4 for a +10 Cha, +2 Int, +2 Str, +2 Con, +4 Dex.
      It's an 8th level character who is almost as good as a Warlock of its level. Generally, it's a far better 8th level character than the than the ECL 14 the MM would have you pay. The fact that its abilities will never grow in power is offset by the fact that it has a high Cha, and so good DCs on its spell-likes.
\end{itemize}

\subsection{Method 2}
This method is the same as Method 1, but it goes a bit further by converting HD to actually appropriate HD by giving the monster the HD that equals its CR and BAB. This corrects problems just as excess HD from giants and undead.

Basically, look that the monster's HD and BAB. What kind of HD would it need to keep about the same BAB and HPs, but would give it the appropriate number of HD to fit its CR/level (which also fixes Saves to reasonably levels). Assign it that HD, and move on with your life.

Here's an example:
Fire Giant. Ok, the Fire Giant is a CR 11 as a PC, and notice that it has a BAB of 11, Great! Normally, it has 15 HD which leads to some craziness if he ever gets a Con boost and it has saves that are a little too big, so lets convert it. Lets give it 11 Barbarian HD(d12s, +1 BAB, good Fort save). We see that he keeps his BAB of 11, his HPs change from 142 to 133, and its base saves are Fort +7, Will/Ref +3 like an actual 11th level character instead of Fort +9, Will/Ref +5.

\subsection{Sample conversions}

Here are some relatively simple character conversions:

\subsubsection{Gnoll (Minimum Level 2)}

Lazy Hyena men filled with awesome? Where do I sign!?

\listone
    \item Medium Size
    \item 30' movement
    \item Humanoid Type (Gnoll subtype)
    \item Darkvision 60'
    \item +4 Strength, +2 Constitution, -2 Intelligence, -2 Charisma
    \item Proficiency in Light Armor, Shields, Simple and Martial Weapons, and the Flindbar.
    \item +1 level in the first Divine Spellcasting class a Gnoll takes.
    \item Scent.
    \item +1 Natural Armor.
    \item Favored Classes: Ranger and Druid
    \item Automatic Languages: Gnoll, Common
    \item Bonus Languages: Abyssal, Blink Dog, Giant, Goblin, Infernal, Loxo, Orc, Sphinx, Sylvan, Worg.
    \item 2 Starting Hit Dice (2d8 HP; 4 + Int Bonus x 5 skill points; +3 Fort Save; +1 BAB)
\end{list}

\subsubsection{Bugbear (Minimum Level 3)}

\listone
    \item Medium Size
    \item 30' movement
    \item Humanoid Type (Goblinoid subtype)
    \item Darkvision 60'
    \item +4 Strength, +2 Constitution, +2 Dexterity, -2 Charisma
    \item Proficiency in Light Armor, Shields, Shuriken, and all Rogue Weapons.
    \item +2 levels in the first Sneak Attack or Sudden Strike class a Bugbear takes.
    \item +3 Natural Armor.
    \item +4 Racial bonus on Move Silently checks.
    \item Favored Classes: Rogue and Ninja
    \item Automatic Languages: Goblin, Common
    \item Bonus Languages: Abyssal, Draconic, Elvish, Giant, Gnoll, Orc, Undercommon.
    \item 3 Starting Hit Dice (3d8 HP; 4 + Int Bonus x 6 skill points; +1 Fort, +3 Reflex, +1 Will; +2 BAB)
\end{list}

\subsubsection{Ogre (Minimum Level 4)}

Giants, even the lowly Ogre, are very specialized creatures. They dominate melee at their level, and really suck at everything else. As monsters, that makes them dangerous. While their glass jaws often leave them in situations that they cannot survive or even put up a decent showing, their laser-like focus can allow them to brutalize characters higher level than themselves if the lighting is just right. As characters, though, this makes them somewhat underwhelming. The ability to win super hard in one encounter only to die horribly in the next is worth less than nothing in a campaign game. An Ogre is a vulnerable and weak character for his level, but he does shine brightly if he can sucker opponents into melee. As such, Ogres really only do well in large, highly varied parties. As long as the remaining characters have potential bases covered extremely well, the fact that a single Ogre can't always pull his weight won't matter as much. For this reason, an Ogre often makes a better cohort than he does a primary character.

\listone
    \item Large Size
    \item 40' movement
    \item Giant Type
    \item Low-light vision and Darkvision (60')
    \item +6 Strength, +2 Constitution, -2 Dexterity, -2 Intelligence, -4 Charisma.
    \item +5 Natural Armor
    \item Proficiency in Light Armor, Medium Armor, Martial Weapons, and Simple Weapons.
    \item Favored Classes: Barbarian and Ranger
    \item Automatic Languages: Giant, Common
    \item Bonus Languages: Draconic, Dwarvish, Goblin, Halfling, Orc, Terran.
    \item 4 Starting Hit Dice (4d10; 4 + Int Bonus x 7 skill points; +4 Fort, +1 Reflex, +1 Will; +4 BAB)
\end{list}

\subsubsection{Frost Giant (Minimum Level 10)}

Right out of the box, the Frost Giant is a bad dude capable of rescuing the head of state from ninjas. Based largely on Norse mythology, these bad boys are big and bad. In fact, at 15 feet tall, they are about as big as you can get and still count as a large creature. That makes it pretty hard for them to find mounts, or fit into small buildings, and do all kinds of other crap that adventurers want to do. But it's not impossible. A Frost Giant isn't a Cloud Giant, he doesn't need people to make new doors to accommodate him, he just needs special doors to get through without it being really inconvenient.

A frost giant gets by in human society mostly because most people wouldn't dare mess with him. And that makes for a decent enough 10th level character.

\listone
    \item Large Size
    \item 40' movement
    \item Giant Type (Cold subtype)
    \item Low-light vision
    \item +12 Strength, +8 Constitution, +2 Wisdom
    \item +9 Natural Armor
    \item Proficiency in Light Armor, Medium Armor, Shields, Simple Weapons, and Martial Weapons.
    \item Rock Throwing and Catching (a Frost Giant's rocks have a range increment of 120 feet).
    \item Favored Classes: Fighter and Barbarian
    \item Cold Immunity and Fire Vulnerability
    \item Automatic Languages: Giant, Common
    \item Bonus Languages: Abyssal, Aquan, Auran, Draconic, Dwarvish, Gnoll, Orc.
    \item 10 Starting Hit Dice (10d10; 4 + Int Bonus x 13 skill points; +7 Fort, +3 Reflex, +3 Will; +10 BAB)
\end{list}