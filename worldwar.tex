\section{A World At War}
\vspace*{-10pt}
\quot{``Our people have fought with their people since the ancient days. It is stupid and wasteful. This cycle must end, which is why I must take up the sword as my ancestors did."}

\subsection{The Stone Ledger: The Dwarves Remember}

Deep in the mountains, the Dwarf people have records that go back to when most of the other races were learning about fire. Second only to the Aboleth themselves, the racial memory of the Dwarves extends to days beyond reckoning. At least, beyond reckoning to anyone who isn't a Dwarf. Dwarves keep their records chiseled into stone and preserved for all time with mystical might. While the spellbooks of the Elves eventually crumble into dust, the Ledgers of the Dwarves will stand in mute testament to their triumphs and failures for as long as day follows night and night follows day.

The Ledgers of the Dwarves measure in exact terms the location of all the cool things that the Dwarven people have found, they give tips for dealing with problems that Dwarves have overcome in the past, and they record in excruciating detail every bad thing that anyone has ever done to the Dwarven race. Remember that when you consider the implications of the fact that every group has at one time or another been at war with any other race you care to name. So the fact that sometimes goblins commit atrocities against Dwarf settlements means that each and every Dwarf child grows up reared on vivid and gory stories of generations of conflicts with goblins -- and goblins really don't. From the goblin perspective\ldots\  nothing is happening at all. Goblins don't live nearly as long as Dwarves do, and that means that they don't have a war with Dwarves even every generation.

This discontinuity leads to Dwarves being much better at the eternal war they are fighting with the Orcs, the Giants, and the Goblins than their opponents. That's because no one else really has the perspective to see that it is an ongoing conflict. The other races see it as a series of separate conflicts that are all individually about something, and mostly their poor record keeping techniques leave them often unable to even recollect the previous conflict. So really, the Dwarves keep winning because they are the only ones playing.

You may be tempted to ask ``If these wars kill thousands, and the only reason they're being kept alive is because of the Dwarf Ledger, doesn't that make the Dwarves the bad guys?" And honestly, that's a pretty good question. The Dwarves are Lawful Good and are the only race involved that understands the epic scale of the over-conflict. But that doesn't mean that they bear sole responsibility. Indeed, while the average Goblin on the street doesn't even know that there's an ancient rivalry between his people and the Dwarves, the list of usual suspects for evil overlords is a laundry list of people who actually also know the whole deal. Liches, Fiend Lords, and of course Maglubiet and Hruggek all know that Dwarves spend large amounts of time training and preparing for battle with the goblin people, and they don't tell the goblins. The thought is that by not telling the goblins that the Dwarves are totally ready for them and have been for thousands of years, that goblins will fight more bravely -- they literally don't know how very unlikely each individual goblin is to make it out alive from any conflict.

So life is pretty weird for a Dwarf. As a Dwarf you know that you are in an eternal struggle with the Goblin people. You also know that several times in your life, goblinoids are going to behave towards the Dwarven people as if nothing was wrong and have flourishing trade relations instead. But you also know that once every couple of goblin generations (which is to say several times in your life if you happen to be a Dwarf) some warlord is going to arise and send hordes of goblins to destroy your family. So if Dwarves come off as being intolerant jerks, that's why.

A special note has to be made about Dwarves and Arcane Magic. They like it. They are really good at it and have tremendous supplies of wizardly goods down in the depths. They can read spellbooks in the dark, and they are encouraged to do so. In some previous editions of D\&D the Dwarven people were not allowed to use Arcane Magic because Gimli wasn't a spellcaster (the actual reasoning, I'm not even making that up), thereby ignoring the Dwarven magicians in many source legends (the Ring Saga for one), and even the Dwarven Magic from the Lord of the Rings. Fortunately, the bad old days are behind us, and Dwarves are back where they are supposed to be -- slinging spells, scribing runes, and crafting magic items in their mountain halls.

\subsubsection{Campaign Seed: Secrets Revealed}
Key pieces of the Stone Ledger have been made public. Pamphlets explaining the situation in Goblin have been given mass distribution. The cycle of violence and peace that has dominated Dwarf/Goblin relations for millennia is coming crashing down. Reactions to the news vary of course. Some Goblins want to mount a final campaign to end the Dwarves once and for all, and others want to simply drop the whole thing and have a permanent peace. With a properly placed word or dagger, you could probably ensure that the proper outcome occurs. But what of the other groups thinking the same thing?

\subsubsection{Campaign Seed: The Blank Spot}
The great map of the Dwarves includes pretty much everything that dwarves have seen and lived to tell about. And yet, there are tunnels in the deep below that lead to\ldots\  nothing. Despite thousands of years of diligent tunneling and mapping, there are still blank spots on that map. Obviously, foul play is involved, but how could something stay so deadly for so long? On the other hand, what if it's simply that the place is so valuable that no one comes back?

\subsection{Gnolls: Too Lazy to Win}
As a race, individual gnolls are powerhouses, each being worth two or three of the lesser races in combat\ldots\ so why don't they rule the world? In short, they are lazy.

Gnolls just don't do the things that would make them successful. They don't organize themselves, they don't amass wealth or build structures, and the reason for this lack of productive behavior is that they are profoundly lazy as race, making them slighty stupid. Being lazy, they know that it takes less work to take such things from weaker races, and so this makes them mean. Gnoll heroes are manic by the standards of their race, since they seek out new experiences rather than stay at home to participate in tribal infighting.

Gnolls take favorable territory, and lesser races have a hard time displacing them, meaning that when someone sufficiently powerful does come along that can challenge the gnolls, it usually kills them outright in order to prevent the need for them to be displaced again. This means that gnolls tend to not have the same kind of shared cultural history as other races. Gnoll clans tend to be undisputed masters of their domain, spawning countless lesser tribes over the years who will attempt to take their own territory until at some point the each of those tribes is destroyed, meaning that each of the tribes has a limited amount of time to exercise its dominion before some greater power strikes them down, but since in that time the tribe spawns smaller tribes, the race as a whole survives.

Being racially lazy, powerful, and prolific means that occasionally one of those greater powers will decide to harness the gnolls, forcing training, discipline and purpose onto them. These Witch Kings or Warlords will then use the gnolls as elite shock troops or peacekeepers that enforce the dominion of their master, but the gnolls obey only as long that they know the hand of their master can reach them. Once the Witch King or Warlord dies, the gnolls revert to form, building tribes and bullying lesser races until they are driven from civilization.

\subsection{Vistas of the Giants: Big and Important Stuff}

Giants are more than just a Jungian representation of the complex feelings and resentments we have for our parents while we're children. Honest. Giants live in a metaphorically separate world from the smaller races, a world where everything is big -- except populations. While Giants eat big food, have big castles, and throw big rocks, a major incursion of Giants is seriously like 8 guys. And while that can really put a giant dent in your day -- the fact is that there's no way for them to kill your people fast enough for that to matter. Giants are simply not going to have a serious effect on your total population no matter what they do -- because there aren't enough of them to ever amount to more than crime.

Giants appear in flavors that correspond to anything you can possibly imagine being much larger and more hardcore than it is. There are giant orcs, there are giant dwarves, there are giant elves, there are giant rogues and there are giant druids. So really any social dynamic you can imagine amongst the small people is replicated in excruciating hugeness among the big people.

An important thing to remember about Giants, however, is that very few of them are as tall as a tree -- let alone a mountain. The vast majority of Giants are Large, not Gargantuan or even Huge. They are Giants like ``Andre the Giant". They're big, but Fire Giants aren't impossibly big. If you saw them walking around on Earth you'd go ``Man, that guy is big." but that's about as far as it would go. Still, for all the fact that Giants are rather disappointingly within scale of normal humans (seriously, the picture of the Giant Slayer in the DMG2 with the chopped off head the size of himself -- that's much larger than even a Titan head), they are amazingly hardcore when it comes to combat. That same Fire Giant can easily wade through a group of 20 orcish warriors, that's not even a major problem for him.

\subsubsection{Campaign Seed: The Land Above}
Those Cloud Castles can't just be built anywhere, they require relatively stabile cloud formation to be built upon. And I know what you're saying ``Relatively what in the what now!?" Right. The D\&D world has cloud formations that are persistent, structurally sound, and capable of supporting several thousand tonnes of weight without buckling. Sure, those sky continents move around much faster than the tectonic plates do, but the surfaces are solid enough to keep a castle afloat for a thousand years.

What's even better of course, is that these Cloud Islands are more than 40' thick. You can't scry on them or teleport to them. It's like having a dungeon that you can still grow beans in. The castles you build here are safe from prying eyes on the ground. And that means exactly what adventurers hope it means: undespoiled ruins. If you have a means to the over world, you have access to new vistas of adventuring populated by empires and monsters that the underworld has never heard of.

\subsubsection{Campaign Seed: Vacillating Terrors}
The Giants are huge. Well, a lot of them are merely Large, but their impact on the field of battle is huge. And there's only a couple of them. That means that with a good assassination, a well placed word, a hefty bribe, or some basic seduction, the giants on one side or another of a conflict can be made to drop out or even switch sides. The impact on the battlefield from these relatively minor acts can be huge, and are totally worth it for both sides.

Giants understand this, and can get pretty greedy. Nevertheless, intelligent kingdoms will often assign adventurers to pampering the whims of these Giants to make sure they stay on the correct side in important confrontations.

\subsection{The Goblin Empire: Silent Loyalty, Silent Dissent}

There are at least three kinds of Goblin. That's important, not only because it means that any group of Goblins has access to a great many opinions and skill sets, but also because it means that the Goblinoid physiology is extremely morphic. And because of this, and because no one really cares if goblins disappear, when a wizard or demon decides to make a new form of super soldier -- chances are good that they use Goblins as a base. Heck, you don't see any halflings with rhino horns on their face, and you don't see any dwarves transformed into undead monstrosities with bone-sucking tentacles popping out of their nipples. That's all the dubious pleasure of the Goblin people.

Goblinoids are, as a people, much quieter and more precise about their movements than other races. And this allows them to live in much higher population densities than other races without going mad. And well, they totally do that. Goblinoid settlements are, by the standards of other races, amazingly claustrophobic. Bugbear settlements traditionally make walls out of paper and place living quarters right next to one another to conserve heat. Those not blessed with the bugbear's natural silence find their every action heard many apartments away. Goblins usually dispense with the paper altogether and simply sleep ten to a room. Fortunately for them, goblins do not snore.

While goblinoid societies are classically short on free space, they are also not generally well organized. Goblins live together not because they like sharing, but because they steal from each other so constantly that it's just a waste of time to put walls between sleeping areas. If a goblin needs something, he'll take it and use it. Goblins aren't socialist utopians or anything, they simply don't respect property rights of others. Oddly enough, the end result is pretty similar to Goblins being really cooperative. Hobgoblin society takes it one step further and even has elaborate rules about who has to submit to who and when people have to take their shoes off and how people have to behave in public and everything. They actually are well organized, and their intricate webs of subjugation allow them to maintain high population densities without eating each other.

Goblinoids go to war for really one reason only: they want your stuff. Hobgoblins need constant influxes of new Slaves to keep everything rolling (even Slaves gain in seniority and prestige in time within Hobgoblin social structures so the bottom rungs of society can really only be replenished from captured enemies). Goblins want your shinies and aren't afraid to torch your village to get them. And finally, the Hruggek demands that the Bugbears slaughter your people from the shadows on a fairly regular basis. That's like wanting your stuff, only in this case what they are taking from your lands is the satisfaction of having seen your last breath from the back.

\subsection{Elves: Servants of the High Wizards}

The individual elf is a fine adventurer, blessed with many attributes that make them well suited to a life of killing monsters for their hard-earned possessions. They are extremely long-lived, quick of eye and reflexes, and blessed with the kind of training that comes from a childhood that spans decades. With all of these benefits, one wonders why they don't rule the world.

The answer is simple: they have a secret. That secret is that elves, as a race, are the pawns of powerful wizards. Just as powerful wizards have taken the heads of giant owls and put them on the bodies of bears, some wizards in the far past decided ``hey, lets make a race that's hot, skinny, and long-lived enough to learn to really please me.'' The end result is a race whose favored class is Wizard, a class requiring study and materials. As engineered servants of powerful wizards, they mystically have the ability to learn their master's arts. The influence of the overlord wizards is the explanation for the variation in the subraces of elves: height, skin coloration, racial abilities, and physical and mental attributes are shaped by the overlords to suit their favored environment and tastes in beauty. Wild elves are physically powerful but dim, while snow elves are hardy but racially arrogant and haughty, and this all stems from the tastes of their wizard overlords in the past. Art and music is encouraged among the young because it makes them more attractive to their overlords.

This doesn't mean that your average elf is directly under the thumb of anyone; elves, like any race, have the ability to grow in power by testing themselves against dangers that can kill them. This means that the wizard overlords of the elven race are in fact elves now; like the githyanki, they threw off the shackles of their overlords eons ago\ldots\ only to wear shackles designed by members of their own race. This is why in places like Faerun, any individual elf can't even go to the elven homeland without doing something drastic (like promise to never leave). Powerful magic protects these places because the elven high wizards that rule the race live in these locations, and they receive only benefit from letting individual members wander the world collecting new experiences and magic to hopefully bring back to them.

As a race, elves of all professions tend to think like long-lived wizards. They know that they can potentially live hundreds of years, so they tend to be very risk adverse. In a word, like any wizard who survives very long in the D\&D universe, they are cowards. They don't allow ideas like ``permanent homes'' or ``pride'' to get in the way of survival. Your average elf lives in the woods because the woods have a lot of hiding places and a native of any particular woods can outrun any non-native trying to catch them, and if your home in a tree burns down you can easily build another home in another tree. Archery is encouraged among elves because it keeps your enemies at a distance, and it grants elves the ability to attack from hiding. Stealth and a distributed cell structure to their society keep them alive long enough for their wizards to prepare a strategy to beat their enemies. They harass and use hit and run tactics to wear down enemies to buy time for their wizards to draw upon their hundreds of years of experience in order to deal with the enemy.

\subsection{Warrens of the Gnomes: Guerilla Illusionists}
Gnomes are one of the few innately magical races. Every gnome starts his day able to speak to burrowing animals, and every gnome of normal intelligence can cast a few simple illusions. Normally, this is not a recipe for a master race. And in truth, gnomes conquer nothing. Their great strength is that they don't fight fair.

Gnomes fight with a siege mentality, but with an extremely creative bent. They build their homes in hills to conceal them, then they cover these up with illusions. They use illusions of sound and light to misdirect enemies, and they use a network of trained burrowing animals to spy on the locations of their enemies. As small creatures, they excel at hiding and as a race they all have the ability to perform minor magical tricks that a creative person can use to any number of effects. They are hardy warriors with a flair for alchemy, so enemies of the gnomes can expect tough and brutal battles in conditions of smoke and flame with troops hindered by caltrops and tanglefoot bags, their horses driven out of control by thunderstones.

Gnomish heroes are well known for fighting giants, but in combat this edge tends to be minimal, as is their knowledge of fighting techniques against goblinoids; in truth, gnomes attack giants at range and from covered and inaccessible positions to avoid being hit with boulders or forced in melee and have learned to strike well at other goblin races to end combats where these races might overrun the gnomish positions. The real enemy of the gnome is the kobold, as these two races tend to cancel out each other's strengths: kobolds use traps that care nothing for illusions, and they hide as well as gnomes but have an enhanced ability to search an area and so find hiding foes and traps, and each race is equally at home in the other's Small-sized tunnels. Kobolds are also better ranged attackers and are naturally armored, making them slightly better combatants. Kobold/gnome wars are masterpieces of misdirection and stealth as each race sets traps and ambushes with gnomes leveraging their innate talent with illusions and kobolds using traps, ranged attacks and melee ambushes, and their own sorcerous talents.

\subsection{Orcs: The Endless War}
Orcs are the product of a generations-long war against the other races. Unfortunately, they haven't realized that they've lost this war. Why the war starts is simple: orcs are, as a race, stupid, ugly, and weak willed, but very strong. Being stupid, ugly and weak willed means that other races tend to always get the upper hand on them and tend to always get the better end of any deal, and other races also tend to not want Orcs around. Orc goods are always a little worse than goods produced by other races, and orcs are generally a little rowdier and less pleasant to be around.

At some point the orcs realize that they are much better in battle than other races, and they decide to fight for a little respect and fair treatment. Then the war is on. The only problem is that orcs win battles, but lose wars. Other races have natural advantages or just greater intelligence, so any war tends to go badly for the orcs in the long run. Powerful melee combat ability doesn't mean much when elves attack from the bushes with longbows and then run away and all the races have superior battle plans and ability to lead their troops.

Once the war has been decisively won, the orcs are driven out of their lands and pushed into some badland, hinterland, or some other undesirable terrain far away from trade routes and civilization and usually full of monsters. The other races then go back to their lives, but here's the trick: the orcs don't. As far as the orcs are concerned, the war is still on because the orcs are still stuck in the worst land in their area, scraping by in the wilderness with minimal natural resources and almost no access to the products of civilization like arable farmlands, centuries-old cities, and trade goods like the products of skilled craftsmen from other lands (which can include magic items).

All of orc culture comes back to this issue. Orcs are constantly warring on other races not out of innate need for violence or evil inclinations, but because they are fighting for their survival as a race in lands considered undesirable by every other major race. Orc raids are not only for food and booty, but for all the things that orc culture cannot produce like tools and weapons. Without these things they cannot survive in the wilderness, and they cannot produce them in the wilderness living as nomads who hunt and gather for survival.

Orc hordes are not an indication of warlike racial tendencies, but of population issues. Once the orcish population in the badlands grows too large to be supportable, they must conquer new lands or else face death by famine and disease. Hordes are formed of ``excess'' young males that are sent off to carve out new lands or die trying\ldots\  both results ease the burden on the few resources in the badlands.

The fact that orcs are constantly in a war footing means that they easily offend other races with their tactics. Rather than fight elven guerilla fighters who sap their resources and manpower, they'll burn the forest down, and rather than fight dwarves in their millennia-old and heavily entrenched deepnesses filled with traps, the orcs will collapse the tunnels and dig the booty out of the rubble. The fact that most races fight defensively means that orcs only gain tactical advantage by being extremely offensively-minded. The fact that orcs do not have supplies coming from the badlands means that while they have no supply trains to cut, they must conduct blitzkrieg-style war or face starvation, and they cannot afford to hold troops in reserve. They often just don't have the resources needed to conduct honorable or civilized war, and their attacks seldom have finesse or timing on their side, meaning that they only win battles through overwhelming force. Night raids are their specialty, as they have darkvision and are sensitive to light.

\subsection{Borderlands of the Sahuagin: Sore Winners}

The first thing to understand about the Sahuagin is that they have already won. Completely. The surface of the world is about \sfrac{3}{4}\  ocean and they own almost all of it. From the standpoint of the Sahuagin, the only places on the planet that have non-Sahuagin races in them are the stale crusts that they already had the presence of mind to cut off their sandwich. All of the non-Sahuagin races are all ghettoized. Even the other aquatic races have been marginalized to the point where they only get the brackish water (Locathah), the rocky shallows (merfolk), the underground darks (Kuo-Toans), or the muddy salt marshes (Lizardfolk). The real real estate -- the ocean and coastline -- are pretty much the private playground of the Sahuagin.

Individually, Sahuagin will kick your ass, and collectively they will kick the ass of any nation you happen to support. The combined populations of all other sapient races on any planet are less than the population of Sahuagin on that planet. The Sahuagin are also much smarter and better organized than you are so their cities are actually more productive than yours per person in addition to the fact that they have more cities than all the other races and their cities are more populous.

The Sahuagin mutate constantly, but are not inclined to Chaos. They just all have different appearances and capabilities. But every one of them is gifted with super intelligence and thick natural armor. The Sahuagin deep seers are some of the most gifted wizards on the planet and honestly have nothing better to do than just scry on crap and tell the armies where there's some cool stuff to go loot. From time to time the Sahuagin will come onto land to beat the living crap out of people and take control of important or valuable items. Then they take the spoils of war and drag it back under water, laughing the whole time.

Against this backdrop of crushing inferiority, how do the other races maintain? Most of them are fighting for stakes so small that they haven't even noticed that the vast majority of the planet is owned and operated by brutally efficient fish men. But one race that certainly has noticed the power discrepancy is the race of elves most likely to be forgotten: the Sea Elves. They actually live in many of the same areas and have a war going with them.

Life is hard for a Sea Elf, because every one of them is born into a post-apocalyptic world where mutants run amok and hunt them for sport. But it's actually even worse than that because in addition to simply being physically and intellectually inferior to the Sahuagin like everyone else is -- they are actually stupid and useless even contrasted with the surface races. An average Sea Elf is as much the intellectual inferior to a Sahuagin as a Griffin is to a normal human. The Sahuagin consider the Sea Elves to be little more than animals, and they aren't wrong.

The Sea Elves keep surviving at all because they see farther than Sahuagin in low-light conditions (and are thus often able to swim away from potential encounters with Sahuagin during the morning and twilight hours that Sea Elves leave their hidden nests), and also because every so often a Sahuagin gets born who looks exactly like a Sea Elf. These Sahuagin mutants, called Malenti, are a little bit worse than a normal Sahuagin in that they lack the rending claws. But they're still stronger and smarter than any Sea Elf that ever swam the 7 seas. So when these Malenti realize that they get a crap deal from Sahuagin society, they often as not run off to join the Sea Elves, where they almost immediately rise to positions of leadership. They also gain crap loads of experience very quickly because the odds are so stacked against them. In short, the reason that the Sea Elves still exist is that they actually are a splinter faction of Sahuagin that uses real sea elves as beasts of burden instead of simply hunting them like the more normal Sahuagin groups do.

And yet, despite the fact that the Sahuagin have won at everything, they still continue to fight the other races and take their children and stuff. Partly this is to feed the insatiable demands of their Baatezu masters, and partly this is because on some deep level the Sahuagin are convinced that it actually couldn't possibly be that easy. In addition to looking for bling and candy to take from the weaker races, the Deep Seers are also combing the world for the one thing that the Great Mothers are pretty sure exists somewhere: the hidden army that the other races are putting together to take the world back from the clutches of the Sahuagin Empire. As far as anyone knows, it doesn't exist, but for some reason the Great Mothers keep insisting that the searching continue. Maybe they know something we don't?

\subsubsection{Campaign Seed: Free Your World}
The Sahuagin have pushed things too far. After the leveling of the city of Kelport, the remaining peoples of the land have at last come to realize the danger that the Sahuagins' unchecked strength poses. The natural alliance of pretty much everyone against the Sahuagin has formed. But how far can you trust your allies? Will the goblins really show up when they said they would? And does everyone together have the strength to topple the coral spires of the Deep Seers?

\subsubsection{Campaign Seed: The Price of Hubris}
In ages past, the Sahuagin conquered the seas of the Kuo-Toa. They crushed their temples, and slaughtered their children. And no one liked the Kuo-Toa because of all the sacrificing people to the Great Evils they used to do, so no one did anything about it at the time. As massively successful empires are wont to do, the Sahuagin have allowed themselves to become decadent and haven't been crossing their Ts particularly, and now the Great Evils are straining to enter the world. That's\ldots\  unfortunate\ldots\  because these ancient and malevolent forces have the power and inclination to destroy everyone on the planet. And to make things worse, while some of the Sahuagin are aware of the problem and contracted our heroes to help solve it, lot's of other Sahuagin refuse to acknowledge that any problem could possibly warrant getting help from outsiders and will work against you at every turn.

\section{After the War}
\vspace*{-10pt}
\quot{``Everything ends, and everything dies."}

Every war has a beginning, middle, and an end. And from a dramatic storytelling point of view, the periods before the war and after the war can honestly be just as awesome as the war itself. Periods before war are, frankly, just like periods of peace and don't warrant being included in this text at all. Periods after wars can be quite compelling as well.

It is a common myth that all wars have winners and losers. The truth is that there are many wars that don't have any winners. Nevertheless we will classify the afterwar campaigns by the signature winner or loser of the last conflict. Often a war will have many winners and losers, so really this can be thought of us a jumping off point for the people the story is most interested in.

\subsection{Triumph of the Halflings: Reconstructing the Shire}

How many of you actually read the Lord of the Rings rather than simply watching the movies? Perhaps the biggest and most awesome part of Halfling lore is the part where they have to pick up the pieces after their shire has been razed. So here you have a situation where the halflings have won, they have conquered and they can invoke their rights as conquerors to impose their culture on the defeated.

But that's a problem. Halfling culture is all about not doing that, it's a very nice society that produces a lot of grain and leads by example. Halfling society has Mayors who rule because they are well liked and have good ideas -- not necessarily the strongest adventurers. The entire point of the ``Outrider" culture is in fact to get powerful Halflings into a prestigious position where they don't control the day-to-day workings of society.

When the Halflings become conquerors, their whole way of life is disrupted. Suddenly the Outriders do run the show -- or at least those parts of it as are on Goblin land. Remember, absolute power corrupts and all that. Halfling society has never really had to contend with a leader who wasn't easily replaceable. With the masters of war in control, how can the shire ever be rebuilt the way the people want it to be? And when it comes down to it, should the Shire be rebuilt the old way? The last time around, The War happened, and that wasn't good for anyone. Maybe a new direction is the best thing.

\subsection{Defeat of the Halflings: They Came and Took Our Land}

Halflings are, as a people, fairly non-confrontational. So it is perhaps unsurprising that Halflings who had been on the losing side of The War would want to leave. Really, most halflings aren't going to disperse into the wilds to conduct a guerilla war against their oppressors and stage a partisan movement to attempt to make the holding of Halfling territory implausibly expensive\ldots\  they're just going to pack up and go. And a perfectly reasonable opening curtain for a D\&D campaign is right there -- in the trains of refugees flooding out of former Halfling territories.

Where will they go? How will other races, even other Halflings, respond to the promised influx of new mouths to feed? It's a nasty proposition, and it really tugs at the heart strings because Halflings look kind of like children anyway, and watching them fleeing with all their worldly possessions into an uncaring world while genocidal enemies pursue them is emotionally effective.

\subsection{Triumph of the Dwarves: Breaking the Cycles}

The Dwarves don't consider themselves to have ``won" just because the goblin invasion has been broken or the last orc warrior has passed out from lack of supplies. No, they understand that the goblins will be back and the orcs remain in the Savage Lands. Team Monster will return, probably within the lifetimes of the Dwarves fighting the last battle, so they've bought themselves a respite, not a victory. But imagine for the moment that the Dwarves actually have won. Maglubiet himself has agreed to order the goblins to leave the Dwarves alone. What now?

The Dwarves have no answer for that question! Their entire way of life depends upon readying themselves for the next battle in an endless struggle. With the actual end of the struggle, their society collapses. Sons do not listen to fathers, and Dwarves of all ages take up beatnik poetry. Cats and dogs live together and currency and hard work lose their value. What would the Dwarven elders do to put things back on track? What new ways could the Dwarves embrace that would allow them to move forward?

\subsection{Defeat of the Dwarves: The Tunnels Forgotten}

It takes a lot for Dwarves to actually lose, just as it takes a lot for them to win. The preponderance of Dwarves really will fight to the death and they are quite good at doing that. But they do have a contingency. They have a backup plan that involves taking a bunch of women and a few men and spiriting them away to various parts of the underdark to rebuild the race in secret. Did you know that sometimes they get excited and activate this plan without actually having lost yet? Then they send a colony pod off into the underdark and are stuck in a position where they can't easily recall them. That's where the weird Dwarf colonies come from. Sometimes it works out, and eventually contact is restored with the ``Deep Dwarves." Sometimes it really doesn't work out well for anyone and you get Duergar.

\subsection{Triumph of the Goblins: What's Yours is Mine}

Getting conquered by the Goblins really has very different effects depending upon which Goblins are in charge when they overrun your defenses. The Hobgoblins have the most intrusive plan -- where your people are enslaved and forced to work for and even join the Hobgoblin clans. The Bugbears have perhaps the least disruptive plan, where they simply run into your village and kill and eat anyone they can catch and then go back to their own lands with everything they can carry. The regular old Goblins, on the other hand, mostly want to fill santa sacks with your stuff, and then come back tomorrow and do it again. It's like taxation, only it's set to ``whatever they can carry" and you have to pay it ``whenever they show up."

Living under the yoke of the Goblins can be anything from an excuse for lots of dangerous random encounters (Bugbears have overrun your nation), to a semi-comic game of fighting semi-organized crime (Goblins), to a role-play heavy pseudo-Japanese setting where the PCs are all ronin or ashigaru or something (Hobgoblins). It can even be more than one of those, in the not-unlikely case that more than one group of Goblinoids is involved. In this case, you're normally going to be forced into a society where Hobgoblins are Samurai, Bugbears are Ninja, Goblins are Yakuza, and you're a serf. This is your chance to do a Kurosawa film from the perspective of those guys in the background harvesting rice with a knife under the disinterested glare of a distrustful Samurai.

\subsection{Defeat of the Goblins: A Land of Banditry}

Again, since the Goblins are really three very different groups, them losing The War represents here extremely different results. The Hobgoblins will probably simply install their conquerors in the highest positions of their Empire and then enthusiastically change their methodology as little as possible. It's like being MacArthur after the handover of Nippon. The Goblins will likewise attempt to ignore their new masters as much as possible, though they differ from the Hobgoblins in that they will place themselves into the command structure of their new conquerors -- to the extent that they happen to be in the presence of said conquerors. The Bugbears, however, are too proud to bother to pay lip service to any so-called conquerors. Mostly, the defeated Bugbears will vanish into the wilderness and proceed to live like werewolves. In that respect, beating the Bugbears is a lot like being beaten by the Bugbears, except that there are less remaining Bugbears.

Regardless, conquered lands of the goblinoid peoples are filled with what the new conquerors could graciously refer to as crime. Pockets of resistance, or just plan stubborn refusal to change to the new program -- goblins are generally quite happy with the new regime but only because they pay it as little heed as possible. And for a goblin, that's a very small amount.

\subsection{Triumph of the Necromancers: Endless Night}

Life sucks when the ravening horde of Wights and Shadows overruns your kingdom. In fact, life probably doesn't even exist. Those that survive will normally have done so by taking shelter in small hallowed areas that the undead will not enter. But here's the exciting part: once all life is gone in the region, the Wights can't replace themselves. Sure, if you start with one Wight and then every day every Wight makes another Wight you'll have an army one million strong in 3 weeks -- but that's already happened. They won, and now the Undead are on the down slop of the Spawn cycle. It's really ugly, but you can retake the world. In fact, you're probably going to. Necropoly isn't really a government that lasts all that long in most D\&D settings.

So here's how it works: you spend your time in the hallowed grounds biding your time. Then, you come out and kill a couple of undead beasties. Then, the various Necromantic Intelligences that have sprung up will direct undead soldiers to go get you, so you'll retreat back to the protected zone. Then you rinse and repeat. It's like a high fantasy post-apocalypse world. As long as you remember that you're small and furry and have to stay out of the way of the dinosaur zombies, you're capable of chipping away at the onyx gauntlet that grips your kingdom.

\subsection{Defeat of the Necromancers: Resource Rush!}

OK, what does a necromantic army do to the land it passes through? Well, for starters it kills everything. Everything. That means that it leaves only the inanimate stuff behind. The soil, the houses, the gold, that sort of thing. In short, if you come in there with some seeds and some dreams after the necromantic army has been destroyed (and remember, many necromantic armies fight to the last), there is a bunch of livable land with no occupants and no monsters.

That is comedy gold right there, and every group of humanoids in the area is going to send all their second sons off to go try to colonize. That means that you have extremely mixed race settlements in the newly opened region. Gnolls live right next to Gnomes for reasons other than alphabetical assignment. But other than getting to live in the newly opened Oklahoma Territory with a bunch of radically different sapient species who don't speak the same language or get along, remember that the monsters are coming back as well. This is empty land, so the monsters going in are doing so at a rate literally infinitely faster than the rate of monsters going out. Sure, it may be a trickle, but it's completely asymmetric. When a displacer beast comes in to the region, it won't have any of its normal food sources or enemies available -- so it's just going to go straight for the villages.

So while the monster presence in the area is almost insanely low by D\&D standards, all of the monsters are going to immediately attack humanoid settlements as soon as they show up. That really makes it easy to DM, let me tell you.
