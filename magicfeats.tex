\section{[Magic] Feats} \label{feats:magic}

Magic feats scale based on the highest level spell you can cast.

\begin{multicols}{2}	

\magicfeat{Battlecaster [Magic]}{You fight with sword and spell both. Cause it's better that way.}{You receive a +2 dodge bonus to AC while casting spells. Additionally, you gain proficiency in one weapon for free.
}{If you hit with a physical attack, the creature struck takes a -2 penalty to saves and SR against your spells until the end of your next turn.
}{You may cast a standard action spell as part of a physical attack. The spell effect is delivered to the target of the attack instead of its usual area and targets if it hits. Any saving throws are still rolled, but the spell does not require an attack roll if it normally has one. If the spell has secondary targets (e.g. chain lightning), they require attack rolls if they would be needed, but area spells only affect the target you strike instead of the normal area. You provoke attacks of opportunity as normal. The physical attack occurs before the spell's effect.
}{As a swift action, you may expend a spell slot or prepared spell to enhance a physical attack. You may add a +1 competence bonus to the attack roll and a +1d8 bonus to damage per spell level of a physical attack this turn per spell level expended this way.
}{When you make a full attack, you may cast a standard action spell as a swift action.}

\magicfeat{Energised Evocation [Magic]}{That, and everyone who disagrees with you has to talk to Mister Fireball.
}{You receive a +2 to saving throws vs evocation spells.
}{Your evocation spells receive +2 to their DCs, and you receive a +2 bonus on Spellcraft checks relating to evocation spells.
}{As a swift action, you may add the [Force] descriptor and change half the energy (acid, cold, electricity, fire, sonic) damage to force damage for the next evocation spell you cast this turn.
}{As a swift action, you may modify the next evocation spell you cast this turn with an area to change its shape to a cylinder(10ft radius, 30ft high), a 40-foot cone, a ball (20-foot-radius spread), or a 120-foot line. The sculpted spell works normally in all respects except for its shape. You may also choose to exclude a 5ft cube from the spell effect for every 3 character levels, whether or not you changed the spell's shape with this.
}{As a swift action, you may cause the the next evocation spell you cast this turn to take effect twice in the same area or on the same target simultaneously. Any variable characteristics (including attack rolls) or decisions you would make about the spell (including target and area), are applied to both spells, with affected creatures receiving all the effects of each spell individually (including getting two saving throws if applicable). A spell whose effects wouldn't stack if it was cast twice under normal circumstances will create redundant effects if successfully twinned (see Combining Magical Effects, page 171 of the Player's Handbook).}

%Level 9 option needs a change, too ccccraaaazzyyy
%\magicfeat{False Reality [Magic]}{Why should you care for truth?
%}{You receive a +2 to saving throws vs illusion spells.
%}{Your illusion spells receive +2 to their DCs, and you receive a +2 bonus on Spellcraft checks relating to illusion spells.
%}{As a swift action, your next illusion spell this turn has no verbal or somatic components.
%}{You can concentrate on a single illusion spell currently active as a free action. Additionally, if a viewer that has penetrated an illusion you cast and informs others of it, the others do not gain their normal +4 bonus to disbelieve it. When presented with incontrovertible proof that it is an illusion, creatures must still succeed on a Will save to disbelieve it, at a +4 bonus.
%}{Your illusions can fool reality itself, and are 50 \% real. If an affected creature disbelieves it, or sees through the illusion (e.g. with true seeing), there is a 50 \% chance that the illusion functions normally. For spells with the [Shadow] descriptor, you take the better of the percentage effectiveness, rather than stack them.}

\magicfeat{Great Oracle [Magic]}{Knowledge is power, and you know everything.
}{You receive a +2 to saving throws vs divination spells.
}{Your divination spells receive +2 to their DCs, and you receive a +2 bonus on Spellcraft checks relating to divination spells.
}{You can concentrate on a single divination spell currently active as a free action and the durations on all your divination spells double.
}{When you cast a divination spell, you gain an insight bonus to initiative, saves and AC equal to the spell level. These bonuses last for 24 hours or until the bonuses are used in some way (e.g. rolling initiative, rolling a saving throw or being attacked by a melee or ranged attack). Such a use eliminates all bonuses.
}{As an immediate action once per encounter, you can change your initiative to before the active creature's. This allows you to take your next turn immediately, and interrupt that creature's actions.}

\magicfeat{Lord of The Thunders [Magic]}{You are a Lightning Warrior.
}{Electricity burns just ain't what they used to be. You gain electricity resistance equal to your character level.
}{You can alter any spell you cast that deals energy damage to deal electricity damage instead. That spell loses any energy descriptors it had, gains the [Electricity] descriptor instead and all its energy (acid, cold, fire or sonic) damage becomes electricity.
}{Any spell you cast with the [Electricity] descriptor ignores an amount of electricity resistance and hardness equal to your character level. Creatures immune to electricity receive half damage from electricity damage instead of none from your spells with the [Electricity] descriptor.
}{By spending a swift action, you can alter any spell with the [Electricity] descriptor to deal half of its electricity damage as sonic damage. That spell also gains the [Sonic] descriptor. Additionally, all creatures that are hit by the spell's attack roll, take damage from this spell (if it deals damage) and/or fail their saves against the spell must make a Fortitude save at the spell's DC or be stunned for 1 round, and a further Reflex save at the spell's DC or be knocked prone.
}{The essence of the storm surrounds you, you gain an aura of wind and lightning. You are surrounded by an stormy aura with a radius of 5ft per 4 character levels that deals 1d6 damage/2 character levels to everything inside. You may change the wind force by one step per 3 character levels, and the wind circle you in a clockwise or counterclockwise fashion, leaving your space calm. Existing wind effects may increase or decrease the wind force. You may activate, deactivate or change the force and direction of this wind as a swift action.}

\magicfeat{Master of Magic [Magic]}{"Magic is everything." - Wizards' creed
}{You receive a +4 competence bonus on caster level checks to defeat spell resistance.
}{You can take 10 on caster level checks.
}{If you miss with a spell that requires a ranged touch attack, you may reroll that attack with a -5 penalty. If the reroll misses, you miss and all the other mages laugh at your poor aim.
}{Pick a spell 4 levels below the maximum you can cast, provided it does not cost XP. You may use that spell as a SLA at will. As you gain access to higher spells you may trade this out accordingly for more powerful SLAs.
}{You ignore Spell Resistance and Immunity to Magic. In fact, you are so awesome that you can ignore antimagic fields with an opposed caster level check. Antimagic fields without caster levels use the generating creature's CR as their caster level. DM's discretion applies to antimagic fields without caster levels and or originating creature.}

\magicfeat{Master Conjurer [Magic]}{You bring everything to the right place, at the right time...for you.
}{You receive a +2 to saving throws vs conjuration spells.
}{Your conjuration spells receive +2 to their DCs, and you receive a +2 bonus on Spellcraft checks relating to conjuration spells.
}{As a swift action, you may cause the next conjuration spell you cast this turn to generate a cloud of foul smoke. The smoke is a 5ft radius emanation centered on either you, a target of the spell or the spell's effect. It acts as a fog cloud spell that fades away at the beginning of your next turn. All creatures within the cloud are sickened. You and creatures you summon or call with any spell you cast are unaffected by the cloud's sickening or concealment effects when they would be negative.
}{As a swift action, the next conjuration spell you cast also teleports you, an adjacent creature or an adjacent object or the spell's target (if any) 5ft per spell level. An unwilling target gets a Will save at the spell's DC to resist.
}{As a swift action, you reduce the casting time of the next conjuration spell you cast this turn by one step (to a minimum of a swift action), from a 1-round casting to a full-round action, a full-round action to a standard action, and a standard action to a move action.}

\magicfeat{Moilian Necromancer [Magic]}{You have been taught by the ancient school of Moil in the Black Lore. And you wear a skeleton hand around your neck, because you're that cool.
}{You gain the ability to rebuke undead as a cleric of your character level.
}{You may expend a swift action to deal an additional 1d6 negative energy damage per character level to all targets of the next necromancy spell you cast this turn. This additional damage can be halved with a successful Fort save at the spell's DC.
}{You may alter any spell with the [Cold] descriptor to deal half of its damage as negative energy damage instead. You may spend a swift action to apply the [Cold] descriptor to any spell you cast that deals hit point damage.
}{You may spend an immediate action after killing a living creature with a spell to have it rise immediately as a skeleton or zombie of the base creature as if animate dead had been cast on its corpse. The maximum CR of the resulting undead cannot exceed your character level-2. This creature is automatically considered to be rebuked. If you wish to gain a creature, but doing so would put you over your rebuking limit, you may turn some of the undead you already command loose to make room. Sentient undead released in this manner act normally, while nonsentient undead are automatically hostile to all living creatures.
}{When you cast any spell that has the [Cold] descriptor, deals negative energy damage or is from the necromancy school, you may spend an immediate action to reroll any attack rolls, or force the rerolls of any saving throws, and choose one. Each roll can only be rerolled once in this manner.}

\magicfeat{Pernicious Pyromancy [Magic]}{You have been trained by the pernicious goblin pyromancers, and have enough burns to make lesser men cringe. You also have the urge to give other people burns just as bad as yours.
}{You are so used to being burnt, singed, torched, scalded and branded that you gain fire resistance equal to your character level.
}{You can alter any spell you cast that deals energy damage to deal fire damage instead as a swift action. That spell loses any energy descriptors it had, gains the [Fire] descriptor instead, and all its energy (acid, cold, electricity or sonic) damage becomes fire. It also deals an additional 1 point of damage per character level.
}{As a swift action, you may make the next spell you cast this turn ignore an amount of fire resistance and hardness equal to your character level. The spell gains the [Fire] descriptor and deals additional fire damage equal to your character level. Creatures immune to fire receive half damage from fire damage instead of none from this spell.
}{By spending a swift action, you may add additional dice of damage to any spell with the [Fire] descriptor, up to a maximum of twice its normal damage dice. This additional damage repeats the next round. However, for each dice that you add, the spell deals 1d6 damage to you. The damage dealt to you ignores any immunity or resistance.
}{By spending a swift action, you can alter a spell with the [Fire] descriptor to deal maximum damage. Any target dealt damage by such a spell is set on fire, dealing your character level in damage each round. The save DC to extinguish it is the spell's DC instead of the normal DC for extinguishing fire. Immersion in water does not automatically extinguish this fire, but grants a +8 bonus to the save instead.}

\magicfeat{Reckless Spellcasting [Magic]}{You cast spells which explode things. Living things.
}{You are a scary mother fucker. Against anyone that has seen you cast a spell, you receive a +4 competence bonus to Intimidate checks.
}{You can alter any spell you cast shaped like a burst, line, emanation or spread to have an additional 100 \% area. Spells without these components cannot be affected.
}{Empower Spell: By spending a swift action, you can alter any spell you cast to have its variable, numeric effects increased by 50\%.
}{By spending a swift action, you can alter any area (cone, cylinder, line or burst) spell which allows a Reflex save to push all affected creatures and objects opposite to the origin point of the spell (in a direction designated by you for each creature). For each 5ft a target would move in this manner, it sustains a further 1d6 damage. If this movement is interrupted by an object or creature, both the collider and the collidee receives the damage, with an additional 1d6 damage for both.
}{By spending a swift action, you can alter any spell you cast to have all of its variable, numeric components increased to the maximum possible, as well as applying the benefits of Empower Spell.}

\magicfeat{Tactical Mage [Magic]}{You think better than other casters. Your spells do too.
}{You never seem to be in the wrong place at the wrong time. As an immediate action, a number times per day equal to your character level, you may teleport 10ft. in any direction. You must have line of sight and line of effect to the spot.
}{Sculpt Spell: By spending a swift action, you can alter any area spell by changing its area to either a cylinder (20ft radius, 30ft high), a 40ft cone, four 10ft cubes, a ball (40ft radius spread), or a 120ft line.
}{You can alter a spell that affects at least one opponent to have no immediate effect. In the next 24 hours, if you are dealt damage, you can choose to trigger the spell as an immediate action, but you must include whoever or whatever dealt you damage as one of the spell's targets. You can only have one spell of this sort active on you at any one time, and any spells stored in this way are considered to be the same as a contingency spell (thus, not allowing other contingencies to be active on the same person).
}{By spending a swift action, you can alter a single-target spell with a range greater than touch to arc to a number of additional targets beyond the first, to a maximum of your character level. All of the secondary targets must be within 30ft of the primary target, and the caster level of the spell is halved against any secondary targets that it affects.
Additionally, Sculpt Spell no longer requires an action to activate.
}{By spending a full round action, you may cast 2 spells with normal casting times of standard action or faster or cast a spell with a 1 round casting time as a standard action.}

\magicfeat{Supreme Aegis [Magic]}{As a master abjurer, you protect people.
}{You receive a +2 bonus on saves vs abjuration spells.
}{Your abjuration spells receive +2 to their DCs, and you receive a +2 bonus on Spellcraft checks relating to abjuration spells.
}{When you cast an abjuration spell, you may grant yourself or one of the spell's targets energy resistance and DR /- equal to spell level x 5. This protection lasts for 1 round per spell level, or until it prevents any damage.
}{When you cast an abjuration spell, you may grant yourself or one of the spell's targets spell resistance equal to 15 + character level. This protection lasts for 1 round per spell level, or until the spell resistance is rolled, whether or not it prevented the spell.
}{When you cast an abjuration spell with a duration on yourself, you may share its effects with allies within 5ft/4 character levels while it lasts.}

\magicfeat{Wand Mastery [Magic]}{Wands don't kill people, you do.
}{You are considered armed when wielding a wand. Wands are considered Tiny melee weapons (or objects two size categories smaller than you, if you are not Medium sized) that deal 1d4 damage. Magical wands and staffs you wield gain an enhancement bonus on attack and damage rolls equal to your character level/3.
}{You may expend spell slots or prepared spells as a swift action to give a staff or wand temporary charges. For each spell level above the staff or wand's highest spell level, it gains a temporary charge which lasts for 1 hour per character level. You may only have one 'set' of temporary charges on a wand at the same time (thus, multiple separate activations of this ability only give the highest number of charges created this way).
}{When you activate a staff or wand, you can spend an additional charge. If you do, the wand's effect is considered to be of the same caster level as you (if that would be higher), and its DC is increased to 10+1/2 character level+your casting stat modifier. Additionally, if you make an AOO with a wand used as a weapon, you may trigger the spell the staff or wand contains instead of taking the AOO, expending a charge as normal, but provoking no AOOs.
}{As a swift action, you may expend charges from a staff or wand to boost your caster level. Add a caster level per spell level of the wand per charge expended to your next spell this turn, up to a maximum of half your character level.
}{You can wield a wand in each hand at no penalty. You may activate two wands or activate a single staff twice with a full round action.}

\end{multicols}

