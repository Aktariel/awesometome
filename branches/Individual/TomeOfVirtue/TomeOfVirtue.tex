\documentclass[10pt]{report}

\usepackage{appendix}
\usepackage{floatflt}
\usepackage{fancyhdr}
\usepackage{textcomp}
\usepackage[usenames]{color}
\usepackage{isoent}    %\sfrac
%\usepackage{palatino} %font
\usepackage{newcent}   %font
\usepackage{sectsty} %custom section headings

\newcommand{\normalsections}{
\sectionfont{\noindent\rule{\textwidth}{0.015in}\\\nohang}
\subsectionfont{\noindent\rule{\textwidth}{0.005in}\\\nohang}
}

\newcommand{\columnsections}{
\sectionfont{\vspace*{-20pt}\noindent\rule{3.5in}{0.015in}\\\nohang}
\subsectionfont{\vspace*{-20pt}\noindent\rule{3.5in}{0.005in}\\\nohang}
}


\sectionfont{\noindent\rule{\textwidth}{0.015in}\\\nohang}
\subsectionfont{\noindent\rule{3.5in}{0.005in}\\\nohang}

\usepackage[Bjarne]{fncychap} %Canned chapter headings


\usepackage{multicol}
\usepackage[bookmarks=true,colorlinks,linkcolor=cyan,breaklinks]{hyperref}



%%%%Margins%%%
\topmargin 0pt
\advance \topmargin by -\headheight
\advance \topmargin by -\headsep
\textheight 8.9in
\oddsidemargin -0.25in
\evensidemargin \oddsidemargin
\textwidth 7in
\oddsidemargin -0.25in
%\setlength {\parindent} {0pt}



%%%Formatting%%%
\newcommand{\ability}[2]{\smallskip \noindent \textbf{#1} #2}
\newcommand{\shortability}[2]{\noindent\textbf{#1} #2\\}
\newcommand{\bolded}[1]{\noindent\textbf{#1}}
\newcommand{\itemability}[2]{\item \textbf{#1} #2}
\newcommand{\featname}[1]{\vspace*{0.1cm plus 0.2cm minus 0.05cm}\noindent\textbf{#1}\\}
\newcommand{\featnamelist}[1]{\vspace*{0.1cm plus 0.2cm minus 0.05cm}\noindent\textbf{#1}}

\newcommand{\descfeat}[2]{\featname{#1}\emph{#2}\\}

\newcommand{\classname}[1]{\subsection{#1}}
\newcommand{\condition}[1]{\emph{#1}}
\newcommand{\quot}[1]{\emph{#1}\medskip}
\newcommand{\desc}[1]{#1 \medskip}
\newcommand{\example}[1]{\emph{#1}}
\newcommand{\magicitem}[1]{\emph{#1}}
\newcommand{\monster}[1]{\subsection{#1} \label{monster:#1}}
\newcommand{\monsterline}[2]{\textbf{#1:} #2\\}
\newcommand{\monstersizetype}[2]{\textbf{#1 #2}\\}
\newcommand{\spell}[1]{\emph{#1}}
\newcommand{\spelllist}[1]{\smallskip \noindent \underline{\textbf{#1}}}

% For the feats -- moved here from feats.tex because we added another 
% section(s) for feats. -Surgo
\newcommand{\minitabular}[1]{\begin{tabular}{p{0.25in}p{2.9in}} #1\\ \end{tabular}}
\newcommand{\babfeat}[7]{
	\noindent\minitabular{\multicolumn{2}{l}{\parbox{3in}{\textbf{#1}}}}
	\minitabular{\multicolumn{2}{l}{\parbox{3in}{\small #2}}}
	\minitabular{\raggedleft\textbf{\small \textbf{+0:}}& {\small #3}}
	\minitabular{\raggedleft\textbf{\small +1:} & {\small #4}}
	\minitabular{\raggedleft\textbf{\small +6:} & {\small #5}}
	\minitabular{\raggedleft\textbf{\small +11:} & {\small #6}}
	\minitabular{\raggedleft\textbf{\small +16:} &{\small #7}}
}
\newcommand{\skillfeat}[8]{
	\noindent\minitabular{\multicolumn{2}{l}{\parbox{3in}{\textbf{#1}}}}
	\minitabular{\multicolumn{2}{l}{\parbox{3in}{\small #2}}}
	\minitabular{\multicolumn{2}{l}{\parbox{3in}{\small\textbf{#3}}}}
	\minitabular{\raggedleft\textbf{\small \textbf{0:}} &{\small #4}}
	\minitabular{\raggedleft\textbf{\small 4:} &{\small #5}}
	\minitabular{\raggedleft\textbf{\small 9:} &{\small #6}}
	\minitabular{\raggedleft\textbf{\small 14:} &{\small #7}}
	\minitabular{\raggedleft\textbf{\small 19:} &{\small #8}}
}

\newcommand{\boxit}[1]{\frame{\parbox{\textwidth}{#1}}}

% A new box command -- the first argument is the box's header, the second the 
% contents of the box. This looks a lot better than \boxit. -Surgo
\newcommand{\abox}[2]{\vspace{5pt}
	\fbox{\begin{minipage}{0.8\linewidth}
	\setlength{\parindent}{0.15in}
	\textbf{#1}

	#2
\end{minipage}}
\vspace{5pt}}

\newcommand{\itemspace}{\setlength{\itemsep}{-1mm}\setlength{\topsep}{-1mm} }

\newcommand{\listone}{\begin{list}{$\bullet$}{\itemspace}}
\newcommand{\listprereq}{\begin{list}{\vspace*{-2pt}}{\itemspace}}
\newcommand{\listtwo}{\begin{list}{$\triangleright$}{\itemspace}}
\newcommand{\listthree}{\begin{list}{--}{\itemspace}}

% Bold the first argument in the listitem
\newcommand{\bolditem}[2]{\item \textbf{#1} #2}

\newcommand{\slashfrac}[2]{${}^{#1}$\hspace*{-1pt}\makebox[2pt]{$\diagup$}${}_{#2}$}
%\newcommand{\half}[0]{\slashfrac{1}{2}}
\newcommand{\half}[0]{\ensuremath{\sfrac{1}{2}} }
\newcommand{\third}[0]{\ensuremath{\sfrac{1}{3}} }
\newcommand{\fourth}[0]{\ensuremath{\sfrac{1}{4}} }
%\newcommand{\half}[0]{\ensuremath{\scriptscriptstyle 1/2}}
%\newcommand{\threefourths}[0]{\ensuremath{\sfrac{3}{4}}}


%\rule{\textwidth}{0.005in}

%\renewcommand{\sectionmark}[1]{\markright{\thesection\ \boldmath\emph{#1}\unboldmath}\\\rule{\textwidth}{0.005in}\\}

\setcounter{tocdepth}{1}

\begin{document}

\pagestyle{plain}
%Cover Page

\begin{center} \Huge

\textsc{Races of War}\end{center}



\vspace{2cm}
\begin{center}\large By Frank Trollman \& K\end{center}


\newpage

\vspace*{4in}

\noindent Please address all complaints and comments about balance to the authors at\\
{\color{blue} \href{http://tgdmb.com/viewforum.php?f=1}{http://tgdmb.com/viewforum.php?f=1}}

\vspace{0.2in}

%\noindent For a hypertext version of some of this information, especially classes, please look at\\
%{\color{blue} \href{http://www.d20ragon.com/frank/}{http://www.d20ragon.com/frank/}}

%\vspace{0.2in}

\noindent Amateur Typesetting by Joshua Middendorf, updated by Morgon ``Surgo'' Kanter, ``Aktariel'', and Stephen ``Quantumboost'' Smith.

\vspace{0.15in}

\noindent Please address all comments regarding the quality (or lack thereof) of the typesetting (that is, formatting of the pdf) to Joshua Middendorf (\href{mailto:middendorfproject@gmail.com}{middendorfproject@gmail.com}), Morgon Kanter (\href{mailto:morgon.kanter@gmail.com}{morgon.kanter@gmail.com}), Aktariel (\href{mailto:aktariel@gmail.com}{aktariel@gmail.com}), or simply comment in the above forum.


%\vspace{0.2in}

%\emph{Enjoy!}


\vspace{1in}
\noindent Published on \today, version 0.6\\
\noindent You may find the most recent version of this document at:\\
{\color{blue} \href{http://www.tgdmb.com/viewtopic.php?t=36046}{http://www.tgdmb.com/viewtopic.php?t=36046}}

\newpage


\pagestyle{fancy}
%Fix the spacing so it's all reasonable
\linespread{.9}  \small  \normalsize \itemspace \normalsections

\tableofcontents

\chapter{Alignment Discussion: Good, Evil. Fanaticism.}

\chapter{Compromise, Coexistence, and Conversion}

\chapter{Classes}

\classname{Paladin}
\vspace*{-8pt}
\quot{``Good for the Good God!''}

\desc{A Paladin fights in melee and casts protective spells, enhancing their own defenses and those of others, and diverting attacks attacks onto themselves despite their fundamentally being a defensive character.  They also have a reasonable supply of information-gathering divination spells.}

\desc{As a Paladin, despite being a melee combatant, you can neglect your physical abilities some, and focus on Wisdom and Charisma, which power your abilities.  Your class features allow you to also rely on Wisdom and Charisma for physical combat to an extent, although Constitution and Strength are useful if you don't neglect them.}

\desc{Depending on the moral system of the setting, Paladins may or may not have a deific preference.  Under flag-based morality, Paladins often have patrons from their sponsoring plane.  On the other hand, if the planes of Good are Good because of what they do, rather than what they are, then Paladins are often not dedicated followers of any specific deity; their dedication to overall Good often interferes with their ability to maintain any other allegience, even a deific one.}

\ability{Alignment:}{Paladins must be of Good alignment.  Under Word is Bond ethics, Paladins must also be Lawful.  Under flag-based alignments, Paladins must be aligned with a plane that sponsors Paladins (Mount Celestia under the core cosmology).}

\ability{Hit Die:}{d8}

\ability{Base Attack Bonus:}{Good}

\ability{Good Saves:}{Fort, Reflex}

\ability{Skills:}{Climb (Str), Concentration (Con), Craft (Int), Diplomacy (Cha), Handle Animal (Cha), Heal (Wis), Knowledge (Nobility, Religion) (Int), Profession, Ride (Dex), Sense Motive (Wis), Spellcraft (Int), and Swim (Str).}

\ability{Skill Points/Level:}{2 + Int Mod}

\begin{table}[htb]
\begin{small}
\begin{tabular}[h]{lp{1.9cm}p{0.7cm}p{0.7cm}p{0.7cm}p{9cm}}
&   Base Attack Bonus&  Fort Save&  Ref Save&   Will Save&  Special\\
1&  +0& +2& +2& +0&  Aura of Good, Healing Touch, Exorcism, Insightful Strike, Personal Warding, Turn Undead\\
2&  +1& +3& +3& +0&  Detect Evil, Divine Grace, Divine Heart, Holy Shield, Smite Evil\\
3&  +1& +3& +3& +1&  Evasion, Expanded Prayers, Interrupting Spell 1/day\\
4&  +2& +4& +4& +1&  Snap Judgement\\
5&  +2& +4& +4& +1&  Divine Resilience, Guided Ward, Reactive Healing\\
6&  +3& +5& +5& +2&  Iron Will, Special Mount\\
7&  +3& +5& +5& +2&  Divine Reach, Expanded Prayers\\
8&  +4& +6& +6& +2&  Exorcising Smite, Interrupting Spell 2/day, Mortal Smite\\
9&  +4& +6& +6& +3&  Divine Chain, Hero's Heart, Prayers of the Faithful\\
10& +5& +7& +7& +3&  Timelessness  \\
11& +5& +7& +7& +3&  Expanded Prayers\\
12& +6& +8& +8& +4&  Radiant Healing\\
13& +6& +8& +8& +4&  Interrupting Spell 3/day\\
14& +7& +9& +9& +4&  Double Smite\\
15& +7& +9& +9& +5&  Expanded Prayers\\
16& +8& +10& +10& +5&  \\
17& +8& +10& +10& +5&  Divine Reaction\\
18& +9& +11& +11& +6&  Interrupting Spell 4/day\\
19& +9& +11& +11& +6&  Expanded Prayers, Guardian Beyond Death\\
20& +10& +12& +12& +6&  Ascendancy\\
\end{tabular}
\end{small}
\end{table}

\smallskip\noindent All of the following are Class Features of the Paladin class.

\ability{Weapon and Armor Proficiency:}{Paladins are proficient with all simple and martial weapons, all armor, and non-tower shields.}

\ability{Spellcasting:}{Paladins use the Sorcerer spells/day table.  Bonus spells and save DCs are based on Wisdom.  They cast divine spells off the Paladin spell list, and automatically know every spell on their spell list.  The somatic components of Paladin spells can be done even if both hands are occupied with a weapon and shield (but not two weapons).}

\ability{Aura of Good:}{A Paladin registers under Detect Good as a Cleric of a Good Deity.}

\ability{Healing Touch (Su):}{A Paladin has a pool of Healing Touch points equal to her class level * her Charisma modifier, which refreshes when her spell slots do.  As a swift action, the Paladin may heal herself or an ally within reach of her touch by any number of hit points up to the current value of their Healing Touch pool, removing that many points from the pool.  If the Paladin recieves healing in excess of her current damage taken, she may replenish her Healing Touch pool, but to no more than its usual maximum.}

\ability{Exorcism (Su):}{As an attack no more than once per round, a Paladin may deliver a melee touch attack to any undead creature, [evil] outsider, or any creature under the influence of a possessing spirit, such as a Ghost's Malevolence ability.  On a successful hit, deduct a number of points from the Paladin's Healing Touch pool equal to her class level, and inflict 1d6 damage per point taken.  This damage comes from pure divine power.  An exorcism on a possessed creature bypasses the possessed creature entirely and does damage directly to the possessor.  Exorcisms cannot make critical hits.  An exorcism, even one channeled through a weapon, does not do damage from any other source, such as weapon or unarmed strike damage.}

\ability{Insightful Strike:}{A 2nd-level Paladin gains Insightful Strike as a bonus feat.  If she already has Insightful Strike, she instead gains another [Combat] feat.}

\ability{Personal Warding:}{As long as she makes a melee attack on the same round, a Paladin may cast any Paladin spell with a casting time of one round or less on herself as a swift action.  She must attack before she can cast the spell.}

\ability{Turn Undead:}{As Cleric. A Paladin can never Rebuke Undead, and doing so is against their code of conduct should they somehow gain the ability from another class.}

\ability{Detect Evil (Sp):}{A Paladin of 2nd level or higher can cast \emph{Detect Evil} at-will as a spell-like ability.}

\ability{Divine Grace:}{A Paladin of 2nd level or higher adds her Charisma bonus to all saving throws.}

\ability{Divine Heart (Ex):}{A Paladin of at least 2nd level is immune to all Fear effects, including the conditions Shaken, Frightened, and Panicked, and all nonmagical diseases, including the Sickened condition, even when it has a magical source.}

\ability{Holy Shield (Ex):}{A Paladin of 2nd level or higher can substitute her Charisma modifier for her Dexterity modifier to armor class as long as she is using a shield in her hand (not an Animated Shield). She gains the full benefit of her shield against touch attacks. This bonus applies even when her Dexterity bonus would normally be denied, but she is still considered to have lost her dexterity bonus to AC for purposes of special abilities such as Sneak Attack under the normal circumstances.}

\ability{Smite Evil (Su):}{A Paladin of 2nd level or higher can attempt to smite a target as part of making a melee attack.  The attempt does not work (at no cost to her other than the attack) unless the target is evil.  If the target is evil, she adds her Charisma modifier to hit, and, if successful, she adds her Charisma modifier to hit and her level to damage.  She also gains supernatural strength, causing her to not take strength penalties to damage (bonuses still apply).

If, after the attack's damage is applied, the target has less than 4 hit points per class level of the Smiting paladin, it is instantly destroyed as though through hit point damage.  A single target may only be smote once per Paladin per day, and the Paladin cannot smite another target after successfully smiting one until either the first target is defeated or five rounds pass since the last Smite.  If the attack misses or is attempted on an invalid target, the Paladin may try again immediately against the same or a different target as soon as she can make another attack.}

\ability{Evasion (Ex):}{As a Rogue or Monk.  Gained at 3rd level.}

\ability{Expanded Prayers:}{At 3rd, 7th, 11th, 15th, and 19th levels, the Paladin adds one Abjuration, Divination, or Necromancy (Healing) spell of any level up to the highest level she can cast from any 9-level pre-Tome caster's spell list to her own.  The spell may not already be on her spell list at a different level.}

\ability{Interrupting Spell:}{A Paladin of 3rd level may, once per day, cast any Paladin spell with a casting time of one round or less as an immediate action.  This increases to twice per day at 8th level, three times at 13th level, and four times per day at 18th level.}

\ability{Snap Judgement (Su):}{As a swift action, a Paladin of fourth level or higher can focus on a target and gain information about them as if she had spent three rounds focusing on them with \emph{Detect Evil}.}

\ability{Divine Resilience (Su):}{A Paladin of 5th level or higher may expend a Turn Undead attempt as a free action on her turn to gain Energy Resistance equal to twice her class level to any two energy types, selected when she uses this ability.  This is shared with all allies within 20'.  This lasts for a number of rounds equal to her Charisma bonus + 1, minimum 1.  Allies that leave the area of this ability lose its effect, and allies that enter it gain the benefit.  If the Paladin activates this ability again, the first one automatically expires even if its duration had not ended.}

\ability{Guided Ward (Ex):}{A Paladin of 5th level or higher may focus on an enemy as a free action, impeding it from attacking her allies.  This ability, and its targeting, are obvious to any observer.  If the enemy attacks any of her allies before her next turn, she may cast any Paladin spell with a casting time of one round or less on any ally that creature attacked as a swift action.  The ally must still be within range and a valid target for the spell.}

\ability{Reactive Healing:}{A Paladin of 5th level or higher can use Healing Touch as an immediate action.}

\ability{Iron Will:}{At 6th level, a Paladin gains Iron Will as a bonus feat.  If she already has the Iron Will feat, she may pick any [Combat] feat.}

\ability{Special Mount:}{At 6th level, a Paladin gains a special mount.  This is a creature of either the same alignment as the Paladin or 4 or lower intelligence (including mindless) and non-evil alignment, and a CR three lower than the Paladin's, taking [Awesome] subtypes into account.  If she really wants to, she may instead use the SRD Paladin's Mount rules instead.  Regardless, the creature must be willing and able to serve as a mount (no Troll, Tendriculous, or Rat mounts, with an exception on the last for small values of Paladin and large values of Rat).  If necessary, the creature may be advanced to keep pace with the Paladin.}

\ability{Divine Reach:}{A Paladin of 7th level or higher may expend a Turn Undead attempt to change any of her Touch-range Paladin spells to Close range, as part of casting the spell. Such a modified spell can only affect willing targets.  She may also extend a Personal spell to Close range, but such a spell may only be cast on targets allowed by Guided Ward.}

\ability{Exorcising Smite (Su):}{A Paladin of 8th level or higher may make an Exorcising Smite.  As part of making the smite, she channels Exorcism through her weapon using the same attack roll; the Exorcism only lands on a successful hit, not just a touch hit unless the whole attack is a touch attack.  The Exorcism takes effect before weapon damage (including the Smite additions) are applied.  She may make no more than one attack in a round in which she makes an Exorcising Smite.  All costs of the Exorcism and the Smite are paid normally.}

\ability{Divine Chain:}{A Paladin of 9th level or higher may apply the Chain Spell metamagic feat to any of her Paladin spells when casting it without increasing the spell level, causing it to affect a number of additional targets equal to her caster level for half damage, or -4 to the save DC of nondamaging spells.  All secondary targets must be valid.  Using this ability costs two uses of her Turn Undead ability.}

\ability{Hero's Heart (Ex):}{A Hero's heart cannot be easily stopped by magic.  A Paladin of 9th level becomes immune to [Death] effects and petrification.}

\ability{Prayers of the Faithful (Su):}{A Paladin of 9th level or higher may pray for healing.  Praying is a full-round action that precludes taking a 5-foot step.  She is considered flat-footed while praying, but gains Fast Healing 10.  Up to one ally per level within 30' with line of sight to her may join her in prayer, taking the same action and gaining the same benefits.}

\ability{Radiant Healing (Su):}{A Paladin of 12th level or higher may use her Healing Touch ability at Close range instead of Touch.  By halving the healing done, she may have it affect all allies in range instead of just herself.}

\ability{Double Smite (Su):}{A Paladin of 14th level or higher gains a second Smite, which can be used just like the first and is recovered independently but in the same way.  So she can smite a target twice, or smite a second target before the first is neutralized, but not both at once.}

\ability{Divine Reaction (Su):}{A Paladin of 17th level or higher may expend a Turn Undead attempt as a free action to gain an extra swift action to use immediately.  She may do this when it is not her turn to gain an extra swift action on her next turn that must be spent immediately (i.e., an immediate action).  Each of her turns may only benefit from this ability once.}

\ability{Guardian Beyond Death (Su):}{A Paladin of 19th level is revived as if by a \emph{True Ressurection} spell where she fell at what would be the end of her next turn after she is slain.  She may use this ability once per day.  She also regains three uses of Turn Undead and her entire Healing Touch pool, as well as both Smites when this happens.  All of her equipment is teleported to its proper place when she returns.  If this ability is suppressed when she dies or when it would activate, it delays until it is not suppressed.}

\ability{Ascendancy (Ex):}{A Paladin of 20th level may be considered as her original type or a Native Outsider, whichever is most advantageous at the moment.  She gains Damage Reduction 10/evil, is no longer subject to aging penalties or death by old age, and may use \emph{Plane Shift} three times per day as a spell-like ability, but may only shift between a chosen Good-aligned plane (under flag morality, her sponsoring plane) or her home plane.  Once she chooses her upper plane, it cannot be changed except for through an alignment change.  Her powers also can no longer be withdrawn by her planar sponsors; she may change alignment freely.}

\ability{Code of Conduct:}{Something paladinish but not shit.  If she violates this code egregiously or habitually, she must recieve an Atonement spell from a character of her alignment and higher level before she can use her Supernatural and Spell-like class features and her spellcasting again.  She intuitively knows when an action pushes the boundaries of or violates the code.}

\spelllist{Paladin Spell List:}

\small\ability{0th level:}{\emph{Create Water, Cure Minor Wounds, Detect Magic, Detect Poison, Disrupt Undead, Guidance, Light, Mending, Read Magic, Resistance, Virtue}}

      \ability{1st level:}{\emph{Atonement, Bless, Bless Water, Bless Weapon, Cure Light Wounds, Deathwatch (no [Evil] descriptor), Delay Poison, Detect Chaos, Detect Good, Detect Law, Detect Undead, Greater Dispel Magic, Divine Favor, Endure Elements, Enlarge Person, Entropic Shield, Magic Vestment, Greater Magic Weapon, Nondetection, Protection From Chaos/Evil/Law (opposed alignments only), Remove Fear, Searing Light, Shield of Faith}}

      \ability{2nd level:}{\emph{Aid, Align Weapon, Bull's Strength, Consecrate, Cure Moderate Wounds, Desecrate (desanctifying use only, no [Evil] descriptor), Eagle's Splendor, Heroism, Owl's Wisdom, Protection from Arrows, Remove Paralysis, Resist Energy, Lesser Restoration, See Invisibility, Shield Other, Status, Zone of Truth}}

      \ability{3rd level:}{\emph{Arcane Sight, Continual Flame, Create Food and Water, Cure Serious Wounds, Daylight, Holy Smite, Magic Circle against Chaos/Evil/Law (Opposed alignments only), Neutralize Poison, Prayer, Protection from Energy, Remove Blindness/Deafness, Remove Curse, Remove Disease, True Strike}}

      \ability{4th level:}{\emph{Cure Critical Wounds, Death Ward, Detect Scrying, Dimensional Anchor, Discern Lies, Dismissal, Freedom of Movement, Haste, Lesser Planar Ally, Restoration, Spell Immunity}}

      \ability{5th level:}{\emph{Break Enchantment, Commune, Mass Cure Light Wounds, Dispel Chaos (Lawful only), Dispel Evil, Dispel Law (Chaotic only), Disrupting Weapon, Hallow, Heal Mount, Holy Sword, Interposing Hand, Mark of Justice (nonchaotic only), Raise Dead, Righteous Might, Spell Resistance, True Seeing}}

      \ability{6th level:}{\emph{Banishment, Mass Cure Moderate Wounds, Find the Path, Heal, Heroes' Feast, Greater Heroism, Planar Ally, Plane Shift, Quest, Undeath to Death}}

      \ability{7th level:}{\emph{Greater Arcane Sight, Mass Cure Serious Wounds, Holy Word, Limited Wish, Regenerate, Resurrection, Greater Restoration, Spell Turning}}

      \ability{8th level:}{\emph{Mass Cure Critical Wounds, Dimensional Lock, Holy Aura, Iron Body, Mind Blank, Moment of Prescience, Greater Planar Ally, Protection from Spells, Greater Spell Immunity}}

      \ability{9th level:}{\emph{Astral Projection, Foresight, Freedom, Gate, Mass Heal, Miracle, True Resurrection}}
\normalsize

\ability{Holy Avengers, Protectors, and Redeemers:}{Paladins use a special class of magic weapons known as Holy Avengers, Holy Protectors, and Holy Redeemers (collectively referred to as Paladin Weapons).  Such items are always magic melee weapons.  They always count as Good-aligned weapons.  They also gain additional powers depending on the class level of the Paladin, as shown on the table below.  A Paladin may only get the special benefits (beyond being a magic weapon) of one Paladin Weapon at a time, even if she fights with two weapons, or wears one and fights with the other.

All Types:
0: The weapon sheds light as a torch when drawn.
1: The Paladin continually benefits from Protection from Evil as long as the weapon is on her body, as a supernatural ability.  This may be suppressed or restored as a swift action while the weapon is stowed or sheathed; while it is in-hand, it is always active.
4: The Paladin continually generates a Magic Circle against Evil as long as the weapon is drawn.  The weapon sheds light as a Daylight spell when drawn.
8: The Paladin and all allies within 10' of her gain Spell Resistance 5 + the Paladin's character level as long as the weapon is drawn.  The weapon suppresses Darkness effects of lower spell level than the highest the Paladin can cast, and dispels those more than three levels lower.

Avenger:
Holy Avengers are carried by Paladins who serve as agents of vengeance.  Such Paladins have the grim duty of stopping those who have already done evil, pursuing justice for those wrongs which cannot be righted.  Paladins carrying Holy Avengers are the best known for their deeds, however, as the tale of the destruction of a powerful demonic temple will be told for far longer than that of the time your home wasn't burned by a dragon.
1: The Paladin may cast spells that target a weapon, such as Bless Weapon, on her Holy Avenger as a swift action as long as it is drawn.
3: When the Paladin successfully uses Smite Evil with her Holy Avenger, she ignores all material and alignment damage reduction, and does lethal damage to targets with Regeneration.  The Holy Avenger is also considered to be a Ghost Touch weapon when used in a Smite.
8: The Holy Avenger may be used to make a smiting attack separate from the Paladin's normal Smites.  She adds her Charisma to attack and her level to damage against an evil creature for one attack.  If the attack is successful, the creature must make a Fortitude save (DC 10 + 1/2 Paladin's level + Paladin's Wisdom modifier) or be stunned and dazed for 4 rounds.  If the attack misses or is made against a non-evil creature, it may be attempted again next round; if it hits, it may be attempted again either as soon as the first target leaves the fight (such as by retreat, surrender, or death) or after five rounds.
13: The Paladin may use Exorcism without spending points from her Healing Touch pool while weilding or wearing the Holy Avenger.  Exorcism damage dice are d8s.
14: The Paladin adds Destruction to her spell list as a 7th-level spell while wielding the Holy Avenger.
18: Any creature smote by the Paladin must make a Fortitude save or be destroyed as per the Destruction spell (normal weapon damage on a successful save).  Any creature killed by a Paladin's smite is destroyed as per the Destruction spell.  Undead are fully affected by this ability.

Protector:
Holy Protectors are used to defend others.  They are the signature weapon of Paladins of a more defensive bent, both those stopping rampaging dragons from burning villages and those stopping hordes of demons from shredding archmages.
1: The Paladin can add her level to her Healing Touch pool maximum, and gains an additional use of Turn Undead, essentially increasing her effective Charisma modifier for purposes of Healing Touch and Turn Undead by 1.  She gains these benefits as long as the Holy Protector is worn.
3: The Paladin may make an additional number of Attacks of Opportunity with her Holy Protector per round equal to her Charisma bonus.  She may also make attacks of opportunity as though it had Reach, although, if it does not, she does not gain an attack of opportunity against enemies closing with her.  When she hits with an attack of opportunity provoked by movement, the creature must make a Will save (DC 10 + Paladin's Charisma Modifier + 1/2 Paladin's level) or have its speed reduced to 0 until the beginning of its next turn.  Flying creatures instead must continue a straight-line course to maintain the minimum distance to avoid a fall, at a 45-degree downward angle.
8: The Holy Protector becomes a Defending weapon.  All allies of the Paladin within 30' gain the same AC bonus as she does from this or the Expertise attack option.
13: The Paladin gains Uncanny Dodge and Improved Uncanny Dodge.  She may also cast Shield Other on any creature within 60' as a free action spell-like ability usable as many times per round as she cares to use it, that lasts until the target goes out of range or it is dismissed as another free action (even when not her turn).  Finally, as long as her weapon is drawn and held, she gains Regeneration 3, with Evil-aligned weapons and spells doing full damage.
18: Three times per day, when she casts a spell with only willing targets, the Paladin may designate the spell to repeat again on the next round.

Redeemer:
Paladins who seek to show their enemies the benefits of switching to Team Good often carry Holy Redeemers for personal defense when higher-ups on Team Evil try to gank them for stealing their minions.  Occasionally a Paladin with this path makes a name for themselves with a retinue of redeemed devils or suchlike, but most work more quietly.
1: Attacks with a Holy Redeemer never take a penalty to inflict nonlethal damage.  Creatures destroyed by a Holy Redeemer's smite may instead merely be knocked unconcious or held for up to one hour per class level, at the Paladin's option.
4: If the target of her Snap Judgement ability knowingly lies before the Paladin's next turn, she knows (as the Discern Lies spell).
8: The Paladin may, as a swift action, allow any ally to repeat one Will save against a Charm, Compulsion, Possession, or [evil] affect, with her Charisma modifier as a bonus on the save.  Success causes the effect to end immediately.
13: The Paladin gains a bonus spell slot of her highest castable spell level.
18: Any creature smote with a Holy Redeemer may be forced to make a Will save (DC 10 + 1/2 Paladin level + her Charisma modifier), or be subject to an Imprisonment spell.  Imprisonment is added to the Paladin's spell list as long as she wears this weapon.}

\ability{Acquiring Paladin Weapons:}{Paladin weapons are intended to not be particularly imbalancing if a Paladin acquires them at any point in her career.  A Paladin that wants one should be allowed to acquire one at some point in the campaign, with the level range of the campaign determining which point.  If the campaign starts above level 12, Paladin characters should probably be allowed to start with Paladin weapons if they want one.  A Paladin that is supposed to find a Paladin Weapon may be given extraplanar help or advice in locating one (i.e., an angel shows up in one of her dreams and says that a Paladin with exactly the weapon they want died in the Dungeon of Dread some generations back, and his weapon is still in the dungeon).}

\ability{Distributing Paladin Weapons:}{Retaining more than one Paladin Weapon for longer than is necessary to give it to a deserving Paladin (including evaluation of deservingness), or deliberately destroying one, is a violation of the Paladin's Code; Paladin Weapons are in short supply, and hoarding them directly impedes the cause of good.}

\ability{Crafting Paladin Weapons:}{Paladin Weapons can only be crafted by Paladins.  They are considered Masterpieces at any level, have a level prerequisite of some kind, and each Paladin is only able to craft one ever.  There's probably some other cost too.  Thus, most Paladins find, rather than make, their weapons.}
%\classname{Cleric} \label{class:cleric}
\vspace{-8pt}
\quot{"Fear my righteous shining holy beacon of... righteousness?"}

\desc{Clerics are the holy (or unholy) warriors, standing fast against the darkness (or light). They are also made of cheese, and thus a prime target for minmaxxers.}
% somebody seriously needs to write some better content.
%   - Ak

\ability{Alignment:}{A cleric's alignment must be within one step of his deity's (that is, it may be one step away on either the lawful-chaotic axis or the good-evil axis, but not both). A cleric may not be neutral unless his deity's alignment is also neutral.}

\ability{Starting Gold:}{5d4x10	gp (125 Gold)}

\ability{Starting Age:}{As Cleric}

\ability{Hit Die:}{d8}

\ability{Class Skills:}{The cleric's class skills (and the key ability for each skill) are Concentration (Con), Craft (Int), Diplomacy (Cha), Heal (Wis), Knowledge (arcana) (Int), Knowledge (history) (Int), Knowledge (religion) (Int), Knowledge (the planes) (Int), Profession (Wis), and Spellcraft (Int).}

\ability{Domains and Class Skills:}{A cleric who chooses the Animal or Plant domain adds Knowledge (nature) (Int) to the cleric class skills listed above. A cleric who chooses the Knowledge domain adds all Knowledge (Int) skills to the list. A cleric who chooses the Travel domain adds Survival (Wis) to the list. A cleric who chooses the Trickery domain adds Bluff (Cha), Disguise (Cha), and Hide (Dex) to the list. See Deity, Domains, and Domain Spells, below, for more information.}

\ability{Skills/Level:}{2 + Intelligence Bonus}

\begin{table}[htb]
\begin{small}
\begin{tabular}{lp{2cm}p{0.7cm}p{0.7cm}p{0.7cm}l llllllllll}
Level  &Base Attack Bonus &Fort Save &Ref Save &Will Save &Special & \multicolumn{10}{c}{Spells Per Day}\\
       &                &    &    &    &                        &0 &1 &2 &3 &4 &5 &6 &7 &8 &9 \\
1st    &+0              &+2  &+0  &+2  & Turn or Rebuke Undead  &3 &1 &- &- &- &- &- &- &- &- \\
2nd    &+1              &+3  &+0  &+3  &                        &4 &2 &- &- &- &- &- &- &- &- \\
3rd    &+2              &+3  &+1  &+3  &                        &4 &2 &1 &- &- &- &- &- &- &- \\
4th    &+3              &+4  &+1  &+4  &                        &5 &3 &2 &- &- &- &- &- &- &- \\
5th    &+3              &+4  &+1  &+4  &                        &5 &3 &2 &1 &- &- &- &- &- &- \\
6th    &+4              &+5  &+2  &+5  &                        &5 &3 &3 &2 &- &- &- &- &- &- \\
7th    &+5              &+5  &+2  &+5  &                        &6 &4 &3 &2 &1 &- &- &- &- &- \\
8th    &+6/+1           &+6  &+2  &+6  &                        &6 &4 &3 &3 &2 &- &- &- &- &- \\
9th    &+6/+1           &+6  &+3  &+6  &                        &6 &4 &4 &3 &2 &1 &- &- &- &- \\
10th   &+7/+2           &+7  &+3  &+7  &                        &6 &4 &4 &3 &3 &2 &- &- &- &- \\
11th   &+8/+3           &+7  &+3  &+7  &                        &6 &5 &4 &4 &3 &2 &1 &- &- &- \\
12th   &+8/+3           &+8  &+4  &+8  &                        &6 &5 &4 &4 &3 &3 &2 &- &- &- \\
13th   &+9/+4           &+8  &+4  &+8  &                        &6 &5 &5 &4 &4 &3 &2 &1 &- &- \\
14th   &+10/+5          &+9  &+4  &+9  &                        &6 &5 &5 &4 &4 &3 &3 &2 &- &- \\
15th   &+11/+6/+6       &+9  &+5  &+9  &                        &6 &5 &5 &5 &4 &4 &3 &2 &1 &- \\
16th   &+12/+7/+7       &+10 &+5  &+10 &                        &6 &5 &5 &5 &4 &4 &3 &3 &2 &- \\
17th   &+12/+7/+7       &+10 &+5  &+10 &                        &6 &5 &5 &5 &5 &4 &4 &3 &2 &1 \\
18th   &+13/+8/+8       &+11 &+6  &+11 &                        &6 &5 &5 &5 &5 &4 &4 &3 &3 &2 \\
19th   &+14/+9/+9       &+11 &+6  &+11 &                        &6 &5 &5 &5 &5 &5 &4 &4 &3 &3 \\
20th   &+15/+10/+10     &+12 &+6  &+12 &                        &6 &5 &5 &5 &5 &5 &4 &4 &4 &4 \\
\end{tabular}
\end{small}
\end{table}

%Theres no way the table would fit with the '+1's in the spells per day section, so I took it out, the text is pretty clear that you get an extra spell slot for your domain.  Feel free to break it into two tables if you think its best.

\smallskip\noindent All of the following are class features of the cleric.

\ability{Weapon and Armor Proficiency:}{Clerics are proficient with all simple weapons, with all types of armor (light, medium, and heavy), and with shields (except tower shields).

\smallskip\noindent A cleric who chooses the War domain receives the Weapon Focus feat related to his deity's weapon as a bonus feat. He also receives the appropriate Martial Weapon Proficiency feat as a bonus feat, if the weapon falls into that category.}

\ability{Aura (Ex):}{A cleric of a chaotic, evil, good, or lawful deity has a particularly powerful aura corresponding to the deity's alignment (see the detect evil spell for details). Clerics who don't worship a specific deity but choose the Chaotic, Evil, Good, or Lawful domain have a similarly powerful aura of the corresponding alignment.}

\ability{Spells:}{A cleric casts divine spells, which are drawn from the cleric spell list. However, his alignment may restrict him from casting certain spells opposed to his moral or ethical beliefs; see Chaotic, Evil, Good, and Lawful Spells, below. A cleric must choose and prepare his spells in advance (see below). To prepare or cast a spell, a cleric must have a Wisdom score equal to at least 10 + the spell level. The Difficulty Class for a saving throw against a cleric's spell is 10 + the spell level + the cleric's Wisdom modifier. Like other spellcasters, a cleric can cast only a certain number of spells of each spell level per day. His base daily spell allotment is given on Table: The Cleric. In addition, he receives bonus spells per day if he has a high Wisdom score.

\smallskip\noindent A cleric also gets one domain spell of each spell level he can cast, starting at 1st level. When a cleric prepares a spell in a domain spell slot, it must come from one of his two domains (see Deities, Domains, and Domain Spells, below). Clerics meditate or pray for their spells. Each cleric must choose a time at which he must spend 1 hour each day in quiet contemplation or supplication to regain his daily allotment of spells.

\smallskip\noindent Time spent resting has no effect on whether a cleric can prepare spells. A cleric may prepare and cast any spell on the cleric spell list, provided that he can cast spells of that level, but he must choose which spells to prepare during his daily meditation.}

\ability{Deity, Domains, and Domain Spells:}{A cleric's deity influences his alignment, what magic he can perform, his values, and how others see him. A cleric chooses two domains from among those belonging to his deity. A cleric can select an alignment domain (Chaos, Evil, Good, or Law) only if his alignment matches that domain. If a cleric is not devoted to a particular deity, he still selects two domains to represent his spiritual inclinations and abilities. The restriction on alignment domains still applies. Each domain gives the cleric access to a domain spell at each spell level he can cast, from 1st on up, as well as a granted power. The cleric gets the granted powers of both the domains selected. With access to two domain spells at a given spell level, a cleric prepares one or the other each day in his domain spell slot. If a domain spell is not on the cleric spell list, a cleric can prepare it only in his domain spell slot.}

\ability{Spontaneous Casting:}{A good cleric (or a neutral cleric of a good deity) can channel stored spell energy into healing spells that the cleric did not prepare ahead of time. The cleric can "lose" any prepared spell that is not a domain spell in order to cast any cure spell of the same spell level or lower (a cure spell is any spell with "cure" in its name). An evil cleric (or a neutral cleric of an evil deity), can't convert prepared spells to cure spells but can convert them to inflict spells (an inflict spell is one with "inflict" in its name). A cleric who is neither good nor evil and whose deity is neither good nor evil can convert spells to either cure spells or inflict spells (player's choice). Once the player makes this choice, it cannot be reversed. This choice also determines whether the cleric turns or commands undead (see below).}

\ability{Chaotic, Evil, Good, and Lawful Spells:}{A cleric can't cast spells of an alignment opposed to his own or his deity's (if he has one). Spells associated with particular alignments are indicated by the chaos, evil, good, and law descriptors in their spell descriptions.}

\ability{Turn or Rebuke Undead (Su):}{Any cleric, regardless of alignment, has the power to affect undead creatures by channeling the power of his faith through his holy (or unholy) symbol (see Turn or Rebuke Undead). A good cleric (or a neutral cleric who worships a good deity) can turn or destroy undead creatures. An evil cleric (or a neutral cleric who worships an evil deity) instead rebukes or commands such creatures. A neutral cleric of a neutral deity must choose whether his turning ability functions as that of a good cleric or an evil cleric. Once this choice is made, it cannot be reversed. This decision also determines whether the cleric can cast spontaneous cure or inflict spells (see above). A cleric may attempt to turn undead a number of times per day equal to 3 + his Charisma modifier. A cleric with 5 or more ranks in Knowledge (religion) gets a +2 bonus on turning checks against undead.}

\ability{Bonus Languages:}{A cleric's bonus language options include Celestial, Abyssal, and Infernal (the languages of good, chaotic evil, and lawful evil outsiders, respectively). These choices are in addition to the bonus languages available to the character because of his race.}

\ability{Ex-Clerics:}{A cleric who grossly violates the code of conduct required by his god loses all spells and class features, except for armor and shield proficiencies and proficiency with simple weapons. He cannot thereafter gain levels as a cleric of that god until he atones (see the atonement spell description).}


\chapter{Bonus Core Classes}

\classname{Curator} \label{comm:class:curator}
\vspace*{-8pt}
\quot{``These people are under my stewardship. Please handle them with care.''}

%\desc{}

\ability{Alignment:}{Any Non-Evil}

%\ability{Races:}{}

\ability{Starting Gold:}{4d4x10 gp (100 gold)}

\ability{Starting Age:}{As Cleric.}

\ability{Hit Die:}{d8}

\ability{Class Skills:}{The Curator's class skills (and the key ability for each skill) are Concentration (Con), Craft (Int), Decipher Script (Int), Diplomacy (Cha), Heal (Wis), Listen (Wis), Knowledge (arcana) (Int), Knowledge (local) (Int), Knowledge (religion) (Int), Profession (Wis), Sense Motive (Wis), Spellcraft (Int), Spot (Wis), Survival (Wis), and Use Magic Device (Cha)}

\ability{Skills/Level:}{4 + Intelligence Bonus}


\begin{table}[htb]
\begin{small}
\begin{tabular}{lp{3cm}p{0.7cm}p{0.7cm}p{0.7cm}l}
Level &Base Attack Bonus &Fort Save &Ref Save &Will Save &Special\\
1st &+0 &+2 &+0 &+2 &Luminous Bolt, Benediction\\
2nd &+1 &+3 &+0 &+3 &Sphere\\
3rd &+2 &+3 &+1 &+3 &Benediction\\
4th &+3 &+4 &+1 &+4 &Heal Injuries\\
5th &+3 &+4 &+1 &+4 &Bendiction\\
6th &+4 &+5 &+2 &+5 &Sphere\\
7th &+5 &+5 &+2 &+5 &Benediction\\
8th &+6/+1 &+6 &+2 &+6 &One and All, Greater Luminous Bolt\\
9th &+6/+1 &+6 &+3 &+6 &Greater Benediction\\
10th &+7/+2 &+7 &+3 &+7 &Sphere\\
11th &+8/+3 &+7 &+3 &+7 &Greater Benediction\\
12th &+8/+3 &+8 &+4 &+8 &Perfect Preservation\\
13th &+9/+4 &+8 &+4 &+8 &Greater Benediction\\
14th &+10/+5 &+9 &+4 &+9 &Sphere\\
15th &+11/+6/+6 &+9 &+5 &+9 &Supreme Benediction\\
16th &+12/+7/+7 &+10 &+5 &+10 &Supreme Luminous Bolt\\
17th &+12/+7/+7 &+10 &+5 &+10 &Supreme Benediction\\
18th &+13/+8/+8 &+11 &+6 &+11 &Sphere\\
19th &+14/+9/+9 &+11 &+6 &+11 &Supreme Benediction\\
20th &+15/+10/+10 &+12 &+6 &+12 &Eternity\\
\end{tabular}
\end{small}
\end{table}

\smallskip\noindent All of the following are Class Features of the Curator class.

\ability{Weapon and Armor Proficiency:}{A Curator is proficient with simple weapons, with light and medium armor, and with shields (but not great shields).}

\ability{Luminous Bolt (Su):}{As an attack action, a Curator may fire a bolt of light at her foes. This has a range of Close (25 feet +5 ft./2 levels), does 1d6 light damage per two levels of Curator (rounded up), and requires a ranged touch attack to hit. Any creature damaged by this attack must make a Fortitude save (DC 10 + CHA + \half your level) or be dazzled for one round. Undead and creatures specifically vulnerable to light take double damage, and incorporeal undead targeted by this effect have no miss chance due to incorporeality.}

\ability{Benedictions (Su):}{At levels 1, 3, 5, and 7, a Curator learns a Benediction. Each Benediction requires an immediate action to activate, may target any creature the Curator can see within Close range (including herself), and is usable at will. The effects of a Benediction end at the start of the Curator's next turn, and the affected creature is surrounded by a faint glow while its effects persist.

Each Benediction has two effects, chosen from the following list:}

\begin{itemize}\itemspace\begin{small}
    \item{The target of your Benediction receives an amount of temporary HP equal to your ranks in the Heal skill.}
    \item{The target of your Benediction is cured of one of the following status effects: dazzled, shaken, sickened, or magical fatigue (your choice at time of use).}
    \item{The target of your Benediction receives a +2 morale bonus to AC.}
    \item{The target of your Benediction receives a +2 morale bonus to all saves.}
    \item{The target of your Benediction receives spell resistance equal to 5 + your character level.}
    \item{The target of your Benediction gains a +3 insight bonus to resist enemy bullrush, disarm, grapple, and trip attempts.}
    \item{The target of your Benediction gains a +5 insight bonus to Balance, Escape Artist, and Tumble checks.}
    \item{The target of your Benediction gains a +5 insight bonus to Climb, Jump, and Swim checks.}
    \item{The target of your Benediction gains a +5 insight bonus to Listen and Spot checks, and to resist enemy feints.}
    \item{The target of your Benediction gains blindsense out to 15'.}
    \item{The target of your Benediction gains a +10' insight bonus to move speed.}
    \item{Anyone attacking the target of your Benediction in melee must make a Fortitude save (DC 10 + CHA + \half your level) or be sickened for one round. This occurs before the attack is resolved.}
    \item{Anyone attacking the target of your Benediction in melee must make a Will save (DC 10 + CHA + \half your level) or be shaken for one round. This occurs before the attack is resolved.}
\end{small}\end{itemize}

\ability{Spheres:}{At 2nd level, and every 4 levels thereafter, the Curator gains basic access to a Sphere, which must be one of: Exorcism, Mystery, Piety, Restraint, Revelation, Splendor, or Vigor (see below). If she selects a Sphere in which she already has basic access, she gains advanced access; if she already has advanced accesss, she gains expert access.}

\ability{Heal Injuries (Sp):}{With a one minute ritual, a Curator of 4th level can heal a number of characters equal to her class level of a number of hit points of damage equal to her ranks in the Heal skill. This action may be taken a number of times per day equal to 3 + the Curator�s Charisma Modifier (if any), and all affected characters must be within close range of the Curator for the entire period.}

\ability{One and All (Su):}{At 8th level, a Curator takes on wider responsibilities. Once per day, when she activates one of her Benedictions, she may apply its effects to herself and all allies within range, rather than just one target.}

\ability{Greater Luminous Bolt (Su):}{At 8th level, any creature that fails the Fortitude save when struck by her Luminous Bolt also becomes blinded for one round.}

\ability{Greater Benedictions (Su):}{At levels 9, 11, and 13, a Curator learns a Greater Benediction. Each Greater Benediction requires an immediate action to activate, may target any creature the Curator can see within Close range (including herself), and is usable at will. The effects of a Greater Benediction end at the start of the Curator's next turn, and the affected creature is surrounded by a warm glow while its effects persist.

When a Curator gains a Greater Benediction, she may replace one of her Benedictions with a different Benediction.

Each Greater Benediction has two effects, chosen from the following list:}

\begin{itemize}\itemspace\begin{small}
    \item{The target of your Greater Benediction receives DR 8/-.}
    \item{The target of your Greater Benediction gains energy resistance 15 against one energy type you select at the time of use.}
    \item{The target of your Greater Benediction cannot be affected by poison, disease, fear or charm effects (this does not suppress or remove effects already present).}
    \item{The target of your Greater Benediction gains the benefits of a protection from alignment effect, using an alignment you choose at the time of use.}
    \item{The target of your Greater Benediction benefits from Evasion and Improved Uncanny Dodge.}
    \item{The target of your Greater Benediction benefits from blindsight out to 30'.}
    \item{The target of your Greater Benediction does not provoke the normal attacks of opportunity for movement, ranged attacks, spells, spell-like abilities, or using items.}
    \item{The target of your Greater Benediction benefits from a freedom of movement effect.}
    \item{Anyone attacking the target of your Greater Benediction in melee must make a Reflex save or be struck by a luminous pulse for 5d6 light damage (undead and light-vulnerable creatures take double damage). Each attacker is subjected to this effect only once per round.}
    \item{Anyone attacking the target of your Greater Benediction must make a Will save (DC 10 + CHA + \half your level) or fail, losing the action but not any resources (spell slots, components, XP, and the like) used to power it.}
    \item{Your Greater Benediction affects one additional target within range.}
    \item{Your Greater Benediction bestows any two effects from the Benediction Effects list.}
\end{small} \end{itemize}

\ability{Perfect Preservation(Ex):}{Once she reaches 12th level, a Curator is immortal. She may also care for a number of willing creatures equal to her charisma modifier, and prevent them from aging for as long as they receive care. Additionally, neither the Curator nor any creature under her care will lose a level if they die and are resurrected.}

\ability{Supreme Benedictions (Su):}{At levels 15, 17, and 19, a Curator learns a Supreme Benediction. Each Supreme Benediction requires an immediate action to activate, may target any creature the Curator can see within Close range (including herself), and is usable at will. The effects of a Supreme Benediction end at the start of the Curator's next turn, and the affected creature is surrounded by a strong glow while its effects persist.

When a Curator gains a Supreme Benediction, she may replace one of her Benedictions or Greater Benedictions with a different Benediction of the same type.

Each Supreme Benediction has two effects, chosen from the following list:}

\begin{itemize}\itemspace\begin{small}
    \item{The target of your Supreme Benediction is immune to fire, cold, electricity, acid, sonic, negative energy, and death effects.}
    \item{The target of your Supreme Benediction is immune to fatigue, exhaustion, fear affects, mind-affecting effects, dazing, stunning, sickening, and nausea, and all such effects are immediately removed from him.}
    \item{The target of your Supreme Benediction has total concealment.}
    \item{The target of your Supreme Benediction benefits from true seeing and blindsight to 120', and his attacks never suffer a miss chance due to concealment.}
    \item{The target of your Supreme Benediction may choose to teleport himself and everything he is physically carrying up to 100' in any direction as a free action once per round.}
    \item{Any creature damaging the target of your Supreme Benediction also damages itself for half that amount (this includes ability damage).}
    \item{Any creature forcing the target of your Supreme Benediction to make a saving throw must itself make a Fortitude save (DC 10 + CHA + \half your level) or be knocked unconscious for two rounds.}
    \item{Any hostile creature that meets the gaze of the target of your Supreme Benediction must make a Reflex save (DC 10 + CHA + \half your level) or be cowered until it loses sight of him.}
    \item{Any creature coming within 60' of the target of your Supreme Benediction with hostile intent must make a Will save (DC 10 + CHA + \half your level) or become helpless for one round. Each creature is only subjected to this effect once per round.}
    \item{The effects of your Supreme Benediction persist for one extra round.}
    \item{Your Supreme Benediction bestows any two effects from the Greater Benediction Effects list.}
\end{small}\end{itemize}

\ability{Supreme Luminous Bolt (Su):}{At 16th level, any creature that fails the Fortitude save when struck by her Luminous Bolt also becomes stunned for one round.}

\ability{Eternity(Ex):}{Upon reaching 20th level, a Curator selects one Benediction or Greater Benediction that she knows. Thereafter, she enjoys its effects unceasingly and at all times.}
\classname{Marshall} \label{class:marshall}
\vspace*{-8pt}
\quot{``The Jarls of Niefelheim think to banish our summers to a land of darkness. If they succeed there will be no second chances, no future generations. We will stop them here. For if our children's children are to even exist, they must remember our bravery on this day!''}

\desc{The Marshall is a leader of men. Whether a member of a squad of Halfling special forces or the head of a rampaging horde of Bugbears, the Marshall's true home is the battlefield.}

\ability{Alignment:}{Leaders arise on every aspect of the wheel. From the hordes of the Orcs to the formations of the Dwarves, there is almost always a charismatic leader at the head of any serious army. Serious armies are often employed by the forces of Law, as the forces of Chaos are often wont to be embodied by individuals pillaging without reference one to another. But the forces of Chaos have their share of tyrants as well.}

\ability{Races:}{Almost all Marshalls are humanoids. There are few races of Giants or Magical Beasts that form warbands sufficient to warrant military leadership.}

\ability{Attributes:}{Charisma is of prime importance to a Marshall. Other than that, a Marshall spends a lot of time yelling in plain view of enemies, so Constitution is of perhaps greater importance for them than it is to many other classes.}

\ability{Starting Gold:}{6d6x10 gp (210 gold)}

\ability{Starting Age:}{As Paladin.}

\ability{Hit Die:}{d12}

\ability{Class Skills:}{The Marshall's class skills (and the key ability for each skill) are Bluff (Cha), Climb (Str), Craft (Int), Concentration (Con), Diplomacy (Cha), Disguise (Cha), Gather Information (Cha), Handle Animal (Cha), Heal (Wis), Hide (Dex), Intimidate (Cha), Jump (Str), Knowledge [Any] (Int), Listen (Wis), Move Silently (Dex), Profession (-), Ride (Dex), Search (Int), Sense Motive (Wis), Spellcraft (Int), Spot (Wis), Survival (Wis), Swim (Str), and Use Rope (Dex).}

\ability{Skills/Level:}{4 + Intelligence Bonus}

\begin{table}[htb]
\begin{small}
\begin{tabular}{lp{3cm}p{0.7cm}p{0.7cm}p{0.7cm}l}
Level &Base Attack Bonus &Fort Save &Ref Save &Will Save &Special\\
1 &+1 &+2 &+2 &+2 &Battle Shout, \spell{Heal Injuries}, Inspire Bravery\\
2 &+2 &+3 &+3 &+3 &Untiring, Magic Circle\\
3 &+3 &+3 &+3 &+3 &Dispelling Glare, War Shout\\
4 &+4 &+4 &+4 &+4 &Bolster Allies, Inspire Heroism\\
5 &+5 &+4 &+4 &+4 &\spell{Aura of Healing}, Project Voice\\
6 &+6/+1 &+5 &+5 &+5 &Leadership, \spell{Restoration}\\
7 &+7/+2 &+5 &+5 &+5 &\spell{Heal}, Terrible Shout\\
8 &+8/+3 &+6 &+6 &+6 &Inspire Foolishness, Delaying Shout\\
9 &+9/+4 &+6 &+6 &+6 &Inspire the Masses, \spell{Revive the Dead}\\
10 &+10/+5 &+7 &+7 &+7 &\spell{Mass Heal}, Command Enemies\\
11 &+11/+6/+6 &+7 &+7 &+7 &Massive Presence, Inspire Greater Heroism\\
12 &+12/+7/+7 &+8 &+8 &+8 &Convert Opponents
\end{tabular}
\end{small}
\end{table}

\smallskip\noindent All of the following are Class Features of the Marshall class.

\ability{Weapon and Armor Proficiency:}{Marshalls are proficient with all simple and Marshall weapons, as well as any Exotic Weapons appropriate to their race, religion, or culture. Marshalls are proficient with light, Medium, and Heavy armor as well as Shields and Great Shields.}

\ability{Battle Shout (Ex):}{With a Swift Action, a Marshall may scream super loud and work up his allies for battle. All allies within short range gain a Morale Bonus to attack and damage rolls for 10 rounds. This bonus is half the Marshall's class level (rounded up) or the Marshall's Charisma modifier, whichever is less. Allies must be able to see and hear the Marshall for this bonus to take effect.}

\ability{Heal Injuries (Sp):}{The blood of an army is the life of its men, and soldiers who cannot fight are less than worthless. With a one minute ritual, a Marshall can heal a number of characters equal to his class level of a number of hit points of damage equal to his ranks in the Heal skill. This action may be taken a number of times per day equal to 3 + the Marshall's Charisma Modifier (if any), and all affected characters must be within close range of the Marshall or the entire period.}

\ability{Inspire Bravery (Ex):}{A Marshall's mere presence is a boon to the morale of his compatriots. All allies within short range of a Marshall may add the Marshall's Charisma modifier to their saves against [Fear] effects.}

\ability{Untiring (Ex):}{A 2nd level Marshall need not ever sleep and is immune to any effects which cause \condition{fatigue}, \condition{exhaustion}, or \condition{sleep}.}

\ability{Magic Circle (Su):}{A 2nd level Marshall radiates a \spell{magic circle} against an alignment of his choice at all times. He can end, resume, or change the alignment of his circle as a Swift action.}

\ability{Dispelling Glare (Su):}{A 3rd level Marshall can attempt to destroy a magical effect by staring at it really hard. As a Swift Action, the Marshall may attempt to dispel a single magical effect or suppress a single magical item within medium range and line of sight. The dispelling effect and DC is as per \spell{dispel magic}, and the Marshall's check is a d20 + Marshall Level.}

\ability{War Shout (Ex):}{At 3rd level, a Marshall's Battle Shout effects all allies within Long Range. The allies must still be able to hear and see the Marshall as normal.}

\ability{Bolster Allies (Ex):}{With a Swift action, a 4th level Marshall may assist all allies within short range of himself. These allies may each reroll one die roll made before the Marshall's next turn. They must be able to see and hear the Marshall when the reroll is called for.}

\ability{Inspire Heroism (Ex):}{The allies of a 4th level Marshall are capable of heroic, even improbable feats so long as they can see and hear the Marshall. All allies within Short Range of the Marshall gain a +1 Luck bonus on attack rolls, skill and ability checks, and saves.}

\ability{Aura of Healing (Su):}{A 5th level Marshall can activate an Aura of Healing as a Swift Action. This aura lasts for 10 rounds and may be activated a number of times per day equal to the Marshall's Charisma modifier. All allies within short range heal a number of hit points equal to the Marshall's ranks in Heal during the Marshall's turn while the aura is in effect.}

\ability{Project Voice (Ex):}{The voice of a 5th level Marshall carries super far. In anything approaching normal conditions, the Marshall can project his voice out for a mile. The Marshall can automatically dispel a \spell{silence} or similar effect as a Swift action.}

\ability{Leadership:}{At 6th level, a Marshall gains a Leadership feat that he qualifies for as an additional feat.}

\ability{Restoration (Sp):}{A 6th level Marshall can cast \spell{restoration} as a spell-like abilty at will.}

\ability{Heal (Sp):}{A 7th level Marshall may use \spell{heal} as a spell-like ability with a caster level equal to his Character Level. This may be used a number of times per day equal to his Charisma modifier.}

\ability{Terrible Shout (Ex):}{A 7th level Marshall can spook all enemies within Medium range with a terrible war howl. Enemies who can see and hear the Marshall within range must make a Will save (DC 10 + \half Level + Charisma Modifier) or become \condition{frightened} for 10 rounds. This is a [Fear] effect and only affects creatures with an Intelligence score. Creatures who succeed in their save are unaffected for the next 24 hours.}

\ability{Inspire Foolishness (Ex):}{People do dumb stuff for an 8th level Marshall. All allies within Short Range of a Marshall are immune to Fear so long as they can see and hear the Marshall.}

\ability{Delaying Glare (Ex):}{An 8th level Marshall can trip up and delay enemies with a steady Glare and an intimidating stance. With a Swift Action, the Marshall may force all enemies within Medium Range to make a Will save (DC 10 + \half Level + Charisma Modifier) or become \spell{slowed} for 10 rounds. Enemies must be able to see the Marshall, but they need not hear him.}

\ability{Inspire the Masses (Ex):}{The Inspiration abiilties of a 9th level Marshall extend to Long range, provided that the allies can still see and hear the Marshall.}

\ability{Revive the Dead (Sp):}{A 9th level Marshall can refill the ranks of his army by returning the fallen to life. With a 10 minute ritual, the Marshall can return a dead creature back to life so long as it has died within a number of days equal to the Marshall's ranks in the Heal skill. The newly alive awaken with a number of hit points as if they had had zero and then rested for 1 day (usually 2 hit points per hit die). The Marshall may do this a number of times per day equal to his Charisma modifier. This abiilty does not affect a character's level, nor does it restore removed body parts.}

\ability{Mass Heal (Sp):}{A 10th level Marshall can cast \spell{mass heal} as a spell-like ability a number of times per day equal to his Charisma modifier.}

\ability{Command Enemies (Ex):}{With a standard action, a Marshall can issue commands to opponents as the spell \spell{greater command}. Unless enemies make a Will Save (DC 10 + \half Level + Charisma Modifier), these commands are obeyed. This is a [Mind Affecting] [Language Dependant] ability.}

\ability{Massive Presence (Su):}{An 11th level Marshall can grow one size category and emit light as a bonfire at will. In addition to making the Marshall stronger and more powerful as normal, the Marshall is of course much easier to see. This effect may be begun or ended as a Swift Action.}

\ability{Inspire Greater Heroism (Ex):}{At 11th level, the Luck bonus provided by Inspire Heroism increases to +2. It still affects all allies who can see and hear him within Long Range (because of the Inspire the Masses ability).}

\ability{Convert Opponents (Ex):}{A 12th level Marshall can make an opponent within Medium Range Friendly to himself and his cause with a Standard Action unless the creature makes a Will Save (DC 10 + \half Level + Charisma Modifier). This is a [Mind Affecting] [Compulsion] effect, but any changes in attitude caused by subsequent Diplomacy or actions are not. In all other respects, this functions as \spell{charm monster}. This ability may be used at will.}

\classname{Sohei} \label{class:comm:sohei}
\vspace{-8pt}
\quot{``Don't you dare hurt my friends.''}

\desc{The Sohei is a bit like the Monk, but with a little more influence (in theory, at least) on being virtuous and enlightened. They also tend to use weapons and armour, but it's by no means necessary.}

\desc{Note that this isn't just a description of the class I have written, it's basically what the Sohei seems to be. To be honest, they're not as well known as, say, the ninja (although at least there's less argument as to what they actually do).}

\desc{As a Sohei, a good Int is not a bad idea, but what you really want is a high Wisdom, and without ignoring your physical abilities. Although, I hear there are feats that could in fact allow you to ignore your physical scores. It has the potential to get around annoying things like walls of force at later levels, and gets a free ticket into the Wish Based Economy. It probably has cool tricks for the advanced player to exploit, but I don't really see anything that stops it being available to beginners.}

\ability{Alignment:}{Anything they could argue as being virtuous (see: any Good) }

\ability{Races:}{Any}

\ability{Starting Gold:}{2d4x10 gp (50 gold)}

\ability{Starting Age:}{As Monk.}

\ability{Hit Die:}{d10}

\ability{Class Skills:}{Make your own, since Koumei doesn't believe in cross-class skills.}

\ability{Skills/Level:}{4 + Intelligence Bonus}

\begin{table}[tbh]
\begin{small}
\begin{tabular}{lp{3cm}p{0.7cm}p{0.7cm}p{0.7cm}p{9cm}l}
Level  &Base Attack Bonus &Fort Save &Ref Save &Will Save &Special\\
1st &+1 &+2 &+2 &+2 &Improved Unarmed Strike, Divine Flurry\\
2nd &+2 &+3 &+3 &+3 &Unforeseen Strike, Virtuous Writings\\
3rd &+3 &+3 &+3 &+3 &Power Slide, Speed: +10'\\
4th &+4 &+4 &+4 &+4 &Counter Attack\\
5th &+5 &+4 &+4 &+4 &Enlightenment: Perfect Understanding\\
6th &+6/+1 &+5 &+5 &+5 &Super Gauge, Divine Go Stone Placement of Perfection\\
7th &+7/+2 &+5 &+5 &+5 &Divine Frenzy\\
8th &+8/+3 &+6 &+6 &+6 &Speed: +15', C-C-C-COMBO BREAKER\\
9th &+9/+4 &+6 &+6 &+6 &Divine Perfection\\
10th &+10/+5 &+7 &+7 &+7 &Enlightenment: Eschew Worldly Possessions, Divine Dodge\\
11th &+11/+6/+6 &+7 &+7 &+7 &One Hand Clapping\\
12th &+12/+7/+7 &+8 &+8 &+8 &Shadow Run\\
13th &+13/+8/+8 &+8 &+8 &+8 &Weight of Thought, Speed: +20'\\
14th &+14/+9/+9 &+9 &+9 &+9 &Divine Meditation\\
15th &+15/+10/+10 &+9 &+9 &+9 &Enlightenment: We Are All Connected\\
16th &+16/+11/+11/+11 &+10 &+10 &+10 &Poetry in Motion\\
17th &+17/+12/+12/+12 &+10 &+10 &+10 &Zen Fist\\
18th &+18/+13/+13/+13 &+11 &+11 &+11 &Speed: +25', Formation of the Eight Trigrams\\
19th &+19/+14/+14/+14 &+11 &+11 &+11 &Deadly Finishing Move\\
20th &+20/+15/+15/+15 &+12 &+12 &+12 &Enlightenment: Transcendence, Secret Scroll \\
\end{tabular}
\end{small}
\end{table}

\smallskip\noindent All of the following are Class Features of the Sohei class.

\ability{Weapon and Armor Proficiency:}{Sohei are proficient with all Simple Weapons, All Polearms, Light Armor, Medium Armor, and Heavy Armor.}

\ability{Improved Unarmed Strike:}{The Sohei gains a natural Slam attack that deals 1d8 damage if Medium sized. This gains an enhancement bonus equal to one quarter of their level (round down), and can be used to make full attack actions if no other weapons are used.}

\ability{Divine Flurry:}{When making a Full Attack action with any weapon or combination thereof, the Sohei may make one additional attack, and all attacks are made at their highest attack bonus, however all of them take a crippling -2 penalty to hit.}

\ability{Unforeseen Strike:}{Those struck by the Attacks of Opportunity of the Sohei must make a Fortitude save (DC 10 + half the Sohei's HD + their Wisdom modifier) or be Stunned for 1 round.}

\ability{Virtuous Writings:}{The Sohei may cast spells from Divine scrolls as though they had the spells on their class list. Yes, even if a jerk of a Warlock takes a spell that isn't Divine (are there any that aren't on some obscure domain?) and makes a Divine scroll of it. Additionally, they may create scrolls of spells on the Cleric*, Good, Nobility, Oracle or Purification spell lists, as though a Cleric (with the listed domains) of a level equal to their Sohei level. Casting the spells are not required for this process.\smallskip
*Does not include [Evil] spells.}

\ability{Power Slide:}{If a 3rd level Sohei takes damage from an attack, she may allow herself to be flung backwards, thereby lessening the impact. She may make a Balance check with a DC equal to the damage inflicted and if she succeeds, she suffers only half damage. This is a skill check, not a Saving Throw, so abilities such as Evasion do not apply. She is moved away from the source of damage by 5' for every 5 points of damage (or part there of) negated in this way. If there is not enough space for her to move, she suffers a d6 of damage for each square not moved. If she passes through an occupied square, the Sohei would have to make a tumble check to avoid attacks of opportunity. \smallskip

If this ability is gained from another class, then the Sohei may choose to increase or decrease the total distance moved by 50\% (so a Power Slide that negated 12 points of damage can cause him to move 5', 10', or 15' at her choice).}

\ability{Speed:}{The Sohei gains an increasing enhancement bonus to Speed. This is reduced by 5 feet when wearing Medium Armour, or 10' when wearing Heavy Armor.}

\ability{Counter Attack:}{Enemies provoke attacks of opportunity from the Sohei whenever they make melee attacks against the Sohei's allies.}

\ability{Enlightenment: Perfect Understanding:}{The Sohei reaches a plateu of enlightenment and understanding. At this stage, they can never be Confused, and always know when a person is lying. Additionally, they can choose to automagically disbelieve illusions and can see the invisible.}

\ability{Super Gauge:}{Whenever the Sohei successfully strikes a foe or is struck by a foe, they gain a point of Tension against that foe. This lasts for a number of rounds equal to their Wisdom modifier, until they gain Tension against anyone else, until the foe they have Tension against is defeated or until used (whichever comes first).
\smallskip
5 Tension points may be spent to gain an additional melee attack as a Free Action or to Empower a spell being cast from a scroll. 10 points may be used to unleash a Super Move, a Full Attack where each hit deals an additional amount of damage equal to 5d6 plus the Sohei's BAB. It should be compulsory that the player shout out the name of the attack.}

\ability{Divine Go Stone Placement of Perfection:}{Whenever combat begins, before Initiative is actually rolled, the Sohei may freeze time for one round. In this time, the Sohei and their allies may each make a Move Action to get into superior positions. Initiative is then rolled, and combat resumes. They may also use this ability once per minute during combat as an Immediate Action.}

\ability{Divine Frenzy:}{Whenever someone deals damage to the Sohei, they may enter a Divine Frenzy. This lasts for as long as they have Tension, and during the effect they may act completely normally and use items normally - they can even cast spells if they have spellcasting ability from another class. The Divine Flurry gains another bonus attack, with every strike gaining +2d6 damage of any Energy Type, and the Sohei's Attacks of Opportunity may, at the Sohei's option, knock the targets away 5' per 5 damage dealt, in a straight line in the direction of their choice. This movement provokes Attacks of Opportunity, but not from the Sohei.}

\ability{C-C-C-COMBO BREAKER:}{Each foe is unable to successfully make more than one attack against each of the Sohei's allies during their turn, as long as the Sohei is not flat-footed. This includes spells and similar effects, as long as they are not Harmless, but does not include attacks of opportunity, as they do not take place during the foe's turn (unless they do, in which case they do count). For multiple-target or area effects, only the targets they already attacked are immune.\smallskip

Linked effects such as Improved Grab, the Trip ability of wolves and Poison will take effect as normal. Even taking continuing damage from an ongoing effect is fine. The enemies are free to take turns beating on the same guy, and one enemy splitting attacks up amongst the team is still fine, but one foe focusing their attacks on one target triggers this ability. For example, a Disintegrate followed by a Quickened Fireball will mean one guy gets hit for Disintegration, and everyone else takes the fire damage. \smallskip

The ally in question must be in a square the Sohei threatens, otherwise they are exempt from this effect.}

\ability{Divine Perfection:}{The Sohei becomes immune to Death Effects, Disease and Poison, and gains Spell Resistance equal to their Hit Dice + 5}

\ability{Enlightenment: Eschew Worldly Possessions:}{At this stage, the Sohei can create items worth up to 15,000 GP as per a Wish, at will, requiring only a minute of meditation. They also need not eat or drink. To use this ability, the Sohei must not own any currency or precious metals. Personal equipment is exempted from the restriction. This is a flavor thing, so use common sense.}

\ability{Divine Dodge:}{Once per minute, as an Immediate Action, the Sohei may suddenly move 10' away without provoking Attacks of Opportunity. If this brings the Sohei out of range for an attack that was about to be resolved, the attack automatically misses, likewise if it brings the Sohei out of the area of effect for a spell that was about to go off. \smallskip

It may be used again before the minute is up, at the cost of 10 Tension. However, this must be against an attack performed by the one who earned the Tension.}

\ability{One Hand Clapping:}{The Sohei cannot be Dazed, Deafened or Stunned. Additionally, their unarmed strikes deal +2d6 Sonic damage when they strike, and by spending 10 Tension as a Swift action, they may cause each blow to unleash a Sonic Pulse: all enemies within 10' take 1d6+BAB Sonic damage (no save), and the foes actually struck must pass a Fortitude save at the end of the Sohei's turn or be knocked Prone, Deaf and Paralyzed for 1d4 rounds. \smallskip

Furthermore, if an ally provokes an Attack of Opportunity, and is in a square the Sohei threatens, they may educate the ally in the wisdom of not provoking, by smacking them upside the head. This uses a single Attack of Opportunity but has no effect on the ally. However, no enemy may take an Attack of Opportunity against the ally for that action.}

\ability{Shadow Run:}{The Sohei can step onto the Ethereal Plane, and thus may walk on water or, for that matter, thin air as long as they end their turn standing on something solid. It can be a flimsy tree branch or a bamboo pole, mind you, as long as it is a solid object. Furthermore, they may step through Walls of Force, treating them merely as 10' movement per wall, and can add their Intelligence modifier to their Initiative.}

\ability{Weight of Thought:}{The Sohei may subtract their Wisdom modifier from armour penalties. If this reduces the penalty to zero, then they may move as if unarmoured, including being able to run at full speed in Heavy armour and tumble and everything.}

\ability{Divine Meditation:}{If the Sohei spends one minute meditating, they gain insight into future events, and may add a +10 Insight bonus on the next d20 roll they make. Additionally, doing so lets them regain Hit Points equal to the sum of all of their ranks in Knowledge skills.}

\ability{Enlightenment: We Are All Connected:}{The Sohei gains an even greater understanding of life, and how hatred is a weapon we forge for others that only hurts ourselves, that the air they exhale is inhaled by their enemies, and that everything is linked together.\smallskip

Any beneficial spell or effect granted to the Sohei or any of their allies they can see is instead, at the Sohei's option, granted to the Sohei and any number of their allies they can see. Additionally, any foe who subjects the Sohei (or any ally the Sohei can reach) to an ongoing condition such as Poison, Disease, Fear, Blindness or Nausea must make a Will save (DC 10 + half the Sohei's Hit Dice + their Wisdom modifier) or also receive the same affliction, for the same duration.}

\ability{Poetry in Motion:}{The Sohei may make a Move Action in addition to any other actions they perform in a turn. This movement may be broken up so as to walk around while making a Full Attack. Additionally, the Sohei counts as being in every square they move through until their turn is over, allowing them to flank with themselves and knock enemies past herself to provoke attacks of opportunity.}

\ability{Zen Fist:}{The actions of the Sohei cannot be interrupted by Immediate Actions, readied actions or attacks of opportunity. Instead, the Sohei declares their intent (such as "to make three attacks against the Hobgoblin Warlord and three against its Dragon mount"), the dice are rolled, and no attacks are actually made. However, at the end of the Sohei's turn, the targets have been attacked, actions performed and so on, just without there being a moment in time in which the actions were made. If you are confused by this, you're not Zen enough. \smallskip

Additionally, they may elect to move as by teleporting - the journey has been made without a single footstep having fallen. Doing this, however, does not allow for Poetry in Motion.}

\ability{Formation of the Eight Trigrams:}{Whenever the Sohei uses the Divine Go Stone Placement of Perfection ability, they form a strong barrier of defence with their allies. Draw an imaginary line between every pair of allies, including the Sohei. These form Walls of Force for one round, and are filled with Acid Fog that does not affect the Sohei or their allies. \smallskip

Additionally, the Sohei and their allies gain the benefits (but not the drawbacks) of a Stoneskin effect for one round. Any foe who attacks the Sohei or any of her allies during this time but fails to deal any damage automatically becomes Exhausted for one minute. In this case, allies are only those within 50' of the Sohei, not those back home in Kansas.}

\ability{Deadly Finishing Move:}{By spending 20 Tension as a Standard Action, the Sohei may unleash a killer Destroyer move. This attack deals an additional 2d6 damage per Hit Die of the Sohei, and bypasses all Damage Reduction (?including /-) and Regeneration. \smallskip

It also deals Strength Damage equal to the Sohei's Wisdom modifier, and the target must make either a Fortitude save or a Will save (Sohei's choice) or be slain instantly and completely destroyed. Not even ashes remain, and the soul is transported to the centre of the planet, on another plane (probably Hell). \smallskip

It should be mandated that the player name this move and either shout it out before unleashing it, or quietly state the name after the effects take place.}

\ability{Enlightenment: Transcendence:}{The Sohei is no longer a normal mortal creature. They become Ageless, and can only die by being killed. However, they are so hardcore they wouldn't die even if you killed them: they rise again 24 hours later unless the body is hacked apart and burned, the ashes then scattered into an Unhallowed land.}

\ability{Secret Scroll:}{The Sohei writes out the secret scrolls to enlightenment and, as they no longer have any goals of their own, having reached transcendence, must now strive to lead others to the door of knowledge. Anyone who reads the scrolls gains a permanent +5 Inherent bonus to Wisdom (the Sohei gains this when writing the scrolls), however from this point on they take a -5 penalty to hit the Sohei, due to having been influenced enough that the Sohei can predict their every move.}

%\input{baseclasses/whitemage}

\chapter{Prestige Classes}

\classname{Holy Crusader} \label{comm:prestige:holycrusader}
\vspace*{-8pt}
\quot{``Through great action, we may make a difference. It is only through inaction under the guise of pacifism that we lose to the great evils.''}

\ability{Requirements:}{}
\listprereq
\itemability{BAB:}{+5}
\itemability{Skills:}{Knowledge (the planes) 8 ranks, Knowledge (religion) 8 ranks}
\itemability{Feats:}{Blitz}
\itemability{Alignment:}{Any Good}
\itemability{Spellcasting:}{Must be able to form a Protection From or Magic Circle Against Evil at least once per day, as a spell or otherwise.}
\itemability{Special:}{Must have \emph{started} a fight with an Evil Outsider. Angelic Knights need only meet one of the two skill requirements (but still must be Good and have the Blitz feat).}
\end{list}\vspace*{8pt}

\ability{Hit Die:}{d10}

\ability{Class Skills:}{Whatever you want, since Koumei doesn't believe in cross-class skills.}

\ability{Skill Points at Each Level:}{4 + Int modifier.}

\begin{table}[tbh]
\begin{small}
\begin{tabular}{lp{1.9cm}p{0.7cm}p{0.7cm}p{0.7cm}l}
Level  &Base Attack  Bonus &Fort Save &Ref Save &Will Save &Special\\
1st &+1 &+2 &+0 &+2 &Aura of Daylight, Celestial Plate, Take The Burden\\
2nd &+2 &+3 &+0 &+3 &Aura of Menace, Divine Mount, Incite Vigilance\\
3rd &+3 &+3 &+1 &+3 &Fiend Slayer, Commanding Voice, Angel Wings\\
4th &+4 &+4 &+1 &+4 &Holy Word, Smite Evil, Empower the Helpless\\
5th &+5 &+4 &+1 &+4 &Resurrection, Decry the Wicked, Angel Wings\\
\end{tabular}
\end{small}
\end{table}

\smallskip\noindent All of the following are Class Features of the Holy Crusader prestige class.

\ability{Weapon and Armor Proficiency:}{The Holy Crusader gains no proficiency with any weapon or armor.}

\ability{Aura of Daylight:}{The Holy Crusader constantly emits Daylight, dealing damage to Undead within 30' equal to 1d6+Cha every round. They may suppress this ability at will if they want to sneak about or, you know, sleep.}

\ability{Celestial Plate:}{The Angels are so impressed by the Crusader that they give her a set of Adamantine Full Plate with an Enhancement bonus equal to her level divided by 4 (round down). Additionally, it grants DR 10/Good (making it useful when fighting Evil), and adds its Enhancement Bonus to armour to any damage rolls against creatures with the [Evil] subtype. Additionally, if the character has flight, they may still fly when wearing this, despite the weight.}

\ability{Take the Burden:}{With a Standard Action, the Holy Crusader may make a touch attack against an ally and ease their suffering. They may Cure 1d6 damage per hit die, receiving half the damage themselves, or may transfer one affliction (such as Nausea, a Poison or Disease, negative levels or Stunning) from the target to themselves. If they are immune to the affliction, they luck out and the affliction is just cured.}

\ability{Aura of Menace:}{A righteous aura surrounds Holy Crusaders that fight or get angry. Any hostile creature within a 20-foot radius of a Holy Crusader must succeed on a Will save (DC 10 + \half Hit Die + Charisma modifier) to resist its effects. Those who fail take a -2 penalty on attacks, AC, and saves for 24 hours or until they successfully hit the Holy Crusader that generated the aura. A creature that has resisted or broken the effect cannot be affected again by the same Holy Crusader�s aura for 24 hours.}

\ability{Divine Mount:}{The Holy Crusader is blessed with a special mount to ride. It is completely loyal, and if slain, another replaces it after 24 hours. The mount may be any [Good] Magical Beast (including Celestial Animals/Vermin), or a Half-Celestial Animal or Vermin, with a CR at least 3 less than the Crusader's hit dice.}

\ability{Incite Vigilance:}{As a Supernatural Ability once per hour, the Holy Crusader may drive allies to their limits, creating a Mass Haste effect. This is a verbal ability that requires a Move-Equivalent Action. The benefits also apply to the Crusader herself.}

\ability{Fiend Slayer:}{The Holy Crusader is especially trained to kill fiends, somewhere between an occupation, hobby and driving goal. When the Crusader strikes an [Evil, Extraplanar] Outsider, they lose access to one Spell-like or Supernatural ability of the Crusader's choice for one round per class level. Multiple strikes remove multiple abilities. If the fiend is left with no abilities and is struck again, they must make a Fortitude save (DC 10 + \half the Crusader's HD + their Charisma modifier) or be Destroyed (as per ``Destruction'').}

\ability{Commanding Voice:}{The Holy Crusader may caste Greater Command as a Spell-like Ability at will to anyone who failed to save against their Aura of Menace. The Save DC is equal to 10 + \half the Crusader's Hit Dice + their Charisma modifier.}

\ability{Angel Wings:}{At levels 3 and 5, the Holy Crusader receives a bonus [Celestial] feat. This must be a feat that grants wings.}

\ability{Holy Word:}{Three times per day, the Holy Crusader may cast Holy Word as a Supernatural Ability. Caster level equals hit dice and cannot be augmented beyond this.}

\ability{Smite Evil:}{Once per hour, the Holy Crusader may Smite Evil. This is a non-action that alters an attack that is about to be made, adding their Charisma modifier to the attack roll and hit dice to the damage roll. These benefits only apply to Evil enemies. Furthermore, Divine flames consume the target, treating them as on fire, except the damage is Holy.}

\ability{Empower the Helpless:}{Once per day as a Supernatural Ability, a Holy Crusader may grant great power to up to 2 NPCs per class level. The NPCs must have a CR of no more than 1 each. Once blessed, they become a frenzied mob of righteousness for 1 hour per class level. Treat them as a single Human Barbarian that happens to take up more space than usual and has the [Swarm] subtype, with a number of levels equal to the combined hit dice of all of the NPCs. They gain Strength and Constitution scores equal to 10 plus the class level of the Crusader, Dexterity 12, and the average Intelligence, Wisdom and Charisma of the mob.}

\ability{Resurrection:}{Once per day as a Supernatural Ability, the Holy Crusader may create a True Resurrection effect.}

\ability{Decry the Wicked:}{As a Standard Action at will, the Holy Crusader can shout out the sins of an Evil aligned foe. This is a verbal Supernatural Ability that extends out to 100 feet. The target must make a Will save or suffer from Heavy Encumbrance and become Fatigued for 1 minute. Every round, they suffer nonlethal damage equal to 2d6 plus the Crusader's hit dice.}


\chapter{Feats/Spells/Spheres}

\section{Feats}

%\newcommand{\babfeat}[7]{\noindent\textbf{#1} \\ \emph{#2} \\ \textbf{Benefit:}#3 \\ \textbf{+1:}#4  \\ \textbf{+6:}#5 \\ \textbf{+11:}#6 \\ \textbf{+16:}#7 \bigskip }
%\newcommand{\skillfeat}[7]{\noindent\textbf{#1} \\ \emph{#2} \\ \textbf{Benefit:} #3 \\ \textbf{4:} #4  \\ \textbf{9:} #5 \\ \textbf{14:} #6 \\ \textbf{19:} #7 \bigskip }

\section{The Failure of Feats}
\vspace*{-10pt}
\quot{``How about instead of being able to travel anywhere in the multiverse, transform yourself into anything you can think of, stop time, and slay everyone you can see, we just give a nice +1 to hit with your secondary weapon? Deal?''}

\noindent\desc{Feats were an interesting idea when they were ported to 3rd edition D\&D. But let's face it; they don't go nearly far enough. Feats were made extremely conservative in their effects on the game because the authors didn't want to offend people with too radical a change. Well, now we've had third edition for 6 years, and we're offended. Feats are an interesting and tangible way to get unique abilities onto a character, but they have fallen prey to two key fallacies that has ended up turning the entire concept to ashes in our mouths. The first is the idea that if you think of something kind of cool for a character to do, you should make it a feat. That sounds compelling, but you only get 7 feats in your whole life. If you have to spend a feat for every cool thing you ever do, you're not going to do very many cool things in the approximately 260 encounters you'll have on your way from 1st to 20th level. The second is the idea that a feat should be equivalent to a cantrip or two. This one is even less excusable, and just makes us cry. A +1 bonus is something that you seriously might forget that you even have. Having one more +1 bonus doesn't make your character unique, it makes you a sucker for spending one of the half dozen feats you'll ever see on a bonus the other players won't even mention when discussing your character.}

\noindent\desc{We all understand this problem, what do we do about it? Well, for starters, Feats have to do more things. Many characters are 5th level or so and they only have 2 feats. Those feats should describe their character in a much more salient way than ``I'm no worse shooting into melee than I am shooting at people with cover that isn't my friends.'' This was begun with the tactical feats, but it didn't go far enough. It's not enough to add additional feats that do something halfway interesting for high level characters to have -- we actually have to replace the stupid one dimensional feats in the PHB with feats that rational people would care about in any way. Spending a single feat should be enough to make you a ``sniper character'' because for a substantial portion of your life you only get one feat. Secondly, we have to clear away feats that don't provide numeric bonuses large enough to care about. The minimum bonus you'll ever notice is +3, because that's actually larger than the difference between having rolled well and having rolled poorly on your starting stats. Numeric bonuses smaller than that are actually insulting and need to be removed from the feats altogether. 3.5 Skill Focus was a nice start, but that's all it was -- a start.}

\noindent{Furthermore, the fundamental structure of feats has been a disaster. The system of prerequisites often ensures that characters won't get an ability before it would be level appropriate for them to do so, but actually does nothing to ensure that such characters are in fact getting level appropriate abilities. Indeed, if a 12th level character decides that they want to pursue a career in shooting people in the face, they have to start all over gaining an ability that is supposed to be level appropriate for a 1st level character. Meanwhile, when a wizard of 12th level decides to pursue some new direction in spellcasting -- he learns a new 6th level spell right off -- and gets an ability that's level appropriate for a 12th level character.}

\subsection{Exploits}

\noindent\desc{Getting proficiency with a weapon isn't worth a feat. They hand that crap out with your character class for free. Seriously, even exotic weapon proficiencies aren't a big deal. Therefore, we're instituting Exploits as something that can be acquired in-game. These are for any of the binary abilities that simply don't have a massive impact on your character's performance at any level.}

\noindent\desc{If you have Martial Weapon Proficiency, it's really unreasonable for it to be that hard to learn how to use a new weapon, whether it's exotic or not. If you spend a week training with a weapon, you can make an Int check (DC 10) to simply gain the Exploit of Exotic Weapon Proficiency. And no, you can't take 10 on that.}

\noindent{If you don't have Martial Weapon Proficiency and you want to use a new weapon, that's touchier. But if you have a weapon for an entire level, you should just gain proficiency in it when you gain your next level whatever level you happen to select.}

\subsection{The New Feat System}

\noindent{So where are we going with this? First of all, feat chains are gone. That seemed like a good idea, but it wasn't. Secondly, the vast majority of feats don't have prerequisites at all, they scale. A [Combat] feat scales to your Base Attack Bonus, a [Skill] feat scales to your ranks in a skill, and a [Metamagic] feat scales to the highest level spell you can cast. And that's because those are the only things in the game that actually have anything to do with the level your character is in any way that we feel good about.}

\section{The New Combat Ready Feats} \label{feats:combat}

\begin{multicols}{2}
\hypertarget{feat:blindfighting}{}\babfeat{Blind Fighting [Combat]}{
You don't have to see to kill.}{
You may reroll your miss chances caused by concealment.}{
While in darkness, you may move your normal speed without difficulty.}{
You have Blindsense out to 60', this allows you to know the location of all creatures within 60'.}{
You have Tremorsense out to 120', this allows you to ``see'' anything within 120' that is touching the earth.}{
You cannot be caught flat footed.}

\hypertarget{feat:blitz}{}\babfeat{Blitz [Combat]}{
You go all out and try to achieve goals in a proactive manner.}{
While charging, you may opt to lose your Dexterity Bonus to AC for one round, but inflicting an extra d6 of damage if you hit.}{
You may go all out when attacking, adding your Base Attack Bonus to your damage, but provoking an Attack of Opportunity.}{
Bonus attacks made in a Full Attack for having a high BAB are made with a -2 penalty instead of a -5 penalty.}{
Every time you inflict damage upon an opponent with your melee attacks, you may immediately use an Intimidate attempt against that opponent as a bonus action.}{
You may make a Full Attack action as a Standard Action.}

\hypertarget{feat:combatlooting}{}\babfeat{Combat Looting [Combat]}{
You can put things into your pants in the middle of combat.}{
You may sheathe or store an object as a free action.}{
You get a +3 bonus to  \hyperlink{combat:disarm}{Disarm} attempts. Picking up objects off the ground does not provoke an attack of opportunity.}{
As a Swift action, you may take a ring, amulet/necklace, headband, bracer, or belt from an opponent you have successfully \hyperlink{combat:grapple}{grappled}. You may pick up an item off the ground in the middle of a move action.}{
If you are grappling with an opponent, you may activate or deactivate their magic items with a successful Use Magic Device check. You may make Appraise checks as a free action.}{
You can take 10 on Use Magic Device and Sleight of Hand checks.}

\vspace{100pt}
\hypertarget{feat:combatschool}{}\babfeat{Combat School [Combat]}{
You are a member of a completely arbitrary fighting school that has a number of recognizable signature fighting moves.}{
First, name your fighting style (such as ``Hammer and Anvil Technique'' or ``Crescent Moon Style'', or ``Way of the Lightning Mace''). This fighting style only works with a small list of melee weapons that you have to describe the connectedness to the DM in a half-way believable way. Now, whenever you are using that technique in melee combat, you gain a +2 bonus on attack rolls.}{
Your immersion in your technique gives you great martial prowess, you gain a +2 to damage rolls in melee combat.}{
When you strike your opponent with the signature moves of your fighting school in melee, they must make a Fortitude Save (DC 10 + 1/2 your level + your Strength bonus) or become dazed for one round.}{
You may take 10 on attack rolls while using your special techniques. The DC to disarm you of a school-appropriate weapon is increased by 4.}{
You may add +5 to-hit on any one attack you make after the first each turn. If you hit an opponent twice in one round, all further attacks this round against that opponent are made with The Edge.}

\hypertarget{feat:command}{}\babfeat{Command [Combat] [Leadership]}{
You lead tiny men.}{
You have a Command Rating equal to your Base Attack Bonus divided by five (round up).}{
You can muster a group of followers. Your leadership score is your Base Attack Bonus plus your Charisma Modifier.}{
You are able to delegate command to a loyal cohort. A cohort is an intelligent and loyal creature with a CR at least 2 less than your character level. Cohorts gain levels when you do.}{
With a Swift Action you may rally troops, allowing all allies within medium range of yourself to reroll their saves vs. Fear and gain a +2 Morale Bonus to attack and damage rolls for 1 minute. This is a language-dependent ability that may be used an unlimited number of times.}{
Your allies gain a +2 morale bonus to all saving throws if they can see you and you are within medium range.}


\hypertarget{feat:dangersense}{}\babfeat{Danger Sense [Combat]}{
Maybe Spiders tell you what's up. You certainly react to danger with uncanny effectiveness.}{
You get a +3 bonus on Initiative checks.}{
For the purpose of Search, Spot, and Listen, you are always considered to be ``actively searching''. You also get Uncanny Dodge.}{
You may take 10 on Listen, Spot, and Search checks.}{
You may make a Sense Motive check (opposed by your opponent's Bluff check) immediately whenever any creature approaches within 60' of you with harmful intent. If you succeed, you know the location of the creature even if you cannot see it.}{
You are never surprised and always act on the first round of any combat.}


\hypertarget{feat:elusivetarget}{}\babfeat{Elusive Target [Combat]}{
You are very hard to hit when you want to be.}{
You gain a +2 Dodge bonus to AC.}{
Your opponents do not gain flanking or higher ground bonuses against you.}{
Your opponents do not inflict extra damage from the \hyperlink{combat:powerattack}{Power Attack} option.}{
Diverting Defense -- As an immediate action, you may redirect an attack against you to any creature in your threatened range, friend or foe. You may not redirect an attack to the creature making the attack.}{
As an immediate action, you may make an attack that would normally hit you miss instead.}


\hypertarget{feat:experttactician}{}\babfeat{Expert Tactician [Combat]}{
You benefit your allies so good they remember you long time.}{
You gain a +4 bonus when flanking instead of the normal +2 bonus. Your allies who flank with you gain the same advantage.}{
You may \hyperlink{combat:feint}{Feint} as an Immediate action.}{
As a move action, you may make any 5' square adjacent to yourself into difficult ground.}{
For determining flanking with your allies, you may count your location as being 5' in any direction from your real location.}{
You ignore Cover bonuses less than full cover.}


\hypertarget{feat:ghosthunter}{}\babfeat{Ghost Hunter [Combat]}{
You smack around those folks in the spirit world.}{
Your attacks have a 50\% chance of striking incorporeal opponents even if they are not magical.}{
You can hear incorporeal and ethereal creatures as if they lacked those traits (note that shadows and the like rarely bother to actively move silently).}{
You can see invisible and ethereal creatures as if they lacked those traits.}{
Your attacks count as if you had the Ghost Touch property on your weapons.}{
Any Armor or shield you use benefits from the Ghost Touch quality.}

\vspace{100pt}
\hypertarget{feat:giantslayer}{}\babfeat{Giant Slayer [Combat]}{
Everyone has a specialty. Yours is miraculously finding ways to stab creatures in the face when it seems improbable that you would be able to reach that high.}{
When you perform a \hyperlink{combat:grabon}{grab on} Grapple maneuver, you do not provoke an attack of opportunity.}{
You gain a +4 Dodge bonus to your AC and Reflex Saves against attacks from any creature with a longer natural reach than your own.}{
You have The Edge against any creature you attack that is larger than you. Also, an opponent using the Improved Grab ability on you provokes an attack of opportunity from you. You may take this attack even if you do not threaten a square occupied by your opponent.}{
When you attempt to trip an opponent, you may choose whether your opponent resists with Strength or Dexterity.}{
When involved in an opposed bull rush, grapple, or trip check as the attacker or defender, you may negate the size modifier of both participants. You may not choose to negate the size modifier of only one character.}


\hypertarget{feat:greatfortitude}{}\babfeat{Great Fortitude [Combat]}{
You are so tough. Your belly is like a prism.}{
You gain a +3 bonus to your Fortitude Saves.}{
You die at -20 instead of -10.}{
You gain 1 hit point per level.}{
You gain DR of 5/-.}{
You are immune to the fatigued and exhausted conditions. If you are already immune to these conditions, you gain 1 hit point per level for each condition you were already immune to.}


\hypertarget{feat:hordebreaker}{}\babfeat{Horde Breaker[Combat]}{
You kill really large numbers of people.}{
You gain a number of extra attacks of opportunity each round equal to your Dexterity Bonus (if positive).}{
Whenever you drop an opponent with a melee attack, you are entitled to a bonus ``cleave'' attack against another opponent you threaten. You may not take a 5' step or otherwise move before taking this bonus attack. This Cleave attack is considered an attack of opportunity.}{
You may take a bonus 5' step every time you are entitled to a Cleave attack, which you may take either before or after the attack.}{
You may generate an aura of fear on any opponents within 10' of yourself whenever you drop an opponent in melee. The save DC is 10 + the Hit Dice of the dropped creature.}{
Opponents you have the Edge against provoke an attack of opportunity from you by moving into your threatened area or attacking you.}

\vspace{30pt}
\hypertarget{feat:hunter}{}\babfeat{Hunter [Combat]}{
You can move around and shoot things with surprising effectiveness.}{
The penalties for using a ranged weapon from an unstable platform (such as a ship or a moving horse) are halved.}{
Shot on the Run -- you may take a standard action to attack with a ranged weapon in the middle of a move action, taking some of your movement before and some of your movement after your attack. That still counts as your standard and move action for the round.}{
You suffer no penalties for firing from unstable ground, a running steed, or any of that.}{
You may take a full round action to take a double move and make a single ranged attack from any point during your movement.}{
You may take a full round action to run a full four times your speed and make a single ranged attack from any point during your movement. You retain your Dexterity modifier to AC while running.}


\hypertarget{feat:insightfulstrike}{}\babfeat{Insightful Strike [Combat]}{
You Hack people down with inherent awesomeness.}{
You may use your Wisdom Modifier in place of your Strength Modifier for your melee attack rolls.}{
Your attacks have The Edge against an opponent who has a lower Wisdom and Dexterity than your own Wisdom, regardless of relative BAB.}{
Your melee attacks have a doubled critical threat range.}{
You make horribly telling blows. The extra critical multiplier of your melee attacks is doubled (x2 becomes x3, x3 becomes x5, and x4 becomes x7).}{
Any Melee attack you make is considered to be made with a magic weapon that has an enhancement bonus equal to your Wisdom Modifier (if positive).}


\hypertarget{feat:ironwill}{}\babfeat{Iron Will [Combat]}{
You are able to grit your teeth and shake off mental influences.}{
You gain a +3 bonus to your Willpower saves.}{
You gain the slippery mind ability of a Rogue.}{
If you are stunned, you are dazed instead.}{
You do not suffer penalties from pain and fear.}{
You are immune to compulsion effects.}

\vspace{300pt}
\hypertarget{feat:juggernaut}{}\babfeat{Juggernaut [Combat]}{
You are an unstoppable Juggernaut.}{
You may be considered one size category larger for the purposes of any size dependant roll you make (such as a Bull Rush, Overrun, or Lift action).}{
You do not provoke an attack of opportunity for entering an opponent's square.}{
You gain a +4 bonus to attack and damage rolls to destroy objects. You may shatter a Force Effect by inflicting 30 damage on it.}{
When you successfully \hyperlink{combat:bullrush}{bullrush} or overrun an opponent, you automatically Trample them, inflicting damage equal to a natural slam attack for a creature of your size.}{
You gain the Rock Throwing ability of any standard Giant with a strength equal to or less than yourself.}


\hypertarget{feat:lightningreflexes}{}\babfeat{Lightning Reflexes [Combat]}{
You are fasty McFastFast. It helps keep you alive.}{
You gain a +3 bonus to your Reflex saves.}{
You gain Evasion, if you already have Evasion, that stacks to Improved Evasion.}{
You may make a Balance Check in place of your Reflex save.}{
You gain a +3 bonus to your Initiative.}{
When you take the Full Defense Action, add your level to your AC.}


\hypertarget{feat:mageslayer}{}\babfeat{Mage Slayer [Combat]}{
You have trained long and hard to kill magic users. Maybe you hate them, maybe you just noticed that most of the really dangerous creatures in the world use magic.}{
You gain Spell Resistance of 5 + Character Level.}{
Damage you inflict is considered ``ongoing damage'' for the purposes of concentration checks made before the beginning of your next round. All your attacks in a round are considered the same source of continuing damage.}{
Creatures cannot cast defensively within your threat range.}{
Your attacks ignore Deflection bonuses to AC.}{
When a creature uses a [Teleportation] effect within medium range of yourself, you may choose to be transported as well. This is not an action.}


\hypertarget{feat:mountedcombat}{}\skillfeat{Mounted Combat [Skill]}{
You are at your best when fighting with an ally that you are sitting on.}{
Ride Ranks:}{
Once per turn, you may attempt to negate an attack that hits your mount by making a Ride skill check with a DC equal to the AC that the attack hit. Attacks that do not require an attack roll cannot be negated in this way.}{
While Mounted, you may take a charge attack at any point along your mount's movement, so long as your mount is moving in a straight line up to the point of your attack.}{
You suffer no penalty to your ride or handle animal skill checks when training or riding unusual mounts such as magical beasts or dragons.}{
You may use your Ride Check in place of your mount's Balance, Jump, Climb, or Reflex Saving Throws.}{
Any time a spell effect would target your mount, you may elect to have it target you instead. Any time a spell effect would target you, you may elect to have it affect your Mount instead.}


\hypertarget{feat:murderousintent}{}\babfeat{Murderous Intent [Combat]}{
You stab people in the face.}{
You may make a \hyperlink{combat:coupdegrace}{Coup de Grace} as a standard action.}{
When you kill an opponent, you gain a +2 Morale Bonus to your attack and damage rolls for 1 minute.}{
Once per round, you may take an attack of opportunity against an opponent who is denied their Dexterity bonus to AC.}{
You may take a \hyperlink{combat:coupdegrace}{Coup de Grace} action against opponents who are stunned.}{
You may take a \hyperlink{combat:coupdegrace}{Coup de Grace} action against opponents who are dazed.}


\hypertarget{feat:phalanxfighter}{}\babfeat{Phalanx Fighter[Combat]}{
You fight well in a group.}{
You may take attacks of opportunity even while flat footed.}{
Any Dodge bonus to AC you gain is also granted to any adjacent allies for as long as you benefit from the bonus and your ally remains adjacent.}{
Charging is an action that provokes an attack of opportunity from you. This attack is considered to be a ``readied attack'' if it matters for purposes like setting against a charge.}{
You may attack with a reach weapon as if it was not a reach weapon. Thus, a medium creature would normally threaten at 5' and 10' with a reach weapon.}{
You may take an Aid Another action once per round as a free action. You provide double normal bonuses from this effect.}

\vspace{300pt}
\hypertarget{feat:pointblankshot}{}\hypertarget{feat:pbshot}{}\babfeat{Point Blank Shot [Combat]}{
You are crazy good using a ranged weapon in close quarters.}{
When you are within 30' of your target, your attacks with a ranged weapon gain a +3 bonus to-hit.}{
You add your base attack bonus to damage with any ranged attack within the first range increment.}{
You do not provoke an attack of opportunity when you make a ranged attack.}{
When armed with a Ranged Weapon, you may make attacks of opportunity against opponents who provoke them within 30' of you. Movement within this area does not provoke an attack of opportunity.}{
With a Full Attack action, you may fire a ranged weapon once at every single opponent within the first range increment of your weapon. You gain no additional attacks for having a high BAB. Make a single attack roll for the entire round, and compare to the armor class of each opponent within range.}


\hypertarget{feat:sniper}{}\babfeat{Sniper [Combat]}{
Your shooting is precise and dangerous.}{
Your range increments are 50\% longer than they would ordinarily be. Any benefit of being within 30' of an opponent is retained out to 60'.}{
Precise Shot -- You do not suffer a -4 penalty when firing a ranged weapon into melee and never hit an unintended target in close combats or grapples.}{
Sharp Shooting -- Your ranged attacks ignore Cover Bonuses (total cover still bones you).}{
Opponents struck by your ranged attacks do not automatically know what square your attack came from, and must attempt to find you normally.}{
Any time you hit an opponent with a ranged weapon, it is counted as a critical threat. If your weapon already had a 19-20 threat range, increase its critical multiplier by 1.}


\hypertarget{feat:subtlecut}{}\babfeat{Subtle Cut [Combat]}{
You cut people so bad they have to ask you about it later.}{
Any time you damage an opponent, that damage is increased by 1.}{
As a standard action, you can make a weapon attack that also reduces a creature's movement rate. For every 5 points of damage this attack does, reduce the creature's movement by 5'. This penalties lasts until the damage is healed.}{
As a standard action, you may make a weapon attack that also does 2d4 points of Dexterity damage.}{
Any weapon attack that you make at this level acts as if the weapon had the wounding property.}{
As a standard action, you may make an attack that dazes your opponent. This effect lasts one round, and has a DC of 10 + half your level + your Intelligence bonus.}

\vspace{10pt}
\hypertarget{feat:twoweaponfighting}{}\babfeat{Two Weapon Fighting [Combat]}{
When armed with two weapons, you fight with two weapons rather than picking and choosing and fighting with only one. Kind of obvious in retrospect.}{
You suffer no penalty for doing things with your off-hand. When you make an attack or full-attack action, you may make a number of attacks with your off-hand weapon equal to the number of attacks you are afforded with your primary weapon.}{
While armed with two weapons, you gain an extra Attack of Opportunity each round for each attack you would be allowed for your BAB, these extra attacks of opportunity must be made with your off-hand.}{
You gain a +2 Shield Bonus to your armor class when fighting with two weapons and not flat footed.}{
You may Feint as a Swift action.}{
While fighting with two weapons and not flat footed you may add the enhancement bonus of either your primary or your off-hand weapon to your Shield Bonus to AC.}


\hypertarget{feat:weaponfinesse}{}\babfeat{Weapon Finesse [Combat]}{
You are incredibly deft with a sword.}{
You may use your Dexterity Modifier instead of your Strength modifier for calculating your melee attack bonus.}{
Your special attacks are considered to have the Edge when you attack an opponent with a Dexterity modifier smaller than yours, even if your Base Attack Bonus is not larger.}{
You may use your Dexterity modifier in place of your Strength modifier when attempting to trip an opponent.}{
You may use your Dexterity modifier in place of your Strength modifier for calculating your melee damage.}{
Opportunist -- Once per turn, when an opponent is struck, you may take an attack of opportunity on that opponent.}

\vspace{300pt}
\hypertarget{feat:whirlwind}{}\hypertarget{feat:whirlwindattack}{}\babfeat{Whirlwind [Combat]}{
You are just as dangerous to everyone around you as to anyone around you.}{
As a full round action, you may make a single attack against each opponent you can reach. Roll one attack roll and compare to each available opponent's AC individually.}{
You gain a +3 bonus to Balance checks.}{
As a full round action, you may take a regular move action and make a single attack against each opponent you can reach at any point during your movement. Roll one attack roll and compare to each available opponent's AC individually.}{
Until your next round after making a whirlwind attack, you may take an attack of opportunity against any opponent that enters your threatened area.}{
As a full round action, you take a charge action, overrunning any creature in your path, and may make a single attack against each opponent you can reach at any point during your movement. Roll one attack roll and compare to each available opponent's AC individually.}


\hypertarget{feat:zenarchery}{}\babfeat{Zen Archery [Combat]}{
You are very calm about shooting people in the face. That's a good place to be.}{
You may use your Wisdom Modifier in place of your Dexterity Modifier on ranged attack rolls.}{
Any opponent you can hear is considered an opponent you can see for purposes of targeting them with ranged attacks.}{
If you cast a Touch Spell, you can deliver it with a ranged weapon (though you must hit with a normal attack to deliver the spell).}{
As a Full Round Action, you may make one ranged attack with a +20 Insight bonus to hit.}{
As a Full Round Action, you may make one ranged attack with a +20 Insight bonus to hit. If this attack hits, your attack is automatically upgraded to a critical threat. If the threat range of your weapon is 19-20, your critical multiplier is increased by one.}

\end{multicols}

\section{Spells}

\section{Spells}

\begin{multicols}{2}
\begin{small}

\featname{Acid Arrow}
\shortability{Conjuration}{(Creation) [Acid]}
\shortability{Level:}{Sor/Wiz 2}
\shortability{Components:}{V, S, M, F}
\shortability{Casting Time:}{1 standard action}
\shortability{Range:}{Long (400 ft. + 40 ft./level)}
\shortability{Effect:}{One arrow of acid}
\shortability{Duration:}{1 round + 1 round per three levels}
\shortability{Saving Throw:}{None}
\shortability{Spell Resistance:}{No}
\smallskip\noindent A magical arrow of acid springs from your hand and speeds to its target. You must succeed on a ranged touch attack to hit your target. The arrow deals 2d4 points of acid damage with no splash damage. For every three caster levels (to a maximum of 18th), the acid, unless somehow neutralized, lasts for another round, dealing another 2d4 points of damage in that round.

\shortability{Material Component:}{Powdered rhubarb leaf and an adder's stomach.}
\shortability{Focus:}{A dart.}

\featname{Acid Fog}
\shortability{Conjuration}{(Creation) [Acid]}
\shortability{Level:}{Sor/Wiz 6, Water 7}
\shortability{Components:}{V, S, M/DF}
\shortability{Casting Time:}{1 standard action}
\shortability{Range:}{Medium (100 ft. + 10 ft./level)}
\shortability{Effect:}{Fog spreads in 20-ft. radius, 20 ft. high}
\shortability{Duration:}{1 round/level}
\shortability{Saving Throw:}{None}
\shortability{Spell Resistance:}{No}
\smallskip\noindent Acid fog creates a billowing mass of misty vapors similar to that produced by a solid fog spell. In addition to slowing creatures down and obscuring sight, this spell's vapors are highly acidic. Each round on your turn, starting when you cast the spell, the fog deals 2d6 points of acid damage to each creature and object within it.

\shortability{Arcane Material Component:}{A pinch of dried, powdered peas combined with powdered animal hoof.}

\featname{Acid Splash}
\shortability{Conjuration}{(Creation) [Acid]}
\shortability{Level:}{Sor/Wiz 0}
\shortability{Components:}{V, S}
\shortability{Casting Time:}{1 standard action}
\shortability{Range:}{Close (25 ft. + 5 ft./2 levels)}
\shortability{Effect:}{One missile of acid}
\shortability{Duration:}{Instantaneous}
\shortability{Saving Throw:}{None}
\shortability{Spell Resistance:}{No}
\smallskip\noindent You fire a small orb of acid at the target. You must succeed on a ranged touch attack to hit your target. The orb deals 1d3 points of acid damage.

\featname{Aid}
\shortability{Enchantment}{(Compulsion) [Mind-Affecting]}
\shortability{Level:}{Clr 2, Good 2, Luck 2}
\shortability{Components:}{V, S, DF}
\shortability{Casting Time:}{1 standard action}
\shortability{Range:}{Touch}
\shortability{Target:}{Living creature touched}
\shortability{Duration:}{1 min./level}
\shortability{Saving Throw:}{None}
\shortability{Spell Resistance:}{Yes (harmless)}
\smallskip\noindent Aid grants the target a +1 morale bonus on attack rolls and saves against fear effects, plus temporary hit points equal to 1d8 + caster level (to a maximum of 1d8+10 temporary hit points at caster level 10th).

\featname{Air Walk}
\shortability{Transmutation}{[Air]}
\shortability{Level:}{Air 4, Clr 4, Drd 4}
\shortability{Components:}{V, S, DF}
\shortability{Casting Time:}{1 standard action}
\shortability{Range:}{Touch}
\shortability{Target:}{Creature (Gargantuan or smaller) touched}
\shortability{Duration:}{10 min./level}
\shortability{Saving Throw:}{None}
\shortability{Spell Resistance:}{Yes (harmless)}
\smallskip\noindent The subject can tread on air as if walking on solid ground. Moving upward is similar to walking up a hill. The maximum upward or downward angle possible is 45 degrees, at a rate equal to one-half the air walker's normal speed.

\smallskip\noindent A strong wind (21+ mph) can push the subject along or hold it back. At the end of its turn each round, the wind blows the air walker 5 feet for each 5 miles per hour of wind speed. The creature may be subject to additional penalties in exceptionally strong or turbulent winds, such as loss of control over movement or physical damage from being buffeted about.

\smallskip\noindent Should the spell duration expire while the subject is still aloft, the magic fails slowly. The subject floats downward 60 feet per round for 1d6 rounds. If it reaches the ground in that amount of time, it lands safely. If not, it falls the rest of the distance, taking 1d6 points of damage per 10 feet of fall. Since dispelling a spell effectively ends it, the subject also descends in this way if the air walk spell is dispelled, but not if it is negated by an antimagic field.

\smallskip\noindent You can cast air walk on a specially trained mount so it can be ridden through the air. You can train a mount to move with the aid of air walk (counts as a trick; see Handle Animal skill) with one week of work and a DC 25 Handle Animal check.

\featname{Alarm}
\shortability{Abjuration}{}
\shortability{Level:}{Brd 1, Rgr 1, Sor/Wiz 1}
\shortability{Components:}{V, S, F/DF}
\shortability{Casting Time:}{1 standard action}
\shortability{Range:}{Close (25 ft. + 5 ft./2 levels)}
\shortability{Area:}{20-ft.-radius emanation centered on a point in space}
\shortability{Duration:}{2 hours/level (D)}
\shortability{Saving Throw:}{None}
\shortability{Spell Resistance:}{No}
\smallskip\noindent Alarm sounds a mental or audible alarm each time a creature of Tiny or larger size enters the warded area or touches it. A creature that speaks the password (determined by you at the time of casting) does not set off the alarm. You decide at the time of casting whether the alarm will be mental or audible.

\shortability{Mental Alarm:}{A mental alarm alerts you (and only you) so long as you remain within 1 mile of the warded area. You note a single mental ``ping'' that awakens you from normal sleep but does not otherwise disturb concentration. A silence spell has no effect on a mental alarm.}
\shortability{Audible Alarm:}{An audible alarm produces the sound of a hand bell, and anyone within 60 feet of the warded area can hear it clearly. Reduce the distance by 10 feet for each interposing closed door and by 20 feet for each substantial interposing wall.}

\smallskip\noindent In quiet conditions, the ringing can be heard faintly as far as 180 feet away. The sound lasts for 1 round. Creatures within a silence spell cannot hear the ringing.

\smallskip\noindent Ethereal or astral creatures do not trigger the alarm.

\smallskip\noindent Alarm can be made permanent with a permanency spell.

\shortability{Arcane Focus:}{A tiny bell and a piece of very fine silver wire}

\featname{Align Weapon}
\shortability{Transmutation}{[see text]}
\shortability{Level:}{Clr 2}
\shortability{Components:}{V, S, DF}
\shortability{Casting Time:}{1 standard action}
\shortability{Range:}{Touch}
\shortability{Target:}{Weapon touched or fifty projectiles (all of which must be in contact with each other at the time of casting)}
\shortability{Duration:}{1 min./level}
\shortability{Saving Throw:}{Will negates (harmless, object)}
\shortability{Spell Resistance:}{Yes (harmless, object)}
\smallskip\noindent Align weapon makes a weapon good, evil, lawful, or chaotic, as you choose. A weapon that is aligned can bypass the damage reduction of certain creatures. This spell has no effect on a weapon that already has an alignment.

\smallskip\noindent You can't cast this spell on a natural weapon, such as an unarmed strike.

\smallskip\noindent When you make a weapon good, evil, lawful, or chaotic, align weapon is a good, evil, lawful, or chaotic spell, respectively.

\featname{Alter Self}
\shortability{Transmutation}{}
\shortability{Level:}{Brd 2, Sor/Wiz 2}
\shortability{Components:}{V, S}
\shortability{Casting Time:}{1 standard action}
\shortability{Range:}{Personal}
\shortability{Target:}{You}
\shortability{Duration:}{10 min./level (D)}

\smallskip\noindent You assume the form of a creature of the same type as your normal form. The new form must be within one size category of your normal size. The maximum HD of an assumed form is equal to your caster level, to a maximum of 5 HD at 5th level. You can change into a member of your own kind or even into yourself.

\smallskip\noindent You retain your own ability scores. Your class and level, hit points, alignment, base attack bonus, and base save bonuses all remain the same. You retain all supernatural and spell-like special attacks and qualities of your normal form, except for those requiring a body part that the new form does not have (such as a mouth for a breath weapon or eyes for a gaze attack).

\smallskip\noindent You keep all extraordinary special attacks and qualities derived from class levels, but you lose any from your normal form that are not derived from class levels.

\smallskip\noindent If the new form is capable of speech, you can communicate normally. You retain any spellcasting ability you had in your original form, but the new form must be able to speak intelligibly (that is, speak a language) to use verbal components and must have limbs capable of fine manipulation to use somatic or material components.

\smallskip\noindent You acquire the physical qualities of the new form while retaining your own mind. Physical qualities include natural size, mundane movement capabilities (such as burrowing, climbing, walking, swimming, and flight with wings, to a maximum speed of 120 feet for flying or 60 feet for nonflying movement), natural armor bonus, natural weapons (such as claws, bite, and so on), racial skill bonuses, racial bonus feats, and any gross physical qualities (presence or absence of wings, number of extremities, and so forth). A body with extra limbs does not allow you to make more attacks (or more advantageous two-weapon attacks) than normal.

\smallskip\noindent You do not gain any extraordinary special attacks or special qualities not noted above under physical qualities, such as darkvision, low-light vision, blindsense, blindsight, fast healing, regeneration, scent, and so forth.

\smallskip\noindent You do not gain any supernatural special attacks, special qualities, or spell-like abilities of the new form. Your creature type and subtype (if any) remain the same regardless of your new form. You cannot take the form of any creature with a template, even if that template doesn't change the creature type or subtype.

\smallskip\noindent You can freely designate the new form's minor physical qualities (such as hair color, hair texture, and skin color) within the normal ranges for a creature of that kind. The new form's significant physical qualities (such as height, weight, and gender) are also under your control, but they must fall within the norms for the new form's kind. You are effectively disguised as an average member of the new form's race. If you use this spell to create a disguise, you get a +10 bonus on your Disguise check.

\smallskip\noindent When the change occurs, your equipment, if any, either remains worn or held by the new form (if it is capable of wearing or holding the item), or melds into the new form and becomes nonfunctional. When you revert to your true form, any objects previously melded into the new form reappear in the same location on your body they previously occupied and are once again functional. Any new items you wore in the assumed form and can't wear in your normal form fall off and land at your feet; any that you could wear in either form or carry in a body part common to both forms at the time of reversion are still held in the same way. Any part of the body or piece of equipment that is separated from the whole reverts to its true form.

\featname{Analyze Dweomer}
\shortability{Divination}{}
\shortability{Level:}{Brd 6, Sor/Wiz 6}
\shortability{Components:}{V, S, F}
\shortability{Casting Time:}{1 standard action}
\shortability{Range:}{Close (25 ft. + 5 ft./2 levels)}
\shortability{Targets:}{One object or creature per caster level}
\shortability{Duration:}{1 round/level (D)}
\shortability{Saving Throw:}{None or Will negates; see text}
\shortability{Spell Resistance:}{No}
\smallskip\noindent You discern all spells and magical properties present in a number of creatures or objects. Each round, you may examine a single creature or object that you can see as a free action. In the case of a magic item, you learn its functions, how to activate its functions (if appropriate), and how many charges are left (if it uses charges). In the case of an object or creature with active spells cast upon it, you learn each spell, its effect, and its caster level.

\smallskip\noindent An attended object may attempt a Will save to resist this effect if its holder so desires. If the save succeeds, you learn nothing about the object except what you can discern by looking at it. An object that makes its save cannot be affected by any other analyze dweomer spells for 24 hours.
Analyze dweomer does not function when used on an artifact.

\shortability{Focus:}{A tiny lens of ruby or sapphire set in a small golden loop. The gemstone must be worth at least 1,500 gp.}

\featname{Animal Growth}
\shortability{Transmutation}{}
\shortability{Level:}{Drd 5, Rgr 4, Sor/Wiz 5}
\shortability{Components:}{V, S}
\shortability{Casting Time:}{1 standard action}
\shortability{Range:}{Medium (100 ft. + 10 ft./level)}
\shortability{Targets:}{Up to one animal (Gargantuan or smaller) per two levels, no two of which can be more than 30 ft. apart}
\shortability{Duration:}{1 min./level}
\shortability{Saving Throw:}{Fortitude negates}
\shortability{Spell Resistance:}{Yes}
\smallskip\noindent A number of animals grow to twice their normal size and eight times their normal weight. This alteration changes each animal's size category to the next largest, grants it a +8 size bonus to Strength and a +4 size bonus to Constitution (and thus an extra 2 hit points per HD), and imposes a -2 size penalty to Dexterity. The creature's existing natural armor bonus increases by 2. The size change also affects the animal's modifier to AC and attack rolls and its base damage. The animal's space and reach change as appropriate to the new size, but its speed does not change.

\smallskip\noindent The spell also grants each subject damage reduction 10/magic and a +4 resistance bonus on saving throws. If insufficient room is available for the desired growth, the creature attains the maximum possible size and may make a Strength check (using its increased Strength) to burst any enclosures in the process. If it fails, it is constrained without harm by the materials enclosing it- the spell cannot be used to crush a creature by increasing its size.

\smallskip\noindent All equipment worn or carried by an animal is similarly enlarged by the spell, though this change has no effect on the magical properties of any such equipment.
Any enlarged item that leaves the enlarged creature's possession instantly returns to its normal size.
\smallskip\noindent 
The spell gives no means of command or influence over the enlarged animals.

\smallskip\noindent Multiple magical effects that increase size do not stack.

\featname{Animal Messenger}
\shortability{Enchantment}{(Compulsion) [Mind-Affecting]}
\shortability{Level:}{Brd 2, Drd 2, Rgr 1}
\shortability{Components:}{V, S, M}
\shortability{Casting Time:}{1 standard action}
\shortability{Range:}{Close (25 ft. + 5 ft./2 levels)}
\shortability{Target:}{One Tiny animal}
\shortability{Duration:}{One day/level}
\shortability{Saving Throw:}{None; see text}
\shortability{Spell Resistance:}{Yes}
\smallskip\noindent You compel a Tiny animal to go to a spot you designate. The most common use for this spell is to get an animal to carry a message to your allies. The animal cannot be one tamed or trained by someone else, including such creatures as familiars and animal companions.

\smallskip\noindent Using some type of food desirable to the animal as a lure, you call the animal to you. It advances and awaits your bidding. You can mentally impress on the animal a certain place well known to you or an obvious landmark. The directions must be simple, because the animal depends on your knowledge and can't find a destination on its own. You can attach some small item or note to the messenger. The animal then goes to the designated location and waits there until the duration of the spell expires, whereupon it resumes its normal activities.

\smallskip\noindent During this period of waiting, the messenger allows others to approach it and remove any scroll or token it carries. The intended recipient gains no special ability to communicate with the animal or read any attached message (if it's written in a language he or she doesn't know, for example).

\shortability{Material Component:}{A morsel of food the animal likes.}

\featname{Animal Shapes}
\shortability{Transmutation}{}
\shortability{Level:}{Animal 7, Drd 8}
\shortability{Components:}{V, S, DF}
\shortability{Casting Time:}{1 standard action}
\shortability{Range:}{Close (25 ft. + 5 ft./2 levels)}
\shortability{Targets:}{Up to one willing creature per level, all within 30 ft. of each other}
\shortability{Duration:}{1 hour/level (D)}
\shortability{Saving Throw:}{None; see text}
\shortability{Spell Resistance:}{Yes (harmless)}
\smallskip\noindent As polymorph, except you polymorph up to one willing creature per caster level into an animal of your choice; the spell has no effect on unwilling creatures. All creatures must take the same kind of animal form. Recipients remain in the animal form until the spell expires or until you dismiss it for all recipients. In addition, an individual subject may choose to resume its normal form as a full-round action; doing so ends the spell for that subject alone. The maximum HD of an assumed form is equal to the subject's HD or your caster level, whichever is lower, to a maximum of 20 HD at 20th level.

\featname{Animal Trance}
\shortability{Enchantment}{(Compulsion) [Mind-Affecting, Sonic]}
\shortability{Level:}{Brd 2, Drd 2}
\shortability{Components:}{V, S}
\shortability{Casting Time:}{1 standard action}
\shortability{Range:}{Close (25 ft. + 5 ft./2 levels)}
\shortability{Targets:}{Animals or magical beasts with Intelligence 1 or 2}
\shortability{Duration:}{Concentration}
\shortability{Saving Throw:}{Will negates; see text}
\shortability{Spell Resistance:}{Yes}
\smallskip\noindent Your swaying motions and music (or singing, or chanting) compel animals and magical beasts to do nothing but watch you. Only a creature with an Intelligence score of 1 or 2 can be fascinated by this spell. Roll 2d6 to determine the total number of HD worth of creatures that you fascinate. The closest targets are selected first until no more targets within range can be affected.

\smallskip\noindent A magical beast, a dire animal, or an animal trained to attack or guard is allowed a saving throw; an animal not trained to attack or guard is not.

\featname{Animate Dead}
\shortability{Necromancy}{[Evil]}
\shortability{Level:}{Clr 3, Death 3, Sor/Wiz 4}
\shortability{Components:}{V, S, M}
\shortability{Casting Time:}{1 standard action}
\shortability{Range:}{Touch}
\shortability{Targets:}{One or more corpses touched}
\shortability{Duration:}{Instantaneous}
\shortability{Saving Throw:}{None}
\shortability{Spell Resistance:}{No}
\smallskip\noindent This spell turns the bones or bodies of dead creatures into undead skeletons or zombies that follow your spoken commands.

\smallskip\noindent The undead can follow you, or they can remain in an area and attack any creature (or just a specific kind of creature) entering the place. They remain animated until they are destroyed. (A destroyed skeleton or zombie can't be animated again.)

\smallskip\noindent Regardless of the type of undead you create with this spell, you can't create more HD of undead than twice your caster level with a single casting of animate dead. (The desecrate spell doubles this limit)

\smallskip\noindent The undead you create remain under your control indefinitely. No matter how many times you use this spell, however, you can control only 4 HD worth of undead creatures per caster level. If you exceed this number, all the newly created creatures fall under your control, and any excess undead from previous castings become uncontrolled. (You choose which creatures are released.) If you are a cleric, any undead you might command by virtue of your power to command or rebuke undead do not count toward the limit. 

\shortability{Skeletons:}{A skeleton can be created only from a mostly intact corpse or skeleton. The corpse must have bones. If a skeleton is made from a corpse, the flesh falls off the bones.}
\shortability{Zombies:}{A zombie can be created only from a mostly intact corpse. The corpse must be that of a creature with a true anatomy.}
\shortability{Material Component:}{You must place a black onyx gem worth at least 25 gp per Hit Die of the undead into the mouth or eye socket of each corpse you intend to animate. The magic of the spell turns these gems into worthless, burned-out shells.}

\featname{Animate Objects}
\shortability{Transmutation}{}
\shortability{Level:}{Brd 6, Chaos 6, Clr 6}
\shortability{Components:}{V, S}
\shortability{Casting Time:}{1 standard action}
\shortability{Range:}{Medium (100 ft. + 10 ft./level)}
\shortability{Targets:}{One Small object per caster level; see text}
\shortability{Duration:}{1 round/level}
\shortability{Saving Throw:}{None}
\shortability{Spell Resistance:}{No}
\smallskip\noindent You imbue inanimate objects with mobility and a semblance of life. Each such animated object then immediately attacks whomever or whatever you initially designate.
An animated object can be of any nonmagical material. You may animate one Small or smaller object or an equivalent number of larger objects per caster level. A Medium object counts as two Small or smaller objects, a Large object as four, a Huge object as eight, a Gargantuan object as sixteen, and a Colossal object as thirty-two. You can change the designated target or targets as a move action, as if directing an active spell.

\smallskip\noindent This spell cannot animate objects carried or worn by a creature.

\smallskip\noindent Animate objects can be made permanent with a permanency spell.

\featname{Animate Plants}
\shortability{Transmutation}{}
\shortability{Level:}{Drd 7, Plant 7}
\shortability{Components:}{V}
\shortability{Casting Time:}{1 standard action}
\shortability{Range:}{Close (25 ft. + 5 ft./2 levels)}
\shortability{Targets:}{One Large plant per three caster levels or all plants within range; see text}
\shortability{Duration:}{1 round/level or 1 hour/level; see text}
\shortability{Saving Throw:}{None}
\shortability{Spell Resistance:}{No}
\smallskip\noindent You imbue inanimate plants with mobility and a semblance of life. Each animated plant then immediately attacks whomever or whatever you initially designate as though it were an animated object of the appropriate size category. You may animate one Large or smaller plant, or an equivalent number of larger plants, per three caster levels. A Huge plant counts as two Large or smaller plants, a Gargantuan plant as four, and a Colossal plant as eight. You can change the designated target or targets as a move action, as if directing an active spell.

\smallskip\noindent Use the statistics for animated objects, except that plants smaller than Large usually don't have hardness.

\smallskip\noindent Animate plants cannot affect plant creatures, nor does it affect nonliving vegetable material.

\shortability{Entangle:}{Alternatively, you may imbue all plants within range with a degree of mobility, which allows them to entwine around creatures in the area. This usage of the spell duplicates the effect of an entangle spell. Spell resistance does not keep creatures from being entangled. This effect lasts 1 hour per caster level.}

\featname{Animate Rope}
\shortability{Transmutation}{}
\shortability{Level:}{Brd 1, Sor/Wiz 1}
\shortability{Components:}{V, S}
\shortability{Casting Time:}{1 standard action}
\shortability{Range:}{Medium (100 ft. + 10 ft./level)}
\shortability{Target:}{One ropelike object, length up to 50 ft. + 5 ft./level; see text}
\shortability{Duration:}{1 round/level}
\shortability{Saving Throw:}{None}
\shortability{Spell Resistance:}{No}
\smallskip\noindent You can animate a nonliving ropelike object. The maximum length assumes a rope with a 1-inch diameter.

\smallskip\noindent Reduce the maximum length by 50\% for every additional inch of thickness, and increase it by 50\% for each reduction of the rope's diameter by half.
The possible commands are ``coil'' (form a neat, coiled stack), ``coil and knot,'' ``loop,'' ``loop and knot,'' ``tie and knot,'' and the opposites of all of the above (``uncoil,'' and so forth). You can give one command each round as a move action, as if directing an active spell.

\smallskip\noindent The rope can enwrap only a creature or an object within 1 foot of it-it does not snake outward-so it must be thrown near the intended target. Doing so requires a successful ranged touch attack roll (range increment 10 feet). A typical 1- inch-diameter hempen rope has 2 hit points, AC 10, and requires a DC 23 Strength check to burst it. The rope does not deal damage, but it can be used as a trip line or to cause a single opponent that fails a Reflex saving throw to become entangled. A creature capable of spellcasting that is bound by this spell must make a DC 15 Concentration check to cast a spell. An entangled creature can slip free with a DC 20 Escape Artist check.

\smallskip\noindent The rope itself and any knots tied in it are not magical.

\smallskip\noindent This spell grants a +2 bonus on any Use Rope checks you make when using the transmuted rope.

\smallskip\noindent The spell cannot animate objects carried or worn by a creature.

\featname{Antilife Shell}
\shortability{Abjuration}{}
\shortability{Level:}{Animal 6, Clr 6, Drd 6}
\shortability{Components:}{V, S, DF}
\shortability{Casting Time:}{1 round}
\shortability{Range:}{10 ft.}
\shortability{Area:}{10-ft.-radius emanation, centered on you}
\shortability{Duration:}{10 min./level (D)}
\shortability{Saving Throw:}{None}
\shortability{Spell Resistance:}{Yes}
\smallskip\noindent You bring into being a mobile, hemispherical energy field that prevents the entrance of most types of living creatures.

\smallskip\noindent The effect hedges out animals, aberrations, dragons, fey, giants, humanoids, magical beasts, monstrous humanoids, oozes, plants, and vermin, but not constructs, elementals, outsiders, or undead.

\smallskip\noindent This spell may be used only defensively, not aggressively. Forcing an abjuration barrier against creatures that the spell keeps at bay collapses the barrier.

\featname{Antimagic Field}
\shortability{Abjuration}{}
\shortability{Level:}{Clr 8, Magic 6, Protection 6, Sor/Wiz 6}
\shortability{Components:}{V, S, M/DF}
\shortability{Casting Time:}{1 standard action}
\shortability{Range:}{10 ft.}
\shortability{Area:}{10-ft.-radius emanation, centered on you}
\shortability{Duration:}{10 min./level (D)}
\shortability{Saving Throw:}{None}
\shortability{Spell Resistance:}{See text}
\smallskip\noindent An invisible barrier surrounds you and moves with you. The space within this barrier is impervious to most magical effects, including spells, spell-like abilities, and supernatural abilities. Likewise, it prevents the functioning of any magic items or spells within its confines.

\smallskip\noindent An antimagic field suppresses any spell or magical effect used within, brought into, or cast into the area, but does not dispel it. Time spent within an antimagic field counts against the suppressed spell's duration.

\smallskip\noindent Summoned creatures of any type and incorporeal undead wink out if they enter an antimagic field. They reappear in the same spot once the field goes away. Time spent winked out counts normally against the duration of the conjuration that is maintaining the creature. If you cast antimagic field in an area occupied by a summoned creature that has spell resistance, you must make a caster level check (1d20 + caster level) against the creature's spell resistance to make it wink out. (The effects of instantaneous conjurations are not affected by an antimagic field because the conjuration itself is no longer in effect, only its result.)

\smallskip\noindent A normal creature can enter the area, as can normal missiles. Furthermore, while a magic sword does not function magically within the area, it is still a sword (and a masterwork sword at that). The spell has no effect on golems and other constructs that are imbued with magic during their creation process and are thereafter self-supporting (unless they have been summoned, in which case they are treated like any other summoned creatures). Elementals, corporeal undead, and outsiders are likewise unaffected unless summoned. These creatures' spell-like or supernatural abilities, however, may be temporarily nullified by the field. Dispel magic does not remove the field.

\smallskip\noindent Two or more antimagic fields sharing any of the same space have no effect on each other. Certain spells, such as wall of force, prismatic sphere, and prismatic wall, remain unaffected by antimagic field (see the individual spell descriptions). Artifacts and deities are unaffected by mortal magic such as this. 

\smallskip\noindent Should a creature be larger than the area enclosed by the barrier, any part of it that lies outside the barrier is unaffected by the field.

\shortability{Arcane Material Component:}{A pinch of powdered iron or iron filings.}

\featname{Antipathy}
\shortability{Enchantment}{(Compulsion) [Mind-Affecting]}
\shortability{Level:}{Drd 9, Sor/Wiz 8}
\shortability{Components:}{V, S, M/DF}
\shortability{Casting Time:}{1 hour}
\shortability{Range:}{Close (25 ft. + 5 ft./2 levels)}
\shortability{Target:}{One location (up to a 10-ft. cube/level) or one object}
\shortability{Duration:}{2 hours/level (D)}
\shortability{Saving Throw:}{Will partial}
\shortability{Spell Resistance:}{Yes}
\smallskip\noindent You cause an object or location to emanate magical vibrations that repel either a specific kind of intelligent creature or creatures of a particular alignment, as defined by you. The kind of creature to be affected must be named specifically. A creature subtype is not specific enough. Likewise, the specific alignment to be repelled must be named.

\smallskip\noindent Creatures of the designated kind or alignment feel an overpowering urge to leave the area or to avoid the affected item.
A compulsion forces them to abandon the area or item, shunning it and never willingly returning to it while the spell is in effect. A creature that makes a successful saving throw can stay in the area or touch the item but feels uncomfortable doing so. This distracting discomfort reduces the creature's Dexterity score by 4 points.
Antipathy counters and dispels sympathy.

\shortability{Arcane Material Component:}{A lump of alum soaked in vinegar.}

\featname{Antiplant Shell}
\shortability{Abjuration}{}
\shortability{Level:}{Drd 4}
\shortability{Components:}{V, S, DF}
\shortability{Casting Time:}{1 standard action}
\shortability{Range:}{10 ft.}
\shortability{Area:}{10-ft.-radius emanation, centered on you}
\shortability{Duration:}{10 min./level (D)}
\shortability{Saving Throw:}{None}
\shortability{Spell Resistance:}{Yes}
\smallskip\noindent The antiplant shell spell creates an invisible, mobile barrier that keeps all creatures within the shell protected from attacks by plant creatures or animated plants. As with many abjuration spells, forcing the barrier against creatures that the spell keeps at bay strains and collapses the field.

\featname{Arcane Eye}
\shortability{Divination}{(Scrying)}
\shortability{Level:}{Sor/Wiz 4}
\shortability{Components:}{V, S, M}
\shortability{Casting Time:}{10 minutes}
\shortability{Range:}{Unlimited}
\shortability{Effect:}{Magical sensor}
\shortability{Duration:}{1 min./level (D)}
\shortability{Saving Throw:}{None}
\shortability{Spell Resistance:}{No}
\smallskip\noindent You create an invisible magical sensor that sends you visual information. You can create the arcane eye at any point you can see, but it can then travel outside your line of sight without hindrance. An arcane eye travels at 30 feet per round (300 feet per minute) if viewing an area ahead as a human would (primarily looking at the floor) or 10 feet per round (100 feet per minute) if examining the ceiling and walls as well as the floor ahead. It sees exactly as you would see if you were there.

\smallskip\noindent The eye can travel in any direction as long as the spell lasts. Solid barriers block its passage, but it can pass through a hole or space as small as 1 inch in diameter. The eye can't enter another plane of existence, even through a gate or similar magical portal.

\smallskip\noindent You must concentrate to use an arcane eye. If you do not concentrate, the eye is inert until you again concentrate.

\shortability{Material Component:}{A bit of bat fur.}

\featname{Arcane Lock}
\shortability{Abjuration}{}
\shortability{Level:}{Sor/Wiz 2}
\shortability{Components:}{V, S, M}
\shortability{Casting Time:}{1 standard action}
\shortability{Range:}{Touch}
\shortability{Target:}{The door, chest, or portal touched, up to 30 sq. ft./level in size}
\shortability{Duration:}{Permanent}
\shortability{Saving Throw:}{None}
\shortability{Spell Resistance:}{No}
\smallskip\noindent An arcane lock spell cast upon a door, chest, or portal magically locks it. You can freely pass your own arcane lock without affecting it; otherwise, a door or object secured with this spell can be opened only by breaking in or with a successful dispel magic or knock spell. Add 10 to the normal DC to break open a door or portal affected by this spell. (A knock spell does not remove an arcane lock; it only suppresses the effect for 10 minutes.)

\shortability{Material Component:}{Gold dust worth 25 gp.}

\featname{Arcane Mark}
\shortability{Universal}{}
\shortability{Level:}{Sor/Wiz 0}
\shortability{Components:}{V, S}
\shortability{Casting Time:}{1 standard action}
\shortability{Range:}{0 ft.}
\shortability{Effect:}{One personal rune or mark, all of which must fit within 1 sq. ft.}
\shortability{Duration:}{Permanent}
\shortability{Saving Throw:}{None}
\shortability{Spell Resistance:}{No}
\smallskip\noindent This spell allows you to inscribe your personal rune or mark, which can consist of no more than six characters. The writing can be visible or invisible. An arcane mark spell enables you to etch the rune upon any substance without harm to the material upon which it is placed. If an invisible mark is made, a detect magic spell causes it to glow and be visible, though not necessarily understandable.

\smallskip\noindent See invisibility, true seeing, a gem of seeing, or a robe of eyes likewise allows the user to see an invisible arcane mark. A read magic spell reveals the words, if any. The mark cannot be dispelled, but it can be removed by the caster or by an erase spell.

\smallskip\noindent If an arcane mark is placed on a living being, normal wear gradually causes the effect to fade in about a month.
Arcane mark must be cast on an object prior to casting instant summons on the same object (see that spell description for details).

\featname{Arcane Sight}
\shortability{Divination}{}
\shortability{Level:}{Sor/Wiz 3}
\shortability{Components:}{V, S}
\shortability{Casting Time:}{1 standard action}
\shortability{Range:}{Personal}
\shortability{Target:}{You}
\shortability{Duration:}{1 min./level (D)}
\smallskip\noindent This spell makes your eyes glow blue and allows you to see magical auras within 120 feet of you. The effect is similar to that of a detect magic spell, but arcane sight does not require concentration and discerns aura location and power more quickly.

\smallskip\noindent You know the location and power of all magical auras within your sight. An aura's power depends on a spell's functioning level or an item's caster level, as noted in the description of the detect magic spell. If the items or creatures bearing the auras are in line of sight, you can make Spellcraft skill checks to determine the school of magic involved in each. (Make one check per aura; DC 15 + spell level, or 15 + one-half caster level for a nonspell effect.)

\smallskip\noindent If you concentrate on a specific creature within 120 feet of you as a standard action, you can determine whether it has any spellcasting or spell-like abilities, whether these are arcane or divine (spell-like abilities register as arcane), and the strength of the most powerful spell or spell-like ability the creature currently has available for use.

\smallskip\noindent Arcane sight can be made permanent with a permanency spell.

\featname{Arcane Sight, Greater}
\shortability{Divination}{}
\shortability{Level:}{Sor/Wiz 7}

\smallskip\noindent This spell functions like arcane sight, except that you automatically know which spells or magical effects are active upon any individual or object you see.
Greater arcane sight doesn't let you identify magic items.

\smallskip\noindent Unlike arcane sight, this spell cannot be made permanent with a permanency spell.

\featname{Astral Projection}
\shortability{Necromancy}{}
\shortability{Level:}{Clr 9, Sor/Wiz 9, Travel 9}
\shortability{Components:}{V, S, M}
\shortability{Casting Time:}{30 minutes}
\shortability{Range:}{Touch}
\shortability{Targets:}{You plus one additional willing creature touched per two caster levels}
\shortability{Duration:}{See text}
\shortability{Saving Throw:}{None}
\shortability{Spell Resistance:}{Yes}
\smallskip\noindent By freeing your spirit from your physical body, this spell allows you to project an astral body onto another plane altogether.

\smallskip\noindent You can bring the astral forms of other willing creatures with you, provided that these subjects are linked in a circle with you at the time of the casting. These fellow travelers are dependent upon you and must accompany you at all times. If something happens to you during the journey, your companions are stranded wherever you left them.

\smallskip\noindent You project your astral self onto the Astral Plane, leaving your physical body behind on the Material Plane in a state of suspended animation. The spell projects an astral copy of you and all you wear or carry onto the Astral Plane. Since the Astral Plane touches upon other planes, you can travel astrally to any of these other planes as you will. To enter one, you leave the Astral Plane, forming a new physical body (and equipment) on the plane of existence you have chosen to enter.

\smallskip\noindent While you are on the Astral Plane, your astral body is connected at all times to your physical body by a silvery cord. If the cord is broken, you are killed, astrally and physically. Luckily, very few things can destroy a silver cord. When a second body is formed on a different plane, the incorporeal silvery cord remains invisibly attached to the new body. If the second body or the astral form is slain, the cord simply returns to your body where it rests on the Material Plane, thereby reviving it from its state of suspended animation. Although astral projections are able to function on the Astral Plane, their actions affect only creatures existing on the Astral Plane; a physical body must be materialized on other planes.

\smallskip\noindent You and your companions may travel through the Astral Plane indefinitely. Your bodies simply wait behind in a state of suspended animation until you choose to return your spirits to them. The spell lasts until you desire to end it, or until it is terminated by some outside means, such as dispel magic cast upon either the physical body or the astral form, the breaking of the silver cord, or the destruction of your body back on the Material Plane (which kills you).

\shortability{Material Component:}{A jacinth worth at least 1,000 gp, plus a silver bar worth 5 gp for each person to be affected.}

\featname{Atonement}
\shortability{Abjuration}{}
\shortability{Level:}{Clr 5, Drd 5}
\shortability{Components:}{V, S, M, F, DF, XP}
\shortability{Casting Time:}{1 hour}
\shortability{Range:}{Touch}
\shortability{Target:}{Living creature touched}
\shortability{Duration:}{Instantaneous}
\shortability{Saving Throw:}{None}
\shortability{Spell Resistance:}{Yes}
\smallskip\noindent This spell removes the burden of evil acts or misdeeds from the subject. The creature seeking atonement must be truly repentant and desirous of setting right its misdeeds. If the atoning creature committed the evil act unwittingly or under some form of compulsion, atonement operates normally at no cost to you. However, in the case of a creature atoning for deliberate misdeeds and acts of a knowing and willful nature, you must intercede with your deity (requiring you to expend 500 XP) in order to expunge the subject's burden. Many casters first assign a subject of this sort a quest (see geas/quest) or similar penance to determine whether the creature is truly contrite before casting the atonement spell on its behalf.

\smallskip\noindent Atonement may be cast for one of several purposes, depending on the version selected.

\shortability{Reverse Magical Alignment Change:}{If a creature has had its alignment magically changed, atonement returns its alignment to its original status at no cost in experience points.}
\shortability{Restore Class:}{A paladin who has lost her class features due to committing an evil act may have her paladinhood restored to her by this spell.}
\shortability{Restore Cleric or Druid Spell Powers:}{A cleric or druid who has lost the ability to cast spells by incurring the anger of his or her deity may regain that ability by seeking atonement from another cleric of the same deity or another druid. If the transgression was intentional, the casting cleric loses 500 XP for his intercession. If the transgression was unintentional, he does not lose XP.}
\shortability{Redemption or Temptation:}{You may cast this spell upon a creature of an opposing alignment in order to offer it a chance to change its alignment to match yours. The prospective subject must be present for the entire casting process. Upon completion of the spell, the subject freely chooses whether it retains its original alignment or acquiesces to your offer and changes to your alignment. No duress, compulsion, or magical influence can force the subject to take advantage of the opportunity offered if it is unwilling to abandon its old alignment. This use of the spell does not work on outsiders or any creature incapable of changing its alignment naturally.

\smallskip\noindent Though the spell description refers to evil acts, atonement can also be used on any creature that has performed acts against its alignment, whether those acts are evil, good, chaotic, or lawful.}

\shortability{Note:}{Normally, changing alignment is up to the player. This use of atonement simply offers a believable way for a character to change his or her alignment drastically, suddenly, and definitively.}
\shortability{Material Component:}{Burning incense.}
\shortability{Focus:}{In addition to your holy symbol or normal divine focus, you need a set of prayer beads (or other prayer device, such as a prayer wheel or prayer book) worth at least 500 gp.}

\shortability{XP Cost:}{When cast for the benefit of a creature whose guilt was the result of deliberate acts, the cost to you is 500 XP per casting (see above).}

\featname{Augury}
\shortability{Divination}{}
\shortability{Level:}{Clr 2}
\shortability{Components:}{V, S, M, F}
\shortability{Casting Time:}{1 minute}
\shortability{Range:}{Personal}
\shortability{Target:}{You}
\shortability{Duration:}{Instantaneous}
\smallskip\noindent An augury can tell you whether a particular action will bring good or bad results for you in the immediate future.

\smallskip\noindent The base chance for receiving a meaningful reply is 70\% + 1\% per caster level, to a maximum of 90\%; this roll is made secretly. A question may be so straightforward that a successful result is automatic, or so vague as to have no chance of success. If the augury succeeds, you get one of four results:

\listone
	\item Weal (if the action will probably bring good results).
  \item Woe (for bad results).
  \item Weal and woe (for both).
  \item Nothing (for actions that don't have especially good or bad results).
\end{list}

\smallskip\noindent If the spell fails, you get the ``nothing'' result. A cleric who gets the ``nothing'' result has no way to tell whether it was the consequence of a failed or successful augury.

\smallskip\noindent The augury can see into the future only about half an hour, so anything that might happen after that does not affect the result. Thus, the result might not take into account the long-term consequences of a contemplated action. All auguries cast by the same person about the same topic use the same dice result as the first casting.

\shortability{Material Component:}{Incense worth at least 25 gp.}
\shortability{Focus:}{A set of marked sticks, bones, or similar tokens of at least 25 gp value.}

\featname{Awaken}
\shortability{Transmutation}{}
\shortability{Level:}{Drd 5}
\shortability{Components:}{V, S, DF, XP}
\shortability{Casting Time:}{24 hours}
\shortability{Range:}{Touch}
\shortability{Target:}{Animal or tree touched}
\shortability{Duration:}{Instantaneous}
\shortability{Saving Throw:}{Will negates}
\shortability{Spell Resistance:}{Yes}
\smallskip\noindent You awaken a tree or animal to humanlike sentience. To succeed, you must make a Will save (DC 10 + the animal's current HD, or the HD the tree will have once awakened).

\smallskip\noindent The awakened animal or tree is friendly toward you. You have no special empathy or connection with a creature you awaken, although it serves you in specific tasks or endeavors if you communicate your desires to it.

\smallskip\noindent An awakened tree has characteristics as if it were an animated object, except that it gains the plant type and its Intelligence, Wisdom, and Charisma scores are each 3d6. An awakened plant gains the ability to move its limbs, roots, vines, creepers, and so forth, and it has senses similar to a human's. An awakened animal gets 3d6 Intelligence, +1d3 Charisma, and +2 HD. Its type becomes magical beast (augmented animal). An awakened animal can't serve as an animal companion, familiar, or special mount. An awakened tree or animal can speak one language that you know, plus one additional language that you know per point of Intelligence bonus (if any).

\shortability{XP Cost:}{250 XP.}

\featname{Baleful Polymorph}
\shortability{Transmutation}{}
\shortability{Level:}{Drd 5, Sor/Wiz 5}
\shortability{Components:}{V, S}
\shortability{Casting Time:}{1 standard action}
\shortability{Range:}{Close (25 ft. + 5 ft./2 levels)}
\shortability{Target:}{One creature}
\shortability{Duration:}{Permanent}
\shortability{Saving Throw:}{Fortitude negates, Will partial; see text}
\shortability{Spell Resistance:}{Yes}
\smallskip\noindent As polymorph, except that you change the subject into a Small or smaller animal of no more than 1 HD. If the new form would prove fatal to the creature the subject gets a +4 bonus on the save.

\smallskip\noindent If the spell succeeds, the subject must also make a Will save. If this second save fails, the creature loses its extraordinary, supernatural, and spell-like abilities, loses its ability to cast spells (if it had the ability), and gains the alignment, special abilities, and Intelligence, Wisdom, and Charisma scores of its new form in place of its own. It still retains its class and level (or HD), as well as all benefits deriving therefrom (such as base attack bonus, base save bonuses, and hit points). It retains any class features (other than spellcasting) that aren't extraordinary, supernatural, or spell-like abilities.

\smallskip\noindent Incorporeal or gaseous creatures are immune to being polymorphed, and a creature with the shapechanger subtype can revert to its natural form as a standard action.

\featname{Bane}
\shortability{Enchantment}{(Compulsion) [Fear, Mind-Affecting]}
\shortability{Level:}{Clr 1}
\shortability{Components:}{V, S, DF}
\shortability{Casting Time:}{1 standard action}
\shortability{Range:}{50 ft.}
\shortability{Area:}{All enemies within 50 ft.}
\shortability{Duration:}{1 min./level}
\shortability{Saving Throw}{Will negates}
\shortability{Spell Resistance:}{Yes}
\smallskip\noindent Bane fills your enemies with fear and doubt. Each affected creature takes a -1 penalty on attack rolls and a -1 penalty on saving throws against fear effects. Bane counters and dispels bless.

\featname{Banishment}
\shortability{Abjuration}{}
\shortability{Level:}{Clr 6, Sor/Wiz 7}
\shortability{Components:}{V, S, F}
\shortability{Casting Time:}{1 standard action}
\shortability{Range:}{Close (25 ft. + 5 ft./2 levels)}
\shortability{Targets:}{One or more extraplanar creatures, no two of which can be more than 30 ft. apart}
\shortability{Duration:}{Instantaneous}
\shortability{Saving Throw:}{Will negates}
\shortability{Spell Resistance:}{Yes}
\smallskip\noindent A banishment spell is a more powerful version of the dismissal spell. It enables you to force extraplanar creatures out of your home plane. As many as 2 Hit Dice of creatures per caster level can be banished.

\smallskip\noindent You can improve the spell's chance of success by presenting at least one object or substance that the target hates, fears, or otherwise opposes. For each such object or substance, you gain a +1 bonus on your caster level check to overcome the target's spell resistance (if any), the saving throw DC increases by 2. 

\smallskip\noindent Certain rare items might work twice as well as a normal item for the purpose of the bonuses (each providing a +2 bonus on the caster level check against spell resistance and increasing the save DC by 4).

\shortability{Arcane Focus:}{Any item that is distasteful to the subject (optional, see above).}

\featname{Barkskin}
\shortability{Transmutation}{}
\shortability{Level:}{Drd 2, Rgr 2, Plant 2}
\shortability{Components:}{V, S, DF}
\shortability{Casting Time:}{1 standard action}
\shortability{Range:}{Touch}
\shortability{Target:}{Living creature touched}
\shortability{Duration:}{10 min./level}
\shortability{Saving Throw:}{None}
\shortability{Spell Resistance:}{Yes (harmless)}
\smallskip\noindent Barkskin toughens a creature's skin. The effect grants a +2 enhancement bonus to the creature's existing natural armor bonus. This enhancement bonus increases by 1 for every three caster levels above 3rd, to a maximum of +5 at caster level 12th.

\smallskip\noindent The enhancement bonus provided by barkskin stacks with the target's natural armor bonus, but not with other enhancement bonuses to natural armor. A creature without natural armor has an effective natural armor bonus of +0.

\featname{Bear's Endurance}
\shortability{Transmutation}{}
\shortability{Level:}{Clr 2, Drd 2, Rgr 2, Sor/Wiz 2}
\shortability{Components:}{V, S, DF}
\shortability{Casting Time:}{1 standard action}
\shortability{Range:}{Touch}
\shortability{Target:}{Creature touched}
\shortability{Duration:}{1 min./level}
\shortability{Saving Throw}{Will negates (harmless)}
\shortability{Spell Resistance:}{Yes}
\smallskip\noindent The affected creature gains greater vitality and stamina. The spell grants the subject a +4 enhancement bonus to Constitution, which adds the usual benefits to hit points, Fortitude saves, Constitution checks, and so forth.

\smallskip\noindent Hit points gained by a temporary increase in Constitution score are not temporary hit points. They go away when the subject's Constitution drops back to normal. They are not lost first as temporary hit points are.

\featname{Bear's Endurance, Mass}
\shortability{Transmutation}{}
\shortability{Level:}{Clr 6, Drd 6, Sor/Wiz 6}
\shortability{Range:}{Close (25 ft. + 5 ft./2 levels)}
\shortability{Targets:}{One creature/level, no two of which can be more than 30 ft. apart}
\smallskip\noindent Mass bear's endurance works like bear 's endurance, except that it affects multiple creatures.

\featname{Bestow Curse}
\shortability{Necromancy}{}
\shortability{Level:}{Clr 3, Sor/Wiz 4}
\shortability{Components:}{V, S}
\shortability{Casting Time:}{1 standard action}
\shortability{Range:}{Touch}
\shortability{Target:}{Creature touched}
\shortability{Duration:}{Permanent}
\shortability{Saving Throw:}{Will negates}
\shortability{Spell Resistance:}{Yes}
\smallskip\noindent You place a curse on the subject. Choose one of the following three effects.

\listone
  \item -6 decrease to an ability score (minimum 1).
  \item -4 penalty on attack rolls, saves, ability checks, and skill checks.
  \item Each turn, the target has a 50\% chance to act normally; otherwise, it takes no action.
\end{list}

\smallskip\noindent You may also invent your own curse, but it should be no more powerful than those described above. The curse bestowed by this spell cannot be dispelled, but it can be removed with a break enchantment, limited wish, miracle, remove curse, or wish spell.

\smallskip\noindent Bestow curse counters remove curse.

\featname{Binding}
\shortability{Enchantment}{(Compulsion) [Mind-Affecting]}
\shortability{Level:}{Sor/Wiz 8}
\shortability{Components:}{V, S, M}
\shortability{Casting Time:}{One minute}
\shortability{Range:}{Close (25 ft. + 5 ft./2 levels)}
\shortability{Target:}{One living creature}
\shortability{Duration:}{See text (D)}
\shortability{Saving Throw:}{Will negates; see text}
\shortability{Spell Resistance:}{Yes}
\smallskip\noindent A binding spell creates a magical restraint to hold a creature. The target gets an initial saving throw only if its Hit Dice equal at least one-half your caster level.

\smallskip\noindent You may have as many as six assistants help you with the spell. For each assistant who casts suggestion, your caster level for this casting of binding increases by 1. For each assistant who casts dominate animal, dominate person, or dominate monster, your caster level for this casting of binding increases by a number equal to one-third of that assistant's level, provided that the spell's target is appropriate for a binding spell. Since the assistants' spells are cast simply to improve your caster level for the purpose of the binding spell, saving throws and spell resistance against the assistants' spells are irrelevant. Your caster level determines whether the target gets an initial Will saving throw and how long the binding lasts. All binding spells are dismissible.

\smallskip\noindent Regardless of the version of binding you cast, you can specify triggering conditions that end the spell and release the creature whenever they occur. These triggers can be as simple or elaborate as you desire, but the condition must be reasonable and have a likelihood of coming to pass. The conditions can be based on a creature's name, identity, or alignment but otherwise must be based on observable actions or qualities. Intangibles such as level, class, Hit Dice, or hit points don't qualify. Once the spell is cast, its triggering conditions cannot be changed. Setting a release condition increases the save DC (assuming a saving throw is allowed) by 2.

\smallskip\noindent If you are casting any of the first three versions of binding (those with limited durations), you may cast additional binding spells to prolong the effect, since the durations overlap. If you do so, the target gets a saving throw at the end of the first spell's duration, even if your caster level was high enough to disallow an initial saving throw. If the creature succeeds on this save, all the binding spells it has received are broken.

\smallskip\noindent The binding spell has six versions. Choose one of the following versions when you cast the spell.

\shortability{Chaining:}{The subject is confined by restraints that generate an antipathy spell affecting all creatures who approach the subject, except you. The duration is one year per caster level. The subject of this form of binding is confined to the spot it occupied when it received the spell.}
\shortability{Slumber:}{This version causes the subject to become comatose for as long as one year per caster level. The subject does not need to eat or drink while slumbering, nor does it age. This form of binding is more difficult to cast than chaining, making it slightly easier to resist. Reduce the spell's save DC by 1.}
\shortability{Bound Slumber:}{This combination of chaining and slumber lasts for as long as one month per caster level. Reduce the save DC by 2.}
\shortability{Hedged Prison:}{The subject is transported to or otherwise brought within a confined area from which it cannot wander by any means. The effect is permanent. Reduce the save DC by 3.}
\shortability{Metamorphosis:}{The subject assumes gaseous form, except for its head or face. It is held harmless in a jar or other container, which may be transparent if you so choose. The creature remains aware of its surroundings and can speak, but it cannot leave the container, attack, or use any of its powers or abilities. The binding is permanent. The subject does not need to breathe, eat, or drink while metamorphosed, nor does it age. Reduce the save DC by 4.}
\shortability{Minimus Containment:}{The subject is shrunk to a height of 1 inch or even less and held within some gem, jar, or similar object. The binding is permanent. The subject does not need to breathe, eat, or drink while contained, nor does it age. Reduce the save DC by 4.}

\smallskip\noindent You can't dispel a binding spell with dispel magic or a similar effect, though an antimagic field or Mage's disjunction affects it normally. A bound extraplanar creature cannot be sent back to its home plane due to dismissal, banishment, or a similar effect.

\shortability{Components:}{The components for a binding spell vary according to the version of the spell, but they always include a continuous chanting utterance read from the scroll or spellbook page containing the spell, somatic gestures, and materials appropriate to the form of binding used. These components can include such items as miniature chains of special metals, soporific herbs of the rarest sort (for slumber bindings), a bell jar of the finest crystal, and the like.} In addition to the specially made props suited to the specific type of binding (cost 500 gp), the spell requires opals worth at least 500 gp for each HD of the target and a vellum depiction or carved statuette of the subject to be captured.

\featname{Black Tentacles}
\shortability{Conjuration}{(Creation)}
\shortability{Level:}{Sor/Wiz 4}
\shortability{Components:}{V, S, M}
\shortability{Casting Time:}{1 standard action}
\shortability{Range:}{Medium (100 ft. + 10 ft./level)}
\shortability{Area:}{20-ft.-radius spread}
\shortability{Duration:}{1 round/level (D)}
\shortability{Saving Throw:}{None}
\shortability{Spell Resistance:}{No}
\smallskip\noindent This spell conjures a field of rubbery black tentacles, each 10 feet long. These waving members seem to spring forth from the earth, floor, or whatever surface is underfoot-including water. They grasp and entwine around creatures that enter the area, holding them fast and crushing them with great strength.

\smallskip\noindent Every creature within the area of the spell must make a grapple check, opposed by the grapple check of the tentacles. Treat the tentacles attacking a particular target as a Large creature with a base attack bonus equal to your caster level and a Strength score of 19. Thus, its grapple check modifier is equal to your caster level +8. The tentacles are immune to all types of damage.

\smallskip\noindent Once the tentacles grapple an opponent, they may make a grapple check each round on your turn to deal 1d6+4 points of bludgeoning damage. The tentacles continue to crush the opponent until the spell ends or the opponent escapes.

\smallskip\noindent Any creature that enters the area of the spell is immediately attacked by the tentacles. Even creatures who aren't grappling with the tentacles may move through the area at only half normal speed.

\shortability{Material Component:}{A piece of tentacle from a giant octopus or a giant squid.}

\featname{Blade Barrier}
\shortability{Evocation}{[Force]}
\shortability{Level:}{Clr 6, Good 6, War 6}
\shortability{Components:}{V, S}
\shortability{Casting Time:}{1 standard action}
\shortability{Range:}{Medium (100 ft. + 10 ft./level)}
\shortability{Effect:}{Wall of whirling blades up to 20 ft. long/ level, or a ringed wall of whirling blades with a radius of up to 5 ft. per two levels; either form 20 ft. high}
\shortability{Duration:}{1 min./level (D)}
\shortability{Saving Throw:}{Reflex half or Reflex negates; see text}
\shortability{Spell Resistance:}{Yes}
\smallskip\noindent An immobile, vertical curtain of whirling blades shaped of pure force springs into existence. Any creature passing through the wall takes 1d6 points of damage per caster level (maximum 15d6), with a Reflex save for half damage.

\smallskip\noindent If you evoke the barrier so that it appears where creatures are, each creature takes damage as if passing through the wall. Each such creature can avoid the wall (ending up on the side of its choice) and thus take no damage by making a successful Reflex save.

\smallskip\noindent A blade barrier provides cover (+4 bonus to AC, +2 bonus on Reflex saves) against attacks made through it.

\featname{Blasphemy}
\shortability{Evocation}{[Evil, Sonic]}
\shortability{Level:}{Clr 7, Evil 7}
\shortability{Components:}{V}
\shortability{Casting Time:}{1 standard action}
\shortability{Range:}{40 ft.}
\shortability{Area:}{Nonevil creatures in a 40-ft.-radius spread centered on you}
\shortability{Duration:}{Instantaneous}
\shortability{Saving Throw:}{None or Will negates; see text}
\shortability{Spell Resistance:}{Yes}
\smallskip\noindent Any nonevil creature within the area of a blasphemy spell suffers the following ill effects.

\begin{tabular}[h!]{l|p{1.7in}}
HD & Effect \\ \hline
Equal to caster level & Dazed \\
Up to caster level -1 & Weakened, dazed \\
Up to caster level -5 & Paralyzed, weakened, dazed \\
Up to caster level -10 & Killed, paralyzed, weakened, dazed \\
\end{tabular}

\smallskip\noindent The effects are cumulative and concurrent.

\smallskip\noindent No saving throw is allowed against these effects.

\shortability{Dazed:}{The creature can take no actions for 1 round, though it defends itself normally.}
\shortability{Weakened:}{The creature's Strength score decreases by 2d6 points for 2d4 rounds.}
\shortability{Paralyzed:}{The creature is paralyzed and helpless for 1d10 minutes.}
\shortability{Killed:}{Living creatures die. Undead creatures are destroyed.}

\smallskip\noindent Furthermore, if you are on your home plane when you cast this spell, nonevil extraplanar creatures within the area are instantly banished back to their home planes. Creatures so banished cannot return for at least 24 hours. This effect takes place regardless of whether the creatures hear the blasphemy. The banishment effect allows a Will save (at a -4 penalty) to negate. Creatures whose Hit Dice exceed your caster level are unaffected by blasphemy.

\featname{Bless}
\shortability{Enchantment}{(Compulsion) [Mind-Affecting]}
\shortability{Level:}{Clr 1, Pal 1}
\shortability{Components:}{V, S, DF}
\shortability{Casting Time:}{1 standard action}
\shortability{Range:}{50 ft.}
\shortability{Area:}{The caster and all allies within a 50-ft. burst, centered on the caster}
\shortability{Duration:}{1 min./level}
\shortability{Saving Throw:}{None}
\shortability{Spell Resistance:}{Yes (harmless)}
\smallskip\noindent Bless fills your allies with courage. Each ally gains a +1 morale bonus on attack rolls and on saving throws against fear effects.

\smallskip\noindent Bless counters and dispels bane.

\featname{Bless Water}
\shortability{Transmutation}{[Good]}
\shortability{Level:}{Clr 1, Pal 1}
\shortability{Components:}{V, S, M}
\shortability{Casting Time:}{1 minute}
\shortability{Range:}{Touch}
\shortability{Target:}{Flask of water touched}
\shortability{Duration:}{Instantaneous}
\shortability{Saving Throw:}{Will negates (object)}
\shortability{Spell Resistance:}{Yes (object)}
\smallskip\noindent This transmutation imbues a flask (1 pint) of water with positive energy, turning it into holy water.

\shortability{Material Component:}{5 pounds of powdered silver (worth 25 gp).}

\featname{Bless Weapon}
\shortability{Transmutation}{}
\shortability{Level:}{Pal 1}
\shortability{Components:}{V, S}
\shortability{Casting Time:}{1 standard action}
\shortability{Range:}{Touch}
\shortability{Target:}{Weapon touched}
\shortability{Duration:}{1 min./level}
\shortability{Saving Throw:}{None}
\shortability{Spell Resistance:}{No}
\smallskip\noindent This transmutation makes a weapon strike true against evil foes. The weapon is treated as having a +1 enhancement bonus for the purpose of bypassing the damage reduction of evil creatures or striking evil incorporeal creatures (though the spell doesn't grant an actual enhancement bonus). The weapon also becomes good, which means it can bypass the damage reduction of certain creatures. (This effect overrides and suppresses any other alignment the weapon might have.) Individual arrows or bolts can be transmuted, but affected projectile weapons (such as bows) don't confer the benefit to the projectiles they shoot.

\smallskip\noindent In addition, all critical hit rolls against evil foes are automatically successful, so every threat is a critical hit. This last effect does not apply to any weapon that already has a magical effect related to critical hits, such as a keen weapon or a vorpal sword.

\featname{Blight}
\shortability{Necromancy}{}
\shortability{Level:}{Drd 4, Sor/Wiz 5}
\shortability{Components:}{V, S, DF}
\shortability{Casting Time:}{1 standard action}
\shortability{Range:}{Touch}
\shortability{Duration:}{Instantaneous}
\shortability{Saving Throw:}{Fortitude half; see text}
\shortability{Spell Resistance:}{Yes}
\smallskip\noindent This spell withers a single plant of any size. An affected plant creature takes 1d6 points of damage per level (maximum 15d6) and may attempt a Fortitude saving throw for half damage. A plant that isn't a creature doesn't receive a save and immediately withers and dies.

\smallskip\noindent This spell has no effect on the soil or surrounding plant life.

\featname{Blindness/Deafness}
\shortability{Necromancy}{}
\shortability{Level:}{Brd 2, Clr 3, Sor/Wiz 2}
\shortability{Components:}{V}
\shortability{Casting Time:}{1 standard action}
\shortability{Range:}{Medium (100 ft. + 10 ft./level)}
\shortability{Target:}{One living creature}
\shortability{Duration:}{Permanent (D)}
\shortability{Saving Throw:}{Fortitude negates}
\shortability{Spell Resistance:}{Yes}
\smallskip\noindent You call upon the powers of unlife to render the subject blinded or deafened, as you choose.

\featname{Blink}
\shortability{Transmutation}{}
\shortability{Level:}{Brd 3, Sor/Wiz 3}
\shortability{Components:}{V, S}
\shortability{Casting Time:}{1 standard action}
\shortability{Range:}{Personal}
\shortability{Target:}{You}
\shortability{Duration:}{1 round/level (D)}
\smallskip\noindent You ``blink'' back and forth between the Material Plane and the Ethereal Plane. You look as though you're winking in and out of reality very quickly and at random.
Blinking has several effects, as follows.

\smallskip\noindent Physical attacks against you have a 50\% miss chance, and the Blind-Fight feat doesn't help opponents, since you're ethereal and not merely invisible. If the attack is capable of striking ethereal creatures, the miss chance is only 20\% (for concealment).

\smallskip\noindent If the attacker can see invisible creatures, the miss chance is also only 20\%. (For an attacker who can both see and strike ethereal creatures, there is no miss chance.) Likewise, your own attacks have a 20\% miss chance, since you sometimes go ethereal just as you are about to strike.

\smallskip\noindent Any individually targeted spell has a 50\% chance to fail against you while you're blinking unless your attacker can target invisible, ethereal creatures. Your own spells have a 20\% chance to activate just as you go ethereal, in which case they typically do not affect the Material Plane.

\smallskip\noindent While blinking, you take only half damage from area attacks (but full damage from those that extend onto the Ethereal Plane). You strike as an invisible creature (with a +2 bonus on attack rolls), denying your target any Dexterity bonus to AC.

\smallskip\noindent You take only half damage from falling, since you fall only while you are material.

\smallskip\noindent While blinking, you can step through (but not see through) solid objects. For each 5 feet of solid material you walk through, there is a 50\% chance that you become material. If this occurs, you are shunted off to the nearest open space and take 1d6 points of damage per 5 feet so traveled. You can move at only three-quarters speed (because movement on the Ethereal Plane is at half speed, and you spend about half your time there and half your time material.)

\smallskip\noindent Since you spend about half your time on the Ethereal Plane, you can see and even attack ethereal creatures. You interact with ethereal creatures roughly the same way you interact with material ones.

\smallskip\noindent An ethereal creature is invisible, incorporeal, and capable of moving in any direction, even up or down. As an incorporeal creature, you can move through solid objects, including living creatures.

\smallskip\noindent An ethereal creature can see and hear the Material Plane, but everything looks gray and insubstantial. Sight and hearing on the Material Plane are limited to 60 feet.
Force effects and abjurations affect you normally. Their effects extend onto the Ethereal Plane from the Material Plane, but not vice versa. An ethereal creature can't attack material creatures, and spells you cast while ethereal affect only other ethereal things. Certain material creatures or objects have attacks or effects that work on the Ethereal Plane. Treat other ethereal creatures and objects as material.

\featname{Blur}
\shortability{Illusion}{(Glamer)}
\shortability{Level:}{Brd 2,Sor/Wiz 2}
\shortability{Components:}{V}
\shortability{Casting Time:}{1 standard action}
\shortability{Range:}{Touch}
\shortability{Target:}{Creature touched}
\shortability{Duration:}{1 min./level (D)}
\shortability{Saving Throw:}{Will negates (harmless)}
\shortability{Spell Resistance:}{Yes (harmless)}
\smallskip\noindent The subject's outline appears blurred, shifting and wavering. This distortion grants the subject concealment (20\% miss chance).

\smallskip\noindent A see invisibility spell does not counteract the blur effect, but a true seeing spell does.

\smallskip\noindent Opponents that cannot see the subject ignore the spell's effect (though fighting an unseen opponent carries penalties of its own).

\featname{Break Enchantment}
\shortability{Abjuration}{}
\shortability{Level:}{Brd 4, Clr 5, Luck 5, Pal 4, Sor/Wiz 5}
\shortability{Components:}{V, S}
\shortability{Casting Time:}{1 minute}
\shortability{Range:}{Close (25 ft. + 5 ft./2 levels)}
\shortability{Targets:}{Up to one creature per level, all within 30 ft. of each other}
\shortability{Duration:}{Instantaneous}
\shortability{Saving Throw:}{See text}
\shortability{Spell Resistance:}{No}
\smallskip\noindent This spell frees victims from enchantments, transmutations, and curses. Break enchantment can reverse even an instantaneous effect. For each such effect, you make a caster level check (1d20 + caster level, maximum +15) against a DC of 11 + caster level of the effect. Success means that the creature is free of the spell, curse, or effect. For a cursed magic item, the DC is 25.

\smallskip\noindent If the spell is one that cannot be dispelled by dispel magic, break enchantment works only if that spell is 5th level or lower. 

\smallskip\noindent If the effect comes from some permanent magic item break enchantment does not remove the curse from the item, but it does frees the victim from the item's effects. 

\featname{Bull's Strength}
\shortability{Transmutation}{}
\shortability{Level:}{Clr 2, Drd 2, Pal 2, Sor/Wiz 2, Strength 2}
\shortability{Components:}{V, S, M/DF}
\shortability{Casting Time:}{1 standard action}
\shortability{Range:}{Touch}
\shortability{Target:}{Creature touched}
\shortability{Duration:}{1 min./level}
\shortability{Saving Throw}{Will negates (harmless)}
\shortability{Spell Resistance:}{Yes (harmless)}
\smallskip\noindent The subject becomes stronger. The spell grants a +4 enhancement bonus to Strength, adding the usual benefits to melee attack rolls, melee damage rolls, and other uses of the Strength modifier.

\shortability{Arcane Material Component:}{A few hairs, or a pinch of dung, from a bull.}

\featname{Bull's Strength, Mass}
\shortability{Transmutation}{}
\shortability{Level:}{Clr 6, Drd 6, Sor/Wiz 6}
\shortability{Range:}{Close (25 ft. + 5 ft./2 levels)}
\shortability{Targets:}{One creature/level, no two of which can be more than 30 ft. apart}
\smallskip\noindent This spell functions like bull's strength, except that it affects multiple creatures.

\featname{Burning Hands}
\shortability{Evocation}{[Fire]}
\shortability{Level:}{Fire 1, Sor/Wiz 1}
\shortability{Components:}{V, S}
\shortability{Casting Time:}{1 standard action}
\shortability{Range:}{15 ft.}
\shortability{Area:}{Cone-shaped burst}
\shortability{Duration:}{Instantaneous}
\shortability{Saving Throw:}{Reflex half}
\shortability{Spell Resistance:}{Yes}
\smallskip\noindent A cone of searing flame shoots from your fingertips. Any creature in the area of the flames takes 1d4 points of fire damage per caster level (maximum 5d4). Flammable materials burn if the flames touch them. A character can extinguish burning items as a full-round action.

\featname{Call Lightning}
\shortability{Evocation}{[Electricity]}
\shortability{Level:{Drd 3}}
\shortability{Components:}{V, S}
\shortability{Casting Time:}{1 round}
\shortability{Range:}{Medium (100 ft. + 10 ft./level)}
\shortability{Effect:}{One or more 30-ft.-long vertical lines of lightning}
\shortability{Duration:}{1 min./level}
\shortability{Saving Throw:}{Reflex half}
\shortability{Spell Resistance:}{Yes}
\smallskip\noindent Immediately upon completion of the spell, and once per round thereafter, you may call down a 5-foot-wide, 30-foot-long, vertical bolt of lightning that deals 3d6 points of electricity damage. The bolt of lightning flashes down in a vertical stroke at whatever target point you choose within the spell's range (measured from your position at the time). Any creature in the target square or in the path of the bolt is affected.

\smallskip\noindent You need not call a bolt of lightning immediately; other actions, even spellcasting, can be performed. However, each round after the first you may use a standard action (concentrating on the spell) to call a bolt. You may call a total number of bolts equal to your caster level (maximum 10 bolts).

\smallskip\noindent If you are outdoors and in a stormy area-a rain shower, clouds and wind, hot and cloudy conditions, or even a tornado (including a whirlwind formed by a djinni or an air elemental of at least Large size)-each bolt deals 3d10 points of electricity damage instead of 3d6.

\smallskip\noindent This spell functions indoors or underground but not underwater.

\featname{Call Lightning Storm}
\shortability{Evocation}{[Electricity]}
\shortability{Level:}{Drd 5}
\shortability{Range:}{Long (400 ft. + 40 ft./level)}
\smallskip\noindent This spell functions like call lightning, except that each bolt deals 5d6 points of electricity damage (or 5d10 if created outdoors in a stormy area), and you may call a maximum of 15 bolts.

\featname{Calm Animals}
\shortability{Enchantment}{(Compulsion) [Mind-Affecting]}
\shortability{Level:}{Animal 1, Drd 1, Rgr 1}
\shortability{Components:}{V, S}
\shortability{Casting Time:}{1 standard action}
\shortability{Range:}{Close (25 ft. + 5 ft./2 levels)}
\shortability{Targets:}{Animals within 30 ft. of each other}
\shortability{Duration:}{1 min./level}
\shortability{Saving Throw:}{Will negates; see text}
\shortability{Spell Resistance:}{Yes}
\smallskip\noindent This spell soothes and quiets animals, rendering them docile and harmless. Only ordinary animals (those with Intelligence scores of 1 or 2) can be affected by this spell. All the subjects must be of the same kind, and no two may be more than 30 feet apart. The maximum number of Hit Dice of animals you can affect is equal to 2d4 + caster level. A dire animal or an animal trained to attack or guard is allowed a saving throw; other animals are not.

\smallskip\noindent The affected creatures remain where they are and do not attack or flee. They are not helpless and defend themselves normally if attacked. Any threat breaks the spell on the threatened creatures.

\featname{Calm Emotions}
\shortability{Enchantment}{(Compulsion) [Mind-Affecting]}
\shortability{Level:}{Brd 2, Clr 2, Law 2}
\shortability{Components:}{V, S, DF}
\shortability{Casting Time:}{1 standard action}
\shortability{Range:}{Medium (100 ft. + 10 ft./level)}
\shortability{Area:}{Creatures in a 20-ft.-radius spread}
\shortability{Duration:}{Concentration, up to 1 round/level (D)}
\shortability{Saving Throw:}{Will negates}
\shortability{Spell Resistance:}{Yes}
\smallskip\noindent This spell calms agitated creatures. You have no control over the affected creatures, but calm emotions can stop raging creatures from fighting or joyous ones from reveling. Creatures so affected cannot take violent actions (although they can defend themselves) or do anything destructive. Any aggressive action against or damage dealt to a calmed creature immediately breaks the spell on all calmed creatures.

\smallskip\noindent This spell automatically suppresses (but does not dispel) any morale bonuses granted by spells such as bless, good hope, and rage, as well as negating a bard's ability to inspire courage or a barbarian's rage ability. It also suppresses any fear effects and removes the confused condition from all targets. While the spell lasts, a suppressed spell or effect has no effect. When the calm emotions spell ends, the original spell or effect takes hold of the creature again, provided that its duration has not expired in the meantime.

\featname{Cat's Grace}
\shortability{Transmutation}{}
\shortability{Level:}{Brd 2, Drd 2, Rgr 2, Sor/Wiz 2}
\shortability{Components:}{V, S, M}
\shortability{Casting Time:}{1 standard action}
\shortability{Range:}{Touch}
\shortability{Target:}{Creature touched}
\shortability{Duration:}{1 min./level}
\shortability{Saving Throw:}{Will negates (harmless)}
\shortability{Spell Resistance:}{Yes}
\smallskip\noindent The transmuted creature becomes more graceful, agile, and coordinated. The spell grants a +4 enhancement bonus to Dexterity, adding the usual benefits to AC, Reflex saves, and other uses of the Dexterity modifier.

\shortability{Material Component:}{A pinch of cat fur.}

\featname{Cat's Grace, Mass}
\shortability{Transmutation}{}
\shortability{Level:}{Brd 6, Drd 6, Sor/Wiz 6}
\shortability{Range:}{Close (25 ft. + 5 ft./2 levels)}
\shortability{Targets:}{One creature/level, no two of which can be more than 30 ft. apart}
\smallskip\noindent This spell functions like cat's grace, except that it affects multiple creatures.

\featname{Cause Fear}
\shortability{Necromancy}{[Fear, Mind-Affecting]}
\shortability{Level:}{Brd 1, Clr 1, Death 1, Sor/Wiz 1}
\shortability{Components:}{V, S}
\shortability{Casting Time:}{1 standard action}
\shortability{Range:}{Close (25 ft. + 5 ft./2 levels)}
\shortability{Target:}{One living creature with 5 or fewer HD}
\shortability{Duration:}{1d4 rounds or 1 round; see text}
\shortability{Saving Throw:}{Will partial}
\shortability{Spell Resistance:}{Yes}
\smallskip\noindent The affected creature becomes frightened. If the subject succeeds on a Will save, it is shaken for 1 round. Creatures with 6 or more Hit Dice are immune to this effect.

\smallskip\noindent Cause fear counters and dispels remove fear.

\featname{Chain Lightning}
\shortability{Evocation}{[Electricity]}
\shortability{Level:}{Air 6, Sor/Wiz 6}
\shortability{Components:}{V, S, F}
\shortability{Casting Time:}{1 standard action}
\shortability{Range:}{Long (400 ft. + 40 ft./level)}
\shortability{Targets:}{One primary target, plus one secondary target/level (each of which must be within 30 ft. of the primary target)}
\shortability{Duration:}{Instantaneous}
\shortability{Saving Throw:}{Reflex half}
\shortability{Spell Resistance:}{Yes}
\smallskip\noindent This spell creates an electrical discharge that begins as a single stroke commencing from your fingertips. Unlike lightning bolt, chain lightning strikes one object or creature initially, then arcs to other targets.

\smallskip\noindent The bolt deals 1d6 points of electricity damage per caster level (maximum 20d6) to the primary target. After it strikes, lightning can arc to a number of secondary targets equal to your caster level (maximum 20). The secondary bolts each strike one target and deal half as much damage as the primary one did (rounded down).

\smallskip\noindent Each target can attempt a Reflex saving throw for half damage. You choose secondary targets as you like, but they must all be within 30 feet of the primary target, and no target can be struck more than once. You can choose to affect fewer secondary targets than the maximum.

\shortability{Focus:}{A bit of fur; a piece of amber, glass, or a crystal rod; plus one silver pin for each of your caster levels.}

\featname{Changestaff}
\shortability{Transmutation}{}
\shortability{Level:}{Drd 7}
\shortability{Components:}{V, S, F}
\shortability{Casting Time:}{1 round}
\shortability{Range:}{Touch}
\shortability{Target:}{Your touched staff}
\shortability{Duration:}{1 hour/level (D)}
\shortability{Saving Throw:}{None}
\shortability{Spell Resistance:}{No}
\smallskip\noindent You change a specially prepared quarterstaff into a Huge treantlike creature, about 24 feet tall. When you plant the end of the staff in the ground and speak a special command to conclude the casting of the spell, your staff turns into a creature that looks and fights just like a treant. The staff-treant defends you and obeys any spoken commands. However, it is by no means a true treant; it cannot converse with actual treants or control trees. If the staff-treant is reduced to 0 or fewer hit points, it crumbles to powder and the staff is destroyed. Otherwise, the staff returns to its normal form when the spell duration expires (or when the spell is dismissed), and it can be used as the focus for another casting of the spell. The staff-treant is always at full strength when created, despite any wounds it may have incurred the last time it appeared.

\shortability{Focus:}{The quarterstaff, which must be specially prepared. The staff must be a sound limb cut from an ash, oak, or yew, then cured, shaped, carved, and polished (a process requiring twenty-eight days).}

\smallskip\noindent You cannot adventure or engage in other strenuous activity during the shaping and carving of the staff.

\featname{Chaos Hammer}
\shortability{Evocation}{[Chaotic]}
\shortability{Level:}{Chaos 4}
\shortability{Components:}{V, S}
\shortability{Casting Time:}{1 standard action}
\shortability{Range:}{Medium (100 ft. + 10 ft./level)}
\shortability{Area:}{20-ft.-radius burst}
\shortability{Duration:}{Instantaneous (1d6 rounds); see text}
\shortability{Saving Throw:}{Will partial; see text}
\shortability{Spell Resistance:}{Yes}
\smallskip\noindent You unleash chaotic power to smite your enemies. The power takes the form of a multicolored explosion of leaping, ricocheting energy. Only lawful and neutral (not chaotic) creatures are harmed by the spell.

\smallskip\noindent The spell deals 1d8 points of damage per two caster levels (maximum 5d8) to lawful creatures (or 1d6 points of damage per caster level, maximum 10d6, to lawful outsiders) and slows them for 1d6 rounds (see the slow spell). A successful Will save reduces the damage by half and negates the slow effect.

\smallskip\noindent The spell deals only half damage against creatures who are neither lawful nor chaotic, and they are not slowed. Such a creature can reduce the damage by half again (down to one-quarter) with a successful Will save.

\featname{Charm Animal}
\shortability{Enchantment}{(Charm) [Mind-Affecting]}
\shortability{Level:}{Drd 1, Rgr 1}
\shortability{Target:}{One animal}
\smallskip\noindent This spell functions like charm person, except that it affects a creature of the animal type.

\featname{Charm Monster}
\shortability{Enchantment}{(Charm) [Mind-Affecting]}
\shortability{Level:}{Brd 3, Sor/Wiz 4}
\shortability{Target:}{One living creature}
\shortability{Duration:}{One day/level}
\smallskip\noindent This spell functions like charm person, except that the effect is not restricted by creature type or size.

\featname{Charm Monster, Mass}
\shortability{Enchantment}{(Charm) [Mind-Affecting]}
\shortability{Level:}{Brd 6, Sor/Wiz 8}
\shortability{Components:}{V}
\shortability{Targets:}{One or more creatures, no two of which can be more than 30 ft. apart}
\shortability{Duration:}{One day/level}
\smallskip\noindent This spell functions like charm monster, except that mass charm monster affects a number of creatures whose combined HD do not exceed twice your level, or at least one creature regardless of HD. If there are more potential targets than you can affect, you choose them one at a time until you choose a creature with too many HD.

\featname{Charm Person}
\shortability{Enchantment}{(Charm) [Mind-Affecting]}
\shortability{Level:}{Brd 1, Sor/Wiz 1}
\shortability{Components:}{V, S}
\shortability{Casting Time:}{1 standard action}
\shortability{Range:}{Close (25 ft. + 5 ft./2 levels)}
\shortability{Target:}{One humanoid creature}
\shortability{Duration:}{1 hour/level}
\shortability{Saving Throw:}{Will negates}
\shortability{Spell Resistance:}{Yes}
\smallskip\noindent This charm makes a humanoid creature regard you as its trusted friend and ally (treat the target's attitude as friendly). If the creature is currently being threatened or attacked by you or your allies, however, it receives a +5 bonus on its saving throw. 

\smallskip\noindent The spell does not enable you to control the charmed person as if it were an automaton, but it perceives your words and actions in the most favorable way. You can try to give the subject orders, but you must win an opposed Charisma check to convince it to do anything it wouldn't ordinarily do. (Retries are not allowed.) An affected creature never obeys suicidal or obviously harmful orders, but it might be convinced that something very dangerous is worth doing.  Any act by you or your apparent allies that threatens the charmed person breaks the spell. You must speak the person's language to communicate your commands, or else be good at pantomiming.

\featname{Chill Metal}
\shortability{Transmutation}{[Cold]}
\shortability{Level:}{Drd 2}
\shortability{Components:}{V, S, DF}
\shortability{Casting Time:}{1 standard action}
\shortability{Range:}{Close (25 ft. + 5 ft./2 levels)}
\shortability{Target:}{Metal equipment of one creature per two levels, no two of which can be more than 30 ft. apart; or 25 lb. of metal/level, none of which can be more than 30 ft. away from any of the rest}
\shortability{Duration:}{7 rounds}
\shortability{Saving Throw:}{Will negates (object)}
\shortability{Spell Resistance:}{Yes (object)}
\smallskip\noindent Chill metal makes metal extremely cold. Unattended, nonmagical metal gets no saving throw. Magical metal is allowed a saving throw against the spell. An item in a creature's possession uses the creature's saving throw bonus unless its own is higher.

\smallskip\noindent A creature takes cold damage if its equipment is chilled. It takes full damage if its armor is affected or if it is holding, touching, wearing, or carrying metal weighing one-fifth of its weight. The creature takes minimum damage (1 point or 2 points; see the table) if it's not wearing metal armor and the metal that it's carrying weighs less than one-fifth of its weight.

\smallskip\noindent On the first round of the spell, the metal becomes chilly and uncomfortable to touch but deals no damage. The same effect also occurs on the last round of the spell's duration. During the second (and also the next-to-last) round, icy coldness causes pain and damage. In the third, fourth, and fifth rounds, the metal is freezing cold, causing more damage, as shown on the table below.

\begin{tabular}[h!]{l|l|l}
Round & Metal Temperature & Damage \\ \hline
1 & Cold & None \\
2 & Icy & 1d4 points \\
3-5 & Freezing & 2d4 points \\
6 & Icy & 1d4 points \\
7 & Cold & None \\
\end{tabular}

\smallskip\noindent Any heat intense enough to damage the creature negates cold damage from the spell (and vice versa) on a point-for-point basis. Underwater, chill metal deals no damage, but ice immediately forms around the affected metal, making it more buoyant.

\smallskip\noindent Chill metal counters and dispels heat metal.

\featname{Chill Touch}
\shortability{Necromancy}{}
\shortability{Level:}{Sor/Wiz 1}
\shortability{Components:}{V, S}
\shortability{Casting Time:}{1 standard action}
\shortability{Range:}{Touch}
\shortability{Targets:}{Creature or creatures touched (up to one/level)}
\shortability{Duration:}{Instantaneous}
\shortability{Saving Throw:}{Fortitude partial or Will negates; see text}
\shortability{Spell Resistance:}{Yes}
\smallskip\noindent A touch from your hand, which glows with blue energy, disrupts the life force of living creatures. Each touch channels negative energy that deals 1d6 points of damage. The touched creature also takes 1 point of Strength damage unless it makes a successful Fortitude saving throw. You can use this melee touch attack up to one time per level.

\smallskip\noindent An undead creature you touch takes no damage of either sort, but it must make a successful Will saving throw or flee as if panicked for 1d4 rounds +1 round per caster level.

\featname{Circle of Death}
\shortability{Necromancy}{[Death]}
\shortability{Level:}{Sor/Wiz 6}
\shortability{Components:}{V, S, M}
\shortability{Casting Time:}{1 standard action}
\shortability{Range:}{Medium (100 ft. + 10 ft./level)}
\shortability{Area:}{Several living creatures within a 40-ft.-radius burst}
\shortability{Duration:}{Instantaneous}
\shortability{Saving Throw:}{Fortitude negates}
\shortability{Spell Resistance:}{Yes}
\smallskip\noindent A circle of death snuffs out the life force of living creatures, killing them instantly.

\smallskip\noindent The spell slays 1d4 HD worth of living creatures per caster level (maximum 20d4). Creatures with the fewest HD are affected first; among creatures with equal HD, those who are closest to the burst's point of origin are affected first. No creature of 9 or more HD can be affected, and Hit Dice that are not sufficient to affect a creature are wasted.

\shortability{Material Component:}{The powder of a crushed black pearl with a minimum value of 500 gp.}

\featname{Clairaudience/Clairvoyance}
\shortability{Divination}{(Scrying)}
\shortability{Level:}{Brd 3, Knowledge 3, Sor/Wiz 3}
\shortability{Components:}{V, S, F/DF}
\shortability{Casting Time:}{10 minutes}
\shortability{Range:}{Long (400 ft. + 40 ft./level)}
\shortability{Effect:}{Magical sensor}
\shortability{Duration:}{1 min./level (D)}
\shortability{Saving Throw:}{None}
\shortability{Spell Resistance:}{No}
\smallskip\noindent Clairaudience/clairvoyance creates an invisible magical sensor at a specific location that enables you to hear or see (your choice) almost as if you were there. You don't need line of sight or line of effect, but the locale must be known-a place familiar to you or an obvious one. Once you have selected the locale, the sensor doesn't move, but you can rotate it in all directions to view the area as desired. Unlike other scrying spells, this spell does not allow magically or supernaturally enhanced senses to work through it. If the chosen locale is magically dark, you see nothing. If it is naturally pitch black, you can see in a 10- foot radius around the center of the spell's effect. Clairaudience/clairvoyance functions only on the plane of existence you are currently occupying.

\shortability{Arcane Focus:}{A small horn (for hearing) or a glass eye (for seeing).}

\featname{Clenched Fist}
\shortability{Evocation}{[Force]}
\shortability{Level:}{Sor/Wiz 8, Strength 8}
\shortability{Components:}{V, S, F/DF}
\smallskip\noindent This spell functions like interposing hand, except that the hand can interpose itself, push, or strike one opponent that you select. The floating hand can move as far as 60 feet and can attack in the same round. Since this hand is directed by you, its ability to notice or attack invisible or concealed creatures is no better than yours.

\smallskip\noindent The hand attacks once per round, and its attack bonus equals your caster level + your Intelligence, Wisdom, or Charisma modifier (for a wizard, cleric, or sorcerer, respectively), +11 for the hand's Strength score (33), -1 for being Large. The hand deals 1d8+11 points of damage on each attack, and any creature struck must make a Fortitude save (against this spell's save DC) or be stunned for 1 round. Directing the spell to a new target is a move action.

\smallskip\noindent The clenched fist can also interpose itself as interposing hand does, or it can bull rush an opponent as forceful hand does, but at a +15 bonus on the Strength check.

\smallskip\noindent Clerics who cast this spell name it for their deities.

\shortability{Arcane Focus:}{A leather glove.}

\featname{Cloak of Chaos}
\shortability{Abjuration}{[Chaotic]}
\shortability{Level:}{Chaos 8, Clr 8}
\shortability{Components:}{V, S, F}
\shortability{Casting Time:}{1 standard action}
\shortability{Range:}{20 ft.}
\shortability{Targets:}{One creature/level in a 20-ft.-radius burst centered on you}
\shortability{Duration:}{1 round/level (D)}
\shortability{Saving Throw:}{See text}
\shortability{Spell Resistance:}{Yes (harmless)}
\smallskip\noindent A random pattern of color surrounds the subjects, protecting them from attacks, granting them resistance to spells cast by lawful creatures, and causing lawful creatures that strike the subjects to become confused. This abjuration has four effects.

\smallskip\noindent First, each warded creature gains a +4 deflection bonus to AC and a +4 resistance bonus on saves. Unlike protection from law, the benefit of this spell applies against all attacks, not just against attacks by lawful creatures.

\smallskip\noindent Second, each warded creature gains spell resistance 25 against lawful spells and spells cast by lawful creatures.

\smallskip\noindent Third, the abjuration blocks possession and mental influence, just as protection from law does.

\smallskip\noindent Finally, if a lawful creature succeeds on a melee attack against a warded creature, the offending attacker is confused for 1 round (Will save negates, as with the confusion spell, but against the save DC of cloak of chaos).

\shortability{Focus:}{A tiny reliquary containing some sacred relic, such as a scrap of parchment from a chaotic text. The reliquary costs at least 500 gp.}

\featname{Clone}
\shortability{Necromancy}{}
\shortability{Level:}{Sor/Wiz 8}
\shortability{Components:}{V, S, M, F}
\shortability{Casting Time:}{10 minutes}
\shortability{Range:}{0 ft.}
\shortability{Effect:}{One clone}
\shortability{Duration:}{Instantaneous}
\shortability{Saving Throw:}{None}
\shortability{Spell Resistance:}{No}
\smallskip\noindent This spell makes an inert duplicate of a creature. If the original individual has been slain, its soul immediately transfers to the clone, creating a replacement (provided that the soul is free and willing to return). The original's physical remains, should they still exist, become inert and cannot thereafter be restored to life. If the original creature has reached the end of its natural life span (that is, it has died of natural causes), any cloning attempt fails.

\smallskip\noindent To create the duplicate, you must have a piece of flesh (not hair, nails, scales, or the like) with a volume of at least 1 cubic inch that was taken from the original creature's living body. The piece of flesh need not be fresh, but it must be kept from rotting. Once the spell is cast, the duplicate must be grown in a laboratory for 2d4 months.

\smallskip\noindent When the clone is completed, the original's soul enters it immediately, if that creature is already dead. The clone is physically identical with the original and possesses the same personality and memories as the original. In other respects, treat the clone as if it were the original character raised from the dead, including the loss of one level or 2 points of Constitution (if the original was a 1st-level character). If this Constitution adjustment would give the clone a Constitution score of 0, the spell fails. If the original creature has lost levels since the flesh sample was taken and died at a lower level than the clone would otherwise be, the clone is one level below the level at which the original died.

\smallskip\noindent The spell duplicates only the original's body and mind, not its equipment.

\smallskip\noindent A duplicate can be grown while the original still lives, or when the original soul is unavailable, but the resulting body is merely a soulless bit of inert flesh, which rots if not preserved.

\shortability{Material Component:}{The piece of flesh and various laboratory supplies (cost 1,000 gp).}
\shortability{Focus:}{Special laboratory equipment (cost 500 gp).}

\featname{Cloudkill}
\shortability{Conjuration}{(Creation)}
\shortability{Level:}{Sor/Wiz 5}
\shortability{Components:}{V, S}
\shortability{Casting Time:}{1 standard action}
\shortability{Range:}{Medium (100 ft. + 10 ft./level)}
\shortability{Effect:}{Cloud spreads in 20-ft. radius, 20 ft. high}
\shortability{Duration:}{1 min./level}
\shortability{Saving Throw:}{Fortitude partial; see text}
\shortability{Spell Resistance:}{No}
\smallskip\noindent This spell generates a bank of fog, similar to a fog cloud, except that its vapors are yellowish green and poisonous. These vapors automatically kill any living creature with 3 or fewer HD (no save). A living creature with 4 to 6 HD is slain unless it succeeds on a Fortitude save (in which case it takes 1d4 points of Constitution damage on your turn each round while in the cloud).

\smallskip\noindent A living creature with 6 or more HD takes 1d4 points of Constitution damage on your turn each round while in the cloud (a successful Fortitude save halves this damage). Holding one's breath doesn't help, but creatures immune to poison are unaffected by the spell.

\smallskip\noindent Unlike a fog cloud, the cloudkill moves away from you at 10 feet per round, rolling along the surface of the ground.

\smallskip\noindent Figure out the cloud's new spread each round based on its new point of origin, which is 10 feet farther away from the point of origin where you cast the spell.

\smallskip\noindent Because the vapors are heavier than air, they sink to the lowest level of the land, even pouring down den or sinkhole openings. It cannot penetrate liquids, nor can it be cast underwater.

\featname{Color Spray}
\shortability{Illusion}{(Pattern) [Mind-Affecting]}
\shortability{Level:}{Sor/Wiz 1}
\shortability{Components:}{V, S, M}
\shortability{Casting Time:}{1 standard action}
\shortability{Range:}{15 ft.}
\shortability{Area:}{Cone-shaped burst}
\shortability{Duration:}{Instantaneous; see text}
\shortability{Saving Throw:}{Will negates}
\shortability{Spell Resistance:}{Yes}
\smallskip\noindent A vivid cone of clashing colors springs forth from your hand, causing creatures to become stunned, perhaps also blinded, and possibly knocking them unconscious.

\smallskip\noindent Each creature within the cone is affected according to its Hit Dice.

\shortability{2 HD or less:}{The creature is unconscious, blinded, and stunned for 2d4 rounds, then blinded and stunned for 1d4 rounds, and then stunned for 1 round. (Only living creatures are knocked unconscious.)}
\shortability{3 or 4 HD:}{The creature is blinded and stunned for 1d4 rounds, then stunned for 1 round.}
\shortability{5 or more HD:}{The creature is stunned for 1 round.}

\smallskip\noindent Sightless creatures are not affected by color spray.

\shortability{Material Component:}{A pinch each of powder or sand that is colored red, yellow, and blue.}

\featname{Command}
\shortability{Enchantment}{(Compulsion) [Language-Dependent, Mind-Affecting]}
\shortability{Level:}{Clr 1}
\shortability{Components:}{V}
\shortability{Casting Time:}{1 standard action}
\shortability{Range:}{Close (25 ft. + 5 ft./2 levels)}
\shortability{Target:}{One living creature}
\shortability{Duration:}{1 round}
\shortability{Saving Throw:}{Will negates}
\shortability{Spell Resistance:}{Yes}
\smallskip\noindent You give the subject a single command, which it obeys to the best of its ability at its earliest opportunity. You may select from the following options.

\shortability{Approach:}{On its turn, the subject moves toward you as quickly and directly as possible for 1 round. The creature may do nothing but move during its turn, and it provokes attacks of opportunity for this movement as normal.}
\shortability{Drop:}{On its turn, the subject drops whatever it is holding. It can't pick up any dropped item until its next turn.}
\shortability{Fall:}{On its turn, the subject falls to the ground and remains prone for 1 round. It may act normally while prone but takes any appropriate penalties.}
\shortability{Flee:}{On its turn, the subject moves away from you as quickly as possible for 1 round. It may do nothing but move during its turn, and it provokes attacks of opportunity for this movement as normal.}
\shortability{Halt:}{The subject stands in place for 1 round. It may not take any actions but is not considered helpless.}

\smallskip\noindent If the subject can't carry out your command on its next turn, the spell automatically fails.

\featname{Command, Greater}
\shortability{Enchantment}{(Compulsion) [Language-Dependent, Mind-Affecting]}
\shortability{Level:}{Clr 5}
\shortability{Targets:}{One creature/level, no two of which can be more than 30 ft. apart}
\shortability{Duration:}{1 round/level}
\smallskip\noindent This spell functions like command, except that up to one creature per level may be affected, and the activities continue beyond 1 round. At the start of each commanded creature's action after the first, it gets another Will save to attempt to break free from the spell. Each creature must receive the same command.

\featname{Command Plants}
\shortability{Transmutation}{}
\shortability{Level:}{Drd 4, Plant 4, Rgr 3}
\shortability{Components:}{V}
\shortability{Casting Time:}{1 standard action}
\shortability{Range:}{Close (25 ft. + 5 ft./2 levels)}
\shortability{Targets:}{Up to 2 HD/level of plant creatures, no two of which can be more than 30 ft. apart}
\shortability{Duration:}{One day/level}
\shortability{Saving Throw:}{Will negates}
\shortability{Spell Resistance:}{Yes}
\smallskip\noindent This spell allows you some degree of control over one or more plant creatures. Affected plant creatures can understand you, and they perceive your words and actions in the most favorable way (treat their attitude as friendly). They will not attack you while the spell lasts. You can try to give a subject orders, but you must win an opposed Charisma check to convince it to do anything it wouldn't ordinarily do. (Retries are not allowed.) A commanded plant never obeys suicidal or obviously harmful orders, but it might be convinced that something very dangerous is worth doing.

\smallskip\noindent You can affect a number of plant creatures whose combined level or HD do not exceed twice your level.

\featname{Command Undead}
\shortability{Necromancy}{}
\shortability{Level:}{Sor/Wiz 2}
\shortability{Components:}{V, S, M}
\shortability{Casting Time:}{1 standard action}
\shortability{Range:}{Close (25 ft. + 5 ft./2 levels)}
\shortability{Targets:}{One undead creature}
\shortability{Duration:}{One day/level}
\shortability{Saving Throw:}{Will negates; see text}
\shortability{Spell Resistance:}{Yes}
\smallskip\noindent This spell allows you some degree of control over an undead creature. Assuming the subject is intelligent, it perceives your words and actions in the most favorable way (treat its attitude as friendly). It will not attack you while the spell lasts. You can try to give the subject orders, but you must win an opposed Charisma check to convince it to do anything it wouldn't ordinarily do. (Retries are not allowed.) An intelligent commanded undead never obeys suicidal or obviously harmful orders, but it might be convinced that something very dangerous is worth doing.

\smallskip\noindent A nonintelligent undead creature gets no saving throw against this spell. When you control a mindless being, you can communicate only basic commands, such as ``come here,'' ``go there,'' ``fight,'' ``stand still,'' and so on. Nonintelligent undead won't resist suicidal or obviously harmful orders.

\smallskip\noindent Any act by you or your apparent allies that threatens the commanded undead (regardless of its Intelligence) breaks the spell.

\smallskip\noindent Your commands are not telepathic. The undead creature must be able to hear you.

\shortability{Material Component:}{A shred of raw meat and a splinter of bone.}

\featname{Commune}
\shortability{Divination}{}
\shortability{Level:}{Clr 5}
\shortability{Components:}{V, S, M, DF, XP}
\shortability{Casting Time:}{10 minutes}
\shortability{Range:}{Personal}
\shortability{Target:}{You}
\shortability{Duration:}{1 round/level}
\smallskip\noindent You contact your deity-or agents thereof -and ask questions that can be answered by a simple yes or no. (A cleric of no particular deity contacts a philosophically allied deity.) You are allowed one such question per caster level. The answers given are correct within the limits of the entity's knowledge. "Unclear" is a legitimate answer, because powerful beings of the Outer Planes are not necessarily omniscient. In cases where a one-word answer would be misleading or contrary to the deity's interests, a short phrase (five words or less) may be given as an answer instead.

\smallskip\noindent The spell, at best, provides information to aid character decisions. The entities contacted structure their answers to further their own purposes. If you lag, discuss the answers, or go off to do anything else, the spell ends.

\shortability{Material Component:}{Holy (or unholy) water and incense.}
\shortability{XP Cost:}{100 XP.}

\featname{Commune with Nature}
\shortability{Divination}{}
\shortability{Level:}{Animal 5, Drd 5, Rgr 4}
\shortability{Components:}{V, S}
\shortability{Casting Time:}{10 minutes}
\shortability{Range:}{Personal}
\shortability{Target:}{You}
\shortability{Duration:}{Instantaneous}
\smallskip\noindent You become one with nature, attaining knowledge of the surrounding territory. You instantly gain knowledge of as many as three facts from among the following subjects: the ground or terrain, plants, minerals, bodies of water, people, general animal population, presence of woodland creatures, presence of powerful unnatural creatures, or even the general state of the natural setting.

\smallskip\noindent In outdoor settings, the spell operates in a radius of 1 mile per caster level. In natural underground settings-caves, caverns, and the like-the radius is limited to 100 feet per caster level. The spell does not function where nature has been replaced by construction or settlement, such as in dungeons and towns.

\featname{Comprehend Languages}
\shortability{Divination}{}
\shortability{Level:}{Brd 1, Clr 1, Sor/Wiz 1}
\shortability{Components:}{V, S, M/DF}
\shortability{Casting Time:}{1 standard action}
\shortability{Range:}{Personal}
\shortability{Target:}{You}
\shortability{Duration:}{10 min./level}
\smallskip\noindent You can understand the spoken words of creatures or read otherwise incomprehensible written messages. In either case, you must touch the creature or the writing. The ability to read does not necessarily impart insight into the material, merely its literal meaning. The spell enables you to understand or read an unknown language, not speak or write it.

\smallskip\noindent Written material can be read at the rate of one page (250 words) per minute. Magical writing cannot be read, though the spell reveals that it is magical. This spell can be foiled by certain warding magic (such as the secret page and illusory script spells). It does not decipher codes or reveal messages concealed in otherwise normal text.

\smallskip\noindent Comprehend languages can be made permanent with a permanency spell.

\shortability{Arcane Material Component:}{A pinch of soot and a few grains of salt.}

\featname{Cone of Cold}
\shortability{Evocation}{[Cold]}
\shortability{Level:}{Sor/Wiz 5, Water 6}
\shortability{Components:}{V, S, M/DF}
\shortability{Casting Time:}{1 standard action}
\shortability{Range:}{60 ft.}
\shortability{Area:}{Cone-shaped burst}
\shortability{Duration:}{Instantaneous}
\shortability{Saving Throw:}{Reflex half}
\shortability{Spell Resistance:}{Yes}
\smallskip\noindent Cone of cold creates an area of extreme cold, originating at your hand and extending outward in a cone. It drains heat, dealing 1d6 points of cold damage per caster level (maximum 15d6).

\shortability{Arcane Material Component:}{A very small crystal or glass cone.}

\featname{Confusion}
\shortability{Enchantment}{(Compulsion) [Mind-Affecting]}
\shortability{Level:}{Brd 3, Sor/Wiz 4, Trickery 4}
\shortability{Components:}{V, S, M/DF}
\shortability{Casting Time:}{1 standard action}
\shortability{Range:}{Medium (100 ft. + 10 ft./level)}
\shortability{Targets:}{All creatures in a 15-ft. radius burst}
\shortability{Duration:}{1 round/level}
\shortability{Saving Throw:}{Will negates}
\shortability{Spell Resistance:}{Yes}
\smallskip\noindent This spell causes the targets to become confused, making them unable to independently determine what they will do.

\smallskip\noindent Roll on the following table at the beginning of each subject's turn each round to see what the subject does in that round.

\begin{tabular}[h!]{l p{2.5in}}
d\% & Behavior \\ \hline
01-10 & Attack caster with melee or ranged weapons (or close with caster if attack is not possible). \\
11-20 & Act normally. \\
21-50 & Do nothing but babble incoherently. \\
51-70 & Flee away from caster at top possible speed. \\
71-100 & Attack nearest creature (for this purpose, a familiar counts as part of the subject's self). \\
\end{tabular}

\smallskip\noindent A confused character who can't carry out the indicated action does nothing but babble incoherently. Attackers are not at any special advantage when attacking a confused character. Any confused character who is attacked automatically attacks its attackers on its next turn, as long as it is still confused when its turn comes. Note that a confused character will not make attacks of opportunity against any creature that it is not already devoted to attacking (either because of its most recent action or because it has just been attacked).

\shortability{Arcane Material Component:}{A set of three nut shells.}

\featname{Confusion, Lesser}
\shortability{Enchantment}{(Compulsion) [Mind-Affecting]}
\shortability{Level:}{Brd 1}
\shortability{Components:}{V, S, DF}
\shortability{Range:}{Close (25 ft. + 5 ft./2 levels)}
\shortability{Target:}{One living creature}
\shortability{Duration:}{1 round}
\smallskip\noindent This spell causes a single creature to become confused for 1 round. See the confusion spell, above, to determine the exact effect on the subject.

\featname{Consecrate}
\shortability{Evocation}{[Good]}
\shortability{Level:}{Clr 2}
\shortability{Components:}{V, S, M, DF}
\shortability{Casting Time:}{1 standard action}
\shortability{Range:}{Close (25 ft. + 5 ft./2 levels)}
\shortability{Area:}{20-ft.-radius emanation}
\shortability{Duration:}{2 hours/level}
\shortability{Saving Throw:}{None}
\shortability{Spell Resistance:}{No}
\smallskip\noindent This spell blesses an area with positive energy. Each Charisma check made to turn undead within this area gains a +3 sacred bonus. Every undead creature entering a consecrated area suffers minor disruption, giving it a -1 penalty on attack rolls, damage rolls, and saves. Undead cannot be created within or summoned into a consecrated area.

\smallskip\noindent If the consecrated area contains an altar, shrine, or other permanent fixture dedicated to your deity, pantheon, or aligned higher power, the modifiers given above are doubled (+6 sacred bonus on turning checks, -2 penalties for undead in the area). You cannot consecrate an area with a similar fixture of a deity other than your own patron.

\smallskip\noindent If the area does contain an altar, shrine, or other permanent fixture of a deity, pantheon, or higher power other than your patron, the consecrate spell instead curses the area, cutting off its connection with the associated deity or power. This secondary function, if used, does not also grant the bonuses and penalties relating to undead, as given above.

\smallskip\noindent Consecrate counters and dispels desecrate.

\shortability{Material Component:}{A vial of holy water and 25 gp worth (5 pounds) of silver dust, all of which must be sprinkled around the area.}

\featname{Contact Other Plane}
\shortability{Divination}{}
\shortability{Level:}{Sor/Wiz 5}
\shortability{Components:}{V}
\shortability{Casting Time:}{10 minutes}
\shortability{Range:}{Personal}
\shortability{Target:}{You}
\shortability{Duration:}{Concentration}
\smallskip\noindent You send your mind to another plane of existence (an Elemental Plane or some plane farther removed) in order to receive advice and information from powers there. (See the accompanying table for possible consequences and results of the attempt.) The powers reply in a language you understand, but they resent such contact and give only brief answers to your questions. (All questions are answered with ``yes,'' ``no,'' ``maybe,'' ``never,'' ``irrelevant,'' or some other one-word answer.)

\smallskip\noindent You must concentrate on maintaining the spell (a standard action) in order to ask questions at the rate of one per round. A question is answered by the power during the same round. For every two caster levels, you may ask one question.

\smallskip\noindent Contact with minds far removed from your home plane increases the probability that you will incur a decrease to Intelligence and Charisma, but the chance of the power knowing the answer, as well as the probability of the entity answering correctly, are likewise increased by moving to distant planes.

\smallskip\noindent Once the Outer Planes are reached, the power of the deity contacted determines the effects. (Random results obtained from the table are subject to the personalities of individual deities.)

\smallskip\noindent On rare occasions, this divination may be blocked by an act of certain deities or forces.

\begin{center}
\begin{table*}[b!]
\caption{Contact Other Plane Results}
\begin{tabular}{l|p{1.2in}llll}
Plane Contacted & Avoid Int/Cha Decrease & True Answer & Don't Know & Lie & Random Answer \\ \hline
Elemental Plane & DC 7/1 week & 01-34 & 35-62 & 63-83 & 84-100 \\
 -(appropriate) & (DC 7/1 week) & (01-68) & (69-75) & (76-98) & (99-100) \\
Positive/Negative Energy Plane & DC 8/1 week & 01-39 & 40-65 & 66-86 & 87-100 \\
Astral Plane & DC 9/1 week & 01-44 & 45-67 & 68-88 & 89-100 \\
Outer Plane, demideity & DC 10/2 weeks & 01-49 & 50-70 & 71-91 & 92-100 \\
Outer Plane, lesser deity & DC 12/3 weeks & 01-60 & 61-75 & 76-95 & 96-100 \\
Outer Plane, intermediate deity & DC 14/4 weeks & 01-73 & 74-81 & 82-98 & 99-100 \\
Outer Plane, greater deity & DC 16/5 weeks & 01-88 & 89-90 & 91-99 & 100 \\
\end{tabular}
\end{table*}
\end{center}

\shortability{Avoid Int/Cha Decrease:}{You must succeed on an Intelligence check against this DC to avoid a decrease in Intelligence and Charisma. If the check fails, your Intelligence and Charisma scores each fall to 8 for the stated duration, and you become unable to cast arcane spells. If you lose Intelligence and Charisma, the effect strikes as soon as the first question is asked, and no answer is received. (The entries in parentheses are for questions that pertain to the appropriate Elemental Plane.)}
\shortability{Results of a Successful Contact:}{d\% is rolled for the result shown on the table}
\shortability{True Answer:}{You get a true, one-word answer. Questions that cannot be answered in this way are answered randomly.}
\shortability{Don't Know:}{The entity tells you that it doesn't know.}
\shortability{Lie:}{The entity intentionally lies to you.}
\shortability{Random Answer:}{The entity tries to lie but doesn't know the answer, so it makes one up.}

\featname{Contagion}
\shortability{Necromancy}{[Evil]}
\shortability{Level:}{Clr 3, Destruction 3, Drd 3, Sor/Wiz 4}
\shortability{Components:}{V, S}
\shortability{Casting Time:}{1 standard action}
\shortability{Range:}{Touch}
\shortability{Target:}{Living creature touched}
\shortability{Duration:}{Instantaneous}
\shortability{Saving Throw:}{Fortitude negates}
\shortability{Spell Resistance:}{Yes}
\smallskip\noindent The subject contracts a disease selected from the table below, which strikes immediately (no incubation period). The DC noted is for the subsequent saves (use contagion's normal save DC for the initial saving throw).

\begin{tabular}[h!]{l|l|l}
Disease & DC & Damage \\ \hline
Blinding sickness & 16 & 1d4 Str\textsuperscript{1} \\
Cackle fever & 16 & 1d6 Wis \\
Filth fever & 12 & 1d3 Dex and 1d3 Con \\
Mindfire & 12 & 1d4 Int \\
Red ache & 15 & 1d6 Str	\\
Shakes & 13 & 1d8 Dex \\
Slimy doom & 14 & 1d4 Con \\
\multicolumn{3}{p{3in}}{\textsuperscript{1}Each time a victim takes 2 or more points of Strength damage from blinding sickness, he or she must make another Fortitude save (using the disease's save DC) or be permanently blinded.} \\
\end{tabular}

\featname{Contingency}
\shortability{Evocatio}{}
\shortability{Level:}{Sor/Wiz 6}
\shortability{Components:}{V, S, M, F}
\shortability{Casting Time:}{At least 10 minutes; see text}
\shortability{Range:}{Personal}
\shortability{Target:}{You}
\shortability{Duration:}{One day/level (D) or until discharged}
\smallskip\noindent You can place another spell upon your person so that it comes into effect under some condition you dictate when casting contingency. The contingency spell and the companion spell are cast at the same time. The 10-minute casting time is the minimum total for both castings; if the companion spell has a casting time longer than 10 minutes, use that instead.

\smallskip\noindent The spell to be brought into effect by the contingency must be one that affects your person and be of a spell level no higher than one-third your caster level (rounded down, maximum 6th level).

\smallskip\noindent The conditions needed to bring the spell into effect must be clear, although they can be general. In all cases, the contingency immediately brings into effect the companion spell, the latter being ``cast'' instantaneously when the prescribed circumstances occur. If complicated or convoluted conditions are prescribed, the whole spell combination (contingency and the companion magic) may fail when called on. The companion spell occurs based solely on the stated conditions, regardless of whether you want it to. 

\smallskip\noindent You can use only one contingency spell at a time; if a second is cast, the first one (if still active) is dispelled.

\shortability{Material Component:}{That of the companion spell, plus quicksilver and an eyelash of an ogre mage, rakshasa, or similar spell-using creature.}
\shortability{Focus:}{A statuette of you carved from elephant ivory and decorated with gems (worth at least 1,500 gp). You must carry the focus for the contingency to work.}

\featname{Continual Flame}
\shortability{Evocation}{[Light]}
\shortability{Level:}{Clr 3, Sor/Wiz 2}
\shortability{Components:}{V, S, M}
\shortability{Casting Time:}{1 standard action}
\shortability{Range:}{Touch}
\shortability{Target:}{Object touched}
\shortability{Effect:}{Magical, heatless flame}
\shortability{Duration:}{Permanent}
\shortability{Saving Throw:}{None}
\shortability{Spell Resistance:}{No}
\smallskip\noindent A flame, equivalent in brightness to a torch, springs forth from an object that you touch. The effect looks like a regular flame, but it creates no heat and doesn't use oxygen. A continual flame can be covered and hidden but not smothered or quenched.
Light spells counter and dispel darkness spells of an equal or lower level.

\shortability{Material Component:}{You sprinkle ruby dust (worth 50 gp) on the item that is to carry the flame.}

\featname{Control Plants}
\shortability{Transmutation}{}
\shortability{Level:}{Drd 8, Plant 8}
\shortability{Components:}{V, S, DF}
\shortability{Casting Time:}{1 standard action}
\shortability{Range:}{Close (25 ft. + 5 ft./2 levels)}
\shortability{Targets:}{Up to 2 HD/level of plant creatures, no two of which can be more than 30 ft. apart}
\shortability{Duration:}{1 min./level}
\shortability{Saving Throw:}{Will negates}
\shortability{Spell Resistance:}{No}
\smallskip\noindent This spell enables you to control the actions of one or more plant creatures for a short period of time. You command the creatures by voice and they understand you, no matter what language you speak. Even if vocal communication is impossible  the controlled plants do not attack you. At the end of the spell, the subjects revert to their normal behavior.

\smallskip\noindent Suicidal or self-destructive commands are simply ignored.

\featname{Control Undead}
\shortability{Necromancy}{}
\shortability{Level:}{Sor/Wiz 7}
\shortability{Components:}{V, S, M}
\shortability{Casting Time:}{1 standard action}
\shortability{Range:}{Close (25 ft. + 5 ft./2 levels)}
\shortability{Targets:}{Up to 2 HD/level of undead creatures, no two of which can be more than 30 ft. apart}
\shortability{Duration:}{1 min./level}
\shortability{Saving Throw:}{Will negates}
\shortability{Spell Resistance:}{Yes}
\smallskip\noindent This spell enables you to command undead creatures for a short period of time. You command them by voice and they understand you, no matter what language you speak. Even if vocal communication is impossible the controlled undead do not attack you. At the end of the spell, the subjects revert to their normal behavior.

\smallskip\noindent Intelligent undead creatures remember that you controlled them.

\shortability{Material Component:}{A small piece of bone and a small piece of raw meat.}

\featname{Control Water}
\shortability{Transmutation}{[Water]}
\shortability{Level:}{Clr 4, Drd 4, Sor/Wiz 6, Water 4}
\shortability{Components:}{V, S, M/DF}
\shortability{Casting Time:}{1 standard action}
\shortability{Range:}{Long (400 ft. + 40 ft./level)}
\shortability{Area:}{Water in a volume of 10 ft./level by 10 ft./level by 2 ft./level (S)}
\shortability{Duration:}{10 min./level (D)}
\shortability{Saving Throw:}{None; see text}
\shortability{Spell Resistance:}{No}
\smallskip\noindent Depending on the version you choose, the control water spell raises or lowers water.

\shortability{Lower Water:}{This causes water or similar liquid to reduce its depth by as much as 2 feet per caster level (to a minimum depth of 1 inch). The water is lowered within a squarish depression whose sides are up to caster level x 10 feet long. In extremely large and deep bodies of water, such as a deep ocean, the spell creates a whirlpool that sweeps ships and similar craft downward, putting them at risk and rendering them unable to leave by normal movement for the duration of the spell. When cast on water elementals and other water-based creatures, this spell acts as a slow spell (Will negates). The spell has no effect on other creatures.}

\shortability{Raise Water:}{This causes water or similar liquid to rise in height, just as the lower water version causes it to lower. Boats raised in this way slide down the sides of the hump that the spell creates. If the area affected by the spell includes riverbanks, a beach, or other land nearby, the water can spill over onto dry land.
With either version, you may reduce one horizontal dimension by half and double the other horizontal dimension.}

\shortability{Arcane Material Component:}{A drop of water (for raise water) or a pinch of dust (for lower water).}

\featname{Control Weather}
\shortability{Transmutation}{}
\shortability{Level:}{Air 7, Clr 7, Drd 7, Sor/Wiz 7}
\shortability{Components:}{V, S}
\shortability{Casting Time:}{10 minutes; see text}
\shortability{Range:}{2 miles}
\shortability{Area:}{2-mile-radius circle, centered on you; see text}
\shortability{Duration:}{4d12 hours; see text}
\shortability{Saving Throw:}{None}
\shortability{Spell Resistance:}{No}
\smallskip\noindent You change the weather in the local area. It takes 10 minutes to cast the spell and an additional 10 minutes for the effects to manifest. You can call forth weather appropriate to the climate and season of the area you are in.
 
\begin{tabular}[h!]{l|p{2in}}
Season & Possible Weather \\
Spring & Tornado, thunderstorm, sleet storm, or hot weather	\\
Summer & Torrential rain, heat wave, or hailstorm	\\
Autumn & Hot or cold weather, fog, or sleet	\\
Winter & Frigid cold, blizzard, or thaw	\\
Late winter & Hurricane-force winds or early spring (coastal area) \\
\end{tabular}

\smallskip\noindent You control the general tendencies of the weather, such as the direction and intensity of the wind. You cannot control specific applications of the weather-where lightning strikes, for example, or the exact path of a tornado. When you select a certain weather condition to occur, the weather assumes that condition 10 minutes later (changing gradually, not abruptly). The weather continues as you left it for the duration, or until you use a standard action to designate a new kind of weather (which fully manifests itself 10 minutes later). Contradictory conditions are not possible simultaneously.

\smallskip\noindent Control weather can do away with atmospheric phenomena (naturally occurring or otherwise) as well as create them.

\smallskip\noindent A druid casting this spell doubles the duration and affects a circle with a 3-mile radius.

\featname{Control Winds}
\shortability{Transmutation}{[Air]}
\shortability{Level:}{Air 5, Drd 5}
\shortability{Components:}{V, S}
\shortability{Casting Time:}{1 standard action}
\shortability{Range:}{40 ft./level}
\shortability{Area:}{40 ft./level radius cylinder 40 ft. high}
\shortability{Duration:}{10 min./level}
\shortability{Saving Throw:}{Fortitude negates}
\shortability{Spell Resistance:}{No}
\smallskip\noindent You alter wind force in the area surrounding you. You can make the wind blow in a certain direction or manner, increase its strength, or decrease its strength. The new wind direction and strength persist until the spell ends or until you choose to alter your handiwork, which requires concentration. You may create an ``eye'' of calm air up to 80 feet in diameter at the center of the area if you so desire, and you may choose to limit the area to any cylindrical area less than your full limit.

\shortability{Wind Direction:}{You may choose one of four basic wind patterns to function over the spell's area.}

\listone
  \item A downdraft blows from the center outward in equal strength in all directions.
  \item An updraft blows from the outer edges in toward the center in equal strength from all directions, veering upward before impinging on the eye in the center.
  \item A rotation causes the winds to circle the center in clockwise or counterclockwise fashion.
  \item A blast simply causes the winds to blow in one direction across the entire area from one side to the other.
\end{list}

\shortability{Wind Strength:}{For every three caster levels, you can increase or decrease wind strength by one level. Each round on your turn, a creature in the wind must make a Fortitude save or suffer the effect of being in the windy area.}

\begin{tabular}[h!]{p{1in}|p{2in}}
Strong winds (21+ mph) & Make sailing difficult. \\ \hline
A severe wind (31+ mph) & Causes minor ship and building damage. \\ \hline
A windstorm (51+ mph) & Drives most flying creatures from the skies, uproots small trees, knocks down light wooden structures, tears off roofs, and endangers ships. \\ \hline
Hurricane force winds (75+ mph) & Destroy wooden buildings, sometimes uproot even large trees, and cause most ships to founder. \\ \hline
A tornado (175+ mph) & Destroys all nonfortified buildings and often uproots large trees. \\
\end{tabular}

\featname{Create Food and Water}
\shortability{Conjuration}{(Creation)}
\shortability{Level:}{Clr 3}
\shortability{Components:}{V, S}
\shortability{Casting Time:}{10 minutes}
\shortability{Range:}{Close (25 ft. + 5 ft./2 levels)}
\shortability{Effect:}{Food and water to sustain three humans or one horse/level for 24 hours}
\shortability{Duration:}{24 hours; see text}
\shortability{Saving Throw:}{None}
\shortability{Spell Resistance:}{No}
\smallskip\noindent The food that this spell creates is simple fare of your choice-highly nourishing, if rather bland. Food so created decays and becomes inedible within 24 hours, although it can be kept fresh for another 24 hours by casting a purify food and water spell on it. The water created by this spell is just like clean rain water, and it doesn't go bad as the food does.

\featname{Create Greater Undead}
\shortability{Necromancy}{[Evil]}
\shortability{Level:}{Clr 8, Death 8, Sor/Wiz 8}

\smallskip\noindent This spell functions like create undead, except that you can create more powerful and intelligent sorts of undead: shadows, wraiths, spectres, and devourers. The type or types of undead you can create is based on your caster level, as shown on the table below.

\begin{tabular}[h!]{l|l}
Caster Level & Undead Created \\ \hline
15th or lower & Shadow \\
16th-17th & Wraith \\
18th-19th & Spectre	\\
20th or higher & Devourer \\
\end{tabular}

\featname{Create Undead}
\shortability{Necromancy}{[Evil]}
\shortability{Level:}{Clr 6, Death 6, Evil 6, Sor/Wiz 6}
\shortability{Components:}{V, S, M}
\shortability{Casting Time:}{1 hour}
\shortability{Range:}{Close (25 ft. + 5 ft./2 levels)}
\shortability{Target:}{One corpse}
\shortability{Duration:}{Instantaneous}
\shortability{Saving Throw:}{None}
\shortability{Spell Resistance:}{No}
\smallskip\noindent A much more potent spell than animate dead, this evil spell allows you to create more powerful sorts of undead: ghouls, ghasts, mummies, and mohrgs. The type or types of undead you can create is based on your caster level, as shown on the table below.

\begin{tabular}[h!]{l|l}
Caster Level & Undead Created \\ \hline   
11th or lower & Ghoul \\
12th-14th & Ghast	\\
15th-17th & Mummy \\
18th or higher & Mohrg \\
\end{tabular}

\smallskip\noindent You may create less powerful undead than your level would allow if you choose. Created undead are not automatically under the control of their animator. If you are capable of commanding undead, you may attempt to command the undead creature as it forms.
This spell must be cast at night.

\shortability{Material Component:}{A clay pot filled with grave dirt and another filled with brackish water. The spell must be cast on a dead body. You must place a black onyx gem worth at least 50 gp per HD of the undead to be created into the mouth or eye socket of each corpse. The magic of the spell turns these gems into worthless shells.}

\featname{Create Water}
\shortability{Conjuration}{(Creation) [Water]}
\shortability{Level:}{Clr 0, Drd 0, Pal 1}
\shortability{Components:}{V, S}
\shortability{Casting Time:}{1 standard action}
\shortability{Range:}{Close (25 ft. + 5 ft./2 levels)}
\shortability{Effect:}{Up to 2 gallons of water/level}
\shortability{Duration:}{Instantaneous}
\shortability{Saving Throw:}{None}
\shortability{Spell Resistance:}{No}
\smallskip\noindent This spell generates wholesome, drinkable water, just like clean rain water. Water can be created in an area as small as will actually contain the liquid, or in an area three times as large-possibly creating a downpour or filling many small receptacles.

\shortability{Note:}{Conjuration spells can't create substances or objects within a creature. Water weighs about 8 pounds per gallon. One cubic foot of water contains roughly 8 gallons and weighs about 60 pounds.}

\featname{Creeping Doom}
\shortability{Conjuration}{(Summoning)}
\shortability{Level:}{Drd 7}
\shortability{Components:}{V, S}
\shortability{Casting Time:}{1 round}
\shortability{Range:}{Close (25 ft. + 5 ft./2 levels)/ 100 ft.; see text}
\shortability{Effect:}{One swarm of centipedes per two levels}
\shortability{Duration:}{1 min./level}
\shortability{Saving Throw:}{None}
\shortability{Spell Resistance:}{No}

\smallskip\noindent When you utter the spell of creeping doom, you call forth a mass of centipede swarms (one per two caster levels, to a maximum of ten swarms at 20th level), which need not appear adjacent to one another.

\smallskip\noindent You may summon the centipede swarms so that they share the area of other creatures. The swarms remain stationary, attacking any creatures in their area, unless you command the creeping doom to move (a standard action). As a standard action, you can command any number of the swarms to move toward any prey within 100 feet of you. You cannot command any swarm to move more than 100 feet away from you, and if you move more than 100 feet from any swarm, that swarm remains stationary, attacking any creatures in its area (but it can be commanded again if you move within 100 feet).

\featname{Crushing Despair}
\shortability{Enchantment}{(Compulsion) [Mind-Affecting]}
\shortability{Level:}{Brd 3, Sor/Wiz 4}
\shortability{Components:}{V, S, M}
\shortability{Casting Time:}{1 standard action}
\shortability{Range:}{30 ft.}
\shortability{Area:}{Cone-shaped burst}
\shortability{Duration:}{1 min./level}
\shortability{Saving Throw:}{Will negates}
\shortability{Spell Resistance:}{Yes}
\smallskip\noindent An invisible cone of despair causes great sadness in the subjects. Each affected creature takes a -2 penalty on attack rolls, saving throws, ability checks, skill checks, and weapon damage rolls.

\smallskip\noindent Crushing despair counters and dispels good hope.

\shortability{Material Component:}{A vial of tears.}

\featname{Crushing Hand}
\shortability{Evocation}{[Force]}
\shortability{Level:}{Sor/Wiz 9, Strength 9}
\shortability{Components:}{V, S, M, F/DF}
\smallskip\noindent This spell functions like interposing hand, except that the hand can interpose itself, push, or crush one opponent that you select.
The crushing hand can grapple an opponent like grasping hand does. Its grapple bonus equals your caster level + your Intelligence, Wisdom, or Charisma modifier (for a wizard, cleric, or sorcerer, respectively), +12 for the hand's Strength score (35), +4 for being Large. The hand deals 2d6+12 points of damage (lethal, not nonlethal) on each successful grapple check against an opponent.

\smallskip\noindent The crushing hand can also interpose itself as interposing hand does, or it can bull rush an opponent as forceful hand does, but at a +18 bonus.

\smallskip\noindent Directing the spell to a new target is a move action.

\smallskip\noindent Clerics who cast this spell name it for their deities.

\shortability{Arcane Material Component:}{The shell of an egg.}
\shortability{Arcane Focus:}{A glove of snakeskin.}

\featname{Cure Critical Wounds}
\shortability{Conjuration}{(Healing)}
\shortability{Level:}{Brd 4, Clr 4, Drd 5, Healing 4}
\smallskip\noindent This spell functions like cure light wounds, except that it cures 4d8 points of damage +1 point per caster level (maximum +20).

\featname{Cure Critical Wounds, Mass}
\shortability{Conjuration}{(Healing)}
\shortability{Level:}{Clr 8, Drd 9, Healing 8}
\smallskip\noindent This spell functions like mass cure light wounds, except that it cures 4d8 points of damage +1 point per caster level (maximum +40).

\featname{Cure Light Wounds}
\shortability{Conjuration}{(Healing)}
\shortability{Level:}{Brd 1, Clr 1, Drd 1, Healing 1, Pal 1, Rgr 2}
\shortability{Components:}{V, S}
\shortability{Casting Time:}{1 standard action}
\shortability{Range:}{Touch}
\shortability{Target:}{Creature touched}
\shortability{Duration:}{Instantaneous}
\shortability{Saving Throw:}{Will half (harmless); see text}
\shortability{Spell Resistance:}{Yes (harmless); see text}
\smallskip\noindent When laying your hand upon a living creature, you channel positive energy that cures 1d8 points of damage +1 point per caster level (maximum +5).

\smallskip\noindent Since undead are powered by negative energy, this spell deals damage to them instead of curing their wounds. An undead creature can apply spell resistance, and can attempt a Will save to take half damage.

\featname{Cure Light Wounds, Mass}
\shortability{Conjuration}{(Healing)}
\shortability{Level:}{Brd 5, Clr 5, Drd 6, Healing 5}
\shortability{Components:}{V, S}
\shortability{Casting Time:}{1 standard action}
\shortability{Range:}{Close (25 ft. + 5 ft./2 levels)}
\shortability{Target:}{One creature/level, no two of which can be more than 30 ft. apart}
\shortability{Duration:}{Instantaneous}
\shortability{Saving Throw:}{Will half (harmless) or Will half; see text}
\shortability{Spell Resistance:}{Yes (harmless) or Yes; see text}
\smallskip\noindent You channel positive energy to cure 1d8 points of damage +1 point per caster level (maximum +25) in each selected creature.

\smallskip\noindent Like other cure spells, mass cure light wounds deals damage to undead in its area rather than curing them. Each affected undead may attempt a Will save for half damage.

\featname{Cure Minor Wounds}
\shortability{Conjuration}{(Healing)}
\shortability{Level:}{Clr 0, Drd 0}
\smallskip\noindent This spell functions like cure light wounds, except that it cures only 1 point of damage.

\featname{Cure Moderate Wounds}
\shortability{Conjuration}{(Healing)}
\shortability{Level:}{Brd 2, Clr 2, Drd 3, Healing 2, Pal 3, Rgr 3}
\smallskip\noindent This spell functions like cure light wounds, except that it cures 2d8 points of damage +1 point per caster level (maximum +10).

\featname{Cure Moderate Wounds, Mass}
\shortability{Conjuration}{(Healing)}
\shortability{Level:}{Brd 6, Clr 6, Drd 7}
\smallskip\noindent This spell functions like mass cure light wounds, except that it cures 2d8 points of damage +1 point per caster level (maximum +30).

\featname{Cure Serious Wounds}
\shortability{Conjuration}{(Healing)}
\shortability{Level:}{Brd 3, Clr 3, Drd 4, Pal 4, Rgr 4, Healing 3}
\smallskip\noindent This spell functions like cure light wounds, except that it cures 3d8 points of damage +1 point per caster level (maximum +15).

\featname{Cure Serious Wounds, Mass}
\shortability{Conjuration}{(Healing)}
\shortability{Level:}{Clr 7, Drd 8}
\smallskip\noindent This spell functions like mass cure light wounds, except that it cures 3d8 points of damage +1 point per caster level (maximum +35).

\featname{Curse Water}
\shortability{Necromancy}{[Evil]}
\shortability{Level:}{Clr 1}
\shortability{Components:}{V, S, M}
\shortability{Casting Time:}{1 minute}
\shortability{Range:}{Touch}
\shortability{Target:}{Flask of water touched}
\shortability{Duration:}{Instantaneous}
\shortability{Saving Throw:}{Will negates (object)}
\shortability{Spell Resistance:}{Yes (object)}
\smallskip\noindent This spell imbues a flask (1 pint) of water with negative energy, turning it into unholy water. Unholy water damages good outsiders the way holy water damages undead and evil outsiders.

\shortability{Material Component:}{5 pounds of powdered silver (worth 25 gp).}

\featname{Dancing Lights}
\shortability{Evocation}{[Light]}
\shortability{Level:}{Brd 0, Sor/Wiz 0}
\shortability{Components:}{V, S}
\shortability{Casting Time:}{1 standard action}
\shortability{Range:}{Medium (100 ft. + 10 ft./level)}
\shortability{Effect:}{Up to four lights, all within a 10- ft.-radius area}
\shortability{Duration:}{1 minute (D)}
\shortability{Saving Throw:}{None}
\shortability{Spell Resistance:}{No}
\smallskip\noindent Depending on the version selected, you create up to four lights that resemble lanterns or torches (and cast that amount of light), or up to four glowing spheres of light (which look like will-o'-wisps), or one faintly glowing, vaguely humanoid shape. The dancing lights must stay within a 10-foot-radius area in relation to each other but otherwise move as you desire (no concentration required): forward or back, up or down, straight or turning corners, or the like. The lights can move up to 100 feet per round. A light winks out if the distance between you and it exceeds the spell's range.

\smallskip\noindent Dancing lights can be made permanent with a permanency spell.

\featname{Darkness}
\shortability{Evocation}{[Darkness]}
\shortability{Level:}{Brd 2, Clr 2, Sor/Wiz 2}
\shortability{Components:}{V, M/DF}
\shortability{Casting Time:}{1 standard action}
\shortability{Range:}{Touch}
\shortability{Target:}{Object touched}
\shortability{Duration:}{10 min./level (D)}
\shortability{Saving Throw:}{None}
\shortability{Spell Resistance:}{No}
\smallskip\noindent This spell causes an object to radiate shadowy illumination out to a 20-foot radius. All creatures in the area gain concealment (20\% miss chance). Even creatures that can normally see in such conditions (such as with darkvision or low-light vision) have the miss chance in an area shrouded in magical darkness.

\smallskip\noindent Normal lights (torches, candles, lanterns, and so forth) are incapable of brightening the area, as are light spells of lower level. Higher level light spells are not affected by darkness.

\smallskip\noindent If darkness is cast on a small object that is then placed inside or under a lightproof covering, the spell's effect is blocked until the covering is removed.
Darkness counters or dispels any light spell of equal or lower spell level.

\shortability{Arcane Material Component:}{A bit of bat fur and either a drop of pitch or a piece of coal.}

\featname{Darkvision}
\shortability{Transmutation}{}
\shortability{Level:}{Rgr 3, Sor/Wiz 2}
\shortability{Components:}{V, S, M}
\shortability{Casting Time:}{1 standard action}
\shortability{Range:}{Touch}
\shortability{Target:}{Creature touched}
\shortability{Duration:}{1 hour/level}
\shortability{Saving Throw:}{Will negates (harmless)}
\shortability{Spell Resistance:}{Yes (harmless)}
\smallskip\noindent The subject gains the ability to see 60 feet even in total darkness. Darkvision is black and white only but otherwise like normal sight. Darkvision does not grant one the ability to see in magical darkness.

\smallskip\noindent Darkvision can be made permanent with a permanency spell.

\shortability{Material Component:}{Either a pinch of dried carrot or an agate.}

\featname{Daylight}
\shortability{Evocation}{[Light]}
\shortability{Level:}{Brd 3, Clr 3, Drd 3, Pal 3, Sor/Wiz 3}
\shortability{Components:}{V, S}
\shortability{Casting Time:}{1 standard action}
\shortability{Range:}{Touch}
\shortability{Target:}{Object touched}
\shortability{Duration:}{10 min./level (D)}
\shortability{Saving Throw:}{None}
\shortability{Spell Resistance:}{No}
\smallskip\noindent The object touched sheds light as bright as full daylight in a 60-foot radius, and dim light for an additional 60 feet beyond that. Creatures that take penalties in bright light also take them while within the radius of this magical light. Despite its name, this spell is not the equivalent of daylight for the purposes of creatures that are damaged or destroyed by bright light.

\smallskip\noindent If daylight is cast on a small object that is then placed inside or under a light- proof covering, the spell's effects are blocked until the covering is removed.

\smallskip\noindent Daylight brought into an area of magical darkness (or vice versa) is temporarily negated, so that the otherwise prevailing light conditions exist in the overlapping areas of effect.

\smallskip\noindent Daylight counters or dispels any darkness spell of equal or lower level, such as darkness.

\featname{Daze}
\shortability{Enchantment}{(Compulsion) [Mind-Affecting]}
\shortability{Level:}{Brd 0, Sor/Wiz 0}
\shortability{Components:}{V, S, M}
\shortability{Casting Time:}{1 standard action}
\shortability{Range:}{Close (25 ft. + 5 ft./2 levels)}
\shortability{Target:}{One humanoid creature of 4 HD or less}
\shortability{Duration:}{1 round}
\shortability{Saving Throw:}{Will negates}
\shortability{Spell Resistance:}{Yes}
\smallskip\noindent This enchantment clouds the mind of a humanoid creature with 4 or fewer Hit Dice so that it takes no actions. Humanoids of 5 or more HD are not affected. A dazed subject is not stunned, so attackers get no special advantage against it.

\shortability{Material Component:}{A pinch of wool or similar substance.}

\featname{Daze Monster}
\shortability{Enchantment}{(Compulsion) [Mind-Affecting]}
\shortability{Level:}{Brd 2, Sor/Wiz 2}
\shortability{Range:}{Medium (100 ft. + 10 ft./level)}
\shortability{Target:}{One living creature of 6 HD or less}
\smallskip\noindent This spell functions like daze, but daze monster can affect any one living creature of any type. Creatures of 7 or more HD are not affected.

\featname{Death Knell}
\shortability{Necromancy}{[Death, Evil]}
\shortability{Level:}{Clr 2, Death 2}
\shortability{Components:}{V, S}
\shortability{Casting Time:}{1 standard action}
\shortability{Range:}{Touch}
\shortability{Target:}{Living creature touched}
\shortability{Duration:}{Instantaneous/10 minutes per HD of subject; see text}
\shortability{Saving Throw:}{Will negates}
\shortability{Spell Resistance:}{Yes}
\smallskip\noindent You draw forth the ebbing life force of a creature and use it to fuel your own power. Upon casting this spell, you touch a living creature that has -1 or fewer hit points. If the subject fails its saving throw, it dies, and you gain 1d8 temporary hit points and a +2 bonus to Strength. Additionally, your effective caster level goes up by +1, improving spell effects dependent on caster level. (This increase in effective caster level does not grant you access to more spells.) These effects last for 10 minutes per HD of the subject creature.

\featname{Death Ward}
\shortability{Necromancy}{}
\shortability{Level:}{Clr 4, Death 4, Drd 5, Pal 4}
\shortability{Components:}{V, S, DF}
\shortability{Casting Time:}{1 standard action}
\shortability{Range:}{Touch}
\shortability{Target:}{Living creature touched}
\shortability{Duration:}{1 min./level}
\shortability{Saving Throw:}{Will negates (harmless)}
\shortability{Spell Resistance:}{Yes (harmless)}
\smallskip\noindent The subject is immune to all death spells, magical death effects, energy drain, and any negative energy effects.

\smallskip\noindent This spell doesn't remove negative levels that the subject has already gained, nor does it affect the saving throw necessary 24 hours after gaining a negative level.

\smallskip\noindent Death ward does not protect against other sorts of attacks even if those attacks might be lethal.

\featname{Deathwatch}
\shortability{Necromancy}{[Evil]}
\shortability{Level:}{Clr 1}
\shortability{Components:}{V, S}
\shortability{Casting Time:}{1 standard action}
\shortability{Range:}{30 ft.}
\shortability{Area:}{Cone-shaped emanation}
\shortability{Duration:}{10 min./level}
\shortability{Saving Throw:}{None}
\shortability{Spell Resistance:}{No}
\smallskip\noindent Using the foul sight granted by the powers of unlife, you can determine the condition of creatures near death within the spell's range. You instantly know whether each creature within the area is dead, fragile (alive and wounded, with 3 or fewer hit points left), fighting off death (alive with 4 or more hit points), undead, or neither alive nor dead (such as a construct).

\smallskip\noindent Deathwatch sees through any spell or ability that allows creatures to feign death.

\featname{Deep Slumber}
\shortability{Enchantment}{(Compulsion) [Mind-Affecting]}
\shortability{Level:}{Brd 3, Sor/Wiz 3}
\shortability{Range:}{Close (25 ft. + 5 ft./2 levels)}
\smallskip\noindent This spell functions like sleep, except that it affects 10 HD of creatures.

\featname{Deeper Darkness}
\shortability{Evocation}{[Darkness]}
\shortability{Level:}{Clr 3}
\shortability{Duration:}{One day/level (D)}
\smallskip\noindent This spell functions like darkness, except that the object radiates shadowy illumination in a 60-foot radius and the darkness lasts longer.

\smallskip\noindent Daylight brought into an area of deeper darkness (or vice versa) is temporarily negated, so that the otherwise prevailing light conditions exist in the overlapping areas of effect.

\smallskip\noindent Deeper darkness counters and dispels any light spell of equal or lower level, including daylight and light.

\featname{Delay Poison}
\shortability{Conjuration}{(Healing)}
\shortability{Level:}{Brd 2, Clr 2, Drd 2, Pal 2, Rgr 1}
\shortability{Components:}{V, S, DF}
\shortability{Casting Time:}{1 standard action}
\shortability{Range:}{Touch}
\shortability{Target:}{Creature touched}
\shortability{Duration:}{1 hour/level}
\shortability{Saving Throw:}{Fortitude negates (harmless)}
\shortability{Spell Resistance:}{Yes (harmless)}
\smallskip\noindent The subject becomes temporarily immune to poison. Any poison in its system or any poison to which it is exposed during the spell's duration does not affect the subject until the spell's duration has expired. Delay poison does not cure any damage that poison may have already done.

\featname{Delayed Blast Fireball}
\shortability{Evocation}{[Fire]}
\shortability{Level:}{Sor/Wiz 7}
\shortability{Duration:}{5 rounds or less; see text}
\smallskip\noindent This spell functions like fireball, except that it is more powerful and can detonate up to 5 rounds after the spell is cast. The burst of flame deals 1d6 points of fire damage per caster level (maximum 20d6).

\smallskip\noindent The glowing bead created by delayed blast fireball can detonate immediately if you desire, or you can choose to delay the burst for as many as 5 rounds. You select the amount of delay upon completing the spell, and that time cannot change once it has been set unless someone touches the bead (see below). If you choose a delay, the glowing bead sits at its destination until it detonates. A creature can pick up and hurl the bead as a thrown weapon (range increment 10 feet). If a creature handles and moves the bead within 1 round of its detonation, there is a 25\% chance that the bead detonates while being handled.

\featname{Demand}
\shortability{Enchantment}{(Compulsion) [Mind-Affecting]}
\shortability{Level:}{Sor/Wiz 8}
\shortability{Saving Throw:}{Will partial}
\shortability{Spell Resistance:}{Yes}
\smallskip\noindent This spell functions like sending, but the message can also contain a suggestion (see the suggestion spell), which the subject does its best to carry out. A successful Will save negates the suggestion effect but not the contact itself. The demand, if received, is understood even if the subject's Intelligence score is as low as 1. If the message is impossible or meaningless according to the circumstances that exist for the subject at the time the demand is issued, the message is understood but the suggestion is ineffective.

\smallskip\noindent The demand's message to the creature must be twenty-five words or less, including the suggestion. The creature can also give a short reply immediately.

\shortability{Material Component:}{A short piece of copper wire and some small part of the subject-a hair, a bit of nail, or the like.}

\featname{Desecrate}
\shortability{Evocation}{[Evil]}
\shortability{Level:}{Clr 2, Evil 2}
\shortability{Components:}{V, S, M, DF}
\shortability{Casting Time:}{1 standard action}
\shortability{Range:}{Close (25 ft. + 5 ft./2 levels)}
\shortability{Area:}{20-ft.-radius emanation}
\shortability{Duration:}{2 hours/level}
\shortability{Saving Throw:}{None}
\shortability{Spell Resistance:}{Yes}
\smallskip\noindent This spell imbues an area with negative energy. Each Charisma check made to turn undead within this area takes a -3 profane penalty, and every undead creature entering a desecrated area gains a +1 profane bonus on attack rolls, damage rolls, and saving throws. An undead creature created within or summoned into such an area gains +1 hit points per HD.

\smallskip\noindent If the desecrated area contains an altar, shrine, or other permanent fixture dedicated to your deity or aligned higher power, the modifiers given above are doubled (-6 profane penalty on turning checks, +2 profane bonus and +2 hit points per HD for undead in the area).

\smallskip\noindent Furthermore, anyone who casts animate dead within this area may create as many as double the normal amount of undead (that is, 4 HD per caster level rather than 2 HD per caster level).

\smallskip\noindent If the area contains an altar, shrine, or other permanent fixture of a deity, pantheon, or higher power other than your patron, the desecrate spell instead curses the area, cutting off its connection with the associated deity or power. This secondary function, if used, does not also grant the bonuses and penalties relating to undead, as given above.

\smallskip\noindent Desecrate counters and dispels consecrate.

\shortability{Material Component:}{A vial of unholy water and 25 gp worth (5 pounds) of silver dust, all of which must be sprinkled around the area.}

\featname{Destruction}
\shortability{Necromancy}{[Death]}
\shortability{Level:}{Clr 7, Death 7}
\shortability{Components:}{V, S, F}
\shortability{Casting Time:}{1 standard action}
\shortability{Range:}{Close (25 ft. + 5 ft./2 levels)}
\shortability{Target:}{One creature}
\shortability{Duration:}{Instantaneous}
\shortability{Saving Throw:}{Fortitude partial}
\shortability{Spell Resistance:}{Yes}
\smallskip\noindent This spell instantly slays the subject and consumes its remains (but not its equipment and possessions) utterly. If the target's Fortitude saving throw succeeds, it instead takes 10d6 points of damage. The only way to restore life to a character who has failed to save against this spell is to use true resurrection, a carefully worded wish spell followed by resurrection, or miracle.

\shortability{Focus:}{A special holy (or unholy) symbol of silver marked with verses of anathema (cost 500 gp).}

\featname{Detect Animals or Plants}
\shortability{Divination}{}
\shortability{Level:}{Drd 1, Rgr 1}
\shortability{Components:}{V, S}
\shortability{Casting Time:}{1 standard action}
\shortability{Range:}{Long (400 ft. + 40 ft./level)}
\shortability{Area:}{Cone-shaped emanation}
\shortability{Duration:}{Concentration, up to 10 min./level (D)}
\shortability{Saving Throw:}{None}
\shortability{Spell Resistance:}{No}

\smallskip\noindent You can detect a particular kind of animal or plant in a cone emanating out from you in whatever direction you face. You must think of a kind of animal or plant when using the spell, but you can change the animal or plant kind each round. The amount of information revealed depends on how long you search a particular area or focus on a specific kind of animal or plant.

\shortability{1st Round:}{Presence or absence of that kind of animal or plant in the area.}
\shortability{2nd Round:}{Number of individuals of the specified kind in the area, and the condition of the healthiest specimen.}
\shortability{3rd Round:}{The condition (see below) and location of each individual present. If an animal or plant is outside your line of sight, then you discern its direction but not its exact location.}
\shortability{Conditions:}{For purposes of this spell, the categories of condition are as follows:}
\shortability{Normal:}{Has at least 90\% of full normal hit points, free of disease.}
\shortability{Fair:}{30\% to 90\% of full normal hit points remaining.}
\shortability{Poor:}{Less than 30\% of full normal hit points remaining, afflicted with a disease, or suffering from a debilitating injury.}
\shortability{Weak:}{0 or fewer hit points remaining, afflicted with a disease in the terminal stage, or crippled.}
If a creature falls into more than one category, the spell indicates the weaker of the two.

\smallskip\noindent Each round you can turn to detect a kind of animal or plant in a new area. The spell can penetrate barriers, but 1 foot of stone, 1 inch of common metal, a thin sheet of lead, or 3 feet of wood or dirt blocks it.

\featname{Detect Chaos}
\shortability{Divination}{}
\shortability{Level:}{Clr 1}
\smallskip\noindent This spell functions like detect evil, except that it detects the auras of chaotic creatures, clerics of chaotic deities, chaotic spells, and chaotic magic items, and you are vulnerable to an overwhelming chaotic aura if you are lawful.

\featname{Detect Evil}
\shortability{Divination}{}
\shortability{Level:}{Clr 1}
\shortability{Components:}{V, S, DF}
\shortability{Casting Time:}{1 standard action}
\shortability{Range:}{60 ft.}
\shortability{Area:}{Cone-shaped emanation}
\shortability{Duration:}{Concentration, up to 10 min./ level (D)}
\shortability{Saving Throw:}{None}
\shortability{Spell Resistance:}{No}
\smallskip\noindent You can sense the presence of evil. The amount of information revealed depends on how long you study a particular area or subject.

\shortability{1st Round:}{Presence or absence of evil.}
\shortability{2nd Round:}{Number of evil auras (creatures, objects, or spells) in the area and the power of the most potent evil aura present. If you are of good alignment, and the strongest evil aura's power is overwhelming (see below), and the HD or level of the aura's source is at least twice your character level, you are stunned for 1 round and the spell ends.}
\shortability{3rd Round:}{The power and location of each aura. If an aura is outside your line of sight, then you discern its direction but not its exact location.}
\shortability{Aura Power:}{An evil aura's power depends on the type of evil creature or object that you're detecting and its HD, caster level, or (in the case of a cleric) class level; see the accompanying table. If an aura falls into more than one strength category, the spell indicates the stronger of the two. Refer to the Detect Evil: Aura Power table.}
\shortability{Lingering Aura:}{An evil aura lingers after its original source dissipates (in the case of a spell) or is destroyed (in the case of a creature or magic item). If detect evil is cast and directed at such a location, the spell indicates an aura strength of dim (even weaker than a faint aura). How long the aura lingers at this dim level depends on its original power.}

%This table needs to be placed on the page above Detect Evil so because table*s will appear on the next page in the multicol environment.
\begin{table*}[b!]
\caption{Detect Evil: Aura Power}
\begin{tabular}[h!]{l|llll}
Creature/Object &	Faint & Moderate & Strong & Overwhelming \\ \hline
Evil creature\textsuperscript{1} (HD) & 10 or lower &	11-25 & 26-50 & 51 or higher \\	   
Undead (HD) & 2 or lower & 3-8 & 9-20	21 & or higher \\
Evil outsider (HD) & 1 or lower & 2-4 & 5-10 & 11 or higher \\
Cleric of an evil deity \textsuperscript{2} (class levels) & 1 & 2-4 & 5-10 & 11 or higher \\
Evil magic item or spell (caster level) & 2nd or lower & 3rd-8th & 9th-20th & 21st or higher \\
\multicolumn{5}{l}{\textsuperscript{1}{Except for undead and outsiders, which have their own entries on the table.}} \\
\multicolumn{5}{l}{\textsuperscript{2}{Some characters who are not clerics may radiate an aura of equivalent power. The class description will indicate whether this applies.}} \\
\end{tabular}
\end{table*}

\begin{tabular}[h!]{l|l}
Original Strength & Duration of Lingering Aura \\
Faint & 1d6 rounds \\
Moderate & 1d6 minutes \\
Strong & 1d6x10 minutes \\
Overwhelming & 1d6 days \\
\end{tabular}

\smallskip\noindent Animals, traps, poisons, and other potential perils are not evil, and as such this spell does not detect them.

\smallskip\noindent Each round, you can turn to detect evil in a new area. The spell can penetrate barriers, but 1 foot of stone, 1 inch of common metal, a thin sheet of lead, or 3 feet of wood or dirt blocks it.

\featname{Detect Good}
\shortability{Divination}{}
\shortability{Level:}{Clr 1}
\smallskip\noindent This spell functions like detect evil, except that it detects the auras of good creatures, clerics or paladins of good deities, good spells, and good magic items, and you are vulnerable to an overwhelming good aura if you are evil. Healing potions, antidotes, and similar beneficial items are not good.

\featname{Detect Law}
\shortability{Divination}{}
\shortability{Level:}{Clr 1}
\smallskip\noindent This spell functions like detect evil, except that it detects the auras of lawful creatures, clerics of lawful deities, lawful spells, and lawful magic items, and you are vulnerable to an overwhelming lawful aura if you are chaotic.

\featname{Detect Magic}
\shortability{Divination}{}
\shortability{Level:}{Brd 0, Clr 0, Drd 0, Sor/Wiz 0}
\shortability{Components:}{V, S}
\shortability{Casting Time:}{1 standard action}
\shortability{Range:}{60 ft.}
\shortability{Area:}{Cone-shaped emanation}
\shortability{Duration:}{Concentration, up to 1 min./level (D)}
\shortability{Saving Throw:}{None}
\shortability{Spell Resistance:}{No}
\smallskip\noindent You detect magical auras. The amount of information revealed depends on how long you study a particular area or subject.

\shortability{1st Round:}{Presence or absence of magical auras.}
\shortability{2nd Round:}{Number of different magical auras and the power of the most potent aura.}
\shortability{3rd Round:}{The strength and location of each aura. If the items or creatures bearing the auras are in line of sight, you can make Spellcraft skill checks to determine the school of magic involved in each. (Make one check per aura; DC 15 + spell level, or 15 + half caster level for a nonspell effect.)}

\smallskip\noindent Magical areas, multiple types of magic, or strong local magical emanations may distort or conceal weaker auras.

\shortability{Aura Strength:}{An aura's power depends on a spell's functioning spell level or an item's caster level. If an aura falls into more than one category, detect magic indicates the stronger of the two. Refer to table Detect Magic: Aura Power}
\shortability{Lingering Aura:}{A magical aura lingers after its original source dissipates (in the case of a spell) or is destroyed (in the case of a magic item). If detect magic is cast and directed at such a location, the spell indicates an aura strength of dim (even weaker than a faint aura). How long the aura lingers at this dim level depends on its original power.}

%This table needs to be on the page before Detect Magic
\begin{table*}[b!]
\caption{Detect Magic: Aura Power}
\begin{tabular}[h!]{l|llll}
Spell or Object & Faint & Moderate & Strong & Overwhelming \\ \hline
Functioning spell (spell level) & 3rd or lower & 4th-6th & 7th-9th & 10th+ (deity-level) \\
Magic item (caster level) & 5th or lower & 6th-11th & 12th-20th & 21st+ (artifact) \\
\end{tabular}
\end{table*}

\begin{tabular}[h!]{l|l}
Original Strength & Duration of Lingering Aura \\
Faint & 1d6 rounds \\
Moderate & 1d6 minutes \\
Strong & 1d6x10 minutes \\
Overwhelming & 1d6 days \\
\end{tabular}
\smallskip\noindent Outsiders and elementals are not magical in themselves, but if they are summoned, the conjuration spell registers.

\smallskip\noindent Each round, you can turn to detect magic in a new area. The spell can penetrate barriers, but 1 foot of stone, 1 inch of common metal, a thin sheet of lead, or 3 feet of wood or dirt blocks it.

\smallskip\noindent Detect magic can be made permanent with a permanency spell.

\featname{Detect Poison}
\shortability{Divination}{}
\shortability{Level:}{Clr 0, Drd 0, Pal 1, Rgr 1, Sor/Wiz 0}
\shortability{Components:}{V, S}
\shortability{Casting Time:}{1 standard action}
\shortability{Range:}{Close (25 ft. + 5 ft./2 levels)}
Target or \shortability{Area:}{One creature, one object, or a 5-ft. cube}
\shortability{Duration:}{Instantaneous}
\shortability{Saving Throw:}{None}
\shortability{Spell Resistance:}{No}
\smallskip\noindent You determine whether a creature, object, or area has been poisoned or is poisonous. You can determine the exact type of poison with a DC 20 Wisdom check. A character with the Craft (alchemy) skill may try a DC 20 Craft (alchemy) check if the Wisdom check fails, or may try the Craft (alchemy) check prior to the Wisdom check.

\smallskip\noindent The spell can penetrate barriers, but 1 foot of stone, 1 inch of common metal, a thin sheet of lead, or 3 feet of wood or dirt blocks it.

\featname{Detect Scrying}
\shortability{Divination}{}
\shortability{Level:}{Brd 4, Sor/Wiz 4}
\shortability{Components:}{V, S, M}
\shortability{Casting Time:}{1 standard action}
\shortability{Range:}{40 ft.}
\shortability{Area:}{40-ft.-radius emanation centered on you}
\shortability{Duration:}{24 hours}
\shortability{Saving Throw:}{None}
\shortability{Spell Resistance:}{No}
\smallskip\noindent You immediately become aware of any attempt to observe you by means of a divination (scrying) spell or effect. The spell's area radiates from you and moves as you move. You know the location of every magical sensor within the spell's area.

\smallskip\noindent If the scrying attempt originates within the area, you also know its location; otherwise, you and the scrier immediately make opposed caster level checks (1d20 + caster level). If you at least match the scrier's result, you get a visual image of the scrier and an accurate sense of his or her direction and distance from you.

\shortability{Material Component:}{A small piece of mirror and a miniature brass hearing trumpet.}

\featname{Detect Secret Doors}
\shortability{Divination}{}
\shortability{Level:}{Brd 1, Knowledge 1, Sor/Wiz 1}
\shortability{Components:}{V, S}
\shortability{Casting Time:}{1 standard action}
\shortability{Range:}{60 ft.}
\shortability{Area:}{Cone-shaped emanation}
\shortability{Duration:}{Concentration, up to 1 min./level (D)}
\shortability{Saving Throw:}{None}
\shortability{Spell Resistance:}{No}
\smallskip\noindent You can detect secret doors, compartments, caches, and so forth. Only passages, doors, or openings that have been specifically constructed to escape detection are detected by this spell. The amount of information revealed depends on how long you study a particular area or subject.

\shortability{1st Round:}{Presence or absence of secret doors.}
\shortability{2nd Round:}{Number of secret doors and the location of each. If an aura is outside your line of sight, then you discern its direction but not its exact location.}
\shortability{Each Additional Round:}{The mechanism or trigger for one particular secret portal closely examined by you. Each round, you can turn to detect secret doors in a new area. The spell can penetrate barriers, but 1 foot of stone, 1 inch of common metal, a thin sheet of lead, or 3 feet of wood or dirt blocks it.}

\featname{Detect Snares and Pits}
\shortability{Divination}{}
\shortability{Level:}{Drd 1, Rgr 1}
\shortability{Components:}{V, S}
\shortability{Casting Time:}{1 standard action}
\shortability{Range:}{60 ft.}
\shortability{Area:}{Cone-shaped emanation}
\shortability{Duration:}{Concentration, up to 10 min./level (D)}
\shortability{Saving Throw:}{None}
\shortability{Spell Resistance:}{No}
\smallskip\noindent You can detect simple pits, deadfalls, and snares as well as mechanical traps constructed of natural materials. The spell does not detect complex traps, including trapdoor traps.

\smallskip\noindent Detect snares and pits does detect certain natural hazards-quicksand (a snare), a sinkhole (a pit), or unsafe walls of natural rock (a deadfall). However, it does not reveal other potentially dangerous conditions. The spell does not detect magic traps (except those that operate by pit, deadfall, or snaring; see the spell snare), nor mechanically complex ones, nor those that have been rendered safe or inactive.

\smallskip\noindent The amount of information revealed depends on how long you study a particular area.

\shortability{1st Round:}{Presence or absence of hazards.}
\shortability{2nd Round:}{Number of hazards and the location of each. If a hazard is outside your line of sight, then you discern its direction but not its exact location.}
\shortability{Each Additional Round:}{The general type and trigger for one particular hazard closely examined by you.}

\smallskip\noindent Each round, you can turn to detect snares and pits in a new area. The spell can penetrate barriers, but 1 foot of stone, 1 inch of common metal, a thin sheet of lead, or 3 feet of wood or dirt blocks it.

\featname{Detect Thoughts}
\shortability{Divination}{[Mind-Affecting]}
\shortability{Level:}{Brd 2, Knowledge 2, Sor/Wiz 2}
\shortability{Components:}{V, S, F/DF}
\shortability{Casting Time:}{1 standard action}
\shortability{Range:}{60 ft.}
\shortability{Area:}{Cone-shaped emanation}
\shortability{Duration:}{Concentration, up to 1 min./level (D)}
\shortability{Saving Throw:}{Will negates; see text}
\shortability{Spell Resistance:}{No}
\smallskip\noindent You detect surface thoughts. The amount of information revealed depends on how long you study a particular area or subject.

\shortability{1st Round:}{Presence or absence of thoughts (from conscious creatures with Intelligence scores of 1 or higher).}
\shortability{2nd Round:}{Number of thinking minds and the Intelligence score of each. If the highest Intelligence is 26 or higher (and at least 10 points higher than your own Intelligence score), you are stunned for 1 round and the spell ends. This spell does not let you determine the location of the thinking minds if you can't see the creatures whose thoughts you are detecting.}
\shortability{3rd Round:}{Surface thoughts of any mind in the area. A target's Will save prevents you from reading its thoughts, and you must cast detect thoughts again to have another chance. Creatures of animal intelligence (Int 1 or 2) have simple, instinctual thoughts that you can pick up.}

\smallskip\noindent Each round, you can turn to detect thoughts in a new area. The spell can penetrate barriers, but 1 foot of stone, 1 inch of common metal, a thin sheet of lead, or 3 feet of wood or dirt blocks it.

\shortability{Arcane Focus:}{A copper piece.}

\featname{Detect Undead}
\shortability{Divination}{}
\shortability{Level:}{Clr 1, Pal 1, Sor/Wiz 1}
\shortability{Components:}{V, S, M/DF}
\shortability{Casting Time:}{1 standard action}
\shortability{Range:}{60 ft.}
\shortability{Area:}{Cone-shaped emanation}
\shortability{Duration:}{Concentration, up to 1 minute/ level (D)}
\shortability{Saving Throw:}{None}
\shortability{Spell Resistance:}{No}
\smallskip\noindent You can detect the aura that surrounds undead creatures. The amount of information revealed depends on how long you study a particular area.

\shortability{1st Round:}{Presence or absence of undead auras.}
\shortability{2nd Round:}{Number of undead auras in the area and the strength of the strongest undead aura present. If you are of good alignment, and the strongest undead aura's strength is overwhelming (see below), and the creature has HD of at least twice your character level, you are stunned for 1 round and the spell ends.}
\shortability{3rd Round:}{The strength and location of each undead aura. If an aura is outside your line of sight, then you discern its direction but not its exact location.}
\shortability{Aura Strength:}{The strength of an undead aura is determined by the HD of the undead creature, as given on the following table.}

\begin{tabular}[h!]{l|l}
HD & Strength \\ \hline
1 or lower & Faint \\	   
2-4 & Moderate \\	   
5-10 & Strong	\\
11 or higher & Overwhelming \\
\end{tabular}
	 
\shortability{Lingering Aura:}{An undead aura lingers after its original source is destroyed. If detect undead is cast and directed at such a location, the spell indicates an aura strength of dim (even weaker than a faint aura). How long the aura lingers at this dim level depends on its original power:}

\begin{tabular}[h!]{l|l} 
Original Strength & Duration of Lingering Aura \\ \hline
Faint & 1d6 rounds \\
Moderate & 1d6 minutes \\
Strong & 1d6x10 minutes	\\
Overwhelming & 1d6 days \\
\end{tabular}
	 
\smallskip\noindent Each round, you can turn to detect undead in a new area. The spell can penetrate barriers, but 1 foot of stone, 1 inch of common metal, a thin sheet of lead, or 3 feet of wood or dirt blocks it.

\shortability{Arcane Material Component:}{A bit of earth from a grave.}

\featname{Dictum}
\shortability{Evocation}{[Lawful, Sonic]}
\shortability{Level:}{Clr 7, Law 7}
\shortability{Components:}{V}
\shortability{Casting Time:}{1 standard action}
\shortability{Range:}{40 ft.}
\shortability{Area:}{Nonlawful creatures in a 40-ft.-radius spread centered on you}
\shortability{Duration:}{Instantaneous}
\shortability{Saving Throw:}{None or Will negates; see text}
\shortability{Spell Resistance:}{Yes}
\smallskip\noindent Any nonlawful creature within the area of a dictum spell suffers the following ill effects.

\begin{tabular}[h!]{l|l}
HD & Effect	\\ \hline
Equal to caster level & Deafened \\
Up to caster level -1 & Slowed, deafened \\
Up to caster level -5 & Paralyzed, slowed, deafened \\
Up to caster level -10 & Killed, paralyzed, slowed, deafened \\
\end{tabular}

\smallskip\noindent The effects are cumulative and concurrent. No saving throw is allowed against these effects.

\shortability{Deafened:}{The creature is deafened for 1d4 rounds.}
\shortability{Slowed:}{The creature is slowed, as by the slow spell, for 2d4 rounds.}
\shortability{Paralyzed:}{The creature is paralyzed and helpless for 1d10 minutes.}
\shortability{Killed:}{Living creatures die. Undead creatures are destroyed.}

\smallskip\noindent Furthermore, if you are on your home plane when you cast this spell, nonlawful extraplanar creatures within the area are instantly banished back to their home planes. Creatures so banished cannot return for at least 24 hours. This effect takes place regardless of whether the creatures hear the dictum. The banishment effect allows a Will save (at a -4 penalty) to negate.

\smallskip\noindent Creatures whose HD exceed your caster level are unaffected by dictum.

\featname{Dimension Door}
\shortability{Conjuration}{(Teleportation)}
\shortability{Level:}{Brd 4, Sor/Wiz 4, Travel 4}
\shortability{Components:}{V}
\shortability{Casting Time:}{1 standard action}
\shortability{Range:}{Long (400 ft. + 40 ft./level)}
\shortability{Target:}{You and touched objects or other touched willing creatures}
\shortability{Duration:}{Instantaneous}
\shortability{Saving Throw:}{None and Will negates (object)}
\shortability{Spell Resistance:}{No and Yes (object)}
\smallskip\noindent You instantly transfer yourself from your current location to any other spot within range. You always arrive at exactly the spot desired-whether by simply visualizing the area or by stating direction. After using this spell, you can't take any other actions until your next turn. You can bring along objects as long as their weight doesn't exceed your maximum load. You may also bring one additional willing Medium or smaller creature (carrying gear or objects up to its maximum load) or its equivalent per three caster levels. A Large creature counts as two Medium creatures, a Huge creature counts as two Large creatures, and so forth. All creatures to be transported must be in contact with one another, and at least one of those creatures must be in contact with you.

\smallskip\noindent If you arrive in a place that is already occupied by a solid body, you and each creature traveling with you take 1d6 points of damage and are shunted to a random open space on a suitable surface within 100 feet of the intended location.

\smallskip\noindent If there is no free space within 100 feet, you and each creature traveling with you take an additional 2d6 points of damage and are shunted to a free space within 1,000 feet. If there is no free space within 1,000 feet, you and each creature travelling with you take an additional 4d6 points of damage and the spell simply fails.

\featname{Dimensional Anchor}
\shortability{Abjuration}{}
\shortability{Level:}{Clr 4, Sor/Wiz 4}
\shortability{Components:}{V, S}
\shortability{Casting Time:}{1 standard action}
\shortability{Range:}{Medium (100 ft. + 10 ft./level)}
\shortability{Effect:}{Ray}
\shortability{Duration:}{1 min./level}
\shortability{Saving Throw:}{None}
\shortability{Spell Resistance:}{Yes (object)}
\smallskip\noindent A green ray springs from your outstretched hand. You must make a ranged touch attack to hit the target. Any creature or object struck by the ray is covered with a shimmering emerald field that completely blocks extradimensional travel. Forms of movement barred by a dimensional anchor include astral projection, blink, dimension door, ethereal jaunt, etherealness, gate, maze, plane shift, shadow walk, teleport, and similar spell-like or psionic abilities. The spell also prevents the use of a gate or teleportation circle for the duration of the spell.

\smallskip\noindent A dimensional anchor does not interfere with the movement of creatures already in ethereal or astral form when the spell is cast, nor does it block extradimensional perception or attack forms. Also, dimensional anchor does not prevent summoned creatures from disappearing at the end of a summoning spell.

\featname{Dimensional Lock}
\shortability{Abjuration}{}
\shortability{Level:}{Clr 8, Sor/Wiz 8}
\shortability{Components:}{V, S}
\shortability{Casting Time:}{1 standard action}
\shortability{Range:}{Medium (100 ft. + 10 ft./level)}
\shortability{Area:}{20-ft.-radius emanation centered on a point in space}
\shortability{Duration:}{One day/level}
\shortability{Saving Throw:}{None}
\shortability{Spell Resistance:}{Yes}
\smallskip\noindent You create a shimmering emerald barrier that completely blocks extradimensional travel. Forms of movement barred include astral projection, blink, dimension door, ethereal jaunt, etherealness, gate, maze, plane shift, shadow walk, teleport, and similar spell-like or psionic abilities. Once dimensional lock is in place, extradimensional travel into or out of the area is not possible.

\smallskip\noindent A dimensional lock does not interfere with the movement of creatures already in ethereal or astral form when the spell is cast, nor does it block extradimensional perception or attack forms. Also, the spell does not prevent summoned creatures from disappearing at the end of a summoning spell.

\featname{Diminish Plants}
\shortability{Transmutation}{}
\shortability{Level:}{Drd 3, Rgr 3}
\shortability{Components:}{V, S, DF}
\shortability{Casting Time:}{1 standard action}
\shortability{Range:}{See text}
\shortability{Target or Area:}{See text}
\shortability{Duration:}{Instantaneous}
\shortability{Saving Throw:}{None}
\shortability{Spell Resistance:}{No}
\smallskip\noindent This spell has two versions.

\shortability{Prune Growth:}{This version causes normal vegetation within long range (400 feet + 40 feet per level) to shrink to about one-third of their normal size, becoming untangled and less bushy. The affected vegetation appears to have been carefully pruned and trimmed.

\smallskip\noindent At your option, the area can be a 100- foot-radius circle, a 150-foot-radius semicircle, or a 200-foot-radius quarter-circle. You may also designate portions of the area that are not affected.}

\shortability{Stunt Growth:}{This version targets normal plants within a range of 1/2 mile, reducing their potential productivity over the course of the following year to one third below normal.}

\smallskip\noindent Diminish plants counters plant growth.

\smallskip\noindent This spell has no effect on plant creatures.

\featname{Discern Lies}
\shortability{Divination}{}
\shortability{Level:}{Clr 4, Pal 3}
\shortability{Components:}{V, S, DF}
\shortability{Casting Time:}{1 standard action}
\shortability{Range:}{Close (25 ft. + 5 ft./2 levels)}
\shortability{Targets:}{One creature/level, no two of which can be more than 30 ft. apart}
\shortability{Duration:}{Concentration, up to 1 round/level}
\shortability{Saving Throw:}{Will negates}
\shortability{Spell Resistance:}{No}
\smallskip\noindent Each round, you concentrate on one subject, who must be within range. You know if the subject deliberately and knowingly speaks a lie by discerning disturbances in its aura caused by lying. The spell does not reveal the truth, uncover unintentional inaccuracies, or necessarily reveal evasions.

\smallskip\noindent Each round, you may concentrate on a different subject.

\featname{Discern Location}
\shortability{Divination}{}
\shortability{Level:}{Clr 8, Knowledge 8, Sor/Wiz 8}
\shortability{Components:}{V, S, DF}
\shortability{Casting Time:}{10 minutes}
\shortability{Range:}{Unlimited}
\shortability{Target:}{One creature or object}
\shortability{Duration:}{Instantaneous}
\shortability{Saving Throw:}{None}
\shortability{Spell Resistance:}{No}
\smallskip\noindent A discern location spell is among the most powerful means of locating creatures or objects. Nothing short of a mind blank spell or the direct intervention of a deity keeps you from learning the exact location of a single individual or object. Discern location circumvents normal means of protection from scrying or location. The spell reveals the name of the creature or object's location (place, name, business name, building name, or the like), community, county (or similar political division), country, continent, and the plane of existence where the target lies.

\smallskip\noindent To find a creature with the spell, you must have seen the creature or have some item that once belonged to it. To find an object, you must have touched it at least once.

\featname{Disguise Self}
\shortability{Illusion}{(Glamer)}
\shortability{Level:}{Brd 1, Sor/Wiz 1, Trickery 1}
\shortability{Components:}{V, S}
\shortability{Casting Time:}{1 standard action}
\shortability{Range:}{Personal}
\shortability{Target:}{You}
\shortability{Duration:}{10 min./level (D)}
\smallskip\noindent You make yourself-including clothing, armor, weapons, and equipment-look different. You can seem 1 foot shorter or taller, thin, fat, or in between. You cannot change your body type. Otherwise, the extent of the apparent change is up to you. You could add or obscure a minor feature or look like an entirely different person.

\smallskip\noindent The spell does not provide the abilities or mannerisms of the chosen form, nor does it alter the perceived tactile (touch) or audible (sound) properties of you or your equipment. 

\smallskip\noindent If you use this spell to create a disguise, you get a +10 bonus on the Disguise check.

\smallskip\noindent A creature that interacts with the glamer gets a Will save to recognize it as an illusion.

\featname{Disintegrate}
\shortability{Transmutation}{}
\shortability{Level:}{Destruction 7, Sor/Wiz 6}
\shortability{Components:}{V, S, M/DF}
\shortability{Casting Time:}{1 standard action}
\shortability{Range:}{Medium (100 ft. + 10 ft./level)}
\shortability{Effect:}{Ray}
\shortability{Duration:}{Instantaneous}
\shortability{Saving Throw:}{Fortitude partial (object)}
\shortability{Spell Resistance:}{Yes}
\smallskip\noindent A thin, green ray springs from your pointing finger. You must make a successful ranged touch attack to hit. Any creature struck by the ray takes 2d6 points of damage per caster level (to a maximum of 40d6). Any creature reduced to 0 or fewer hit points by this spell is entirely disintegrated, leaving behind only a trace of fine dust. A disintegrated creature's equipment is unaffected.

\smallskip\noindent When used against an object, the ray simply disintegrates as much as one 10- foot cube of nonliving matter. Thus, the spell disintegrates only part of any very large object or structure targeted. The ray affects even objects constructed entirely of force, such as forceful hand or a wall of force, but not magical effects such as a globe of invulnerability or an antimagic field.

\smallskip\noindent A creature or object that makes a successful Fortitude save is partially affected, taking only 5d6 points of damage. If this damage reduces the creature or object to 0 or fewer hit points, it is entirely disintegrated.
Only the first creature or object struck can be affected; that is, the ray affects only one target per casting.

\shortability{Arcane Material Component:}{A lodestone and a pinch of dust.}

\featname{Dismissal}
\shortability{Abjuration}{}
\shortability{Level:}{Clr 4, Sor/Wiz 5}
\shortability{Components:}{V, S, DF}
\shortability{Casting Time:}{1 standard action}
\shortability{Range:}{Close (25 ft. + 5 ft./2 levels)}
\shortability{Target:}{One extraplanar creature}
\shortability{Duration:}{Instantaneous}
\shortability{Saving Throw:}{Will negates; see text}
\shortability{Spell Resistance:}{Yes}
\smallskip\noindent This spell forces an extraplanar creature back to its proper plane if it fails a special Will save (DC = spell's save DC - creature's HD + your caster level). If the spell is successful, the creature is instantly whisked away, but there is a 20\% chance of actually sending the subject to a plane other than its own.

\featname{Dispel Chaos}
\shortability{Abjuration}{[Lawful]}
\shortability{Level:}{Clr 5, Law 5, Pal 4}
\smallskip\noindent This spell functions like dispel evil, except that you are surrounded by constant, blue, lawful energy, and the spell affects chaotic creatures and spells rather than evil ones.

\featname{Dispel Evil}
\shortability{Abjuration}{[Good]}
\shortability{Level:}{Clr 5, Good 5, Pal 4}
\shortability{Components:}{V, S, DF}
\shortability{Casting Time:}{1 standard action}
\shortability{Range:}{Touch}
\shortability{Target or Targets:}{You and a touched evil creature from another plane; or you and an enchantment or evil spell on a touched creature or object}
\shortability{Duration:}{1 round/level or until discharged, whichever comes first}
\shortability{Saving Throw:}{See text}
\shortability{Spell Resistance:}{See text}
\smallskip\noindent Shimmering, white, holy energy surrounds you. This power has three effects.

\smallskip\noindent First, you gain a +4 deflection bonus to AC against attacks by evil creatures.

\smallskip\noindent Second, on making a successful melee touch attack against an evil creature from another plane, you can choose to drive that creature back to its home plane. The creature can negate the effects with a successful Will save (spell resistance applies). This use discharges and ends the spell.

\smallskip\noindent Third, with a touch you can automatically dispel any one enchantment spell cast by an evil creature or any one evil spell. Exception: Spells that can't be dispelled by dispel magic also can't be dispelled by dispel evil. Saving throws and spell resistance do not apply to this effect. This use discharges and ends the spell.

\featname{Dispel Good}
\shortability{Abjuration}{[Evil]}
\shortability{Level:}{Clr 5, Evil 5}

\smallskip\noindent This spell functions like dispel evil, except that you are surrounded by dark, wavering, unholy energy, and the spell affects good creatures and spells rather than evil ones.

\featname{Dispel Law}
\shortability{Abjuration}{[Chaotic]}
\shortability{Level:}{Chaos 5, Clr 5}

\smallskip\noindent This spell functions like dispel evil, except that you are surrounded by flickering, yellow, chaotic energy, and the spell affects lawful creatures and spells rather than evil ones.

\featname{Dispel Magic}
\shortability{Abjuration}{}
\shortability{Level:}{Brd 3, Clr 3, Drd 4, Magic 3, Pal 3, Sor/Wiz 3}
\shortability{Components:}{V, S}
\shortability{Casting Time:}{1 standard action}
\shortability{Range:}{Medium (100 ft. + 10 ft./level)}
\shortability{Target or Area:}{One spellcaster, creature, or object; or 20-ft.-radius burst}
\shortability{Duration:}{Instantaneous}
\shortability{Saving Throw:}{None}
\shortability{Spell Resistance:}{No}
\smallskip\noindent You can use dispel magic to end ongoing spells that have been cast on a creature or object, to temporarily suppress the magical abilities of a magic item, to end ongoing spells (or at least their effects) within an area, or to counter another spellcaster's spell. A dispelled spell ends as if its duration had expired. Some spells, as detailed in their descriptions, can't be defeated by dispel magic. Dispel magic can dispel (but not counter) spell-like effects just as it does spells.

\shortability{Note:}{The effect of a spell with an instantaneous duration can't be dispelled, because the magical effect is already over before the dispel magic can take effect.}
You choose to use dispel magic in one of three ways: a targeted dispel, an area dispel, or a counterspell.
\shortability{Targeted Dispel:}{One object, creature, or spell is the target of the dispel magic spell. You make a dispel check (1d20 + your caster level, maximum +10) against the spell or against each ongoing spell currently in effect on the object or creature. The DC for this dispel check is 11 + the spell's caster level. If you succeed on a particular check, that spell is dispelled; if you fail, that spell remains in effect.

\smallskip\noindent If you target an object or creature that is the effect of an ongoing spell (such as a monster summoned by monster summoning), you make a dispel check to end the spell that conjured the object or creature.

\smallskip\noindent If the object that you target is a magic item, you make a dispel check against the item's caster level. If you succeed, all the item's magical properties are suppressed for 1d4 rounds, after which the item recovers on its own. A suppressed item becomes nonmagical for the duration of the effect. An interdimensional interface (such as a bag of holding) is temporarily closed. A magic item's physical properties are unchanged: A suppressed magic sword is still a sword (a masterwork sword, in fact). Artifacts and deities are unaffected by mortal magic such as this.

\smallskip\noindent You automatically succeed on your dispel check against any spell that you cast yourself.}

\shortability{Area Dispel:}{When dispel magic is used in this way, the spell affects everything within a 20-foot radius.

\smallskip\noindent For each creature within the area that is the subject of one or more spells, you make a dispel check against the spell with the highest caster level. If that check fails, you make dispel checks against progressively weaker spells until you dispel one spell (which discharges the dispel magic spell so far as that target is concerned) or until you fail all your checks. The creature's magic items are not affected.

\smallskip\noindent For each object within the area that is the target of one or more spells, you make dispel checks as with creatures. Magic items are not affected by an area dispel.

\smallskip\noindent For each ongoing area or effect spell whose point of origin is within the area of the dispel magic spell, you can make a dispel check to dispel the spell.

\smallskip\noindent For each ongoing spell whose area overlaps that of the dispel magic spell, you can make a dispel check to end the effect, but only within the overlapping area.

\smallskip\noindent If an object or creature that is the effect of an ongoing spell (such as a monster summoned by monster summoning) is in the area, you can make a dispel check to end the spell that conjured that object or creature (returning it whence it came) in addition to attempting to dispel spells targeting the creature or object.

\smallskip\noindent You may choose to automatically succeed on dispel checks against any spell that you have cast.}

\shortability{Counterspell:}{When dispel magic is used in this way, the spell targets a spellcaster and is cast as a counterspell. Unlike a true counterspell, however, dispel magic may not work; you must make a dispel check to counter the other spellcaster's spell.}

\featname{Dispel Magic, Greater}
\shortability{Abjuration}{}
\shortability{Level:}{Brd 5, Clr 6, Drd 6, Sor/Wiz 6}

\smallskip\noindent This spell functions like dispel magic, except that the maximum caster level on your dispel check is +20 instead of +10.

\smallskip\noindent Additionally, greater dispel magic has a chance to dispel any effect that remove curse can remove, even if dispel magic can't dispel that effect.

\featname{Displacement}
\shortability{Illusion}{(Glamer)}
\shortability{Level:}{Brd 3, Sor/Wiz 3}
\shortability{Components:}{V, M}
\shortability{Casting Time:}{1 standard action}
\shortability{Range:}{Touch}
\shortability{Target:}{Creature touched}
\shortability{Duration:}{1 round/level (D)}
\shortability{Saving Throw:}{Will negates (harmless)}
\shortability{Spell Resistance:}{Yes (harmless)}
\smallskip\noindent The subject of this spell appears to be about 2 feet away from its true location. The creature benefits from a 50\% miss chance as if it had total concealment. However, unlike actual total concealment, displacement does not prevent enemies from targeting the creature normally. True seeing reveals its true location.

\shortability{Material Component:}{A small strip of leather twisted into a loop.}

\featname{Disrupt Undead}
\shortability{Necromancy}{}
\shortability{Level:}{Sor/Wiz 0}
\shortability{Components:}{V, S}
\shortability{Casting Time:}{1 standard action}
\shortability{Range:}{Close (25 ft. + 5 ft./2 levels)}
\shortability{Effect:}{Ray}
\shortability{Duration:}{Instantaneous}
\shortability{Saving Throw:}{None}
\shortability{Spell Resistance:}{Yes}
\smallskip\noindent You direct a ray of positive energy. You must make a ranged touch attack to hit, and if the ray hits an undead creature, it deals 1d6 points of damage to it.

\featname{Disrupting Weapon}
\shortability{Transmutation}{}
\shortability{Level:}{Clr 5}
\shortability{Components:}{V, S}
\shortability{Casting Time:}{1 standard action}
\shortability{Range:}{Touch}
\shortability{Targets:}{One melee weapon}
\shortability{Duration:}{1 round/level}
\shortability{Saving Throw:}{Will negates (harmless, object); see text}
\shortability{Spell Resistance:}{Yes (harmless, object)}
\smallskip\noindent This spell makes a melee weapon deadly to undead. Any undead creature with HD equal to or less than your caster level must succeed on a Will save or be destroyed utterly if struck in combat with this weapon. Spell resistance does not apply against the destruction effect.

\featname{Divination}
\shortability{Divination}{}
\shortability{Level:}{Clr 4, Knowledge 4}
\shortability{Components:}{V, S, M}
\shortability{Casting Time:}{10 minutes}
\shortability{Range:}{Personal}
\shortability{Target:}{You}
\shortability{Duration:}{Instantaneous }
\smallskip\noindent Similar to augury but more powerful, a divination spell can provide you with a useful piece of advice in reply to a question concerning a specific goal, event, or activity that is to occur within one week. The advice can be as simple as a short phrase, or it might take the form of a cryptic rhyme or omen. If your party doesn't act on the information, the conditions may change so that the information is no longer useful. The base chance for a correct divination is 70\% + 1\% per caster level, to a maximum of 90\%. If the dice roll fails, you know the spell failed, unless specific magic yielding false information is at work.

\smallskip\noindent As with augury, multiple divinations about the same topic by the same caster use the same dice result as the first divination spell and yield the same answer each time.

\shortability{Material Component:}{Incense and a sacrificial offering appropriate to your religion, together worth at least 25 gp.}

\featname{Divine Favor}
\shortability{Evocatio}{}
\shortability{Level:}{Clr 1, Pal 1}
\shortability{Components:}{V, S, DF}
\shortability{Casting Time:}{1 standard action}
\shortability{Range:}{Personal}
\shortability{Target:}{You}
\shortability{Duration:}{1 minute}
\smallskip\noindent Calling upon the strength and wisdom of a deity, you gain a +1 luck bonus on attack and weapon damage rolls for every three caster levels you have (at least +1, maximum +6). The bonus doesn't apply to spell damage.

\featname{Divine Power}
\shortability{Evocatio}{}
\shortability{Level:}{Clr 4, War 4}
\shortability{Components:}{V, S, DF}
\shortability{Casting Time:}{1 standard action}
\shortability{Range:}{Personal}
\shortability{Target:}{You}
\shortability{Duration:}{1 round/level}
\smallskip\noindent Calling upon the divine power of your patron, you imbue yourself with strength and skill in combat. Your base attack bonus becomes equal to your character level (which may give you additional attacks), you gain a +6 enhancement bonus to Strength, and you gain 1 temporary hit point per caster level.

\featname{Dominate Animal}
\shortability{Enchantment}{(Compulsion) [Mind-Affecting]}
\shortability{Level:}{Animal 3, Drd 3}
\shortability{Components:}{V, S}
\shortability{Casting Time:}{1 round}
\shortability{Range:}{Close (25 ft. + 5 ft./2 levels)}
\shortability{Target:}{One animal}
\shortability{Duration:}{1 round/level}
\shortability{Saving Throw:}{Will negates}
\shortability{Spell Resistance:}{Yes}
\smallskip\noindent You can enchant an animal and direct it with simple commands such as ``Attack,'' ``Run,'' and ``Fetch.'' Suicidal or self-destructive commands (including an order to attack a creature two or more size categories larger than the dominated animal) are simply ignored.

\smallskip\noindent Dominate animal establishes a mental link between you and the subject creature. The animal can be directed by silent mental command as long as it remains in range. You need not see the creature to control it. You do not receive direct sensory input from the creature, but you know what it is experiencing. Because you are directing the animal with your own intelligence, it may be able to undertake actions normally beyond its own comprehension. You need not concentrate exclusively on controlling the creature unless you are trying to direct it to do something it normally couldn't do. Changing your instructions or giving a dominated creature a new command is the equivalent of redirecting a spell, so it is a move action.

\featname{Dominate Monster}
\shortability{Enchantment}{(Compulsion) [Mind-Affecting]}
\shortability{Level:}{Sor/Wiz 9}
\shortability{Target:}{One creature}
\smallskip\noindent This spell functions like dominate person, except that the spell is not restricted by creature type.

\featname{Dominate Person}
\shortability{Enchantment}{(Compulsion) [Mind-Affecting]}
\shortability{Level:}{Brd 4, Sor/Wiz 5}
\shortability{Components:}{V, S}
\shortability{Casting Time:}{1 round}
\shortability{Range:}{Close (25 ft. + 5 ft./2 levels)}
\shortability{Target:}{One humanoid}
\shortability{Duration:}{One day/level}
\shortability{Saving Throw:}{Will negates}
\shortability{Spell Resistance:}{Yes}
\smallskip\noindent You can control the actions of any humanoid creature through a telepathic link that you establish with the subject's mind.

\smallskip\noindent If you and the subject have a common language, you can generally force the subject to perform as you desire, within the limits of its abilities. If no common language exists, you can communicate only basic commands, such as ``Come here,'' ``Go there,'' ``Fight,'' and ``Stand still.'' You know what the subject is experiencing, but you do not receive direct sensory input from it, nor can it communicate with you telepathically.

\smallskip\noindent Once you have given a dominated creature a command, it continues to attempt to carry out that command to the exclusion of all other activities except those necessary for day-to-day survival (such as sleeping, eating, and so forth). Because of this limited range of activity, a Sense Motive check against DC 15 (rather than DC 25) can determine that the subject's behavior is being influenced by an enchantment effect (see the Sense Motive skill description).

\smallskip\noindent Changing your instructions or giving a dominated creature a new command is the equivalent of redirecting a spell, so it is a move action.
By concentrating fully on the spell (a standard action), you can receive full sensory input as interpreted by the mind of the subject, though it still can't communicate with you. You can't actually see through the subject's eyes, so it's not as good as being there yourself, but you still get a good idea of what's going on.

\smallskip\noindent Subjects resist this control, and any subject forced to take actions against its nature receives a new saving throw with a +2 bonus. Obviously self-destructive orders are not carried out. Once control is established, the range at which it can be exercised is unlimited, as long as you and the subject are on the same plane. You need not see the subject to control it.

\smallskip\noindent If you don't spend at least 1 round concentrating on the spell each day, the subject receives a new saving throw to throw off the domination.

\smallskip\noindent Protection from evil or a similar spell can prevent you from exercising control or using the telepathic link while the subject is so warded, but such an effect neither prevents the establishment of domination nor dispels it.

\featname{Doom}
\shortability{Necromancy}{[Fear, Mind-Affecting]}
\shortability{Level:}{Clr 1}
\shortability{Components:}{V, S, DF}
\shortability{Casting Time:}{1 standard action}
\shortability{Range:}{Medium (100 ft. + 10 ft./level)}
\shortability{Target:}{One living creature}
\shortability{Duration:}{1 min./level}
\shortability{Saving Throw:}{Will negates}
\shortability{Spell Resistance:}{Yes}
\smallskip\noindent This spell fills a single subject with a feeling of horrible dread that causes it to become shaken.

\featname{Dream}
\shortability{Illusion}{(Phantasm) [Mind-Affecting]}
\shortability{Level:}{Brd 5, Sor/Wiz 5}
\shortability{Components:}{V, S}
\shortability{Casting Time:}{1 minute}
\shortability{Range:}{Unlimited}
\shortability{Target:}{One living creature touched}
\shortability{Duration:}{See text}
\shortability{Saving Throw:}{None}
\shortability{Spell Resistance:}{Yes}
\smallskip\noindent You, or a messenger touched by you, sends a phantasmal message to others in the form of a dream. At the beginning of the spell, you must name the recipient or identify him or her by some title that leaves no doubt as to identity. The messenger then enters a trance, appears in the intended recipient's dream, and delivers the message. The message can be of any length, and the recipient remembers it perfectly upon waking. The communication is one-way. The recipient cannot ask questions or offer information, nor can the messenger gain any information by observing the dreams of the recipient.

\smallskip\noindent Once the message is delivered, the messenger's mind returns instantly to its body. The duration of the spell is the time required for the messenger to enter the recipient's dream and deliver the message.

\smallskip\noindent If the recipient is awake when the spell begins, the messenger can choose to wake up (ending the spell) or remain in the trance. The messenger can remain in the trance until the recipient goes to sleep, then enter the recipient's dream and deliver the message as normal. A messenger that is disturbed during the trance comes awake, ending the spell.
Creatures who don't sleep (such as elves, but not half-elves) or don't dream cannot be contacted by this spell.

\smallskip\noindent The messenger is unaware of its own surroundings or of the activities around it while in the trance. It is defenseless both physically and mentally (always fails any saving throw) while in the trance.

\featname{Eagle's Splendor}
\shortability{Transmutation}{}
\shortability{Level:}{Brd 2, Clr 2, Pal 2, Sor/Wiz 2}
\shortability{Components:}{V, S, M/DF}
\shortability{Casting Time:}{1 standard action}
\shortability{Range:}{Touch}
\shortability{Target:}{Creature touched}
\shortability{Duration:}{1 min./level}
\shortability{Saving Throw:}{Will negates (harmless)}
\shortability{Spell Resistance:}{Yes}
\smallskip\noindent The transmuted creature becomes more poised, articulate, and personally forceful. The spell grants a +4 enhancement bonus to Charisma, adding the usual benefits to Charisma-based skill checks and other uses of the Charisma modifier. Sorcerers and bards (and other spellcasters who rely on Charisma) affected by this spell do not gain any additional bonus spells for the increased Charisma, but the save DCs for spells they cast while under this spell's effect do increase.

\shortability{Arcane Material Component:}{A few feathers or a pinch of droppings from an eagle.}

\featname{Eagle's Splendor, Mass}
\shortability{Transmutation}{}
\shortability{Level:}{Brd 6, Clr 6, Sor/Wiz 6}
\shortability{Range:}{Close (25 ft. + 5 ft./2 levels)}
\shortability{Target:}{One creature/level, no two of which can be more than 30 ft. apart}
\smallskip\noindent This spell functions like eagle's splendor, except that it affects multiple creatures.

\featname{Earthquake}
\shortability{Evocation}{[Earth]}
\shortability{Level:}{Clr 8, Destruction 8, Drd 8, Earth 7}
\shortability{Components:}{V, S, DF}
\shortability{Casting Time:}{1 standard action}
\shortability{Range:}{Long (400 ft. + 40 ft./level)}
\shortability{Area:}{80-ft.-radius spread (S)}
\shortability{Duration:}{1 round}
\shortability{Saving Throw:}{See text}
\shortability{Spell Resistance:}{No}
\smallskip\noindent When you cast earthquake, an intense but highly localized tremor rips the ground. The shock knocks creatures down, collapses structures, opens cracks in the ground, and more. The effect lasts for 1 round, during which time creatures on the ground can't move or attack. A spellcaster on the ground must make a Concentration check (DC 20 + spell level) or lose any spell he or she tries to cast. The earthquake affects all terrain, vegetation, structures, and creatures in the area. The specific effect of an earthquake spell depends on the nature of the terrain where it is cast.

\shortability{Cave, Cavern, or Tunnel:}{The spell collapses the roof, dealing 8d6 points of bludgeoning damage to any creature caught under the cave-in (Reflex DC 15 half ) and pinning that creature beneath the rubble (see below). An earthquake cast on the roof of a very large cavern could also endanger those outside the actual area but below the falling debris.
Cliffs: Earthquake causes a cliff to crumble, creating a landslide that travels horizontally as far as it fell vertically. Any creature in the path takes 8d6 points of bludgeoning damage (Reflex DC 15 half ) and is pinned beneath the rubble (see below).}
\shortability{Open Ground:}{Each creature standing in the area must make a DC 15 Reflex save or fall down. Fissures open in the earth, and every creature on the ground has a 25\% chance to fall into one (Reflex DC 20 to avoid a fissure). At the end of the spell, all fissures grind shut, killing any creatures still trapped within.}
\shortability{Structure:}{Any structure standing on open ground takes 100 points of damage, enough to collapse a typical wooden or masonry building, but not a structure built of stone or reinforced masonry. Hardness does not reduce this damage, nor is it halved as damage dealt to objects normally is. Any creature caught inside a collapsing structure takes 8d6 points of bludgeoning damage (Reflex DC 15 half ) and is pinned beneath the rubble (see below).}
\shortability{River, Lake, or Marsh:}{Fissures open underneath the water, draining away the water from that area and forming muddy ground. Soggy marsh or swampland becomes quicksand for the duration of the spell, sucking down creatures and structures. Each creature in the area must make a DC 15 Reflex save or sink down in the mud and quicksand. At the end of the spell, the rest of the body of water rushes in to replace the drained water, possibly drowning those caught in the mud.}
\shortability{Pinned beneath Rubble:}{Any creature pinned beneath rubble takes 1d6 points of nonlethal damage per minute while pinned. If a pinned character falls unconscious, he or she must make a DC 15 Constitution check or take 1d6 points of lethal damage each minute thereafter until freed or dead.}

\featname{Elemental Swarm}
\shortability{Conjuration}{(Summoning) [see text]}
\shortability{Level:}{Air 9, Drd 9, Earth 9, Fire 9, Water 9}
\shortability{Components:}{V, S}
\shortability{Casting Time:}{10 minutes}
\shortability{Range:}{Medium (100 ft. + 10 ft./level)}
\shortability{Effect:}{Two or more summoned creatures, no two of which can be more than 30 ft. apart}
\shortability{Duration:}{10 min./level (D)}
\shortability{Saving Throw:}{None}
\shortability{Spell Resistance:}{No}
\smallskip\noindent This spell opens a portal to an Elemental Plane and summons elementals from it. A druid can choose the plane (Air, Earth, Fire, or Water); a cleric opens a portal to the plane matching his domain.

\smallskip\noindent When the spell is complete, 2d4 Large elementals appear. Ten minutes later, 1d4 Huge elementals appear. Ten minutes after that, one greater elemental appears. Each elemental has maximum hit points per HD. Once these creatures appear, they serve you for the duration of the spell.

\smallskip\noindent The elementals obey you explicitly and never attack you, even if someone else manages to gain control over them. You do not need to concentrate to maintain control over the elementals. You can dismiss them singly or in groups at any time.

\smallskip\noindent When you use a summoning spell to summon an air, earth, fire, or water creature, it is a spell of that type.

\featname{Endure Elements}
\shortability{Abjuration}{}
\shortability{Level:}{Clr 1, Drd 1, Pal 1, Rgr 1, Sor/Wiz 1, Sun 1}
\shortability{Components:}{V, S}
\shortability{Casting Time:}{1 standard action}
\shortability{Range:}{Touch}
\shortability{Target:}{Creature touched}
\shortability{Duration:}{24 hours}
\shortability{Saving Throw:}{Will negates (harmless)}
\shortability{Spell Resistance:}{Yes (harmless)}
\smallskip\noindent A creature protected by endure elements suffers no harm from being in a hot or cold environment. It can exist comfortably in conditions between -50 and 140 degrees Fahrenheit without having to make Fortitude saves). The creature's equipment is likewise protected.

\smallskip\noindent Endure elements doesn't provide any protection from fire or cold damage, nor does it protect against other environmental hazards such as smoke, lack of air, and so forth.

\featname{Energy Drain}
\shortability{Necromancy}{}
\shortability{Level:}{Clr 9, Sor/Wiz 9}
\shortability{Saving Throw:}{Fortitude partial; see text for enervation}
\smallskip\noindent This spell functions like enervation, except that the creature struck gains 2d4 negative levels, and the negative levels last longer.

\smallskip\noindent There is no saving throw to avoid gaining the negative levels, but 24 hours after gaining them, the subject must make a Fortitude saving throw (DC = energy drain spell's save DC) for each negative level. If the save succeeds, that negative level is removed. If it fails, the negative level also goes away, but one of the subject's character levels is permanently drained.

\smallskip\noindent An undead creature struck by the ray gains 2d4x5 temporary hit points for 1 hour.

\featname{Enervation}
\shortability{Necromancy}{}
\shortability{Level:}{Sor/Wiz 4}
\shortability{Components:}{V, S}
\shortability{Casting Time:}{1 standard action}
\shortability{Range:}{Close (25 ft. + 5 ft./2 levels)}
\shortability{Effect:}{Ray of negative energy}
\shortability{Duration:}{Instantaneous}
\shortability{Saving Throw:}{None}
\shortability{Spell Resistance:}{Yes}
\smallskip\noindent You point your finger and utter the incantation, releasing a black ray of crackling negative energy that suppresses the life force of any living creature it strikes. You must make a ranged touch attack to hit. If the attack succeeds, the subject gains 1d4 negative levels.

\smallskip\noindent If the subject has at least as many negative levels as HD, it dies. Each negative level gives a creature a -1 penalty on attack rolls, saving throws, skill checks, ability checks, and effective level (for determining the power, duration, DC, and other details of spells or special abilities).

\smallskip\noindent Additionally, a spellcaster loses one spell or spell slot from his or her highest available level. Negative levels stack.

\smallskip\noindent Assuming the subject survives, it regains lost levels after a number of hours equal to your caster level (maximum 15 hours). Usually, negative levels have a chance of permanently draining the victim's levels, but the negative levels from enervation don't last long enough to do so.

\smallskip\noindent An undead creature struck by the ray gains 1d4x5 temporary hit points for 1 hour.

\featname{Enlarge Person}
\shortability{Transmutation}{}
\shortability{Level:}{Sor/Wiz 1, Strength 1}
\shortability{Components:}{V, S, M}
\shortability{Casting Time:}{1 round}
\shortability{Range:}{Close (25 ft. + 5 ft./2 levels)}
\shortability{Target:}{One humanoid creature}
\shortability{Duration:}{1 min./level (D)}
\shortability{Saving Throw:}{Fortitude negates}
\shortability{Spell Resistance:}{Yes}
\smallskip\noindent This spell causes instant growth of a humanoid creature, doubling its height and multiplying its weight by 8. This increase changes the creature's size category to the next larger one. The target gains a +2 size bonus to Strength, a -2 size penalty to Dexterity (to a minimum of 1), and a -1 penalty on attack rolls and AC due to its increased size.

\smallskip\noindent A humanoid creature whose size increases to Large has a space of 10 feet and a natural reach of 10 feet. This spell does not change the target's speed.

\smallskip\noindent If insufficient room is available for the desired growth, the creature attains the maximum possible size and may make a Strength check (using its increased Strength) to burst any enclosures in the process. If it fails, it is constrained without harm by the materials enclosing it- the spell cannot be used to crush a creature by increasing its size.

\smallskip\noindent All equipment worn or carried by a creature is similarly enlarged by the spell. Melee and projectile weapons affected by this spell deal more damage. Other magical properties are not affected by this spell. Any enlarged item that leaves an enlarged creature's possession (including a projectile or thrown weapon) instantly returns to its normal size. This means that thrown weapons deal their normal damage, and projectiles deal damage based on the size of the weapon that fired them. Magical properties of enlarged items are not increased by this spell.

\smallskip\noindent Multiple magical effects that increase size do not stack

\smallskip\noindent Enlarge person counters and dispels reduce person.

\smallskip\noindent Enlarge person can be made permanent with a permanency spell.

\shortability{Material Component:}{A pinch of powdered iron.}

\featname{Enlarge Person, Mass}
\shortability{Transmutation}{}
\shortability{Level:}{Sor/Wiz 4}
\shortability{Target:}{One humanoid creature/level, no two of which can be more than 30 ft. apart}
\smallskip\noindent This spell functions like enlarge person, except that it affects multiple creatures.

\featname{Entangle}
\shortability{Transmutation}{}
\shortability{Level:}{Drd 1, Plant 1, Rgr 1}
\shortability{Components:}{V, S, DF}
\shortability{Casting Time:}{1 standard action}
\shortability{Range:}{Long (400 ft. + 40 ft./level)}
\shortability{Area:}{Plants in a 40-ft.-radius spread}
\shortability{Duration:}{1 min./level (D)}
\shortability{Saving Throw:}{Reflex partial; see text}
\shortability{Spell Resistance:}{No}
\smallskip\noindent Grasses, weeds, bushes, and even trees wrap, twist, and entwine about creatures in the area or those that enter the area, holding them fast and causing them to become entangled. The creature can break free and move half its normal speed by using a full-round action to make a DC 20 Strength check or a DC 20 Escape Artist check. A creature that succeeds on a Reflex save is not entangled but can still move at only half speed through the area. Each round on your turn, the plants once again attempt to entangle all creatures that have avoided or escaped entanglement.

\smallskip\noindent Note: The effects of the spell may be altered somewhat, based on the nature of the entangling plants.

\featname{Enthrall}
\shortability{Enchantment}{(Charm) [Language Dependent, Mind-Affecting, Sonic]}
\shortability{Level:}{Brd 2, Clr 2}
\shortability{Components:}{V, S}
\shortability{Casting Time:}{1 round}
\shortability{Range:}{Medium (100 ft. + 10 ft./level)}
\shortability{Targets:}{Any number of creatures}
\shortability{Duration:}{1 hour or less}
\shortability{Saving Throw:}{Will negates; see text}
\shortability{Spell Resistance:}{Yes}
\smallskip\noindent If you have the attention of a group of creatures, you can use this spell to hold them spellbound. To cast the spell, you must speak or sing without interruption for 1 full round. Thereafter, those affected give you their undivided attention, ignoring their surroundings. They are considered to have an attitude of friendly while under the effect of the spell. Any potentially affected creature of a race or religion unfriendly to yours gets a +4 bonus on the saving throw.
\smallskip\noindent A creature with 4 or more HD or with a Wisdom score of 16 or higher remains aware of its surroundings and has an attitude of indifferent. It gains a new saving throw if it witnesses actions that it opposes.

\smallskip\noindent The effect lasts as long as you speak or sing, to a maximum of 1 hour. Those enthralled by your words take no action while you speak or sing and for 1d3 rounds thereafter while they discuss the topic or performance. Those entering the area during the performance must also successfully save or become enthralled. The speech ends (but the 1d3-round delay still applies) if you lose concentration or do anything other than speak or sing.

\smallskip\noindent If those not enthralled have unfriendly or hostile attitudes toward you, they can collectively make a Charisma check to try to end the spell by jeering and heckling. For this check, use the Charisma bonus of the creature with the highest Charisma in the group; others may make Charisma checks to assist. The heckling ends the spell if this check result beats your Charisma check result. Only one such challenge is allowed per use of the spell.

\smallskip\noindent If any member of the audience is attacked or subjected to some other overtly hostile act, the spell ends and the previously enthralled members become immediately unfriendly toward you. Each creature with 4 or more HD or with a Wisdom score of 16 or higher becomes hostile.

\featname{Entropic Shield}
\shortability{Abjuration}{}
\shortability{Level:}{Clr 1, Luck 1}
\shortability{Components:}{V, S}
\shortability{Casting Time:}{1 standard action}
\shortability{Range:}{Personal}
\shortability{Target:}{You}
\shortability{Duration:}{1 min./level (D)}
\smallskip\noindent A magical field appears around you, glowing with a chaotic blast of multicolored hues. This field deflects incoming arrows, rays, and other ranged attacks. Each ranged attack directed at you for which the attacker must make an attack roll has a 20\% miss chance (similar to the effects of concealment). Other attacks that simply work at a distance are not affected.

\featname{Erase}
\shortability{Transmutation}{}
\shortability{Level:}{Brd 1, Sor/Wiz 1}
\shortability{Components:}{V, S}
\shortability{Casting Time:}{1 standard action}
\shortability{Range:}{Close (25 ft. + 5 ft./2 levels)}
\shortability{Target:}{One scroll or two pages}
\shortability{Duration:}{Instantaneous}
\shortability{Saving Throw:}{See text}
\shortability{Spell Resistance:}{No}
\smallskip\noindent Erase removes writings of either magical or mundane nature from a scroll or from one or two pages of paper, parchment, or similar surfaces. With this spell, you can remove explosive runes, a glyph of warding, a sepia snake sigil, or an arcane mark, but not illusory script or a symbol spell. Nonmagical writing is automatically erased if you touch it and no one else is holding it. Otherwise, the chance of erasing nonmagical writing is 90\%.

\smallskip\noindent Magic writing must be touched to be erased, and you also must succeed on a caster level check (1d20 + caster level) against DC 15. (A natural 1 or 2 is always a failure on this check.) If you fail to erase explosive runes, a glyph of warding, or a sepia snake sigil, you accidentally activate that writing instead.

\featname{Ethereal Jaunt}
\shortability{Transmutation}{}
\shortability{Level:}{Clr 7, Sor/Wiz 7}
\shortability{Components:}{V, S}
\shortability{Casting Time:}{1 standard action}
\shortability{Range:}{Personal}
\shortability{Target:}{You}
\shortability{Duration:}{1 round/level (D)}
\smallskip\noindent You become ethereal, along with your equipment. For the duration of the spell, you are in a place called the Ethereal Plane, which overlaps the normal, physical, Material Plane. When the spell expires, you return to material existence.

\smallskip\noindent An ethereal creature is invisible, insubstantial, and capable of moving in any direction, even up or down, albeit at half normal speed. As an insubstantial creature, you can move through solid objects, including living creatures. An ethereal creature can see and hear on the Material Plane, but everything looks gray and ephemeral. Sight and hearing onto the Material Plane are limited to 60 feet.

\smallskip\noindent Force effects and abjurations affect an ethereal creature normally. Their effects extend onto the Ethereal Plane from the Material Plane, but not vice versa. An ethereal creature can't attack material creatures, and spells you cast while ethereal affect only other ethereal things. Certain material creatures or objects have attacks or effects that work on the Ethereal Plane.

\smallskip\noindent Treat other ethereal creatures and ethereal objects as if they were material. 

\smallskip\noindent If you end the spell and become material while inside a material object (such as a solid wall), you are shunted off to the nearest open space and take 1d6 points of damage per 5 feet that you so travel.

\featname{Etherealness}
\shortability{Transmutation}{}
\shortability{Level:}{Clr 9, Sor/Wiz 9}
\shortability{Range:}{Touch; see text}
\shortability{Targets:}{You and one other touched creature per three levels}
\shortability{Duration:}{1 min./level (D)}
\shortability{Spell Resistance:}{Yes}
\smallskip\noindent This spell functions like ethereal jaunt, except that you and other willing creatures joined by linked hands (along with their equipment) become ethereal. Besides yourself, you can bring one creature per three caster levels to the Ethereal Plane. Once ethereal, the subjects need not stay together.

\smallskip\noindent When the spell expires, all affected creatures on the Ethereal Plane return to material existence.

\featname{Expeditious Retreat}
\shortability{Transmutation}{}
\shortability{Level:}{Brd 1, Sor/Wiz 1}
\shortability{Components:}{V, S}
\shortability{Casting Time:}{1 standard action}
\shortability{Range:}{Personal}
\shortability{Target:}{You}
\shortability{Duration:}{1 min./level (D)}
\smallskip\noindent This spell increases your base land speed by 30 feet. (This adjustment is treated as an enhancement bonus.) There is no effect on other modes of movement, such as burrow, climb, fly, or swim. As with any effect that increases your speed, this spell affects your jumping distance (see the Jump skill).

\featname{Explosive Runes}
\shortability{Abjuration}{[Force]}
\shortability{Level:}{Sor/Wiz 3}
\shortability{Components:}{V, S}
\shortability{Casting Time:}{1 standard action}
\shortability{Range:}{Touch}
\shortability{Target:}{One touched object weighing no more than 10 lb.}
\shortability{Duration:}{Permanent until discharged (D)}
\shortability{Saving Throw:}{See text}
\shortability{Spell Resistance:}{Yes}
\smallskip\noindent You trace these mystic runes upon a book, map, scroll, or similar object bearing written information. The runes detonate when read, dealing 6d6 points of force damage. Anyone next to the runes (close enough to read them) takes the full damage with no saving throw; any other creature within 10 feet of the runes is entitled to a Reflex save for half damage. The object on which the runes were written also takes full damage (no saving throw).

\smallskip\noindent You and any characters you specifically instruct can read the protected writing without triggering the runes. Likewise, you can remove the runes whenever desired. Another creature can remove them with a successful dispel magic or erase spell, but attempting to dispel or erase the runes and failing to do so triggers the explosion.

\smallskip\noindent Note: Magic traps such as explosive runes are hard to detect and disable. A rogue (only) can use the Search skill to find the runes and Disable Device to thwart them. The DC in each case is 25 + spell level, or 28 for explosive runes.

\featname{Eyebite}
\shortability{Necromancy}{[Evil]}
\shortability{Level:}{Brd 6, Sor/Wiz 6}
\shortability{Components:}{V, S}
\shortability{Casting Time:}{1 standard action}
\shortability{Range:}{Close (25 ft. + 5 ft./2 levels)}
\shortability{Target:}{One living creature}
\shortability{Duration:}{1 round per three levels; see text}
\shortability{Saving Throw:}{Fortitude negates}
\shortability{Spell Resistance:}{Yes}
\smallskip\noindent Each round, you may target a single living creature, striking it with waves of evil power. Depending on the target's HD, this attack has as many as three effects.
 
\begin{tabular}[h!]{l|l}
HD & Effect \\ \hline
10 or more & Sickened \\
5-9 & Panicked, sickened \\
4 or less & Comatose, panicked, sickened \\
\end{tabular}

\smallskip\noindent The effects are cumulative and concurrent.

\shortability{Sickened:}{Sudden pain and fever sweeps over the subject's body. A sickened creature takes a -2 penalty on attack rolls, weapon damage rolls, saving throws, skill checks, and ability checks. A creature affected by this spell remains sickened for 10 minutes per caster level. The effects cannot be negated by a remove disease or heal spell, but a remove curse is effective.}
\shortability{Panicked:}{The subject becomes panicked for 1d4 rounds. Even after the panic ends, the creature remains shaken for 10 minutes per caster level, and it automatically becomes panicked again if it comes within sight of you during that time. This is a fear effect.}
\shortability{Comatose:}{The subject falls into a catatonic coma for 10 minutes per caster level. During this time, it cannot be awakened by any means short of dispelling the effect. This is not a sleep effect, and thus elves are not immune to it.}

\smallskip\noindent The spell lasts for 1 round per three caster levels. You must spend a move action each round after the first to target a foe.

\featname{Fabricate}
\shortability{Transmutation}{}
\shortability{Level:}{Sor/Wiz 5}
\shortability{Components:}{V, S, M}
\shortability{Casting Time:}{See text}
\shortability{Range:}{Close (25 ft. + 5 ft./2 levels)}
\shortability{Target:}{Up to 10 cu. ft./level; see text}
\shortability{Duration:}{Instantaneous}
\shortability{Saving Throw:}{None}
\shortability{Spell Resistance:}{No}
\smallskip\noindent You convert material of one sort into a product that is of the same material. Creatures or magic items cannot be created or transmuted by the fabricate spell. The quality of items made by this spell is commensurate with the quality of material used as the basis for the new fabrication. If you work with a mineral, the target is reduced to 1 cubic foot per level instead of 10 cubic feet.

\smallskip\noindent You must make an appropriate Craft check to fabricate articles requiring a high degree of craftsmanship.

\smallskip\noindent Casting requires 1 round per 10 cubic feet (or 1 cubic foot) of material to be affected by the spell.

\shortability{Material Component:}{The original material, which costs the same amount as the raw materials required to craft the item to be created.}

\featname{Faerie Fire}
\shortability{Evocation}{[Light]}
\shortability{Level:}{Drd 1}
\shortability{Components:}{V, S, DF}
\shortability{Casting Time:}{1 standard action}
\shortability{Range:}{Long (400 ft. + 40 ft./level)}
\shortability{Area:}{Creatures and objects within a 5-ft.-radius burst}
\shortability{Duration:}{1 min./level (D)}
\shortability{Saving Throw:}{None}
\shortability{Spell Resistance:}{Yes}
\smallskip\noindent A pale glow surrounds and outlines the subjects. Outlined subjects shed light as candles. Outlined creatures do not benefit from the concealment normally provided by darkness (though a 2nd-level or higher magical darkness effect functions normally), blur, displacement, invisibility, or similar effects. The light is too dim to have any special effect on undead or dark-dwelling creatures vulnerable to light. The faerie fire can be blue, green, or violet, according to your choice at the time of casting. The faerie fire does not cause any harm to the objects or creatures thus outlined.

\featname{False Life}
\shortability{Necromancy}{}
\shortability{Level:}{Sor/Wiz 2}
\shortability{Components:}{V, S, M}
\shortability{Casting Time:}{1 standard action}
\shortability{Range:}{Personal}
\shortability{Target:}{You}
\shortability{Duration:}{1 hour/level or until discharged; see text}
\smallskip\noindent You harness the power of unlife to grant yourself a limited ability to avoid death. While this spell is in effect, you gain temporary hit points equal to 1d10 +1 per caster level (maximum +10).

\shortability{Material Component:}{A small amount of alcohol or distilled spirits, which you use to trace certain sigils on your body during casting. These sigils cannot be seen once the alcohol or spirits evaporate.}

\featname{False Vision}
\shortability{Illusion}{(Glamer)}
\shortability{Level:}{Brd 5, Sor/Wiz 5, Trickery 5}
\shortability{Components:}{V, S, M}
\shortability{Casting Time:}{1 standard action}
\shortability{Range:}{Touch}
\shortability{Area:}{40-ft.-radius emanation}
\shortability{Duration:}{1 hour/level (D)}
\shortability{Saving Throw:}{None}
\shortability{Spell Resistance:}{No}
\smallskip\noindent Any divination (scrying) spell used to view anything within the area of this spell instead receives a false image (as the major image spell), as defined by you at the time of casting. As long as the duration lasts, you can concentrate to change the image as desired. While you aren't concentrating, the image remains static.

\shortability{Arcane Material Component:}{The ground dust of a piece of jade worth at least 250 gp, which is sprinkled into the air when the spell is cast.}

\featname{Fear}
\shortability{Necromancy}{[Fear, Mind-Affecting]}
\shortability{Level:}{Brd 3, Sor/Wiz 4}
\shortability{Components:}{V, S, M}
\shortability{Casting Time:}{1 standard action}
\shortability{Range:}{30 ft.}
\shortability{Area:}{Cone-shaped burst}
\shortability{Duration:}{1 round/level or 1 round; see text}
\shortability{Saving Throw:}{Will partial}
\shortability{Spell Resistance:}{Yes}
\smallskip\noindent An invisible cone of terror causes each living creature in the area to become panicked unless it succeeds on a Will save. If cornered, a panicked creature begins cowering. If the Will save succeeds, the creature is shaken for 1 round.

\shortability{Material Component:}{Either the heart of a hen or a white feather.}

\featname{Feather Fall}
\shortability{Transmutation}{}
\shortability{Level:}{Brd 1, Sor/Wiz 1}
\shortability{Components:}{V}
\shortability{Casting Time:}{1 free action}
\shortability{Range:}{Close (25 ft. + 5 ft./2 levels)}
\shortability{Targets:}{One Medium or smaller freefalling object or creature/level, no two of which may be more than 20 ft. apart}
\shortability{Duration:}{Until landing or 1 round/level}
\shortability{Saving Throw:}{Will negates (harmless) or Will negates (object)}
\shortability{Spell Resistance:}{Yes (object)}
\smallskip\noindent The affected creatures or objects fall slowly. Feather fall instantly changes the rate at which the targets fall to a mere 60 feet per round (equivalent to the end of a fall from a few feet), and the subjects take no damage upon landing while the spell is in effect. However, when the spell duration expires, a normal rate of falling resumes.

\smallskip\noindent The spell affects one or more Medium or smaller creatures (including gear and carried objects up to each creature's maximum load) or objects, or the equivalent in larger creatures: A Large creature or object counts as two Medium creatures or objects, a Huge creature or object counts as two Large creatures or objects, and so forth.

\smallskip\noindent You can cast this spell with an instant utterance, quickly enough to save yourself if you unexpectedly fall. Casting the spell is a free action, like casting a quickened spell, and it counts toward the normal limit of one quickened spell per round. You may even cast this spell when it isn't your turn.

\smallskip\noindent This spell has no special effect on ranged weapons unless they are falling quite a distance. If the spell is cast on a falling item the object does half normal damage based on its weight, with no bonus for the height of the drop.

\smallskip\noindent Feather fall works only upon free-falling objects. It does not affect a sword blow or a charging or flying creature.

\featname{Feeblemind}
\shortability{Enchantment}{(Compulsion) [Mind-Affecting]}
\shortability{Level:}{Sor/Wiz 5}
\shortability{Components:}{V, S, M}
\shortability{Casting Time:}{1 standard action}
\shortability{Range:}{Medium (100 ft. + 10 ft./level)}
\shortability{Target:}{One creature}
\shortability{Duration:}{Instantaneous}
\shortability{Saving Throw:}{Will negates; see text}
\shortability{Spell Resistance:}{Yes}
\smallskip\noindent If the target creature fails a Will saving throw, its Intelligence and Charisma scores each drop to 1. The affected creature is unable to use Intelligence- or Charisma-based skills, cast spells, understand language, or communicate coherently. Still, it knows who its friends are and can follow them and even protect them. The subject remains in this state until a heal, limited wish, miracle, or wish spell is used to cancel the effect of the feeblemind. A creature that can cast arcane spells, such as a sorcerer or a wizard, takes a -4 penalty on its saving throw.

\shortability{Material Component:}{A handful of clay, crystal, glass, or mineral spheres.}

\featname{Find the Path}
\shortability{Divination}{}
\shortability{Level:}{Brd 6, Clr 6, Drd 6, Knowledge 6, Travel 6}
\shortability{Components:}{V, S, F}
\shortability{Casting Time:}{3 rounds}
\shortability{Range:}{Personal or touch}
\shortability{Target:}{You or creature touched}
\shortability{Duration:}{10 min./level}
\shortability{Saving Throw:}{None or Will negates (harmless)}
\shortability{Spell Resistance:}{No or Yes (harmless)}
\smallskip\noindent The recipient of this spell can find the shortest, most direct physical route to a specified destination, be it the way into or out of a locale. The locale can be outdoors, underground, or even inside a maze spell. Find the path works with respect to locations, not objects or creatures at a locale. The location must be on the same plane as you are at the time of casting.

\smallskip\noindent The spell enables the subject to sense the correct direction that will eventually lead it to its destination, indicating at appropriate times the exact path to follow or physical actions to take. For example, the spell enables the subject to sense trip wires or the proper word to bypass a glyph of warding. The spell ends when the destination is reached or the duration expires, whichever comes first. Find the path can be used to remove the subject and its companions from the effect of a maze spell in a single round.

\smallskip\noindent This divination is keyed to the recipient, not its companions, and its effect does not predict or allow for the actions of creatures (including guardians).

\shortability{Focus:}{A set of divination counters of the sort you favor.}

\featname{Find Traps}
\shortability{Divination}{}
\shortability{Level:}{Clr 2}
\shortability{Components:}{V, S}
\shortability{Casting Time:}{1 standard action}
\shortability{Range:}{Personal}
\shortability{Target:}{You}
\shortability{Duration:}{1 min./level}
\smallskip\noindent You gain intuitive insight into the workings of traps. You can use the Search skill to detect traps just as a rogue can. In addition, you gain an insight bonus equal to one-half your caster level (maximum +10) on Search checks made to find traps while the spell is in effect.

\smallskip\noindent Note that find traps grants no ability to disable the traps that you may find.

\featname{Finger of Death}
\shortability{Necromancy}{[Death]}
\shortability{Level:}{Drd 8, Sor/Wiz 7}
\shortability{Components:}{V, S}
\shortability{Casting Time:}{1 standard action}
\shortability{Range:}{Close (25 ft. + 5 ft./2 levels)}
\shortability{Target:}{One living creature}
\shortability{Duration:}{Instantaneous}
\shortability{Saving Throw:}{Fortitude partial}
\shortability{Spell Resistance:}{Yes}
\smallskip\noindent You can slay any one living creature within range. The target is entitled to a Fortitude saving throw to survive the attack. If the save is successful, the creature instead takes 3d6 points of damage +1 point per caster level (maximum +25).

\smallskip\noindent The subject might die from damage even if it succeeds on its saving throw.

\featname{Fire Seeds}
\shortability{Conjuration}{(Creation) [Fire]}
\shortability{Level:}{Drd 6, Fire 6, Sun 6}
\shortability{Components:}{V, S, M}
\shortability{Casting Time:}{1 standard action}
\shortability{Range:}{Touch}
\shortability{Targets:}{Up to four touched acorns or up to eight touched holly berries}
\shortability{Duration:}{10 min./level or until used}
\shortability{Saving Throw:}{None or Reflex half; see text}
\shortability{Spell Resistance:}{No}
\smallskip\noindent Depending on the version of fire seeds you choose, you turn acorns into splash weapons that you or another character can throw, or you turn holly berries into bombs that you can detonate on command.

\shortability{Acorn Grenades:}{As many as four acorns turn into special splash weapons that can be hurled as far as 100 feet. A ranged touch attack roll is required to strike the intended target. Together, the acorns are capable of dealing 1d6 points of fire damage per caster level (maximum 20d6), divided up among the acorns as you wish.

\smallskip\noindent Each acorn explodes upon striking any hard surface. In addition to its regular fire damage, it deals 1 point of splash damage per die, and it ignites any combustible materials within 10 feet. A creature within this area that makes a successful Reflex saving throw takes only half damage; a creature struck directly is not allowed a saving throw.}

\shortability{Holly Berry Bombs:}{You turn as many as eight holly berries into special bombs. The holly berries are usually placed by hand, since they are too light to make effective thrown weapons (they can be tossed only 5 feet). If you are within 200 feet and speak a word of command, each berry instantly bursts into flame, causing 1d8 points of fire damage +1 point per caster level to every creature in a 5-foot radius burst and igniting any combustible materials within 5 feet. A creature in the area that makes a successful Reflex saving throw takes only half damage.}

\shortability{Material Component:}{The acorns or holly berries.}

\featname{Fire Shield}
\shortability{Evocation}{[Fire or Cold]}
\shortability{Level:}{Fire 5, Sor/Wiz 4, Sun 4}
\shortability{Components:}{V, S, M/DF}
\shortability{Casting Time:}{1 standard action}
\shortability{Range:}{Personal}
\shortability{Target:}{You}
\shortability{Duration:}{1 round/level (D)}
\smallskip\noindent This spell wreathes you in flame and causes damage to each creature that attacks you in melee. The flames also protect you from either cold-based or fire-based attacks (your choice).

\smallskip\noindent Any creature striking you with its body or a handheld weapon deals normal damage, but at the same time the attacker takes 1d6 points of damage +1 point per caster level (maximum +15). This damage is either cold damage (if the shield protects against fire-based attacks) or fire damage (if the shield protects against cold-based attacks). If the attacker has spell resistance, it applies to this effect. Creatures wielding weapons with exceptional reach are not subject to this damage if they attack you.

\smallskip\noindent When casting this spell, you appear to immolate yourself, but the flames are thin and wispy, giving off light equal to only half the illumination of a normal torch (10 feet). The color of the flames is determined randomly (50\% chance of either color)-blue or green if the chill shield is cast, violet or blue if the warm shield is employed. The special powers of each version are as follows.

\shortability{Warm Shield:}{The flames are warm to the touch. You take only half damage from cold-based attacks. If such an attack allows a Reflex save for half damage, you take no damage on a successful save.}
\shortability{Chill Shield:}{The flames are cool to the touch. You take only half damage from fire-based attacks. If such an attack allows a Reflex save for half damage, you take no damage on a successful save.}
\shortability{Arcane Material Component:}{A bit of phosphorus for the warm shield; a live firefly or glowworm or the tail portions of four dead ones for the chill shield.}

\featname{Fire Storm}
\shortability{Evocation}{[Fire]}
\shortability{Level:}{Clr 8, Drd 7, Fire 7}
\shortability{Components:}{V, S}
\shortability{Casting Time:}{1 round}
\shortability{Range:}{Medium (100 ft. + 10 ft./level)}
\shortability{Area:}{Two 10-ft. cubes per level (S)}
\shortability{Duration:}{Instantaneous}
\shortability{Saving Throw:}{Reflex half}
\shortability{Spell Resistance:}{Yes}
\smallskip\noindent When a fire storm spell is cast, the whole area is shot through with sheets of roaring flame. The raging flames do not harm natural vegetation, ground cover, and any plant creatures in the area that you wish to exclude from damage. Any other creature within the area takes 1d6 points of fire damage per caster level (maximum 20d6).

\featname{Fire Trap}
\shortability{Abjuration}{[Fire]}
\shortability{Level:}{Drd 2, Sor/Wiz 4}
\shortability{Components:}{V, S, M}
\shortability{Casting Time:}{10 minutes}
\shortability{Range:}{Touch}
\shortability{Target:}{Object touched}
\shortability{Duration:}{Permanent until discharged (D)}
\shortability{Saving Throw:}{Reflex half; see text}
\shortability{Spell Resistance:}{Yes}
\smallskip\noindent Fire trap creates a fiery explosion when an intruder opens the item that the trap protects. A fire trap can ward any object that can be opened and closed.

\smallskip\noindent When casting fire trap, you select a point on the object as the spell's center. When someone other than you opens the object, a fiery explosion fills the area within a 5-foot radius around the spell's center. The flames deal 1d4 points of fire damage +1 point per caster level (maximum +20). The item protected by the trap is not harmed by this explosion.

\smallskip\noindent A fire trapped item cannot have a second closure or warding spell placed on it.

\smallskip\noindent A knock spell does not bypass a fire trap.  An unsuccessful dispel magic spell does not detonate the spell.

\smallskip\noindent Underwater, this ward deals half damage and creates a large cloud of steam.

\smallskip\noindent You can use the fire trapped object without discharging it, as can any individual to whom the object was specifically attuned when cast. Attuning a fire trapped object to an individual usually involves setting a password that you can share with friends.

\shortability{Note:}{Magic traps such as fire trap are hard to detect and disable. A rogue (only) can use the Search skill to find a fire trap and Disable Device to thwart it. The DC in each case is 25 + spell level (DC 27 for a druid's fire trap or DC 29 for the arcane version).}
\shortability{Material Component:}{A half-pound of gold dust (cost 25 gp) sprinkled on the warded object.}

\featname{Fireball}
\shortability{Evocation}{[Fire]}
\shortability{Level:}{Sor/Wiz 3}
\shortability{Components:}{V, S, M}
\shortability{Casting Time:}{1 standard action}
\shortability{Range:}{Long (400 ft. + 40 ft./level)}
\shortability{Area:}{20-ft.-radius spread}
\shortability{Duration:}{Instantaneous}
\shortability{Saving Throw:}{Reflex half}
\shortability{Spell Resistance:}{Yes}
\smallskip\noindent A fireball spell is an explosion of flame that detonates with a low roar and deals 1d6 points of fire damage per caster level (maximum 10d6) to every creature within the area. Unattended objects also take this damage. The explosion creates almost no pressure.

\smallskip\noindent You point your finger and determine the range (distance and height) at which the fireball is to burst. A glowing, pea-sized bead streaks from the pointing digit and, unless it impacts upon a material body or solid barrier prior to attaining the prescribed range, blossoms into the fireball at that point. (An early impact results in an early detonation.) If you attempt to send the bead through a narrow passage, such as through an arrow slit, you must ``hit'' the opening with a ranged touch attack, or else the bead strikes the barrier and detonates prematurely.

\smallskip\noindent The fireball sets fire to combustibles and damages objects in the area. It can melt metals with low melting points, such as lead, gold, copper, silver, and bronze. If the damage caused to an interposing barrier shatters or breaks through it, the fireball may continue beyond the barrier if the area permits; otherwise it stops at the barrier just as any other spell effect does.

\shortability{Material Component:}{A tiny ball of bat guano and sulfur.}

\featname{Flame Arrow}
\shortability{Transmutation}{[Fire]}
\shortability{Level:}{Sor/Wiz 3}
\shortability{Components:}{V, S, M}
\shortability{Casting Time:}{1 standard action}
\shortability{Range:}{Close (25 ft. + 5 ft./2 levels)}
\shortability{Target:}{Fifty projectiles, all of which must be in contact with each other at the time of casting}
\shortability{Duration:}{10 min./level}
\shortability{Saving Throw:}{None}
\shortability{Spell Resistance:}{No}
\smallskip\noindent You turn ammunition (such as arrows, bolts, shuriken, and stones) into fiery projectiles. Each piece of ammunition deals an extra 1d6 points of fire damage to any target it hits. A flaming projectile can easily ignite a flammable object or structure, but it won't ignite a creature it strikes.

\shortability{Material Component:}{A drop of oil and a small piece of flint.}

\featname{Flame Blade}
\shortability{Evocation}{[Fire]}
\shortability{Level:}{Drd 2}
\shortability{Components:}{V, S, DF}
\shortability{Casting Time:}{1 standard action}
\shortability{Range:}{0 ft.}
\shortability{Effect:}{Sword-like beam}
\shortability{Duration:}{1 min./level (D)}
\shortability{Saving Throw:}{None}
\shortability{Spell Resistance:}{Yes}
\smallskip\noindent A 3-foot-long, blazing beam of red-hot fire springs forth from your hand. You wield this bladelike beam as if it were a scimitar. Attacks with the flame blade are melee touch attacks. The blade deals 1d8 points of fire damage +1 point per two caster levels (maximum +10). Since the blade is immaterial, your Strength modifier does not apply to the damage. A flame blade can ignite combustible materials such as parchment, straw, dry sticks, and cloth.

\smallskip\noindent The spell does not function underwater.

\featname{Flame Strike}
\shortability{Evocation}{[Fire]}
\shortability{Level:}{Clr 5, Drd 4, Sun 5, War 5}
\shortability{Components:}{V, S, DF}
\shortability{Casting Time:}{1 standard action}
\shortability{Range:}{Medium (100 ft. + 10 ft./level)}
\shortability{Area:}{Cylinder (10-ft. radius, 40 ft. high)}
\shortability{Duration:}{Instantaneous}
\shortability{Saving Throw:}{Reflex half}
\shortability{Spell Resistance:}{Yes}
\smallskip\noindent A flame strike produces a vertical column of divine fire roaring downward. The spell deals 1d6 points of damage per caster level (maximum 15d6). Half the damage is fire damage, but the other half results directly from divine power and is therefore not subject to being reduced by resistance to fire-based attacks.

\featname{Flaming Sphere}
\shortability{Evocation}{[Fire]}
\shortability{Level:}{Drd 2, Sor/Wiz 2}
\shortability{Components:}{V, S, M/DF}
\shortability{Casting Time:}{1 standard action}
\shortability{Range:}{Medium (100 ft. + 10 ft./level)}
\shortability{Effect:}{5-ft.-diameter sphere}
\shortability{Duration:}{1 round/level}
\shortability{Saving Throw:}{Reflex negates}
\shortability{Spell Resistance:}{Yes}
\smallskip\noindent A burning globe of fire rolls in whichever direction you point and burns those it strikes. It moves 30 feet per round. As part of this movement, it can ascend or jump up to 30 feet to strike a target. If it enters a space with a creature, it stops moving for the round and deals 2d6 points of fire damage to that creature, though a successful Reflex save negates that damage. A flaming sphere rolls over barriers less than 4 feet tall. It ignites flammable substances it touches and illuminates the same area as a torch would.

\smallskip\noindent The sphere moves as long as you actively direct it (a move action for you); otherwise, it merely stays at rest and burns. It can be extinguished by any means that would put out a normal fire of its size. The surface of the sphere has a spongy, yielding consistency and so does not cause damage except by its flame. It cannot push aside unwilling creatures or batter down large obstacles. A flaming sphere winks out if it exceeds the spell's range.

\shortability{Arcane Material Component:}{A bit of tallow, a pinch of brimstone, and a dusting of powdered iron.}

\featname{Flare}
\shortability{Evocation}{[Light]}
\shortability{Level:}{Brd 0, Drd 0, Sor/Wiz 0}
\shortability{Components:}{V}
\shortability{Casting Time:}{1 standard action}
\shortability{Range:}{Close (25 ft. + 5 ft./2 levels)}
\shortability{Effect:}{Burst of light}
\shortability{Duration:}{Instantaneous}
\shortability{Saving Throw:}{Fortitude negates}
\shortability{Spell Resistance:}{Yes}
\smallskip\noindent This cantrip creates a burst of light. If you cause the light to burst directly in front of a single creature, that creature is dazzled for 1 minute unless it makes a successful Fortitude save. Sightless creatures, as well as creatures already dazzled, are not affected by flare.

\featname{Flesh to Stone}
\shortability{Transmutation}{}
\shortability{Level:}{Sor/Wiz 6}
\shortability{Components:}{V, S, M}
\shortability{Casting Time:}{1 standard action}
\shortability{Range:}{Medium (100 ft. + 10 ft./level)}
\shortability{Target:}{One creature}
\shortability{Duration:}{Instantaneous}
\shortability{Saving Throw:}{Fortitude negates}
\shortability{Spell Resistance:}{Yes}
\smallskip\noindent The subject, along with all its carried gear, turns into a mindless, inert statue. If the statue resulting from this spell is broken or damaged, the subject (if ever returned to its original state) has similar damage or deformities. The creature is not dead, but it does not seem to be alive either when viewed with spells such as deathwatch.

\smallskip\noindent Only creatures made of flesh are affected by this spell.

\shortability{Material Component:}{Lime, water, and earth.}

\featname{Fly}
\shortability{Transmutation}{}
\shortability{Level:}{Sor/Wiz 3, Travel 3}
\shortability{Components:}{V, S, F/DF}
\shortability{Casting Time:}{1 standard action}
\shortability{Range:}{Touch}
\shortability{Target:}{Creature touched}
\shortability{Duration:}{1 min./level}
\shortability{Saving Throw:}{Will negates (harmless)}
\shortability{Spell Resistance:}{Yes (harmless)}
\smallskip\noindent The subject can fly at a speed of 60 feet (or 40 feet if it wears medium or heavy armor, or if it carries a medium or heavy load). It can ascend at half speed and descend at double speed, and its maneuverability is good. Using a fly spell requires only as much concentration as walking, so the subject can attack or cast spells normally. The subject of a fly spell can charge but not run, and it cannot carry aloft more weight than its maximum load, plus any armor it wears.

\smallskip\noindent Should the spell duration expire while the subject is still aloft, the magic fails slowly. The subject floats downward 60 feet per round for 1d6 rounds. If it reaches the ground in that amount of time, it lands safely. If not, it falls the rest of the distance, taking 1d6 points of damage per 10 feet of fall. Since dispelling a spell effectively ends it, the subject also descends in this way if the fly spell is dispelled, but not if it is negated by an antimagic field.

\shortability{Arcane Focus:}{A wing feather from any bird.}

\featname{Floating Disk}
\shortability{Evocation}{[Force]}
\shortability{Level:}{Sor/Wiz 1}
\shortability{Components:}{V, S, M}
\shortability{Casting Time:}{1 standard action}
\shortability{Range:}{Close (25 ft. + 5 ft./2 levels)}
\shortability{Effect:}{3-ft.-diameter disk of force}
\shortability{Duration:}{1 hour/level}
\shortability{Saving Throw:}{None}
\shortability{Spell Resistance:}{No}
\smallskip\noindent You create a slightly concave, circular plane of force that follows you about and carries loads for you. The disk is 3 feet in diameter and 1 inch deep at its center. It can hold 100 pounds of weight per caster level. (If used to transport a liquid, its capacity is 2 gallons.) The disk floats approximately 3 feet above the ground at all times and remains level. It floats along horizontally within spell range and will accompany you at a rate of no more than your normal speed each round. If not otherwise directed, it maintains a constant interval of 5 feet between itself and you. The disk winks out of existence when the spell duration expires. The disk also winks out if you move beyond range or try to take the disk more than 3 feet away from the surface beneath it. When the disk winks out, whatever it was supporting falls to the surface beneath it.

\shortability{Material Component:}{A drop of mercury.}

\featname{Fog Cloud}
\shortability{Conjuration}{(Creation)}
\shortability{Level:}{Drd 2, Sor/Wiz 2, Water 2}
\shortability{Components:}{V, S}
\shortability{Casting Time:}{1 standard action}
\shortability{Range:}{Medium (100 ft. + 10 ft. level)}
\shortability{Effect:}{Fog spreads in 20-ft. radius, 20 ft. high}
\shortability{Duration:}{10 min./level}
\shortability{Saving Throw:}{None}
\shortability{Spell Resistance:}{No}
\smallskip\noindent A bank of fog billows out from the point you designate. The fog obscures all sight, including darkvision, beyond 5 feet. A creature within 5 feet has concealment (attacks have a 20\% miss chance). Creatures farther away have total concealment (50\% miss chance, and the attacker can't use sight to locate the target).

\smallskip\noindent A moderate wind (11+ mph) disperses the fog in 4 rounds; a strong wind (21+ mph) disperses the fog in 1 round.

\smallskip\noindent The spell does not function underwater.

\featname{Forbiddance}
\shortability{Abjuration}{}
\shortability{Level:}{Clr 6}
\shortability{Components:}{V, S, M, DF}
\shortability{Casting Time:}{6 rounds}
\shortability{Range:}{Medium (100 ft. + 10 ft./level)}
\shortability{Area:}{60-ft. cube/level (S)}
\shortability{Duration:}{Permanent}
\shortability{Saving Throw:}{See text}
\shortability{Spell Resistance:}{Yes}
\smallskip\noindent Forbiddance seals an area against all planar travel into or within it. This includes all teleportation spells (such as dimension door and teleport), plane shifting, astral travel, ethereal travel, and all summoning spells. Such effects simply fail automatically.

\smallskip\noindent In addition, it damages entering creatures whose alignments are different from yours. The effect on those attempting to enter the warded area is based on their alignment relative to yours (see below). A creature inside the area when the spell is cast takes no damage unless it exits the area and attempts to reenter, at which time it is affected as normal.

\shortability{Alignments identical:}{No effect. The creature may enter the area freely (although not by planar travel).}
\shortability{Alignments different with respect to either law/chaos or good/evil:}{The creature takes 6d6 points of damage. A successful Will save halves the damage, and spell resistance applies.}
\shortability{Alignments different with respect to both law/chaos and good/evil:}{The creature takes 12d6 points of damage. A successful Will save halves the damage, and spell resistance applies.}

\smallskip\noindent At your option, the abjuration can include a password, in which case creatures of alignments different from yours can avoid the damage by speaking the password as they enter the area. You must select this option (and the password) at the time of casting.
Dispel magic does not dispel a forbiddance effect unless the dispeller's level is at least as high as your caster level.

\smallskip\noindent You can't have multiple overlapping forbiddance effects. In such a case, the more recent effect stops at the boundary of the older effect.

\shortability{Material Component:}{A sprinkling of holy water and rare incenses worth at least 1,500 gp, plus 1,500 gp per 60-foot cube. If a password is desired, this requires the burning of additional rare incenses worth at least 1,000 gp, plus 1,000 gp per 60-foot cube.}

\featname{Forcecage}
\shortability{Evocation}{[Force]}
\shortability{Level:}{Sor/Wiz 7}
\shortability{Components:}{V, S, M}
\shortability{Casting Time:}{1 standard action}
\shortability{Range:}{Close (25 ft. + 5 ft./2 levels)}
\shortability{Area:}{Barred cage (20-ft. cube) or windowless cell (10-ft. cube)}
\shortability{Duration:}{2 hours/level (D)}
\shortability{Saving Throw:}{None}
\shortability{Spell Resistance:}{No}
\smallskip\noindent This powerful spell brings into being an immobile, invisible cubical prison composed of either bars of force or solid walls of force (your choice).

\smallskip\noindent Creatures within the area are caught and contained unless they are too big to fit inside, in which case the spell automatically fails. Teleportation and other forms of astral travel provide a means of escape, but the force walls or bars extend into the Ethereal Plane, blocking ethereal travel.

\smallskip\noindent Like a wall of force spell, a forcecage resists dispel magic, but it is vulnerable to a disintegrate spell, and it can be destroyed by a sphere of annihilation or a rod of cancellation.

\shortability{Barred Cage:}{This version of the spell produces a 20-foot cube made of bands of force (similar to a wall of force spell) for bars. The bands are a half-inch wide, with half-inch gaps between them. Any creature capable of passing through such a small space can escape; others are confined. You can't attack a creature in a barred cage with a weapon unless the weapon can fit between the gaps. Even against such weapons (including arrows and similar ranged attacks), a creature in the barred cage has cover. All spells and breath weapons can pass through the gaps in the bars.}
\shortability{Windowless Cell:}{This version of the spell produces a 10-foot cube with no way in and no way out. Solid walls of force form its six sides.}
\shortability{Material Component:}{Ruby dust worth 1,500 gp, which is tossed into the air and disappears when you cast the spell.}

\featname{Forceful Hand}
\shortability{Evocation}{[Force]}
\shortability{Level:}{Sor/Wiz 6}
\shortability{Components:}{V, S, F}
\smallskip\noindent This spell functions like interposing hand, except that the forceful hand pursues and pushes away the opponent that you designate. Treat this attack as a bull rush with a +14 bonus on the Strength check (+8 for Strength 27, +4 for being Large, and a +2 bonus for charging, which it always gets). The hand always moves with the opponent to push that target back the full distance allowed, and it has no speed limit. Directing the spell to a new target is a move action.

\smallskip\noindent A very strong creature could not push the hand out of its way because the latter would instantly reposition itself between the creature and you, but an opponent could push the hand up against you by successfully bull rushing it.

\shortability{Focus:}{A sturdy glove made of leather or heavy cloth.}

\featname{Foresight}
\shortability{Divination}{}
\shortability{Level:}{Drd 9, Knowledge 9, Sor/Wiz 9}
\shortability{Components:}{V, S, M/DF}
\shortability{Casting Time:}{1 standard action}
\shortability{Range:}{Personal or touch}
\shortability{Target:}{See text}
\shortability{Duration:}{10 min./level}
\shortability{Saving Throw:}{None or Will negates (harmless)}
\shortability{Spell Resistance:}{No or Yes (harmless)}
\smallskip\noindent This spell grants you a powerful sixth sense in relation to yourself or another. Once foresight is cast, you receive instantaneous warnings of impending danger or harm to the subject of the spell. You are never surprised or flat-footed. In addition, the spell gives you a general idea of what action you might take to best protect yourself and gives you a +2 insight bonus to AC and Reflex saves. This insight bonus is lost whenever you would lose a Dexterity bonus to AC.

\smallskip\noindent When another creature is the subject of the spell, you receive warnings about that creature. You must communicate what you learn to the other creature for the warning to be useful, and the creature can be caught unprepared in the absence of such a warning. Shouting a warning, yanking a person back, and even telepathically communicating (via an appropriate spell) can all be accomplished before some danger befalls the subject, provided you act on the warning without delay. The subject, however, does not gain the insight bonus to AC and Reflex saves.

\shortability{Arcane Material Component:}{A hummingbird's feather.}

\featname{Fox's Cunning}
\shortability{Transmutation}{}
\shortability{Level:}{Brd 2, Sor/Wiz 2}
\shortability{Components:}{V, S, M/DF}
\shortability{Casting Time:}{1 standard action}
\shortability{Range:}{Touch}
\shortability{Target:}{Creature touched}
\shortability{Duration:}{1 min./level}
\shortability{Saving Throw:}{Will negates (harmless)}
\shortability{Spell Resistance:}{Yes}
\smallskip\noindent The transmuted creature becomes smarter. The spell grants a +4 enhancement bonus to Intelligence, adding the usual benefits to Intelligence-based skill checks and other uses of the Intelligence modifier. Wizards (and other spellcasters who rely on Intelligence) affected by this spell do not gain any additional bonus spells for the increased Intelligence, but the save DCs for spells they cast while under this spell's effect do increase. This spell doesn't grant extra skill points.

\shortability{Arcane Material Component:}{A few hairs, or a pinch of dung, from a fox.}

\featname{Fox's Cunning, Mass}
\shortability{Transmutation}{}
\shortability{Level:}{Brd 6, Sor/Wiz 6}
\shortability{Range:}{Close (25 ft. + 5 ft./2 levels)}
\shortability{Target:}{One creature/level, no two of which can be more than 30 ft. apart}
\smallskip\noindent This spell functions like fox's cunning, except that it affects multiple creatures.

\featname{Freedom}
\shortability{Abjuration}{}
\shortability{Level:}{Sor/Wiz 9}
\shortability{Components:}{V, S}
\shortability{Casting Time:}{1 standard action}
\shortability{Range:}{Close (25 ft. + 5 ft./2 levels) or see text}
\shortability{Target:}{One creature}
\shortability{Duration:}{Instantaneous}
\shortability{Saving Throw:}{Will negates (harmless)}
\shortability{Spell Resistance:}{Yes}
\smallskip\noindent The subject is freed from spells and effects that restrict its movement, including binding, entangle, grappling, imprisonment, maze, paralysis, petrification, pinning, sleep, slow, stunning, temporal stasis, and web. To free a creature from imprisonment or maze, you must know its name and background, and you must cast this spell at the spot where it was entombed or banished into the maze.

\featname{Freedom of Movement}
\shortability{Abjuration}{}
\shortability{Level:}{Brd 4, Clr 4, Drd 4, Luck 4, Rgr 4}
\shortability{Components:}{V, S, M, DF}
\shortability{Casting Time:}{1 standard action}
\shortability{Range:}{Personal or touch}
\shortability{Target:}{You or creature touched}
\shortability{Duration:}{10 min./level}
\shortability{Saving Throw:}{Will negates (harmless)}
\shortability{Spell Resistance:}{Yes (harmless)}
\smallskip\noindent This spell enables you or a creature you touch to move and attack normally for the duration of the spell, even under the influence of magic that usually impedes movement, such as paralysis, solid fog, slow, and web. The subject automatically succeeds on any grapple check made to resist a grapple attempt, as well as on grapple checks or Escape Artist checks made to escape a grapple or a pin.

\smallskip\noindent The spell also allows the subject to move and attack normally while underwater, even with slashing weapons such as axes and swords or with bludgeoning weapons such as flails, hammers, and maces, provided that the weapon is wielded in the hand rather than hurled. The freedom of movement spell does not, however, allow water breathing.

\shortability{Material Component:}{A leather thong, bound around the arm or a similar appendage.}

\featname{Freezing Sphere}
\shortability{Evocation}{[Cold]}
\shortability{Level:}{Sor/Wiz 6}
\shortability{Components:}{V, S, F}
\shortability{Casting Time:}{1 standard action}
\shortability{Range:}{Long (400 ft. + 40 ft./level)}
\shortability{Target, Effect, or Area:}{See text}
\shortability{Duration:}{Instantaneous or 1 round/level; see text}
\shortability{Saving Throw:}{Reflex half; see text}
\shortability{Spell Resistance:}{Yes}
\smallskip\noindent Freezing sphere creates a frigid globe of cold energy that streaks from your fingertips to the location you select, where it explodes in a 10-foot-radius burst, dealing 1d6 points of cold damage per caster level (maximum 15d6) to each creature in the area. An elemental (water) creature instead takes 1d8 points of cold damage per caster level (maximum 15d8).

\smallskip\noindent If the freezing sphere strikes a body of water or a liquid that is principally water (not including water-based creatures), it freezes the liquid to a depth of 6 inches over an area equal to 100 square feet (a 10- foot square) per caster level (maximum 1,500 square feet). This ice lasts for 1 round per caster level. Creatures that were swimming on the surface of frozen water become trapped in the ice. Attempting to break free is a full-round action. A trapped creature must make a DC 25 Strength check or a DC 25 Escape Artist check to do so.

\smallskip\noindent You can refrain from firing the globe after completing the spell, if you wish. Treat this as a touch spell for which you are holding the charge. You can hold the charge for as long as 1 round per level, at the end of which time the freezing sphere bursts centered on you (and you receive no saving throw to resist its effect). Firing the globe in a later round is a standard action.

\shortability{Focus:}{A small crystal sphere.}

\featname{Gaseous Form}
\shortability{Transmutation}{}
\shortability{Level:}{Air 3, Brd 3, Sor/Wiz 3}
\shortability{Components:}{S, M/DF}
\shortability{Casting Time:}{1 standard action}
\shortability{Range:}{Touch}
\shortability{Target:}{Willing corporeal creature touched}
\shortability{Duration:}{2 min./level (D)}
\shortability{Saving Throw:}{None}
\shortability{Spell Resistance:}{No}
\smallskip\noindent The subject and all its gear become insubstantial, misty, and translucent. Its material armor (including natural armor) becomes worthless, though its size, Dexterity, deflection bonuses, and armor bonuses from force effects still apply. The subject gains damage reduction 10/magic and becomes immune to poison and critical hits. It can't attack or cast spells with verbal, somatic, material, or focus components while in gaseous form. (This does not rule out the use of certain spells that the subject may have prepared using the feats Silent Spell, Still Spell, and Eschew Materials.) The subject also loses supernatural abilities while in gaseous form. If it has a touch spell ready to use, that spell is discharged harmlessly when the gaseous form spell takes effect.

\smallskip\noindent A gaseous creature can't run, but it can fly at a speed of 10 feet (maneuverability perfect). It can pass through small holes or narrow openings, even mere cracks, with all it was wearing or holding in its hands, as long as the spell persists. The creature is subject to the effects of wind, and it can't enter water or other liquid. It also can't manipulate objects or activate items, even those carried along with its gaseous form. Continuously active items remain active, though in some cases their effects may be moot.

\shortability{Arcane Material Component:}{A bit of gauze and a wisp of smoke.}

\featname{Gate}
\shortability{Conjuration}{(Creation or Calling)}
\shortability{Level:}{Clr 9, Sor/Wiz 9}
\shortability{Components:}{V, S, XP; see text}
\shortability{Casting Time:}{1 standard action}
\shortability{Range:}{Medium (100 ft. + 10 ft./level)}
\shortability{Effect:}{See text}
\shortability{Duration:}{Instantaneous or concentration (up to 1 round/level); see text}
\shortability{Saving Throw:}{None}
\shortability{Spell Resistance:}{No}
\smallskip\noindent Casting a gate spell has two effects. First, it creates an interdimensional connection between your plane of existence and a plane you specify, allowing travel between those two planes in either direction.

\smallskip\noindent Second, you may then call a particular individual or kind of being through the gate.

\smallskip\noindent The gate itself is a circular hoop or disk from 5 to 20 feet in diameter (caster's choice), oriented in the direction you desire when it comes into existence (typically vertical and facing you). It is a two-dimensional window looking into the plane you specified when casting the spell, and anyone or anything that moves through is shunted instantly to the other side.

\smallskip\noindent A gate has a front and a back. Creatures moving through the gate from the front are transported to the other plane; creatures moving through it from the back are not.

\shortability{Planar Travel:}{As a mode of planar travel, a gate spell functions much like a plane shift spell, except that the gate opens precisely at the point you desire (a creation effect). Deities and other beings who rule a planar realm can prevent a gate from opening in their presence or personal demesnes if they so desire. Travelers need not join hands with you-anyone who chooses to step through the portal is transported. A gate cannot be opened to another point on the same plane; the spell works only for interplanar travel.

\smallskip\noindent You may hold the gate open only for a brief time (no more than 1 round per caster level), and you must concentrate on doing so, or else the interplanar connection is severed.}

\shortability{Calling Creatures:}{The second effect of the gate spell is to call an extraplanar creature to your aid (a calling effect). By naming a particular being or kind of being as you cast the spell, you cause the gate to open in the immediate vicinity of the desired creature and pull the subject through, willing or unwilling. Deities and unique beings are under no compulsion to come through the gate, although they may choose to do so of their own accord. This use of the spell creates a gate that remains open just long enough to transport the called creatures. This use of the spell has an XP cost (see below).

\smallskip\noindent If you choose to call a kind of creature instead of a known individual you may call either a single creature (of any HD) or several creatures. You can call and control several creatures as long as their HD total does not exceed your caster level. In the case of a single creature, you can control it if its HD do not exceed twice your caster level. A single creature with more HD than twice your caster level can't be controlled. Deities and unique beings cannot be controlled in any event. An uncontrolled being acts as it pleases, making the calling of such creatures rather dangerous. An uncontrolled being may return to its home plane at any time.

\smallskip\noindent A controlled creature can be commanded to perform a service for you. Such services fall into two categories: immediate tasks and contractual service. Fighting for you in a single battle or taking any other actions that can be accomplished within 1 round per caster level counts as an immediate task; you need not make any agreement or pay any reward for the creature's help. The creature departs at the end of the spell.

\smallskip\noindent If you choose to exact a longer or more involved form of service from a called creature, you must offer some fair trade in return for that service. The service exacted must be reasonable with respect to the promised favor or reward; see the lesser planar ally spell for appropriate rewards. (Some creatures may want their payment in ``livestock'' rather than in coin, which could involve complications.) Immediately upon completion of the service, the being is transported to your vicinity, and you must then and there turn over the promised reward. After this is done, the creature is instantly freed to return to its own plane.

\smallskip\noindent Failure to fulfill the promise to the letter results in your being subjected to service by the creature or by its liege and master, at the very least. At worst, the creature or its kin may attack you.}

\shortability{Note:}{When you use a calling spell such as gate to call an air, chaotic, earth, evil, fire, good, lawful, or water creature, it becomes a spell of that type.}
\shortability{XP Cost:}{1,000 XP (only for the calling creatures function).}

\featname{Geas/Quest}
\shortability{Enchantment}{(Compulsion) [Language-Dependent, Mind-Affecting]}
\shortability{Level:}{Brd 6, Clr 6, Sor/Wiz 6}
\shortability{Casting Time:}{10 minutes}
\shortability{Target:}{One living creature}
\shortability{Saving Throw:}{None}
\smallskip\noindent This spell functions similarly to lesser geas, except that it affects a creature of any HD and allows no saving throw.

\smallskip\noindent Instead of taking penalties to ability scores (as with lesser geas), the subject takes 3d6 points of damage each day it does not attempt to follow the geas/quest. Additionally, each day it must make a Fortitude saving throw or become sickened. These effects end 24 hours after the creature attempts to resume the geas/ quest.

\smallskip\noindent A remove curse spell ends a geas/quest spell only if its caster level is at least two higher than your caster level. Break enchantment does not end a geas/quest, but limited wish, miracle, and wish do.

\smallskip\noindent Bards, sorcerers, and wizards usually refer to this spell as geas, while clerics call the same spell quest.

\featname{Geas, Lesser}
\shortability{Enchantment}{(Compulsion) [Language-Dependent, Mind-Affecting]}
\shortability{Level:}{Brd 3, Sor/Wiz 4}
\shortability{Components:}{V}
\shortability{Casting Time:}{1 round}
\shortability{Range:}{Close (25 ft. + 5 ft./2 levels)}
\shortability{Target:}{One living creature with 7 HD or less}
\shortability{Duration:}{One day/level or until discharged (D)}
\shortability{Saving Throw:}{Will negates}
\shortability{Spell Resistance:}{Yes}
\smallskip\noindent A lesser geas places a magical command on a creature to carry out some service or to refrain from some action or course of activity, as desired by you. The creature must have 7 or fewer Hit Dice and be able to understand you. While a geas cannot compel a creature to kill itself or perform acts that would result in certain death, it can cause almost any other course of activity.

\smallskip\noindent The geased creature must follow the given instructions until the geas is completed, no matter how long it takes.

\smallskip\noindent If the instructions involve some open-ended task that the recipient cannot complete through his own actions the spell remains in effect for a maximum of one day per caster level. A clever recipient can subvert some instructions:

\smallskip\noindent If the subject is prevented from obeying the lesser geas for 24 hours, it takes a -2 penalty to each of its ability scores. Each day, another -2 penalty accumulates, up to a total of -8. No ability score can be reduced to less than 1 by this effect. The ability score penalties are removed 24 hours after the subject resumes obeying the lesser geas.

\smallskip\noindent A lesser geas (and all ability score penalties) can be ended by break enchantment, limited wish, remove curse, miracle, or wish. Dispel magic does not affect a lesser geas.

\featname{Gentle Repose}
\shortability{Necromancy}{}
\shortability{Level:}{Clr 2, Sor/Wiz 3}
\shortability{Components:}{V, S, M/DF}
\shortability{Casting Time:}{1 standard action}
\shortability{Range:}{Touch}
\shortability{Target:}{Corpse touched}
\shortability{Duration:}{One day/level}
\shortability{Saving Throw:}{Will negates (object)}
\shortability{Spell Resistance:}{Yes (object)}
\smallskip\noindent You preserve the remains of a dead creature so that they do not decay. Doing so effectively extends the time limit on raising that creature from the dead (see raise dead). Days spent under the influence of this spell don't count against the time limit. Additionally, this spell makes transporting a fallen comrade more pleasant.

\smallskip\noindent The spell also works on severed body parts and the like.

\shortability{Arcane Material Component:}{A pinch of salt, and a copper piece for each eye the corpse has (or had).}

\featname{Ghost Sound}
\shortability{Illusion}{(Figment)}
\shortability{Level:}{Brd 0, Sor/Wiz 0}
\shortability{Components:}{V, S, M}
\shortability{Casting Time:}{1 standard action}
\shortability{Range:}{Close (25 ft. + 5 ft./2 levels)}
\shortability{Effect:}{Illusory sounds}
\shortability{Duration:}{1 round/level (D)}
\shortability{Saving Throw:}{Will disbelief (if interacted with)}
\shortability{Spell Resistance:}{No}
\smallskip\noindent Ghost sound allows you to create a volume of sound that rises, recedes, approaches, or remains at a fixed place. You choose what type of sound ghost sound creates when casting it and cannot thereafter change the sound's basic character.

\smallskip\noindent The volume of sound created depends on your level. You can produce as much noise as four normal humans per caster level (maximum twenty humans). Thus, talking, singing, shouting, walking, marching, or running sounds can be created. The noise a ghost sound spell produces can be virtually any type of sound within the volume limit. A horde of rats running and squeaking is about the same volume as eight humans running and shouting. A roaring lion is equal to the noise from sixteen humans, while a roaring dire tiger is equal to the noise from twenty humans.

\smallskip\noindent Ghost sound can enhance the effectiveness of a silent image spell.

\smallskip\noindent Ghost sound can be made permanent with a permanency spell.

\shortability{Material Component:}{A bit of wool or a small lump of wax.}

\featname{Ghoul Touch}
\shortability{Necromancy}{}
\shortability{Level:}{Sor/Wiz 2}
\shortability{Components:}{V, S, M}
\shortability{Casting Time:}{1 standard action}
\shortability{Range:}{Touch}
\shortability{Target:}{Living humanoid touched}
\shortability{Duration:}{1d6+2 rounds}
\shortability{Saving Throw:}{Fortitude negates}
\shortability{Spell Resistance:}{Yes}
\smallskip\noindent Imbuing you with negative energy, this spell allows you to paralyze a single living humanoid for the duration of the spell with a successful melee touch attack.

\smallskip\noindent Additionally, the paralyzed subject exudes a carrion stench that causes all living creatures (except you) in a 10-foot-radius spread to become sickened (Fortitude negates). A neutralize poison spell removes the effect from a sickened creature, and creatures immune to poison are unaffected by the stench.

\shortability{Material Component:}{A small scrap of cloth taken from clothing worn by a ghoul, or a pinch of earth from a ghoul's lair.}

\featname{Giant Vermin}
\shortability{Transmutation}{}
\shortability{Level:}{Clr 4, Drd 4}
\shortability{Components:}{V, S, DF}
\shortability{Casting Time:}{1 standard action}
\shortability{Range:}{Close (25 ft. + 5 ft./2 levels)}
\shortability{Targets:}{Up to three vermin, no two of which can be more than 30 ft. apart}
\shortability{Duration:}{1 min./level}
\shortability{Saving Throw:}{None}
\shortability{Spell Resistance:}{Yes}
\smallskip\noindent You turn three normal-sized centipedes, two normal-sized spiders, or a single normal-sized scorpion into larger forms. Only one type of vermin can be transmuted (so a single casting cannot affect both a centipede and a spider), and all must be grown to the same size. The size to which the vermin can be grown depends on your level; see the table below.

\smallskip\noindent Any giant vermin created by this spell do not attempt to harm you, but your control of such creatures is limited to simple commands ("Attack," "Defend," "Stop," and so forth). Orders to attack a certain creature when it appears or guard against a particular occurrence are too complex for the vermin to understand. Unless commanded to do otherwise, the giant vermin attack whoever or whatever is near them.

\begin{tabular}[h!]{l|l}
Caster Level & Vermin Size \\ \hline
9th or lower & Medium \\
10th-13th & Large \\
14th-17th & Huge \\
18th-19th & Gargantuan \\
20th or higher & Colossal \\
\end{tabular}

\featname{Glibness}
\shortability{Transmutation}{}
\shortability{Level:}{Brd 3}
\shortability{Components:}{S}
\shortability{Casting Time:}{1 standard action}
\shortability{Range:}{Personal}
\shortability{Target:}{You}
\shortability{Duration:}{10 min./level (D)}
\smallskip\noindent Your speech becomes fluent and more believable. You gain a +30 bonus on Bluff checks made to convince another of the truth of your words. (This bonus doesn't apply to other uses of the Bluff skill, such as feinting in combat, creating a diversion to hide, or communicating a hidden message via innuendo.)

\smallskip\noindent If a magical effect is used against you that would detect your lies or force you to speak the truth the user of the effect must succeed on a caster level check (1d20 + caster level) against a DC of 15 + your caster level to succeed. Failure means the effect does not detect your lies or force you to speak only the truth.

\featname{Glitterdust}
\shortability{Conjuration}{(Creation)}
\shortability{Level:}{Brd 2, Sor/Wiz 2}
\shortability{Components:}{V, S, M}
\shortability{Casting Time:}{1 standard action}
\shortability{Range:}{Medium (100 ft. + 10 ft./level)}
\shortability{Area:}{Creatures and objects within 10-ft.-radius spread}
\shortability{Duration:}{1 round/level}
\shortability{Saving Throw:}{Will negates (blinding only)}
\shortability{Spell Resistance:}{No}
\smallskip\noindent A cloud of golden particles covers everyone and everything in the area, causing creatures to become blinded and visibly outlining invisible things for the duration of the spell. All within the area are covered by the dust, which cannot be removed and continues to sparkle until it fades.

\smallskip\noindent Any creature covered by the dust takes a -40 penalty on Hide checks.

\shortability{Material Component:}{Ground mica.}

\featname{Globe of Invulnerability}
\shortability{Abjuration}{}
\shortability{Level:}{Sor/Wiz 6}
\smallskip\noindent This spell functions like lesser globe of invulnerability, except that it also excludes 4th-level spells and spell-like effects.

\featname{Globe of Invulnerability, Lesser}
\shortability{Abjuration}{}
\shortability{Level:}{Sor/Wiz 4}
\shortability{Components:}{V, S, M}
\shortability{Casting Time:}{1 standard action}
\shortability{Range:}{10 ft.}
\shortability{Area:}{10-ft.-radius spherical emanation, centered on you}
\shortability{Duration:}{1 round/level (D)}
\shortability{Saving Throw:}{None}
\shortability{Spell Resistance:}{No}
\smallskip\noindent An immobile, faintly shimmering magical sphere surrounds you and excludes all spell effects of 3rd level or lower. The area or effect of any such spells does not include the area of the lesser globe of invulnerability. Such spells fail to affect any target located within the globe. Excluded effects include spell-like abilities and spells or spell-like effects from items. However, any type of spell can be cast through or out of the magical globe. Spells of 4th level and higher are not affected by the globe, nor are spells already in effect when the globe is cast. The globe can be brought down by a targeted dispel magic spell, but not by an area dispel magic. You can leave and return to the globe without penalty.

\smallskip\noindent Note that spell effects are not disrupted unless their effects enter the globe, and even then they are merely suppressed, not dispelled. 

\smallskip\noindent If a given spell has more than one level depending on which character class is casting it, use the level appropriate to the caster to determine whether lesser globe of invulnerability stops it.

\shortability{Material Component:}{A glass or crystal bead that shatters at the expiration of the spell.}

\featname{Glyph of Warding}
\shortability{Abjuration}{}
\shortability{Level:}{Clr 3}
\shortability{Components:}{V, S, M}
\shortability{Casting Time:}{10 minutes}
\shortability{Range:}{Touch}
\shortability{Target or Area:}{Object touched or up to 5 sq. ft./level}
\shortability{Duration:}{Permanent until discharged (D)}
\shortability{Saving Throw:}{See text}
\shortability{Spell Resistance:}{No (object) and Yes; see text}
\smallskip\noindent This powerful inscription harms those who enter, pass, or open the warded area or object. A glyph of warding can guard a bridge or passage, ward a portal, trap a chest or box, and so on.

\smallskip\noindent You set the conditions of the ward. Typically, any creature entering the warded area or opening the warded object without speaking a password (which you set when casting the spell) is subject to the magic it stores. Alternatively or in addition to a password trigger, glyphs can be set according to physical characteristics (such as height or weight) or creature type, subtype, or kind. Glyphs can also be set with respect to good, evil, law, or chaos, or to pass those of your religion. They cannot be set according to class, Hit Dice, or level. Glyphs respond to invisible creatures normally but are not triggered by those who travel past them ethereally. Multiple glyphs cannot be cast on the same area. However, if a cabinet has three drawers, each can be separately warded.

\smallskip\noindent When casting the spell, you weave a tracery of faintly glowing lines around the warding sigil. A glyph can be placed to conform to any shape up to the limitations of your total square footage. When the spell is completed, the glyph and tracery become nearly invisible.

\smallskip\noindent Glyphs cannot be affected or bypassed by such means as physical or magical probing, though they can be dispelled. Mislead, polymorph, and nondetection (and similar magical effects) can fool a glyph, though nonmagical disguises and the like can't. Read magic allows you to identify a glyph of warding with a DC 13 Spellcraft check. Identifying the glyph does not discharge it and allows you to know the basic nature of the glyph (version, type of damage caused, what spell is stored).

\shortability{Note:}{Magic traps such as glyph of warding are hard to detect and disable. A rogue (only) can use the Search skill to find the glyph and Disable Device to thwart it. The DC in each case is 25 + spell level, or 28 for glyph of warding.}

\smallskip\noindent Depending on the version selected, a glyph either blasts the intruder or activates a spell.

\shortability{Blast Glyph:}{A blast glyph deals 1d8 points of damage per two caster levels (maximum 5d8) to the intruder and to all within 5 feet of him or her. This damage is acid, cold, fire, electricity, or sonic (caster's choice, made at time of casting). Each creature affected can attempt a Reflex save to take half damage. Spell resistance applies against this effect.}
\shortability{Spell Glyph:}{You can store any harmful spell of 3rd level or lower that you know. All level-dependent features of the spell are based on your caster level at the time of casting the glyph. If the spell has a target, it targets the intruder. If the spell has an area or an amorphous effect the area or effect is centered on the intruder. If the spell summons creatures, they appear as close as possible to the intruder and attack. Saving throws and spell resistance operate as normal, except that the DC is based on the level of the spell stored in the glyph.}
\shortability{Material Component:}{You trace the glyph with incense, which must first be sprinkled with powdered diamond worth at least 200 gp.}

\featname{Glyph of Warding, Greater}
\shortability{Abjuration}{}
\shortability{Level:}{Clr 6}

\smallskip\noindent This spell functions like glyph of warding, except that a greater blast glyph deals up to 10d8 points of damage, and a greater spell glyph can store a spell of 6th level or lower.

\smallskip\noindent \shortability{Material Component:}{You trace the glyph with incense, which must first be sprinkled with powdered diamond worth at least 400 gp.}

\featname{Goodberry}
\shortability{Transmutation}{}
\shortability{Level:}{Drd 1}
\shortability{Components:}{V, S, DF}
\shortability{Casting Time:}{1 standard action}
\shortability{Range:}{Touch}
\shortability{Targets:}{2d4 fresh berries touched}
\shortability{Duration:}{One day/level}
\shortability{Saving Throw:}{None}
\shortability{Spell Resistance:}{Yes}
\smallskip\noindent Casting goodberry upon a handful of freshly picked berries makes 2d4 of them magical. You (as well as any other druid of 3rd or higher level) can immediately discern which berries are affected. Each transmuted berry provides nourishment as if it were a normal meal for a Medium creature. The berry also cures 1 point of damage when eaten, subject to a maximum of 8 points of such curing in any 24-hour period.

\featname{Good Hope}
\shortability{Enchantment}{(Compulsion) [Mind-Affecting]}
\shortability{Level:}{Brd 3}
\shortability{Components:}{V, S}
\shortability{Casting Time:}{1 standard action}
\shortability{Range:}{Medium (100 ft. + 10 ft./level)}
\shortability{Targets:}{One living creature/level, no two of which may be more than 30 ft. apart}
\shortability{Duration:}{1 min./level}
\shortability{Saving Throw:}{Will negates (harmless)}
\shortability{Spell Resistance:}{Yes (harmless)}
\smallskip\noindent This spell instills powerful hope in the subjects. Each affected creature gains a +2 morale bonus on saving throws, attack rolls, ability checks, skill checks, and weapon damage rolls.

\smallskip\noindent Good hope counters and dispels crushing despair.

\featname{Grasping Hand}
\shortability{Evocation}{[Force]}
\shortability{Level:}{Sor/Wiz 7, Strength 7}
\shortability{Components:}{V, S, F/DF}
\smallskip\noindent This spell functions like interposing hand, except the hand can also grapple one opponent that you select. The grasping hand gets one grapple attack per round.

\smallskip\noindent Its attack bonus to make contact equals your caster level + your Intelligence, Wisdom, or Charisma modifier (for wizards, clerics, and sorcerers, respectively), +10 for the hand's Strength score (31), -1 for being Large. Its grapple bonus is this same figure, except with a +4 modifier for being Large instead of -1. 

\smallskip\noindent The hand holds but does not harm creatures it grapples.

\smallskip\noindent Directing the spell to a new target is a move action.

\smallskip\noindent The grasping hand can also bull rush an opponent as forceful hand does, but at a +16 bonus on the Strength check (+10 for Strength 35, +4 for being Large, and a +2 bonus for charging, which it always gets), or interpose itself as interposing hand does.

\smallskip\noindent Clerics who cast this spell name it for their deities.

\shortability{Arcane Focus:}{A leather glove.}

\featname{Grease}
\shortability{Conjuration}{(Creation)}
\shortability{Level:}{Brd 1, Sor/Wiz 1}
\shortability{Components:}{V, S, M}
\shortability{Casting Time:}{1 standard action}
\shortability{Range:}{Close (25 ft. + 5 ft./2 levels)}
\shortability{Target or Area:}{One object or a 10-ft. square}
\shortability{Duration:}{1 round/level (D)}
\shortability{Saving Throw:}{See text}
\shortability{Spell Resistance:}{No}
\smallskip\noindent A grease spell covers a solid surface with a layer of slippery grease. Any creature in the area when the spell is cast must make a successful Reflex save or fall. This save is repeated on your turn each round that the creature remains within the area. A creature can walk within or through the area of grease at half normal speed with a DC 10 Balance check. Failure means it can't move that round (and must then make a Reflex save or fall), while failure by 5 or more means it falls (see the Balance skill for details).

\smallskip\noindent The spell can also be used to create a greasy coating on an item. Material objects not in use are always affected by this spell, while an object wielded or employed by a creature receives a Reflex saving throw to avoid the effect. If the initial saving throw fails, the creature immediately drops the item. A saving throw must be made in each round that the creature attempts to pick up or use the greased item. A creature wearing greased armor or clothing gains a +10 circumstance bonus on Escape Artist checks and on grapple checks made to resist or escape a grapple or to escape a pin.

\shortability{Material Component:}{A bit of pork rind or butter.}

\featname{Greater (Spell Name)}
\smallskip\noindent Any spell whose name begins with greater is alphabetized in this chapter according to the second word of the spell name. Thus, the description of a greater spell appears near the description of the spell on which it is based. Spell chains that have greater spells in them include those based on the spells arcane sight, command, dispel magic, glyph of warding, invisibility, magic fang, magic weapon, planar ally, planar binding, prying eyes, restoration, scrying, shadow conjuration, shadow evocation, shout, and teleport.

\featname{Guards and Wards}
\shortability{Abjuration}{}
\shortability{Level:}{Sor/Wiz 6}
\shortability{Components:}{V, S, M, F}
\shortability{Casting Time:}{30 minutes}
\shortability{Range:}{Anywhere within the area to be warded}
\shortability{Area:}{Up to 200 sq. ft./level (S)}
\shortability{Duration:}{2 hours/level (D)}
\shortability{Saving Throw:}{See text}
\shortability{Spell Resistance:}{See text}
\smallskip\noindent This powerful spell is primarily used to defend your stronghold. The ward protects 200 square feet per caster level. The warded area can be as much as 20 feet high, and shaped as you desire. You can ward several stories of a stronghold by dividing the area among them; you must be somewhere within the area to be warded to cast the spell. The spell creates the following magical effects within the warded area.

\shortability{Fog:}{Fog fills all corridors, obscuring all sight, including darkvision, beyond 5 feet. A creature within 5 feet has concealment (attacks have a 20\% miss chance). Creatures farther away have total concealment (50\% miss chance, and the attacker cannot use sight to locate the target). \textbf{Saving Throw:}None. \textbf{Spell Resistance:}No.}
\shortability{Arcane Locks:}{All doors in the warded area are arcane locked. \shortability{Saving Throw:}{None. \textbf{Spell Resistance:}No.}}
\shortability{Webs:}{Webs fill all stairs from top to bottom. These strands are identical with those created by the web spell, except that they regrow in 10 minutes if they are burned or torn away while the guards and wards spell lasts. \textbf{Saving Throw:}Reflex negates; see text for web. \textbf{Spell Resistance:}No.}
\shortability{Confusion:}{Where there are choices in direction-such as a corridor intersection or side passage-a minor confusion-type effect functions so as to make it 50\% probable that intruders believe they are going in the opposite direction from the one they actually chose. This is an enchantment, mind-affecting effect. \textbf{Saving Throw:}None. \textbf{Spell Resistance:}Yes.}
\shortability{Lost Doors:}{One door per caster level is covered by a silent image to appear as if it were a plain wall. \textbf{Saving Throw:}Will disbelief (if interacted with). \shortability{Spell Resistance:}{No.}}

\smallskip\noindent In addition, you can place your choice of one of the following five magical effects.

\listtwo
  \item Dancing lights in four corridors. You can designate a simple program that causes the lights to repeat as long as the guards and wards spell lasts. \shortability{Saving Throw:}{None. \shortability{Spell Resistance:}{No.}}
  \item A magic mouth in two places. \shortability{Saving Throw:}{None. \shortability{Spell Resistance:}{No.}}
  \item A stinking cloud in two places. The vapors appear in the places you designate; they return within 10 minutes if dispersed by wind while the guards and wards spell lasts. \shortability{Saving Throw:}{Fortitude negates; see text for stinking cloud. \shortability{Spell Resistance:}{No.}}
  \item A gust of wind in one corridor or room. \shortability{Saving Throw:}{Fortitude negates. \shortability{Spell Resistance:}{Yes.}}
  \item A suggestion in one place. You select an area of up to 5 feet square, and any creature who enters or passes through the area receives the suggestion mentally. \shortability{Saving Throw:}{Will negates. \shortability{Spell Resistance:}{Yes.}}
\end{list}

\smallskip\noindent The whole warded area radiates strong magic of the abjuration school. A dispel magic cast on a specific effect, if successful, removes only that effect. A successful Mage's disjunction destroys the entire guards and wards effect.

\shortability{Material Component:}{Burning incense, a small measure of brimstone and oil, a knotted string, and a small amount of blood.}
\shortability{Focus:}{A small silver rod.}

\featname{Guidance}
\shortability{Divination}{}
\shortability{Level:}{Clr 0, Drd 0}
\shortability{Components:}{V, S}
\shortability{Casting Time:}{1 standard action}
\shortability{Range:}{Touch}
\shortability{Target:}{Creature touched}
\shortability{Duration:}{1 minute or until discharged}
\shortability{Saving Throw:}{Will negates (harmless)}
\shortability{Spell Resistance:}{Yes}
\smallskip\noindent This spell imbues the subject with a touch of divine guidance. The creature gets a +1 competence bonus on a single attack roll, saving throw, or skill check. It must choose to use the bonus before making the roll to which it applies.

\featname{Gust of Wind}
\shortability{Evocation}{[Air]}
\shortability{Level:}{Drd 2, Sor/Wiz 2}
\shortability{Components:}{V, S}
\shortability{Casting Time:}{1 standard action}
\shortability{Range:}{60 ft.}
\shortability{Effect:}{Line-shaped gust of severe wind emanating out from you to the extreme of the range}
\shortability{Duration:}{1 round}
\shortability{Saving Throw:}{Fortitude negates}
\shortability{Spell Resistance:}{Yes}
\smallskip\noindent This spell creates a severe blast of air (approximately 50 mph) that originates from you, affecting all creatures in its path.

\smallskip\noindent A Tiny or smaller creature on the ground is knocked down and rolled 1d4x10 feet, taking 1d4 points of nonlethal damage per 10 feet. If flying, a Tiny or smaller creature is blown back 2d6x10 feet and takes 2d6 points of nonlethal damage due to battering and buffeting.

\smallskip\noindent Small creatures are knocked prone by the force of the wind, or if flying are blown back 1d6x10 feet.

\smallskip\noindent Medium creatures are unable to move forward against the force of the wind, or if flying are blown back 1d6x5 feet.

\smallskip\noindent Large or larger creatures may move normally within a gust of wind effect.

\smallskip\noindent A gust of wind can't move a creature beyond the limit of its range.

\smallskip\noindent Any creature, regardless of size, takes a -4 penalty on ranged attacks and Listen checks in the area of a gust of wind.

\smallskip\noindent The force of the gust automatically extinguishes candles, torches, and similar unprotected flames. It causes protected flames, such as those of lanterns, to dance wildly and has a 50\% chance to extinguish those lights.

\smallskip\noindent In addition to the effects noted, a gust of wind can do anything that a sudden blast of wind would be expected to do. It can create a stinging spray of sand or dust, fan a large fire, overturn delicate awnings or hangings, heel over a small boat, and blow gases or vapors to the edge of its range.

\smallskip\noindent Gust of wind can be made permanent with a permanency spell.

\featname{Hallow}
\shortability{Evocation}{[Good]}
\shortability{Level:}{Clr 5, Drd 5}
\shortability{Components:}{V, S, M, DF}
\shortability{Casting Time:}{24 hours}
\shortability{Range:}{Touch}
\shortability{Area:}{40-ft. radius emanating from the touched point}
\shortability{Duration:}{Instantaneous}
\shortability{Saving Throw:}{See text}
\shortability{Spell Resistance:}{See text}
\smallskip\noindent Hallow makes a particular site, building, or structure a holy site. This has four major effects.

\smallskip\noindent First, the site or structure is guarded by a magic circle against evil effect.

\smallskip\noindent Second, all Charisma checks made to turn undead gain a +4 sacred bonus, and Charisma checks to command undead take a -4 penalty. Spell resistance does not apply to this effect. (This provision does not apply to the druid version of the spell.)

\smallskip\noindent Third, any dead body interred in a hallowed site cannot be turned into an undead creature.

\smallskip\noindent Finally, you may choose to fix a single spell effect to the hallowed site. The spell effect lasts for one year and functions throughout the entire site, regardless of the normal duration and area or effect. You may designate whether the effect applies to all creatures, creatures who share your faith or alignment, or creatures who adhere to another faith or alignment. At the end of the year, the chosen effect lapses, but it can be renewed or replaced simply by casting hallow again.

\smallskip\noindent Spell effects that may be tied to a hallowed site include aid, bane, bless, cause fear, darkness, daylight, death ward, deeper darkness, detect evil, detect magic, dimensional anchor, discern lies, dispel magic, endure elements, freedom of movement, invisibility purge, protection from energy, remove fear, resist energy, silence, tongues, and zone of truth. Saving throws and spell resistance might apply to these spells' effects. (See the individual spell descriptions for details.)

\smallskip\noindent An area can receive only one hallow spell (and its associated spell effect) at a time. Hallow counters but does not dispel unhallow.

\shortability{Material Component:}{Herbs, oils, and incense worth at least 1,000 gp, plus 1,000 gp per level of the spell to be included in the hallowed area.}

\featname{Hallucinatory Terrain}
\shortability{Illusion}{(Glamer)}
\shortability{Level:}{Brd 4, Sor/Wiz 4}
\shortability{Components:}{V, S, M}
\shortability{Casting Time:}{10 minutes}
\shortability{Range:}{Long (400 ft. + 40 ft./level)}
\shortability{Area:}{One 30-ft. cube/level (S)}
\shortability{Duration:}{2 hours/level (D)}
\shortability{Saving Throw:}{Will disbelief (if interacted with)}
\shortability{Spell Resistance:}{No}
\smallskip\noindent You make natural terrain look, sound, and smell like some other sort of natural terrain. Structures, equipment, and creatures within the area are not hidden or changed in appearance.

\shortability{Material Component:}{A stone, a twig, and a bit of green plant.}

\featname{Halt Undead}
\shortability{Necromancy}{}
\shortability{Level:}{Sor/Wiz 3}
\shortability{Components:}{V, S, M}
\shortability{Casting Time:}{1 standard action}
\shortability{Range:}{Medium (100 ft. + 10 ft./level)}
\shortability{Targets:}{Up to three undead creatures, no two of which can be more than 30 ft. apart}
\shortability{Duration:}{1 round/level}
\shortability{Saving Throw:}{Will negates (see text)}
\shortability{Spell Resistance:}{Yes}
\smallskip\noindent This spell renders as many as three undead creatures immobile. A nonintelligent undead creature gets no saving throw; an intelligent undead creature does. If the spell is successful, it renders the undead creature immobile for the duration of the spell (similar to the effect of hold person on a living creature). The effect is broken if the halted creatures are attacked or take damage.

\shortability{Material Component:}{A pinch of sulfur and powdered garlic.}

\featname{Harm}
\shortability{Necromancy}{}
\shortability{Level:}{Clr 6, Destruction 6}
\shortability{Components:}{V, S}
\shortability{Casting Time:}{1 standard action}
\shortability{Range:}{Touch}
\shortability{Target:}{Creature touched}
\shortability{Duration:}{Instantaneous}
\shortability{Saving Throw:}{Will half; see text}
\shortability{Spell Resistance:}{Yes}
\smallskip\noindent Harm charges a subject with negative energy that deals 10 points of damage per caster level (to a maximum of 150 points at 15th level). If the creature successfully saves, harm deals half this amount, but it cannot reduce the target's hit points to less than 1.

\smallskip\noindent If used on an undead creature, harm acts like heal.

\featname{Haste}
\shortability{Transmutation}{}
\shortability{Level:}{Brd 3, Sor/Wiz 3}
\shortability{Components:}{V, S, M}
\shortability{Casting Time:}{1 standard action}
\shortability{Range:}{Close (25 ft. + 5 ft./2 levels)}
\shortability{Targets:}{One creature/level, no two of which can be more than 30 ft. apart}
\shortability{Duration:}{1 round/level}
\shortability{Saving Throw:}{Fortitude negates (harmless)}
\shortability{Spell Resistance:}{Yes (harmless)}
\smallskip\noindent The transmuted creatures move and act more quickly than normal. This extra speed has several effects.

\smallskip\noindent When making a full attack action, a hasted creature may make one extra attack with any weapon he is holding. The attack is made using the creature's full base attack bonus, plus any modifiers appropriate to the situation. (This effect is not cumulative with similar effects, such as that provided by a weapon of speed, nor does it actually grant an extra action, so you can't use it to cast a second spell or otherwise take an extra action in the round.)

\smallskip\noindent A hasted creature gains a +1 bonus on attack rolls and a +1 dodge bonus to AC and Reflex saves. Any condition that makes you lose your Dexterity bonus to Armor Class (if any) also makes you lose dodge bonuses.

\smallskip\noindent All of the hasted creature's modes of movement (including land movement, burrow, climb, fly, and swim) increase by 30 feet, to a maximum of twice the subject's normal speed using that form of movement. This increase counts as an enhancement bonus, and it affects the creature's jumping distance as normal for increased speed.

\smallskip\noindent Multiple haste effects don't stack. Haste dispels and counters slow.

\shortability{Material Component:}{A shaving of licorice root.}

\featname{Heal}
\shortability{Conjuration}{(Healing)}
\shortability{Level:}{Clr 6, Drd 7, Healing 6}
\shortability{Components:}{V, S}
\shortability{Casting Time:}{1 standard action}
\shortability{Range:}{Touch}
\shortability{Target:}{Creature touched}
\shortability{Duration:}{Instantaneous}
\shortability{Saving Throw:}{Will negates (harmless)}
\shortability{Spell Resistance:}{Yes (harmless)}
\smallskip\noindent Heal enables you to channel positive energy into a creature to wipe away injury and afflictions. It immediately ends any and all of the following adverse conditions affecting the Target: ability damage, blinded, confused, dazed, dazzled, deafened, diseased, exhausted, fatigued, feebleminded, insanity, nauseated, sickened, stunned, and poisoned. It also cures 10 hit points of damage per level of the caster, to a maximum of 150 points at 15th level.

\smallskip\noindent Heal does not remove negative levels, restore permanently drained levels, or restore permanently drained ability score points.

\smallskip\noindent If used against an undead creature, heal instead acts like harm.

\featname{Heal, Mass}
\shortability{Conjuration}{(Healing)}
\shortability{Level:}{Clr 9, Healing 9}
\shortability{Range:}{Close (25 ft. + 5 ft./2 levels)}
\shortability{Targets:}{One or more creatures, no two of which can be more than 30 ft. apart}
\smallskip\noindent This spell functions like heal, except as noted above. The maximum number of hit points restored to each creature is 250.

\featname{Heal Mount}
\shortability{Conjuration}{(Healing)}
\shortability{Level:}{Pal 3}
\shortability{Components:}{V, S}
\shortability{Casting Time:}{1 standard action}
\shortability{Range:}{Touch}
\shortability{Target:}{Your mount touched}
\shortability{Duration:}{Instantaneous}
\shortability{Saving Throw:}{Will negates (harmless)}
\shortability{Spell Resistance:}{Yes (harmless)}
\smallskip\noindent This spell functions like heal, but it affects only the paladin's special mount (typically a warhorse).

\featname{Heat Metal}
\shortability{Transmutation}{[Fire]}
\shortability{Level:}{Drd 2, Sun 2}
\shortability{Components:}{V, S, DF}
\shortability{Casting Time:}{1 standard action}
\shortability{Range:}{Close (25 ft. + 5 ft./2 levels)}
\shortability{Target:}{Metal equipment of one creature per two levels, no two of which can be more than 30 ft. apart; or 25 lb. of metal/level, all of which must be within a 30-ft. circle}
\shortability{Duration:}{7 rounds}
\shortability{Saving Throw:}{Will negates (object)}
\shortability{Spell Resistance:}{Yes (object)}
\smallskip\noindent Heat metal makes metal extremely warm. Unattended, nonmagical metal gets no saving throw. Magical metal is allowed a saving throw against the spell. An item in a creature's possession uses the creature's saving throw bonus unless its own is higher.

\smallskip\noindent A creature takes fire damage if its equipment is heated. It takes full damage if its armor is affected or if it is holding, touching, wearing, or carrying metal weighing one-fifth of its weight. The creature takes minimum damage (1 point or 2 points; see the table) if it's not wearing metal armor and the metal that it's carrying weighs less than one-fifth of its weight.

\smallskip\noindent On the first round of the spell, the metal becomes warm and uncomfortable to touch but deals no damage. The same effect also occurs on the last round of the spell's duration. During the second (and also the next-to-last) round, intense heat causes pain and damage. In the third, fourth, and fifth rounds, the metal is searing hot, causing more damage, as shown on the table below.

\begin{tabular}[h!]{l|ll}
Round & Metal Temperature & Damage \\ \hline
1 & Warm & None \\	   
2 & Hot & 1d4 points \\
3-5 & Searing & 2d4 points \\
6 & Hot & 1d4 points \\
7 & Warm & None \\
\end{tabular}

\smallskip\noindent Any cold intense enough to damage the creature negates fire damage from the spell (and vice versa) on a point-for-point basis. If cast underwater, heat metal deals half damage and boils the surrounding water.

\smallskip\noindent Heat metal counters and dispels chill metal.

\featname{Helping Hand}
\shortability{Evocatio}{}
\shortability{Level:}{Clr 3}
\shortability{Components:}{V, S, DF}
\shortability{Casting Time:}{1 standard action}
\shortability{Range:}{5 miles}
\shortability{Effect:}{Ghostly hand}
\shortability{Duration:}{1 hour/level}
\shortability{Saving Throw:}{None}
\shortability{Spell Resistance:}{No}
\smallskip\noindent You create the ghostly image of a hand, which you can send to find a creature within 5 miles. The hand then beckons to that creature and leads it to you if the creature is willing to follow.

\smallskip\noindent When the spell is cast, the hand appears in front of you. You then specify a person (or any creature) by physical description, which can include race, gender, and appearance but not ambiguous factors such as level, alignment, or class. When the description is complete, the hand streaks off in search of a subject that fits the description. The amount of time it takes to find the subject depends on how far away she is.

\begin{tabular}[h!]{l|l}
Distance & Time to Locate \\ \hline
100 ft. or less & 1 round \\
1,000 ft. & 1 minute \\
1 mile & 10 minutes \\
2 miles & 1 hour \\
3 miles & 2 hours	\\
4 miles & 3 hours	\\
5 miles & 4 hours \\
\end{tabular}

\smallskip\noindent Once the hand locates the subject, it beckons the creature to follow it. If the subject does so, the hand points in your direction, indicating the most direct feasible route. The hand hovers 10 feet in front of the subject, moving before it at a speed of as much as 240 feet per round. Once the hand leads the subject back to you, it disappears.

\smallskip\noindent The subject is not compelled to follow the hand or act in any particular way toward you. If the subject chooses not to follow, the hand continues to beckon for the duration of the spell, then disappears. If the spell expires while the subject is en route to you, the hand disappears; the subject must then rely on her own devices to locate you.

\smallskip\noindent If more than one subject in a 5-mile radius meets the description, the hand locates the closest creature. If that creature refuses to follow the hand, the hand does not seek out a second subject.

\smallskip\noindent If, at the end of 4 hours of searching, the hand has found no subject that matches the description within 5 miles, it returns to you, displays an outstretched palm (indicating that no such creature was found), and disappears.

\smallskip\noindent The ghostly hand has no physical form. It is invisible to anyone except you and a potential subject. It cannot engage in combat or execute any other task aside from locating a subject and leading it back to you. The hand can't pass through solid objects but can ooze through small cracks and slits. The hand cannot travel more than 5 miles from the spot it appeared when you cast the spell.

\featname{Heroes' Feast}
\shortability{Conjuration}{[Creation]}
\shortability{Level:}{Brd 6, Clr 6}
\shortability{Components:}{V, S, DF}
\shortability{Casting Time:}{10 minutes}
\shortability{Range:}{Close (25 ft. + 5 ft./2 levels)}
\shortability{Effect:}{Feast for one creature/level}
\shortability{Duration:}{1 hour plus 12 hours; see text}
\shortability{Saving Throw:}{None}
\shortability{Spell Resistance:}{No}
\smallskip\noindent You bring forth a great feast, including a magnificent table, chairs, service, and food and drink. The feast takes 1 hour to consume, and the beneficial effects do not set in until this hour is over. Every creature partaking of the feast is cured of all diseases, sickness, and nausea; becomes immune to poison for 12 hours; and gains 1d8 temporary hit points +1 point per two caster levels (maximum +10) after imbibing the nectar-like beverage that is part of the feast. The ambrosial food that is consumed grants each creature that partakes a +1 morale bonus on attack rolls and Will saves and immunity to fear effects for 12 hours.

\smallskip\noindent If the feast is interrupted for any reason, the spell is ruined and all effects of the spell are negated.

\featname{Heroism}
\shortability{Enchantment}{(Compulsion) [Mind-Affecting]}
\shortability{Level:}{Brd 2, Sor/Wiz 3}
\shortability{Components:}{V, S}
\shortability{Casting Time:}{1 standard action}
\shortability{Range:}{Touch}
\shortability{Target:}{Creature touched}
\shortability{Duration:}{10 min./level}
\shortability{Saving Throw:}{Will negates (harmless)}
\shortability{Spell Resistance:}{Yes (harmless)}
\smallskip\noindent This spell imbues a single creature with great bravery and morale in battle. The target gains a +2 morale bonus on attack rolls, saves, and skill checks.

\featname{Heroism, Greater}
\shortability{Enchantment}{(Compulsion) [Mind-Affecting]}
\shortability{Level:}{Brd 5, Sor/Wiz 6}
\shortability{Duration:}{1 min./level}
\smallskip\noindent This spell functions like heroism, except the creature gains a +4 morale bonus on attack rolls, saves, and skill checks, immunity to fear effects, and temporary hit points equal to your caster level (maximum 20).

\featname{Hide from Animals}
\shortability{Abjuration}{}
\shortability{Level:}{Drd 1, Rgr 1}
\shortability{Components:}{S, DF}
\shortability{Casting Time:}{1 standard action}
\shortability{Range:}{Touch}
\shortability{Targets:}{One creature touched/level}
\shortability{Duration:}{10 min./level (D)}
\shortability{Saving Throw:}{Will negates (harmless)}
\shortability{Spell Resistance:}{Yes}
\smallskip\noindent Animals cannot see, hear, or smell the warded creatures. Even extraordinary or supernatural sensory capabilities, such as blindsense, blindsight, scent, and tremorsense, cannot detect or locate warded creatures. Animals simply act as though the warded creatures are not there. If a warded character touches an animal or attacks any creature, even with a spell, the spell ends for all recipients.

\featname{Hide from Undead}
\shortability{Abjuration}{}
\shortability{Level:}{Clr 1}
\shortability{Components:}{V, S, DF}
\shortability{Casting Time:}{1 standard action}
\shortability{Range:}{Touch}
\shortability{Targets:}{One touched creature/level}
\shortability{Duration:}{10 min./level (D)}
\shortability{Saving Throw:}{Will negates (harmless); see text}
\shortability{Spell Resistance:}{Yes}
\smallskip\noindent Undead cannot see, hear, or smell the warded creatures. Even extraordinary or supernatural sensory capabilities, such as blindsense, blindsight, scent, and tremorsense, cannot detect or locate warded creatures. Nonintelligent undead creatures are automatically affected and act as though the warded creatures are not there. An intelligent undead creature gets a single Will saving throw. If it fails, the subject can't see any of the warded creatures. However, if it has reason to believe unseen opponents are present, it can attempt to find or strike them. If a warded creature attempts to turn or command undead, touches an undead creature, or attacks any creature (even with a spell), the spell ends for all recipients.

\featname{Hideous Laughter}
\shortability{Enchantment}{(Compulsion) [Mind-Affecting]}
\shortability{Level:}{Brd 1, Sor/Wiz 2}
\shortability{Components:}{V, S, M}
\shortability{Casting Time:}{1 standard action}
\shortability{Range:}{Close (25 ft. + 5 ft./2 levels)}
\shortability{Target:}{One creature; see text}
\shortability{Duration:}{1 round/level}
\shortability{Saving Throw:}{Will negates}
\shortability{Spell Resistance:}{Yes}
\smallskip\noindent This spell afflicts the subject with uncontrollable laughter. It collapses into gales of manic laughter, falling prone. The subject can take no actions while laughing, but is not considered helpless. After the spell ends, it can act normally.

\smallskip\noindent A creature with an Intelligence score of 2 or lower is not affected. A creature whose type is different from the caster's receives a +4 bonus on its saving throw, because humor doesn't ``translate'' well.

\shortability{Material Component:}{Tiny tarts that are thrown at the target and a feather that is waved in the air.}

\featname{Hold Animal}
\shortability{Enchantment}{(Compulsion) [Mind-Affecting]}
\shortability{Level:}{Animal 2, Drd 2, Rgr 2}
\shortability{Components:}{V, S}
\shortability{Target:}{One animal}
\smallskip\noindent This spell functions like hold person, except that it affects an animal instead of a humanoid.

\featname{Hold Monster}
\shortability{Enchantment}{(Compulsion) [Mind-Affecting]}
\shortability{Level:}{Brd 4, Law 6, Sor/Wiz 5}
\shortability{Components:}{V, S, M/DF}
\shortability{Target:}{One living creature}
\smallskip\noindent This spell functions like hold person, except that it affects any living creature that fails its Will save.

\shortability{Arcane Material Component:}{One hard metal bar or rod, which can be as small as a three-penny nail.}

\featname{Hold Monster, Mass}
\shortability{Enchantment}{(Compulsion) [Mind-Affecting]}
\shortability{Level:}{Sor/Wiz 9}
\shortability{Targets:}{One or more creatures, no two of which can be more than 30 ft. apart}
\smallskip\noindent This spell functions like hold person, except that it affects multiple creatures and holds any living creature that fails its Will save.

\featname{Hold Person}
\shortability{Enchantment}{(Compulsion) [Mind-Affecting]}
\shortability{Level:}{Brd 2, Clr 2, Sor/Wiz 3}
\shortability{Components:}{V, S, F/DF}
\shortability{Casting Time:}{1 standard action}
\shortability{Range:}{Medium (100 ft. + 10 ft./level)}
\shortability{Target:}{One humanoid creature}
\shortability{Duration:}{1 round/level (D); see text}
\shortability{Saving Throw:}{Will negates; see text}
\shortability{Spell Resistance:}{Yes}
\smallskip\noindent The subject becomes paralyzed and freezes in place. It is aware and breathes normally but cannot take any actions, even speech. Each round on its turn, the subject may attempt a new saving throw to end the effect. (This is a full-round action that does not provoke attacks of opportunity.)

\smallskip\noindent A winged creature who is paralyzed cannot flap its wings and falls. A swimmer can't swim and may drown.

\shortability{Arcane Focus:}{A small, straight piece of iron.}

\featname{Hold Person, Mass}
\shortability{Enchantment}{(Compulsion) [Mind-Affecting]}
\shortability{Level:}{Sor/Wiz 7}
\shortability{Targets:}{One or more humanoid creatures, no two of which can be more than 30 ft. apart}
\smallskip\noindent This spell functions like hold person, except as noted above.

\featname{Hold Portal}
\shortability{Abjuration}{}
\shortability{Level:}{Sor/Wiz 1}
\shortability{Component:}{V}
\shortability{Casting Time:}{1 standard action}
\shortability{Range:}{Medium (100 ft. + 10 ft./level)}
\shortability{Target:}{One portal, up to 20 sq. ft./level}
\shortability{Duration:}{1 min./level (D)}
\shortability{Saving Throw:}{None}
\shortability{Spell Resistance:}{No}
\smallskip\noindent This spell magically holds shut a door, gate, window, or shutter of wood, metal, or stone. The magic affects the portal just as if it were securely closed and normally locked. A knock spell or a successful dispel magic spell can negate a hold portal spell.

\smallskip\noindent For a portal affected by this spell, add 5 to the normal DC for forcing open the portal.

\featname{Holy Aura}
\shortability{Abjuration}{[Good]}
\shortability{Level:}{Clr 8, Good 8}
\shortability{Components:}{V, S, F}
\shortability{Casting Time:}{1 standard action}
\shortability{Range:}{20 ft.}
\shortability{Targets:}{One creature/level in a 20-ft.-radius burst centered on you}
\shortability{Duration:}{1 round/level (D)}
\shortability{Saving Throw:}{See text}
\shortability{Spell Resistance:}{Yes (harmless)}
\smallskip\noindent A brilliant divine radiance surrounds the subjects, protecting them from attacks, granting them resistance to spells cast by evil creatures, and causing evil creatures to become blinded when they strike the subjects. This abjuration has four effects.

\smallskip\noindent First, each warded creature gains a +4 deflection bonus to AC and a +4 resistance bonus on saves. Unlike protection from evil, this benefit applies against all attacks, not just against attacks by evil creatures.
Second, each warded creature gains spell resistance 25 against evil spells and spells cast by evil creatures.

\smallskip\noindent Third, the abjuration blocks possession and mental influence, just as protection from evil does.

\smallskip\noindent Finally, if an evil creature succeeds on a melee attack against a warded creature, the offending attacker is blinded (Fortitude save negates, as blindness/deafness, but against holy aura's save DC).

\shortability{Focus:}{A tiny reliquary containing some sacred relic. The reliquary costs at least 500 gp.}

\featname{Holy Smite}
\shortability{Evocation}{[Good]}
\shortability{Level:}{Good 4}
\shortability{Components:}{V, S}
\shortability{Casting Time:}{1 standard action}
\shortability{Range:}{Medium (100 ft. + 10 ft./level)}
\shortability{Area:}{20-ft.-radius burst}
\shortability{Duration:}{Instantaneous (1 round); see text}
\shortability{Saving Throw:}{Will partial; see text}
\shortability{Spell Resistance:}{Yes}
\smallskip\noindent You draw down holy power to smite your enemies. Only evil and neutral creatures are harmed by the spell; good creatures are unaffected.

\smallskip\noindent The spell deals 1d8 points of damage per two caster levels (maximum 5d8) to each evil creature in the area (or 1d6 points of damage per caster level, maximum 10d6, to an evil outsider) and causes it to become blinded for 1 round. A successful Will saving throw reduces damage to half and negates the blinded effect.

\smallskip\noindent The spell deals only half damage to creatures who are neither good nor evil, and they are not blinded. Such a creature can reduce that damage by half (down to one-quarter of the roll) with a successful Will save.

\featname{Holy Sword}
\shortability{Evocation}{[Good]}
\shortability{Level:}{Pal 4}
\shortability{Components:}{V, S}
\shortability{Casting Time:}{1 standard action}
\shortability{Range:}{Touch}
\shortability{Target:}{Melee weapon touched}
\shortability{Duration:}{1 round/level}
\shortability{Saving Throw:}{None}
\shortability{Spell Resistance:}{No}
\smallskip\noindent This spell allows you to channel holy power into your sword, or any other melee weapon you choose. The weapon acts as a +5 holy weapon (+5 enhancement bonus on attack and damage rolls, extra 2d6 damage against evil opponents). It also emits a magic circle against evil effect (as the spell). If the magic circle ends, the sword creates a new one on your turn as a free action. The spell is automatically canceled 1 round after the weapon leaves your hand. You cannot have more than one holy sword at a time.

\smallskip\noindent If this spell is cast on a magic weapon, the powers of the spell supersede any that the weapon normally has, rendering the normal enhancement bonus and powers of the weapon inoperative for the duration of the spell. This spell is not cumulative with bless weapon or any other spell that might modify the weapon in any way.

\smallskip\noindent This spell does not work on artifacts.

\shortability{Note:}{A masterwork weapon's bonus to attack does not stack with an enhancement bonus to attack.}

\featname{Holy Word}
\shortability{Evocation}{[Good, Sonic]}
\shortability{Level:}{Clr 7, Good 7}
\shortability{Components:}{V}
\shortability{Casting Time:}{1 standard action}
\shortability{Range:}{40 ft.}
\shortability{Area:}{Nongood creatures in a 40-ft.-radius spread centered on you}
\shortability{Duration:}{Instantaneous}
\shortability{Saving Throw:}{None or Will negates; see text}
\shortability{Spell Resistance:}{Yes}
\smallskip\noindent Any nongood creature within the area that hears the holy word suffers the following ill effects.

\begin{tabular}[h!]{l|l}
HD & Effect \\ \hline
Equal to caster level & Deafened \\
Up to caster level \textendash1 & Blinded, deafened \\
Up to caster level \textendash5 & Paralyzed, blinded, deafened	\\
Up to caster level \textendash10 & Killed, paralyzed, blinded, deafened \\

\smallskip\noindent The effects are cumulative and concurrent. No saving throw is allowed against these effects.

\shortability{Deafened:}{The creature is deafened for 1d4 rounds.}
\shortability{Blinded:}{The creature is blinded for 2d4 rounds.}
\shortability{Paralyzed:}{The creature is paralyzed and helpless for 1d10 minutes.}
\shortability{Killed:}{Living creatures die. Undead creatures are destroyed.}

\smallskip\noindent Furthermore, if you are on your home plane when you cast this spell, nongood extraplanar creatures within the area are instantly banished back to their home planes. Creatures so banished cannot return for at least 24 hours. This effect takes place regardless of whether the creatures hear the holy word. The banishment effect allows a Will save (at a -4 penalty) to negate.

\smallskip\noindent Creatures whose HD exceed your caster level are unaffected by holy word.

\featname{Horrid Wilting}
\shortability{Necromancy}{}
\shortability{Level:}{Sor/Wiz 8, Water 8}
\shortability{Components:}{V, S, M/DF}
\shortability{Casting Time:}{1 standard action}
\shortability{Range:}{Long (400 ft. + 40 ft./level)}
\shortability{Targets:}{Living creatures, no two of which can be more than 60 ft. apart}
\shortability{Duration:}{Instantaneous}
\shortability{Saving Throw:}{Fortitude half}
\shortability{Spell Resistance:}{Yes}
\smallskip\noindent This spell evaporates moisture from the body of each subject living creature, dealing 1d6 points of damage per caster level (maximum 20d6). This spell is especially devastating to water elementals and plant creatures, which instead take 1d8 points of damage per caster level (maximum 20d8).

\shortability{Arcane Material Component:}{A bit of sponge.}

\featname{Hypnotic Pattern}
\shortability{Illusion}{(Pattern) [Mind-Affecting]}
\shortability{Level:}{Brd 2, Sor/Wiz 2}
\shortability{Components:}{V (Brd only), S, M; see text}
\shortability{Casting Time:}{1 standard action}
\shortability{Range:}{Medium (100 ft. + 10 ft./level)}
\shortability{Effect:}{Colorful lights in a 10-ft.-radius spread}
\shortability{Duration:}{Concentration + 2 rounds}
\shortability{Saving Throw:}{Will negates}
\shortability{Spell Resistance:}{Yes}
\smallskip\noindent A twisting pattern of subtle, shifting colors weaves through the air, fascinating creatures within it. Roll 2d4 and add your caster level (maximum 10) to determine the total number of Hit Dice of creatures affected. Creatures with the fewest HD are affected first; and, among creatures with equal HD, those who are closest to the spell's point of origin are affected first. Hit Dice that are not sufficient to affect a creature are wasted. Affected creatures become fascinated by the pattern of colors. Sightless creatures are not affected.

\smallskip\noindent A wizard or sorcerer need not utter a sound to cast this spell, but a bard must sing, play music, or recite a rhyme as a verbal component.

\shortability{Material Component:}{A glowing stick of incense or a crystal rod filled with phosphorescent material.}

\featname{Hypnotism}
\shortability{Enchantment}{(Compulsion) [Mind-Affecting]}
\shortability{Level:}{Brd 1, Sor/Wiz 1}
\shortability{Components:}{V, S}
\shortability{Casting Time:}{1 round}
\shortability{Range:}{Close (25 ft. + 5 ft./2 levels)}
\shortability{Area:}{Several living creatures, no two of which may be more than 30 ft. apart}
\shortability{Duration:}{2d4 rounds (D)}
\shortability{Saving Throw:}{Will negates}
\shortability{Spell Resistance:}{Yes}
\smallskip\noindent Your gestures and droning incantation fascinate nearby creatures, causing them to stop and stare blankly at you. In addition, you can use their rapt attention to make your suggestions and requests seem more plausible. Roll 2d4 to see how many total Hit Dice of creatures you affect. Creatures with fewer HD are affected before creatures with more HD. Only creatures that can see or hear you are affected, but they do not need to understand you to be fascinated.

\smallskip\noindent If you use this spell in combat, each target gains a +2 bonus on its saving throw. If the spell affects only a single creature not in combat at the time, the saving throw has a penalty of -2.

\smallskip\noindent While the subject is fascinated by this spell, it reacts as though it were two steps more friendly in attitude. This allows you to make a single request of the affected creature (provided you can communicate with it). The request must be brief and reasonable. Even after the spell ends, the creature retains its new attitude toward you, but only with respect to that particular request.

\smallskip\noindent A creature that fails its saving throw does not remember that you enspelled it.

\featname{Ice Storm}
\shortability{Evocation}{[Cold]}
\shortability{Level:}{Drd 4, Sor/Wiz 4, Water 5}
\shortability{Components:}{V, S, M/DF}
\shortability{Casting Time:}{1 standard action}
\shortability{Range:}{Long (400 ft. + 40 ft./level)}
\shortability{Area:}{Cylinder (20-ft. radius, 40 ft. high)}
\shortability{Duration:}{1 full round}
\shortability{Saving Throw:}{None}
\shortability{Spell Resistance:}{Yes}
\smallskip\noindent Great magical hailstones pound down for 1 full round, dealing 3d6 points of bludgeoning damage and 2d6 points of cold damage to every creature in the area. A -4 penalty applies to each Listen check made within the ice storm's effect, and all land movement within its area is at half speed. At the end of the duration, the hail disappears, leaving no aftereffects (other than the damage dealt).

\shortability{Arcane Material Component:}{A pinch of dust and a few drops of water.}

\featname{Identify}
\shortability{Divination}{}
\shortability{Level:}{Brd 1, Magic 2, Sor/Wiz 1}
\shortability{Components:}{V, S, M/DF}
\shortability{Casting Time:}{1 hour}
\shortability{Range:}{Touch}
\shortability{Targets:}{One touched object}
\shortability{Duration:}{Instantaneous}
\shortability{Saving Throw:}{None}
\shortability{Spell Resistance:}{No}
\smallskip\noindent The spell determines all magic properties of a single magic item, including how to activate those functions (if appropriate), and how many charges are left (if any).

\smallskip\noindent Identify does not function when used on an artifact.

\shortability{Arcane Material Component:}{A pearl of at least 100 gp value, crushed and stirred into wine with an owl feather; the infusion must be drunk prior to spellcasting.}

\featname{Illusory Script}
\shortability{Illusion}{(Phantasm) [Mind-Affecting]}
\shortability{Level:}{Brd 3, Sor/Wiz 3}
\shortability{Components:}{V, S, M}
\shortability{Casting Time:}{1 minute or longer; see text}
\shortability{Range:}{Touch}
\shortability{Target:}{One touched object weighing no more than 10 lb.}
\shortability{Duration:}{One day/level (D)}
\shortability{Saving Throw:}{Will negates; see text}
\shortability{Spell Resistance:}{Yes}
\smallskip\noindent You write instructions or other information on parchment, paper, or any suitable writing material. The illusory script appears to be some form of foreign or magical writing. Only the person (or people) designated by you at the time of the casting are able to read the writing; it's unintelligible to any other character, although an illusionist recognizes it as illusory script.

\smallskip\noindent Any unauthorized creature attempting to read the script triggers a potent illusory effect and must make a saving throw. A successful saving throw means the creature can look away with only a mild sense of disorientation. Failure means the creature is subject to a suggestion implanted in the script by you at the time the illusory script spell was cast. The suggestion lasts only 30 minutes. Typical suggestions include "Close the book and leave," "Forget the existence of the book," and so forth. If successfully dispelled by dispel magic, the illusory script and its secret message disappear. The hidden message can be read by a combination of the true seeing spell with the read magic or comprehend languages spell.

\smallskip\noindent The casting time depends on how long a message you wish to write, but it is always at least 1 minute.

\shortability{Material Component:}{A lead-based ink (cost of not less than 50 gp).}

\featname{Illusory Wall}
\shortability{Illusion}{(Figment)}
\shortability{Level:}{Sor/Wiz 4}
\shortability{Components:}{V, S}
\shortability{Casting Time:}{1 standard action}
\shortability{Range:}{Close (25 ft. + 5 ft./2 levels)}
\shortability{Effect:}{Image 1 ft. by 10 ft. by 10 ft.}
\shortability{Duration:}{Permanent}
\shortability{Saving Throw:}{Will disbelief (if interacted with)}
\shortability{Spell Resistance:}{No}
\smallskip\noindent This spell creates the illusion of a wall, floor, ceiling, or similar surface. It appears absolutely real when viewed, but physical objects can pass through it without difficulty. When the spell is used to hide pits, traps, or normal doors, any detection abilities that do not require sight work normally. Touch or a probing search reveals the true nature of the surface, though such measures do not cause the illusion to disappear.

\featname{Imbue with Spell Ability}
\shortability{Evocation}{}
\shortability{Level:}{Clr 4, Magic 4}
\shortability{Components:}{V, S, DF}
\shortability{Casting Time:}{10 minutes}
\shortability{Range:}{Touch}
\shortability{Target:}{Creature touched; see text}
\shortability{Duration:}{Permanent until discharged (D)}
\shortability{Saving Throw:}{Will negates (harmless)}
\shortability{Spell Resistance:}{Yes (harmless)}
\smallskip\noindent You transfer some of your currently prepared spells, and the ability to cast them, to another creature. Only a creature with an Intelligence score of at least 5 and a Wisdom score of at least 9 can receive this bestowal. Only cleric spells from the schools of abjuration, divination, and conjuration (healing) can be transferred. The number and level of spells that the subject can be granted depends on its Hit Dice; even multiple castings of imbue with spell ability can't exceed this limit.

\begin{tabular}[h!]{l|l}
HD of Recipient & Spells Imbued \\ \hline
2 or lower & One 1st-level spell \\
3-4 & One or two 1st-level spells	\\
5 or higher & One or two 1st-level spells and one 2nd-level spell	 \\
\end{tabular}

\smallskip\noindent The transferred spell's variable characteristics (range, duration, area, and the like) function according to your level, not the level of the recipient.

\smallskip\noindent Once you cast imbue with spell ability, you cannot prepare a new 4th-level spell to replace it until the recipient uses the imbued spells or is slain, or until you dismiss the imbue with spell ability spell. In the meantime, you remain responsible to your deity or your principles for the use to which the spell is put. If the number of 4th-level spells you can cast decreases, and that number drops below your current number of active imbue with spell ability spells, the more recently cast imbued spells are dispelled.

\smallskip\noindent To cast a spell with a verbal component, the subject must be able to speak. To cast a spell with a somatic component, it must have humanlike hands. To cast a spell with a material component or focus, it must have the materials or focus.

\featname{Implosion}
\shortability{Evocation}{}
\shortability{Level:}{Clr 9, Destruction 9}
\shortability{Components:}{V, S}
\shortability{Casting Time:}{1 standard action}
\shortability{Range:}{Close (25 ft. + 5 ft./2 levels)}
\shortability{Targets:}{One corporeal creature/round}
\shortability{Duration:}{Concentration (up to 4 rounds)}
\shortability{Saving Throw:}{Fortitude negates}
\shortability{Spell Resistance:}{Yes}
\smallskip\noindent You create a destructive resonance in a corporeal creature's body. For each round you concentrate, you cause one creature to collapse in on itself, killing it. (This effect, being instantaneous, cannot be dispelled.)

\smallskip\noindent You can target a particular creature only once with each casting of the spell.

\smallskip\noindent Implosion has no effect on creatures in gaseous form or on incorporeal creatures.

\featname{Imprisonment}
\shortability{Abjuration}{}
\shortability{Level:}{Sor/Wiz 9}
\shortability{Components:}{V, S}
\shortability{Casting Time:}{1 standard action}
\shortability{Range:}{Touch}
\shortability{Target:}{Creature touched}
\shortability{Duration:}{Instantaneous}
\shortability{Saving Throw:}{Will negates; see text}
\shortability{Spell Resistance:}{Yes}
\smallskip\noindent When you cast imprisonment and touch a creature, it is entombed in a state of suspended animation (see the temporal stasis spell) in a small sphere far beneath the surface of the earth. The subject remains there unless a freedom spell is cast at the locale where the imprisonment took place. Magical search by a crystal ball, a locate object spell, or some other similar divination does not reveal the fact that a creature is imprisoned, but discern location does. A wish or miracle spell will not free the recipient, but will reveal where it is entombed. If you know the target's name and some facts about its life, the target takes a -4 penalty on its save.

\featname{Incendiary Cloud}
\shortability{Conjuration}{(Creation) [Fire]}
\shortability{Level:}{Fire 8, Sor/Wiz 8}
\shortability{Components:}{V, S}
\shortability{Casting Time:}{1 standard action}
\shortability{Range:}{Medium (100 ft. + 10 ft./level)}
\shortability{Effect:}{Cloud spreads in 20-ft. radius, 20 ft. high}
\shortability{Duration:}{1 round/level}
\shortability{Saving Throw:}{Reflex half; see text}
\shortability{Spell Resistance:}{No}
\smallskip\noindent An incendiary cloud spell creates a cloud of roiling smoke shot through with white-hot embers. The smoke obscures all sight as a fog cloud does. In addition, the white-hot embers within the cloud deal 4d6 points of fire damage to everything within the cloud on your turn each round. All targets can make Reflex saves each round to take half damage.

\smallskip\noindent As with a cloudkill spell, the smoke moves away from you at 10 feet per round. Figure out the smoke's new spread each round based on its new point of origin, which is 10 feet farther away from where you were when you cast the spell. By concentrating, you can make the cloud (actually its point of origin) move as much as 60 feet each round. Any portion of the cloud that would extend beyond your maximum range dissipates harmlessly, reducing the remainder's spread thereafter.

\smallskip\noindent As with fog cloud, wind disperses the smoke, and the spell can't be cast underwater.

\featname{Inflict Critical Wounds}
\shortability{Necromancy}{}
\shortability{Level:}{Clr 4, Destruction 4}
\smallskip\noindent This spell functions like inflict light wounds, except that you deal 4d8 points of damage +1 point per caster level (maximum +20).

\featname{Inflict Critical Wounds, Mass}
\shortability{Necromancy}{}
\shortability{Level:}{Clr 8}
\smallskip\noindent This spell functions like mass inflict light wounds, except that it deals 4d8 points of damage +1 point per caster level (maximum +40).

\featname{Inflict Light Wounds}
\shortability{Necromancy}{}
\shortability{Level:}{Clr 1, Destruction 1}
\shortability{Components:}{V, S}
\shortability{Casting Time:}{1 standard action}
\shortability{Range:}{Touch}
\shortability{Target:}{Creature touched}
\shortability{Duration:}{Instantaneous}
\shortability{Saving Throw:}{Will half}
\shortability{Spell Resistance:}{Yes}
\smallskip\noindent When laying your hand upon a creature, you channel negative energy that deals 1d8 points of damage +1 point per caster level (maximum +5).

\smallskip\noindent Since undead are powered by negative energy, this spell cures such a creature of a like amount of damage, rather than harming it.

\featname{Inflict Light Wounds, Mass}
\shortability{Necromancy}{}
\shortability{Level:}{Clr 5, Destruction 5}
\shortability{Components:}{V, S}
\shortability{Casting Time:}{1 standard action}
\shortability{Range:}{Close (25 ft. + 5 ft./2 levels)}
\shortability{Target:}{One creature/level, no two of which can be more than 30 ft. apart}
\shortability{Duration:}{Instantaneous}
\shortability{Saving Throw:}{Will half}
\shortability{Spell Resistance:}{Yes}
\smallskip\noindent Negative energy spreads out in all directions from the point of origin, dealing 1d8 points of damage +1 point per caster level (maximum +25) to nearby living enemies.

\smallskip\noindent Like other inflict spells, mass inflict light wounds cures undead in its area rather than damaging them. A cleric capable of spontaneously casting inflict spells can also spontaneously cast mass inflict spells.

\featname{Inflict Minor Wounds}
\shortability{Necromancy}{}
\shortability{Level:}{Clr 0}
\shortability{Saving Throw:}{Will negates}
\smallskip\noindent This spell functions like inflict light wounds, except that you deal 1 point of damage and a Will save negates the damage instead of halving it.

\featname{Inflict Moderate Wounds}
\shortability{Necromancy}{}
\shortability{Level:}{Clr 2}
\smallskip\noindent This spell functions like inflict light wounds, except that you deal 2d8 points of damage +1 point per caster level (maximum +10).

\featname{Inflict Moderate Wounds, Mass}
\shortability{Necromancy}{}
\shortability{Level:}{Clr 6}
\smallskip\noindent This spell functions like mass inflict light wounds, except that it deals 2d8 points of damage +1 point per caster level (maximum +30).

\featname{Inflict Serious Wounds}
\shortability{Necromancy}{}
\shortability{Level:}{Clr 3}
\smallskip\noindent This spell functions like inflict light wounds, except that you deal 3d8 points of damage +1 point per caster level (maximum +15).

\featname{Inflict Serious Wounds, Mass}
\shortability{Necromancy}{}
\shortability{Level:}{Clr 7}
\smallskip\noindent This spell functions like mass inflict light wounds, except that it deals 3d8 points of damage +1 point per caster level (maximum +35).

\featname{Insanity}
\shortability{Enchantment}{(Compulsion) [Mind-Affecting]}
\shortability{Level:}{Sor/Wiz 7}
\shortability{Components:}{V, S}
\shortability{Casting Time:}{1 standard action}
\shortability{Range:}{Medium (100 ft. + 10 ft./level)}
\shortability{Target:}{One living creature}
\shortability{Duration:}{Instantaneous}
\shortability{Saving Throw:}{Will negates}
\shortability{Spell Resistance:}{Yes}
\smallskip\noindent The affected creature suffers from a continuous confusion effect, as the spell.

\smallskip\noindent Remove curse does not remove insanity. Greater restoration, heal, limited wish, miracle, or wish can restore the creature.

\featname{Insect Plague}
\shortability{Conjuration}{(Summoning)}
\shortability{Level:}{Clr 5, Drd 5}
\shortability{Components:}{V, S, DF}
\shortability{Casting Time:}{1 round}
\shortability{Range:}{Long (400 ft. + 40 ft./level)}
\shortability{Effect:}{One swarm of locusts per three levels, each of which must be adjacent to at least one other swarm}
\shortability{Duration:}{1 min./level}
\shortability{Saving Throw:}{None}
\shortability{Spell Resistance:}{No}
\smallskip\noindent You summon a number of swarms of locusts (one per three levels, to a maximum of six swarms at 18th level). The swarms must be summoned so that each one is adjacent to at least one other swarm (that is, the swarms must fill one contiguous area). You may summon the locust swarms so that they share the area of other creatures. Each swarm attacks any creatures occupying its area. The swarms are stationary after being summoned, and won't pursue creatures that flee.

\featname{Instant Summons}
\shortability{Conjuration}{(Summoning)}
\shortability{Level:}{Sor/Wiz 7}
\shortability{Components:}{V, S, M}
\shortability{Casting Time:}{1 standard action}
\shortability{Range:}{See text}
\shortability{Target:}{One object weighing 10 lb. or less whose longest dimension is 6 ft. or less}
\shortability{Duration:}{Permanent until discharged}
\shortability{Saving Throw:}{None}
\shortability{Spell Resistance:}{No}
\smallskip\noindent You call some nonliving item from virtually any location directly to your hand.

\smallskip\noindent First, you must place your arcane mark on the item. Then you cast this spell, which magically and invisibly inscribes the name of the item on a sapphire worth at least 1,000 gp. Thereafter, you can summon the item by speaking a special word (set by you when the spell is cast) and crushing the gem. The item appears instantly in your hand. Only you can use the gem in this way.

\smallskip\noindent If the item is in the possession of another creature, the spell does not work, but you know who the possessor is and roughly where that creature is located when the summons occurs.

\smallskip\noindent The inscription on the gem is invisible. It is also unreadable, except by means of a read magic spell, to anyone but you.

\smallskip\noindent The item can be summoned from another plane, but only if no other creature has claimed ownership of it.

\shortability{Material Component:}{A sapphire worth at least 1,000 gp.}

\featname{Interposing Hand}
\shortability{Evocation}{[Force]}
\shortability{Level:}{Sor/Wiz 5}
\shortability{Components:}{V, S, F}
\shortability{Casting Time:}{1 standard action}
\shortability{Range:}{Medium (100 ft. + 10 ft./level)}
\shortability{Effect:}{10-ft. hand}
\shortability{Duration:}{1 round/level (D)}
\shortability{Saving Throw:}{None}
\shortability{Spell Resistance:}{Yes}
\smallskip\noindent Interposing hand creates a Large magic hand that appears between you and one opponent. This floating, disembodied hand then moves to remain between the two of you, regardless of where you move or how the opponent tries to get around it, providing cover (+4 AC) for you against that opponent. Nothing can fool the hand-it sticks with the selected opponent in spite of darkness, invisibility, polymorphing, or any other attempt at hiding or disguise. The hand does not pursue an opponent, however.

\smallskip\noindent An interposing hand is 10 feet long and about that wide with its fingers outstretched. It has as many hit points as you do when you're undamaged, and its AC is 20 (-1 size, +11 natural). It takes damage as a normal creature, but most magical effects that don't cause damage do not affect it.
The hand never provokes attacks of opportunity from opponents. It cannot push through a wall of force or enter an antimagic field, but it suffers the full effect of a prismatic wall or prismatic sphere. The hand makes saving throws as its caster.

\smallskip\noindent Disintegrate or a successful dispel magic destroys it.

\smallskip\noindent Any creature weighing 2,000 pounds or less that tries to push past the hand is slowed to half its normal speed. The hand cannot reduce the speed of a creature weighing more than 2,000 pounds, but it still affects the creature's attacks.

\smallskip\noindent Directing the spell to a new target is a move action.

\shortability{Focus:}{A soft glove.}

\featname{Invisibility}
\shortability{Illusion}{(Glamer)}
\shortability{Level:}{Brd 2, Sor/Wiz 2, Trickery 2}
\shortability{Components:}{V, S, M/DF}
\shortability{Casting Time:}{1 standard action}
\shortability{Range:}{Personal or touch}
\shortability{Target:}{You or a creature or object weighing no more than 100 lb./level}
\shortability{Duration:}{1 min./level (D)}
\shortability{Saving Throw:}{Will negates (harmless) or Will negates (harmless, object)}
\shortability{Spell Resistance:}{Yes (harmless) or Yes (harmless, object)}
\smallskip\noindent The creature or object touched becomes invisible, vanishing from sight, even from darkvision. If the recipient is a creature carrying gear, that vanishes, too. If you cast the spell on someone else, neither you nor your allies can see the subject, unless you can normally see invisible things or you employ magic to do so.

\smallskip\noindent Items dropped or put down by an invisible creature become visible; items picked up disappear if tucked into the clothing or pouches worn by the creature. Light, however, never becomes invisible, although a source of light can become so (thus, the effect is that of a light with no visible source). Any part of an item that the subject carries but that extends more than 10 feet from it becomes visible.

\smallskip\noindent Of course, the subject is not magically silenced, and certain other conditions can render the recipient detectable (such as stepping in a puddle). The spell ends if the subject attacks any creature. For purposes of this spell, an attack includes any spell targeting a foe or whose area or effect includes a foe. (Exactly who is a foe depends on the invisible character's perceptions.) Actions directed at unattended objects do not break the spell. Causing harm indirectly is not an attack. Thus, an invisible being can open doors, talk, eat, climb stairs, summon monsters and have them attack, cut the ropes holding a rope bridge while enemies are on the bridge, remotely trigger traps, open a portcullis to release attack dogs, and so forth. If the subject attacks directly, however, it immediately becomes visible along with all its gear. Spells such as bless that specifically affect allies but not foes are not attacks for this purpose, even when they include foes in their area.

\smallskip\noindent Invisibility can be made permanent (on objects only) with a permanency spell.

\shortability{Arcane Material Component:}{An eyelash encased in a bit of gum arabic.}

\featname{Invisibility, Greater}
\shortability{Illusion}{(Glamer)}
\shortability{Level:}{Brd 4, Sor/Wiz 4}
\shortability{Components:}{V, S}
\shortability{Target:}{You or creature touched}
\shortability{Duration:}{1 round/level (D)}
\shortability{Saving Throw:}{Will negates (harmless)}
\smallskip\noindent This spell functions like invisibility, except that it doesn't end if the subject attacks.

\featname{Invisibility, Mass}
\shortability{Illusion}{(Glamer)}
\shortability{Level:}{Sor/Wiz 7}
\shortability{Components:}{V, S, M}
\shortability{Range:}{Long (400 ft. + 40 ft./level)}
\shortability{Targets:}{Any number of creatures, no two of which can be more than 180 ft. apart}
\smallskip\noindent This spell functions like invisibility, except that the effect is mobile with the group and is broken when anyone in the group attacks. Individuals in the group cannot see each other. The spell is broken for any individual who moves more than 180 feet from the nearest member of the group. (If only two individuals are affected, the one moving away from the other one loses its invisibility. If both are moving away from each other, they both become visible when the distance between them exceeds 180 feet.)

\shortability{Material Component:}{An eyelash encased in a bit of gum arabic.}

\featname{Invisibility Purge}
\shortability{Evocatio}{}
\shortability{Level:}{Clr 3}
\shortability{Components:}{V, S}
\shortability{Casting Time:}{1 standard action}
\shortability{Range:}{Personal}
\shortability{Target:}{You}
\shortability{Duration:}{1 min./level (D)}
\smallskip\noindent You surround yourself with a sphere of power with a radius of 5 feet per caster level that negates all forms of invisibility.

\smallskip\noindent Anything invisible becomes visible while in the area.

\featname{Invisibility Sphere}
\shortability{Illusion}{(Glamer)}
\shortability{Level:}{Brd 3, Sor/Wiz 3}
\shortability{Components:}{V, S, M}
\shortability{Area:}{10-ft.-radius emanation around the creature or object touched}
\smallskip\noindent This spell functions like invisibility, except that this spell confers invisibility upon all creatures within 10 feet of the recipient. The center of the effect is mobile with the recipient.

\smallskip\noindent Those affected by this spell can see each other and themselves as if unaffected by the spell. Any affected creature moving out of the area becomes visible, but creatures moving into the area after the spell is cast do not become invisible. Affected creatures (other than the recipient) who attack negate the invisibility only for themselves. If the spell recipient attacks, the invisibility sphere ends.

\featname{Iron Body}
\shortability{Transmutation}{}
\shortability{Level:}{Earth 8, Sor/Wiz 8}
\shortability{Components:}{V, S, M/DF}
\shortability{Casting Time:}{1 standard action}
\shortability{Range:}{Personal}
\shortability{Target:}{You}
\shortability{Duration:}{1 min./level (D)}
\smallskip\noindent This spell transforms your body into living iron, which grants you several powerful resistances and abilities.

\smallskip\noindent You gain damage reduction 15/adamantine. You are immune to blindness, critical hits, ability score damage, deafness, disease, drowning, electricity, poison, stunning, and all spells or attacks that affect your physiology or respiration, because you have no physiology or respiration while this spell is in effect. You take only half damage from acid and fire of all kinds. However, you also become vulnerable to all special attacks that affect iron golems.

\smallskip\noindent You gain a +6 enhancement bonus to your Strength score, but you take a -6 penalty to Dexterity as well (to a minimum Dexterity score of 1), and your speed is reduced to half normal. You have an arcane spell failure chance of 50\% and a -8 armor check penalty, just as if you were clad in full plate armor. You cannot drink (and thus can't use potions) or play wind instruments.

\smallskip\noindent Your unarmed attacks deal damage equal to a club sized for you (1d4 for Small characters or 1d6 for Medium characters), and you are considered armed when making unarmed attacks.

\smallskip\noindent Your weight increases by a factor of ten, causing you to sink in water like a stone. However, you could survive the crushing pressure and lack of air at the bottom of the ocean-at least until the spell duration expires.

\shortability{Arcane Material Component:}{A small piece of iron that was once part of either an iron golem, a hero's armor, or a war machine.}

\featname{Ironwood}
\shortability{Transmutation}{}
\shortability{Level:}{Drd 6}
\shortability{Components:}{V, S, M}
\shortability{Casting Time:}{1 minute/lb. created}
\shortability{Range:}{0 ft.}
\shortability{Effect:}{An ironwood object weighing up to 5 lb./level}
\shortability{Duration:}{One day/level (D)}
\shortability{Saving Throw:}{None}
\shortability{Spell Resistance:}{No}
\smallskip\noindent Ironwood is a magical substance created by druids from normal wood. While remaining natural wood in almost every way, ironwood is as strong, heavy, and resistant to fire as steel. Spells that affect metal or iron do not function on ironwood. Spells that affect wood do affect ironwood, although ironwood does not burn. Using this spell with wood shape or a wood-related Craft check, you can fashion wooden items that function as steel items. Thus, wooden plate armor and wooden swords can be created that are as durable as their normal steel counterparts. These items are freely usable by druids.

\smallskip\noindent Further, if you make only half as much ironwood as the spell would normally allow, any weapon, shield, or suit of armor so created is treated as a magic item with a +1 enhancement bonus.

\shortability{Material Component:}{Wood shaped into the form of the intended ironwood object.}

\featname{Irresistible Dance}
\shortability{Enchantment}{(Compulsion) [Mind-Affecting]}
\shortability{Level:}{Brd 6, Sor/Wiz 8}
\shortability{Components:}{V}
\shortability{Casting Time:}{1 standard action}
\shortability{Range:}{Touch}
\shortability{Target:}{Living creature touched}
\shortability{Duration:}{1d4+1 rounds}
\shortability{Saving Throw:}{None}
\shortability{Spell Resistance:}{Yes}
\smallskip\noindent The subject feels an undeniable urge to dance and begins doing so, complete with foot shuffling and tapping. The spell effect makes it impossible for the subject to do anything other than caper and prance in place. The effect imposes a -4 penalty to Armor Class and a -10 penalty on Reflex saves, and it negates any AC bonus granted by a shield the target holds. The dancing subject provokes attacks of opportunity each round on its turn.

\featname{Jump}
\shortability{Transmutation}{}
\shortability{Level:}{Drd 1, Rgr 1, Sor/Wiz 1}
\shortability{Components:}{V, S, M}
\shortability{Casting Time:}{1 standard action}
\shortability{Range:}{Touch}
\shortability{Target:}{Creature touched}
\shortability{Duration:}{1 min./level (D)}
\shortability{Saving Throw:}{Will negates (harmless)}
\shortability{Spell Resistance:}{Yes}
\smallskip\noindent The subject gets a +10 enhancement bonus on Jump checks. The enhancement bonus increases to +20 at caster level 5th, and to +30 (the maximum) at caster level 9th.

\shortability{Material Component:}{A grasshopper's hind leg, which you break when the spell is cast.}

\featname{Keen Edge}
\shortability{Transmutation}{}
\shortability{Level:}{Sor/Wiz 3}
\shortability{Components:}{V, S}
\shortability{Casting Time:}{1 standard action}
\shortability{Range:}{Close (25 ft. + 5 ft./2 levels)}
\shortability{Targets:}{One weapon or fifty projectiles, all of which must be in contact with each other at the time of casting}
\shortability{Duration:}{10 min./level}
\shortability{Saving Throw:}{Will negates (harmless, object)}
\shortability{Spell Resistance:}{Yes (harmless, object)}
\smallskip\noindent This spell makes a weapon magically keen, improving its ability to deal telling blows. This transmutation doubles the threat range of the weapon. A threat range of 20 becomes 19-20, a threat range of 19-20 becomes 17-20, and a threat range of 18-20 becomes 15-20. The spell can be cast only on piercing or slashing weapons. If cast on arrows or crossbow bolts, the keen edge on a particular projectile ends after one use, whether or not the missile strikes its intended target. (Treat shuriken as arrows, rather than as thrown weapons, for the purpose of this spell.)

\smallskip\noindent Multiple effects that increase a weapon's threat range (such as the keen edge spell and the Improved Critical feat) don't stack. You can't cast this spell on a natural weapon, such as a claw.

\featname{Knock}
\shortability{Transmutation}{}
\shortability{Level:}{Sor/Wiz 2}
\shortability{Components:}{V}
\shortability{Casting Time:}{1 standard action}
\shortability{Range:}{Medium (100 ft. + 10 ft./level)}
\shortability{Target:}{One door, box, or chest with an area of up to 10 sq. ft./level}
\shortability{Duration:}{Instantaneous; see text}
\shortability{Saving Throw:}{None}
\shortability{Spell Resistance:}{No}
\smallskip\noindent The knock spell opens stuck, barred, locked, held, or arcane locked doors. It opens secret doors, as well as locked or trick-opening boxes or chests. It also loosens welds, shackles, or chains (provided they serve to hold closures shut). If used to open a arcane locked door, the spell does not remove the arcane lock but simply suspends its functioning for 10 minutes. In all other cases, the door does not relock itself or become stuck again on its own. Knock does not raise barred gates or similar impediments (such as a portcullis), nor does it affect ropes, vines, and the like. The effect is limited by the area. Each spell can undo as many as two means of preventing egress. 

\featname{Know Direction}
\shortability{Divination}{}
\shortability{Level:}{Brd 0, Drd 0}
\shortability{Components:}{V, S}
\shortability{Casting Time:}{1 standard action}
\shortability{Range:}{Personal}
\shortability{Target:}{You}
\shortability{Duration:}{Instantaneous}
\smallskip\noindent You instantly know the direction of north from your current position. The spell is effective in any environment in which "north" exists, but it may not work in extraplanar settings. Your knowledge of north is correct at the moment of casting, but you can get lost again within moments if you don't find some external reference point to help you keep track of direction.

\featname{Legend Lore}
\shortability{Divination}{}
\shortability{Level:}{Brd 4, Knowledge 7, Sor/Wiz 6}
\shortability{Components:}{V, S, M, F}
\shortability{Casting Time:}{See text}
\shortability{Range:}{Personal}
\shortability{Target:}{You}
\shortability{Duration:}{See text}
\smallskip\noindent Legend lore brings to your mind legends about an important person, place, or thing. If the person or thing is at hand, or if you are in the place in question, the casting time is only 1d4x10 minutes. If you have only detailed information on the person, place, or thing, the casting time is 1d10 days, and the resulting lore is less complete and specific (though it often provides enough information to help you find the person, place, or thing, thus allowing a better legend lore result next time). If you know only rumors, the casting time is 2d6 weeks, and the resulting lore is vague and incomplete (though it often directs you to more detailed information, thus allowing a better legend lore result next time).

\smallskip\noindent During the casting, you cannot engage in other than routine activities: eating, sleeping, and so forth. When completed, the divination brings legends (if any) about the person, place, or things to your mind. These may be legends that are still current, legends that have been forgotten, or even information that has never been generally known. If the person, place, or thing is not of legendary importance, you gain no information. As a rule of thumb, characters who are 11th level and higher are ``legendary,'' as are the sorts of creatures they contend with, the major magic items they wield, and the places where they perform their key deeds.

\shortability{Material Component:}{Incense worth at least 250 gp.}
\shortability{Focus:}{Four strips of ivory (worth 50 gp each) formed into a rectangle.}

\featname{Lesser (Spell Name)}
\shortability{Any}{spell whose name begins with lesser is alphabetized in this chapter according to the second word of the spell name. Thus, the description of a lesser spell appears near the description of the spell on which it is based. Spell chains that have lesser spells in them include those based on the spells confusion, geas, globe of invulnerability, planar ally, planar binding, and restoration.}

\featname{Levitate}
\shortability{Transmutation}{}
\shortability{Level:}{Sor/Wiz 2}
\shortability{Components:}{V, S, F}
\shortability{Casting Time:}{1 standard action}
\shortability{Range:}{Personal or close (25 ft. + 5 ft./2 levels)}
\shortability{Target:}{You or one willing creature or one object (total weight up to 100 lb./level)}
\shortability{Duration:}{1 min./level (D)}
\shortability{Saving Throw:}{None}
\shortability{Spell Resistance:}{No}
\smallskip\noindent Levitate allows you to move yourself, another creature, or an object up and down as you wish. A creature must be willing to be levitated, and an object must be unattended or possessed by a willing creature. You can mentally direct the recipient to move up or down as much as 20 feet each round; doing so is a move action. You cannot move the recipient horizontally, but the recipient could clamber along the face of a cliff, for example, or push against a ceiling to move laterally (generally at half its base land speed).

\smallskip\noindent A levitating creature that attacks with a melee or ranged weapon finds itself increasingly unstable; the first attack has a -1 penalty on attack rolls, the second -2, and so on, to a maximum penalty of -5. A full round spent stabilizing allows the creature to begin again at -1.

\shortability{Focus:}{Either a small leather loop or a piece of golden wire bent into a cup shape with a long shank on one end.}

\featname{Light}
\shortability{Evocation}{[Light]}
\shortability{Level:}{Brd 0, Clr 0, Drd 0, Sor/Wiz 0}
\shortability{Components:}{V, M/DF}
\shortability{Casting Time:}{1 standard action}
\shortability{Range:}{Touch}
\shortability{Target:}{Object touched}
\shortability{Duration:}{10 min./level (D)}
\shortability{Saving Throw:}{None}
\shortability{Spell Resistance:}{No}
\smallskip\noindent This spell causes an object to glow like a torch, shedding bright light in a 20-foot radius (and dim light for an additional 20 feet) from the point you touch. The effect is immobile, but it can be cast on a movable object. Light taken into an area of magical darkness does not function.

\smallskip\noindent A light spell (one with the light descriptor) counters and dispels a darkness spell (one with the darkness descriptor) of an equal or lower level.

\shortability{Arcane Material Component:}{A firefly or a piece of phosphorescent moss.}

\featname{Lightning Bolt}
\shortability{Evocation}{[Electricity]}
\shortability{Level:}{Sor/Wiz 3}
\shortability{Components:}{V, S, M}
\shortability{Casting Time:}{1 standard action}
\shortability{Range:}{120 ft.}
\shortability{Area:}{120-ft. line}
\shortability{Duration:}{Instantaneous}
\shortability{Saving Throw:}{Reflex half}
\shortability{Spell Resistance:}{Yes}
\smallskip\noindent You release a powerful stroke of electrical energy that deals 1d6 points of electricity damage per caster level (maximum 10d6) to each creature within its area. The bolt begins at your fingertips.

\smallskip\noindent The lightning bolt sets fire to combustibles and damages objects in its path. It can melt metals with a low melting point, such as lead, gold, copper, silver, or bronze. If the damage caused to an interposing barrier shatters or breaks through it, the bolt may continue beyond the barrier if the spell's range permits; otherwise, it stops at the barrier just as any other spell effect does.

\shortability{Material Component:}{A bit of fur and an amber, crystal, or glass rod.}

\featname{Limited Wish}
\shortability{Universa}{}
\shortability{Level:}{Sor/Wiz 7}
\shortability{Components:}{V, S, XP}
\shortability{Casting Time:}{1 standard action}
\shortability{Range:}{See text}
\shortability{Target, Effect, or Area:}{See text}
\shortability{Duration:}{See text}
\shortability{Saving Throw:}{None; see text}
\shortability{Spell Resistance:}{Yes}
\smallskip\noindent A limited wish lets you create nearly any type of effect. For example, a limited wish can do any of the following things.

\listone
  \item Duplicate any sorcerer/wizard spell of 6th level or lower, provided the spell is not of a school prohibited to you.
  \item Duplicate any other spell of 5th level or lower, provided the spell is not of a school prohibited to you.
  \item Duplicate any sorcerer/wizard spell of 5th level or lower, even if it's of a prohibited school.
  \item Duplicate any other spell of 4th level or lower, even if it's of a prohibited school.
  \item Undo the harmful effects of many spells, such as geas/quest or insanity.
  \item Produce any other effect whose power level is in line with the above effects, such as a single creature automatically hitting on its next attack or taking a -7 penalty on its next saving throw.
\end{list}
  
\smallskip\noindent A duplicated spell allows saving throws and spell resistance as normal (but the save DC is for a 7th-level spell). When a limited wish duplicates a spell that has an XP cost, you must pay that cost or 300 XP, whichever is more. When a limited wish spell duplicates a spell with a material component that costs more than 1,000 gp, you must provide that component.

\shortability{XP Cost:}{300 XP or more (see above).}

\featname{Liveoak}
\shortability{Transmutation}{}
\shortability{Level:}{Drd 6}
\shortability{Components:}{V, S}
\shortability{Casting Time:}{10 minutes}
\shortability{Range:}{Touch}
\shortability{Target:}{Tree touched}
\shortability{Duration:}{One day/level (D)}
\shortability{Saving Throw:}{None}
\shortability{Spell Resistance:}{No}
\smallskip\noindent This spell turns an oak tree into a protector or guardian. The spell can be cast on only a single tree at a time; while liveoak is in effect, you can't cast it again on another tree. The tree on which the spell is cast must be within 10 feet of your dwelling place, within a place sacred to you, or within 300 feet of something that you wish to guard or protect.

\smallskip\noindent Liveoak must be cast on a healthy, Huge oak. A triggering phrase of up to one word per caster level is placed on the targeted oak. The liveoak spell triggers the tree into animating as a treant. 

\smallskip\noindent If liveoak is dispelled, the tree takes root immediately, wherever it happens to be. If released by you, the tree tries to return to its original location before taking root.

\featname{Locate Creature}
\shortability{Divination}{}
\shortability{Level:}{Brd 4, Sor/Wiz 4}
\shortability{Components:}{V, S, M}
\shortability{Duration:}{10 min./level}
\smallskip\noindent This spell functions like locate object, except this spell locates a known or familiar creature.

\smallskip\noindent You slowly turn and sense when you are facing in the direction of the creature to be located, provided it is within range. You also know in which direction the creature is moving, if any.

\smallskip\noindent The spell can locate a creature of a specific kind or a specific creature known to you. It cannot find a creature of a certain type. To find a kind of creature, you must have seen such a creature up close (within 30 feet) at least once.

\smallskip\noindent Running water blocks the spell. It cannot detect objects. It can be fooled by mislead, nondetection, and polymorph spells.

\shortability{Material Component:}{A bit of fur from a bloodhound.}

\featname{Locate Object}
\shortability{Divination}{}
\shortability{Level:}{Brd 2, Clr 3, Sor/Wiz 2, Travel 2}
\shortability{Components:}{V, S, F/DF}
\shortability{Casting Time:}{1 standard action}
\shortability{Range:}{Long (400 ft. + 40 ft./level)}
\shortability{Area:}{Circle, centered on you, with a radius of 400 ft. + 40 ft./level}
\shortability{Duration:}{1 min./level}
\shortability{Saving Throw:}{None}
\shortability{Spell Resistance:}{No}
\smallskip\noindent You sense the direction of a well-known or clearly visualized object. You can search for general items, in which case you locate the nearest one of its kind if more than one is within range. Attempting to find a certain item requires a specific and accurate mental image; if the image is not close enough to the actual object, the spell fails. You cannot specify a unique item unless you have observed that particular item firsthand (not through divination).

\smallskip\noindent The spell is blocked by even a thin sheet of lead. Creatures cannot be found by this spell. Polymorph any object fools it.

\shortability{Arcane Focus:}{A forked twig.}

\featname{Longstrider}
\shortability{Transmutation}{}
\shortability{Level:}{Drd 1, Rgr 1, Travel 1}
\shortability{Components:}{V, S, M}
\shortability{Casting Time:}{1 standard action}
\shortability{Range:}{Personal}
\shortability{Target:}{You}
\shortability{Duration:}{1 hour/level (D)}
\smallskip\noindent This spell increases your base land speed by 10 feet. (This adjustment counts as an enhancement bonus.) It has no effect on other modes of movement, such as burrow, climb, fly, or swim.

\shortability{Material Component:}{A pinch of dirt.}

\featname{Lullaby}
\shortability{Enchantment}{(Compulsion) [Mind-Affecting]}
\shortability{Level:}{Brd 0}
\shortability{Components:}{V, S}
\shortability{Casting Time:}{1 standard action}
\shortability{Range:}{Medium (100 ft. + 10 ft./level)}
\shortability{Area:}{Living creatures within a 10-ft.-radius burst}
\shortability{Duration:}{Concentration + 1 round/level (D)}
\shortability{Saving Throw:}{Will negates}
\shortability{Spell Resistance:}{Yes}
\smallskip\noindent Any creature within the area that fails a Will save becomes drowsy and inattentive, taking a -5 penalty on Listen and Spot checks and a -2 penalty on Will saves against sleep effects while the lullaby is in effect. Lullaby lasts for as long as the caster concentrates, plus up to 1 round per caster level thereafter.

\featname{Mage Armor}
\shortability{Conjuration}{(Creation) [Force]}
\shortability{Level:}{Sor/Wiz 1}
\shortability{Components:}{V, S, F}
\shortability{Casting Time:}{1 standard action}
\shortability{Range:}{Touch}
\shortability{Target:}{Creature touched}
\shortability{Duration:}{1 hour/level (D)}
\shortability{Saving Throw:}{Will negates (harmless)}
\shortability{Spell Resistance:}{No}
\smallskip\noindent An invisible but tangible field of force surrounds the subject of a mage armor spell, providing a +4 armor bonus to AC.

\smallskip\noindent Unlike mundane armor, mage armor entails no armor check penalty, arcane spell failure chance, or speed reduction. Since mage armor is made of force, incorporeal creatures can't bypass it the way they do normal armor.

\shortability{Focus:}{A piece of cured leather.}

\featname{Mage Hand}
\shortability{Transmutation}{}
\shortability{Level:}{Brd 0, Sor/Wiz 0}
\shortability{Components:}{V, S}
\shortability{Casting Time:}{1 standard action}
\shortability{Range:}{Close (25 ft. + 5 ft./2 levels)}
\shortability{Target:}{One nonmagical, unattended object weighing up to 5 lb.}
\shortability{Duration:}{Concentration}
\shortability{Saving Throw:}{None}
\shortability{Spell Resistance:}{No}
\smallskip\noindent You point your finger at an object and can lift it and move it at will from a distance. As a move action, you can propel the object as far as 15 feet in any direction, though the spell ends if the distance between you and the object ever exceeds the spell's range.

\featname{Mage's Disjunction}
\shortability{Abjuration}{}
\shortability{Level:}{Magic 9, Sor/Wiz 9}
\shortability{Components:}{V}
\shortability{Casting Time:}{1 standard action}
\shortability{Range:}{Close (25 ft. + 5 ft./2 levels)}
\shortability{Area:}{All magical effects and magic items within a 40-ft.-radius burst}
\shortability{Duration:}{Instantaneous}
\shortability{Saving Throw:}{Will negates (object)}
\shortability{Spell Resistance:}{No}
\smallskip\noindent All magical effects and magic items within the radius of the spell, except for those that you carry or touch, are disjoined. That is, spells and spell-like effects are separated into their individual components (ending the effect as a dispel magic spell does), and each permanent magic item must make a successful Will save or be turned into a normal item. An item in a creature's possession uses its own Will save bonus or its possessor's Will save bonus, whichever is higher.

\smallskip\noindent You also have a 1\% chance per caster level of destroying an antimagic field. If the antimagic field survives the disjunction, no items within it are disjoined.

\smallskip\noindent Even artifacts are subject to disjunction, though there is only a 1\% chance per caster level of actually affecting such powerful items. Additionally, if an artifact is destroyed, you must make a DC 25 Will save or permanently lose all spellcasting abilities. (These abilities cannot be recovered by mortal magic, not even miracle or wish.)

\shortability{Note:}{Destroying artifacts is a dangerous business, and it is 95\% likely to attract the attention of some powerful being who has an interest in or connection with the device.}

\featname{Mage's Faithful Hound}
\shortability{Conjuration}{(Creation)}
\shortability{Level:}{Sor/Wiz 5}
\shortability{Components:}{V, S, M}
\shortability{Casting Time:}{1 standard action}
\shortability{Range:}{Close (25 ft. + 5 ft./2 levels)}
\shortability{Effect:}{Phantom watchdog}
\shortability{Duration:}{1 hour/caster level or until discharged, then 1 round/caster level; see text}
\shortability{Saving Throw:}{None}
\shortability{Spell Resistance:}{No}
\smallskip\noindent You conjure up a phantom watchdog that is invisible to everyone but yourself. It then guards the area where it was conjured (it does not move). The hound immediately starts barking loudly if any Small or larger creature approaches within 30 feet of it. (Those within 30 feet of the hound when it is conjured may move about in the area, but if they leave and return, they activate the barking.) The hound sees invisible and ethereal creatures. It does not react to figments, but it does react to shadow illusions.

\smallskip\noindent If an intruder approaches to within 5 feet of the hound, the dog stops barking and delivers a vicious bite (+10 attack bonus, 2d6+3 points of piercing damage) once per round. The dog also gets the bonuses appropriate to an invisible creature.

\smallskip\noindent The dog is considered ready to bite intruders, so it delivers its first bite on the intruder's turn. Its bite is the equivalent of a magic weapon for the purpose of damage reduction. The hound cannot be attacked, but it can be dispelled.
The spell lasts for 1 hour per caster level, but once the hound begins barking, it lasts only 1 round per caster level. If you are ever more than 100 feet distant from the hound, the spell ends.

\shortability{Material Component:}{A tiny silver whistle, a piece of bone, and a thread.}

\featname{Mage's Lucubration}
\shortability{Transmutation}{}
\shortability{Level:}{Wiz 6}
\shortability{Components:}{V, S}
\shortability{Casting Time:}{1 standard action}
\shortability{Range:}{Personal}
\shortability{Target:}{You}
\shortability{Duration:}{Instantaneous}
\smallskip\noindent You instantly recall any one spell of 5th level or lower that you have used during the past 24 hours. The spell must have been actually cast during that period. The recalled spell is stored in your mind as through prepared in the normal fashion.

\smallskip\noindent If the recalled spell requires material components, you must provide them. The recovered spell is not usable until the material components are available.

\featname{Mage's Magnificent Mansion}
\shortability{Conjuration}{(Creation)}
\shortability{Level:}{Sor/Wiz 7}
\shortability{Components:}{V, S, F}
\shortability{Casting Time:}{1 standard action}
\shortability{Range:}{Close (25 ft. + 5 ft./2 levels)}
\shortability{Effect:}{Extradimensional mansion, up to three 10-ft. cubes/level (S)}
\shortability{Duration:}{2 hours/level (D)}
\shortability{Saving Throw:}{None}
\shortability{Spell Resistance:}{No}
\smallskip\noindent You conjure up an extradimensional dwelling that has a single entrance on the plane from which the spell was cast. The entry point looks like a faint shimmering in the air that is 4 feet wide and 8 feet high. Only those you designate may enter the mansion, and the portal is shut and made invisible behind you when you enter. You may open it again from your own side at will. Once observers have passed beyond the entrance, they are in a magnificent foyer with numerous chambers beyond. The atmosphere is clean, fresh, and warm.

\smallskip\noindent You can create any floor plan you desire to the limit of the spell's effect. The place is furnished and contains sufficient foodstuffs to serve a nine-course banquet to a dozen people per caster level. A staff of near-transparent servants (as many as two per caster level), liveried and obedient, wait upon all who enter. The servants function as unseen servant spells except that they are visible and can go anywhere in the mansion.

\smallskip\noindent Since the place can be entered only through its special portal, outside conditions do not affect the mansion, nor do conditions inside it pass to the plane beyond.

\shortability{Focus:}{A miniature portal carved from ivory, a small piece of polished marble, and a tiny silver spoon (each item worth 5 gp).}

\featname{Mage's Private Sanctum}
\shortability{Abjuration}{}
\shortability{Level:}{Sor/Wiz 5}
\shortability{Components:}{V, S, M}
\shortability{Casting Time:}{10 minutes}
\shortability{Range:}{Close (25 ft. + 5 ft./2 levels)}
\shortability{Area:}{30-ft. cube/level (S)}
\shortability{Duration:}{24 hours (D)}
\shortability{Saving Throw:}{None}
\shortability{Spell Resistance:}{No}
\smallskip\noindent This spell ensures privacy. Anyone looking into the area from outside sees only a dark, foggy mass. Darkvision cannot penetrate it. No sounds, no matter how loud, can escape the area, so nobody can eavesdrop from outside. Those inside can see out normally.

\smallskip\noindent Divination (scrying) spells cannot perceive anything within the area, and those within are immune to detect thoughts. The ward prevents speech between those inside and those outside (because it blocks sound), but it does not prevent other communication, such as a sending or message spell, or telepathic communication, such as that between a wizard and her familiar.

\smallskip\noindent The spell does not prevent creatures or objects from moving into and out of the area.

\smallskip\noindent Mage's private sanctum can be made permanent with a permanency spell.

\shortability{Material Component:}{A thin sheet of lead, a piece of opaque glass, a wad of cotton or cloth, and powdered chrysolite.}

\featname{Mage's Sword}
\shortability{Evocation}{[Force]}
\shortability{Level:}{Sor/Wiz 7}
\shortability{Components:}{V, S, F}
\shortability{Casting Time:}{1 standard action}
\shortability{Range:}{Close (25 ft. + 5 ft./2 levels)}
\shortability{Effect:}{One sword}
\shortability{Duration:}{1 round/level (D)}
\shortability{Saving Throw:}{None}
\shortability{Spell Resistance:}{Yes}
\smallskip\noindent This spell brings into being a shimmering, swordlike plane of force. The sword strikes at any opponent within its range, as you desire, starting in the round that you cast the spell. The sword attacks its designated target once each round on your turn. Its attack bonus is equal to your caster level + your Int bonus or your Cha bonus (for wizards or sorcerers, respectively) with an additional +3 enhancement bonus. As a force effect, it can strike ethereal and incorporeal creatures. It deals 4d6+3 points of force damage, with a threat range of 19-20 and a critical multiplier of x2.

\smallskip\noindent The sword always strikes from your direction. It does not get a bonus for flanking or help a combatant get one. If the sword goes beyond the spell range from you, if it goes out of your sight, or if you are not directing it, the sword returns to you and hovers.

\smallskip\noindent Each round after the first, you can use a standard action to switch the sword to a new target. If you do not, the sword continues to attack the previous round's target.

\smallskip\noindent The sword cannot be attacked or harmed by physical attacks, but dispel magic, disintegrate, a sphere of annihilation, or a rod of cancellation affects it. The sword's AC is 13 (10, +0 size bonus for Medium object, +3 deflection bonus).
If an attacked creature has spell resistance, the resistance is checked the first time Mage's sword strikes it. If the sword is successfully resisted, the spell is dispelled. If not, the sword has its normal full effect on that creature for the duration of the spell.

\shortability{Focus:}{A miniature platinum sword with a grip and pommel of copper and zinc. It costs 250 gp to construct.}

\featname{Magic Aura}
\shortability{Illusion}{(Glamer)}
\shortability{Level:}{Brd 1, Magic 1, Sor/Wiz 1}
\shortability{Components:}{V, S, F}
\shortability{Casting Time:}{1 standard action}
\shortability{Range:}{Touch}
\shortability{Target:}{One touched object weighing up to 5 lb./level}
\shortability{Duration:}{One day/level (D)}
\shortability{Saving Throw:}{None; see text}
\shortability{Spell Resistance:}{No}
\smallskip\noindent You alter an item's aura so that it registers to detect spells (and spells with similar capabilities) as though it were nonmagical, or a magic item of a kind you specify, or the subject of a spell you specify. 

\smallskip\noindent If the object bearing magic aura has identify cast on it or is similarly examined, the examiner recognizes that the aura is false and detects the object's actual qualities if he succeeds on a Will save. Otherwise, he believes the aura and no amount of testing reveals what the true magic is.

\smallskip\noindent If the targeted item's own aura is exceptionally powerful (if it is an artifact, for instance), magic aura doesn't work.

\shortability{Note:}{A magic weapon, shield, or suit of armor must be a masterwork item, so a sword of average make, for example, looks suspicious if it has a magical aura.}
\shortability{Focus:}{A small square of silk that must be passed over the object that receives the aura.}

\featname{Magic Circle against Chaos}
\shortability{Abjuration}{[Lawful]}
\shortability{Level:}{Clr 3, Law 3, Pal 3, Sor/Wiz 3}
\smallskip\noindent This spell functions like magic circle against evil, except that it is similar to protection from chaos instead of protection from evil, and it can imprison a nonlawful called creature.

\featname{Magic Circle against Evil}
\shortability{Abjuration}{[Good]}
\shortability{Level:}{Clr 3, Good 3, Pal 3, Sor/Wiz 3}
\shortability{Components:}{V, S, M/DF}
\shortability{Casting Time:}{1 standard action}
\shortability{Range:}{Touch}
\shortability{Area:}{10-ft.-radius emanation from touched creature}
\shortability{Duration:}{10 min./level}
\shortability{Saving Throw:}{Will negates (harmless)}
\shortability{Spell Resistance:}{No; see text}
\smallskip\noindent All creatures within the area gain the effects of a protection from evil spell, and no nongood summoned creatures can enter the area either. You must overcome a creature's spell resistance in order to keep it at bay (as in the third function of protection from evil), but the deflection and resistance bonuses and the protection from mental control apply regardless of enemies' spell resistance.

\smallskip\noindent This spell has an alternative version that you may choose when casting it. A magic circle against evil can be focused inward rather than outward. When focused inward, the spell binds a nongood called creature (such as those called by the lesser planar binding, planar binding, and greater planar binding spells) for a maximum of 24 hours per caster level, provided that you cast the spell that calls the creature within 1 round of casting the magic circle. The creature cannot cross the circle's boundaries. If a creature too large to fit into the spell's area is the subject of the spell, the spell acts as a normal protection from evil spell for that creature only.

\smallskip\noindent A magic circle leaves much to be desired as a trap. If the circle of powdered silver laid down in the process of spellcasting is broken, the effect immediately ends. The trapped creature can do nothing that disturbs the circle, directly or indirectly, but other creatures can. If the called creature has spell resistance, it can test the trap once a day. If you fail to overcome its spell resistance, the creature breaks free, destroying the circle. A creature capable of any form of dimensional travel (astral projection, blink, dimension door, etherealness, gate, plane shift, shadow walk, teleport, and similar abilities) can simply leave the circle through that means. You can prevent the creature's extradimensional escape by casting a dimensional anchor spell on it, but you must cast the spell before the creature acts. If you are successful, the anchor effect lasts as long as the magic circle does. The creature cannot reach across the magic circle, but its ranged attacks (ranged weapons, spells, magical abilities, and the like) can. The creature can attack any target it can reach with its ranged attacks except for the circle itself.

\smallskip\noindent You can add a special diagram (a two-dimensional bounded figure with no gaps along its circumference, augmented with various magical sigils) to make the magic circle more secure. Drawing the diagram by hand takes 10 minutes and requires a DC 20 Spellcraft check. You do not know the result of this check. If the check fails, the diagram is ineffective. You can take 10 when drawing the diagram if you are under no particular time pressure to complete the task. This task also takes 10 full minutes. If time is no factor at all, and you devote 3 hours and 20 minutes to the task, you can take 20.

\smallskip\noindent A successful diagram allows you to cast a dimensional anchor spell on the magic circle during the round before casting any summoning spell. The anchor holds any called creatures in the magic circle for 24 hours per caster level. A creature cannot use its spell resistance against a magic circle prepared with a diagram, and none of its abilities or attacks can cross the diagram. If the creature tries a Charisma check to break free of the trap (see the lesser planar binding spell), the DC increases by 5. The creature is immediately released if anything disturbs the diagram-even a straw laid across it. However, the creature itself cannot disturb the diagram either directly or indirectly, as noted above.

\smallskip\noindent This spell is not cumulative with protection from evil and vice versa.

\shortability{Arcane Material Component:}{A little powdered silver with which you trace a 3-footdiameter circle on the floor (or ground) around the creature to be warded.}

\featname{Magic Circle against Good}
\shortability{Abjuration}{[Evil]}
\shortability{Level:}{Clr 3, Evil 3, Sor/Wiz 3}
\smallskip\noindent This spell functions like magic circle against evil, except that it is similar to protection from good instead of protection from evil, and it can imprison a nonevil called creature.

\featname{Magic Circle against Law}
\shortability{Abjuration}{[Chaotic]}
\shortability{Level:}{Chaos 3, Clr 3, Sor/Wiz 3}
\smallskip\noindent This spell functions like magic circle against evil, except that it is similar to protection from law instead of protection from evil, and it can imprison a nonchaotic called creature.

\featname{Magic Fang}
\shortability{Transmutation}{}
\shortability{Level:}{Drd 1, Rgr 1}
\shortability{Components:}{V, S, DF}
\shortability{Casting Time:}{1 standard action}
\shortability{Range:}{Touch}
\shortability{Target:}{Living creature touched}
\shortability{Duration:}{1 min./level}
\shortability{Saving Throw:}{Will negates (harmless)}
\shortability{Spell Resistance:}{Yes (harmless)}
\smallskip\noindent Magic fang gives one natural weapon of the subject a +1 enhancement bonus on attack and damage rolls. The spell can affect a slam attack, fist, bite, or other natural weapon. (The spell does not change an unarmed strike's damage from nonlethal damage to lethal damage.)

\smallskip\noindent Magic fang can be made permanent with a permanency spell.

\featname{Magic Fang, Greater}
\shortability{Transmutation}{}
\shortability{Level:}{Drd 3, Rgr 3}
\shortability{Range:}{Close (25 ft. + 5 ft./2 levels)}
\shortability{Target:}{One living creature}
\shortability{Duration:}{1 hour/level}
\smallskip\noindent This spell functions like magic fang, except that the enhancement bonus on attack and damage rolls is +1 per four caster levels (maximum +5).

\smallskip\noindent Alternatively, you may imbue all of the creature's natural weapons with a +1 enhancement bonus (regardless of your caster level).

\smallskip\noindent Greater magic fang can be made permanent with a permanency spell.

\featname{Magic Jar}
\shortability{Necromancy}{}
\shortability{Level:}{Sor/Wiz 5}
\shortability{Components:}{V, S, F}
\shortability{Casting Time:}{1 standard action}
\shortability{Range:}{Medium (100 ft. + 10 ft./level)}
\shortability{Target:}{One creature}
\shortability{Duration:}{1 hour/level or until you return to your body}
\shortability{Saving Throw:}{Will negates; see text}
\shortability{Spell Resistance:}{Yes}
\smallskip\noindent By casting magic jar, you place your soul in a gem or large crystal (known as the magic jar), leaving your body lifeless. Then you can attempt to take control of a nearby body, forcing its soul into the magic jar. You may move back to the jar (thereby returning the trapped soul to its body) and attempt to possess another body. The spell ends when you send your soul back to your own body, leaving the receptacle empty.

\smallskip\noindent To cast the spell, the magic jar must be within spell range and you must know where it is, though you do not need line of sight or line of effect to it. When you transfer your soul upon casting, your body is, as near as anyone can tell, dead. 

\smallskip\noindent While in the magic jar, you can sense and attack any life force within 10 feet per caster level (and on the same plane of existence). You do need line of effect from the jar to the creatures. You cannot determine the exact creature types or positions of these creatures. In a group of life forces, you can sense a difference of 4 or more Hit Dice between one creature and another and can determine whether a life force is powered by positive or negative energy. (Undead creatures are powered by negative energy. Only sentient undead creatures have, or are, souls.)

\smallskip\noindent You could choose to take over either a stronger or a weaker creature, but which particular stronger or weaker creature you attempt to possess is determined randomly.

\smallskip\noindent Attempting to possess a body is a full-round action. It is blocked by protection from evil or a similar ward. You possess the body and force the creature's soul into the magic jar unless the subject succeeds on a Will save. Failure to take over the host leaves your life force in the magic jar, and the target automatically succeeds on further saving throws if you attempt to possess its body again.

\smallskip\noindent If you are successful, your life force occupies the host body, and the host's life force is imprisoned in the magic jar. You keep your Intelligence, Wisdom, Charisma, level, class, base attack bonus, base save bonuses, alignment, and mental abilities. The body retains its Strength, Dexterity, Constitution, hit points, natural abilities, and automatic abilities. A body with extra limbs does not allow you to make more attacks (or more advantageous two-weapon attacks) than normal. You can't choose to activate the body's extraordinary or supernatural abilities. The creature's spells and spell-like abilities do not stay with the body.

\smallskip\noindent As a standard action, you can shift freely from a host to the magic jar if within range, sending the trapped soul back to its body. The spell ends when you shift from the jar to your own body.

\smallskip\noindent If the host body is slain, you return to the magic jar, if within range, and the life force of the host departs (it is dead). If the host body is slain beyond the range of the spell, both you and the host die. Any life force with nowhere to go is treated as slain.

\smallskip\noindent If the spell ends while you are in the magic jar, you return to your body (or die if your body is out of range or destroyed). If the spell ends while you are in a host, you return to your body (or die, if it is out of range of your current position), and the soul in the magic jar returns to its body (or dies if it is out of range). Destroying the receptacle ends the spell, and the spell can be dispelled at either the magic jar or at the host's location.

\shortability{Focus:}{A gem or crystal worth at least 100 gp.}

\featname{Magic Missile}
\shortability{Evocation}{[Force]}
\shortability{Level:}{Sor/Wiz 1}
\shortability{Components:}{V, S}
\shortability{Casting Time:}{1 standard action}
\shortability{Range:}{Medium (100 ft. + 10 ft./level)}
\shortability{Targets:}{Up to five creatures, no two of which can be more than 15 ft. apart}
\shortability{Duration:}{Instantaneous}
\shortability{Saving Throw:}{None}
\shortability{Spell Resistance:}{Yes}
\smallskip\noindent A missile of magical energy darts forth from your fingertip and strikes its target, dealing 1d4+1 points of force damage.

\smallskip\noindent The missile strikes unerringly, even if the target is in melee combat or has less than total cover or total concealment. Specific parts of a creature can't be singled out. Inanimate objects are not damaged by the spell.

\smallskip\noindent For every two caster levels beyond 1st, you gain an additional missile-two at 3rd level, three at 5th, four at 7th, and the maximum of five missiles at 9th level or higher. If you shoot multiple missiles, you can have them strike a single creature or several creatures. A single missile can strike only one creature. You must designate targets before you check for spell resistance or roll damage.

\featname{Magic Mouth}
\shortability{Illusion}{(Glamer)}
\shortability{Level:}{Brd 1, Sor/Wiz 2}
\shortability{Components:}{V, S, M}
\shortability{Casting Time:}{1 standard action}
\shortability{Range:}{Close (25 ft. + 5 ft./2 levels)}
\shortability{Target:}{One creature or object}
\shortability{Duration:}{Permanent until discharged}
\shortability{Saving Throw:}{Will negates (object)}
\shortability{Spell Resistance:}{Yes (object)}
\smallskip\noindent This spell imbues the chosen object or creature with an enchanted mouth that suddenly appears and speaks its message the next time a specified event occurs. The message, which must be twenty-five or fewer words long, can be in any language known by you and can be delivered over a period of 10 minutes. The mouth cannot utter verbal components, use command words, or activate magical effects. It does, however, move according to the words articulated; if it were placed upon a statue, the mouth of the statue would move and appear to speak. Of course, magic mouth can be placed upon a tree, rock, or any other object or creature.

\smallskip\noindent The spell functions when specific conditions are fulfilled according to your command as set in the spell. Commands can be as general or as detailed as desired, although only visual and audible triggers can be used. Triggers react to what appears to be the case. Disguises and illusions can fool them. Normal darkness does not defeat a visual trigger, but magical darkness or invisibility does. Silent movement or magical silence defeats audible triggers. Audible triggers can be keyed to general types of noises or to a specific noise or spoken word. Actions can serve as triggers if they are visible or audible. A magic mouth cannot distinguish alignment, level, Hit Dice, or class except by external garb.

\smallskip\noindent The range limit of a trigger is 15 feet per caster level, so a 6th-level caster can command a magic mouth to respond to triggers as far as 90 feet away. Regardless of range, the mouth can respond only to visible or audible triggers and actions in line of sight or within hearing distance.

\smallskip\noindent Magic mouth can be made permanent with a permanency spell.

\shortability{Material Component:}{A small bit of honeycomb and jade dust worth 10 gp.}

\featname{Magic Stone}
\shortability{Transmutation}{}
\shortability{Level:}{Clr 1, Drd 1, Earth 1}
\shortability{Components:}{V, S, DF}
\shortability{Casting Time:}{1 standard action}
\shortability{Range:}{Touch}
\shortability{Targets:}{Up to three pebbles touched}
\shortability{Duration:}{30 minutes or until discharged}
\shortability{Saving Throw:}{Will negates (harmless, object)}
\shortability{Spell Resistance:}{Yes (harmless, object)}
\smallskip\noindent You transmute as many as three pebbles, which can be no larger than sling bullets, so that they strike with great force when thrown or slung. If hurled, they have a range increment of 20 feet. If slung, treat them as sling bullets (range increment 50 feet). The spell gives them a +1 enhancement bonus on attack and damage rolls. The user of the stones makes a normal ranged attack. Each stone that hits deals 1d6+1 points of damage (including the spell's enhancement bonus), or 2d6+2 points against undead.

\featname{Magic Vestment}
\shortability{Transmutation}{}
\shortability{Level:}{Clr 3, Strength 3, War 3}
\shortability{Components:}{V, S, DF}
\shortability{Casting Time:}{1 standard action}
\shortability{Range:}{Touch}
\shortability{Target:}{Armor or shield touched}
\shortability{Duration:}{1 hour/level}
\shortability{Saving Throw:}{Will negates (harmless, object)}
\shortability{Spell Resistance:}{Yes (harmless, object)}
\smallskip\noindent You imbue a suit of armor or a shield with an enhancement bonus of +1 per four caster levels (maximum +5 at 20th level).

\smallskip\noindent An outfit of regular clothing counts as armor that grants no AC bonus for the purpose of this spell.

\featname{Magic Weapon}
\shortability{Transmutation}{}
\shortability{Level:}{Clr 1, Pal 1, Sor/Wiz 1, War 1}
\shortability{Components:}{V, S, DF}
\shortability{Casting Time:}{1 standard action}
\shortability{Range:}{Touch}
\shortability{Target:}{Weapon touched}
\shortability{Duration:}{1 min./level}
\shortability{Saving Throw:}{Will negates (harmless, object)}
\shortability{Spell Resistance:}{Yes (harmless, object)}
\smallskip\noindent Magic weapon gives a weapon a +1 enhancement bonus on attack and damage rolls. (An enhancement bonus does not stack with a masterwork weapon's +1 bonus on attack rolls.)

\smallskip\noindent You can't cast this spell on a natural weapon, such as an unarmed strike (instead, see magic fang). A monk's unarmed strike is considered a weapon, and thus it can be enhanced by this spell.

\featname{Magic Weapon, Greater}
\shortability{Transmutation}{}
\shortability{Level:}{Clr 4, Pal 3, Sor/Wiz 3}
\shortability{Components:}{V, S, M/DF}
\shortability{Casting Time:}{1 standard action}
\shortability{Range:}{Close (25 ft. + 5 ft./2 levels)}
\shortability{Target:}{One weapon or fifty projectiles (all of which must be in contact with each other at the time of casting)}
\shortability{Duration:}{1 hour/level}
\shortability{Saving Throw:}{Will negates (harmless, object)}
\shortability{Spell Resistance:}{Yes (harmless, object)}
\smallskip\noindent This spell functions like magic weapon, except that it gives a weapon an enhancement bonus on attack and damage rolls of +1 per four caster levels (maximum +5).

\smallskip\noindent Alternatively, you can affect as many as fifty arrows, bolts, or bullets. The projectiles must be of the same kind, and they have to be together (in the same quiver or other container). Projectiles, but not thrown weapons, lose their transmutation when used. (Treat shuriken as projectiles, rather than as thrown weapons, for the purpose of this spell.)

\shortability{Arcane Material Component:}{Powdered lime and carbon.}

\featname{Major Creation}
\shortability{Conjuration}{(Creation)}
\shortability{Level:}{Sor/Wiz 5}
\shortability{Casting Time:}{10 minutes}
\shortability{Range:}{Close (25 ft. + 5 ft./2 levels)}
\shortability{Duration:}{See text}
\smallskip\noindent This spell functions like minor creation, except that you can also create an object of mineral nature: stone, crystal, metal, or the like. The duration of the created item varies with its relative hardness and rarity, as indicated on the following table.
 
\begin{tabular}[h!]{l|l}
Hardness and Rarity Examples & Duration \\ \hline
Vegetable matter & 2 hr./level \\
Stone, crystal, base metals & 1 hr./level \\
Precious metals & 20 min./level \\
Gems & 10 min./level \\
Rare metal\textsuperscript{1} & 1 round/level \\
\multicolumn{2}{p{3in}}{\textsuperscript{1} Includes adamantine, alchemical silver, and mithral. You can't use major creation to create a cold iron item.}
\end{tabular}

\featname{Major Image}
\shortability{Illusion}{(Figment)}
\shortability{Level:}{Brd 3, Sor/Wiz 3}
\shortability{Duration:}{Concentration + 3 rounds}
\smallskip\noindent This spell functions like silent image, except that sound, smell, and thermal illusions are included in the spell effect. While concentrating, you can move the image within the range. 

\smallskip\noindent The image disappears when struck by an opponent unless you cause the illusion to react appropriately.

\featname{Make Whole}
\shortability{Transmutation}{}
\shortability{Level:}{Clr 2}
\shortability{Casting Time:}{1 standard action}
\shortability{Range:}{Close (25 ft. + 5 ft./2 levels)}
\shortability{Target:}{One object of up to 10 cu. ft./ level}
\smallskip\noindent This spell functions like mending, except that make whole completely repairs an object made of any substance, even one with multiple breaks, to be as strong as new. The spell does not restore the magical abilities of a broken magic item made whole, and it cannot mend broken magic rods, staffs, or wands. The spell does not repair items that have been warped, burned, disintegrated, ground to powder, melted, or vaporized, nor does it affect creatures (including constructs).

\featname{Mark of Justice}
\shortability{Necromancy}{}
\shortability{Level:}{Clr 5, Pal 4}
\shortability{Components:}{V, S, DF}
\shortability{Casting Time:}{10 minutes}
\shortability{Range:}{Touch}
\shortability{Target:}{Creature touched}
\shortability{Duration:}{Permanent; see text}
\shortability{Saving Throw:}{None}
\shortability{Spell Resistance:}{Yes}
\smallskip\noindent You draw an indelible mark on the subject and state some behavior on the part of the subject that will activate the mark. When activated, the mark curses the subject. Typically, you designate some sort of criminal behavior that activates the mark, but you can pick any act you please. The effect of the mark is identical with the effect of bestow curse.

\smallskip\noindent Since this spell takes 10 minutes to cast and involves writing on the target, you can cast it only on a creature that is willing or restrained.

\smallskip\noindent Like the effect of bestow curse, a mark of justice cannot be dispelled, but it can be removed with a break enchantment, limited wish, miracle, remove curse, or wish spell. Remove curse works only if its caster level is equal to or higher than your mark of justice caster level. These restrictions apply regardless of whether the mark has activated.

\featname{Mass (Spell Name)}
\smallskip\noindent Any spell whose name begins with mass is alphabetized in this chapter according to the second word of the spell name. Thus, the description of a mass spell appears near the description of the spell on which it is based. Spell chains that have mass spells in them include those based on the spells bear's endurance, bull's strength, cat's grace, charm monster, cure critical wounds, cure light wounds, cure moderate wounds, cure serious wounds, eagle's splendor, enlarge person, fox's cunning, heal, hold monster, hold person, inflict critical wounds, inflict light wounds, inflict moderate wounds, inflict serious wounds, invisibility, owl's wisdom, reduce person, and suggestion.

\featname{Maze}
\shortability{Conjuration}{(Teleportation)}
\shortability{Level:}{Sor/Wiz 8}
\shortability{Components:}{V, S}
\shortability{Casting Time:}{1 standard action}
\shortability{Range:}{Close (25 ft. + 5 ft./2 levels)}
\shortability{Target:}{One creature}
\shortability{Duration:}{See text}
\shortability{Saving Throw:}{None}
\shortability{Spell Resistance:}{Yes}
\smallskip\noindent You banish the subject into an extradimensional labyrinth of force planes. Each round on its turn, it may attempt a DC 20 Intelligence check to escape the labyrinth as a full-round action. If the subject doesn't escape, the maze disappears after 10 minutes, forcing the subject to leave.

\smallskip\noindent On escaping or leaving the maze, the subject reappears where it had been when the maze spell was cast. If this location is filled with a solid object, the subject appears in the nearest open space. Spells and abilities that move a creature within a plane, such as teleport and dimension door, do not help a creature escape a maze spell, although a plane shift spell allows it to exit to whatever plane is designated in that spell. Minotaurs are not affected by this spell.

\featname{Meld into Stone}
\shortability{Transmutation}{[Earth]}
\shortability{Level:}{Clr 3, Drd 3}
\shortability{Components:}{V, S, DF}
\shortability{Casting Time:}{1 standard action}
\shortability{Range:}{Personal}
\shortability{Target:}{You}
\shortability{Duration:}{10 min./level}
\smallskip\noindent Meld into stone enables you to meld your body and possessions into a single block of stone. The stone must be large enough to accommodate your body in all three dimensions. When the casting is complete, you and not more than 100 pounds of nonliving gear merge with the stone. If either condition is violated, the spell fails and is wasted.

\smallskip\noindent While in the stone, you remain in contact, however tenuous, with the face of the stone through which you melded. You remain aware of the passage of time and can cast spells on yourself while hiding in the stone. Nothing that goes on outside the stone can be seen, but you can still hear what happens around you. Minor physical damage to the stone does not harm you, but its partial destruction (to the extent that you no longer fit within it) expels you and deals you 5d6 points of damage. The stone's complete destruction expels you and slays you instantly unless you make a DC 18 Fortitude save.

\smallskip\noindent Any time before the duration expires, you can step out of the stone through the surface that you entered. If the spell's duration expires or the effect is dispelled before you voluntarily exit the stone, you are violently expelled and take 5d6 points of damage.

\smallskip\noindent The following spells harm you if cast upon the stone that you are occupying: Stone to flesh expels you and deals you 5d6 points of damage. Stone shape deals you 3d6 points of damage but does not expel you. Transmute rock to mud expels you and then slays you instantly unless you make a DC 18 Fortitude save, in which case you are merely expelled. Finally, passwall expels you without damage.

\featname{Mending}
\shortability{Transmutation}{}
\shortability{Level:}{Brd 0, Clr 0, Drd 0, Sor/Wiz 0}
\shortability{Components:}{V, S}
\shortability{Casting Time:}{1 standard action}
\shortability{Range:}{10 ft.}
\shortability{Target:}{One object of up to 1 lb.}
\shortability{Duration:}{Instantaneous}
\shortability{Saving Throw:}{Will negates (harmless, object)}
\shortability{Spell Resistance:}{Yes (harmless, object)}
\smallskip\noindent Mending repairs small breaks or tears in objects (but not warps, such as might be caused by a warp wood spell). It will weld broken metallic objects such as a ring, a chain link, a medallion, or a slender dagger, providing but one break exists.

\smallskip\noindent Ceramic or wooden objects with multiple breaks can be invisibly rejoined to be as strong as new. A hole in a leather sack or a wineskin is completely healed over by mending. The spell can repair a magic item, but the item's magical abilities are not restored. The spell cannot mend broken magic rods, staffs, or wands, nor does it affect creatures (including constructs).

\featname{Message}
\shortability{Transmutation}{[Language-Dependent]}
\shortability{Level:}{Brd 0, Sor/Wiz 0}
\shortability{Components:}{V, S, F}
\shortability{Casting Time:}{1 standard action}
\shortability{Range:}{Medium (100 ft. + 10 ft./level)}
\shortability{Targets:}{One creature/level}
\shortability{Duration:}{10 min./level}
\shortability{Saving Throw:}{None}
\shortability{Spell Resistance:}{No}
\smallskip\noindent You can whisper messages and receive whispered replies with little chance of being overheard. You point your finger at each creature you want to receive the message. When you whisper, the whispered message is audible to all targeted creatures within range. Magical silence, 1 foot of stone, 1 inch of common metal (or a thin sheet of lead), or 3 feet of wood or dirt blocks the spell. The message does not have to travel in a straight line. It can circumvent a barrier if there is an open path between you and the subject, and the path's entire length lies within the spell's range. The creatures that receive the message can whisper a reply that you hear. The spell transmits sound, not meaning. It doesn't transcend language barriers.

\shortability{Note:}{To speak a message, you must mouth the words and whisper, possibly allowing observers the opportunity to read your lips.}
\shortability{Focus:}{A short piece of copper wire.}

\featname{Meteor Swarm}
\shortability{Evocation}{[Fire]}
\shortability{Level:}{Sor/Wiz 9}
\shortability{Components:}{V, S}
\shortability{Casting Time:}{1 standard action}
\shortability{Range:}{Long (400 ft. + 40 ft./level)}
\shortability{Area:}{Four 40-ft.-radius spreads; see text}
\shortability{Duration:}{Instantaneous}
\shortability{Saving Throw:}{None or Reflex half; see text}
\shortability{Spell Resistance:}{Yes}
\smallskip\noindent Meteor swarm is a very powerful and spectacular spell that is similar to fireball in many aspects. When you cast it, four 2- foot-diameter spheres spring from your outstretched hand and streak in straight lines to the spots you select. The meteor spheres leave a fiery trail of sparks.

\smallskip\noindent If you aim a sphere at a specific creature, you may make a ranged touch attack to strike the target with the meteor. Any creature struck by one of these spheres takes 2d6 points of bludgeoning damage (no save) and receives no saving throw against the sphere's fire damage (see below). If a targeted sphere misses its target, it simply explodes at the nearest corner of the target's space. You may aim more than one meteor at the same target.

\smallskip\noindent Once a sphere reaches its destination, it explodes in a 40-foot-radius spread, dealing 6d6 points of fire damage to each creature in the area. If a creature is within the area of more than one sphere, it must save separately against each. (Fire resistance applies to each sphere's damage individually.)

\featname{Mind Blank}
\shortability{Abjuration}{}
\shortability{Level:}{Protection 8, Sor/Wiz 8}
\shortability{Components:}{V, S}
\shortability{Casting Time:}{1 standard action}
\shortability{Range:}{Close (25 ft. + 5 ft./2 levels)}
\shortability{Target:}{One creature}
\shortability{Duration:}{24 hours}
\shortability{Saving Throw:}{Will negates (harmless)}
\shortability{Spell Resistance:}{Yes (harmless)}
\smallskip\noindent The subject is protected from all devices and spells that detect, influence, or read emotions or thoughts. This spell protects against all mind-affecting spells and effects as well as information gathering by divination spells or effects. Mind blank even foils limited wish, miracle, and wish spells when they are used in such a way as to affect the subject's mind or to gain information about it. In the case of scrying that scans an area the creature is in, such as arcane eye, the spell works but the creature simply isn't detected. Scrying attempts that are targeted specifically at the subject do not work at all.

\featname{Mind Fog}
\shortability{Enchantment}{(Compulsion) [Mind-Affecting]}
\shortability{Level:}{Brd 5, Sor/Wiz 5}
\shortability{Components:}{V, S}
\shortability{Casting Time:}{1 standard action}
\shortability{Range:}{Medium (100 ft. + 10 ft./level)}
\shortability{Effect:}{Fog spreads in 20-ft. radius, 20 ft. high}
\shortability{Duration:}{30 minutes and 2d6 rounds; see text}
\shortability{Saving Throw:}{Will negates}
\shortability{Spell Resistance:}{Yes}
\smallskip\noindent Mind fog produces a bank of thin mist that weakens the mental resistance of those caught in it. Creatures in the mind fog take a -10 competence penalty on Wisdom checks and Will saves. (A creature that successfully saves against the fog is not affected and need not make further saves even if it remains in the fog.) Affected creatures take the penalty as long as they remain in the fog and for 2d6 rounds thereafter. The fog is stationary and lasts for 30 minutes (or until dispersed by wind).

\smallskip\noindent A moderate wind (11+ mph) disperses the fog in four rounds; a strong wind (21+ mph) disperses the fog in 1 round.

\smallskip\noindent The fog is thin and does not significantly hamper vision.

\featname{Minor Creation}
\shortability{Conjuration}{(Creation)}
\shortability{Level:}{Sor/Wiz 4}
\shortability{Components:}{V, S, M}
\shortability{Casting Time:}{1 minute}
\shortability{Range:}{0 ft.}
\shortability{Effect:}{Unattended, nonmagical object of nonliving plant matter, up to 1 cu. ft./level}
\shortability{Duration:}{1 hour/level (D)}
\shortability{Saving Throw:}{None}
\shortability{Spell Resistance:}{No}

\smallskip\noindent You create a nonmagical, unattended object of nonliving, vegetable matter. The volume of the item created cannot exceed 1 cubic foot per caster level. You must succeed on an appropriate skill check to make a complex item.

\smallskip\noindent Attempting to use any created object as a material component causes the spell to fail.

\shortability{Material Component:}{A tiny piece of matter of the same sort of item you plan to create with minor creation.}

\featname{Minor Image}
\shortability{Illusion}{(Figment)}
\shortability{Level:}{Brd 2, Sor/Wiz 2}
\shortability{Duration:}{Concentration +2 rounds}

\smallskip\noindent This spell functions like silent image, except that minor image includes some minor sounds but not understandable speech.

\featname{Miracle}
\shortability{Evocation}{}
\shortability{Level:}{Clr 9, Luck 9}
\shortability{Components:}{V, S, XP; see text}
\shortability{Casting Time:}{1 standard action}
\shortability{Range:}{See text}
\shortability{Target, Effect, or Area:}{See text}
\shortability{Duration:}{See text}
\shortability{Saving Throw:}{See text}
\shortability{Spell Resistance:}{Yes}
\smallskip\noindent You don't so much cast a miracle as request one. You state what you would like to have happen and request that your deity (or the power you pray to for spells) intercede.

\smallskip\noindent A miracle can do any of the following things.

\listone
  \item Duplicate any cleric spell of 8th level or lower (including spells to which you have access because of your domains). 
  \item Duplicate any other spell of 7th level or lower.
  \item Undo the harmful effects of certain spells, such as feeblemind or insanity.
  \item Have any effect whose power level is in line with the above effects.
\end{list}
  
\smallskip\noindent If the miracle has any of the above effects, casting it has no experience point cost.

\smallskip\noindent Alternatively, a cleric can make a very powerful request. Casting such a miracle costs the cleric 5,000 XP because of the powerful divine energies involved. Examples of especially powerful miracles of this sort could include the following.

\listone
  \item Swinging the tide of a battle in your favor by raising fallen allies to continue fighting.
  \item Moving you and your allies, with all your and their gear, from one plane to another through planar barriers to a specific locale with no chance of error.
  \item Protecting a city from an earthquake, volcanic eruption, flood, or other major natural disaster.
\end{list}

\smallskip\noindent In any event, a request that is out of line with the deity's (or alignment's) nature is refused.

\smallskip\noindent A duplicated spell allows saving throws and spell resistance as normal, but the save DCs are as for a 9th-level spell. When a miracle duplicates a spell that has an XP cost, you must pay that cost. When a miracle spell duplicates a spell with a material component that costs more than 100 gp, you must provide that component.

\shortability{XP Cost:}{5,000 XP (for some uses of the miracle spell; see above).}

\featname{Mirage Arcana}
\shortability{Illusion}{(Glamer)}
\shortability{Level:}{Brd 5, Sor/Wiz 5}
\shortability{Components:}{V, S}
\shortability{Casting Time:}{1 standard action}
\shortability{Area:}{One 20-ft. cube/level (S)}
\shortability{Duration:}{Concentration +1 hour/ level (D)}
\smallskip\noindent This spell functions like hallucinatory terrain, except that it enables you to make any area appear to be something other than it is. The illusion includes audible, visual, tactile, and olfactory elements. Unlike hallucinatory terrain, the spell can alter the appearance of structures (or add them where none are present). Still, it can't disguise, conceal, or add creatures (though creatures within the area might hide themselves within the illusion just as they can hide themselves within a real location).

\featname{Mirror Image}
\shortability{Illusion}{(Figment)}
\shortability{Level:}{Brd 2, Sor/Wiz 2}
\shortability{Components:}{V, S}
\shortability{Casting Time:}{1 standard action}
\shortability{Range:}{Personal; see text}
\shortability{Target:}{You}
\shortability{Duration:}{1 min./level (D)}
\smallskip\noindent Several illusory duplicates of you pop into being, making it difficult for enemies to know which target to attack. The figments stay near you and disappear when struck.

\smallskip\noindent Mirror image creates 1d4 images plus one image per three caster levels (maximum eight images total). These figments separate from you and remain in a cluster, each within 5 feet of at least one other figment or you. You can move into and through a mirror image. When you and the mirror image separate, observers can't use vision or hearing to tell which one is you and which the image. The figments may also move through each other. The figments mimic your actions, pretending to cast spells when you cast a spell, drink potions when you drink a potion, levitate when you levitate, and so on.

\smallskip\noindent Enemies attempting to attack you or cast spells at you must select from among indistinguishable targets. Generally, roll randomly to see whether the selected target is real or a figment. Any successful attack against an image destroys it. An image's AC is 10 + your size modifier + your Dex modifier. Figments seem to react normally to area spells (such as looking like they're burned or dead after being hit by a fireball).

\smallskip\noindent While moving, you can merge with and split off from figments so that enemies who have learned which image is real are again confounded.

\smallskip\noindent An attacker must be able to see the images to be fooled. If you are invisible or an attacker shuts his or her eyes, the spell has no effect. (Being unable to see carries the same penalties as being blinded.)

\featname{Misdirection}
\shortability{Illusion}{(Glamer)}
\shortability{Level:}{Brd 2, Sor/Wiz 2}
\shortability{Components:}{V, S}
\shortability{Casting Time:}{1 standard action}
\shortability{Range:}{Close (25 ft. + 5 ft./2 levels)}
\shortability{Target:}{One creature or object, up to a 10-ft. cube in size}
\shortability{Duration:}{1 hour/level}
\shortability{Saving Throw:}{None or Will negates; see text}
\shortability{Spell Resistance:}{No}
\smallskip\noindent By means of this spell, you misdirect the information from divination spells that reveal auras (detect evil, detect magic, discern lies, and the like). On casting the spell, you choose another object within range. For the duration of the spell, the subject of misdirection is detected as if it were the other object. (Neither the subject nor the other object gets a saving throw against this effect.) Detection spells provide information based on the second object rather than on the actual target of the detection unless the caster of the detection succeeds on a Will save. For instance, you could make yourself detect as a tree if one were within range at casting: not evil, not lying, not magical, neutral in alignment, and so forth. This spell does not affect other types of divination magic (augury, detect thoughts, clairaudience/clairvoyance, and the like).

\featname{Mislead}
\shortability{Illusion}{(Figment, Glamer)}
\shortability{Level:}{Brd 5, Luck 6, Sor/Wiz 6, Trickery 6}
\shortability{Components:}{S}
\shortability{Casting Time:}{1 standard action}
\shortability{Range:}{Close (25 ft. + 5 ft./2 levels)}
Target/\shortability{Effect:}{You/one illusory double}
\shortability{Duration:}{1 round/level (D) and concentration + 3 rounds; see text}
\shortability{Saving Throw:}{None or Will disbelief (if interacted with); see text}
\shortability{Spell Resistance:}{No}
\smallskip\noindent You become invisible (as improved invisibility, a glamer), and at the same time, an illusory double of you (as major image, a figment) appears. You are then free to go elsewhere while your double moves away. The double appears within range but thereafter moves as you direct it (which requires concentration beginning on the first round after the casting). You can make the figment appear superimposed perfectly over your own body so that observers don't notice an image appearing and you turning invisible. You and the figment can then move in different directions. The double moves at your speed and can talk and gesture as if it were real, but it cannot attack or cast spells, though it can pretend to do so.

\smallskip\noindent The illusory double lasts as long as you concentrate upon it, plus 3 additional rounds. After you cease concentration, the illusory double continues to carry out the same activity until the duration expires. The improved invisibility lasts for 1 round per level, regardless of concentration.

\featname{Mnemonic Enhancer}
\shortability{Transmutation}{}
\shortability{Level:}{Wiz 4}
\shortability{Components:}{V, S, M, F}
\shortability{Casting Time:}{10 minutes}
\shortability{Range:}{Personal}
\shortability{Target:}{You}
\shortability{Duration:}{Instantaneous}
\smallskip\noindent Casting this spell allows you to prepare additional spells or retain spells recently cast. Pick one of these two versions when the spell is cast.

\shortability{Prepare:}{You prepare up to three additional levels of spells. A cantrip counts as 1/2 level for this purpose. You prepare and cast these spells normally.}
\shortability{Retain:}{You retain any spell of 3rd level or lower that you had cast up to 1 round before you started casting the mnemonic enhancer. This restores the previously cast spell to your mind.}

\smallskip\noindent In either event, the spell or spells prepared or retained fade after 24 hours (if not cast).

\shortability{Material Component:}{A piece of string, and ink consisting of squid secretion with black dragon's blood.}
\shortability{Focus:}{An ivory plaque of at least 50 gp value.}

\featname{Modify Memory}
\shortability{Enchantment}{(Compulsion) [Mind-Affecting]}
\shortability{Level:}{Brd 4}
\shortability{Components:}{V, S}
\shortability{Casting Time:}{1 round; see text}
\shortability{Range:}{Close (25 ft. + 5 ft./2 levels)}
\shortability{Target:}{One living creature}
\shortability{Duration:}{Permanent}
\shortability{Saving Throw:}{Will negates}
\shortability{Spell Resistance:}{Yes}
\smallskip\noindent You reach into the subject's mind and modify as many as 5 minutes of its memories in one of the following ways.

\listone
  \item Eliminate all memory of an event the subject actually experienced. This spell cannot negate charm, geas/quest, suggestion, or similar spells.
  \item  Allow the subject to recall with perfect clarity an event it actually experienced.
  \item  Change the details of an event the subject actually experienced.
  \item  Implant a memory of an event the subject never experienced.
\end{list}

\smallskip\noindent Casting the spell takes 1 round. If the subject fails to save, you proceed with the spell by spending as much as 5 minutes (a period of time equal to the amount of memory time you want to modify) visualizing the memory you wish to modify in the subject. If your concentration is disturbed before the visualization is complete, or if the subject is ever beyond the spell's range during this time, the spell is lost.

\smallskip\noindent A modified memory does not necessarily affect the subject's actions, particularly if it contradicts the creature's natural inclinations. An illogical modified memory is dismissed by the creature as a bad dream or a memory muddied by too much wine. 

\featname{Moment of Prescience}
\shortability{Divination}{}
\shortability{Level:}{Luck 8, Sor/Wiz 8}
\shortability{Components:}{V, S}
\shortability{Casting Time:}{1 standard action}
\shortability{Range:}{Personal}
\shortability{Target:}{You}
\shortability{Duration:}{1 hour/level or until discharged}
\smallskip\noindent This spell grants you a powerful sixth sense in relation to yourself. Once during the spell's duration, you may choose to use its effect. This spell grants you an insight bonus equal to your caster level (maximum +25) on any single attack roll, opposed ability or skill check, or saving throw. Alternatively, you can apply the insight bonus to your AC against a single attack (even if flatfooted). Activating the effect doesn't take an action; you can even activate it on another character's turn if needed. You must choose to use the moment of prescience before you make the roll it is to modify. Once used, the spell ends.

\smallskip\noindent You can't have more than one moment of prescience active on you at the same time.

\featname{Mount}
\shortability{Conjuration}{(Summoning)}
\shortability{Level:}{Sor/Wiz 1}
\shortability{Components:}{V, S, M}
\shortability{Casting Time:}{1 round}
\shortability{Range:}{Close (25 ft. + 5 ft./2 levels)}
\shortability{Effect:}{One mount}
\shortability{Duration:}{2 hours/level (D)}
\shortability{Saving Throw:}{None}
\shortability{Spell Resistance:}{No}
\smallskip\noindent You summon a light horse or a pony (your choice) to serve you as a mount. The steed serves willingly and well. The mount comes with a bit and bridle and a riding saddle.

\shortability{Material Component:}{A bit of horse hair.}

\featname{Move Earth}
\shortability{Transmutation}{[Earth]}
\shortability{Level:}{Drd 6, Sor/Wiz 6}
\shortability{Components:}{V, S, M}
\shortability{Casting Time:}{See text}
\shortability{Range:}{Long (400 ft. + 40 ft./level)}
\shortability{Area:}{Dirt in an area up to 750 ft. square and up to 10 ft. deep (S)}
\shortability{Duration:}{Instantaneous}
\shortability{Saving Throw:}{None}
\shortability{Spell Resistance:}{No}
\smallskip\noindent Move earth moves dirt (clay, loam, sand), possibly collapsing embankments, moving hillocks, shifting dunes, and so forth.

\smallskip\noindent However, in no event can rock formations be collapsed or moved. The area to be affected determines the casting time. For every 150-foot square (up to 10 feet deep), casting takes 10 minutes. The maximum area, 750 feet by 750 feet, takes 4 hours and 10 minutes to move.

\smallskip\noindent This spell does not violently break the surface of the ground. Instead, it creates wavelike crests and troughs, with the earth reacting with glacierlike fluidity until the desired result is achieved. Trees, structures, rock formations, and such are mostly unaffected except for changes in elevation and relative topography.

\smallskip\noindent The spell cannot be used for tunneling and is generally too slow to trap or bury creatures. Its primary use is for digging or filling moats or for adjusting terrain contours before a battle.

\smallskip\noindent This spell has no effect on earth creatures.

\shortability{Material Component:}{A mixture of soils (clay, loam, and sand) in a small bag, and an iron blade.}

\featname{Neutralize Poison}
\shortability{Conjuration}{(Healing)}
\shortability{Level:}{Brd 4, Clr 4, Drd 3, Pal 4, Rgr 3}
\shortability{Components:}{V, S, M/DF}
\shortability{Casting Time:}{1 standard action}
\shortability{Range:}{Touch}
\shortability{Target:}{Creature or object of up to 1 cu. ft./level touched}
\shortability{Duration:}{10 min./level}
\shortability{Saving Throw:}{Will negates (harmless, object)}
\shortability{Spell Resistance:}{Yes (harmless, object)}
\smallskip\noindent You detoxify any sort of venom in the creature or object touched. A poisoned creature suffers no additional effects from the poison, and any temporary effects are ended, but the spell does not reverse instantaneous effects, such as hit point damage, temporary ability damage, or effects that don't go away on their own.

\smallskip\noindent The creature is immune to any poison it is exposed to during the duration of the spell. Unlike with delay poison, such effects aren't postponed until after the duration, the creature need not make any saves against poison effects applied to it during the length of the spell.

\smallskip\noindent This spell can instead neutralize the poison in a poisonous creature or object for the duration of the spell, at the caster's option.

\shortability{Arcane Material Component:}{A bit of charcoal.}

\featname{Nightmare}
\shortability{Illusion}{(Phantasm) [Mind-Affecting, Evil]}
\shortability{Level:}{Brd 5, Sor/Wiz 5}
\shortability{Components:}{V, S}
\shortability{Casting Time:}{10 minutes}
\shortability{Range:}{Unlimited}
\shortability{Target:}{One living creature}
\shortability{Duration:}{Instantaneous}
\shortability{Saving Throw:}{Will negates; see text}
\shortability{Spell Resistance:}{Yes}
\smallskip\noindent You send a hideous and unsettling phantasmal vision to a specific creature that you name or otherwise specifically designate.

\smallskip\noindent The nightmare prevents restful sleep and causes 1d10 points of damage. The nightmare leaves the subject fatigued and unable to regain arcane spells for the next 24 hours.

\smallskip\noindent The difficulty of the save depends on how well you know the subject and what sort of physical connection (if any) you have to that creature.

\begin{tabular}[h!]{p{2in}|l}
Knowledge & Will Save Modifier \\ \hline
None\textsuperscript{1} & +10 \\
Secondhand (you have heard of the subject) & +5	\\
Firsthand (you have met the subject) & +0	\\
Familiar (you know the subject well) & -5	\\
\multicolumn{2}{p{3in}}{\textsuperscript{1} You must have some sort of connection to a creature you have no knowledge of.}

\begin{tabular}[h!]{l|l}
Connection & Will Save Modifier \\ \hline
Likeness or picture & -2 \\
Possession or garment & -4 \\
Body part, lock of hair, bit of nail, etc. & -10 \\
\end{tabular}

\smallskip\noindent Dispel evil cast on the subject while you are casting the spell dispels the nightmare and causes you to be stunned for 10 minutes per caster level of the dispel evil.

\smallskip\noindent If the recipient is awake when the spell begins, you can choose to cease casting (ending the spell) or to enter a trance until the recipient goes to sleep, whereupon you become alert again and complete the casting. If you are disturbed during the trance, you must succeed on a Concentration check as if you were in the midst of casting a spell or the spell ends.

\smallskip\noindent If you choose to enter a trance, you are not aware of your surroundings or the activities around you while in the trance.

\smallskip\noindent You are defenseless, both physically and mentally, while in the trance. (You always fail any saving throw, for example.)

\smallskip\noindent Creatures who don't sleep (such as elves, but not half-elves) or dream are immune to this spell.

\featname{Nondetection}
\shortability{Abjuration}{}
\shortability{Level:}{Rgr 4, Sor/Wiz 3, Trickery 3}
\shortability{Components:}{V, S, M}
\shortability{Casting Time:}{1 standard action}
\shortability{Range:}{Touch}
\shortability{Target:}{Creature or object touched}
\shortability{Duration:}{1 hour/level}
\shortability{Saving Throw:}{Will negates (harmless, object)}
\shortability{Spell Resistance:}{Yes (harmless, object)}
\smallskip\noindent The warded creature or object becomes difficult to detect by divination spells such as clairaudience/clairvoyance, locate object, and detect spells. Nondetection also prevents location by such magic items as crystal balls. If a divination is attempted against the warded creature or item, the caster of the divination must succeed on a caster level check (1d20 + caster level) against a DC of 11 + the caster level of the spellcaster who cast nondetection. If you cast nondetection on yourself or on an item currently in your possession, the DC is 15 + your caster level.

\smallskip\noindent If cast on a creature, nondetection wards the creature's gear as well as the creature itself.

\shortability{Material Component:}{A pinch of diamond dust worth 50 gp.}

\featname{Obscure Object}
\shortability{Abjuration}{}
\shortability{Level:}{Brd 1, Clr 3, Sor/Wiz 2}
\shortability{Components:}{V, S, M/DF}
\shortability{Casting Time:}{1 standard action}
\shortability{Range:}{Touch}
\shortability{Target:}{One object touched of up to 100 lb./level}
\shortability{Duration:}{8 hours (D)}
\shortability{Saving Throw:}{Will negates (object)}
\shortability{Spell Resistance:}{Yes (object)}
\smallskip\noindent This spell hides an object from location by divination (scrying) effects, such as the scrying spell or a crystal ball. Such an attempt automatically fails (if the divination is targeted on the object) or fails to perceive the object (if the divination is targeted on a nearby location, object, or person).

\shortability{Arcane Material Component:}{A piece of chameleon skin.}

\featname{Obscuring Mist}
\shortability{Conjuration}{(Creation)}
\shortability{Level:}{Air 1, Clr 1, Drd 1, Sor/Wiz 1, Water 1}
\shortability{Components:}{V, S}
\shortability{Casting Time:}{1 standard action}
\shortability{Range:}{20 ft.}
\shortability{Effect:}{Cloud spreads in 20-ft. radius from you, 20 ft. high}
\shortability{Duration:}{1 min./level}
\shortability{Saving Throw:}{None}
\shortability{Spell Resistance:}{No}
\smallskip\noindent A misty vapor arises around you. It is stationary once created. The vapor obscures all sight, including darkvision, beyond 5 feet. A creature 5 feet away has concealment (attacks have a 20\% miss chance). Creatures farther away have total concealment (50\% miss chance, and the attacker cannot use sight to locate the target).

\smallskip\noindent A moderate wind (11+ mph), such as from a gust of wind spell, disperses the fog in 4 rounds. A strong wind (21+ mph) disperses the fog in 1 round. A fireball, flame strike, or similar spell burns away the fog in the explosive or fiery spell's area. A wall of fire burns away the fog in the area into which it deals damage.

\smallskip\noindent This spell does not function underwater.

\featname{Open/Close}
\shortability{Transmutation}{}
\shortability{Level:}{Brd 0, Sor/Wiz 0}
\shortability{Components:}{V, S, F}
\shortability{Casting Time:}{1 standard action}
\shortability{Range:}{Close (25 ft. + 5 ft./2 levels)}
\shortability{Target:}{Object weighing up to 30 lb. or portal that can be opened or closed}
\shortability{Duration:}{Instantaneous}
\shortability{Saving Throw:}{Will negates (object)}
\shortability{Spell Resistance:}{Yes (object)}
\smallskip\noindent You can open or close (your choice) a door, chest, box, window, bag, pouch, bottle, barrel, or other container. If anything resists this activity (such as a bar on a door or a lock on a chest), the spell fails. In addition, the spell can only open and close things weighing 30 pounds or less. Thus, doors, chests, and similar objects sized for enormous creatures may be beyond this spell's ability to affect.

\shortability{Focus:}{A brass key.}

\featname{Order's Wrath}
\shortability{Evocation}{[Lawful]}
\shortability{Level:}{Law 4}
\shortability{Components:}{V, S}
\shortability{Casting Time:}{1 standard action}
\shortability{Range:}{Medium (100 ft. + 10 ft./level)}
\shortability{Area:}{Nonlawful creatures within a burst that fills a 30-ft. cube}
\shortability{Duration:}{Instantaneous (1 round); see text}
\shortability{Saving Throw:}{Will partial; see text}
\shortability{Spell Resistance:}{Yes}
\smallskip\noindent You channel lawful power to smite enemies. The power takes the form of a three-dimensional grid of energy. Only chaotic and neutral (not lawful) creatures are harmed by the spell.

\smallskip\noindent The spell deals 1d8 points of damage per two caster levels (maximum 5d8) to chaotic creatures (or 1d6 points of damage per caster level, maximum 10d6, to chaotic outsiders) and causes them to be dazed for 1 round. A successful Will save reduces the damage to half and negates the daze effect.

\smallskip\noindent The spell deals only half damage to creatures who are neither chaotic nor lawful, and they are not dazed. They can reduce the damage in half again (down to one-quarter of the roll) with a successful Will save.

\featname{Overland Flight}
\shortability{Transmutation}{}
\shortability{Level:}{Sor/Wiz 5}
\shortability{Components:}{V, S}
\shortability{Range:}{Personal}
\shortability{Target:}{You}
\shortability{Duration:}{1 hour/level}

\smallskip\noindent This spell functions like a fly spell, except you can fly at a speed of 40 feet (30 feet if wearing medium or heavy armor, or if carrying a medium or heavy load) with average maneuverability. When using this spell for long-distance movement, you can hustle without taking nonlethal damage (a forced march still requires Constitution checks). This means you can cover 64 miles in an eight-hour period of flight (or 48 miles at a speed of 30 feet).

\featname{Owl's Wisdom}
\shortability{Transmutation}{}
\shortability{Level:}{Clr 2, Drd 2, Pal 2, Rgr 2, Sor/Wiz 2}
\shortability{Components:}{V, S, M/DF}
\shortability{Casting Time:}{1 standard action}
\shortability{Range:}{Touch}
\shortability{Target:}{Creature touched}
\shortability{Duration:}{1 min./level}
\shortability{Saving Throw:}{Will negates (harmless)}
\shortability{Spell Resistance:}{Yes}

\smallskip\noindent The transmuted creature becomes wiser. The spell grants a +4 enhancement bonus to Wisdom, adding the usual benefit to Wisdom-related skills. Clerics, druids, paladins, and rangers (and other Wisdom-based spellcasters) who receive owl's wisdom do not gain any additional bonus spells for the increased Wisdom, but the save DCs for their spells increase.

\shortability{Arcane Material Component:}{A few feathers, or a pinch of droppings, from an owl.}

\featname{Owl's Wisdom, Mass}
\shortability{Transmutation}{}
\shortability{Level:}{Clr 6, Drd 6, Sor/Wiz 6}
\shortability{Range:}{Close (25 ft. + 5 ft./2 levels)}
\shortability{Target:}{One creature/level, no two of which can be more than 30 ft. apart}
This spell functions like owl's wisdom, except that it affects multiple creatures.

\featname{Passwall}
\shortability{Transmutation}{}
\shortability{Level:}{Sor/Wiz 5}
\shortability{Components:}{V, S, M}
\shortability{Casting Time:}{1 standard action}
\shortability{Range:}{Touch}
\shortability{Effect:}{5 ft. by 8 ft. opening, 10 ft. deep plus 5 ft. deep per three additional levels}
\shortability{Duration:}{1 hour/level (D)}
\shortability{Saving Throw:}{None}
\shortability{Spell Resistance:}{No}
\smallskip\noindent You create a passage through wooden, plaster, or stone walls, but not through metal or other harder materials. The passage is 10 feet deep plus an additional 5 feet deep per three caster levels above 9th (15 feet at 12th, 20 feet at 15th, and a maximum of 25 feet deep at 18th level). If the wall's thickness is more than the depth of the passage created, then a single passwall simply makes a niche or short tunnel. Several passwall spells can then form a continuing passage to breach very thick walls. When passwall ends, creatures within the passage are ejected out the nearest exit. If someone dispels the passwall or you dismiss it, creatures in the passage are ejected out the far exit, if there is one, or out the sole exit if there is only one.

\shortability{Material Component:}{A pinch of sesame seeds.}

\featname{Pass without Trace}
\shortability{Transmutation}{}
\shortability{Level:}{Drd 1, Rgr 1}
\shortability{Components:}{V, S, DF}
\shortability{Casting Time:}{1 standard action}
\shortability{Range:}{Touch}
\shortability{Targets:}{One creature/level touched}
\shortability{Duration:}{1 hour/level (D)}
\shortability{Saving Throw:}{Will negates (harmless)}
\shortability{Spell Resistance:}{Yes (harmless)}
\smallskip\noindent The subject or subjects can move through any type of terrain and leave neither footprints nor scent. Tracking the subjects is impossible by nonmagical means.

\featname{Permanency}
\shortability{Universal}{}
\shortability{Level:}{Sor/Wiz 5}
\shortability{Components:}{V, S, XP}
\shortability{Casting Time:}{2 rounds}
\shortability{Range:}{See text}
\shortability{Target, Effect, or Area:}{See text}
\shortability{Duration:}{Permanent; see text}
\shortability{Saving Throw:}{None}
\shortability{Spell Resistance:}{No}
\smallskip\noindent This spell makes certain other spells permanent.

\smallskip\noindent Depending on the spell, you must be of a minimum caster level and must expend a number of XP.

\smallskip\noindent You can make the following spells permanent in regard to yourself.

\begin{tabular}[h!]{l|ll}
Spell & Minimum Caster Level & XP Cost \\ \hline
Arcane sight & 11th & 1,500 XP \\
Comprehend languages & 9th & 500 XP \\
Darkvision & 10th & 1,000 XP \\
Detect magic & 9th & 500 XP \\
Read magic & 9th & 500 XP \\
See invisibility & 10th & 1,000 XP \\
Tongues & 11th & 1,500 XP \\
\end{tabular}

\smallskip\noindent You cast the desired spell and then follow it with the permanency spell. You cannot cast these spells on other creatures. This application of permanency can be dispelled only by a caster of higher level than you were when you cast the spell.

\smallskip\noindent In addition to personal use, permanency can be used to make the following spells permanent on yourself, another creature, or an object (as appropriate).
 
\begin{tabular}[h!]{l|ll}
Spell & Minimum Caster Level & XP Cost \\
Enlarge person & 9th & 500 XP	\\
Magic fang & 9th & 500 XP \\	   
Magic fang, greater & 11th & 1,500 XP	\\
Reduce person & 9th & 500 XP \\
Resistance & 9th & 500 XP	\\
Telepathic bond\textsuperscript{1} & 13th & 2,500 XP \\ \hline
\multicolumn{3}{p{3in}}{\textsuperscript{1} Only bonds two creatures per casting of permanency.} \\
\end{tabular}

\smallskip\noindent Additionally, the following spells can be cast upon objects or areas only and rendered permanent.

\begin{tabular}[h!]{l|ll}
Spell & Minimum Caster Level & XP Cost \\
Alarm & 9th & 500 XP \\
Animate objects & 14th & 3,000 XP	\\
Dancing lights & 9th & 500 XP	\\
Ghost sound & 9th & 500 XP \\
Gust of wind & 11th & 1,500 XP \\
Invisibility & 10th & 1,000 XP \\
Mage's private sanctum & 13th & 2,500 XP \\
Magic mouth & 10th & 1,000 XP	\\
Phase door & 15th & 3,500 XP \\
Prismatic sphere & 17th & 4,500 XP \\
Prismatic wall & 16th & 4,000 XP \\
Shrink item & 11th & 1,500 XP \\
Solid fog & 12th & 2,000 XP \\
Stinking cloud & 11th & 1,500 XP \\
Symbol of death & 16th & 4,000 XP \\
Symbol of fear & 14th & 3,000 XP \\
Symbol of insanity & 16th & 4,000 XP \\
Symbol of pain & 13th & 2,500 XP \\
Symbol of persuasion & 14th & 3,000 XP \\
Symbol of sleep & 16th & 4,000 XP \\
Symbol of stunning & 15th & 3,500 XP \\
Symbol of weakness & 15th & 3,500 XP \\
Teleportation circle & 17th & 4,500 XP \\
Wall of fire & 12th & 2,000 XP \\
Wall of force & 13th & 2,500 XP \\
Web & 10th & 1,000 XP \\
\end{tabular}

\smallskip\noindent Spells cast on other creatures, objects, or locations (not on you) are vulnerable to dispel magic as normal.

\shortability{XP Cost:}{See tables above.}

\featname{Permanent Image}
\shortability{Illusion}{(Figment)}
\shortability{Level:}{Brd 6, Sor/Wiz 6}
\shortability{Effect:}{Figment that cannot extend beyond a 20-ft. cube + one 10-ft. cube/level (S)}
\shortability{Duration:}{Permanent (D)}
\smallskip\noindent This spell functions like silent image, except that the figment includes visual, auditory, olfactory, and thermal elements, and the spell is permanent. By concentrating, you can move the image within the limits of the range, but it is static while you are not concentrating.

\shortability{Material Component:}{A bit of fleece plus powdered jade worth 100 gp.}

\featname{Persistent Image}
\shortability{Illusion}{(Figment)}
\shortability{Level:}{Brd 5, Sor/Wiz 5}
\shortability{Duration:}{1 min./level (D)}
\smallskip\noindent This spell functions like silent image, except that the figment includes visual, auditory, olfactory, and thermal components, and the figment follows a script determined by you. The figment follows that script without your having to concentrate on it. The illusion can include intelligible speech if you wish. 

\shortability{Material Component:}{A bit of fleece and several grains of sand.}

\featname{Phantasmal Killer}
\shortability{Illusion}{(Phantasm) [Fear, Mind-Affecting]}
\shortability{Level:}{Sor/Wiz 4}
\shortability{Components:}{V, S}
\shortability{Casting Time:}{1 standard action}
\shortability{Range:}{Medium (100 ft. + 10 ft./level)}
\shortability{Target:}{One living creature}
\shortability{Duration:}{Instantaneous}
\shortability{Saving Throw:}{Will disbelief (if interacted with), then Fortitude partial; see text}
\shortability{Spell Resistance:}{Yes}
\smallskip\noindent You create a phantasmal image of the most fearsome creature imaginable to the subject simply by forming the fears of the subject's subconscious mind into something that its conscious mind can visualize: this most horrible beast. Only the spell's subject can see the phantasmal killer. You see only a vague shape. The target first gets a Will save to recognize the image as unreal. If that save fails, the phantasm touches the subject, and the subject must succeed on a Fortitude save or die from fear. Even if the Fortitude save is successful, the subject takes 3d6 points of damage.

\smallskip\noindent If the subject of a phantasmal killer attack succeeds in disbelieving and is wearing a helm of telepathy, the beast can be turned upon you. You must then disbelieve it or become subject to its deadly fear attack.

\featname{Phantom Steed}
\shortability{Conjuration}{(Creation)}
\shortability{Level:}{Brd 3, Sor/Wiz 3}
\shortability{Components:}{V, S}
\shortability{Casting Time:}{10 minutes}
\shortability{Range:}{0 ft.}
\shortability{Effect:}{One quasi-real, horselike creature}
\shortability{Duration:}{1 hour/level (D)}
\shortability{Saving Throw:}{None}
\shortability{Spell Resistance:}{No}
\smallskip\noindent You conjure a Large, quasi-real, horselike creature. The steed can be ridden only by you or by the one person for whom you specifically created the mount. A phantom steed has a black head and body, gray mane and tail, and smoke-colored, insubstantial hooves that make no sound. It has what seems to be a saddle, bit, and bridle. It does not fight, but animals shun it and refuse to attack it.

\smallskip\noindent The mount has an AC of 18 (-1 size, +4 natural armor, +5 Dex) and 7 hit points +1 hit point per caster level. If it loses all its hit points, the phantom steed disappears. A phantom steed has a speed of 20 feet per caster level, to a maximum of 240 feet. It can bear its rider's weight plus up to 10 pounds per caster level.

\smallskip\noindent These mounts gain certain powers according to caster level. A mount's abilities include those of mounts of lower caster levels. 

\shortability{8th Leve:}{The mount can ride over sandy, muddy, or even swampy ground without difficulty or decrease in speed.}
\shortability{10th Leve:}{The mount can use water walk at will (as the spell, no action required to activate this ability).}
\shortability{12th Leve:}{The mount can use air walk at will (as the spell, no action required to activate this ability) for up to 1 round at a time, after which it falls to the ground.}
\shortability{14th Leve:}{The mount can fly at its speed (average maneuverability).}

\featname{Phantom Trap}
\shortability{Illusion}{(Glamer)}
\shortability{Level:}{Sor/Wiz 2}
\shortability{Components:}{V, S, M}
\shortability{Casting Time:}{1 standard action}
\shortability{Range:}{Touch}
\shortability{Target:}{Object touched}
\shortability{Duration:}{Permanent (D)}
\shortability{Saving Throw:}{None}
\shortability{Spell Resistance:}{No}
\smallskip\noindent This spell makes a lock or other small mechanism seem to be trapped to anyone who can detect traps. You place the spell upon any small mechanism or device, such as a lock, hinge, hasp, cork, cap, or ratchet. Any character able to detect traps, or who uses any spell or device enabling trap detection, is 100\% certain a real trap exists. Of course, the effect is illusory and nothing happens if the trap is "sprung"; its primary purpose is to frighten away thieves or make them waste precious time.

\smallskip\noindent If another phantom trap is active within 50 feet when the spell is cast, the casting fails.

\shortability{Material Component:}{A piece of iron pyrite touched to the object to be trapped while the object is sprinkled with a special dust requiring 50 gp to prepare.}

\featname{Phase Door}
\shortability{Conjuration}{(Creation)}
\shortability{Level:}{Sor/Wiz 7, Travel 8}
\shortability{Components:}{V}
\shortability{Casting Time:}{1 standard action}
\shortability{Range:}{0 ft.}
\shortability{Effect:}{Ethereal 5 ft. by 8 ft. opening, 10 ft. deep + 5 ft. deep per three levels}
\shortability{Duration:}{One usage per two levels}
\shortability{Saving Throw:}{None}
\shortability{Spell Resistance:}{No}
\smallskip\noindent This spell creates an ethereal passage through wooden, plaster, or stone walls, but not other materials. The phase door is invisible and inaccessible to all creatures except you, and only you can use the passage. You disappear when you enter the phase door and appear when you exit. If you desire, you can take one other creature (Medium or smaller) through the door. This counts as two uses of the door. The door does not allow light, sound, or spell effects through it, nor can you see through it without using it. Thus, the spell can provide an escape route, though certain creatures, such as phase spiders, can follow with ease. A gem of true seeing or similar magic reveals the presence of a phase door but does not allow its use.

\smallskip\noindent A phase door is subject to dispel magic. If anyone is within the passage when it is dispelled, he is harmlessly ejected just as if he were inside a passwall effect.

\smallskip\noindent You can allow other creatures to use the phase door by setting some triggering condition for the door. Such conditions can be as simple or elaborate as you desire. They can be based on a creature's name, identity, or alignment, but otherwise must be based on observable actions or qualities. Intangibles such as level, class, Hit Dice, and hit points don't qualify.

\smallskip\noindent Phase door can be made permanent with a permanency spell.

\featname{Planar Ally}
\shortability{Conjuration}{(Calling) [see text for lesser planar ally]}
\shortability{Level:}{Clr 6}
\shortability{Effect:}{One or two called elementals or outsiders, totaling no more than 12 HD, which cannot be more than 30 ft. apart when they appear}
\smallskip\noindent This spell functions like lesser planar ally, except you may call a single creature of 12 HD or less, or two creatures of the same kind whose Hit Dice total no more than 12. The creatures agree to help you and request your return payment together.

\shortability{XP Cost:}{250 XP.}

\featname{Planar Ally, Greater}
\shortability{Conjuration}{(Calling) [see text for lesser planar ally]}
\shortability{Level:}{Clr 8}
\shortability{Effect:}{Up to three called elementals or outsiders, totaling no more than 18 HD, no two of which can be more than 30 ft. apart when they appear.}
\smallskip\noindent This spell functions like lesser planar ally, except that you may call a single creature of 18 HD or less, or up to three creatures of the same kind whose Hit Dice total no more than 18. The creatures agree to help you and request your return payment together.

\shortability{XP Cost:}{500 XP.}

\featname{Planar Ally, Lesser}
\shortability{Conjuration}{(Calling) [see text]}
\shortability{Level:}{Clr 4}
\shortability{Components:}{V, S, DF, XP}
\shortability{Casting Time:}{10 minutes}
\shortability{Range:}{Close (25 ft. + 5 ft./2 levels)}
\shortability{Effect:}{One called elemental or outsider of 6 HD or less}
\shortability{Duration:}{Instantaneous}
\shortability{Saving Throw:}{None}
\shortability{Spell Resistance:}{No}
\smallskip\noindent By casting this spell, you request your deity to send you an elemental or outsider (of 6 HD or less) of the deity's choice. If you serve no particular deity, the spell is a general plea answered by a creature sharing your philosophical alignment. If you know an individual creature's name, you may request that individual by speaking the name during the spell (though you might get a different creature anyway).

\smallskip\noindent You may ask the creature to perform one task in exchange for a payment from you. Tasks might range from the simple to the complex. You must be able to communicate with the creature called in order to bargain for its services.

\smallskip\noindent The creature called requires a payment for its services. This payment can take a variety of forms, from donating gold or magic items to an allied temple, to a gift given directly to the creature, to some other action on your part that matches the creature's alignment and goals. Regardless, this payment must be made before the creature agrees to perform any services. The bargaining takes at least 1 round, so any actions by the creature begin in the round after it arrives.

\smallskip\noindent A task taking up to 1 minute per caster level requires a payment of 100 gp per HD of the creature called. For a task taking up to 1 hour per caster level, the creature requires a payment of 500 gp per HD. A long-term task, one requiring up to one day per caster level, requires a payment of 1,000 gp per HD.

\smallskip\noindent A nonhazardous task requires only half the indicated payment, while an especially hazardous task might require a greater gift. Few if any creatures will accept a task that seems suicidal (remember, a called creature actually dies when it is killed, unlike a summoned creature). However, if the task is strongly aligned with the creature's ethos, it may halve or even waive the payment. 

\smallskip\noindent At the end of its task, or when the duration bargained for expires, the creature returns to its home plane (after reporting back to you, if appropriate and possible).

\shortability{Note:}{When you use a calling spell that calls an air, chaotic, earth, evil, fire, good, lawful, or water creature, it is a spell of that type.}
\shortability{XP Cost:}{100 XP.}

\featname{Planar Binding}
\shortability{Conjuration}{(Calling) [see text for lesser planar binding]}
\shortability{Level:}{Sor/Wiz 6}
\shortability{Components:}{V, S}
\shortability{Targets:}{Up to three elementals or outsiders, totaling no more than 12 HD, no two of which can be more than 30 ft. apart when they appear}
\smallskip\noindent This spell functions like lesser planar binding, except that you may call a single creature of 12 HD or less, or up to three creatures of the same kind whose Hit Dice total no more than 12. Each creature gets a save, makes an independent attempt to escape, and must be individually persuaded to aid you.

\featname{Planar Binding, Greater}
\shortability{Conjuration}{(Calling) [see text for lesser planar binding]}
\shortability{Level:}{Sor/Wiz 8}
\shortability{Components:}{V, S}
\shortability{Targets:}{Up to three elementals or outsiders, totaling no more than 18 HD, no two of which can be more than 30 ft. apart when they appear.}
\smallskip\noindent This spell functions like lesser planar binding, except that you may call a single creature of 18 HD or less, or up to three creatures of the same kind whose Hit Dice total no more than 18. Each creature gets a saving throw, makes independent attempts to escape, and must be persuaded to aid you individually.

\featname{Planar Binding, Lesser}
\shortability{Conjuration}{(Calling) [see text]}
\shortability{Level:}{Sor/Wiz 5}
\shortability{Components:}{V, S}
\shortability{Casting Time:}{10 minutes}
\shortability{Range:}{Close (25 ft. + 5 ft./2 levels); see text}
\shortability{Target:}{One elemental or outsider with 6 HD or less}
\shortability{Duration:}{Instantaneous}
\shortability{Saving Throw:}{Will negates}
\shortability{Spell Resistance:}{No and Yes; see text}
\smallskip\noindent Casting this spell attempts a dangerous act: to lure a creature from another plane to a specifically prepared trap, which must lie within the spell's range. The called creature is held in the trap until it agrees to perform one service in return for its freedom.

\smallskip\noindent To create the trap, you must use a magic circle spell, focused inward. The kind of creature to be bound must be known and stated. If you wish to call a specific individual, you must use that individual's proper name in casting the spell.

\smallskip\noindent The target creature is allowed a Will saving throw. If the saving throw succeeds, the creature resists the spell. If the saving throw fails, the creature is immediately drawn to the trap (spell resistance does not keep it from being called). The creature can escape from the trap with by successfully pitting its spell resistance against your caster level check, by dimensional travel, or with a successful Charisma check (DC 15 + 1/2 your caster level + your Cha modifier). It can try each method once per day. If it breaks loose, it can flee or attack you. A dimensional anchor cast on the creature prevents its escape via dimensional travel. You can also employ a calling diagram (see magic circle against evil) to make the trap more secure.

\smallskip\noindent If the creature does not break free of the trap, you can keep it bound for as long as you dare. You can attempt to compel the creature to perform a service by describing the service and perhaps offering some sort of reward. You make a Charisma check opposed by the creature's Charisma check. The check is assigned a bonus of +0 to +6 based on the nature of the service and the reward. If the creature wins the opposed check, it refuses service. New offers, bribes, and the like can be made or the old ones reoffered every 24 hours. This process can be repeated until the creature promises to serve, until it breaks free, or until you decide to get rid of it by means of some other spell. Impossible demands or unreasonable commands are never agreed to. If you roll a 1 on the Charisma check, the creature breaks free of the binding and can escape or attack you.

\smallskip\noindent Once the requested service is completed, the creature need only so inform you to be instantly sent back whence it came. The creature might later seek revenge. If you assign some open-ended task that the creature cannot complete though its own actions the spell remains in effect for a maximum of one day per caster level, and the creature gains an immediate chance to break free. Note that a clever recipient can subvert some instructions.
When you use a calling spell to call an air, chaotic, earth, evil, fire, good, lawful, or water creature, it is a spell of that type. 

\featname{Plane Shift}
\shortability{Conjuration}{(Teleportation)}
\shortability{Level:}{Clr 5, Sor/Wiz 7}
\shortability{Components:}{V, S, F}
\shortability{Casting Time:}{1 standard action}
\shortability{Range:}{Touch}
\shortability{Target:}{Creature touched, or up to eight willing creatures joining hands}
\shortability{Duration:}{Instantaneous}
\shortability{Saving Throw:}{Will negates}
\shortability{Spell Resistance:}{Yes}

\smallskip\noindent You move yourself or some other creature to another plane of existence or alternate dimension. If several willing persons link hands in a circle, as many as eight can be affected by the plane shift at the same time. Precise accuracy as to a particular arrival location on the intended plane is nigh impossible. From the Material Plane, you can reach any other plane, though you appear 5 to 500 miles (5d\%) from your intended destination.

\shortability{Note:}{Plane shift transports creatures instantaneously and then ends. The creatures need to find other means if they are to travel back.}
\shortability{Focus:}{A small, forked metal rod. The size and metal type dictates to which plane of existence or alternate dimension the spell sends the affected creatures. }

\featname{Plant Growth}
\shortability{Transmutation}{}
\shortability{Level:}{Drd 3, Plant 3, Rgr 3}
\shortability{Components:}{V, S, DF}
\shortability{Casting Time:}{1 standard action}
\shortability{Range:}{See text}
Target or \shortability{Area:}{See text}
\shortability{Duration:}{Instantaneous}
\shortability{Saving Throw:}{None}
\shortability{Spell Resistance:}{No}
\smallskip\noindent Plant growth has different effects depending on the version chosen.

\shortability{Overgrowth:}{This effect causes normal vegetation (grasses, briars, bushes, creepers, thistles, trees, vines) within long range (400 feet + 40 feet per caster level) to become thick and overgrown. The plants entwine to form a thicket or jungle that creatures must hack or force a way through. Speed drops to 5 feet, or 10 feet for Large or larger creatures. The area must have brush and trees in it for this spell to take effect.

\smallskip\noindent At your option, the area can be a 100-foot-radius circle, a 150-foot-radius semicircle, or a 200-foot-radius quarter circle.

\smallskip\noindent You may designate places within the area that are not affected.}

\shortability{Enrichment:}{This effect targets plants within a range of one-half mile, raising their potential productivity over the course of the next year to one-third above normal.

\smallskip\noindent Plant growth counters diminish plants.}

\smallskip\noindent This spell has no effect on plant creatures.

\featname{Poison}
\shortability{Necromancy}{}
\shortability{Level:}{Clr 4, Drd 3}
\shortability{Components:}{V, S, DF}
\shortability{Casting Time:}{1 standard action}
\shortability{Range:}{Touch}
\shortability{Target:}{Living creature touched}
\shortability{Duration:}{Instantaneous; see text}
\shortability{Saving Throw:}{Fortitude negates; see text}
\shortability{Spell Resistance:}{Yes}
\smallskip\noindent Calling upon the venomous powers of natural predators, you infect the subject with a horrible poison by making a successful melee touch attack. The poison deals 1d10 points of temporary Constitution damage immediately and another 1d10 points of temporary Constitution damage 1 minute later. Each instance of damage can be negated by a Fortitude save (DC 10 + 1/2 your caster level + your Wis modifier).

\featname{Polar Ray}
\shortability{Evocation}{[Cold]}
\shortability{Level:}{Sor/Wiz 8}
\shortability{Components:}{V, S, F}
\shortability{Casting Time:}{1 standard action}
\shortability{Range:}{Close (25 ft. + 5 ft./2 levels)}
\shortability{Effect:}{Ray}
\shortability{Duration:}{Instantaneous}
\shortability{Saving Throw:}{None}
\shortability{Spell Resistance:}{Yes}
\smallskip\noindent A blue-white ray of freezing air and ice springs from your hand. You must succeed on a ranged touch attack with the ray to deal damage to a target. The ray deals 1d6 points of cold damage per caster level (maximum 25d6).

\shortability{Focus:}{A small, white ceramic cone or prism.}

\featname{Polymorph}
\shortability{Transmutation}{}
\shortability{Level:}{Sor/Wiz 4}
\shortability{Components:}{V, S, M}
\shortability{Casting Time:}{1 standard action}
\shortability{Range:}{Touch}
\shortability{Target:}{Willing living creature touched}
\shortability{Duration:}{1 min./level (D)}
\shortability{Saving Throw:}{None}
\shortability{Spell Resistance:}{No}
\smallskip\noindent This spell functions like alter self, except that you change the willing subject into another form of living creature. The new form may be of the same type as the subject or any of the following types: aberration, animal, dragon, fey, giant, humanoid, magical beast, monstrous humanoid, ooze, plant, or vermin. The assumed form can't have more Hit Dice than your caster level (or the subject's HD, whichever is lower), to a maximum of 15 HD at 15th level. You can't cause a subject to assume a form smaller than Fine, nor can you cause a subject to assume an incorporeal or gaseous form. The subject's creature type and subtype (if any) change to match the new form.

\smallskip\noindent Upon changing, the subject regains lost hit points as if it had rested for a night (though this healing does not restore temporary ability damage and provide other benefits of resting; and changing back does not heal the subject further). If slain, the subject reverts to its original form, though it remains dead.

\smallskip\noindent The subject gains the Strength, Dexterity, and Constitution scores of the new form but retains its own Intelligence, Wisdom, and Charisma scores. It also gains all extraordinary special attacks possessed by the form but does not gain the extraordinary special qualities possessed by the new form or any supernatural or spell-like abilities.

\smallskip\noindent Incorporeal or gaseous creatures are immune to being polymorphed, and a creature with the shapechanger subtype can revert to its natural form as a standard action.

\shortability{Material Component:}{An empty cocoon.}

\featname{Polymorph Any Object}
\shortability{Transmutation}{}
\shortability{Level:}{Sor/Wiz 8, Trickery 8}
\shortability{Components:}{V, S, M/DF}
\shortability{Casting Time:}{1 standard action}
\shortability{Range:}{Close (25 ft. + 5 ft./2 levels)}
\shortability{Target:}{One creature, or one nonmagical object of up to 100 cu. ft./level}
\shortability{Duration:}{See text}
\shortability{Saving Throw:}{Fortitude negates (object); see text}
\shortability{Spell Resistance:}{Yes (object)}
\smallskip\noindent This spell functions like polymorph, except that it changes one object or creature into another. The duration of the spell depends on how radical a change is made from the original state to its enchanted state. The duration is determined by using the following guidelines.

\begin{tabular}[h!]{l|l}
Changed Subject Is: & Increase to Duration Factor\textsuperscript{1} \\ \hline
Same kingdom (animal, vegetable, mineral) & +5 \\
Same class (mammals, fungi, metals, etc.) & +2 \\
Same size & +2 \\
Related (twig is to tree, wolf fur is to wolf, etc.) & +2	 \\
Same or lower Intelligence & +2 \\
\multicolumn{2}{p{3in}{\textsuperscript{1} Add all that apply. Look up the total on the next table.}} \\
\end{tabular}
  
\begin{tabular}[h!]{l|ll}
Duration Factor & Duration & Example \\
0 & 20 minutes & Pebble to human \\
2 & 1 hour & Marionette to human \\
4 & 3 hours & Human to marionette	\\
5 & 12 hours & Lizard to manticore \\
6 & 2 days & Sheep to wool coat	\\
7 & 1 week & Shrew to manticore	\\
9+& Permanent & Manticore to shrew \\
\end{tabular}

\smallskip\noindent Unlike polymorph, polymorph any object does grant the creature the Intelligence score of its new form. If the original form didn't have a Wisdom or Charisma score, it gains those scores as appropriate for the new form.

\smallskip\noindent Damage taken by the new form can result in the injury or death of the polymorphed creature. In general, damage occurs when the new form is changed through physical force.

\smallskip\noindent A nonmagical object cannot be made into a magic item with this spell. Magic items aren't affected by this spell.

\smallskip\noindent This spell cannot create material of great intrinsic value, such as copper, silver, gems, silk, gold, platinum, mithral, or adamantine. It also cannot reproduce the special properties of cold iron in order to overcome the damage reduction of certain creatures.

\smallskip\noindent This spell can also be used to duplicate the effects of baleful polymorph, polymorph, flesh to stone, stone to flesh, transmute mud to rock, transmute metal to wood, or transmute rock to mud.

\shortability{Arcane Material Component:}{Mercury, gum arabic, and smoke.}

\featname{Power Word Blind}
\shortability{Enchantment}{(Compulsion) [Mind-Affecting]}
\shortability{Level:}{Sor/Wiz 7, War 7}
\shortability{Components:}{V}
\shortability{Casting Time:}{1 standard action}
\shortability{Range:}{Close (25 ft. + 5 ft./2 levels)}
\shortability{Target:}{One creature with 200 hp or less}
\shortability{Duration:}{See text}
\shortability{Saving Throw:}{None}
\shortability{Spell Resistance:}{Yes}
\smallskip\noindent You utter a single word of power that causes one creature of your choice to become blinded, whether the creature can hear the word or not. The duration of the spell depends on the target's current hit point total. Any creature that currently has 201 or more hit points is unaffected by power word blind.

\begin{tabular}[h!]{l|l}
Hit Points & Duration	\\ \hline
50 or less & Permanent \\
51-100 & 1d4+1 minutes \\
101-200 & 1d4+1 rounds \\
\end{tabular}

\featname{Power Word Kill}
\shortability{Enchantment}{(Compulsion) [Death, Mind-Affecting]}
\shortability{Level:}{Sor/Wiz 9, War 9}
\shortability{Components:}{V}
\shortability{Casting Time:}{1 standard action}
\shortability{Range:}{Close (25 ft. + 5 ft./2 levels)}
\shortability{Target:}{One living creature with 100 hp or less}
\shortability{Duration:}{Instantaneous}
\shortability{Saving Throw:}{None}
\shortability{Spell Resistance:}{Yes}
\smallskip\noindent You utter a single word of power that instantly kills one creature of your choice, whether the creature can hear the word or not. Any creature that currently has 101 or more hit points is unaffected by power word kill.

\featname{Power Word Stun}
\shortability{Enchantment}{(Compulsion) [Mind-Affecting]}
\shortability{Level:}{Sor/Wiz 8, War 8}
\shortability{Components:}{V}
\shortability{Casting Time:}{1 standard action}
\shortability{Range:}{Close (25 ft. + 5 ft./2 levels)}
\shortability{Target:}{One creature with 150 hp or less}
\shortability{Duration:}{See text}
\shortability{Saving Throw:}{None}
\shortability{Spell Resistance:}{Yes}
\smallskip\noindent You utter a single word of power that instantly causes one creature of your choice to become stunned, whether the creature can hear the word or not. The duration of the spell depends on the target's current hit point total. Any creature that currently has 151 or more hit points is unaffected by power word stun.

\begin{tabular}[h!]{l|l}
Hit Points & Duration	\\ \hline
50 or less & 4d4 rounds \\
51-100 & 2d4 rounds	\\
101-150 & 1d4 rounds \\
\end{tabular}

\featname{Prayer}
\shortability{Enchantment}{(Compulsion) [Mind-Affecting]}
\shortability{Level:}{Clr 3, Pal 3}
\shortability{Components:}{V, S, DF}
\shortability{Casting Time:}{1 standard action}
\shortability{Range:}{40 ft.}
\shortability{Area:}{All allies and foes within a 40-ft.-radius burst centered on you}
\shortability{Duration:}{1 round/level}
\shortability{Saving Throw:}{None}
\shortability{Spell Resistance:}{Yes}
\smallskip\noindent You bring special favor upon yourself and your allies while bringing disfavor to your enemies. You and your each of your allies gain a +1 luck bonus on attack rolls, weapon damage rolls, saves, and skill checks, while each of your foes takes a -1 penalty on such rolls.

\featname{Prestidigitation}
\shortability{Universa}{}
\shortability{Level:}{Brd 0, Sor/Wiz 0}
\shortability{Components:}{V, S}
\shortability{Casting Time:}{1 standard action}
\shortability{Range:}{10 ft.}
\shortability{Target, Effect, or Area:}{See text}
\shortability{Duration:}{1 hour}
\shortability{Saving Throw:}{See text}
\shortability{Spell Resistance:}{No}
\smallskip\noindent Prestidigitations are minor tricks that novice spellcasters use for practice. Once cast, a prestidigitation spell enables you to perform simple magical effects for 1 hour. The effects are minor and have severe limitations. A prestidigitation can slowly lift 1 pound of material. It can color, clean, or soil items in a 1-foot cube each round. It can chill, warm, or flavor 1 pound of nonliving material. It cannot deal damage or affect the concentration of spellcasters. Prestidigitation can create small objects, but they look crude and artificial. The materials created by a prestidigitation spell are extremely fragile, and they cannot be used as tools, weapons, or spell components. Finally, a prestidigitation lacks the power to duplicate any other spell effects. Any actual change to an object (beyond just moving, cleaning, or soiling it) persists only 1 hour.

\featname{Prismatic Sphere}
\shortability{Abjuration}{}
\shortability{Level:}{Protection 9, Sor/Wiz 9, Sun 9}
\shortability{Components:}{V}
\shortability{Range:}{10 ft.}
\shortability{Effect:}{10-ft.-radius sphere centered on you}
\smallskip\noindent This spell functions like prismatic wall, except you conjure up an immobile, opaque globe of shimmering, multicolored light that surrounds you and protects you from all forms of attack. The sphere flashes in all colors of the visible spectrum. 

\smallskip\noindent The sphere's blindness effect on creatures with less than 8 HD lasts 2d4x10 minutes.

\smallskip\noindent You can pass into and out of the prismatic sphere and remain near it without harm. However, when you're inside it, the sphere blocks any attempt to project something through the sphere (including spells). Other creatures that attempt to attack you or pass through suffer the effects of each color, one at a time.

\smallskip\noindent Typically, only the upper hemisphere of the globe will exist, since you are at the center of the sphere, so the lower half is usually excluded by the floor surface you are standing on.

\smallskip\noindent The colors of the sphere have the same effects as the colors of a prismatic wall.

\smallskip\noindent Prismatic sphere can be made permanent with a permanency spell.

\featname{Prismatic Spray}
\shortability{Evocatio}{}
\shortability{Level:}{Sor/Wiz 7}
\shortability{Components:}{V, S}
\shortability{Casting Time:}{1 standard action}
\shortability{Range:}{60 ft.}
\shortability{Area:}{Cone-shaped burst}
\shortability{Duration:}{Instantaneous}
\shortability{Saving Throw:}{See text}
\shortability{Spell Resistance:}{Yes}

\smallskip\noindent This spell causes seven shimmering, intertwined, multicolored beams of light to spray from your hand. Each beam has a different power. Creatures in the area of the spell with 8 HD or less are automatically blinded for 2d4 rounds. Every creature in the area is randomly struck by one or more beams, which have additional effects.

\begin{tabular}[h!]{l|l|p{2in}}
1d8 & Color of Beam & Effect \\
1 & Red & 20 points fire damage (Reflex half) \\
2 & Orange & 40 points acid damage (Reflex half) \\
3 & Yellow & 80 points electricity damage (Reflex half) \\	   
4 & Green & Poison (Kills; Fortitude partial, take 1d6 points of Con damage instead) \\
5 & Blue & Turned to stone (Fortitude negates) \\	   
6 & Indigo & Insane, as insanity spell (Will negates) \\
7 & Violet & Sent to another plane (Will negates) \\
8 & -- & Struck by two rays; roll twice more, ignoring any ``8'' results. \\
\end{tabular}

\featname{Prismatic Wall}
\shortability{Abjuration}{}
\shortability{Level:}{Sor/Wiz 8}
\shortability{Components:}{V, S}
\shortability{Casting Time:}{1 standard action}
\shortability{Range:}{Close (25 ft. + 5 ft./2 levels)}
\shortability{Effect}{Wall 4 ft./level wide, 2 ft./level high}
\shortability{Duration:}{10 min./level (D)}
\shortability{Saving Throw:}{See text}
\shortability{Spell Resistance:}{See text}
\smallskip\noindent Prismatic wall creates a vertical, opaque wall-a shimmering, multicolored plane of light that protects you from all forms of attack. The wall flashes with seven colors, each of which has a distinct power and purpose. The wall is immobile, and you can pass through and remain near the wall without harm. However, any other creature with less than 8 HD that is within 20 feet of the wall is blinded for 2d4 rounds by the colors if it looks at the wall.

\smallskip\noindent The wall's maximum proportions are 4 feet wide per caster level and 2 feet high per caster level. A prismatic wall spell cast to materialize in a space occupied by a creature is disrupted, and the spell is wasted.

\smallskip\noindent Each color in the wall has a special effect. The accompanying table shows the seven colors of the wall, the order in which they appear, their effects on creatures trying to attack you or pass through the wall, and the magic needed to negate each color.

\smallskip\noindent The wall can be destroyed, color by color, in consecutive order, by various magical effects; however, the first color must be brought down before the second can be affected, and so on. A rod of cancellation or a mage's disjunction spell destroys a prismatic wall, but an antimagic field fails to penetrate it. Dispel magic and greater dispel magic cannot dispel the wall or anything beyond it. Spell resistance is effective against a prismatic wall, but the caster level check must be repeated for each color present.

\smallskip\noindent Prismatic wall can be made permanent with a permanency spell.

\begin{tabular}[h!]{p{.25in}|p{.25in}|p{2in}|p{.5in}}
Color  & Order & Effect of Color & Negated By	\\ \hline
Red    & 1st & Stops nonmagical ranged weapons. Deals 20 points of fire damage (Reflex half). & Cone of cold \\
Orange & 2nd & Stops magical ranged weapons. Deals 40 points of acid damage (Reflex half). & Gust of wind	\\
Yellow & 3rd & Stops poisons, gases, and petrification. Deals 80 points of electricity damage (Reflex half). & Disintegrate	\\
Green  & 4th & Stops breath weapons. Poison (Kills; Fortitude partial for 1d6 points of Con damage instead). & Passwall	\\
Blue   & 5th & Stops divination and mental attacks. Turned to stone (Fortitude negates). & Magic missile \\
Indigo & 6th & Stops all spells. Will save or become insane (as insanity spell). & Daylight	\\
Violet & 7th & Energy field destroys all objects and effects.1 Creatures sent to another plane (Will negates). & Dispel magic	\\
\multicolumn{4}{p{3in}}{\textsuperscript{1} The violet effect makes the special effects of the other six colors redundant, but these six effects are included here because certain magic items can create prismatic effects one color at a time, and spell resistance might render some colors ineffective (see above).} \\
\end{tabular}

\featname{Produce Flame}
\shortability{Evocation}{[Fire]}
\shortability{Level:}{Drd 1, Fire 2}
\shortability{Components:}{V, S}
\shortability{Casting Time:}{1 standard action}
\shortability{Range:}{0 ft.}
\shortability{Effect:}{Flame in your palm}
\shortability{Duration:}{1 min./level (D)}
\shortability{Saving Throw:}{None}
\shortability{Spell Resistance:}{Yes}
\smallskip\noindent Flames as bright as a torch appear in your open hand. The flames harm neither you nor your equipment.

\smallskip\noindent In addition to providing illumination, the flames can be hurled or used to touch enemies. You can strike an opponent with a melee touch attack, dealing fire damage equal to 1d6 +1 point per caster level (maximum +5). Alternatively, you can hurl the flames up to 120 feet as a thrown weapon. When doing so, you attack with a ranged touch attack (with no range penalty) and deal the same damage as with the melee attack. No sooner do you hurl the flames than a new set appears in your hand. Each attack you make reduces the remaining duration by 1 minute. If an attack reduces the remaining duration to 0 minutes or less, the spell ends after the attack resolves.

\smallskip\noindent This spell does not function underwater.

\featname{Programmed Image}
\shortability{Illusion}{(Figment)}
\shortability{Level:}{Brd 6, Sor/Wiz 6}
\shortability{Effect:}{Visual figment that cannot extend beyond a 20-ft. cube + one 10-ft. cube/level (S)}
\shortability{Duration:}{Permanent until triggered, then 1 round/level}
\smallskip\noindent This spell functions like silent image, except that this spell's figment activates when a specific condition occurs. The figment includes visual, auditory, olfactory, and thermal elements, including intelligible speech.

\smallskip\noindent You set the triggering condition (which may be a special word) when casting the spell. The event that triggers the illusion can be as general or as specific and detailed as desired but must be based on an audible, tactile, olfactory, or visual trigger. The trigger cannot be based on some quality not normally obvious to the senses, such as alignment. (See magic mouth for more details about such triggers.)

\shortability{Material Component:}{A bit of fleece and jade dust worth 25 gp.}

\featname{Project Image}
\shortability{Illusion}{(Shadow)}
\shortability{Level:}{Brd 6, Sor/Wiz 7}
\shortability{Components:}{V, S, M}
\shortability{Casting Time:}{1 standard action}
\shortability{Range:}{Medium (100 ft. + 10 ft./level)}
\shortability{Effect:}{One shadow duplicate}
\shortability{Duration:}{1 round/level (D)}
\shortability{Saving Throw:}{Will disbelief (if interacted with)}
\shortability{Spell Resistance:}{No}
\smallskip\noindent You tap energy from the Plane of Shadow to create a quasi-real, illusory version of yourself. The projected image looks, sounds, and smells like you but is intangible. The projected image mimics your actions (including speech) unless you direct it to act differently (which is a move action).

\smallskip\noindent You can see through its eyes and hear through its ears as if you were standing where it is, and during your turn you can switch from using its senses to using your own, or back again, as a free action. While you are using its senses, your body is considered blinded and deafened.

\smallskip\noindent If you desire, any spell you cast whose range is touch or greater can originate from the projected image instead of from you. The projected image can't cast any spells on itself except for illusion spells. The spells affect other targets normally, despite originating from the projected image.

\smallskip\noindent Objects are affected by the projected image as if they had succeeded on their Will save.

\smallskip\noindent You must maintain line of effect to the projected image at all times. If your line of effect is obstructed, the spell ends. If you use dimension door, teleport, plane shift, or a similar spell that breaks your line of effect, even momentarily, the spell ends.

\shortability{Material Component:}{A small replica of you (a doll), which costs 5 gp to create.}

\featname{Protection from Arrows}
\shortability{Abjuration}{}
\shortability{Level:}{Sor/Wiz 2}
\shortability{Components:}{V, S, F}
\shortability{Casting Time:}{1 standard action}
\shortability{Range:}{Touch}
\shortability{Target:}{Creature touched}
\shortability{Duration:}{1 hour/level or until discharged}
\shortability{Saving Throw:}{Will negates (harmless)}
\shortability{Spell Resistance:}{Yes (harmless)}
\smallskip\noindent The warded creature gains resistance to ranged weapons. The subject gains damage reduction 10/magic against ranged weapons. (This spell doesn't grant you the ability to damage creatures with similar damage reduction.) Once the spell has prevented a total of 10 points of damage per caster level (maximum 100 points), it is discharged.

\shortability{Focus:}{A piece of shell from a tortoise or a turtle.}

\featname{Protection from Chaos}
\shortability{Abjuration}{[Lawful]}
\shortability{Level:}{Clr 1, Law 1, Pal 1, Sor/Wiz 1}
\smallskip\noindent This spell functions like protection from evil, except that the deflection and resistance bonuses apply to attacks from chaotic creatures, and chaotic summoned creatures cannot touch the subject.

\featname{Protection from Energy}
\shortability{Abjuration}{}
\shortability{Level:}{Clr 3, Drd 3, Luck 3, Protection 3, Rgr 2, Sor/Wiz 3}
\shortability{Components:}{V, S, DF}
\shortability{Casting Time:}{1 standard action}
\shortability{Range:}{Touch}
\shortability{Target:}{Creature touched}
\shortability{Duration:}{10 min./level or until discharged}
\shortability{Saving Throw:}{Fortitude negates (harmless)}
\shortability{Spell Resistance:}{Yes (harmless)}
\smallskip\noindent Protection from energy grants temporary immunity to the type of energy you specify when you cast it (acid, cold, electricity, fire, or sonic). When the spell absorbs 12 points per caster level of energy damage (to a maximum of 120 points at 10th level), it is discharged.

\shortability{Note:}{Protection from energy overlaps (and does not stack with) resist energy. If a character is warded by protection from energy and resist energy, the protection spell absorbs damage until its power is exhausted.}

\featname{Protection from Evil}
\shortability{Abjuration}{[Good]}
\shortability{Level:}{Clr 1, Good 1, Pal 1, Sor/Wiz 1}
\shortability{Components:}{V, S, M/DF}
\shortability{Casting Time:}{1 standard action}
\shortability{Range:}{Touch}
\shortability{Target:}{Creature touched}
\shortability{Duration:}{1 min./level (D)}
\shortability{Saving Throw:}{Will negates (harmless)}
\shortability{Spell Resistance:}{No; see text}
\smallskip\noindent This spell wards a creature from attacks by evil creatures, from mental control, and from summoned creatures. It creates a magical barrier around the subject at a distance of 1 foot. The barrier moves with the subject and has three major effects.

\smallskip\noindent First, the subject gains a +2 deflection bonus to AC and a +2 resistance bonus on saves. Both these bonuses apply against attacks made or effects created by evil creatures.

\smallskip\noindent Second, the barrier blocks any attempt to possess the warded creature (by a magic jar attack, for example) or to exercise mental control over the creature (including enchantment (charm) effects and enchantment (compulsion) effects that grant the caster ongoing control over the subject, such as dominate person). The protection does not prevent such effects from targeting the protected creature, but it suppresses the effect for the duration of the protection from evil effect. If the protection from evil effect ends before the effect granting mental control does, the would-be controller would then be able to mentally command the controlled creature. Likewise, the barrier keeps out a possessing life force but does not expel one if it is in place before the spell is cast. This second effect works regardless of alignment.

\smallskip\noindent Third, the spell prevents bodily contact by summoned creatures. This causes the natural weapon attacks of such creatures to fail and the creatures to recoil if such attacks require touching the warded creature. Good summoned creatures are immune to this effect. The protection against contact by summoned creatures ends if the warded creature makes an attack against or tries to force the barrier against the blocked creature. Spell resistance can allow a creature to overcome this protection and touch the warded creature.

\shortability{Arcane Material Component:}{A little powdered silver with which you trace a 3-foot -diameter circle on the floor (or ground) around the creature to be warded.}

\featname{Protection from Good}
\shortability{Abjuration}{[Evil]}
\shortability{Level:}{Clr 1, Evil 1, Sor/Wiz 1}
\smallskip\noindent This spell functions like protection from evil, except that the deflection and resistance bonuses apply to attacks from good creatures, and good summoned creatures cannot touch the subject.

\featname{Protection from Law}
\shortability{Abjuration}{[Chaotic]}
\shortability{Level:}{Chaos 1, Clr 1, Sor/Wiz 1}
\smallskip\noindent This spell functions like protection from evil, except that the deflection and resistance bonuses apply to attacks from lawful creatures, and lawful summoned creatures cannot touch the subject.

\featname{Protection from Spells}
\shortability{Abjuration}{}
\shortability{Level:}{Magic 8, Sor/Wiz 8}
\shortability{Components:}{V, S, M, F}
\shortability{Casting Time:}{1 standard action}
\shortability{Range:}{Touch}
\shortability{Targets:}{Up to one creature touched per four levels}
\shortability{Duration:}{10 min./level}
\shortability{Saving Throw:}{Will negates (harmless)}
\shortability{Spell Resistance:}{Yes (harmless)}
\smallskip\noindent The subject gains a +8 resistance bonus on saving throws against spells and spell-like abilities (but not against supernatural and extraordinary abilities).

\shortability{Material Component:}{A diamond of at least 500 gp value, which must be crushed and sprinkled over the targets.}
\shortability{Focus:}{One 1,000 gp diamond per creature to be granted the protection. Each subject must carry one such gem for the duration of the spell. If a subject loses the gem, the spell ceases to affect him.}

\featname{Prying Eyes}
\shortability{Divination}{}
\shortability{Level:}{Sor/Wiz 5}
\shortability{Components:}{V, S, M}
\shortability{Casting Time:}{1 minute}
\shortability{Range:}{One mile}
\shortability{Effect:}{Ten or more levitating eyes}
\shortability{Duration:}{1 hour/level; see text (D)}
\shortability{Saving Throw:}{None}
\shortability{Spell Resistance:}{No}
\smallskip\noindent You create a number of semitangible, visible magical orbs (called "eyes") equal to 1d4 + your caster level. These eyes move out, scout around, and return as you direct them when casting the spell. Each eye can see 120 feet (normal vision only) in all directions.

\smallskip\noindent While the individual eyes are quite fragile, they're small and difficult to spot. Each eye is a Fine construct, about the size of a small apple, that has 1 hit point, AC 18 (+8 bonus for its size), flies at a speed of 30 feet with perfect maneuverability, and has a +16 Hide modifier. It has a Spot modifier equal to your caster level (maximum +15) and is subject to illusions, darkness, fog, and any other factors that would affect your ability to receive visual information about your surroundings. An eye traveling through darkness must find its way by touch.

\smallskip\noindent When you create the eyes, you specify instructions you want them to follow in a command of no more than twenty-five words. Any knowledge you possess is known by the eyes as well.

\smallskip\noindent In order to report their findings, the eyes must return to your hand. Each replays in your mind all it has seen during its existence. It takes an eye 1 round to replay 1 hour of recorded images. After relaying its findings, an eye disappears. 

\smallskip\noindent If an eye ever gets more than 1 mile away from you, it instantly ceases to exist. However, your link with the eye is such that you won't know if the eye was destroyed because it wandered out of range or because of some other event.

\smallskip\noindent The eyes exist for up to 1 hour per caster level or until they return to you. Dispel magic can destroy eyes. Roll separately for each eye caught in an area dispel. Of course, if an eye is sent into darkness, it could hit a wall or similar obstacle and destroy itself.

\shortability{Material Component:}{A handful of crystal marbles.}

\featname{Prying Eyes, Greater}
\shortability{Divination}{}
\shortability{Level:}{Sor/Wiz 8}
This spell functions like prying eyes, except that the eyes can see all things as they actually are, just as if they had true seeing with a range of 120 feet. Thus, they can navigate darkened areas at full normal speed. Also, a greater prying eye's maximum Spot modifier is +25 instead of +15.

\featname{Purify Food and Drink}
\shortability{Transmutation}{}
\shortability{Level:}{Clr 0, Drd 0}
\shortability{Components:}{V, S}
\shortability{Casting Time:}{1 standard action}
\shortability{Range:}{10 ft.}
\shortability{Target:}{1 cu. ft./level of contaminated food and water}
\shortability{Duration:}{Instantaneous}
\shortability{Saving Throw:}{Will negates (object)}
\shortability{Spell Resistance:}{Yes (object)}
\smallskip\noindent This spell makes spoiled, rotten, poisonous, or otherwise contaminated food and water pure and suitable for eating and drinking. This spell does not prevent subsequent natural decay or spoilage. Unholy water and similar food and drink of significance is spoiled by purify food and drink, but the spell has no effect on creatures of any type nor upon magic potions.

\shortability{Note:}{Water weighs about 8 pounds per gallon. One cubic foot of water contains roughly 8 gallons and weighs about 60 pounds.}

\featname{Pyrotechnics}
\shortability{Transmutation}{}
\shortability{Level:}{Brd 2, Sor/Wiz 2}
\shortability{Components:}{V, S, M}
\shortability{Casting Time:}{1 standard action}
\shortability{Range:}{Long (400 ft. + 40 ft./level)}
\shortability{Target:}{One fire source, up to a 20-ft. cube}
\shortability{Duration:}{1d4+1 rounds, or 1d4+1 rounds after creatures leave the smoke cloud; see text}
\shortability{Saving Throw:}{Will negates or Fortitude negates; see text}
\shortability{Spell Resistance:}{Yes or No; see text}
\smallskip\noindent Pyrotechnics turns a fire into either a burst of blinding fireworks or a thick cloud of choking smoke, depending on the version you choose.

\shortability{Fireworks:}{The fireworks are a flashing, fiery, momentary burst of glowing, colored aerial lights. This effect causes creatures within 120 feet of the fire source to become blinded for 1d4+1 rounds (Will negates). These creatures must have line of sight to the fire to be affected. Spell resistance can prevent blindness.}
\shortability{Smoke Cloud:}{A writhing stream of smoke billows out from the source, forming a choking cloud. The cloud spreads 20 feet in all directions and lasts for 1 round per caster level. All sight, even darkvision, is ineffective in or through the cloud. All within the cloud take -4 penalties to Strength and Dexterity (Fortitude negates). These effects last for 1d4+1 rounds after the cloud dissipates or after the creature leaves the area of the cloud. Spell resistance does not apply.}
\shortability{Material Component:}{The spell uses one fire source, which is immediately extinguished. A fire so large that it exceeds a 20-foot cube is only partly extinguished. Magical fires are not extinguished, although a fire-based creature used as a source takes 1 point of damage per caster level.}

\featname{Quench}
\shortability{Transmutation}{}
\shortability{Level:}{Drd 3}
\shortability{Components:}{V, S, DF}
\shortability{Casting Time:}{1 standard action}
\shortability{Range:}{Medium (100 ft. + 10 ft./level)}
\shortability{Area or Target:}{One 20-ft. cube/level (S) or one fire-based magic item}
\shortability{Duration:}{Instantaneous}
\shortability{Saving Throw:}{None or Will negates (object)}
\shortability{Spell Resistance:}{No or Yes (object)}
\smallskip\noindent Quench is often used to put out forest fires and other conflagrations. It extinguishes all nonmagical fires in its area. The spell also dispels any fire spells in its area, though you must succeed on a dispel check (1d20 +1 per caster level, maximum +15) against each spell to dispel it. The DC to dispel such spells is 11 + the caster level of the fire spell.

\smallskip\noindent Each elemental (fire) creature within the area of a quench spell takes 1d6 points of damage per caster level (maximum 15d6, no save allowed).

\smallskip\noindent Alternatively, you can target the spell on a single magic item that creates or controls flame. The item loses all its fire-based magical abilities for 1d4 hours unless it succeeds on a Will save. (Artifacts are immune to this effect.)

\featname{Rage}
\shortability{Enchantment}{(Compulsion) [Mind-Affecting]}
\shortability{Level:}{Brd 2, Sor/Wiz 3}
\shortability{Components:}{V, S}
\shortability{Casting Time:}{1 standard action}
\shortability{Range:}{Medium (100 ft. + 10 ft./level)}
\shortability{Targets:}{One willing living creature per three levels, no two of which may be more than 30 ft. apart}
\shortability{Duration:}{Concentration + 1 round/level (D)}
\shortability{Saving Throw:}{None}
\shortability{Spell Resistance:}{Yes}
\smallskip\noindent Each affected creature gains a +2 morale bonus to Strength and Constitution, a +1 morale bonus on Will saves, and a -2 penalty to AC. The effect is otherwise identical with a barbarian's rage except that the subjects aren't fatigued at the end of the rage.

\featname{Rainbow Pattern}
\shortability{Illusion}{(Pattern) [Mind-Affecting]}
\shortability{Level:}{Brd 4, Sor/Wiz 4}
\shortability{Components:}{V (Brd only), S, M, F; see text}
\shortability{Casting Time:}{1 standard action}
\shortability{Range:}{Medium (100 ft. + 10 ft./level)}
\shortability{Effect:}{Colorful lights with a 20-ft.-radius spread}
\shortability{Duration:}{Concentration +1 round/ level (D)}
\shortability{Saving Throw:}{Will negates}
\shortability{Spell Resistance:}{Yes}
\smallskip\noindent A glowing, rainbow-hued pattern of interweaving colors fascinates those within it. Rainbow pattern fascinates a maximum of 24 Hit Dice of creatures. Creatures with the fewest HD are affected first. Among creatures with equal HD, those who are closest to the spell's point of origin are affected first. An affected creature that fails its saves is fascinated by the pattern.

\smallskip\noindent With a simple gesture (a free action), you can make the rainbow pattern move up to 30 feet per round (moving its effective point of origin). All fascinated creatures follow the moving rainbow of light, trying to get or remain within the effect. Fascinated creatures who are restrained and removed from the pattern still try to follow it. If the pattern leads its subjects into a dangerous area each fascinated creature gets a second save. If the view of the lights is completely blocked creatures who can't see them are no longer affected.

\smallskip\noindent The spell does not affect sightless creatures.

\shortability{Verbal Component:}{A wizard or sorcerer need not utter a sound to cast this spell, but a bard must sing, play music, or recite a rhyme as a verbal component.}
\shortability{Material Component:}{A piece of phosphor.}
\shortability{Focus:}{A crystal prism.}

\featname{Raise Dead}
\shortability{Conjuration}{(Healing)}
\shortability{Level:}{Clr 5}
\shortability{Components:}{V, S, M, DF}
\shortability{Casting Time:}{1 minute}
\shortability{Range:}{Touch}
\shortability{Target:}{Dead creature touched}
\shortability{Duration:}{Instantaneous}
\shortability{Saving Throw:}{None; see text}
\shortability{Spell Resistance:}{Yes (harmless)}
\smallskip\noindent You restore life to a deceased creature. You can raise a creature that has been dead for no longer than one day per caster level. In addition, the subject's soul must be free and willing to return. If the subject's soul is not willing to return, the spell does not work; therefore, a subject that wants to return receives no saving throw.

\smallskip\noindent Coming back from the dead is an ordeal. The subject of the spell loses one level (or 1 Hit Die) when it is raised, just as if it had lost a level or a Hit Die to an energy-draining creature. If the subject is 1st level, it loses 2 points of Constitution instead (if this would reduce its Con to 0 or less, it can't be raised). This level/HD loss or Constitution loss cannot be repaired by any means. A character who died with spells prepared has a 50\% chance of losing any given spell upon being raised, in addition to losing spells for losing a level. A spellcasting creature that doesn't prepare spells (such as a sorcerer) has a 50\% chance of losing any given unused spell slot as if it had been used to cast a spell, in addition to losing spell slots for losing a level.

\smallskip\noindent A raised creature has a number of hit points equal to its current Hit Dice. Any ability scores damaged to 0 are raised to 1. Normal poison and normal disease are cured in the process of raising the subject, but magical diseases and curses are not undone. While the spell closes mortal wounds and repairs lethal damage of most kinds, the body of the creature to be raised must be whole. Otherwise, missing parts are still missing when the creature is brought back to life. None of the dead creature's equipment or possessions are affected in any way by this spell.

\smallskip\noindent A creature who has been turned into an undead creature or killed by a death effect can't be raised by this spell. Constructs, elementals, outsiders, and undead creatures can't be raised. The spell cannot bring back a creature that has died of old age.

\shortability{Material Component:}{Diamonds worth a total of least 5,000 gp.}

\featname{Ray of Enfeeblement}
\shortability{Necromancy}{}
\shortability{Level:}{Sor/Wiz 1}
\shortability{Components:}{V, S}
\shortability{Casting Time:}{1 standard action}
\shortability{Range:}{Close (25 ft. + 5 ft./2 levels)}
\shortability{Effect:}{Ray}
\shortability{Duration:}{1 min./level}
\shortability{Saving Throw:}{None}
\shortability{Spell Resistance:}{Yes}
\smallskip\noindent A coruscating ray springs from your hand. You must succeed on a ranged touch attack to strike a target. The subject takes a penalty to Strength equal to 1d6+1 per two caster levels (maximum 1d6+5). The subject's Strength score cannot drop below 1.

\featname{Ray of Exhaustion}
\shortability{Necromancy}{}
\shortability{Level:}{Sor/Wiz 3}
\shortability{Components:}{V, S, M}
\shortability{Casting Time:}{1 standard action}
\shortability{Range:}{Close (25 ft. + 5 ft./2 levels)}
\shortability{Effect:}{Ray}
\shortability{Duration:}{1 min./level}
\shortability{Saving Throw:}{Fortitude partial; see text}
\shortability{Spell Resistance:}{Yes}
\smallskip\noindent A black ray projects from your pointing finger. You must succeed on a ranged touch attack with the ray to strike a target.

\smallskip\noindent The subject is immediately exhausted for the spell's duration. A successful Fortitude save means the creature is only fatigued.

\smallskip\noindent A character that is already fatigued instead becomes exhausted.

\smallskip\noindent This spell has no effect on a creature that is already exhausted. Unlike normal exhaustion or fatigue, the effect ends as soon as the spell's duration expires.

\shortability{Material Component:}{A drop of sweat.}

\featname{Ray of Frost}
\shortability{Evocation}{[Cold]}
\shortability{Level:}{Sor/Wiz 0}
\shortability{Components:}{V, S}
\shortability{Casting Time:}{1 standard action}
\shortability{Range:}{Close (25 ft. + 5 ft./2 levels)}
\shortability{Effect:}{Ray}
\shortability{Duration:}{Instantaneous}
\shortability{Saving Throw:}{None}
\shortability{Spell Resistance:}{Yes}
\smallskip\noindent A ray of freezing air and ice projects from your pointing finger. You must succeed on a ranged touch attack with the ray to deal damage to a target. The ray deals 1d3 points of cold damage.

\featname{Read Magic}
\shortability{Divination}{}
\shortability{Level:}{Brd 0, Clr 0, Drd 0, Pal 1, Rgr 1, Sor/Wiz 0}
\shortability{Components:}{V, S, F}
\shortability{Casting Time:}{1 standard action}
\shortability{Range:}{Personal}
\shortability{Target:}{You}
\shortability{Duration:}{10 min./level}
\smallskip\noindent By means of read magic, you can decipher magical inscriptions on objects-books, scrolls, weapons, and the like-that would otherwise be unintelligible. This deciphering does not normally invoke the magic contained in the writing, although it may do so in the case of a cursed scroll. Furthermore, once the spell is cast and you have read the magical inscription, you are thereafter able to read that particular writing without recourse to the use of read magic. You can read at the rate of one page (250 words) per minute. The spell allows you to identify a glyph of warding with a DC 13 Spellcraft check, a greater glyph of warding with a DC 16 Spellcraft check, or any symbol spell with a Spellcraft check (DC 10 + spell level).

\smallskip\noindent Read magic can be made permanent with a permanency spell.

\shortability{Focus:}{A clear crystal or mineral prism.}

\featname{Reduce Animal}
\shortability{Transmutation}{}
\shortability{Level:}{Drd 2, Rgr 3}
\shortability{Components:}{V, S}
\shortability{Casting Time:}{1 standard action}
\shortability{Range:}{Touch}
\shortability{Target:}{One willing animal of Small, Medium, Large, or Huge size}
\shortability{Duration:}{1 hour/level (D)}
\shortability{Saving Throw:}{None}
\shortability{Spell Resistance:}{No}
\smallskip\noindent This spell functions like reduce person, except that it affects a single willing animal. Reduce the damage dealt by the animal's natural attacks as appropriate for its new size.

\featname{Reduce Person}
\shortability{Transmutation}{}
\shortability{Level:}{Sor/Wiz 1}
\shortability{Components:}{V, S, M}
\shortability{Casting Time:}{1 round}
\shortability{Range:}{Close (25 ft. + 5 ft./2 levels)}
\shortability{Target:}{One humanoid creature}
\shortability{Duration:}{1 min./level (D)}
\shortability{Saving Throw:}{Fortitude negates}
\shortability{Spell Resistance:}{Yes}
\smallskip\noindent This spell causes instant diminution of a humanoid creature, halving its height, length, and width and dividing its weight by 8. This decrease changes the creature's size category to the next smaller one. The target gains a +2 size bonus to Dexterity, a -2 size penalty to Strength (to a minimum of 1), and a +1 bonus on attack rolls and AC due to its reduced size.

\smallskip\noindent A Small humanoid creature whose size decreases to Tiny has a space of 2-1/2 feet and a natural reach of 0 feet (meaning that it must enter an opponent's square to attack). A Large humanoid creature whose size decreases to Medium has a space of 5 feet and a natural reach of 5 feet. This spell doesn't change the target's speed.

\smallskip\noindent All equipment worn or carried by a creature is similarly reduced by the spell.

\smallskip\noindent Melee and projectile weapons deal less damage. Other magical properties are not affected by this spell. Any reduced item that leaves the reduced creature's possession (including a projectile or thrown weapon) instantly returns to its normal size. This means that thrown weapons deal their normal damage (projectiles deal damage based on the size of the weapon that fired them).

\smallskip\noindent Multiple magical effects that reduce size do not stack.

\smallskip\noindent Reduce person counters and dispels enlarge person.

\smallskip\noindent Reduce person can be made permanent with a permanency spell.

\shortability{Material Component:}{A pinch of powdered iron.}

\featname{Reduce Person, Mass}
\shortability{Transmutation}{}
\shortability{Level:}{Sor/Wiz 4}
\shortability{Target:}{One humanoid creature/level, no two of which can be more than 30 ft. apart}
\smallskip\noindent This spell functions like reduce person, except that it affects multiple creatures.

\featname{Refuge}
\shortability{Conjuration}{(Teleportation)}
\shortability{Level:}{Clr 7, Sor/Wiz 9}
\shortability{Components:}{V, S, M}
\shortability{Casting Time:}{1 standard action}
\shortability{Range:}{Touch}
\shortability{Target:}{Object touched}
\shortability{Duration:}{Permanent until discharged}
\shortability{Saving Throw:}{None}
\shortability{Spell Resistance:}{No}
\smallskip\noindent You create powerful magic in some specially prepared object. This object contains the power to instantly transport its possessor across any distance within the same plane to your abode. Once the item is transmuted, you must give it willingly to a creature and at the same time inform it of a command word to be spoken when the item is used. To make use of the item, the subject speaks the command word at the same time that it rends or breaks the item (a standard action). When this is done, the individual and all objects it is wearing and carrying (to a maximum of the character's heavy load) are instantly transported to your abode. No other creatures are affected (aside from a familiar that is touching the subject).

\smallskip\noindent You can alter the spell when casting it so that it transports you to within 10 feet of the possessor of the item when it is broken and the command word spoken. You will have a general idea of the location and situation of the item possessor at the time the refuge spell is discharged, but once you decide to alter the spell in this fashion, you have no choice whether or not to be transported.

\shortability{Material Component:}{The specially prepared object, whose construction requires gems worth 1,500 gp.}

\featname{Regenerate}
\shortability{Conjuration}{(Healing)}
\shortability{Level:}{Clr 7, Drd 9, Healing 7}
\shortability{Components:}{V, S, DF}
\shortability{Casting Time:}{3 full rounds}
\shortability{Range:}{Touch}
\shortability{Target:}{Living creature touched}
\shortability{Duration:}{Instantaneous}
\shortability{Saving Throw:}{Fortitude negates (harmless)}
\shortability{Spell Resistance:}{Yes (harmless)}
\smallskip\noindent The subject's severed body members (fingers, toes, hands, feet, arms, legs, tails, or even heads of multiheaded creatures), broken bones, and ruined organs grow back. After the spell is cast, the physical regeneration is complete in 1 round if the severed members are present and touching the creature. It takes 2d10 rounds otherwise.

\smallskip\noindent Regenerate also cures 4d8 points of damage +1 point per caster level (maximum +35), rids the subject of exhaustion and/or fatigue, and eliminates all nonlethal damage the subject has taken. It has no effect on nonliving creatures (including undead).

\featname{Reincarnate}
\shortability{Transmutation}{}
\shortability{Level:}{Drd 4}
\shortability{Components:}{V, S, M, DF}
\shortability{Casting Time:}{10 minutes}
\shortability{Range:}{Touch}
\shortability{Target:}{Dead creature touched}
\shortability{Duration:}{Instantaneous}
\shortability{Saving Throw:}{None; see text}
\shortability{Spell Resistance:}{Yes (harmless)}
\smallskip\noindent With this spell, you bring back a dead creature in another body, provided that its death occurred no more than one week before the casting of the spell and the subject's soul is free and willing to return. If the subject's soul is not willing to return, the spell does not work; therefore, a subject that wants to return receives no saving throw.

\smallskip\noindent Since the dead creature is returning in a new body, all physical ills and afflictions are repaired. The condition of the remains is not a factor. So long as some small portion of the creature's body still exists, it can be reincarnated, but the portion receiving the spell must have been part of the creature's body at the time of death. The magic of the spell creates an entirely new young adult body for the soul to inhabit from the natural elements at hand. This process takes 1 hour to complete.  When the body is ready, the subject is reincarnated.

\smallskip\noindent A reincarnated creature recalls the majority of its former life and form. It retains any class abilities, feats, or skill ranks it formerly possessed. Its class, base attack bonus, base save bonuses, and hit points are unchanged. Strength, Dexterity, and Constitution scores depend partly on the new body. First eliminate the subject's racial adjustments (since it is no longer of his previous race) and then apply the adjustments found below to its remaining ability scores. The subject's level (or Hit Dice) is reduced by 1. If the subject was 1st level, its new Constitution score is reduced by 2. (If this reduction would put its Con at 0 or lower, it can't be reincarnated). This level/HD loss or Constitution loss cannot be repaired by any means.

\smallskip\noindent It's possible for the change in the subject's ability scores to make it difficult for it to pursue its previous character class. If this is the case, the subject is well advised to become a multiclass character.

\smallskip\noindent For a humanoid creature, the new incarnation is determined using the following table. For nonhumanoid creatures, a similar table of creatures of the same type should be created.

\smallskip\noindent A creature that has been turned into an undead creature or killed by a death effect can't be returned to life by this spell.  Constructs, elementals, outsiders, and undead creatures can't be reincarnated. The spell cannot bring back a creature who has died of old age.

\begin{tabular}[h!]{l|llll}
d\%   & Incarnation & Str & Dex & Con \\ \hline
01    & Bugbear & +4 & +2 & +2 \\	   
02-13 & Dwarf & +0 & +0 & +2 \\	   
14-25 & Elf & +0 & +2 & -2 \\	   
26    & Gnoll & +4 & +0 & +2 \\	   
27-38 & Gnome & -2 & +0 & +2 \\	   
39-42 & Goblin  & -2 & +2 & +0 \\	   
43-52 & Half-elf & +0 & +0 & +0 \\	   
53-62 & Half-orc & +2 & +0 & +0	\\   
63-74 & Halfling & -2 & +2 & +0 \\   
75-89 & Human & +0 & +0 & +0 \\   
90-93 & Kobold & -4 & +2 & -2 \\   
94    & Lizardfolk & +2 & +0 & +2	\\   
95-98 & Orc & +4 & +0 & +0 \\   
99    & Troglodyte & +0 & -2 & +4 \\	   
100   & Other & ? & ? & ? \\
\end{tabular}

\smallskip\noindent The reincarnated creature gains all abilities associated with its new form, including forms of movement and speeds, natural armor, natural attacks, extraordinary abilities, and the like, but it doesn't automatically speak the language of the new form. 

\smallskip\noindent A wish or a miracle spell can restore a reincarnated character to his or her original form.

\shortability{Material Component:}{Rare oils and unguents worth a total of least 1,000 gp, spread over the remains.}

\featname{Remove Blindness/Deafness}
\shortability{Conjuration}{(Healing)}
\shortability{Level:}{Clr 3, Pal 3}
\shortability{Components:}{V, S}
\shortability{Casting Time:}{1 standard action}
\shortability{Range:}{Touch}
\shortability{Target:}{Creature touched}
\shortability{Duration:}{Instantaneous}
\shortability{Saving Throw:}{Fortitude negates (harmless)}
\shortability{Spell Resistance:}{Yes (harmless)}
\smallskip\noindent Remove blindness/deafness cures blindness or deafness (your choice), whether the effect is normal or magical in nature. The spell does not restore ears or eyes that have been lost, but it repairs them if they are damaged.

\smallskip\noindent Remove blindness/deafness counters and dispels blindness/deafness.

\featname{Remove Curse}
\shortability{Abjuration}{}
\shortability{Level:}{Brd 3, Clr 3, Pal 3, Sor/Wiz 4}
\shortability{Components:}{V, S}
\shortability{Casting Time:}{1 standard action}
\shortability{Range:}{Touch}
\shortability{Target:}{Creature or item touched}
\shortability{Duration:}{Instantaneous}
\shortability{Saving Throw:}{Will negates (harmless)}
\shortability{Spell Resistance:}{Yes (harmless)}
\smallskip\noindent Remove curse instantaneously removes all curses on an object or a creature. Remove curse does not remove the curse from a cursed shield, weapon, or suit of armor, although the spell typically enables the creature afflicted with any such cursed item to remove and get rid of it. Certain special curses may not be countered by this spell or may be countered only by a caster of a certain level or higher.

\smallskip\noindent Remove curse counters and dispels bestow curse.

\featname{Remove Disease}
\shortability{Conjuration}{(Healing)}
\shortability{Level:}{Clr 3, Drd 3, Rgr 3}
\shortability{Components:}{V, S}
\shortability{Casting Time:}{1 standard action}
\shortability{Range:}{Touch}
\shortability{Target:}{Creature touched}
\shortability{Duration:}{Instantaneous}
\shortability{Saving Throw:}{Fortitude negates (harmless)}
\shortability{Spell Resistance:}{Yes (harmless)}
\smallskip\noindent Remove disease cures all diseases that the subject is suffering from. The spell also kills parasites, including green slime and others. Certain special diseases may not be countered by this spell or may be countered only by a caster of a certain level or higher.

\shortability{Note:}{Since the spell's duration is instantaneous, it does not prevent reinfection after a new exposure to the same disease at a later date.}

\featname{Remove Fear}
\shortability{Abjuration}{}
\shortability{Level:}{Brd 1, Clr 1}
\shortability{Components:}{V, S}
\shortability{Casting Time:}{1 standard action}
\shortability{Range:}{Close (25 ft. + 5 ft./2 levels)}
\shortability{Targets:}{One creature plus one additional creature per four levels, no two of which can be more than 30 ft. apart}
\shortability{Duration:}{10 minutes; see text}
\shortability{Saving Throw:}{Will negates (harmless)}
\shortability{Spell Resistance:}{Yes (harmless)}
\smallskip\noindent You instill courage in the subject, granting it a +4 morale bonus against fear effects for 10 minutes. If the subject is under the influence of a fear effect when receiving the spell, that effect is suppressed for the duration of the spell.

\smallskip\noindent Remove fear counters and dispels cause fear.

\featname{Remove Paralysis}
\shortability{Conjuration}{(Healing)}
\shortability{Level:}{Clr 2, Pal 2}
\shortability{Components:}{V, S}
\shortability{Casting Time:}{1 standard action}
\shortability{Range:}{Close (25 ft. + 5 ft./2 levels)}
\shortability{Targets:}{Up to four creatures, no two of which can be more than 30 ft. apart}
\shortability{Duration:}{Instantaneous}
\shortability{Saving Throw:}{Will negates (harmless)}
\shortability{Spell Resistance:}{Yes (harmless)}
\smallskip\noindent You can free one or more creatures from the effects of any temporary paralysis or related magic, including a ghoul's touch or a slow spell. If the spell is cast on one creature, the paralysis is negated. If cast on two creatures, each receives another save with a +4 resistance bonus against the effect that afflicts it. If cast on three or four creatures, each receives another save with a +2 resistance bonus.

\smallskip\noindent The spell does not restore ability scores reduced by penalties, damage, or drain.

\featname{Repel Metal or Stone}
\shortability{Abjuration}{[Earth]}
\shortability{Level:}{Drd 8}
\shortability{Components:}{V, S}
\shortability{Casting Time:}{1 standard action}
\shortability{Range:}{60 ft.}
\shortability{Area:}{60-ft. line from you}
\shortability{Duration:}{1 round/level (D)}
\shortability{Saving Throw:}{None}
\shortability{Spell Resistance:}{No}
\smallskip\noindent Like repel wood, this spell creates waves of invisible and intangible energy that roll forth from you. All metal or stone objects in the path of the spell are pushed away from you to the limit of the range. Fixed metal or stone objects larger than 3 inches in diameter and loose objects weighing more than 500 pounds are not affected. Anything else, including animated objects, small boulders, and creatures in metal armor, moves back. Fixed objects 3 inches in diameter or smaller bend or break, and the pieces move with the wave of energy. Objects affected by the spell are repelled at the rate of 40 feet per round.

\smallskip\noindent Objects such as metal armor, swords, and the like are pushed back, dragging their bearers with them. Even magic items with metal components are repelled, although an antimagic field blocks the effects.

\smallskip\noindent The waves of energy continue to sweep down the set path for the spell's duration. After you cast the spell, the path is set, and you can then do other things or go elsewhere without affecting the spell's power.

\featname{Repel Vermin}
\shortability{Abjuration}{}
\shortability{Level:}{Brd 4, Clr 4, Drd 4, Rgr 3}
\shortability{Components:}{V, S, DF}
\shortability{Casting Time:}{1 standard action}
\shortability{Range:}{10 ft.}
\shortability{Area:}{10-ft.-radius emanation centered on you}
\shortability{Duration:}{10 min./level (D)}
\shortability{Saving Throw:}{None or Will negates; see text}
\shortability{Spell Resistance:}{Yes}
\smallskip\noindent An invisible barrier holds back vermin. A vermin with Hit Dice of less than one-third your level cannot penetrate the barrier.

\smallskip\noindent A vermin with Hit Dice of one-third your level or more can penetrate the barrier if it succeeds on a Will save. Even so, crossing the barrier deals the vermin 2d6 points of damage, and pressing against the barrier causes pain, which deters most vermin.

\featname{Repel Wood}
\shortability{Transmutation}{}
\shortability{Level:}{Drd 6, Plant 6}
\shortability{Components:}{V, S}
\shortability{Casting Time:}{1 standard action}
\shortability{Range:}{60 ft.}
\shortability{Area:}{60-ft. line-shaped emanation from you}
\shortability{Duration:}{1 min./level (D)}
\shortability{Saving Throw:}{None}
\shortability{Spell Resistance:}{No}
\smallskip\noindent Waves of energy roll forth from you, moving in the direction that you determine, causing all wooden objects in the path of the spell to be pushed away from you to the limit of the range. Wooden objects larger than 3 inches in diameter that are fixed firmly are not affected, but loose objects are. Objects 3 inches in diameter or smaller that are fixed in place splinter and break, and the pieces move with the wave of energy. Objects affected by the spell are repelled at the rate of 40 feet per round.

\smallskip\noindent Objects such as wooden shields, spears, wooden weapon shafts and hafts, and arrows and bolts are pushed back, dragging those carrying them along. (A creature being dragged by an item it is carrying can let go. A creature being dragged by a shield can loose it as a move action and drop it as a free action.) If a spear is planted (set) to prevent this forced movement, it splinters. Even magic items with wooden sections are repelled, although an antimagic field blocks the effects.

\smallskip\noindent The waves of energy continue to sweep down the set path for the spell's duration. After you cast the spell, the path is set, and you can then do other things or go elsewhere without affecting the spell's power.

\featname{Repulsion}
\shortability{Abjuration}{}
\shortability{Level:}{Clr 7, Protection 7, Sor/Wiz 6}
\shortability{Components:}{V, S, F/DF}
\shortability{Casting Time:}{1 standard action}
\shortability{Range:}{Up to 10 ft./level}
\shortability{Area:}{Up to 10-ft.-radius/level emanation centered on you}
\shortability{Duration:}{1 round/level (D)}
\shortability{Saving Throw:}{Will negates}
\shortability{Spell Resistance:}{Yes}
\smallskip\noindent An invisible, mobile field surrounds you and prevents creatures from approaching you. You decide how big the field is at the time of casting (to the limit your level allows). Any creature within or entering the field must attempt a save. If it fails, it becomes unable to move toward you for the duration of the spell. Repelled creatures' actions are not otherwise restricted.

\smallskip\noindent They can fight other creatures and can cast spells and attack you with ranged weapons. If you move closer to an affected creature, nothing happens. (The creature is not forced back.) The creature is free to make melee attacks against you if you come within reach. If a repelled creature moves away from you and then tries to turn back toward you, it cannot move any closer if it is still within the spell's area.

\shortability{Arcane Focus:}{A pair of small iron bars attached to two small canine statuettes, one black and one white, the whole array worth 50 gp.}

\featname{Resilient Sphere}
\shortability{Evocation}{[Force]}
\shortability{Level:}{Sor/Wiz 4}
\shortability{Components:}{V, S, M}
\shortability{Casting Time:}{1 standard action}
\shortability{Range:}{Close (25 ft. + 5 ft./2 levels)}
\shortability{Effect:}{1-ft.-diameter/level sphere, centered around a creature}
\shortability{Duration:}{1 min./level (D)}
\shortability{Saving Throw:}{Reflex negates}
\shortability{Spell Resistance:}{Yes}
\smallskip\noindent A globe of shimmering force encloses a creature, provided the creature is small enough to fit within the diameter of the sphere. The sphere contains its subject for the spell's duration. The sphere is not subject to damage of any sort except from a rod of cancellation, a rod of negation, a disintegrate spell, or a targeted dispel magic spell. These effects destroy the sphere without harm to the subject. Nothing can pass through the sphere, inside or out, though the subject can breathe normally.

\smallskip\noindent The subject may struggle, but the sphere cannot be physically moved either by people outside it or by the struggles of those within.

\shortability{Material Component:}{A hemispherical piece of clear crystal and a matching hemispherical piece of gum arabic.}

\featname{Resistance}
\shortability{Abjuration}{}
\shortability{Level:}{Brd 0, Clr 0, Drd 0, Pal 1, Sor/Wiz 0}
\shortability{Components:}{V, S, M/DF}
\shortability{Casting Time:}{1 standard action}
\shortability{Range:}{Touch}
\shortability{Target:}{Creature touched}
\shortability{Duration:}{1 minute}
\shortability{Saving Throw:}{Will negates (harmless)}
\shortability{Spell Resistance:}{Yes (harmless)}
\smallskip\noindent You imbue the subject with magical energy that protects it from harm, granting it a +1 resistance bonus on saves.

\smallskip\noindent Resistance can be made permanent with a permanency spell.

\shortability{Arcane Material Component:}{A miniature cloak.}

\featname{Resist Energy}
\shortability{Abjuration}{}
\shortability{Level:}{Clr 2, Drd 2, Fire 3, Pal 2, Rgr 1, Sor/Wiz 2}
\shortability{Components:}{V, S, DF}
\shortability{Casting Time:}{1 standard action}
\shortability{Range:}{Touch}
\shortability{Target:}{Creature touched}
\shortability{Duration:}{10 min./level}
\shortability{Saving Throw:}{Fortitude negates (harmless)}
\shortability{Spell Resistance:}{Yes (harmless)}
\smallskip\noindent This abjuration grants a creature limited protection from damage of whichever one of five energy types you select: acid, cold, electricity, fire, or sonic. The subject gains energy resistance 10 against the energy type chosen, meaning that each time the creature is subjected to such damage (whether from a natural or magical source), that damage is reduced by 10 points before being applied to the creature's hit points. The value of the energy resistance granted increases to 20 points at 7th level and to a maximum of 30 points at 11th level. The spell protects the recipient's equipment as well.

\smallskip\noindent Resist energy absorbs only damage. The subject could still suffer unfortunate side effects.

\shortability{Note:}{Resist energy overlaps (and does not stack with) protection from energy. If a character is warded by protection from energy and resist energy, the protection spell absorbs damage until its power is exhausted.}

\featname{Restoration}
\shortability{Conjuration}{(Healing)}
\shortability{Level:}{Clr 4, Pal 4}
\shortability{Components:}{V, S, M}
\smallskip\noindent This spell functions like lesser restoration, except that it also dispels negative levels and restores one experience level to a creature who has had a level drained. The drained level is restored only if the time since the creature lost the level is equal to or less than one day per caster level. A character who has a level restored by restoration has exactly the minimum number of experience points necessary to restore him or her to his or her previous level.

\smallskip\noindent Restoration cures all temporary ability damage, and it restores all points permanently drained from a single ability score (your choice if more than one is drained). It also eliminates any fatigue or exhaustion suffered by the target. 

\smallskip\noindent Restoration does not restore levels or Constitution points lost due to death.

\shortability{Material Component:}{Diamond dust worth 100 gp that is sprinkled over the target.}

\featname{Restoration, Greater}
\shortability{Conjuration}{(Healing)}
\shortability{Level:}{Clr 7}
\shortability{Components:}{V, S, XP}
\shortability{Casting Time:}{10 minutes}
\smallskip\noindent This spell functions like lesser restoration, except that it dispels all negative levels afflicting the healed creature. This effect also reverses level drains by a force or creature, restoring the creature to the highest level it had previously attained. The drained levels are restored only if the time since the creature lost the level is no more than one week per caster level.

\smallskip\noindent Greater restoration also dispels all magical effects penalizing the creature's abilities, cures all temporary ability damage, and restores all points permanently drained from all ability scores. It also eliminates fatigue and exhaustion, and removes all forms of insanity, confusion, and similar mental effects. Greater restoration does not restore levels or Constitution points lost due to death.

\shortability{XP Cost:}{500 XP.}

\featname{Restoration, Lesser}
\shortability{Conjuration}{(Healing)}
\shortability{Level:}{Clr 2, Drd 2, Pal 1}
\shortability{Components:}{V, S}
\shortability{Casting Time:}{3 rounds}
\shortability{Range:}{Touch}
\shortability{Target:}{Creature touched}
\shortability{Duration:}{Instantaneous}
\shortability{Saving Throw:}{Will negates (harmless)}
\shortability{Spell Resistance:}{Yes (harmless)}
\smallskip\noindent Lesser restoration dispels any magical effects reducing one of the subject's ability scores or cures 1d4 points of temporary ability damage to one of the subject's ability scores. It also eliminates any fatigue suffered by the character, and improves an exhausted condition to fatigued. It does not restore permanent ability drain.

\featname{Resurrection}
\shortability{Conjuration}{(Healing)}
\shortability{Level:}{Clr 7}
\shortability{Casting Time:}{10 minutes}
\smallskip\noindent This spell functions like raise dead, except that you are able to restore life and complete strength to any deceased creature.

\smallskip\noindent The condition of the remains is not a factor. So long as some small portion of the creature's body still exists, it can be resurrected, but the portion receiving the spell must have been part of the creature's body at the time of death. (The remains of a creature hit by a disintegrate spell count as a small portion of its body.) The creature can have been dead no longer than 10 years per caster level.

\smallskip\noindent Upon completion of the spell, the creature is immediately restored to full hit points, vigor, and health, with no loss of prepared spells. However, the subject loses one level, or 2 points of Constitution if the subject was 1st level. (If this reduction would bring its Con to 0 or lower, it can't be resurrected). This level loss or Constitution loss cannot be repaired by any means.

\smallskip\noindent You can resurrect someone killed by a death effect or someone who has been turned into an undead creature and then destroyed. You cannot resurrect someone who has died of old age. Constructs, elementals, outsiders, and undead creatures can't be resurrected.

\shortability{Material Component:}{A sprinkle of holy water and diamonds worth a total of at least 10,000 gp.}

\featname{Reverse Gravity}
\shortability{Transmutation}{}
\shortability{Level:}{Drd 8, Sor/Wiz 7}
\shortability{Components:}{V, S, M/DF}
\shortability{Casting Time:}{1 standard action}
\shortability{Range:}{Medium (100 ft. + 10 ft./level)}
\shortability{Area:}{Up to one 10-ft. cube per two levels (S)}
\shortability{Duration:}{1 round/level (D)}
\shortability{Saving Throw:}{None; see text}
\shortability{Spell Resistance:}{No}
\smallskip\noindent This spell reverses gravity in an area, causing all unattached objects and creatures within that area to fall upward and reach the top of the area in 1 round. If some solid object (such as a ceiling) is encountered in this fall, falling objects and creatures strike it in the same manner as they would during a normal downward fall. If an object or creature reaches the top of the area without striking anything, it remains there, oscillating slightly, until the spell ends. At the end of the spell duration, affected objects and creatures fall downward.

\smallskip\noindent Provided it has something to hold onto, a creature caught in the area can attempt a Reflex save to secure itself when the spell strikes. Creatures who can fly or levitate can keep themselves from falling.

\shortability{Arcane Material Component:}{A lodestone and iron filings.}

\featname{Righteous Might}
\shortability{Transmutation}{}
\shortability{Level:}{Clr 5, Strength 5}
\shortability{Components:}{V, S, DF}
\shortability{Casting Time:}{1 standard action}
\shortability{Range:}{Personal}
\shortability{Target:}{You}
\shortability{Duration:}{1 round/level (D)}
\smallskip\noindent Your height immediately doubles, and your weight increases by a factor of eight. This increase changes your size category to the next larger one, and you gain a +8 size bonus to Strength and a +4 size bonus to Constitution. You gain a +4 enhancement bonus to your natural armor. You gain damage reduction 5/evil (if you normally channel positive energy) or damage reduction 5/good (if you normally channel negative energy). At 12th level this damage reduction becomes 10/evil or 10/good, and at 15th level it becomes 15/evil or 15/good (the maximum). Your size modifier for AC and attacks changes as appropriate to your new size category. This spell doesn't change your speed.  Determine space and reach as appropriate to your new size.

\smallskip\noindent If insufficient room is available for the desired growth, you attain the maximum possible size and may make a Strength check (using your increased Strength) to burst any enclosures in the process. If you fail, you are constrained without harm by the materials enclosing you- the spell cannot crush you by increasing your size.

\smallskip\noindent All equipment you wear or carry is similarly enlarged by the spell. Melee and projectile weapons deal more damage. Other magical properties are not affected by this spell. Any enlarged item that leaves your possession (including a projectile or thrown weapon) instantly returns to its normal size. This means that thrown weapons deal their normal damage (projectiles deal damage based on the size of the weapon that fired them).

\smallskip\noindent Multiple magical effects that increase size do not stack.

\featname{Rope Trick}
\shortability{Transmutation}{}
\shortability{Level:}{Sor/Wiz 2}
\shortability{Components:}{V, S, M}
\shortability{Casting Time:}{1 standard action}
\shortability{Range:}{Touch}
\shortability{Target:}{One touched piece of rope from 5 ft. to 30 ft. long}
\shortability{Duration:}{1 hour/level (D)}
\shortability{Saving Throw:}{None}
\shortability{Spell Resistance:}{No}
\smallskip\noindent When this spell is cast upon a piece of rope from 5 to 30 feet long, one end of the rope rises into the air until the whole rope hangs perpendicular to the ground, as if affixed at the upper end. The upper end is, in fact, fastened to an extradimensional space that is outside the multiverse of extradimensional spaces (``planes''). Creatures in the extradimensional space are hidden, beyond the reach of spells (including divinations), unless those spells work across planes. The space holds as many as eight creatures (of any size). Creatures in the space can pull the rope up into the space, making the rope "disappear." In that case, the rope counts as one of the eight creatures that can fit in the space. The rope can support up to 16,000 pounds. A weight greater than that can pull the rope free.

\smallskip\noindent Spells cannot be cast across the extradimensional interface, nor can area effects cross it. Those in the extradimensional space can see out of it as if a 3-foot-by- 5-foot window were centered on the rope. The window is present on the Material Plane, but it's invisible, and even creatures that can see the window can't see through it. Anything inside the extradimensional space drops out when the spell ends. The rope can be climbed by only one person at a time. The rope trick spell enables climbers to reach a normal place if they do not climb all the way to the extradimensional space.

\shortability{Note:}{It is hazardous to create an extradimensional space within an existing extradimensional space or to take an extradimensional space into an existing one.}
\shortability{Material Component:}{Powdered corn extract and a twisted loop of parchment.}

\featname{Rusting Grasp}
\shortability{Transmutation}{}
\shortability{Level:}{Drd 4}
\shortability{Components:}{V, S, DF}
\shortability{Casting Time:}{1 standard action}
\shortability{Range:}{Touch}
\shortability{Target:}{One nonmagical ferrous object (or the volume of the object within 3 ft. of the touched point) or one ferrous creature}
\shortability{Duration:}{See text}
\shortability{Saving Throw:}{None}
\shortability{Spell Resistance:}{No}
\smallskip\noindent Any iron or iron alloy item you touch becomes instantaneously rusted, pitted, and worthless, effectively destroyed. If the item is so large that it cannot fit within a 3-foot radius a 3-foot-radius volume of the metal is rusted and destroyed. Magic items made of metal are immune to this spell.

\smallskip\noindent You may employ rusting grasp in combat with a successful melee touch attack. Rusting grasp used in this way instantaneously destroys 1d6 points of Armor Class gained from metal armor (to the maximum amount of protection the armor offered) through corrosion. 

\smallskip\noindent Weapons in use by an opponent targeted by the spell are more difficult to grasp. You must succeed on a melee touch attack against the weapon. A metal weapon that is hit is destroyed.

\shortability{Note:}{Striking at an opponent's weapon provokes an attack of opportunity. Also, you must touch the weapon and not the other way around.}

\smallskip\noindent Against a ferrous creature, rusting grasp instantaneously deals 3d6 points of damage +1 per caster level (maximum +15) per successful attack. The spell lasts for 1 round per level, and you can make one melee touch attack per round.

\featname{Sanctuary}
\shortability{Abjuration}{}
\shortability{Level:}{Clr 1, Protection 1}
\shortability{Components:}{V, S, DF}
\shortability{Casting Time:}{1 standard action}
\shortability{Range:}{Touch}
\shortability{Target:}{Creature touched}
\shortability{Duration:}{1 round/level}
\shortability{Saving Throw:}{Will negates}
\shortability{Spell Resistance:}{No}
\smallskip\noindent Any opponent attempting to strike or otherwise directly attack the warded creature, even with a targeted spell, must attempt a Will save. If the save succeeds, the opponent can attack normally and is unaffected by that casting of the spell. If the save fails, the opponent can't follow through with the attack, that part of its action is lost, and it can't directly attack the warded creature for the duration of the spell. Those not attempting to attack the subject remain unaffected. This spell does not prevent the warded creature from being attacked or affected by area or effect spells. The subject cannot attack without breaking the spell but may use nonattack spells or otherwise act.

\featname{Scare}
\shortability{Necromancy}{[Fear, Mind-Affecting]}
\shortability{Level:}{Brd 2, Sor/Wiz 2}
\shortability{Components:}{V, S, M}
\shortability{Casting Time:}{1 standard action}
\shortability{Range:}{Medium (100 ft. + 10 ft./level)}
\shortability{Targets:}{One living creature per three levels, no two of which can be more than 30 ft. apart}
\shortability{Duration:}{1 round/level or 1 round; see text for cause fear}
\shortability{Saving Throw:}{Will partial}
\shortability{Spell Resistance:}{Yes}
\smallskip\noindent This spell functions like cause fear, except that it causes all targeted creatures of less than 6 HD to become frightened.

\shortability{Material Component:}{A bit of bone from an undead skeleton, zombie, ghoul, ghast, or mummy.}

\featname{Scintillating Pattern}
\shortability{Illusion}{(Pattern) [Mind-Affecting]}
\shortability{Level:}{Sor/Wiz 8}
\shortability{Components:}{V, S, M}
\shortability{Casting Time:}{1 standard action}
\shortability{Range:}{Close (25 ft. + 5 ft./2 levels)}
\shortability{Effect:}{Colorful lights in a 20-ft.-radius spread}
\shortability{Duration:}{Concentration + 2 rounds}
\shortability{Saving Throw:}{None}
\shortability{Spell Resistance:}{Yes}

\smallskip\noindent A twisting pattern of discordant, coruscating colors weaves through the air, affecting creatures within it. The spell affects a total number of Hit Dice of creatures equal to your caster level (maximum 20). Creatures with the fewest HD are affected first; and, among creatures with equal HD, those who are closest to the spell's point of origin are affected first. Hit Dice that are not sufficient to affect a creature are wasted. The spell affects each subject according to its Hit Dice.

\smallskip\noindent 6 or less: Unconscious for 1d4 rounds, then stunned for 1d4 rounds, and then confused for 1d4 rounds. (Treat an unconscious result as stunned for nonliving creatures.)

\smallskip\noindent 7 to 12: Stunned for 1d4 rounds, then confused for 1d4 rounds. 

\smallskip\noindent 13 or more: Confused for 1d4 rounds. 

\smallskip\noindent Sightless creatures are not affected by scintillating pattern.

\shortability{Material Component:}{A small crystal prism.}

\featname{Scorching Ray}
\shortability{Evocation}{[Fire]}
\shortability{Level:}{Sor/Wiz 2}
\shortability{Components:}{V, S}
\shortability{Casting Time:}{1 standard action}
\shortability{Range:}{Close (25 ft. + 5 ft./2 levels)}
\shortability{Effect:}{One or more rays}
\shortability{Duration:}{Instantaneous}
\shortability{Saving Throw:}{None}
\shortability{Spell Resistance:}{Yes}
\smallskip\noindent You blast your enemies with fiery rays. You may fire one ray, plus one additional ray for every four levels beyond 3rd (to a maximum of three rays at 11th level). Each ray requires a ranged touch attack to hit and deals 4d6 points of fire damage.

\smallskip\noindent The rays may be fired at the same or different targets, but all bolts must be aimed at targets within 30 feet of each other and fired simultaneously.

\featname{Screen}
\shortability{Illusion}{(Glamer)}
\shortability{Level:}{Sor/Wiz 8, Trickery 7}
\shortability{Components:}{V, S}
\shortability{Casting Time:}{10 minutes}
\shortability{Range:}{Close (25 ft. + 5 ft./2 levels)}
\shortability{Area:}{30-ft. cube/level (S)}
\shortability{Duration:}{24 hours}
\shortability{Saving Throw:}{None or Will disbelief (if interacted with); see text}
\shortability{Spell Resistance:}{No}
\smallskip\noindent This spell combines several elements to create a powerful protection from scrying and direct observation. When casting the spell, you dictate what will and will not be observed in the spell's area. The illusion created must be stated in general terms. Once the conditions are set, they cannot be changed.

\smallskip\noindent Attempts to scry the area automatically detect the image stated by you with no save allowed. Sight and sound are appropriate to the illusion created. 

\smallskip\noindent Direct observation may allow a save (as per a normal illusion), if there is cause to disbelieve what is seen. Even entering the area does not cancel the illusion or necessarily allow a save, assuming that hidden beings take care to stay out of the way of those affected by the illusion.

\featname{Scrying}
\shortability{Divination}{(Scrying)}
\shortability{Level:}{Brd 3, Clr 5, Drd 4, Sor/Wiz 4}
\shortability{Components:}{V, S, M/DF, F}
\shortability{Casting Time:}{1 hour}
\shortability{Range:}{See text}
\shortability{Effect:}{Magical sensor}
\shortability{Duration:}{1 min./level}
\shortability{Saving Throw:}{Will negates}
\shortability{Spell Resistance:}{Yes}
\smallskip\noindent You can see and hear some creature, which may be at any distance. If the subject succeeds on a Will save, the scrying attempt simply fails. The difficulty of the save depends on how well you know the subject and what sort of physical connection (if any) you have to that creature. Furthermore, if the subject is on another plane, it gets a +5 bonus on its Will save.
 
\begin{tabular}[h!]{p{2.5in}|p{.5in}}
Knowledge & Will Save Modifier \\
None\textsuperscript{1} & +10	\\
Secondhand (you have heard of the subject) & +5	\\
Firsthand (you have met the subject) & +0	\\
Familiar (you know the subject well) & -5	\\
\multicolumn{2}{p{3in}}{/textsuperscript{1} You must have some sort of connection to a creature you have no knowledge of.} \\
\end{tabular}

\begin{tabular}[h!]{l|l}
Connection & Will Save Modifier \\
Likeness or picture &	-2 \\
Possession or garment & -4 \\
Body part, lock of hair, bit of nail, etc. & -10 \\
\end{tabular}

\smallskip\noindent If the save fails, you can see and hear the subject and the subject's immediate surroundings (approximately 10 feet in all directions of the subject). If the subject moves, the sensor follows at a speed of up to 150 feet.

\smallskip\noindent As with all divination (scrying) spells, the sensor has your full visual acuity, including any magical effects. In addition, the following spells have a 5\% chance per caster level of operating through the sensor: detect chaos, detect evil, detect good, detect law, detect magic, and message.

\smallskip\noindent If the save succeeds, you can't attempt to scry on that subject again for at least 24 hours.

\shortability{Arcane Material Component:}{The eye of a hawk, an eagle, or a roc, plus nitric acid, copper, and zinc.}
\shortability{Wizard,}{Sorcerer, or Bard Focus: A mirror of finely wrought and highly polished silver costing not less than 1,000 gp. The mirror must be at least 2 feet by 4 feet.}
\shortability{Cleric Focus}{: A holy water font costing not less than 100 gp.}
\shortability{Druid Focus}{: A natural pool of water. }

\featname{Scrying, Greater}
\shortability{Divination}{(Scrying)}
\shortability{Level:}{Brd 6, Clr 7, Drd 7, Sor/Wiz 7}
\shortability{Components:}{V, S}
\shortability{Casting Time:}{1 standard action}
\shortability{Duration:}{1 hour/level}
\smallskip\noindent This spell functions like scrying, except as noted above. Additionally, all of the following spells function reliably through the sensor: detect chaos, detect evil, detect good, detect law, detect magic, message, read magic, and tongues.

\featname{Sculpt Sound}
\shortability{Transmutation}{}
\shortability{Level:}{Brd 3}
\shortability{Components:}{V, S}
\shortability{Casting Time:}{1 standard action}
\shortability{Range:}{Close (25 ft. + 5 ft./2 levels)}
\shortability{Targets:}{One creature or object/level, no two of which can be more than 30 ft. apart}
\shortability{Duration:}{1 hour/level (D)}
\shortability{Saving Throw:}{Will negates (object)}
\shortability{Spell Resistance:}{Yes (object)}
\smallskip\noindent You change the sounds that creatures or objects make. You can create sounds where none exist, deaden sounds, or transform sounds into other sounds. All affected creatures or objects must be transmuted in the same way. Once the transmutation is made, you cannot change it.

\smallskip\noindent You can change the qualities of sounds but cannot create words with which you are unfamiliar yourself.

\smallskip\noindent A spellcaster whose voice is changed dramatically is unable to cast spells with verbal components.

\featname{Searing Light}
\shortability{Evocatio}{}
\shortability{Level:}{Clr 3, Sun 3}
\shortability{Components:}{V, S}
\shortability{Casting Time:}{1 standard action}
\shortability{Range:}{Medium (100 ft. + 10 ft./level)}
\shortability{Effect:}{Ray}
\shortability{Duration:}{Instantaneous}
\shortability{Saving Throw:}{None}
\shortability{Spell Resistance:}{Yes}
\smallskip\noindent Focusing divine power like a ray of the sun, you project a blast of light from your open palm. You must succeed on a ranged touch attack to strike your target. A creature struck by this ray of light takes 1d8 points of damage per two caster levels (maximum 5d8). An undead creature takes 1d6 points of damage per caster level (maximum 10d6), and an undead creature particularly vulnerable to bright light takes 1d8 points of damage per caster level (maximum 10d8). A construct or inanimate object takes only 1d6 points of damage per two caster levels (maximum 5d6).

\featname{Secret Chest}
\shortability{Conjuration}{(Summoning)}
\shortability{Level:}{Sor/Wiz 5}
\shortability{Components:}{V, S, F}
\shortability{Casting Time:}{10 minutes}
\shortability{Range:}{See text}
\shortability{Target:}{One chest and up to 1 cu. ft. of goods/caster level}
\shortability{Duration:}{Sixty days or until discharged}
\shortability{Saving Throw:}{None}
\shortability{Spell Resistance:}{No}
\smallskip\noindent You hide a chest on the Ethereal Plane for as long as sixty days and can retrieve it at will. The chest can contain up to 1 cubic foot of material per caster level (regardless of the chest's actual size, which is about 3 feet by 2 feet by 2 feet). If any living creatures are in the chest, there is a 75\% chance that the spell simply fails. Once the chest is hidden, you can retrieve it by concentrating (a standard action), and it appears next to you.

\smallskip\noindent The chest must be exceptionally well crafted and expensive, constructed for you by master crafters. The cost of such a chest is never less than 5,000 gp. Once it is constructed, you must make a tiny replica (of the same materials and perfect in every detail), so that the miniature of the chest appears to be a perfect copy. (The replica costs 50 gp.) You can have but one pair of these chests at any given time-even a wish spell does not allow more. The chests are nonmagical and can be fitted with locks, wards, and so on, just as any normal chest can be.

\smallskip\noindent To hide the chest, you cast the spell while touching both the chest and the replica. The chest vanishes into the Ethereal Plane. You need the replica to recall the chest. After sixty days, there is a cumulative chance of 5\% per day that the chest is irretrievably lost. If the miniature of the chest is lost or destroyed, there is no way, not even with a wish spell, that the large chest can be summoned back, although an extraplanar expedition might be mounted to find it.

\smallskip\noindent Living things in the chest eat, sleep, and age normally, and they die if they run out of food, air, water, or whatever they need to survive.

\shortability{Focus:}{The chest and its replica.}

\featname{Secret Page}
\shortability{Transmutation}{}
\shortability{Level:}{Brd 3, Sor/Wiz 3}
\shortability{Components:}{V, S, M}
\shortability{Casting Time:}{10 minutes}
\shortability{Range:}{Touch}
\shortability{Target:}{Page touched, up to 3 sq. ft. in size}
\shortability{Duration:}{Permanent}
\shortability{Saving Throw:}{None}
\shortability{Spell Resistance:}{No}
\smallskip\noindent Secret page alters the contents of a page so that they appear to be something entirely different. The text of a spell can be changed to show even another spell. Explosive runes or sepia snake sigil can be cast upon the secret page.

\smallskip\noindent A comprehend languages spell alone cannot reveal a secret page's contents. You are able to reveal the original contents by speaking a special word. You can then peruse the actual page, and return it to its secret page form at will. You can also remove the spell by double repetition of the special word. A detect magic spell reveals dim magic on the page in question but does not reveal its true contents. True seeing reveals the presence of the hidden material but does not reveal the contents unless cast in combination with comprehend languages. A secret page spell can be dispelled, and the hidden writings can be destroyed by means of an erase spell.

\shortability{Material Component:}{Powdered herring scales and will-o'-wisp essence.}

\featname{Secure Shelter}
\shortability{Conjuration}{(Creation)}
\shortability{Level:}{Brd 4, Sor/Wiz 4}
\shortability{Components:}{V, S, M, F; see text}
\shortability{Casting Time:}{10 minutes}
\shortability{Range:}{Close (25 ft. + 5 ft./2 levels)}
\shortability{Effect:}{20-ft.-square structure}
\shortability{Duration:}{2 hours/level (D)}
\shortability{Saving Throw:}{None}
\shortability{Spell Resistance:}{No}
\smallskip\noindent You conjure a sturdy cottage or lodge made of material that is common in the area where the spell is cast. The floor is level, clean, and dry. In all respects the lodging resembles a normal cottage, with a sturdy door, two shuttered windows, and a small fireplace.

\smallskip\noindent The shelter has no heating or cooling source (other than natural insulation qualities). Therefore, it must be heated as a normal dwelling, and extreme heat adversely affects it and its occupants. The dwelling does, however, provide considerable security otherwise-it is as strong as a normal stone building, regardless of its material composition. The dwelling resists flames and fire as if it were stone. It is impervious to normal missiles (but not the sort cast by siege engines or giants).

\smallskip\noindent The door, shutters, and even chimney are secure against intrusion, the former two being arcane locked and the latter secured by an iron grate at the top and a narrow flue. In addition, these three areas are protected by an alarm spell. Finally, an unseen servant is conjured to provide service to you for the duration of the shelter.

\smallskip\noindent The secure shelter contains rude furnishings -eight bunks, a trestle table, eight stools, and a writing desk.

\shortability{Material Component:}{A square chip of stone, crushed lime, a few grains of sand, a sprinkling of water, and several splinters of wood. These must be augmented by the components of the unseen servant spell (string and a bit of wood) if this benefit is to be included.}
\shortability{Focus:}{The focus of the alarm spell (silver wire and a tiny bell) if this benefit is to be included.}

\featname{See Invisibility}
\shortability{Divination}{}
\shortability{Level:}{Brd 3, Sor/Wiz 2}
\shortability{Components:}{V, S, M}
\shortability{Casting Time:}{1 standard action}
\shortability{Range:}{Personal}
\shortability{Target:}{You}
\shortability{Duration:}{10 min./level (D)}
\smallskip\noindent You can see any objects or beings that are invisible within your range of vision, as well as any that are ethereal, as if they were normally visible. Such creatures are visible to you as translucent shapes, allowing you easily to discern the difference between visible, invisible, and ethereal creatures.

\smallskip\noindent The spell does not reveal the method used to obtain invisibility. It does not reveal illusions or enable you to see through opaque objects. It does not reveal creatures who are simply hiding, concealed, or otherwise hard to see.

\smallskip\noindent See invisibility can be made permanent with a permanency spell.

\shortability{Material Component:}{A pinch of talc and a small sprinkling of powdered silver.}

\featname{Seeming}
\shortability{Illusion}{(Glamer)}
\shortability{Level:}{Brd 5, Sor/Wiz 5}
\shortability{Components:}{V, S}
\shortability{Casting Time:}{1 standard action}
\shortability{Range:}{Close (25 ft. + 5 ft./2 levels)}
\shortability{Targets:}{One creature per two levels, no two of which can be more than 30 ft. apart}
\shortability{Duration:}{12 hours (D)}
\shortability{Saving Throw:}{Will negates or Will disbelief (if interacted with)}
\shortability{Spell Resistance:}{Yes or No; see text}
\smallskip\noindent This spell functions like disguise self, except that you can change the appearance of other people as well. Affected creatures resume their normal appearances if slain.

\smallskip\noindent Unwilling targets can negate the spell's effect on them by making Will saves or with spell resistance.

\featname{Sending}
\shortability{Evocatio}{}
\shortability{Level:}{Clr 4, Sor/Wiz 5}
\shortability{Components:}{V, S, M/DF}
\shortability{Casting Time:}{10 minutes}
\shortability{Range:}{See text}
\shortability{Target:}{One creature}
\shortability{Duration:}{1 round; see text}
\shortability{Saving Throw:}{None}
\shortability{Spell Resistance:}{No}
\smallskip\noindent You contact a particular creature with which you are familiar and send a short message of twenty-five words or less to the subject. The subject recognizes you if it knows you. It can answer in like manner immediately. A creature with an Intelligence score as low as 1 can understand the sending, though the subject's ability to react is limited as normal by its Intelligence score. Even if the sending is received, the subject is not obligated to act upon it in any manner.

\smallskip\noindent If the creature in question is not on the same plane of existence as you are, there is a 5\% chance that the sending does not arrive. (Local conditions on other planes may worsen this chance considerably.)

\shortability{Arcane Material Component:}{A short piece of fine copper wire.}

\featname{Sepia Snake Sigil}
\shortability{Conjuration}{(Creation) [Force]}
\shortability{Level:}{Brd 3, Sor/Wiz 3}
\shortability{Components:}{V, S, M}
\shortability{Casting Time:}{10 minutes}
\shortability{Range:}{Touch}
\shortability{Target:}{One touched book or written work}
\shortability{Duration:}{Permanent or until discharged; until released or 1d4 days + one day/level; see text}
\shortability{Saving Throw:}{Reflex negates}
\shortability{Spell Resistance:}{No}
\smallskip\noindent When you cast sepia snake sigil, a small symbol appears in the text of one written work such as a book, scroll, or map. The text containing the symbol must be at least twenty-five words long. When anyone reads the text containing the symbol, the sepia snake springs into being and strikes the reader, provided there is line of effect between the symbol and the reader.

\smallskip\noindent Simply seeing the enspelled text is not sufficient to trigger the spell; the subject must deliberately read it. The target is entitled to a save to evade the snake's strike. If it succeeds, the sepia snake dissipates in a flash of brown light accompanied by a puff of dun-colored smoke and a loud noise. If the target fails its save, it is engulfed in a shimmering amber field of force and immobilized until released, either at your command or when 1d4 days + one day per caster level have elapsed.

\smallskip\noindent While trapped in the amber field of force, the subject does not age, breathe, grow hungry, sleep, or regain spells. It is preserved in a state of suspended animation, unaware of its surroundings. It can be damaged by outside forces (and perhaps even killed), since the field provides no protection against physical injury. However, a dying subject does not lose hit points or become stable until the spell ends.

\smallskip\noindent The hidden sigil cannot be detected by normal observation, and detect magic reveals only that the entire text is magical.

\smallskip\noindent A dispel magic can remove the sigil. An erase spell destroys the entire page of text.

\smallskip\noindent Sepia snake sigil can be cast in combination with other spells that hide or garble text, such as secret page.

\shortability{Material Component:}{500 gp worth of powdered amber, a scale from any snake, and a pinch of mushroom spores.}

\featname{Sequester}
\shortability{Abjuration}{}
\shortability{Level:}{Sor/Wiz 7}
\shortability{Components:}{V, S, M}
\shortability{Casting Time:}{1 standard action}
\shortability{Range:}{Touch}
\shortability{Target:}{One willing creature or object (up to a 2-ft. cube/level) touched}
\shortability{Duration:}{One day/level (D)}
\shortability{Saving Throw:}{None or Will negates (object)}
\shortability{Spell Resistance:}{No or Yes (object)}
\smallskip\noindent When cast, this spell not only prevents divination spells from working to detect or locate the creature or object affected by sequester, it also renders the affected creature or object invisible to any form of sight or seeing (as the invisibility spell). The spell does not prevent the subject from being discovered through tactile means or through the use of devices. Creatures affected by sequester become comatose and are effectively in a state of suspended animation until the spell wears off or is dispelled.

\shortability{Note:}{The Will save prevents an attended or magical object from being sequestered. There is no save to see the sequestered creature or object or to detect it with a divination spell.}
\shortability{Material Component:}{A basilisk eyelash, gum arabic, and a dram of whitewash.}

\featname{Shades}
\shortability{Illusion}{(Shadow)}
\shortability{Level:}{Sor/Wiz 9}
\smallskip\noindent This spell functions like shadow conjuration, except that it mimics sorcerer and wizard conjuration spells of 8th level or lower. The illusory conjurations created deal four-fifths (80\%) damage to nonbelievers, and nondamaging effects are 80\% likely to work against nonbelievers.

\featname{Shadow Conjuration}
\shortability{Illusion}{(Shadow)}
\shortability{Level:}{Brd 4, Sor/Wiz 4}
\shortability{Components:}{V, S}
\shortability{Casting Time:}{1 standard action}
\shortability{Range:}{See text}
\shortability{Effect:}{See text}
\shortability{Duration:}{See text}
\shortability{Saving Throw:}{Will disbelief (if interacted with); varies; see text}
\shortability{Spell Resistance:}{Yes; see text}
\smallskip\noindent You use material from the Plane of Shadow to shape quasi-real illusions of one or more creatures, objects, or forces. Shadow conjuration can mimic any sorcerer or wizard conjuration (summoning) or conjuration (creation) spell of 3rd level or lower.

\smallskip\noindent Shadow conjurations are actually one-fifth (20\%) as strong as the real things, though creatures who believe the shadow conjurations to be real are affected by them at full strength.

\smallskip\noindent Any creature that interacts with the conjured object, force, or creature can make a Will save to recognize its true nature.

\smallskip\noindent Spells that deal damage have normal effects unless the affected creature succeeds on a Will save. Each disbelieving creature takes only one-fifth (20\%) damage from the attack. If the disbelieved attack has a special effect other than damage, that effect is only 20\% likely to occur. Regardless of the result of the save to disbelieve, an affected creature is also allowed any save that the spell being simulated allows, but the save DC is set according to shadow conjuration's level (4th) rather than the spell's normal level. In addition, any effect created by shadow conjuration allows spell resistance, even if the spell it is simulating does not. Shadow objects or substances have normal effects except against those who disbelieve them.

\smallskip\noindent Against disbelievers, they are 20\% likely to work.

\smallskip\noindent A shadow creature has one-fifth the hit points of a normal creature of its kind (regardless of whether it's recognized as shadowy). It deals normal damage and has all normal abilities and weaknesses. Against a creature that recognizes it as a shadow creature, however, the shadow creature's damage is one-fifth (20\%) normal, and all special abilities that do not deal lethal damage are only 20\% likely to work. (Roll for each use and each affected character separately.) Furthermore, the shadow creature's AC bonuses are one-fifth as large.

\smallskip\noindent A creature that succeeds on its save sees the shadow conjurations as transparent images superimposed on vague, shadowy forms.

\smallskip\noindent Objects automatically succeed on their Will saves against this spell.

\featname{Shadow Conjuration, Greater}
\shortability{Illusion}{(Shadow)}
\shortability{Level:}{Sor/Wiz 7}

\smallskip\noindent This spell functions like shadow conjuration, except that it can duplicate any sorcerer or wizard conjuration (summoning) or conjuration (creation) spell of 6th level or lower. The illusory conjurations created deal three-fifths (60\%) damage to nonbelievers, and nondamaging effects are 60\% likely to work against nonbelievers.

\featname{Shadow Evocation}
\shortability{Illusion}{(Shadow)}
\shortability{Level:}{Brd 5, Sor/Wiz 5}
\shortability{Components:}{V, S}
\shortability{Casting Time:}{1 standard action}
\shortability{Range:}{See text}
\shortability{Effect:}{See text}
\shortability{Duration:}{See text}
\shortability{Saving Throw:}{Will disbelief (if interacted with)}
\shortability{Spell Resistance:}{Yes}
\smallskip\noindent You tap energy from the Plane of Shadow to cast a quasi-real, illusory version of a sorcerer or wizard evocation spell of 4th level or lower. (For a spell with more than one level, use the best one applicable to you.)

\smallskip\noindent Spells that deal damage have normal effects unless an affected creature succeeds on a Will save. Each disbelieving creature takes only one-fifth damage from the attack. If the disbelieved attack has a special effect other than damage, that effect is one-fifth as strong (if applicable) or only 20\% likely to occur. If recognized as a shadow evocation, a damaging spell deals only one-fifth (20\%) damage. Regardless of the result of the save to disbelieve, an affected creature is also allowed any save (or spell resistance) that the spell being simulated allows, but the save DC is set according to shadow evocation's level (5th) rather than the spell's normal level.

\smallskip\noindent Nondamaging effects have normal effects except against those who disbelieve them. Against disbelievers, they have no effect.

\smallskip\noindent Objects automatically succeed on their Will saves against this spell.

\featname{Shadow Evocation, Greater}
\shortability{Illusion}{(Shadow)}
\shortability{Level:}{Sor/Wiz 8}
\smallskip\noindent This spell functions like shadow evocation, except that it enables you to create partially real, illusory versions of sorcerer or wizard evocation spells of 7th level or lower. If recognized as a greater shadow evocation, a damaging spell deals only three-fifths (60\%) damage.

\featname{Shadow Walk}
\shortability{Illusion}{(Shadow)}
\shortability{Level:}{Brd 5, Sor/Wiz 6}
\shortability{Components:}{V, S}
\shortability{Casting Time:}{1 standard action}
\shortability{Range:}{Touch}
\shortability{Targets:}{Up to one touched creature/ level}
\shortability{Duration:}{1 hour/level (D)}
\shortability{Saving Throw:}{Will negates}
\shortability{Spell Resistance:}{Yes}
\smallskip\noindent To use the shadow walk spell, you must be in an area of shadowy illumination. You and any creature you touch are then transported along a coiling path of shadowstuff to the edge of the Material Plane where it borders the Plane of Shadow. The effect is largely illusory, but the path is quasi-real. You can take more than one creature along with you (subject to your level limit), but all must be touching each other.

\smallskip\noindent In the region of shadow, you move at a rate of 50 miles per hour, moving normally on the borders of the Plane of Shadow but much more rapidly relative to the Material Plane. Thus, you can use this spell to travel rapidly by stepping onto the Plane of Shadow, moving the desired distance, and then stepping back onto the Material Plane.

\smallskip\noindent Because of the blurring of reality between the Plane of Shadow and the Material Plane, you can't make out details of the terrain or areas you pass over during transit, nor can you predict perfectly where your travel will end. It's impossible to judge distances accurately, making the spell virtually useless for scouting or spying. Furthermore, when the spell effect ends, you are shunted 1d10x100 feet in a random horizontal direction from your desired endpoint. If this would place you within a solid object, you are shunted 1d10x1,000 feet in the same direction. If this would still place you within a solid object, you (and any creatures with you) are shunted to the nearest empty space available, but the strain of this activity renders each creature fatigued (no save).

\smallskip\noindent Shadow walk can also be used to travel to other planes that border on the Plane of Shadow, but this usage requires the transit of the Plane of Shadow to arrive at a border with another plane of reality. The transit of the Plane of Shadow requires 1d4 hours.

\smallskip\noindent Any creatures touched by you when shadow walk is cast also make the transition to the borders of the Plane of Shadow.

\smallskip\noindent They may opt to follow you, wander off through the plane, or stumble back into the Material Plane (50\% chance for either of the latter results if they are lost or abandoned by you). Creatures unwilling to accompany you into the Plane of Shadow receive a Will saving throw, negating the effect if successful.

\featname{Shambler}
\shortability{Conjuration}{(Creation)}
\shortability{Level:}{Drd 9, Plant 9}
\shortability{Components:}{V, S}
\shortability{Casting Time:}{1 standard action}
\shortability{Range:}{Medium (100 ft. + 10 ft./level)}
\shortability{Effect:}{Three or more shambling mounds, no two of which can be more than 30 ft. apart; see text}
\shortability{Duration:}{Seven days or seven months (D); see text}
\shortability{Saving Throw:}{None}
\shortability{Spell Resistance:}{No}
\smallskip\noindent The shambler spell creates 1d4+2 shambling mounds with 11 HD each. The creatures willingly aid you in combat or battle, perform a specific mission, or serve as bodyguards. The creatures remain with you for seven days unless you dismiss them. If the shamblers are created only for guard duty, however, the duration of the spell is seven months. In this case, the shamblers can only be ordered to guard a specific site or location. Shamblers summoned to guard duty cannot move outside the spell's range, which is measured from the point where each first appeared. 

\smallskip\noindent The shamblers have resistance to fire as normal shambling mounds do only if the terrain is rainy, marshy, or damp.

\featname{Shapechange}
\shortability{Transmutation}{}
\shortability{Level:}{Animal 9, Drd 9, Sor/Wiz 9}
\shortability{Components:}{V, S, F}
\shortability{Casting Time:}{1 standard action}
\shortability{Range:}{Personal}
\shortability{Target:}{You}
\shortability{Duration:}{10 min./level (D)}
\smallskip\noindent This spell functions like polymorph, except that it enables you to assume the form of any single nonunique creature (of any type) from Fine to Colossal size. The assumed form cannot have more than your caster level in Hit Dice (to a maximum of 25 HD). Unlike polymorph, this spell allows incorporeal or gaseous forms to be assumed.

\smallskip\noindent You gain all extraordinary and supernatural abilities (both attacks and qualities) of the assumed form, but you lose your own supernatural abilities. You also gain the type of the new form in place of your own. The new form does not disorient you. Parts of your body or pieces of equipment that are separated from you do not revert to their original forms.

\smallskip\noindent You can become just about anything you are familiar with. You can change form once each round as a free action. The change takes place either immediately before your regular action or immediately after it, but not during the action. If you use this spell to create a disguise, you get a +10 bonus on your Disguise check.

\shortability{Focus:}{A jade circlet worth no less than 1,500 gp, which you must place on your head when casting the spell. (The focus melds into your new form when you change shape.)}

\featname{Shatter}
\shortability{Evocation}{[Sonic]}
\shortability{Level:}{Brd 2, Chaos 2, Clr 2, Destruction 2, Sor/Wiz 2}
\shortability{Components:}{V, S, M/DF}
\shortability{Casting Time:}{1 standard action}
\shortability{Range:}{Close (25 ft. + 5 ft./2 levels)}
Area or \shortability{Target:}{5-ft.-radius spread; or one solid object or one crystalline creature}
\shortability{Duration:}{Instantaneous}
\shortability{Saving Throw:}{Will negates (object); Will negates (object) or Fortitude half; see text}
\shortability{Spell Resistance:}{Yes (object)}
\smallskip\noindent Shatter creates a loud, ringing noise that breaks brittle, nonmagical objects; sunders a single solid, nonmagical object; or damages a crystalline creature.

\smallskip\noindent Used as an area attack, shatter destroys nonmagical objects of crystal, glass, ceramic, or porcelain. All such objects within a 5-foot radius of the point of origin are smashed into dozens of pieces by the spell. Objects weighing more than 1 pound per your level are not affected, but all other objects of the appropriate composition are shattered.

\smallskip\noindent Alternatively, you can target shatter against a single solid object, regardless of composition, weighing up to 10 pounds per caster level. Targeted against a crystalline creature (of any weight), shatter deals 1d6 points of sonic damage per caster level (maximum 10d6), with a Fortitude save for half damage.

\shortability{Arcane Material Component:}{A chip of mica.}

\featname{Shield}
\shortability{Abjuration}{[Force]}
\shortability{Level:}{Sor/Wiz 1}
\shortability{Components:}{V, S}
\shortability{Casting Time:}{1 standard action}
\shortability{Range:}{Personal}
\shortability{Target:}{You}
\shortability{Duration:}{1 min./level (D)}
\smallskip\noindent Shield creates an invisible, tower shield-sized mobile disk of force that hovers in front of you. It negates magic missile attacks directed at you. The disk also provides a +4 shield bonus to AC. This bonus applies against incorporeal touch attacks, since it is a force effect. The shield has no armor check penalty or arcane spell failure chance. Unlike with a normal tower shield, you can't use the shield spell for cover.

\featname{Shield of Faith}
\shortability{Abjuration}{}
\shortability{Level:}{Clr 1}
\shortability{Components:}{V, S, M}
\shortability{Casting Time:}{1 standard action}
\shortability{Range:}{Touch}
\shortability{Target:}{Creature touched}
\shortability{Duration:}{1 min./level}
\shortability{Saving Throw:}{Will negates (harmless)}
\shortability{Spell Resistance:}{Yes (harmless)}
\smallskip\noindent This spell creates a shimmering, magical field around the touched creature that averts attacks. The spell grants the subject a +2 deflection bonus to AC, with an additional +1 to the bonus for every six levels you have (maximum +5 deflection bonus at 18th level).

\shortability{Material Component:}{A small parchment with a bit of holy text written upon it.}

\featname{Shield of Law}
\shortability{Abjuration}{[Lawful]}
\shortability{Level:}{Clr 8, Law 8}
\shortability{Components:}{V, S, F}
\shortability{Casting Time:}{1 standard action}
\shortability{Range:}{20 ft.}
\shortability{Targets:}{One creature/level in a 20-ft.-radius burst centered on you}
\shortability{Duration:}{1 round/level (D)}
\shortability{Saving Throw:}{See text}
\shortability{Spell Resistance:}{Yes (harmless)}
\smallskip\noindent A dim, blue glow surrounds the subjects, protecting them from attacks, granting them resistance to spells cast by chaotic creatures, and slowing chaotic creatures when they strike the subjects. This abjuration has four effects.

\smallskip\noindent First, each warded creature gains a +4 deflection bonus to AC and a +4 resistance bonus on saves. Unlike protection from chaos, this benefit applies against all attacks, not just against attacks by chaotic creatures.

\smallskip\noindent Second, a warded creature gains spell resistance 25 against chaotic spells and spells cast by chaotic creatures.

\smallskip\noindent Third, the abjuration blocks possession and mental influence, just as protection from chaos does.

\smallskip\noindent Finally, if a chaotic creature succeeds on a melee attack against a warded creature, the attacker is slowed (Will save negates, as the slow spell, but against shield of law's save DC).

\shortability{Focus:}{A tiny reliquary containing some sacred relic, such as a scrap of parchment from a lawful text. The reliquary costs at least 500 gp.}

\featname{Shield Other}
\shortability{Abjuration}{}
\shortability{Level:}{Clr 2, Pal 2, Protection 2}
\shortability{Components:}{V, S, F}
\shortability{Casting Time:}{1 standard action}
\shortability{Range:}{Close (25 ft. + 5 ft./2 levels)}
\shortability{Target:}{One creature}
\shortability{Duration:}{1 hour/level (D)}
\shortability{Saving Throw:}{Will negates (harmless)}
\shortability{Spell Resistance:}{Yes (harmless)}
\smallskip\noindent This spell wards the subject and creates a mystic connection between you and the subject so that some of its wounds are transferred to you. The subject gains a +1 deflection bonus to AC and a +1 resistance bonus on saves. Additionally, the subject takes only half damage from all wounds and attacks (including that dealt by special abilities) that deal hit point damage. The amount of damage not taken by the warded creature is taken by you. Forms of harm that do not involve hit points, such as charm effects, temporary ability damage, level draining, and death effects, are not affected. If the subject suffers a reduction of hit points from a lowered Constitution score, the reduction is not split with you because it is not hit point damage. When the spell ends, subsequent damage is no longer divided between the subject and you, but damage already split is not reassigned to the subject.

\smallskip\noindent If you and the subject of the spell move out of range of each other, the spell ends.

\shortability{Focus:}{A pair of platinum rings (worth at least 50 gp each) worn by both you and the warded creature.}

\featname{Shillelagh}
\shortability{Transmutation}{}
\shortability{Level:}{Drd 1}
\shortability{Components:}{V, S, DF}
\shortability{Casting Time:}{1 standard action}
\shortability{Range:}{Touch}
\shortability{Target:}{One touched nonmagical oak club or quarterstaff}
\shortability{Duration:}{1 min./level}
\shortability{Saving Throw:}{Will negates (object)}
\shortability{Spell Resistance:}{Yes (object)}
\smallskip\noindent Your own nonmagical club or quarterstaff becomes a weapon with a +1 enhancement bonus on attack and damage rolls. (A quarterstaff gains this enhancement for both ends of the weapon.) It deals damage as if it were two size categories larger (a Small club or quarterstaff so transmuted deals 1d8 points of damage, a Medium 2d6, and a Large 3d6), +1 for its enhancement bonus. These effects only occur when the weapon is wielded by you. If you do not wield it, the weapon behaves as if unaffected by this spell.

\featname{Shocking Grasp}
\shortability{Evocation}{[Electricity]}
\shortability{Level:}{Sor/Wiz 1}
\shortability{Components:}{V, S}
\shortability{Casting Time:}{1 standard action}
\shortability{Range:}{Touch}
\shortability{Target:}{Creature or object touched}
\shortability{Duration:}{Instantaneous}
\shortability{Saving Throw:}{None}
\shortability{Spell Resistance:}{Yes}
\smallskip\noindent Your successful melee touch attack deals 1d6 points of electricity damage per caster level (maximum 5d6). When delivering the jolt, you gain a +3 bonus on attack rolls if the opponent is wearing metal armor (or made out of metal, carrying a lot of metal, or the like).

\featname{Shout}
\shortability{Evocation}{[Sonic]}
\shortability{Level:}{Brd 4, Sor/Wiz 4}
\shortability{Components:}{V}
\shortability{Casting Time:}{1 standard action}
\shortability{Range:}{30 ft.}
\shortability{Area:}{Cone-shaped burst}
\shortability{Duration:}{Instantaneous}
\shortability{Saving Throw:}{Fortitude partial or Reflex negates (object); see text}
\shortability{Spell Resistance:}{Yes (object)}
\smallskip\noindent You emit an ear-splitting yell that deafens and damages creatures in its path. Any creature within the area is deafened for 2d6 rounds and takes 5d6 points of sonic damage. A successful save negates the deafness and reduces the damage by half. Any exposed brittle or crystalline object or crystalline creature takes 1d6 points of sonic damage per caster level (maximum 15d6). An affected creature is allowed a Fortitude save to reduce the damage by half, and a creature holding fragile objects can negate damage to them with a successful Reflex save.

\smallskip\noindent A shout spell cannot penetrate a silence spell.

\featname{Shout, Greater}
\shortability{Evocation}{[Sonic]}
\shortability{Level:}{Brd 6, Sor/Wiz 8}
\shortability{Components:}{V, S, F}
\shortability{Range:}{60 ft.}
\shortability{Saving Throw:}{Fortitude partial or Reflex negates (object); see text}
\smallskip\noindent This spell functions like shout, except that the cone deals 10d6 points of sonic damage (or 1d6 points of sonic damage per caster level, maximum 20d6, against exposed brittle or crystalline objects or crystalline creatures). It also causes creatures to be stunned for 1 round and deafened for 4d6 rounds. A creature in the area of the cone can negate the stunning and halve both the damage and the duration of the deafness with a successful Fortitude save. A creature holding vulnerable objects can attempt a Reflex save to negate the damage to those objects.

\shortability{Arcane Focus:}{A small metal or ivory horn.}

\featname{Shrink Item}
\shortability{Transmutation}{}
\shortability{Level:}{Sor/Wiz 3}
\shortability{Components:}{V, S}
\shortability{Casting Time:}{1 standard action}
\shortability{Range:}{Touch}
\shortability{Target:}{One touched object of up to 2 cu. ft./level}
\shortability{Duration:}{One day/level; see text}
\shortability{Saving Throw:}{Will negates (object)}
\shortability{Spell Resistance:}{Yes (object)}
\smallskip\noindent You are able to shrink one nonmagical item (if it is within the size limit) to 1/16 of its normal size in each dimension (to about 1/4,000 the original volume and mass). This change effectively reduces the object's size by four categories. Optionally, you can also change its now shrunken composition to a clothlike one. Objects changed by a shrink item spell can be returned to normal composition and size merely by tossing them onto any solid surface or by a word of command from the original caster. Even a burning fire and its fuel can be shrunk by this spell. Restoring the shrunken object to its normal size and composition ends the spell.

\smallskip\noindent Shrink item can be made permanent with a permanency spell, in which case the affected object can be shrunk and expanded an indefinite number of times, but only by the original caster.

\featname{Silence}
\shortability{Illusion}{(Glamer)}
\shortability{Level:}{Brd 2, Clr 2}
\shortability{Components:}{V, S}
\shortability{Casting Time:}{1 standard action}
\shortability{Range:}{Long (400 ft. + 40 ft./level)}
\shortability{Area:}{20-ft.-radius emanation centered on a creature, object, or point in space}
\shortability{Duration:}{1 min./level (D)}
\shortability{Saving Throw:}{Will negates; see text or none (object)}
\shortability{Spell Resistance:}{Yes; see text or no (object)}
\smallskip\noindent Upon the casting of this spell, complete silence prevails in the affected area. All sound is stopped: Conversation is impossible, spells with verbal components cannot be cast, and no noise whatsoever issues from, enters, or passes through the area. The spell can be cast on a point in space, but the effect is stationary unless cast on a mobile object. The spell can be centered on a creature, and the effect then radiates from the creature and moves as it moves. An unwilling creature can attempt a Will save to negate the spell and can use spell resistance, if any. Items in a creature's possession or magic items that emit sound receive the benefits of saves and spell resistance, but unattended objects and points in space do not. This spell provides a defense against sonic or language-based attacks.

\featname{Silent Image}
\shortability{Illusion}{(Figment)}
\shortability{Level:}{Brd 1, Sor/Wiz 1}
\shortability{Components:}{V, S, F}
\shortability{Casting Time:}{1 standard action}
\shortability{Range:}{Long (400 ft. + 40 ft./level)}
\shortability{Effect:}{Visual figment that cannot extend beyond four 10-ft. cubes + one 10-ft. cube/level (S)}
\shortability{Duration:}{Concentration}
\shortability{Saving Throw:}{Will disbelief (if interacted with)}
\shortability{Spell Resistance:}{No}
\smallskip\noindent This spell creates the visual illusion of an object, creature, or force, as visualized by you. The illusion does not create sound, smell, texture, or temperature. You can move the image within the limits of the size of the effect.

\shortability{Focus:}{A bit of fleece.}

\featname{Simulacrum}
\shortability{Illusion}{(Shadow)}
\shortability{Level:}{Sor/Wiz 7}
\shortability{Components:}{V, S, M, XP}
\shortability{Casting Time:}{12 hours}
\shortability{Range:}{0 ft.}
\shortability{Effect:}{One duplicate creature}
\shortability{Duration:}{Instantaneous}
\shortability{Saving Throw:}{None}
\shortability{Spell Resistance:}{No}
\smallskip\noindent Simulacrum creates an illusory duplicate of any creature. The duplicate creature is partially real and formed from ice or snow. It appears to be the same as the original, but it has only one-half of the real creature's levels or Hit Dice (and the appropriate hit points, feats, skill ranks, and special abilities for a creature of that level or HD). You can't create a simulacrum of a creature whose Hit Dice or levels exceed twice your caster level. You must make a Disguise check when you cast the spell to determine how good the likeness is. A creature familiar with the original might detect the ruse with a successful Spot check (opposed by the caster's Disguise check) or a DC 20 Sense Motive check.

\smallskip\noindent At all times the simulacrum remains under your absolute command. No special telepathic link exists, so command must be exercised in some other manner. A simulacrum has no ability to become more powerful. It cannot increase its level or abilities. If reduced to 0 hit points or otherwise destroyed, it reverts to snow and melts instantly into nothingness. A complex process requiring at least 24 hours, 100 gp per hit point, and a fully equipped magical laboratory can repair damage to a simulacrum.

\shortability{Material Component:}{The spell is cast over the rough snow or ice form, and some piece of the creature to be duplicated (hair, nail, or the like) must be placed inside the snow or ice. Additionally, the spell requires powdered ruby worth 100 gp per HD of the simulacrum to be created.}
\shortability{XP Cost:}{100 XP per HD of the simulacrum to be created (minimum 1,000 XP).}

\featname{Slay Living}
\shortability{Necromancy}{[Death]}
\shortability{Level:}{Clr 5, Death 5}
\shortability{Components:}{V, S}
\shortability{Casting Time:}{1 standard action}
\shortability{Range:}{Touch}
\shortability{Target:}{Living creature touched}
\shortability{Duration:}{Instantaneous}
\shortability{Saving Throw:}{Fortitude partial}
\shortability{Spell Resistance:}{Yes}
\smallskip\noindent You can slay any one living creature. You must succeed on a melee touch attack to touch the subject, and it can avoid death with a successful Fortitude save. If it succeeds, it instead takes 3d6 points of damage +1 point per caster level.

\featname{Sleep}
\shortability{Enchantment}{(Compulsion) [Mind-Affecting]}
\shortability{Level:}{Brd 1, Sor/Wiz 1}
\shortability{Components:}{V, S, M}
\shortability{Casting Time:}{1 round}
\shortability{Range:}{Medium (100 ft. + 10 ft./level)}
\shortability{Area:}{One or more living creatures within a 10-ft.-radius burst}
\shortability{Duration:}{1 min./level}
\shortability{Saving Throw:}{Will negates}
\shortability{Spell Resistance:}{Yes}
\smallskip\noindent A sleep spell causes a magical slumber to come upon 4 Hit Dice of creatures. Creatures with the fewest HD are affected first.

\smallskip\noindent Among creatures with equal HD, those who are closest to the spell's point of origin are affected first. Hit Dice that are not sufficient to affect a creature are wasted.

\smallskip\noindent Sleeping creatures are helpless. Slapping or wounding awakens an affected creature, but normal noise does not. Awakening a creature is a standard action (an application of the aid another action).

\smallskip\noindent Sleep does not target unconscious creatures, constructs, or undead creatures.

\shortability{Material Component:}{A pinch of fine sand, rose petals, or a live cricket.}

\featname{Sleet Storm}
\shortability{Conjuration}{(Creation) [Cold]}
\shortability{Level:}{Drd 3, Sor/Wiz 3}
\shortability{Components:}{V, S, M/DF}
\shortability{Casting Time:}{1 standard action}
\shortability{Range:}{Long (400 ft. + 40 ft./level)}
\shortability{Area:}{Cylinder (40-ft. radius, 20 ft. high)}
\shortability{Duration:}{1 round/level}
\shortability{Saving Throw:}{None}
\shortability{Spell Resistance:}{No}
\smallskip\noindent Driving sleet blocks all sight (even darkvision) within it and causes the ground in the area to be icy. A creature can walk within or through the area of sleet at half normal speed with a DC 10 Balance check. Failure means it can't move in that round, while failure by 5 or more means it falls (see the Balance skill for details).

\smallskip\noindent The sleet extinguishes torches and small fires.

\shortability{Arcane Material Component:}{A pinch of dust and a few drops of water.}

\featname{Slow}
\shortability{Transmutation}{}
\shortability{Level:}{Brd 3, Sor/Wiz 3}
\shortability{Components:}{V, S, M}
\shortability{Casting Time:}{1 standard action}
\shortability{Range:}{Close (25 ft. + 5 ft./2 levels)}
\shortability{Targets:}{One creature/level, no two of which can be more than 30 ft. apart}
\shortability{Duration:}{1 round/level}
\shortability{Saving Throw:}{Will negates}
\shortability{Spell Resistance:}{Yes}
\smallskip\noindent An affected creature moves and attacks at a drastically slowed rate. A slowed creature can take only a single move action or standard action each turn, but not both (nor may it take full-round actions). Additionally, it takes a -1 penalty on attack rolls, AC, and Reflex saves. A slowed creature moves at half its normal speed (round down to the next 5-foot increment), which affects the creature's jumping distance as normal for decreased speed.

\smallskip\noindent Multiple slow effects don't stack. Slow counters and dispels haste.

\shortability{Material Component:}{A drop of molasses.}

\featname{Snare}
\shortability{Transmutation}{}
\shortability{Level:}{Rgr 2, Drd 3}
\shortability{Components:}{V, S, DF}
\shortability{Casting Time:}{3 rounds}
\shortability{Range:}{Touch}
\shortability{Target:}{Touched nonmagical circle of vine, rope, or thong with a 2 ft. diameter + 2 ft./level}
\shortability{Duration:}{Until triggered or broken}
\shortability{Saving Throw:}{None}
\shortability{Spell Resistance:}{No}
\smallskip\noindent This spell enables you to make a snare that functions as a magic trap. The snare can be made from any supple vine, a thong, or a rope. When you cast snare upon it, the cordlike object blends with its surroundings (Search DC 23 for a character with the trapfinding ability to locate). One end of the snare is tied in a loop that contracts around one or more of the limbs of any creature stepping inside the circle.

\smallskip\noindent If a strong and supple tree is nearby, the snare can be fastened to it. The spell causes the tree to bend and then straighten when the loop is triggered, dealing 1d6 points of damage to the creature trapped and lifting it off the ground by the trapped limb or limbs. If no such tree is available, the cordlike object tightens around the creature, dealing no damage but causing it to be entangled.

\smallskip\noindent The snare is magical. To escape, a trapped creature must make a DC 23 Escape Artist check or a DC 23 Strength check that is a full-round action. The snare has AC 7 and 5 hit points. A successful escape from the snare breaks the loop and ends the spell.

\featname{Soften Earth and Stone}
\shortability{Transmutation}{[Earth]}
\shortability{Level:}{Drd 2, Earth 2}
\shortability{Components:}{V, S, DF}
\shortability{Casting Time:}{1 standard action}
\shortability{Range:}{Close (25 ft. + 5 ft./2 levels)}
\shortability{Area:}{10-ft. square/level; see text}
\shortability{Duration:}{Instantaneous}
\shortability{Saving Throw:}{None}
\shortability{Spell Resistance:}{No}
\smallskip\noindent When this spell is cast, all natural, undressed earth or stone in the spell's area is softened. Wet earth becomes thick mud, dry earth becomes loose sand or dirt, and stone becomes soft clay that is easily molded or chopped. You affect a 10-footsquare area to a depth of 1 to 4 feet, depending on the toughness or resilience of the ground at that spot. Magical, enchanted, dressed, or worked stone cannot be affected. Earth or stone creatures are not affected.

\smallskip\noindent A creature in mud must succeed on a Reflex save or be caught for 1d2 rounds and unable to move, attack, or cast spells. A creature that succeeds on its save can move through the mud at half speed, and it can't run or charge.

\smallskip\noindent Loose dirt is not as troublesome as mud, but all creatures in the area can move at only half their normal speed and can't run or charge over the surface.

\smallskip\noindent Stone softened into clay does not hinder movement, but it does allow characters to cut, shape, or excavate areas they may not have been able to affect before.

\smallskip\noindent While soften earth and stone does not affect dressed or worked stone, cavern ceilings or vertical surfaces such as cliff faces can be affected. Usually, this causes a moderate collapse or landslide as the loosened material peels away from the face of the wall or roof and falls.

\smallskip\noindent A moderate amount of structural damage can be dealt to a manufactured structure by softening the ground beneath it, causing it to settle. However, most well-built structures will only be damaged by this spell, not destroyed.

\featname{Solid Fog}
\shortability{Conjuration}{(Creation)}
\shortability{Level:}{Sor/Wiz 4}
\shortability{Components:}{V, S, M}
\shortability{Duration:}{1 min./level}
\shortability{Spell Resistance:}{No}
\smallskip\noindent This spell functions like fog cloud, but in addition to obscuring sight, the solid fog is so thick that any creature attempting to move through it progresses at a speed of 5 feet, regardless of its normal speed, and it takes a -2 penalty on all melee attack and melee damage rolls. The vapors prevent effective ranged weapon attacks (except for magic rays and the like). A creature or object that falls into solid fog is slowed, so that each 10 feet of vapor that it passes through reduces falling damage by 1d6. A creature can't take a 5-foot step while in solid fog.

\smallskip\noindent However, unlike normal fog, only a severe wind (31+ mph) disperses these vapors, and it does so in 1 round.

\smallskip\noindent Solid fog can be made permanent with a permanency spell. A permanent solid fog dispersed by wind reforms in 10 minutes.

\shortability{Material Component:}{A pinch of dried, powdered peas combined with powdered animal hoof.}

\featname{Song of Discord}
\shortability{Enchantment}{(Compulsion) [Mind-Affecting, Sonic]}
\shortability{Level:}{Brd 5}
\shortability{Components:}{V, S}
\shortability{Casting Time:}{1 standard action}
\shortability{Range:}{Medium (100 ft. + 10 ft./level)}
\shortability{Area:}{Creatures within a 20-ft.-radius spread}
\shortability{Duration:}{1 round/level}
\shortability{Saving Throw:}{Will negates}
\shortability{Spell Resistance:}{Yes}
\smallskip\noindent This spell causes those within the area to turn on each other rather than attack their foes. Each affected creature has a 50\% chance to attack the nearest target each round. (Roll to determine each creature's behavior every round at the beginning of its turn.) A creature that does not attack its nearest neighbor is free to act normally for that round.

\smallskip\noindent Creatures forced by a song of discord to attack their fellows employ all methods at their disposal, choosing their deadliest spells and most advantageous combat tactics. They do not, however, harm targets that have fallen unconscious.

\featname{Soul Bind}
\shortability{Necromancy}{}
\shortability{Level:}{Clr 9, Sor/Wiz 9}
\shortability{Components:}{V, S, F}
\shortability{Casting Time:}{1 standard action}
\shortability{Range:}{Close (25 ft. + 5 ft./2 levels)}
\shortability{Target:}{Corpse}
\shortability{Duration:}{Permanent}
\shortability{Saving Throw:}{Will negates}
\shortability{Spell Resistance:}{No}
\smallskip\noindent You draw the soul from a newly dead body and imprison it in a black sapphire gem. The subject must have been dead no more than 1 round per caster level. The soul, once trapped in the gem, cannot be returned through clone, raise dead, reincarnation, resurrection, true resurrection, or even a miracle or a wish. Only by destroying the gem or dispelling the spell on the gem can one free the soul (which is then still dead).

\shortability{Focus:}{A black sapphire of at least 1,000 gp value for every Hit Die possessed by the creature whose soul is to be bound. If the gem is not valuable enough, it shatters when the binding is attempted. (While creatures have no concept of level or Hit Dice as such, the value of the gem needed to trap an individual can be researched. Remember that this value can change over time as creatures gain more Hit Dice.)}

\featname{Sound Burst}
\shortability{Evocation}{[Sonic]}
\shortability{Level:}{Brd 2, Clr 2}
\shortability{Components:}{V, S, F/DF}
\shortability{Casting Time:}{1 standard action}
\shortability{Range:}{Close (25 ft. + 5 ft./2 levels)}
\shortability{Area:}{10-ft.-radius spread}
\shortability{Duration:}{Instantaneous}
\shortability{Saving Throw:}{Fortitude partial}
\shortability{Spell Resistance:}{Yes}
\smallskip\noindent You blast an area with a tremendous cacophony. Every creature in the area takes 1d8 points of sonic damage and must succeed on a Fortitude save to avoid being stunned for 1 round. 

\smallskip\noindent Creatures that cannot hear are not stunned but are still damaged.

\shortability{Arcane Focus:}{A musical instrument.}

\featname{Speak with Animals}
\shortability{Divination}{}
\shortability{Level:}{Brd 3, Drd 1, Rgr 1}
\shortability{Components:}{V, S}
\shortability{Casting Time:}{1 standard action}
\shortability{Range:}{Personal}
\shortability{Target:}{You}
\shortability{Duration:}{1 min./level}
\smallskip\noindent You can comprehend and communicate with animals. You are able to ask questions of and receive answers from animals, although the spell doesn't make them any more friendly or cooperative than normal. Furthermore, wary and cunning animals are likely to be terse and evasive, while the more stupid ones make inane comments. If an animal is friendly toward you, it may do some favor or service for you.

\featname{Speak with Dead}
\shortability{Necromancy}{[Language-Dependent]}
\shortability{Level:}{Clr 3}
\shortability{Components:}{V, S, DF}
\shortability{Casting Time:}{10 minutes}
\shortability{Range:}{10 ft.}
\shortability{Target:}{One dead creature}
\shortability{Duration:}{1 min./level}
\shortability{Saving Throw:}{Will negates; see text}
\shortability{Spell Resistance:}{No}
\smallskip\noindent You grant the semblance of life and intellect to a corpse, allowing it to answer several questions that you put to it. You may ask one question per two caster levels. Unasked questions are wasted if the duration expires. The corpse's knowledge is limited to what the creature knew during life, including the languages it spoke (if any). Answers are usually brief, cryptic, or repetitive. If the creature's alignment was different from yours, the corpse gets a Will save to resist the spell as if it were alive.

\smallskip\noindent If the corpse has been subject to speak with dead within the past week, the new spell fails. You can cast this spell on a corpse that has been deceased for any amount of time, but the body must be mostly intact to be able to respond. A damaged corpse may be able to give partial answers or partially correct answers, but it must at least have a mouth in order to speak at all. 

\smallskip\noindent This spell does not let you actually speak to the person (whose soul has departed). It instead draws on the imprinted knowledge stored in the corpse. The partially animated body retains the imprint of the soul that once inhabited it, and thus it can speak with all the knowledge that the creature had while alive. The corpse, however, cannot learn new information.

\smallskip\noindent Indeed, it can't even remember being questioned.

\smallskip\noindent This spell does not affect a corpse that has been turned into an undead creature.

\featname{Speak with Plants}
\shortability{Divination}{}
\shortability{Level:}{Brd 4, Drd 3, Rgr 2}
\shortability{Components:}{V, S}
\shortability{Casting Time:}{1 standard action}
\shortability{Range:}{Personal}
\shortability{Target:}{You}
\shortability{Duration:}{1 min./level}
\smallskip\noindent You can comprehend and communicate with plants, including both normal plants and plant creatures. You are able to ask questions of and receive answers from plants. A regular plant's sense of its surroundings is limited, so it won't be able to give (or recognize) detailed descriptions of creatures or answer questions about events outside its immediate vicinity.

\smallskip\noindent The spell doesn't make plant creatures any more friendly or cooperative than normal. Furthermore, wary and cunning plant creatures are likely to be terse and evasive, while the more stupid ones may make inane comments. If a plant creature is friendly toward you, it may do some favor or service for you.

\featname{Spectral Hand}
\shortability{Necromancy}{}
\shortability{Level:}{Sor/Wiz 2}
\shortability{Components:}{V, S}
\shortability{Casting Time:}{1 standard action}
\shortability{Range:}{Medium (100 ft. + 10 ft./level)}
\shortability{Effect:}{One spectral hand}
\shortability{Duration:}{1 min./level (D)}
\shortability{Saving Throw:}{None}
\shortability{Spell Resistance:}{No}
\smallskip\noindent A ghostly, glowing hand shaped from your life force materializes and moves as you desire, allowing you to deliver low-level, touch range spells at a distance. On casting the spell, you lose 1d4 hit points that return when the spell ends (even if it is dispelled), but not if the hand is destroyed. (The hit points can be healed as normal.) For as long as the spell lasts, any touch range spell of 4th level or lower that you cast can be delivered by the spectral hand. The spell gives you a +2 bonus on your melee touch attack roll, and attacking with the hand counts normally as an attack. The hand always strikes from your direction. The hand cannot flank targets like a creature can. After it delivers a spell, or if the hand goes beyond the spell range, goes out of your sight, the hand returns to you and hovers.

\smallskip\noindent The hand is incorporeal and thus cannot be harmed by normal weapons. It has improved evasion (half damage on a failed Reflex save and no damage on a successful save), your save bonuses, and an AC of at least 22. Your Intelligence modifier applies to the hand's AC as if it were the hand's Dexterity modifier. The hand has 1 to 4 hit points, the same number that you lost in creating it.

\featname{Spell Immunity}
\shortability{Abjuration}{}
\shortability{Level:}{Clr 4, Protection 4, Strength 4}
\shortability{Components:}{V, S, DF}
\shortability{Casting Time:}{1 standard action}
\shortability{Range:}{Touch}
\shortability{Target:}{Creature touched}
\shortability{Duration:}{10 min./level}
\shortability{Saving Throw:}{Will negates (harmless)}
\shortability{Spell Resistance:}{Yes (harmless)}
\smallskip\noindent The warded creature is immune to the effects of one specified spell for every four levels you have. The spells must be of 4th level or lower. The warded creature effectively has unbeatable spell resistance regarding the specified spell or spells. Naturally, that immunity doesn't protect a creature from spells for which spell resistance doesn't apply. Spell immunity protects against spells, spell-like effects of magic items, and innate spell-like abilities of creatures. It does not protect against supernatural or extraordinary abilities, such as breath weapons or gaze attacks.

\smallskip\noindent Only a particular spell can be protected against, not a certain domain or school of spells or a group of spells that are similar in effect. 

\smallskip\noindent A creature can have only one spell immunity or greater spell immunity spell in effect on it at a time.

\featname{Spell Immunity, Greater}
\shortability{Abjuration}{}
\shortability{Level:}{Clr 8}
\smallskip\noindent This spell functions like spell immunity, except the immunity applies to spells of 8th level or lower.

\smallskip\noindent A creature can have only one spell immunity or greater spell immunity spell in effect on it at a time.

\featname{Spell Resistance}
\shortability{Abjuration}{}
\shortability{Level:}{Clr 5, Magic 5, Protection 5}
\shortability{Components:}{V, S, DF}
\shortability{Casting Time:}{1 standard action}
\shortability{Range:}{Touch}
\shortability{Target:}{Creature touched}
\shortability{Duration:}{1 min./level}
\shortability{Saving Throw:}{Will negates (harmless)}
\shortability{Spell Resistance:}{Yes (harmless)}
\smallskip\noindent The creature gains spell resistance equal to 12 + your caster level.

\featname{Spellstaff}
\shortability{Transmutation}{}
\shortability{Level:}{Drd 6}
\shortability{Components:}{V, S, F}
\shortability{Casting Time:}{10 minutes}
\shortability{Range:}{Touch}
\shortability{Target:}{Wooden quarterstaff touched}
\shortability{Duration:}{Permanent until discharged (D)}
\shortability{Saving Throw:}{Will negates (object)}
\shortability{Spell Resistance:}{Yes (object)}
\smallskip\noindent You store one spell that you can normally cast in a wooden quarterstaff. Only one such spell can be stored in a staff at a given time, and you cannot have more than one spellstaff at any given time. You can cast a spell stored within a staff just as though it were among those you had prepared, but it does not count against your normal allotment for a given day. You use up any applicable material components required to cast the spell when you store it in the spellstaff.

\shortability{Focus:}{The staff that stores the spell.}

\featname{Spell Turning}
\shortability{Abjuration}{}
\shortability{Level:}{Luck 7, Magic 7, Sor/Wiz 7}
\shortability{Components:}{V, S, M/DF}
\shortability{Casting Time:}{1 standard action}
\shortability{Range:}{Personal}
\shortability{Target:}{You}
\shortability{Duration:}{Until expended or 10 min./level}
\smallskip\noindent Spells and spell-like effects targeted on you are turned back upon the original caster. The abjuration turns only spells that have you as a target. Effect and area spells are not affected. Spell turning also fails to stop touch range spells. 

\smallskip\noindent From seven to ten (1d4+6) spell levels are affected by the turning. The exact number is rolled secretly.

\smallskip\noindent When you are targeted by a spell of higher level than the amount of spell turning you have left, that spell is partially turned. The subtract the amount of spell turning left from the spell level of the incoming spell, then divide the result by the spell level of the incoming spell to see what fraction of the effect gets through. For damaging spells, you and the caster each take a fraction of the damage. For nondamaging spells, each of you has a proportional chance to be affected.

\smallskip\noindent If you and a spellcasting attacker are both warded by spell turning effects in operation, a resonating field is created.

\smallskip\noindent Roll randomly to determine the result.

\begin{tabular}[h!]{l|p{2.5in}}
d\% & Effect \\ \hline
01-70 & Spell drains away without effect.	\\
71-80 & Spell affects both of you equally at full effect.	\\
81-97 & Both turning effects are rendered nonfunctional for 1d4 minutes. \\
98-100 & Both of you go through a rift into another plane. \\
\end{tabular}

\shortability{Arcane Material Component:}{A small silver mirror.}

\featname{Spider Climb}
\shortability{Transmutation}{}
\shortability{Level:}{Drd 2, Sor/Wiz 2}
\shortability{Components:}{V, S, M}
\shortability{Casting Time:}{1 standard action}
\shortability{Range:}{Touch}
\shortability{Target:}{Creature touched}
\shortability{Duration:}{10 min./level}
\shortability{Saving Throw:}{Will negates (harmless)}
\shortability{Spell Resistance:}{Yes (harmless)}
\smallskip\noindent The subject can climb and travel on vertical surfaces or even traverse ceilings as well as a spider does. The affected creature must have its hands free to climb in this manner. The subject gains a climb speed of 20 feet; furthermore, it need not make Climb checks to traverse a vertical or horizontal surface (even upside down). A spider climbing creature retains its Dexterity bonus to Armor Class (if any) while climbing, and opponents get no special bonus to their attacks against it. It cannot, however, use the run action while climbing.

\shortability{Material Component:}{A drop of bitumen and a live spider, both of which must be eaten by the subject.}

\featname{Spike Growth}
\shortability{Transmutation}{}
\shortability{Level:}{Drd 3, Rgr 2}
\shortability{Components:}{V, S, DF}
\shortability{Casting Time:}{1 standard action}
\shortability{Range:}{Medium (100 ft. + 10 ft./level)}
\shortability{Area:}{One 20-ft. square/level}
\shortability{Duration:}{1 hour/level (D)}
\shortability{Saving Throw:}{Reflex partial}
\shortability{Spell Resistance:}{Yes}
\smallskip\noindent Any ground-covering vegetation in the spell's area becomes very hard and sharply pointed without changing its appearance.

\smallskip\noindent In areas of bare earth, roots and rootlets act in the same way. Typically, spike growth can be cast in any outdoor setting except open water, ice, heavy snow, sandy desert, or bare stone. Any creature moving on foot into or through the spell's area takes 1d4 points of piercing damage for each 5 feet of movement through the spiked area.

\smallskip\noindent Any creature that takes damage from this spell must also succeed on a Reflex save or suffer injuries to its feet and legs that slow its land speed by one-half. This speed penalty lasts for 24 hours or until the injured creature receives a cure spell (which also restores lost hit points). Another character can remove the penalty by taking 10 minutes to dress the injuries and succeeding on a Heal check against the spell's save DC.

\smallskip\noindent Spike growth can't be disabled with the Disable Device skill.

\shortabiliy{Note:}{Magic traps such as spike growth are hard to detect. A rogue (only) can use the Search skill to find a spike growth. The DC is 25 + spell level, or DC 28 for spike growth (or DC 27 for spike growth cast by a ranger).}

\featname{Spike Stones}
\shortability{Transmutation}{[Earth]}
\shortability{Level:}{Drd 4, Earth 4}
\shortability{Components:}{V, S, DF}
\shortability{Casting Time:}{1 standard action}
\shortability{Range:}{Medium (100 ft. + 10 ft./level)}
\shortability{Area:}{One 20-ft. square/level}
\shortability{Duration:}{1 hour/level (D)}
\shortability{Saving Throw:}{Reflex partial}
\shortability{Spell Resistance:}{Yes}
\smallskip\noindent Rocky ground, stone floors, and similar surfaces shape themselves into long, sharp points that blend into the background.

\smallskip\noindent Spike stones impede progress through an area and deal damage. Any creature moving on foot into or through the spell's area moves at half speed.

\smallskip\noindent In addition, each creature moving through the area takes 1d8 points of piercing damage for each 5 feet of movement through the spiked area.

\smallskip\noindent Any creature that takes damage from this spell must also succeed on a Reflex save to avoid injuries to its feet and legs. A failed save causes the creature's speed to be reduced to half normal for 24 hours or until the injured creature receives a cure spell (which also restores lost hit points). Another character can remove the penalty by taking 10 minutes to dress the injuries and succeeding on a Heal check against the spell's save DC.
Spike stones is a magic trap that can't be disabled with the Disable Device skill.

\shortbability{Note:}{Magic traps such as spike stones are hard to detect. A rogue (only) can use the Search skill to find spike stones. The DC is 25 + spell level, or DC 29 for spike stones.}

\featname{Spiritual Weapon}
\shortability{Evocation}{[Force]}
\shortability{Level:}{Clr 2, War 2}
\shortability{Components:}{V, S, DF}
\shortability{Casting Time:}{1 standard action}
\shortability{Range:}{Medium (100 ft. + 10 ft./level)}
\shortability{Effect:}{Magic weapon of force}
\shortability{Duration:}{1 round/level (D)}
\shortability{Saving Throw:}{None}
\shortability{Spell Resistance:}{Yes}
\smallskip\noindent A weapon made of pure force springs into existence and attacks opponents at a distance, as you direct it, dealing 1d8 force damage per hit, +1 point per three caster levels (maximum +5 at 15th level). The weapon takes the shape of a weapon favored by your deity or a weapon with some spiritual significance or symbolism to you (see below) and has the same threat range and critical multipliers as a real weapon of its form. It strikes the opponent you designate, starting with one attack in the round the spell is cast and continuing each round thereafter on your turn. It uses your base attack bonus (possibly allowing it multiple attacks per round in subsequent rounds) plus your Wisdom modifier as its attack bonus. It strikes as a spell, not as a weapon, so, for example, it can damage creatures that have damage reduction. As a force effect, it can strike incorporeal creatures without the normal miss chance associated with incorporeality. The weapon always strikes from your direction. It does not get a flanking bonus or help a combatant get one. Your feats or combat actions do not affect the weapon. If the weapon goes beyond the spell range, if it goes out of your sight, or if you are not directing it, the weapon returns to you and hovers.

\smallskip\noindent Each round after the first, you can use a move action to redirect the weapon to a new target. If you do not, the weapon continues to attack the previous round's target. On any round that the weapon switches targets, it gets one attack. Subsequent rounds of attacking that target allow the weapon to make multiple attacks if your base attack bonus would allow it to. Even if the spiritual weapon is a ranged weapon, use the spell's range, not the weapon's normal range increment, and switching targets still is a move action.

\smallskip\noindent A spiritual weapon cannot be attacked or harmed by physical attacks, but dispel magic, disintegrate, a sphere of annihilation, or a rod of cancellation affects it. A spiritual weapon's AC against touch attacks is 12 (10 + size bonus for Tiny object).

\smallskip\noindent If an attacked creature has spell resistance, you make a caster level check (1d20 + caster level) against that spell resistance the first time the spiritual weapon strikes it. If the weapon is successfully resisted, the spell is dispelled. If not, the weapon has its normal full effect on that creature for the duration of the spell.

\smallskip\noindent The weapon that you get is often a force replica of your deity's own personal weapon. A cleric without a deity gets a weapon based on his alignment. A neutral cleric without a deity can create a spiritual weapon of any alignment, provided he is acting at least generally in accord with that alignment at the time. The weapons associated with each alignment are as follows.

\shorability{Chaos:}{Battleaxe}
\shorability{Evil:}{Light flail}
\shorability{Good:}{Warhammer}
\shorability{Law:}{Longsword}

\featname{Statue}
\shortability{Transmutation}{}
\shortability{Level:}{Sor/Wiz 7}
\shortability{Components:}{V, S, M}
\shortability{Casting Time:}{1 round}
\shortability{Range:}{Touch}
\shortability{Target:}{Creature touched}
\shortability{Duration:}{1 hour/level (D)}
\shortability{Saving Throw:}{Will negates (harmless)}
\shortability{Spell Resistance:}{Yes (harmless)}
\smallskip\noindent A statue spell turns the subject to solid stone, along with any garments and equipment worn or carried. In statue form, the subject gains hardness 8. The subject retains its own hit points.

\smallskip\noindent The subject can see, hear, and smell normally, but it does not need to eat or breathe. Feeling is limited to those sensations that can affect the granite-hard substance of the individual's body. Chipping is equal to a mere scratch, but breaking off one of the statue's arms constitutes serious damage.

\smallskip\noindent The subject of a statue spell can return to its normal state, act, and then return instantly to the statue state (a free action) if it so desires, as long as the spell duration is in effect.

\shortability{Material Component:}{Lime, sand, and a drop of water stirred by an iron bar, such as a nail or spike.}

\featname{Status}
\shortability{Divination}{}
\shortability{Level:}{Clr 2}
\shortability{Components:}{V, S}
\shortability{Casting Time:}{1 standard action}
\shortability{Range:}{Touch}
\shortability{Targets:}{One living creature touched per three levels}
\shortability{Duration:}{1 hour/level}
\shortability{Saving Throw:}{Will negates (harmless)}
\shortability{Spell Resistance:}{Yes (harmless)}
\smallskip\noindent When you need to keep track of comrades who may get separated, status allows you to mentally monitor their relative positions and general condition. You are aware of direction and distance to the creatures and any conditions affecting them: unharmed, wounded, disabled, staggered, unconscious, dying, nauseated, panicked, stunned, poisoned, diseased, confused, or the like. Once the spell has been cast upon the subjects, the distance between them and the caster does not affect the spell as long as they are on the same plane of existence. If a subject leaves the plane, or if it dies, the spell ceases to function for it.

\featname{Stinking Cloud}
\shortability{Conjuration}{(Creation)}
\shortability{Level:}{Sor/Wiz 3}
\shortability{Components:}{V, S, M}
\shortability{Casting Time:}{1 standard action}
\shortability{Range:}{Medium (100 ft. + 10 ft./level)}
\shortability{Effect:}{Cloud spreads in 20-ft. radius, 20 ft. high}
\shortability{Duration:}{1 round/level}
\shortability{Saving Throw:}{Fortitude negates; see text}
\shortability{Spell Resistance:}{No}
\smallskip\noindent Stinking cloud creates a bank of fog like that created by fog cloud, except that the vapors are nauseating. Living creatures in the cloud become nauseated. This condition lasts as long as the creature is in the cloud and for 1d4+1 rounds after it leaves. (Roll separately for each nauseated character.) Any creature that succeeds on its save but remains in the cloud must continue to save each round on your turn.
Stinking cloud can be made permanent with a permanency spell. A permanent stinking cloud dispersed by wind reforms in 10 minutes.

\shortability{Material Component:}{A rotten egg or several skunk cabbage leaves.}

\featname{Stone Shape}
\shortability{Transmutation}{[Earth]}
\shortability{Level:}{Clr 3, Drd 3, Earth 3, Sor/Wiz 4}
\shortability{Components:}{V, S, M/DF}
\shortability{Casting Time:}{1 standard action}
\shortability{Range:}{Touch}
\shortability{Target:}{Stone or stone object touched, up to 10 cu. ft. + 1 cu. ft./level}
\shortability{Duration:}{Instantaneous}
\shortability{Saving Throw:}{None}
\shortability{Spell Resistance:}{No}
\smallskip\noindent You can form an existing piece of stone into any shape that suits your purpose. While it's possible to make crude coffers, doors, and so forth with stone shape, fine detail isn't possible. There is a 30\% chance that any shape including moving parts simply doesn't work.

\shortability{Arcane Material Component:}{Soft clay, which must be worked into roughly the desired shape of the stone object and then touched to the stone while the verbal component is uttered.}

\featname{Stoneskin}
\shortability{Abjuration}{}
\shortability{Level:}{Drd 5, Earth 6, Sor/Wiz 4, Strength 6}
\shortability{Components:}{V, S, M}
\shortability{Casting Time:}{1 standard action}
\shortability{Range:}{Touch}
\shortability{Target:}{Creature touched}
\shortability{Duration:}{10 min./level or until discharged}
\shortability{Saving Throw:}{Will negates (harmless)}
\shortability{Spell Resistance:}{Yes (harmless)}
\smallskip\noindent The warded creature gains resistance to blows, cuts, stabs, and slashes. The subject gains damage reduction 10/adamantine. (It ignores the first 10 points of damage each time it takes damage from a weapon, though an adamantine weapon bypasses the reduction.) Once the spell has prevented a total of 10 points of damage per caster level (maximum 150 points), it is discharged.

\shortability{Material Component:}{Granite and 250 gp worth of diamond dust sprinkled on the target's skin.}

\featname{Stone Tell}
\shortability{Divination}{}
\shortability{Level:}{Drd 6}
\shortability{Components:}{V, S, DF}
\shortability{Casting Time:}{10 minutes}
\shortability{Range:}{Personal}
\shortability{Target:}{You}
\shortability{Duration:}{1 min./level}
\smallskip\noindent You gain the ability to speak with stones, which relate to you who or what has touched them as well as revealing what is covered or concealed behind or under them. The stones relate complete descriptions if asked. A stone's perspective, perception, and knowledge may prevent the stone from providing the details you are looking for.

\smallskip\noindent You can speak with natural or worked stone.

\featname{Stone to Flesh}
\shortability{Transmutation}{}
\shortability{Level:}{Sor/Wiz 6}
\shortability{Components:}{V, S, M}
\shortability{Casting Time:}{1 standard action}
\shortability{Range:}{Medium (100 ft. + 10 ft./level)}
\shortability{Target:}{One petrified creature or a cylinder of stone from 1 ft. to 3 ft. in diameter and up to 10 ft. long}
\shortability{Duration:}{Instantaneous}
\shortability{Saving Throw:}{Fortitude negates (object); see text}
\shortability{Spell Resistance:}{Yes}

\smallskip\noindent This spell restores a petrified creature to its normal state, restoring life and goods. The creature must make a DC 15 Fortitude save to survive the process. Any petrified creature, regardless of size, can be restored.

\smallskip\noindent The spell also can convert a mass of stone into a fleshy substance. Such flesh is inert and lacking a vital life force unless a life force or magical energy is available. (For example, this spell would turn a stone golem into a flesh golem, but an ordinary statue would become a corpse.) You can affect an object that fits within a cylinder from 1 foot to 3 feet in diameter and up to 10 feet long or a cylinder of up to those dimensions in a larger mass of stone.

\shortability{Material Component:}{A pinch of earth and a drop of blood.}

\featname{Storm of Vengeance}
\shortability{Conjuration}{(Summoning)}
\shortability{Level:}{Drd 9, Clr 9}
\shortability{Components:}{V, S}
\shortability{Casting Time:}{1 round}
\shortability{Range:}{Long (400 ft. + 40 ft./level)}
\shortability{Effect:}{360-ft.-radius storm cloud}
\shortability{Duration:}{Concentration (maximum 10 rounds) (D)}
\shortability{Saving Throw:}{See text}
\shortability{Spell Resistance:}{Yes}
\smallskip\noindent This spell creates an enormous black storm cloud. Lightning and crashing claps of thunder appear within the storm. Each creature beneath the cloud must succeed on a Fortitude save or be deafened for 1d4x10 minutes.

\smallskip\noindent If you do not maintain concentration on the spell after casting it, the spell ends.  If you continue to concentrate, the spell generates additional effects in each following round, as noted below. Each effect occurs during your turn.

\shortabliy{2nd Round:}{Acid rains down in the area, dealing 1d6 points of acid damage (no save).}
\shortabliy{3rd Round:}{You call six bolts of lightning down from the cloud. You decide where the bolts strike. No two bolts may be directed at the same target. Each bolt deals 10d6 points of electricity damage. A creature struck can attempt a Reflex save for half damage.}
\shortabliy{4th Round:}{Hailstones rain down in the area, dealing 5d6 points of bludgeoning damage (no save).}
\shortabliy{5th through 10th Rounds:}{Violent rain and wind gusts reduce visibility. The rain obscures all sight, including darkvision, beyond 5 feet. A creature 5 feet away has concealment (attacks have a 20\% miss chance). Creatures farther away have total concealment (50\% miss chance, and the attacker cannot use sight to locate the target).}

\smallskip\noindent Speed is reduced by three-quarters.

\smallskip\noindent Ranged attacks within the area of the storm are impossible. Spells cast within the area are disrupted unless the caster succeeds on a Concentration check against a DC equal to the storm of vengeance's save DC + the level of the spell the caster is trying to cast.

\featname{Suggestion}
\shortability{Enchantment}{(Compulsion) [Language-Dependent, Mind-Affecting]}
\shortability{Level:}{Brd 2, Sor/Wiz 3}
\shortability{Components:}{V, M}
\shortability{Casting Time:}{1 standard action}
\shortability{Range:}{Close (25 ft. + 5 ft./2 levels)}
\shortability{Target:}{One living creature}
\shortability{Duration:}{1 hour/level or until completed}
\shortability{Saving Throw:}{Will negates}
\shortability{Spell Resistance:}{Yes}
\smallskip\noindent You influence the actions of the target creature by suggesting a course of activity (limited to a sentence or two). The suggestion must be worded in such a manner as to make the activity sound reasonable. Asking the creature to do some obviously harmful act automatically negates the effect of the spell. 

\smallskip\noindent The suggested course of activity can continue for the entire duration. If the suggested activity can be completed in a shorter time, the spell ends when the subject finishes what it was asked to do. You can instead specify conditions that will trigger a special activity during the duration. If the condition is not met before the spell duration expires, the activity is not performed.

\smallskip\noindent A very reasonable suggestion causes the save to be made with a penalty (such as -1 or -2).

\shortability{Material Component:}{A snake's tongue and either a bit of honeycomb or a drop of sweet oil.}

\featname{Suggestion, Mass}
\shortability{Enchantment}{(Compulsion) [Language-Dependent, Mind-Affecting]}
\shortability{Level:}{Brd 5, Sor/Wiz 6}
\shortability{Range:}{Medium (100 ft. + 10 ft./level)}
\shortability{Targets:}{One creature/level, no two of which can be more than 30 ft. apart}
\smallskip\noindent This spell functions like suggestion, except that it can affect more creatures. The same suggestion applies to all these creatures.

\featname{Summon Instrument}
\shortability{Conjuration}{(Summoning)}
\shortability{Level:}{Brd 0}
\shortability{Components:}{V, S}
\shortability{Casting Time:}{1 round}
\shortability{Range:}{0 ft.}
\shortability{Effect:}{One summoned handheld musical instrument}
\shortability{Duration:}{1 min./level (D)}
\shortability{Saving Throw:}{None}
\shortability{Spell Resistance:}{No}
\smallskip\noindent This spell summons one handheld musical instrument of your choice. This instrument appears in your hands or at your feet (your choice). The instrument is typical for its type. Only one instrument appears per casting, and it will play only for you. You can't summon an instrument too large to be held in two hands.

\featname{Summon Monster I}
\shortability{Conjuration}{(Summoning) [see text]}
\shortability{Level:}{Brd 1, Clr 1, Sor/Wiz 1}
\shortability{Components:}{V, S, F/DF}
\shortability{Casting Time:}{1 round}
\shortability{Range:}{Close (25 ft. + 5 ft./2 levels)}
\shortability{Effect:}{One summoned creature}
\shortability{Duration:}{1 round/level (D)}
\shortability{Saving Throw:}{None}
\shortability{Spell Resistance:}{No}
\smallskip\noindent This spell summons an extraplanar creature (typically an outsider, elemental, or magical beast native to another plane). It appears where you designate and acts immediately, on your turn. It attacks your opponents to the best of its ability. If you can communicate with the creature, you can direct it not to attack, to attack particular enemies, or to perform other actions.

\smallskip\noindent The spell conjures one of the creatures from the 1st-level list on the accompanying Summon Monster table. You choose which kind of creature to summon, and you can change that choice each time you cast the spell.

\smallskip\noindent A summoned monster cannot summon or otherwise conjure another creature, nor can it use any teleportation or planar travel abilities. Creatures cannot be summoned into an environment that cannot support them.

\smallskip\noindent When you use a summoning spell to summon an air, chaotic, earth, evil, fire, good, lawful, or water creature, it is a spell of that type.

\shortability{Arcane Focus:}{A tiny bag and a small (not necessarily lit) candle.}

\featname{Summon Monster II}
\shortability{Conjuration}{(Summoning) [see text for summon monster I]}
\shortability{Level:}{Brd 2, Clr 2, Sor/Wiz 2}
\shortability{Effect:}{One or more summoned creatures, no two of which can be more than 30 ft. apart}
\smallskip\noindent This spell functions like summon monster I, except that you can summon one creature from the 2nd-level list or 1d3 creatures of the same kind from the 1st-level list.

\featname{Summon Monster III}
\shortability{Conjuration}{(Summoning) [see text for summon monster I]}
\shortability{Level:}{Brd 3, Clr 3, Sor/Wiz 3}
\shortability{Effect:}{One or more summoned creatures, no two of which can be more than 30 ft. apart}
\smallskip\noindent This spell functions like summon monster I, except that you can summon one creature from the 3rd-level list, 1d3 creatures of the same kind from the 2nd-level list, or 1d4+1 creatures of the same kind from the 1st-level list.

\featname{Summon Monster IV}
\shortability{Conjuration}{(Summoning) [see text for summon monster I]}
\shortability{Level:}{Brd 4, Clr 4, Sor/Wiz 4}
\shortability{Effect:}{One or more summoned creatures, no two of which can be more than 30 ft. apart}
\smallskip\noindent This spell functions like summon monster I, except that you can summon one creature from the 4th-level list, 1d3 creatures of the same kind from the 3rd-level list, or 1d4+1 creatures of the same kind from a lower-level list.

\featname{Summon Monster V}
\shortability{Conjuration}{(Summoning) [see text for summon monster I]}
\shortability{Level:}{Brd 5, Clr 5, Sor/Wiz 5}
\shortability{Effect:}{One or more summoned creatures, no two of which can be more than 30 ft. apart}
\smallskip\noindent This spell functions like summon monster I, except that you can summon one creature from the 5th-level list, 1d3 creatures of the same kind from the 4th-level list, or 1d4+1 creatures of the same kind from a lower-level list.

\featname{Summon Monster VI}
\shortability{Conjuration}{(Summoning) [see text for summon monster I]}
\shortability{Level:}{Brd 6, Clr 6, Sor/Wiz 6}
\shortability{Effect:}{One or more summoned creatures, no two of which can be more than 30 ft. apart}
\smallskip\noindent This spell functions like summon monster I, except you can summon one creature from the 6th-level list, 1d3 creatures of the same kind from the 5th-level list, or 1d4+1 creatures of the same kind from a lower-level list.

\featname{Summon Monster VII}
\shortability{Conjuration}{(Summoning) [see text for summon monster I]}
\shortability{Level:}{Clr 7, Sor/Wiz 7}
\smallskip\noindent This spell functions like summon monster I, except that you can summon one creature from the 7th-level list, 1d3 creatures of the same kind from the 6th-level list, or 1d4+1 creatures of the same kind from a lower-level list.

\featname{Summon Monster VIII}
\shortability{Conjuration}{(Summoning) [see text for summon monster I]}
\shortability{Level:}{Clr 8, Sor/Wiz 8}
\smallskip\noindent This spell functions like summon monster I, except that you can summon one creature from the 8th-level list, 1d3 creatures of the same kind from the 7th-level list, or 1d4+1 creatures of the same kind from a lower-level list.

\featname{Summon Monster IX}
\shortability{Conjuration}{(Summoning) [see text for summon monster I]}
\shortability{Level:}{Chaos 9, Clr 9, Evil 9, Good 9, Law 9, Sor/Wiz 9}
\smallskip\noindent This spell functions like summon monster I, except that you can summon one creature from the 9th-level list, 1d3 creatures of the same kind from the 8th-level list, or 1d4+1 creatures of the same kind from a lower-level list.

%Fuck this is a big table, I'll have to figure out something to do with it
%Summon Monster		   
%1st Level		   
%Celestial dog	LG	   
%Celestial owl	LG	   
%Celestial giant fire beetle	NG	   
%Celestial porpoise1	NG	   
%Celestial badger	CG	   
%Celestial monkey	CG	   
%Fiendish dire rat	LE	   
%Fiendish raven	LE	   
%Fiendish monstrous centipede, Medium	NE	   
%Fiendish monstrous scorpion, Small	NE	   
%Fiendish hawk	CE	   
%Fiendish monstrous spider, Small	CE	   
%Fiendish octopus1	CE	   
%Fiendish snake, Small viper	CE	   
%2nd Level		   
%Celestial giant bee	LG	   
%Celestial giant bombardier beetle	NG	   
%Celestial riding dog	NG	   
%Celestial eagle	CG	   
%Lemure (devil)	LE	   
%Fiendish squid1	LE	   
%Fiendish wolf	LE	   
%Fiendish monstrous centipede, Large	NE	   
%Fiendish monstrous scorpion, Medium	NE	   
%Fiendish shark, Medium1	NE	   
%Fiendish monstrous spider, Medium	CE	   
%Fiendish snake, Medium viper	CE	   
%3rd Level		   
%Celestial black bear	LG	   
%Celestial bison	NG	   
%Celestial dire badger	CG	   
%Celestial hippogriff	CG	   
%Elemental, Small (any)	N	   
%Fiendish ape	LE	   
%Fiendish dire weasel	LE	   
%Hell hound	LE	   
%Fiendish snake, constrictor 	LE	   
%Fiendish boar	NE	   
%Fiendish dire bat	NE	   
%Fiendish monstrous centipede, Huge	NE	   
%Fiendish crocodile	CE	   
%Dretch (demon)	CE	   
%Fiendish snake, Large viper	CE	   
%Fiendish wolverine	CE	   
%4th Level		   
%Archon, lantern	LG	   
%Celestial giant owl	LG	   
%Celestial giant eagle	CG	   
%Celestial lion	CG	   
%Mephit (any)	N	   
%Fiendish dire wolf	LE	   
%Fiendish giant wasp	LE	   
%Fiendish giant praying mantis	NE	   
%Fiendish shark, Large1	NE	   
%Yeth hound	NE	   
%Fiendish monstrous spider, Large	CE	   
%Fiendish snake, Huge viper	CE	   
%Howler	CE	   
%5th Level		   
%Archon, hound	LG	   
%Celestial brown bear	LG	   
%Celestial giant stag beetle	NG	   
%Celestial sea cat1	NG	   
%Celestial griffon	CG	   
%Elemental, Medium (any)	N	   
%Achaierai	LE	   
%Devil, bearded	LE	   
%Fiendish deinonychus	LE	   
%Fiendish dire ape	LE	   
%Fiendish dire boar	NE	   
%Fiendish shark, Huge	NE	   
%Fiendish monstrous scorpion, Large	NE	   
%Shadow mastiff	NE	   
%Fiendish dire wolverine	CE	   
%Fiendish giant crocodile	CE	   
%Fiendish tiger	CE	   
%6th Level		   
%Celestial polar bear	LG	   
%Celestial orca whale1	NG	   
%Bralani (eladrin)	CG	   
%Celestial dire lion	CG	   
%Elemental, Large (any)	N	   
%Janni (genie)	N	   
%Chaos beast	CN	   
%Devil, chain	LE	   
%Xill	LE	   
%Fiendish monstrous centipede, Gargantuan	NE	   
%Fiendish rhinoceros	NE	   
%Fiendish elasmosaurus1	CE	   
%Fiendish monstrous spider, Huge	CE	   
%Fiendish snake, giant constrictor	CE	   
%7th Level		   
%Celestial elephant	LG	   
%Avoral (guardinal)	NG	   
%Celestial baleen whale1	NG	   
%Djinni (genie)	CG	   
%Elemental, Huge (any)	N	   
%Invisible stalker	N	   
%Devil, bone	LE	   
%Fiendish megaraptor	LE	   
%Fiendish monstrous scorpion, Huge	NE	   
%Babau (demon)	CE	   
%Fiendish giant octopus1	CE	   
%Fiendish girallon	CE	   
%8th Level		   
%Celestial dire bear	LG	   
%Celestial cachalot whale1	NG	   
%Celestial triceratops	NG	   
%Lillend	CG	   
%Elemental, greater (any)	N	   
%Fiendish giant squid1	LE	   
%Hellcat	LE	   
%Fiendish monstrous centipede, Colossal	NE	   
%Fiendish dire tiger	CE	   
%Fiendish monstrous spider, Gargantuan	CE	   
%Fiendish tyrannosaurus	CE	   
%Vrock (demon)	CE	   
%9th Level		   
%Couatl	LG	   
%Leonal (guardinal)	NG	   
%Celestial roc	CG	   
%Elemental, elder (any)	N	   
%Devil, barbed	LE	   
%Fiendish dire shark1	NE	   
%Fiendish monstrous scorpion, Gargantuan	NE	   
%Night hag	NE	   
%Bebilith (demon)	CE	   
%Fiendish monstrous spider, Colossal	CE	   
%Hezrou (demon)	CE	   
%1 May be summoned only into an aquatic or watery environment.	 

\featname{Summon Nature's Ally I}
\shortability{Conjuration}{(Summoning)}
\shortability{Level:}{Drd 1, Rgr 1}
\shortability{Components:}{V, S, DF}
\shortability{Casting Time:}{1 round}
\shortability{Range:}{Close (25 ft. + 5 ft./2 levels)}
\shortability{Effect:}{One summoned creature}
\shortability{Duration:}{1 round/level (D)}
\shortability{Saving Throw:}{None}
\shortability{Spell Resistance:}{No}
\smallskip\noindent This spell summons a natural creature. It appears where you designate and acts immediately, on your turn. It attacks your opponents to the best of its ability. If you can communicate with the creature, you can direct it not to attack, to attack particular enemies, or to perform other actions.

\smallskip\noindent A summoned monster cannot summon or otherwise conjure another creature, nor can it use any teleportation or planar travel abilities. Creatures cannot be summoned into an environment that cannot support them.

\smallskip\noindent The spell conjures one of the creatures from the 1st-level list on the accompanying Summon Nature's Ally table. You choose which kind of creature to summon, and you can change that choice each time you cast the spell. All the creatures on the table are neutral unless otherwise noted.

\featname{Summon Nature's Ally II}
\shortability{Conjuration}{(Summoning)}
\shortability{Level:}{Drd 2, Rgr 2}
\shortability{Effect:}{One or more creatures, no two of which can be more than 30 ft. apart}
\smallskip\noindent This spell functions like summon nature's ally I, except that you can summon one 2nd-level creature or 1d3 1st-level creatures of the same kind.

\featname{Summon Nature's Ally III}
\shortability{Conjuration}{(Summoning) [see text]}
\shortability{Level:}{Drd 3, Rgr 3}
\shortability{Effect:}{One or more creatures, no two of which can be more than 30 ft. apart}
\smallskip\noindent This spell functions like summon nature's ally I, except that you can summon one 3rd-level creature, 1d3 2nd-level creatures of the same kind, or 1d4+1 1st-level creatures of the same kind.

\smallskip\noindent When you use a summoning spell to summon an air, chaotic, earth, evil, fire, good, lawful, or water creature, it is a spell of that type. 

\featname{Summon Nature's Ally IV}
\shortability{Conjuration}{(Summoning) [see text]}
\shortability{Level:}{Animal 4, Drd 4, Rgr 4}
\shortability{Effect:}{One or more creatures, no two of which can be more than 30 ft. apart}
\smallskip\noindent This spell functions like summon nature's ally I, except that you can summon one 4th-level creature, 1d3 3rd-level creatures of the same kind, or 1d4+1 lower-level creatures of the same kind.

\smallskip\noindent When you use a summoning spell to summon an air, chaotic, earth, evil, fire, good, lawful, or water creature, it is a spell of that type.

\featname{Summon Nature's Ally V}
\shortability{Conjuration}{(Summoning) [see text]}
\shortability{Level:}{Drd 5}
\shortability{Effect:}{One or more creatures, no two of which can be more than 30 ft. apart}
\smallskip\noindent This spell functions like summon nature's ally I, except that you can summon one 5th-level creature, 1d3 4th-level creatures of the same kind, or 1d4+1 lower-level creatures of the same kind.

\smallskip\noindent When you use a summoning spell to summon an air, chaotic, earth, evil, fire, good, lawful, or water creature, it is a spell of that type.

\featname{Summon Nature's Ally VI}
\shortability{Conjuration}{(Summoning) [see text]}
\shortability{Level:}{Drd 6}
\shortability{Effect:}{One or more creatures, no two of which can be more than 30 ft. apart}
\smallskip\noindent This spell functions like summon nature's ally I, except that you can summon one 6th-level creature, 1d3 5th-level creatures of the same kind, or 1d4+1 lower-level creatures of the same kind.

\smallskip\noindent When you use a summoning spell to summon an air, chaotic, earth, evil, fire, good, lawful, or water creature, it is a spell of that type.

\featname{Summon Nature's Ally VII}
\shortability{Conjuration}{(Summoning) [see text]}
\shortability{Level:}{Drd 7}
\shortability{Effect:}{One or more creatures, no two of which can be more than 30 ft. apart}
\smallskip\noindent This spell functions like summon nature's ally I, except that you can summon one 7th-level creature, 1d3 6th-level creatures of the same kind, or 1d4+1 lower-level creatures of the same kind.

\smallskip\noindent When you use a summoning spell to summon an air, chaotic, earth, evil, fire, good, lawful, or water creature, it is a spell of that type.

\featname{Summon Nature's Ally VIII}
\shortability{Conjuration}{(Summoning) [see text]}
\shortability{Level:}{Animal 8, Drd 8}
\shortability{Effect:}{One or more creatures, no two of which can be more than 30 ft. apart}
\smallskip\noindent This spell functions like summon nature's ally I, except that you can summon one 8th-level creature, 1d3 7th-level creatures of the same kind, or 1d4+1 lower-level creatures of the same kind.

\smallskip\noindent When you use a summoning spell to summon an air, chaotic, earth, evil, fire, good, lawful, or water creature, it is a spell of that type.

\featname{Summon Nature's Ally IX}
\shortability{Conjuration}{(Summoning) [see text]}
\shortability{Level:}{Drd 9}
\shortability{Effect:}{One or more creatures, no two of which can be more than 30 ft. apart}
\smallskip\noindent This spell functions like summon nature's ally I, except that you can summon one 9th-level creature, 1d3 8th-level creatures of the same kind, or 1d4+1 lower-level creatures of the same kind.

\smallskip\noindent When you use a summoning spell to summon an air, chaotic, earth, evil, fire, good, lawful, or water creature, it is a spell of that type.

%another big fucking table
%Summon Nature's Ally	   
%1st Level	   
%Dire rat	   
%Eagle (animal)	   
%onkey (animal)	   
%Octopus1 (animal)	   
%Owl (animal)	   
%Porpoise1 (animal)	   
%Snake, Small viper (animal)	   
%Wolf (animal)	   
%2nd Level	   
%Bear, black (animal)	   
%Crocodile (animal)	   
%Dire badger	   
%Dire bat	   
%Elemental, Small (any)	   
%Hippogriff	   
%Shark, Medium1 (animal)	   
%Snake, Medium viper (animal)	   
%Squid1 (animal)	   
%Wolverine (animal)	   
%3rd Level	   
%Ape (animal)	   
%Dire weasel	   
%Dire wolf	   
%Eagle, giant [NG]	   
%Lion	   
%Owl, giant [NG]	   
%Satyr [CN; without pipes]	   
%Shark, Large1 (animal)	   
%Snake, constrictor (animal)	   
%Snake, Large viper (animal)	   
%Thoqqua	   
%4th Level	   
%Arrowhawk, juvenile	   
%Bear, brown (animal)	   
%Crocodile, giant (animal)	   
%Deinonychus (dinosaur)	   
%Dire ape	   
%Dire boar	   
%Dire wolverine	   
%Elemental, Medium (any)	   
%Salamander, flamebrother [NE]	   
%Sea cat1	   
%Shark, Huge1 (animal)	   
%Snake, Huge viper (animal)	   
%Tiger (animal)	   
%Tojanida, juvenile1	   
%Unicorn [CG]	   
%Xorn, minor	   
%5th Level	   
%Arrowhawk, adult	   
%Bear, polar (animal)	   
%Dire lion	   
%Elasmosaurus1 (dinosaur)	   
%Elemental, Large (any)	   
%Griffon	   
%Janni (genie)	   
%Rhinoceros (animal)	   
%Satyr [CN; with pipes]	   
%Snake, giant constrictor (animal)	   
%Nixie (sprite)	   
%Tojanida, adult1	   
%Whale, orca1 (animal)	   
%6th Level	   
%Dire bear	   
%Elemental, Huge (any)	   
%Elephant (animal)	   
%Girallon	   
%Megaraptor (dinosaur)	   
%Octopus, giant1 (animal)	   
%Pixie* (sprite) [NG; no special arrows]	   
%Salamander, average [NE]	   
%Whale, baleen1	   
%Xorn, average	   
%*Can't cast irresistible dance	   
%7th Level	   
%Arrowhawk, elder	   
%Dire tiger	   
%Elemental, greater (any)	   
%Djinni (genie) [NG]	   
%Invisible stalker	   
%Pixie* (sprite) [NG; with sleep arrows]	   
%Squid, giant1 (animal)	   
%Triceratops (dinosaur)	   
%Tyrannosaurus (dinosaur)	   
%Whale, cachalot1 (animal)	   
%Xorn, elder	   
%*Can't cast irresistible dance	   
%8th Level	   
%Dire shark1	   
%Roc	   
%Salamander, noble [NE]	   
%Tojanida, elder	   
%9th Level	   
%Elemental, elder	   
%Grig [NG; with fiddle] (sprite)	   
%Pixie* (sprite) [NG; with sleep and memory loss arrows]	   
%Unicorn, celestial charger	   
%*Can cast irresistible dance	   
%1 May be summoned only into an aquatic or watery environment.	 

\featname{Summon Swarm}
\shortability{Conjuration}{(Summoning)}
\shortability{Level:}{Brd 2, Drd 2, Sor/Wiz 2}
\shortability{Components:}{V, S, M/DF}
\shortability{Casting Time:}{1 round}
\shortability{Range:}{Close (25 ft. + 5 ft./2 levels)}
\shortability{Effect:}{One swarm of bats, rats, or spiders}
\shortability{Duration:}{Concentration + 2 rounds}
\shortability{Saving Throw:}{None}
\shortability{Spell Resistance:}{No}
\smallskip\noindent You summon a swarm of bats, rats, or spiders (your choice), which attacks all other creatures within its area. (You may summon the swarm so that it shares the area of other creatures.) If no living creatures are within its area, the swarm attacks or pursues the nearest creature as best it can. The caster has no control over its target or direction of travel.

\shortability{Arcane Material Component:}{A square of red cloth.}

\featname{Sunbeam}
\shortability{Evocation}{[Light]}
\shortability{Level:}{Drd 7, Sun 7}
\shortability{Components:}{V, S, DF}
\shortability{Casting Time:}{1 standard action}
\shortability{Range:}{60 ft.}
\shortability{Area:}{Line from your hand}
\shortability{Duration:}{1 round/level or until all beams are exhausted}
\shortability{Saving Throw:}{Reflex negates and Reflex half; see text}
\shortability{Spell Resistance:}{Yes}
\smallskip\noindent For the duration of this spell, you can use a standard action to evoke a dazzling beam of intense light each round. You can call forth one beam per three caster levels (maximum six beams at 18th level). The spell ends when its duration runs out or your allotment of beams is exhausted.

\smallskip\noindent Each creature in the beam is blinded and takes 4d6 points of damage. Any creatures to which sunlight is harmful or unnatural take double damage. A successful Reflex save negates the blindness and reduces the damage by half.

\smallskip\noindent An undead creature caught within the beam takes 1d6 points of damage per caster level (maximum 20d6), or half damage if a Reflex save is successful. In addition, the beam results in the destruction of any undead creature specifically harmed by bright light if it fails its save.

\smallskip\noindent The ultraviolet light generated by the spell deals damage to fungi, mold, oozes, and slimes just as if they were undead creatures.

\featname{Sunburst}
\shortability{Evocation}{[Light]}
\shortability{Level:}{Drd 8, Sor/Wiz 8, Sun 8}
\shortability{Components:}{V, S, M/DF}
\shortability{Casting Time:}{1 standard action}
\shortability{Range:}{Long (400 ft. + 40 ft./level)}
\shortability{Area:}{80-ft.-radius burst}
\shortability{Duration:}{Instantaneous}
\shortability{Saving Throw:}{Reflex partial; see text}
\shortability{Spell Resistance:}{Yes}
\smallskip\noindent Sunburst causes a globe of searing radiance to explode silently from a point you select. All creatures in the globe are blinded and take 6d6 points of damage. A creature to which sunlight is harmful or unnatural takes double damage. A successful Reflex save negates the blindness and reduces the damage by half.

\smallskip\noindent An undead creature caught within the globe takes 1d6 points of damage per caster level (maximum 25d6), or half damage if a Reflex save is successful. In addition, the burst results in the destruction of any undead creature specifically harmed by bright light if it fail its save.

\smallskip\noindent The ultraviolet light generated by the spell deals damage to fungi, mold, oozes, and slimes just as if they were undead creatures.

\smallskip\noindent Sunburst dispels any darkness spells of lower than 9th level within its area.

\shortability{Arcane Material Component:}{A piece of sunstone and a naked flame.}

\featname{Symbol of Death}
\shortability{Necromancy}{[Death]}
\shortability{Level:}{Clr 8, Sor/Wiz 8}
\shortability{Components:}{V, S, M}
\shortability{Casting Time:}{10 minutes}
\shortability{Range:}{0 ft.; see text}
\shortability{Effect:}{One symbol}
\shortability{Duration:}{See text}
\shortability{Saving Throw:}{Fortitude negates}
\shortability{Spell Resistance:}{Yes}
\smallskip\noindent This spell allows you to scribe a potent rune of power upon a surface. When triggered, a symbol of death slays one or more creatures within 60 feet of the symbol (treat as a burst) whose combined total current hit points do not exceed 150. The symbol of death affects the closest creatures first, skipping creatures with too many hit points to affect. Once triggered, the symbol becomes active and glows, lasting for 10 minutes per caster level or until it has affected 150 hit points' worth of creatures, whichever comes first. Any creature that enters the area while the symbol of death is active is subject to its effect, whether or not that creature was in the area when it was triggered. A creature need save against the symbol only once as long as it remains within the area, though if it leaves the area and returns while the symbol is still active, it must save again.

\smallskip\noindent Until it is triggered, the symbol of death is inactive (though visible and legible at a distance of 60 feet). To be effective, a symbol of death must always be placed in plain sight and in a prominent location. Covering or hiding the rune renders the symbol of death ineffective, unless a creature removes the covering, in which case the symbol of death works normally.
As a default, a symbol of death is triggered whenever a creature does one or more of the following, as you select: looks at the rune; reads the rune; touches the rune; passes over the rune; or passes through a portal bearing the rune. Regardless of the trigger method or methods chosen, a creature more than 60 feet from a symbol of death can't trigger it (even if it meets one or more of the triggering conditions, such as reading the rune). Once the spell is cast, a symbol of death's triggering conditions cannot be changed.

\smallskip\noindent In this case, ``reading'' the rune means any attempt to study it, identify it, or fathom its meaning. Throwing a cover over a symbol of death to render it inoperative triggers it if the symbol reacts to touch. You can't use a symbol of death offensively; for instance, a touch-triggered symbol of death remains untriggered if an item bearing the symbol of death is used to touch a creature. Likewise, a symbol of death cannot be placed on a weapon and set to activate when the weapon strikes a foe.

\smallskip\noindent You can also set special triggering limitations of your own. These can be as simple or elaborate as you desire. Special conditions for triggering a symbol of death can be based on a creature's name, identity, or alignment, but otherwise must be based on observable actions or qualities. Intangibles such as level, class, Hit Dice, and hit points don't qualify. 

\smallskip\noindent When scribing a symbol of death, you can specify a password or phrase that prevents a creature using it from triggering the effect. Anyone using the password remains immune to that particular rune's effects so long as the creature remains within 60 feet of the rune. If the creature leaves the radius and returns later, it must use the password again.

\smallskip\noindent You also can attune any number of creatures to the symbol of death, but doing this can extend the casting time. Attuning one or two creatures takes negligible time, and attuning a small group (as many as ten creatures) extends the casting time to 1 hour. Attuning a large group (as many as twenty-five creatures) takes 24 hours. Attuning larger groups takes proportionately longer. Any creature attuned to a symbol of death cannot trigger it and is immune to its effects, even if within its radius when triggered. You are automatically considered attuned to your own symbols of death, and thus always ignore the effects and cannot inadvertently trigger them.
Read magic allows you to identify a symbol of death with a DC 19 Spellcraft check. Of course, if the symbol of death is set to be triggered by reading it, this will trigger the symbol.

\smallskip\noindent A symbol of death can be removed by a successful dispel magic targeted solely on the rune. An erase spell has no effect on a symbol of death. Destruction of the surface where a symbol of death is inscribed destroys the symbol but also triggers it.

\smallskip\noindent Symbol of death can be made permanent with a permanency spell. A permanent symbol of death that is disabled or that has affected its maximum number of hit points becomes inactive for 10 minutes, then can be triggered again as normal.

\shortability{Note:}{Magic traps such as symbol of death are hard to detect and disable. A rogue (only) can use the Search skill to find a symbol of death and Disable Device to thwart it. The DC in each case is 25 + spell level, or 33 for symbol of death.}
\shortability{Material Component:}{Mercury and phosphorus, plus powdered diamond and opal with a total value of at least 5,000 gp each.}

\featname{Symbol of Fear}
\shortability{Necromancy}{[Fear, Mind-Affecting]}
\shortability{Level:}{Clr 6, Sor/Wiz 6}
\shortability{Saving Throw:}{Will negates}
\smallskip\noindent This spell functions like symbol of death, except that all creatures within 60 feet of the symbol of fear instead become panicked for 1 round per caster level.

\shortability{Note:}{Magic traps such as symbol of fear are hard to detect and disable. A rogue (only) can use the Search skill to find a symbol of fear and Disable Device to thwart it. The DC in each case is 25 + spell level, or 31 for symbol of fear.}
\shortability{Material Component:}{Mercury and phosphorus, plus powdered diamond and opal with a total value of at least 1,000 gp.}

\featname{Symbol of Insanity}
\shortability{Enchantment}{(Compulsion) [Mind-Affecting]}
\shortability{Level:}{Clr 8, Sor/Wiz 8}
\shortability{Saving Throw:}{Will negates}
\smallskip\noindent This spell functions like symbol of death, except that all creatures within the radius of the symbol of insanity instead become permanently insane (as the insanity spell).

\smallskip\noindent Unlike symbol of death, symbol of insanity has no hit point limit; once triggered, a symbol of insanity simply remains active for 10 minutes per caster level.

\shortability{Note:}{Magic traps such as symbol of insanity are hard to detect and disable. A rogue (only) can use the Search skill to find a symbol of insanity and Disable Device to thwart it. The DC in each case is 25 + spell level, or 33 for symbol of insanity.}
\shortability{Material Component:}{Mercury and phosphorus, plus powdered diamond and opal with a total value of at least 5,000 gp.}

\featname{Symbol of Pain}
\shortability{Necromancy}{[Evil]}
\shortability{Level:}{Clr 5, Sor/Wiz 5}
\smallskip\noindent This spell functions like symbol of death, except that each creature within the radius of a symbol of pain instead suffers wracking pains that impose a -4 penalty on attack rolls, skill checks, and ability checks. These effects last for 1 hour after the creature moves farther than 60 feet from the symbol.

\smallskip\noindent Unlike symbol of death, symbol of pain has no hit point limit; once triggered, a symbol of pain simply remains active for 10 minutes per caster level.

\shortability{Note:}{Magic traps such as symbol of pain are hard to detect and disable. A rogue (only) can use the Search skill to find a symbol of pain and Disable Device to thwart it. The DC in each case is 25 + spell level, or 30 for symbol of pain.}
\shortability{Material Component:}{Mercury and phosphorus, plus powdered diamond and opal with a total value of at least 1,000 gp.}

\featname{Symbol of Persuasion}
\shortability{Enchantment}{(Charm) [Mind-Affecting]}
\shortability{Level:}{Clr 6, Sor/Wiz 6}
\shortability{Saving Throw:}{Will negates}
\smallskip\noindent This spell functions like symbol of death, except that all creatures within the radius of a symbol of persuasion instead become charmed by the caster (as the charm monster spell) for 1 hour per caster level.

\smallskip\noindent Unlike symbol of death, symbol of persuasion has no hit point limit; once triggered, a symbol of persuasion simply remains active for 10 minutes per caster level.

\shortability{Note:}{Magic traps such as symbol of persuasion are hard to detect and disable. A rogue (only) can use the Search skill to find a symbol of persuasion and Disable Device to thwart it. The DC in each case is 25 + spell level, or 31 for symbol of persuasion.}
\shortability{Material Component:}{Mercury and phosphorus, plus powdered diamond and opal with a total value of at least 5,000 gp.}

\featname{Symbol of Sleep}
\shortability{Enchantment}{(Compulsion) [Mind-Affecting]}
\shortability{Level:}{Clr 5, Sor/Wiz 5}
\shortability{Saving Throw:}{Will negates}
\smallskip\noindent This spell functions like symbol of death, except that all creatures of 10 HD or less within 60 feet of the symbol of sleep instead fall into a catatonic slumber for 3d6x10 minutes. Unlike with the sleep spell, sleeping creatures cannot be awakened by nonmagical means before this time expires.

\smallskip\noindent Unlike symbol of death, symbol of sleep has no hit point limit; once triggered, a symbol of sleep simply remains active for 10 minutes per caster level.

\shortability{Note:}{Magic traps such as symbol of sleep are hard to detect and disable. A rogue (only) can use the Search skill to find a symbol of sleep and Disable Device to thwart it. The DC in each case is 25 + spell level, or 30 for symbol of sleep.}
\shortability{Material Component:}{Mercury and phosphorus, plus powdered diamond and opal with a total value of at least 1,000 gp.}

\featname{Symbol of Stunning}
\shortability{Enchantment}{(Compulsion) [Mind-Affecting]}
\shortability{Level:}{Clr 7, Sor/Wiz 7}
\shortability{Saving Throw:}{Will negates}
\smallskip\noindent This spell functions like symbol of death, except that all creatures within 60 feet of a symbol of stunning instead become stunned for 1d6 rounds.

\shortability{Note:}{Magic traps such as symbol of stunning are hard to detect and disable. A rogue (only) can use the Search skill to find a symbol of stunning and Disable Device to thwart it. The DC in each case is 25 + spell level, or 32 for symbol of stunning.}
\shortability{Material Component:}{Mercury and phosphorus, plus powdered diamond and opal with a total value of at least 5,000 gp.}

\featname{Symbol of Weakness}
\shortability{Necromancy}{}
\shortability{Level:}{Clr 7, Sor/Wiz 7}
\smallskip\noindent This spell functions like symbol of death, except that every creature within 60 feet of a symbol of weakness instead suffers crippling weakness that deals 3d6 points of Strength damage.

\smallskip\noindent Unlike symbol of death, symbol of weakness has no hit point limit; once triggered, a symbol of weakness simply remains active for 10 minutes per caster level.

\shortability{Note:}{Magic traps such as symbol of weakness are hard to detect and disable. A rogue (only) can use the Search skill to find a symbol of weakness and Disable Device to thwart it. The DC in each case is 25 + spell level, or 32 for symbol of weakness.}
\shortability{Material Component:}{Mercury and phosphorus, plus powdered diamond and opal with a total value of at least 5,000 gp.}

\featname{Sympathetic Vibration}
\shortability{Evocation}{[Sonic]}
\shortability{Level:}{Brd 6}
\shortability{Components:}{V, S, F}
\shortability{Casting Time:}{10 minutes}
\shortability{Range:}{Touch}
\shortability{Target:}{One freestanding structure}
\shortability{Duration:}{Up to 1 round/level}
\shortability{Saving Throw:}{None; see text}
\shortability{Spell Resistance:}{Yes}
\smallskip\noindent By attuning yourself to a freestanding structure such you can create a damaging vibration within it. Once it begins, the vibration deals 2d10 points of damage per round to the target structure. (Hardness has no effect on the spell's damage.) You can choose at the time of casting to limit the duration of the spell; otherwise it lasts for 1 round/ level. If the spell is cast upon a target that is not freestanding the surrounding stone dissipates the effect and no damage occurs.

\smallskip\noindent Sympathetic vibration cannot affect creatures (including constructs). Since a structure is an unattended object, it gets no saving throw to resist the effect.

\shortability{Focus:}{A tuning fork.}

\featname{Sympathy}
\shortability{Enchantment}{(Compulsion) [Mind-Affecting]}
\shortability{Level:}{Drd 9, Sor/Wiz 8}
\shortability{Components:}{V, S, M}
\shortability{Casting Time:}{1 hour}
\shortability{Range:}{Close (25 ft. + 5 ft./2 levels)}
\shortability{Target:}{One location (up to a 10-ft. cube/level) or one object}
\shortability{Duration:}{2 hours/level (D)}
\shortability{Saving Throw:}{Will negates; see text}
\shortability{Spell Resistance:}{Yes}
\smallskip\noindent You cause an object or location to emanate magical vibrations that attract either a specific kind of intelligent creature or creatures of a particular alignment, as defined by you. The particular kind of creature to be affected must be named specifically. A creature subtype is not specific enough. Likewise, the specific alignment must be named.

\smallskip\noindent Creatures of the specified kind or alignment feel elated and pleased to be in the area or desire to touch or to possess the object. The compulsion to stay in the area or touch the object is overpowering. If the save is successful, the creature is released from the enchantment, but a subsequent save must be made 1d6x10 minutes later. If this save fails, the affected creature attempts to return to the area or object.

\smallskip\noindent Sympathy counters and dispels antipathy.

\shortability{Material Component:}{1,500 gp worth of crushed pearls and a drop of honey.}

\featname{Telekinesis}
\shortability{Transmutation}{}
\shortability{Level:}{Sor/Wiz 5}
\shortability{Components:}{V, S}
\shortability{Casting Time:}{1 standard action}
\shortability{Range:}{Long (400 ft. + 40 ft./level)}
\shortability{Target or Targets:}{See text}
\shortability{Duration:}{Concentration (up to 1 round/ level) or instantaneous; see text}
\shortability{Saving Throw:}{Will negates (object) or None; see text}
\shortability{Spell Resistance:}{Yes (object); see text}
\smallskip\noindent You move objects or creatures by concentrating on them. Depending on the version selected, the spell can provide a gentle, sustained force, perform a variety of combat maneuvers, or exert a single short, violent thrust.

\shortability{Sustained Force:}{A sustained force moves an object weighing no more than 25 pounds per caster level (maximum 375 pounds at 15th level) up to 20 feet per round. A creature can negate the effect on an object it possesses with a successful Will save or with spell resistance.

\smallskip\noindent This version of the spell can last 1 round per caster level, but it ends if you cease concentration. The weight can be moved vertically, horizontally, or in both directions. An object cannot be moved beyond your range. The spell ends if the object is forced beyond the range. If you cease concentration for any reason, the object falls or stops.

\smallskip\noindent An object can be telekinetically manipulated as if with one hand. For example, a lever or rope can be pulled, a key can be turned, an object rotated, and so on, if the force required is within the weight limitation. You might even be able to untie simple knots, though delicate activities such as these require Intelligence checks.}

\shortability{Combat Maneuver:}{Alternatively, once per round, you can use telekinesis to perform a bull rush, disarm, grapple (including pin), or trip. Resolve these attempts as normal, except that they don't provoke attacks of opportunity, you use your caster level in place of your base attack bonus (for disarm and grapple), you use your Intelligence modifier (if a wizard) or Charisma modifier (if a sorcerer) in place of your Strength or Dexterity modifier, and a failed attempt doesn't allow a reactive attempt by the target (such as for disarm or trip). No save is allowed against these attempts, but spell resistance applies normally. This version of the spell can last 1 round per caster level, but it ends if you cease concentration.}
\shortability{Violent Thrust:}{Alternatively, the spell energy can be spent in a single round. You can hurl one object or creature per caster level (maximum 15) that are within range and all within 10 feet of each other toward any target within 10 feet per level of all the objects. You can hurl up to a total weight of 25 pounds per caster level (maximum 375 pounds at 15th level).

\smallskip\noindent You must succeed on attack rolls (one per creature or object thrown) to hit the target with the items, using your base attack bonus + your Intelligence modifier (if a wizard) or Charisma modifier (if a sorcerer). Weapons cause standard damage (with no Strength bonus; note that arrows or bolts deal damage as daggers of their size when used in this manner). Other objects cause damage ranging from 1 point per 25 pounds (for less dangerous objects) to 1d6 points of damage per 25 pounds (for hard, dense objects).

\smallskip\noindent Creatures who fall within the weight capacity of the spell can be hurled, but they are allowed Will saves (and spell resistance) to negate the effect, as are those whose held possessions are targeted by the spell. If a telekinesed creature is hurled against a solid surface, it takes damage as if it had fallen 10 feet (1d6 points).}

\featname{Telekinetic Sphere}
\shortability{Evocation}{[Force]}
\shortability{Level:}{Sor/Wiz 8}
\shortability{Components:}{V, S, M}
\shortability{Casting Time:}{1 standard action}
\shortability{Range:}{Close (25 ft. + 5 ft./2 levels)}
\shortability{Effect:}{1-ft.-diameter/level sphere, centered around creatures or objects}
\shortability{Duration:}{1 min./level (D)}
\shortability{Saving Throw:}{Reflex negates (object)}
\shortability{Spell Resistance:}{Yes (object)}
\smallskip\noindent This spell functions like resilient sphere, with the addition that the creatures or objects inside the globe are nearly weightless. Anything contained within an telekinetic sphere weighs only one-sixteenth of its normal weight. You can telekinetically lift anything in the sphere that normally weighs 5,000 pounds or less. The telekinetic control extends from you out to medium range (100 feet + 10 feet per caster level) after the sphere has succeeded in encapsulating its contents.

\smallskip\noindent You can move objects or creatures in the sphere that weigh a total of 5,000 pounds or less by concentrating on the sphere. You can begin moving a sphere in the round after casting the spell. If you concentrate on doing so (a standard action), you can move the sphere as much as 30 feet in a round. If you cease concentrating, the sphere does not move in that round (if on a level surface) or descends at its falling rate (if aloft) until it reaches a level surface, or the spell's duration expires, or you begin concentrating again. If you cease concentrating (voluntarily or due to failing a Concentration check), you can resume concentrating on your next turn or any later turn during the spell's duration.

\smallskip\noindent The sphere falls at a rate of only 60 feet per round, which is not fast enough to cause damage to the contents of the sphere.

\smallskip\noindent You can move the sphere telekinetically even if you are in it.

\shortability{Material Component:}{A hemispherical piece of clear crystal, a matching hemispherical piece of gum arabic, and a pair of small bar magnets.}

\featname{Telepathic Bond}
\shortability{Divination}{}
\shortability{Level:}{Sor/Wiz 5}
\shortability{Components:}{V, S, M}
\shortability{Casting Time:}{1 standard action}
\shortability{Range:}{Close (25 ft. + 5 ft./2 levels)}
\shortability{Targets:}{You plus one willing creature per three levels, no two of which can be more than 30 ft. apart}
\shortability{Duration:}{10 min./level (D)}
\shortability{Saving Throw:}{None}
\shortability{Spell Resistance:}{No}
\smallskip\noindent You forge a telepathic bond among yourself and a number of willing creatures, each of which must have an Intelligence score of 3 or higher. Each creature included in the link is linked to all the others. The creatures can communicate telepathically through the bond regardless of language. No special power or influence is established as a result of the bond. Once the bond is formed, it works over any distance (although not from one plane to another).

\smallskip\noindent If desired, you may leave yourself out of the telepathic bond forged. This decision must be made at the time of casting.

\smallskip\noindent Telepathic bond can be made permanent with a permanency spell, though it only bonds two creatures per casting of permanency.

\shortability{Material Component:}{Pieces of eggshell from two different kinds of creatures.}

\featname{Teleport}
\shortability{Conjuration}{(Teleportation)}
\shortability{Level:}{Sor/Wiz 5, Travel 5}
\shortability{Components:}{V}
\shortability{Casting Time:}{1 standard action}
\shortability{Range:}{Personal and touch}
\shortability{Target:}{You and touched objects or other touched willing creatures}
\shortability{Duration:}{Instantaneous}
\shortability{Saving Throw:}{None and Will negates (object)}
\shortability{Spell Resistance:}{No and Yes (object)}
\smallskip\noindent This spell instantly transports you to a designated destination, which may be as distant as 100 miles per caster level. Interplanar travel is not possible. You can bring along objects as long as their weight doesn't exceed your maximum load. You may also bring one additional willing Medium or smaller creature (carrying gear or objects up to its maximum load) or its equivalent (see below) per three caster levels. A Large creature counts as two Medium creatures, a Huge creature counts as two Large creatures, and so forth. All creatures to be transported must be in contact with one another, and at least one of those creatures must be in contact with you. As with all spells where the range is personal and the target is you, you need not make a saving throw, nor is spell resistance applicable to you. Only objects held or in use (attended) by another person receive saving throws and spell resistance.

\smallskip\noindent You must have some clear idea of the location and layout of the destination. The clearer your mental image, the more likely the teleportation works. Areas of strong physical or magical energy may make teleportation more hazardous or even impossible.

\smallskip\noindent To see how well the teleportation works, roll d\% and consult the Teleport table. Refer to the following information for definitions of the terms on the table.

\shortability{Familiarity:}{``Very familiar'' is a place where you have been very often and where you feel at home. ``Studied carefully'' is a place you know well, either because you can currently see it, you've been there often, or you have used other means (such as scrying) to study the place for at least one hour. ``Seen casually'' is a place that you have seen more than once but with which you are not very familiar. ``Viewed once'' is a place that you have seen once, possibly using magic.}

\smallskip\noindent ``False destination'' is a place that does not truly exist or if you are teleporting to an otherwise familiar location that no longer exists as such or has been so completely altered as to no longer be familiar to you. When traveling to a false destination, roll 1d20+80 to obtain results on the table, rather than rolling d\%, since there is no real destination for you to hope to arrive at or even be off target from.

\shortability{On Target:}{You appear where you want to be.}
\shortability{Off Target:}{You appear safely a random distance away from the destination in a random direction. Distance off target is 1d10x1d10\% of the distance that was to be traveled. The direction off target is determined randomly}
\shortability{Similar Area:}{You wind up in an area that's visually or thematically similar to the target area. Generally, you appear in the closest similar place within range. If no such area exists within the spell's range, the spell simply fails instead.}
\shortability{Mishap: You and anyone else teleporting with you have gotten "scrambled." You each take 1d10 points of damage, and you reroll on the chart to see where you wind up. For these rerolls, roll 1d20+80. Each time "Mishap" comes up, the characters take more damage and must reroll.}
 
Familiarity	On Target	Off Target	Similar Area	Mishap	   
Very familiar	01-97	98-99	100	-	   
Studied carefully	01-94	95-97	98-99	100	   
Seen casually	01-88	89-94	95-98	99-100	   
Viewed once	01-76	77-88	89-96	97-100	   
False destination (1d20+80)	-	-	81-92	93-100	 

\featname{Teleport Object}
\shortability{Conjuration}{(Teleportation)}
\shortability{Level:}{Sor/Wiz 7}
\shortability{Range:}{Touch}
\shortability{Target:}{One touched object of up to 50 lb./level and 3 cu. ft./level}
\shortability{Saving Throw:}{Will negates (object)}
\shortability{Spell Resistance:}{Yes (object)}
This spell functions like teleport, except that it teleports an object, not you. Creatures and magical forces cannot be teleported.
If desired, the target object can be sent to a distant location on the Ethereal Plane. In this case, the point from which the object was teleported remains faintly magical until the item is retrieved. A successful targeted dispel magic spell cast on that point brings the vanished item back from the Ethereal Plane.

\featname{Teleport, Greater}
\shortability{Conjuration}{(Teleportation)}
\shortability{Level:}{Sor/Wiz 7, Travel 7}
This spell functions like teleport, except that there is no range limit and there is no chance you arrive off target. In addition, you need not have seen the destination, but in that case you must have at least a reliable description of the place to which you are teleporting. If you attempt to teleport with insufficient information (or with misleading information), you disappear and simply reappear in your original location. Interplanar travel is not possible.

\featname{Teleportation Circle}
\shortability{Conjuration}{(Teleportation)}
\shortability{Level:}{Sor/Wiz 9}
\shortability{Components:}{V, M}
\shortability{Casting Time:}{10 minutes}
\shortability{Range:}{0 ft.}
\shortability{Effect:}{5-ft.-radius circle that teleports those who activate it}
\shortability{Duration:}{10 min./level (D)}
\shortability{Saving Throw:}{None}
\shortability{Spell Resistance:}{Yes}
You create a circle on the floor or other horizontal surface that teleports, as greater teleport, any creature who stands on it to a designated spot. Once you designate the destination for the circle, you can't change it. The spell fails if you attempt to set the circle to teleport creatures into a solid object, to a place with which you are not familiar and have no clear description, or to another plane.
The circle itself is subtle and nearly impossible to notice. If you intend to keep creatures from activating it accidentally, you need to mark the circle in some way.
Teleportation circle can be made permanent with a permanency spell. A permanent teleportation circle that is disabled becomes inactive for 10 minutes, then can be triggered again as normal.
Note: Magic traps such as teleportation circle are hard to detect and disable. A rogue (only) can use the Search skill to find the circle and Disable Device to thwart it. The DC in each case is 25 + spell level, or 34 in the case of teleportation circle.
\shortability{Material Component:}{Amber dust to cover the area of the circle (cost 1,000 gp).}

\featname{Temporal Stasis}
\shortability{Transmutation}{}
\shortability{Level:}{Sor/Wiz 8}
\shortability{Components:}{V, S, M}
\shortability{Casting Time:}{1 standard action}
\shortability{Range:}{Touch}
\shortability{Target:}{Creature touched}
\shortability{Duration:}{Permanent}
\shortability{Saving Throw:}{Fortitude negates}
\shortability{Spell Resistance:}{Yes}
You must succeed on a melee touch attack. You place the subject into a state of suspended animation. For the creature, time ceases to flow and its condition becomes fixed. The creature does not grow older. Its body functions virtually cease, and no force or effect can harm it. This state persists until the magic is removed (such as by a successful dispel magic spell or a freedom spell).
\shortability{Material Component:}{A powder composed of diamond, emerald, ruby, and sapphire dust with a total value of at least 5,000 gp.}

\featname{Time Stop}
\shortability{Transmutation}{}
\shortability{Level:}{Sor/Wiz 9, Trickery 9}
\shortability{Components:}{V}
\shortability{Casting Time:}{1 standard action}
\shortability{Range:}{Personal}
\shortability{Target:}{You}
\shortability{Duration:}{1d4+1 rounds (apparent time); see text}
This spell seems to make time cease to flow for everyone but you. In fact, you speed up so greatly that all other creatures seem frozen, though they are actually still moving at their normal speeds. You are free to act for 1d4+1 rounds of apparent time. Normal and magical fire, cold, gas, and the like can still harm you. While the time stop is in effect, other creatures are invulnerable to your attacks and spells; you cannot target such creatures with any attack or spell. A spell that affects an area and has a duration longer than the remaining duration of the time stop have their normal effects on other creatures once the time stop ends. Most spellcasters use the additional time to improve their defenses, summon allies, or flee from combat.
You cannot move or harm items held, carried, or worn by a creature stuck in normal time, but you can affect any item that is not in another creature's possession.
You are undetectable while time stop lasts. You cannot enter an area protected by an antimagic field while under the effect of time stop.

\featname{Tiny Hut}
\shortability{Evocation}{[Force]}
\shortability{Level:}{Brd 3, Sor/Wiz 3}
\shortability{Components:}{V, S, M}
\shortability{Casting Time:}{1 standard action}
\shortability{Range:}{20 ft.}
\shortability{Effect:}{20-ft.-radius sphere centered on your location}
\shortability{Duration:}{2 hours/level (D)}
\shortability{Saving Throw:}{None}
\shortability{Spell Resistance:}{No}
You create an unmoving, opaque sphere of force of any color you desire around yourself. Half the sphere projects above the ground, and the lower hemisphere passes through the ground. As many as nine other Medium creatures can fit into the field with you; they can freely pass into and out of the hut without harming it. However, if you remove yourself from the hut, the spell ends.
The temperature inside the hut is 70� F if the exterior temperature is between 0� and 100� F. An exterior temperature below 0� or above 100� lowers or raises the interior temperature on a 1-degree-for-1 basis. The hut also provides protection against the elements, such as rain, dust, and sandstorms. The hut withstands any wind of less than hurricane force, but a hurricane (75+ mph wind speed) or greater force destroys it.
The interior of the hut is a hemisphere. You can illuminate it dimly upon command or extinguish the light as desired. Although the force field is opaque from the outside, it is transparent from within. Missiles, weapons, and most spell effects can pass through the hut without affecting it, although the occupants cannot be seen from outside the hut (they have total concealment).
\shortability{Material Component:}{A small crystal bead that shatters when the spell duration expires or the hut is dispelled.}

\featname{Tongues}
\shortability{Divination}{}
\shortability{Level:}{Brd 2, Clr 4, Sor/Wiz 3}
\shortability{Components:}{V, M/DF}
\shortability{Casting Time:}{1 standard action}
\shortability{Range:}{Touch}
\shortability{Target:}{Creature touched}
\shortability{Duration:}{10 min./level}
\shortability{Saving Throw:}{Will negates (harmless)}
\shortability{Spell Resistance:}{No}
This spell grants the creature touched the ability to speak and understand the language of any intelligent creature, whether it is a racial tongue or a regional dialect. The subject can speak only one language at a time, although it may be able to understand several languages. Tongues does not enable the subject to speak with creatures who don't speak. The subject can make itself understood as far as its voice carries. This spell does not predispose any creature addressed toward the subject in any way.
Tongues can be made permanent with a permanency spell.
\shortability{Arcane Material Component:}{A small clay model of a ziggurat, which shatters when the verbal component is pronounced.}

\featname{Touch of Fatigue}
\shortability{Necromancy}{}
\shortability{Level:}{Sor/Wiz 0}
\shortability{Components:}{V, S, M}
\shortability{Casting Time:}{1 standard action}
\shortability{Range:}{Touch}
\shortability{Target:}{Creature touched}
\shortability{Duration:}{1 round/level}
\shortability{Saving Throw:}{Fortitude negates}
\shortability{Spell Resistance:}{Yes}
You channel negative energy through your touch, fatiguing the target. You must succeed on a touch attack to strike a target.
The subject is immediately fatigued for the spell's duration.
This spell has no effect on a creature that is already fatigued. Unlike with normal fatigue, the effect ends as soon as the spell's duration expires.
\shortability{Material Component:}{A drop of sweat.}

\featname{Touch of Idiocy}
\shortability{Enchantment}{(Compulsion) [Mind-Affecting]}
\shortability{Level:}{Sor/Wiz 2}
\shortability{Components:}{V, S}
\shortability{Casting Time:}{1 standard action}
\shortability{Range:}{Touch}
\shortability{Target:}{Living creature touched}
\shortability{Duration:}{10 min./level}
\shortability{Saving Throw:}{No}
\shortability{Spell Resistance:}{Yes}
With a touch, you reduce the target's mental faculties. Your successful melee touch attack applies a 1d6 penalty to the target's Intelligence, Wisdom, and Charisma scores. This penalty can't reduce any of these scores below 1.
This spell's effect may make it impossible for the target to cast some or all of its spells, if the requisite ability score drops below the minimum required to cast spells of that level.

\featname{Transformation}
\shortability{Transmutation}{}
\shortability{Level:}{Sor/Wiz 6}
\shortability{Components:}{V, S, M}
\shortability{Casting Time:}{1 standard action}
\shortability{Range:}{Personal}
\shortability{Target:}{You}
\shortability{Duration:}{1 round/level}
You become a virtual fighting machine- stronger, tougher, faster, and more skilled in combat. Your mind-set changes so that you relish combat and you can't cast spells, even from magic items.
You gain a +4 enhancement bonus to Strength, Dexterity, and Constitution, a +4 natural armor bonus to AC, a +5 competence bonus on Fortitude saves, and proficiency with all simple and martial weapons. Your base attack bonus equals your character level (which may give you multiple attacks).
You lose your spellcasting ability, including your ability to use spell activation or spell completion magic items, just as if the spells were no longer on your class list.
\shortability{Material Component:}{A potion of bull's strength, which you drink (and whose effects are subsumed by the spell effects).}

\featname{Transmute Metal to Wood}
\shortability{Transmutation}{}
\shortability{Level:}{Drd 7}
\shortability{Components:}{V, S, DF}
\shortability{Casting Time:}{1 standard action}
\shortability{Range:}{Long (400 ft. + 40 ft./level)}
\shortability{Area:}{All metal objects within a 40-ft.-radius burst}
\shortability{Duration:}{Instantaneous}
\shortability{Saving Throw:}{None}
\shortability{Spell Resistance:}{Yes (object; see text)}
This spell enables you to change all metal objects within its area to wood. Weapons, armor, and other metal objects carried by creatures are affected as well. A magic object made of metal effectively has spell resistance equal to 20 + its caster level against this spell. Artifacts cannot be transmuted. Weapons converted from metal to wood take a -2 penalty on attack and damage rolls. The armor bonus of any armor converted from metal to wood is reduced by 2. Weapons changed by this spell splinter and break on any natural attack roll of 1 or 2, and armor changed by this spell loses an additional point of armor bonus every time it is struck with a natural attack roll of 19 or 20.
Only limited wish, miracle, wish, or similar magic can restore a transmuted object to its metallic state.

\featname{Transmute Mud to Rock}
\shortability{Transmutation}{[Earth]}
\shortability{Level:}{Drd 5, Sor/Wiz 5}
\shortability{Components:}{V, S, M/DF}
\shortability{Casting Time:}{1 standard action}
\shortability{Range:}{Medium (100 ft. + 10 ft./level)}
\shortability{Area:}{Up to two 10-ft. cubes/level (S)}
\shortability{Duration:}{Permanent}
\shortability{Saving Throw:}{See text}
\shortability{Spell Resistance:}{No}
This spell transforms normal mud or quicksand of any depth into soft stone (sandstone or a similar mineral) permanently.
Any creature in the mud is allowed a Reflex save to escape before the area is hardened to stone.
Transmute mud to rock counters and dispels transmute rock to mud.
\shortability{Arcane Material Component:}{Sand, lime, and water.}

\featname{Transmute Rock to Mud}
\shortability{Transmutation}{[Earth]}
\shortability{Level:}{Drd 5, Sor/Wiz 5}
\shortability{Components:}{V, S, M/DF}
\shortability{Casting Time:}{1 standard action}
\shortability{Range:}{Medium (100 ft. + 10 ft./level)}
\shortability{Area:}{Up to two 10-ft. cubes/level (S)}
\shortability{Duration:}{Permanent; see text}
\shortability{Saving Throw:}{See text}
\shortability{Spell Resistance:}{No}
This spell turns natural, uncut or unworked rock of any sort into an equal volume of mud. Magical stone is not affected by the spell. The depth of the mud created cannot exceed 10 feet. A creature unable to levitate, fly, or otherwise free itself from the mud sinks until hip- or chest-deep, reducing its speed to 5 feet and causing a -2 penalty on attack rolls and AC. Brush thrown atop the mud can support creatures able to climb on top of it. Creatures large enough to walk on the bottom can wade through the area at a speed of 5 feet.
If transmute rock to mud is cast upon the ceiling of a cavern or tunnel, the mud falls to the floor and spreads out in a pool at a depth of 5 feet. The falling mud and the ensuing cave-in deal 8d6 points of bludgeoning damage to anyone caught directly beneath the area, or half damage to those who succeed on Reflex saves.
Castles and large stone buildings are generally immune to the effect of the spell, since transmute rock to mud can't affect worked stone and doesn't reach deep enough to undermine such buildings' foundations. However, small buildings or structures often rest upon foundations shallow enough to be damaged or even partially toppled by this spell.
The mud remains until a successful dispel magic or transmute mud to rock spell restores its substance-but not necessarily its form. Evaporation turns the mud to normal dirt over a period of days. The exact time depends on exposure to the sun, wind, and normal drainage.
\shortability{Arcane Material Component:}{Clay and water.}

\featname{Transport via Plants}
\shortability{Conjuration}{(Teleportation)}
\shortability{Level:}{Drd 6}
\shortability{Components:}{V, S}
\shortability{Casting Time:}{1 standard action}
\shortability{Range:}{Unlimited}
\shortability{Target:}{You and touched objects or other touched willing creatures}
\shortability{Duration:}{1 round}
\shortability{Saving Throw:}{None}
\shortability{Spell Resistance:}{No}
You can enter any normal plant (Medium or larger) and pass any distance to a plant of the same kind in a single round, regardless of the distance separating the two. The entry plant must be alive. The destination plant need not be familiar to you, but it also must be alive. If you are uncertain of the location of a particular kind of destination plant, you need merely designate direction and distance and the transport via plants spell moves you as close as possible to the desired location. If a particular destination plant is desired but the plant is not living, the spell fails and you are ejected from the entry plant.
You can bring along objects as long as their weight doesn't exceed your maximum load. You may also bring one additional willing Medium or smaller creature (carrying gear or objects up to its maximum load) or its equivalent per three caster levels. Use the following equivalents to determine the maximum number of larger creatures you can bring along: A Large creature counts as two Medium creatures, a Huge creature counts as two Large creatures, and so forth. All creatures to be transported must be in contact with one another, and at least one of those creatures must be in contact with you.
You can't use this spell to travel through plant creatures.
The destruction of an occupied plant slays you and any creatures you have brought along, and ejects the bodies and all carried objects from the tree.

\featname{Trap the Soul}
\shortability{Conjuration}{(Summoning)}
\shortability{Level:}{Sor/Wiz 8}
\shortability{Components:}{V, S, M, (F); see text}
\shortability{Casting Time:}{1 standard action or see text}
\shortability{Range:}{Close (25 ft. + 5 ft./2 levels)}
\shortability{Target:}{One creature}
\shortability{Duration:}{Permanent; see text}
\shortability{Saving Throw:}{See text}
\shortability{Spell Resistance:}{Yes; see text}
Trap the soul forces a creature's life force (and its material body) into a gem. The gem holds the trapped entity indefinitely or until the gem is broken and the life force is released, which allows the material body to reform. If the trapped creature is a powerful creature from another plane it can be required to perform a service immediately upon being freed. Otherwise, the creature can go free once the gem imprisoning it is broken.
Depending on the version selected, the spell can be triggered in one of two ways.
Spell Completion: First, the spell can be completed by speaking its final word as a standard action as if you were casting a regular spell at the subject. This allows spell resistance (if any) and a Will save to avoid the effect. If the creature's name is spoken as well, any spell resistance is ignored and the save DC increases by 2. If the save or spell resistance is successful, the gem shatters.
Trigger Object: The second method is far more insidious, for it tricks the subject into accepting a trigger object inscribed with the final spell word, automatically placing the creature's soul in the trap. To use this method, both the creature's name and the trigger word must be inscribed on the trigger object when the gem is enspelled. A sympathy spell can also be placed on the trigger object. As soon as the subject picks up or accepts the trigger object, its life force is automatically transferred to the gem without the benefit of spell resistance or a save.
\shortability{Material Component:}{Before the actual casting of trap the soul, you must procure a gem of at least 1,000 gp value for every Hit Die possessed by the creature to be trapped. If the gem is not valuable enough, it shatters when the entrapment is attempted. (While creatures have no concept of level or Hit Dice as such, the value of the gem needed to trap an individual can be researched. Remember that this value can change over time as creatures gain more Hit Dice.)}
Focus (Trigger Object Only): If the trigger object method is used, a special trigger object, prepared as described above, is needed.

\featname{Tree Shape}
\shortability{Transmutation}{}
\shortability{Level:}{Drd 2, Rgr 3}
\shortability{Components:}{V, S, DF}
\shortability{Casting Time:}{1 standard action}
\shortability{Range:}{Personal}
\shortability{Target:}{You}
\shortability{Duration:}{1 hour/level (D)}
By means of this spell, you are able to assume the form of a Large living tree or shrub or a Large dead tree trunk with a small number of limbs. The closest inspection cannot reveal that the tree in question is actually a magically concealed creature. To all normal tests you are, in fact, a tree or shrub, although a detect magic spell reveals a faint transmutation on the tree. While in tree form, you can observe all that transpires around you just as if you were in your normal form, and your hit points and save bonuses remain unaffected. You gain a +10 natural armor bonus to AC but have an effective Dexterity score of 0 and a speed of 0 feet. You are immune to critical hits while in tree form. All clothing and gear carried or worn changes with you.
You can dismiss tree shape as a free action (instead of as a standard action).

\featname{Tree Stride}
\shortability{Conjuration}{(Teleportation)}
\shortability{Level:}{Drd 5, Rgr 4}
\shortability{Components:}{V, S, DF}
\shortability{Casting Time:}{1 standard action}
\shortability{Range:}{Personal}
\shortability{Target:}{You}
\shortability{Duration:}{1 hour/level or until expended; see text}
You gain the ability to enter trees and move from inside one tree to inside another tree. The first tree you enter and all others you enter must be of the same kind, must be living, and must have girth at least equal to yours. By moving into an oak tree (for example), you instantly know the location of all other oak trees within transport range (see below) and may choose whether you want to pass into one or simply step back out of the tree you moved into. You may choose to pass to any tree of the appropriate kind within the transport range as shown on the following table.
 
Type of Tree	Transport Range	   
Oak, ash, yew	3,000 feet	   
Elm, linden	2,000 feet	   
Other deciduous	1,500 feet	   
Any coniferous	1,000 feet	   
All other trees	500 feet	 
You may move into a tree up to one time per caster level (passing from one tree to another counts only as moving into one tree). The spell lasts until the duration expires or you exit a tree. Each transport is a full-round action.
You can, at your option, remain within a tree without transporting yourself, but you are forced out when the spell ends. If the tree in which you are concealed is chopped down or burned, you are slain if you do not exit before the process is complete.

\featname{True Resurrection}
\shortability{Conjuration}{(Healing)}
\shortability{Level:}{Clr 9}
\shortability{Casting Time:}{10 minutes}
This spell functions like raise dead, except that you can resurrect a creature that has been dead for as long as 10 years per caster level. This spell can even bring back creatures whose bodies have been destroyed, provided that you unambiguously identify the deceased in some fashion (reciting the deceased's time and place of birth or death is the most common method).
Upon completion of the spell, the creature is immediately restored to full hit points, vigor, and health, with no loss of level (or Constitution points) or prepared spells.
You can revive someone killed by a death effect or someone who has been turned into an undead creature and then destroyed. This spell can also resurrect elementals or outsiders, but it can't resurrect constructs or undead creatures.
Even true resurrection can't restore to life a creature who has died of old age.
\shortability{Material Component:}{A sprinkle of holy water and diamonds worth a total of at least 25,000 gp.}

\featname{True Seeing}
\shortability{Divination}{}
\shortability{Level:}{Clr 5, Drd 7, Knowledge 5, Sor/Wiz 6}
\shortability{Components:}{V, S, M}
\shortability{Casting Time:}{1 standard action}
\shortability{Range:}{Touch}
\shortability{Target:}{Creature touched}
\shortability{Duration:}{1 min./level}
\shortability{Saving Throw:}{Will negates (harmless)}
\shortability{Spell Resistance:}{Yes (harmless)}
You confer on the subject the ability to see all things as they actually are. The subject sees through normal and magical darkness, notices secret doors hidden by magic, sees the exact locations of creatures or objects under blur or displacement effects, sees invisible creatures or objects normally, sees through illusions, and sees the true form of polymorphed, changed, or transmuted things. Further, the subject can focus its vision to see into the Ethereal Plane (but not into extradimensional spaces). The range of true seeing conferred is 120 feet.
True seeing, however, does not penetrate solid objects. It in no way confers X-ray vision or its equivalent. It does not negate concealment, including that caused by fog and the like. True seeing does not help the viewer see through mundane disguises, spot creatures who are simply hiding, or notice secret doors hidden by mundane means. In addition, the spell effects cannot be further enhanced with known magic, so one cannot use true seeing through a crystal ball or in conjunction with clairaudience/clairvoyance.
\shortability{Material Component:}{An ointment for the eyes that costs 250 gp and is made from mushroom powder, saffron, and fat.}

\featname{True Strike}
\shortability{Divination}{}
\shortability{Level:}{Sor/Wiz 1}
\shortability{Components:}{V, F}
\shortability{Casting Time:}{1 standard action}
\shortability{Range:}{Personal}
\shortability{Target:}{You}
\shortability{Duration:}{See text}
You gain temporary, intuitive insight into the immediate future during your next attack. Your next single attack roll (if it is made before the end of the next round) gains a +20 insight bonus. Additionally, you are not affected by the miss chance that applies to attackers trying to strike a concealed target.
\shortability{Focus:}{A small wooden replica of an archery target.}

\featname{Undeath to Death}
\shortability{Necromancy}{}
\shortability{Level:}{Clr 6, Sor/Wiz 6}
\shortability{Components:}{V, S, M/DF}
\shortability{Area:}{Several undead creatures within a 40-ft.-radius burst}
\shortability{Saving Throw:}{Will negates}
This spell functions like circle of death, except that it destroys undead creatures as noted above.
\shortability{Material Component:}{The powder of a crushed diamond worth at least 500 gp.}

\featname{Undetectable Alignment}
\shortability{Abjuration}{}
\shortability{Level:}{Brd 1, Clr 2, Pal 2}
\shortability{Components:}{V, S}
\shortability{Casting Time:}{1 standard action}
\shortability{Range:}{Close (25 ft. + 5 ft./2 levels)}
\shortability{Target:}{One creature or object}
\shortability{Duration:}{24 hours}
\shortability{Saving Throw:}{Will negates (object)}
\shortability{Spell Resistance:}{Yes (object)}
An undetectable alignment spell conceals the alignment of an object or a creature from all forms of divination.

\featname{Unhallow}
\shortability{Evocation}{[Evil]}
\shortability{Level:}{Clr 5, Drd 5}
\shortability{Components:}{V, S, M}
\shortability{Casting Time:}{24 hours}
\shortability{Range:}{Touch}
\shortability{Area:}{40-ft. radius emanating from the touched point}
\shortability{Duration:}{Instantaneous}
\shortability{Saving Throw:}{See text}
\shortability{Spell Resistance:}{See text}
Unhallow makes a particular site, building, or structure an unholy site. This has three major effects.
First, the site or structure is guarded by a magic circle against good effect.
Second, all turning checks made to turn undead take a -4 penalty, and turning checks to rebuke undead gain a +4 profane bonus. Spell resistance does not apply to this effect. (This provision does not apply to the druid version of the spell.)
Finally, you may choose to fix a single spell effect to the unhallowed site. The spell effect lasts for one year and functions throughout the entire site, regardless of its normal duration and area or effect. You may designate whether the effect applies to all creatures, creatures that share your faith or alignment, or creatures that adhere to another faith or alignment. At the end of the year, the chosen effect lapses, but it can be renewed or replaced simply by casting unhallow again.
Spell effects that may be tied to an unhallowed site include aid, bane, bless, cause fear, darkness, daylight, death ward, deeper darkness, detect magic, detect good, dimensional anchor, discern lies, dispel magic, endure elements, freedom of movement, invisibility purge, protection from energy, remove fear, resist energy, silence, tongues, and zone of truth.
Saving throws and spell resistance might apply to these spells' effects. (See the individual spell descriptions for details.)
An area can receive only one unhallow spell (and its associated spell effect) at a time.
Unhallow counters but does not dispel hallow.
\shortability{Material Component:}{Herbs, oils, and incense worth at least 1,000 gp, plus 1,000 gp per level of the spell to be tied to the unhallowed area.}

\featname{Unholy Aura}
\shortability{Abjuration}{[Evil]}
\shortability{Level:}{Clr 8, Evil 8}
\shortability{Components:}{V, S, F}
\shortability{Casting Time:}{1 standard action}
\shortability{Range:}{20 ft.}
\shortability{Targets:}{One creature/level in a 20-ft.-radius burst centered on you}
\shortability{Duration:}{1 round/level (D)}
\shortability{Saving Throw:}{See text}
\shortability{Spell Resistance:}{Yes (harmless)}
A malevolent darkness surrounds the subjects, protecting them from attacks, granting them resistance to spells cast by good creatures, and weakening good creatures when they strike the subjects. This abjuration has four effects.
First, each warded creature gains a +4 deflection bonus to AC and a +4 resistance bonus on saves. Unlike the effect of protection from good, this benefit applies against all attacks, not just against attacks by good creatures.
Second, a warded creature gains spell resistance 25 against good spells and spells cast by good creatures.
Third, the abjuration blocks possession and mental influence, just as protection from good does.
Finally, if a good creature succeeds on a melee attack against a warded creature, the offending attacker takes 1d6 points of temporary Strength damage (Fortitude negates).
\shortability{Focus:}{A tiny reliquary containing some sacred relic, such as a piece of parchment from an unholy text. The reliquary costs at least 500 gp.}

\featname{Unholy Blight}
\shortability{Evocation}{[Evil]}
\shortability{Level:}{Evil 4}
\shortability{Components:}{V, S}
\shortability{Casting Time:}{1 standard action}
\shortability{Range:}{Medium (100 ft. + 10 ft./level)}
\shortability{Area:}{20-ft.-radius spread}
\shortability{Duration:}{Instantaneous (1d4 rounds); see text}
\shortability{Saving Throw:}{Will partial}
\shortability{Spell Resistance:}{Yes}
You call up unholy power to smite your enemies. The power takes the form of a cold, cloying miasma of greasy darkness.
Only good and neutral (not evil) creatures are harmed by the spell.
The spell deals 1d8 points of damage per two caster levels (maximum 5d8) to a good creature (or 1d6 per caster level, maximum 10d6, to a good outsider) and causes it to be sickened for 1d4 rounds. A successful Will save reduces damage to half and negates the sickened effect. The effects cannot be negated by remove disease or heal, but remove curse is effective.
The spell deals only half damage to creatures who are neither evil nor good, and they are not sickened. Such a creature can reduce the damage in half again (down to one-quarter) with a successful Will save.

\featname{Unseen Servant}
\shortability{Conjuration}{(Creation)}
\shortability{Level:}{Brd 1, Sor/Wiz 1}
\shortability{Components:}{V, S, M}
\shortability{Casting Time:}{1 standard action}
\shortability{Range:}{Close (25 ft. + 5 ft./2 levels)}
\shortability{Effect:}{One invisible, mindless, shapeless servant}
\shortability{Duration:}{1 hour/level}
\shortability{Saving Throw:}{None}
\shortability{Spell Resistance:}{No}
An unseen servant is an invisible, mindless, shapeless force that performs simple tasks at your command. It can run and fetch things, open unstuck doors, and hold chairs, as well as clean and mend. The servant can perform only one activity at a time, but it repeats the same activity over and over again if told to do so as long as you remain within range. It can open only normal doors, drawers, lids, and the like. It has an effective Strength score of 2 (so it can lift 20 pounds or drag 100 pounds). It can trigger traps and such, but it can exert only 20 pounds of force, which is not enough to activate certain pressure plates and other devices. It can't perform any task that requires a skill check with a DC higher than 10 or that requires a check using a skill that can't be used untrained. Its speed is 15 feet.
The servant cannot attack in any way; it is never allowed an attack roll. It cannot be killed, but it dissipates if it takes 6 points of damage from area attacks. (It gets no saves against attacks.) If you attempt to send it beyond the spell's range (measured from your current position), the servant ceases to exist.
\shortability{Material Component:}{A piece of string and a bit of wood.}

\featname{Vampiric Touch}
\shortability{Necromancy}{}
\shortability{Level:}{Sor/Wiz 3}
\shortability{Components:}{V, S}
\shortability{Casting Time:}{1 standard action}
\shortability{Range:}{Touch}
\shortability{Target:}{Living creature touched}
\shortability{Duration:}{Instantaneous/1 hour; see text}
\shortability{Saving Throw:}{None}
\shortability{Spell Resistance:}{Yes}
You must succeed on a melee touch attack. Your touch deals 1d6 points of damage per two caster levels (maximum 10d6). You gain temporary hit points equal to the damage you deal. However, you can't gain more than the subject's current hit points +10, which is enough to kill the subject. The temporary hit points disappear 1 hour later.

\featname{Veil}
\shortability{Illusion}{(Glamer)}
\shortability{Level:}{Brd 6, Sor/Wiz 6}
\shortability{Components:}{V, S}
\shortability{Casting Time:}{1 standard action}
\shortability{Range:}{Long (400 ft. + 40 ft./level)}
\shortability{Targets:}{One or more creatures, no two of which can be more than 30 ft. apart}
\shortability{Duration:}{Concentration + 1 hour/level (D)}
\shortability{Saving Throw:}{Will negates; see text}
\shortability{Spell Resistance:}{Yes; see text}
You instantly change the appearance of the subjects and then maintain that appearance for the spell's duration. You can make the subjects appear to be anything you wish. The subjects look, feel, and smell just like the creatures the spell makes them resemble. Affected creatures resume their normal appearances if slain. You must succeed on a Disguise check to duplicate the appearance of a specific individual. This spell gives you a +10 bonus on the check.
Unwilling targets can negate the spell's effect on them by making Will saves or with spell resistance. Those who interact with the subjects can attempt Will disbelief saves to see through the glamer, but spell resistance doesn't help.

\featname{Ventriloquism}
\shortability{Illusion}{(Figment)}
\shortability{Level:}{Brd 1, Sor/Wiz 1}
\shortability{Components:}{V, F}
\shortability{Casting Time:}{1 standard action}
\shortability{Range:}{Close (25 ft. + 5 ft./2 levels)}
\shortability{Effect:}{Intelligible sound, usually speech}
\shortability{Duration:}{1 min./level (D)}
\shortability{Saving Throw:}{Will disbelief (if interacted with)}
\shortability{Spell Resistance:}{No}
You can make your voice (or any sound that you can normally make vocally) seem to issue from someplace else. You can speak in any language you know. With respect to such voices and sounds, anyone who hears the sound and rolls a successful save recognizes it as illusory (but still hears it).
\shortability{Focus:}{A parchment rolled up into a small cone.}

\featname{Virtue}
\shortability{Transmutation}{}
\shortability{Level:}{Clr 0, Drd 0, Pal 1}
\shortability{Components:}{V, S, DF}
\shortability{Casting Time:}{1 standard action}
\shortability{Range:}{Touch}
\shortability{Target:}{Creature touched}
\shortability{Duration:}{1 min.}
\shortability{Saving Throw:}{Fortitude negates (harmless)}
\shortability{Spell Resistance:}{Yes (harmless)}
The subject gains 1 temporary hit point.

\featname{Vision}
\shortability{Divination}{}
\shortability{Level:}{Sor/Wiz 7}
\shortability{Components:}{V, S, M, XP}
\shortability{Casting Time:}{1 standard action}
This spell functions like legend lore, except that it works more quickly but produces some strain on you. You pose a question about some person, place, or object, then cast the spell. If the person or object is at hand or if you are in the place in question, you receive a vision about it by succeeding on a caster level check (1d20 +1 per caster level; maximum +25) against DC 20. If only detailed information on the person, place, or object is known, the DC is 25, and the information gained is incomplete. If only rumors are known, the DC is 30, and the information gained is vague.
\shortability{XP Cost:}{100 XP.}

\featname{Wail of the Banshee}
\shortability{Necromancy}{[Death, Sonic]}
\shortability{Level:}{Death 9, Sor/Wiz 9}
\shortability{Components:}{V}
\shortability{Casting Time:}{1 standard action}
\shortability{Range:}{Close (25 ft. + 5 ft./2 levels)}
\shortability{Area:}{One living creature/level within a 40-ft.-radius spread}
\shortability{Duration:}{Instantaneous}
\shortability{Saving Throw:}{Fortitude negates}
\shortability{Spell Resistance:}{Yes}
You emit a terrible scream that kills creatures that hear it (except for yourself ). Creatures closest to the point of origin are affected first.

\featname{Wall of Fire}
\shortability{Evocation}{[Fire]}
\shortability{Level:}{Drd 5, Fire 4, Sor/Wiz 4}
\shortability{Components:}{V, S, M/DF}
\shortability{Casting Time:}{1 standard action}
\shortability{Range:}{Medium (100 ft. + 10 ft./level)}
\shortability{Effect:}{Opaque sheet of flame up to 20 ft. long/level or a ring of fire with a radius of up to 5 ft. per two levels; either form 20 ft. high}
\shortability{Duration:}{Concentration + 1 round/level}
\shortability{Saving Throw:}{None}
\shortability{Spell Resistance:}{Yes}
An immobile, blazing curtain of shimmering violet fire springs into existence. One side of the wall, selected by you, sends forth waves of heat, dealing 2d4 points of fire damage to creatures within 10 feet and 1d4 points of fire damage to those past 10 feet but within 20 feet. The wall deals this damage when it appears and on your turn each round to all creatures in the area. In addition, the wall deals 2d6 points of fire damage +1 point of fire damage per caster level (maximum +20) to any creature passing through it. The wall deals double damage to undead creatures.
If you evoke the wall so that it appears where creatures are, each creature takes damage as if passing through the wall. If any 5-foot length of wall takes 20 points of cold damage or more in 1 round, that length goes out. (Do not divide cold damage by 4, as normal for objects.)
Wall of fire can be made permanent with a permanency spell. A permanent wall of fire that is extinguished by cold damage becomes inactive for 10 minutes, then reforms at normal strength.
\shortability{Arcane Material Component:}{A small piece of phosphorus.}

\featname{Wall of Force}
\shortability{Evocation}{[Force]}
\shortability{Level:}{Sor/Wiz 5}
\shortability{Components:}{V, S, M}
\shortability{Casting Time:}{1 standard action}
\shortability{Range:}{Close (25 ft. + 5 ft./2 levels)}
\shortability{Effect:}{Wall whose area is up to one 10-ft. square/level}
\shortability{Duration:}{1 round /level (D)}
\shortability{Saving Throw:}{None}
\shortability{Spell Resistance:}{No}
A wall of force spell creates an invisible wall of force. The wall cannot move, it is immune to damage of all kinds, and it is unaffected by most spells, including dispel magic. However, disintegrate immediately destroys it, as does a rod of cancellation, a sphere of annihilation, or a mage's disjunction spell. Breath weapons and spells cannot pass through the wall in either direction, although dimension door, teleport, and similar effects can bypass the barrier. It blocks ethereal creatures as well as material ones (though ethereal creatures can usually get around the wall by floating under or over it through material floors and ceilings). Gaze attacks can operate through a wall of force.
The caster can form the wall into a flat, vertical plane whose area is up to one 10- foot square per level. The wall must be continuous and unbroken when formed. If its surface is broken by any object or creature, the spell fails.
Wall of force can be made permanent with a permanency spell.
\shortability{Material Component:}{A pinch of powder made from a clear gem.}

\featname{Wall of Ice}
\shortability{Evocation}{[Cold]}
\shortability{Level:}{Sor/Wiz 4}
\shortability{Components:}{V, S, M}
\shortability{Casting Time:}{1 standard action}
\shortability{Range:}{Medium (100 ft. + 10 ft./level)}
\shortability{Effect:}{Anchored plane of ice, up to one 10-ft. square/level, or hemisphere of ice with a radius of up to 3 ft. + 1 ft./level}
\shortability{Duration:}{1 min./level}
\shortability{Saving Throw:}{Reflex negates; see text}
\shortability{Spell Resistance:}{Yes}
This spell creates an anchored plane of ice or a hemisphere of ice, depending on the version selected. A wall of ice cannot form in an area occupied by physical objects or creatures. Its surface must be smooth and unbroken when created. Any creature adjacent to the wall when it is created may attempt a Reflex save to disrupt the wall as it is being formed. A successful save indicates that the spell automatically fails. Fire can melt a wall of ice, and it deals full damage to the wall (instead of the normal half damage taken by objects). Suddenly melting a wall of ice creates a great cloud of steamy fog that lasts for 10 minutes.
Ice Plane: A sheet of strong, hard ice appears. The wall is 1 inch thick per caster level. It covers up to a 10-foot-square area per caster level (so a 10th-level wizard can create a wall of ice 100 feet long and 10 feet high, a wall 50 feet long and 20 feet high, or some other combination of length and height that does not exceed 1,000 square feet). The plane can be oriented in any fashion as long as it is anchored. A vertical wall need only be anchored on the floor, while a horizontal or slanting wall must be anchored on two opposite sides.
Each 10-foot square of wall has 3 hit points per inch of thickness. Creatures can hit the wall automatically. A section of wall whose hit points drop to 0 is breached. If a creature tries to break through the wall with a single attack, the DC for the Strength check is 15 + caster level.
Even when the ice has been broken through, a sheet of frigid air remains. Any creature stepping through it (including the one who broke through the wall) takes 1d6 points of cold damage +1 point per caster level (no save).
Hemisphere: The wall takes the form of a hemisphere whose maximum radius is 3 feet + 1 foot per caster level. The hemisphere is as hard to break through as the ice plane form, but it does not deal damage to those who go through a breach.
\shortability{Material Component:}{A small piece of quartz or similar rock crystal.}

\featname{Wall of Iron}
\shortability{Conjuration}{(Creation)}
\shortability{Level:}{Sor/Wiz 6}
\shortability{Components:}{V, S, M}
\shortability{Casting Time:}{1 standard action}
\shortability{Range:}{Medium (100 ft. + 10 ft./level)}
\shortability{Effect:}{Iron wall whose area is up to one 5-ft. square/level; see text}
\shortability{Duration:}{Instantaneous}
\shortability{Saving Throw:}{See text}
\shortability{Spell Resistance:}{No}
You cause a flat, vertical iron wall to spring into being. The wall inserts itself into any surrounding nonliving material if its area is sufficient to do so. The wall cannot be conjured so that it occupies the same space as a creature or another object. It must always be a flat plane, though you can shape its edges to fit the available space.
A wall of iron is 1 inch thick per four caster levels. You can double the wall's area by halving its thickness. Each 5- foot square of the wall has 30 hit points per inch of thickness and hardness 10. A section of wall whose hit points drop to 0 is breached. If a creature tries to break through the wall with a single attack, the DC for the Strength check is 25 + 2 per inch of thickness.
If you desire, the wall can be created vertically resting on a flat surface but not attached to the surface, so that it can be tipped over to fall on and crush creatures beneath it. The wall is 50\% likely to tip in either direction if left unpushed. Creatures can push the wall in one direction rather than letting it fall randomly. A creature must make a DC 40 Strength check to push the wall over. Creatures with room to flee the falling wall may do so by making successful Reflex saves. Any Large or smaller creature that fails takes 10d6 points of damage. The wall cannot crush Huge and larger creatures.
Like any iron wall, this wall is subject to rust, perforation, and other natural phenomena.
\shortability{Material Component:}{A small piece of sheet iron plus gold dust worth 50 gp (1 pound of gold dust).}

\featname{Wall of Stone}
\shortability{Conjuration}{(Creation) [Earth]}
\shortability{Level:}{Clr 5, Drd 6, Earth 5, Sor/Wiz 5}
\shortability{Components:}{V, S, M/DF}
\shortability{Casting Time:}{1 standard action}
\shortability{Range:}{Medium (100 ft. + 10 ft./level)}
\shortability{Effect:}{Stone wall whose area is up to one 5-ft. square/level (S)}
\shortability{Duration:}{Instantaneous}
\shortability{Saving Throw:}{See text}
\shortability{Spell Resistance:}{No}
This spell creates a wall of rock that merges into adjoining rock surfaces. A wall of stone is 1 inch thick per four caster levels and composed of up to one 5-foot square per level. You can double the wall's area by halving its thickness. The wall cannot be conjured so that it occupies the same space as a creature or another object.
Unlike a wall of iron, you can create a wall of stone in almost any shape you desire. The wall created need not be vertical, nor rest upon any firm foundation; however, it must merge with and be solidly supported by existing stone. It can be used to bridge a chasm, for instance, or as a ramp. For this use, if the span is more than 20 feet, the wall must be arched and buttressed. This requirement reduces the spell's area by half. The wall can be crudely shaped to allow crenellations, battlements, and so forth by likewise reducing the area.
Like any other stone wall, this one can be destroyed by a disintegrate spell or by normal means such as breaking and chipping. Each 5-foot square of the wall has 15 hit points per inch of thickness and hardness 8. A section of wall whose hit points drop to 0 is breached. If a creature tries to break through the wall with a single attack, the DC for the Strength check is 20 + 2 per inch of thickness.
It is possible, but difficult, to trap mobile opponents within or under a wall of stone, provided the wall is shaped so it can hold the creatures. Creatures can avoid entrapment with successful Reflex saves.
\shortability{Arcane Material Component:}{A small block of granite.}

\featname{Wall of Thorns}
\shortability{Conjuration}{(Creation)}
\shortability{Level:}{Drd 5, Plant 5}
\shortability{Components:}{V, S}
\shortability{Casting Time:}{1 standard action}
\shortability{Range:}{Medium (100 ft. + 10 ft./level)}
\shortability{Effect:}{Wall of thorny brush, up to one 10-ft. cube/level (S)}
\shortability{Duration:}{10 min./level (D)}
\shortability{Saving Throw:}{None}
\shortability{Spell Resistance:}{No}
A wall of thorns spell creates a barrier of very tough, pliable, tangled brush bearing needle-sharp thorns as long as a human's finger. Any creature forced into or attempting to move through a wall of thorns takes slashing damage per round of movement equal to 25 minus the creature's AC. Dexterity and dodge bonuses to AC do not count for this calculation. (Creatures with an Armor Class of 25 or higher, without considering Dexterity and dodge bonuses, take no damage from contact with the wall.)
You can make the wall as thin as 5 feet thick, which allows you to shape the wall as a number of 10-by-10-by-5-foot blocks equal to twice your caster level. This has no effect on the damage dealt by the thorns, but any creature attempting to break through takes that much less time to force its way through the barrier.
Creatures can force their way slowly through the wall by making a Strength check as a full-round action. For every 5 points by which the check exceeds 20, a creature moves 5 feet (up to a maximum distance equal to its normal land speed). Of course, moving or attempting to move through the thorns incurs damage as described above. A creature trapped in the thorns can choose to remain motionless in order to avoid taking any more damage.
Any creature within the area of the spell when it is cast takes damage as if it had moved into the wall and is caught inside. In order to escape, it must attempt to push its way free, or it can wait until the spell ends. Creatures with the ability to pass through overgrown areas unhindered can pass through a wall of thorns at normal speed without taking damage.
A wall of thorns can be breached by slow work with edged weapons. Chopping away at the wall creates a safe passage 1 foot deep for every 10 minutes of work. Normal fire cannot harm the barrier, but magical fire burns it away in 10 minutes.
Despite its appearance, a wall of thorns is not actually a living plant, and thus is unaffected by spells that affect plants.

\featname{Warp Wood}
\shortability{Transmutation}{}
\shortability{Level:}{Drd 2}
\shortability{Components:}{V, S}
\shortability{Casting Time:}{1 standard action}
\shortability{Range:}{Close (25 ft. + 5 ft./2 levels)}
\shortability{Target:}{1 Small wooden object/level, all within a 20-ft. radius}
\shortability{Duration:}{Instantaneous}
\shortability{Saving Throw:}{Will negates (object)}
\shortability{Spell Resistance:}{Yes (object)}
You cause wood to bend and warp, permanently destroying its straightness, form, and strength. A warped door springs open (or becomes stuck, requiring a Strength check to open, at your option). A boat or ship springs a leak. Warped ranged weapons are useless. A warped melee weapon causes a -4 penalty on attack rolls.
You may warp one Small or smaller object or its equivalent per caster level. A Medium object counts as two Small objects, a Large object as four, a Huge object as eight, a Gargantuan object as sixteen, and a Colossal object as thirty-two.
Alternatively, you can unwarp wood (effectively warping it back to normal) with this spell, straightening wood that has been warped by this spell or by other means. Make whole, on the other hand, does no good in repairing a warped item.
You can combine multiple consecutive warp wood spells to warp (or unwarp) an object that is too large for you to warp with a single spell. 
Until the object is completely warped, it suffers no ill effects.

\featname{Water Breathing}
\shortability{Transmutation}{}
\shortability{Level:}{Clr 3, Drd 3, Sor/Wiz 3, Water 3}
\shortability{Components:}{V, S, M/DF}
\shortability{Casting Time:}{1 standard action}
\shortability{Range:}{Touch}
\shortability{Target:}{Living creatures touched}
\shortability{Duration:}{2 hours/level; see text}
\shortability{Saving Throw:}{Will negates (harmless)}
\shortability{Spell Resistance:}{Yes (harmless)}
The transmuted creatures can breathe water freely. Divide the duration evenly among all the creatures you touch.
The spell does not make creatures unable to breathe air.
\shortability{Arcane Material Component:}{A short reed or piece of straw.}

\featname{Water Walk}
\shortability{Transmutation}{[Water]}
\shortability{Level:}{Clr 3, Rgr 3}
\shortability{Components:}{V, S, DF}
\shortability{Casting Time:}{1 standard action}
\shortability{Range:}{Touch}
\shortability{Targets:}{One touched creature/level}
\shortability{Duration:}{10 min./level (D)}
\shortability{Saving Throw:}{Will negates (harmless)}
\shortability{Spell Resistance:}{Yes (harmless)}
The transmuted creatures can tread on any liquid as if it were firm ground. Mud, oil, snow, quicksand, running water, ice, and even lava can be traversed easily, since the subjects' feet hover an inch or two above the surface. (Creatures crossing molten lava still take damage from the heat because they are near it.) The subjects can walk, run, charge, or otherwise move across the surface as if it were normal ground.
If the spell is cast underwater (or while the subjects are partially or wholly submerged in whatever liquid they are in), the subjects are borne toward the surface at 60 feet per round until they can stand on it.

\featname{Waves of Exhaustion}
\shortability{Necromancy}{}
\shortability{Level:}{Sor/Wiz 7}
\shortability{Components:}{V, S}
\shortability{Casting Time:}{1 standard action}
\shortability{Range:}{60 ft.}
\shortability{Area:}{Cone-shaped burst}
\shortability{Duration:}{Instantaneous}
\shortability{Saving Throw:}{No}
\shortability{Spell Resistance:}{Yes}
Waves of negative energy cause all living creatures in the spell's area to become exhausted. This spell has no effect on a creature that is already exhausted.

\featname{Waves of Fatigue}
\shortability{Necromancy}{}
\shortability{Level:}{Sor/Wiz 5}
\shortability{Components:}{V, S}
\shortability{Casting Time:}{1 standard action}
\shortability{Range:}{30 ft.}
\shortability{Area:}{Cone-shaped burst}
\shortability{Duration:}{Instantaneous}
\shortability{Saving Throw:}{No}
\shortability{Spell Resistance:}{Yes}
Waves of negative energy render all living creatures in the spell's area fatigued. This spell has no effect on a creature that is already fatigued.

\featname{Web}
\shortability{Conjuration}{(Creation)}
\shortability{Level:}{Sor/Wiz 2}
\shortability{Components:}{V, S, M}
\shortability{Casting Time:}{1 standard action}
\shortability{Range:}{Medium (100 ft. + 10 ft./level)}
\shortability{Effect:}{Webs in a 20-ft.-radius spread}
\shortability{Duration:}{10 min./level (D)}
\shortability{Saving Throw:}{Reflex negates; see text}
\shortability{Spell Resistance:}{No}
Web creates a many-layered mass of strong, sticky strands. These strands trap those caught in them. The strands are similar to spider webs but far larger and tougher. These masses must be anchored to two or more solid and diametrically opposed points or else the web collapses upon itself and disappears. Creatures caught within a web become entangled among the gluey fibers. Attacking a creature in a web won't cause you to become entangled.
Anyone in the effect's area when the spell is cast must make a Reflex save. If this save succeeds, the creature is entangled, but not prevented from moving, though moving is more difficult than normal for being entangled (see below). If the save fails, the creature is entangled and can't move from its space, but can break loose by spending 1 round and making a DC 20 Strength check or a DC 25 Escape Artist check. Once loose (either by making the initial Reflex save or a later Strength check or Escape Artist check), a creature remains entangled, but may move through the web very slowly. Each round devoted to moving allows the creature to make a new Strength check or Escape Artist check. The creature moves 5 feet for each full 5 points by which the check result exceeds 10.
If you have at least 5 feet of web between you and an opponent, it provides cover. If you have at least 20 feet of web between you, it provides total cover.
The strands of a web spell are flammable. A magic flaming sword can slash them away as easily as a hand brushes away cobwebs. Any fire can set the webs alight and burn away 5 square feet in 1 round. All creatures within flaming webs take 2d4 points of fire damage from the flames.
Web can be made permanent with a permanency spell. A permanent web that is damaged (but not destroyed) regrows in 10 minutes.
\shortability{Material Component:}{A bit of spider web.}

\featname{Weird}
\shortability{Illusion}{(Phantasm) [Fear, Mind-Affecting]}
\shortability{Level:}{Sor/Wiz 9}
\shortability{Targets:}{Any number of creatures, no two of which can be more than 30 ft. apart}
This spell functions like phantasmal killer, except it can affect more than one creature. Only the affected creatures see the phantasmal creatures attacking them, though you see the attackers as shadowy shapes.
If a subject's Fortitude save succeeds, it still takes 3d6 points of damage and is stunned for 1 round. The subject also takes 1d4 points of temporary Strength damage.

\featname{Whirlwind}
\shortability{Evocation}{[Air]}
\shortability{Level:}{Air 8, Drd 8}
\shortability{Components:}{V, S, DF}
\shortability{Casting Time:}{1 standard action}
\shortability{Range:}{Long (400 ft. + 40 ft./level)}
\shortability{Effect:}{Cyclone 10 ft. wide at base, 30 ft. wide at top, and 30 ft. tall}
\shortability{Duration:}{1 round/level (D)}
\shortability{Saving Throw:}{Reflex negates; see text}
\shortability{Spell Resistance:}{Yes}
This spell creates a powerful cyclone of raging wind that moves through the air, along the ground, or over water at a speed of 60 feet per round. You can concentrate on controlling the cyclone's every movement or specify a simple program. Directing the cyclone's movement or changing its programmed movement is a standard action for you. The cyclone always moves during your turn. If the cyclone exceeds the spell's range, it moves in a random, uncontrolled fashion for 1d3 rounds and then dissipates. (You can't regain control of the cyclone, even if comes back within range.)
Any Large or smaller creature that comes in contact with the spell effect must succeed on a Reflex save or take 3d6 points of damage. A Medium or smaller creature that fails its first save must succeed on a second one or be picked up bodily by the cyclone and held suspended in its powerful winds, taking 1d8 points of damage each round on your turn with no save allowed. You may direct the cyclone to eject any carried creatures whenever you wish, depositing the hapless souls wherever the cyclone happens to be when they are released.

\featname{Whispering Wind}
\shortability{Transmutation}{[Air]}
\shortability{Level:}{Brd 2, Sor/Wiz 2}
\shortability{Components:}{V, S}
\shortability{Casting Time:}{1 standard action}
\shortability{Range:}{1 mile/level}
\shortability{Area:}{10-ft.-radius spread}
\shortability{Duration:}{No more than 1 hour/level or until discharged (destination is reached)}
\shortability{Saving Throw:}{None}
\shortability{Spell Resistance:}{No}
You send a message or sound on the wind to a designated spot. The whispering wind travels to a specific location within range that is familiar to you, provided that it can find a way to the location. A whispering wind is as gentle and unnoticed as a zephyr until it reaches the location. It then delivers its whisper-quiet message or other sound. Note that the message is delivered regardless of whether anyone is present to hear it. The wind then dissipates.
You can prepare the spell to bear a message of no more than twenty-five words, cause the spell to deliver other sounds for 1 round, or merely have the whispering wind seem to be a faint stirring of the air. You can likewise cause the whispering wind to move as slowly as 1 mile per hour or as quickly as 1 mile per 10 minutes.
When the spell reaches its objective, it swirls and remains in place until the message is delivered. As with magic mouth, whispering wind cannot speak verbal components, use command words, or activate magical effects.

\featname{Wind Walk}
\shortability{Transmutation}{[Air]}
\shortability{Level:}{Clr 6, Drd 7}
\shortability{Components:}{V, S, DF}
\shortability{Casting Time:}{1 standard action}
\shortability{Range:}{Touch}
\shortability{Targets:}{You and one touched creature per three levels}
\shortability{Duration:}{1 hour/level (D); see text}
\shortability{Saving Throw:}{No and Will negates (harmless)}
\shortability{Spell Resistance:}{No and Yes (harmless)}
You alter the substance of your body to a cloudlike vapor (as the gaseous form spell) and move through the air, possibly at great speed. You can take other creatures with you, each of which acts independently.
Normally, a wind walker flies at a speed of 10 feet with perfect maneuverability. If desired by the subject, a magical wind wafts a wind walker along at up to 600 feet per round (60 mph) with poor maneuverability. Wind walkers are not invisible but rather appear misty and translucent. If fully clothed in white, they are 80\% likely to be mistaken for clouds, fog, vapors, or the like.
A wind walker can regain its physical form as desired and later resume the cloud form. Each change to and from vaporous form takes 5 rounds, which counts toward the duration of the spell (as does any time spent in physical form). As noted above, you can dismiss the spell, and you can even dismiss it for individual wind walkers and not others.
For the last minute of the spell's duration, a wind walker in cloud form automatically descends 60 feet per round (for a total of 600 feet), though it may descend faster if it wishes. This descent serves as a warning that the spell is about to end.

\featname{Wind Wall}
\shortability{Evocation}{[Air]}
\shortability{Level:}{Air 2, Clr 3, Drd 3, Rgr 2, Sor/Wiz 3}
\shortability{Components:}{V, S, M/DF}
\shortability{Casting Time:}{1 standard action}
\shortability{Range:}{Medium (100 ft. + 10 ft./level)}
\shortability{Effect:}{Wall up to 10 ft./level long and 5 ft./level high (S)}
\shortability{Duration:}{1 round/level}
\shortability{Saving Throw:}{None; see text}
\shortability{Spell Resistance:}{Yes}
An invisible vertical curtain of wind appears. It is 2 feet thick and of considerable strength. It is a roaring blast sufficient to blow away any bird smaller than an eagle, or tear papers and similar materials from unsuspecting hands. (A Reflex save allows a creature to maintain its grasp on an object.) Tiny and Small flying creatures cannot pass through the barrier. Loose materials and cloth garments fly upward when caught in a wind wall. Arrows and bolts are deflected upward and miss, while any other normal ranged weapon passing through the wall has a 30\% miss chance. (A giant-thrown boulder, a siege engine projectile, and other massive ranged weapons are not affected.) Gases, most gaseous breath weapons, and creatures in gaseous form cannot pass through the wall (although it is no barrier to incorporeal creatures).
While the wall must be vertical, you can shape it in any continuous path along the ground that you like. It is possible to create cylindrical or square wind walls to enclose specific points. 
\shortability{Arcane Material Component:}{A tiny fan and a feather of exotic origin.}

\featname{Wish}
\shortability{Universa}{}
\shortability{Level:}{Sor/Wiz 9}
\shortability{Components:}{V, XP}
\shortability{Casting Time:}{1 standard action}
\shortability{Range:}{See text}
\shortability{Target, Effect, or Area:}{See text}
\shortability{Duration:}{See text}
\shortability{Saving Throw:}{See text}
\shortability{Spell Resistance:}{Yes}
Wish is the mightiest spell a wizard or sorcerer can cast. By simply speaking aloud, you can alter reality to better suit you.
Even wish, however, has its limits.
A wish can produce any one of the following effects.
o Duplicate any wizard or sorcerer spell of 8th level or lower, provided the spell is not of a school prohibited to you.
o Duplicate any other spell of 6th level or lower, provided the spell is not of a school prohibited to you.
o Duplicate any wizard or sorcerer spell of 7th level or lower even if it's of a prohibited school.
o Duplicate any other spell of 5th level or lower even if it's of a prohibited school. 
o Undo the harmful effects of many other spells, such as geas/quest or insanity.
o Create a nonmagical item of up to 25,000 gp in value.
o Create a magic item, or add to the powers of an existing magic item.
o Grant a creature a +1 inherent bonus to an ability score. Two to five wish spells cast in immediate succession can grant a creature a +2 to +5 inherent bonus to an ability score (two wishes for a +2 inherent bonus, three for a +3 inherent bonus, and so on). Inherent bonuses are instantaneous, so they cannot be dispelled. Note: An inherent bonus may not exceed +5 for a single ability score, and inherent bonuses to a particular ability score do not stack, so only the best one applies.
o Remove injuries and afflictions. A single wish can aid one creature per caster level, and all subjects are cured of the same kind of affliction. For example, you could heal all the damage you and your companions have taken, or remove all poison effects from everyone in the party, but not do both with the same wish. A wish can never restore the experience point loss from casting a spell or the level or Constitution loss from being raised from the dead.
o Revive the dead. A wish can bring a dead creature back to life by duplicating a resurrection spell. A wish can revive a dead creature whose body has been destroyed, but the task takes two wishes, one to recreate the body and another to infuse the body with life again. A wish cannot prevent a character who was brought back to life from losing an experience level.
o Transport travelers. A wish can lift one creature per caster level from anywhere on any plane and place those creatures anywhere else on any plane regardless of local conditions. An unwilling target gets a Will save to negate the effect, and spell resistance (if any) applies.
o Undo misfortune. A wish can undo a single recent event. The wish forces a reroll of any roll made within the last round (including your last turn). Reality reshapes itself to accommodate the new result. For example, a wish could undo an opponent's successful save, a foe's successful critical hit (either the attack roll or the critical roll), a friend's failed save, and so on. The reroll, however, may be as bad as or worse than the original roll. An unwilling target gets a Will save to negate the effect, and spell resistance (if any) applies.
You may try to use a wish to produce greater effects than these, but doing so is dangerous. (The wish may pervert your intent into a literal but undesirable fulfillment or only a partial fulfillment.)
Duplicated spells allow saves and spell resistance as normal (but save DCs are for 9th-level spells).
\shortability{Material Component:}{When a wish duplicates a spell with a material component that costs more than 10,000 gp, you must provide that component.}
\shortability{XP Cost:}{The minimum XP cost for casting wish is 5,000 XP. When a wish duplicates a spell that has an XP cost, you must pay 5,000 XP or that cost, whichever is more. When a wish creates or improves a magic item, you must pay twice the normal XP cost for crafting or improving the item, plus an additional 5,000 XP.}

\featname{Wood Shape}
\shortability{Transmutation}{}
\shortability{Level:}{Drd 2}
\shortability{Components:}{V, S, DF}
\shortability{Casting Time:}{1 standard action}
\shortability{Range:}{Touch}
\shortability{Target:}{One touched piece of wood no larger than 10 cu. ft. + 1 cu. ft./level}
\shortability{Duration:}{Instantaneous}
\shortability{Saving Throw:}{Will negates (object)}
\shortability{Spell Resistance:}{Yes (object)}
Wood shape enables you to form one existing piece of wood into any shape that suits your purpose. While it is possible to make crude coffers, doors, and so forth, fine detail isn't possible. There is a 30\% chance that any shape that includes moving parts simply doesn't work.

\featname{Word of Chaos}
\shortability{Evocation}{[Chaotic, Sonic]}
\shortability{Level:}{Chaos 7, Clr 7}
\shortability{Components:}{V}
\shortability{Casting Time:}{1 standard action}
\shortability{Range:}{40 ft.}
\shortability{Area:}{Nonchaotic creatures in a 40-ft.- radius spread centered on you}
\shortability{Duration:}{Instantaneous}
\shortability{Saving Throw:}{None or Will negates; see text}
\shortability{Spell Resistance:}{Yes}
Any nonchaotic creature within the area who hears the word of chaos suffers the following ill effects.
The effects are cumulative and concurrent. No saving throw is allowed against these effects.
Deafened: The creature is deafened for 1d4 rounds.
Stunned: The creature is stunned for 1 round.
Confused: The creature is confused, as by the confusion spell, for 1d10 minutes. This is a mind-affecting enchantment effect.
Killed: Living creatures die. Undead creatures are destroyed.
 
HD	Effect	   
Equal to caster level	Deafened	   
Up to caster level -1	Stunned, deafened	   
Up to caster level -5	Confused, stunned, deafened	   
Up to caster level -10	Killed, confused, stunned, deafened	 
Furthermore, if you are on your home plane when you cast this spell, nonchaotic extraplanar creatures within the area are instantly banished back to their home planes. Creatures so banished cannot return for at least 24 hours. This effect takes place regardless of whether the creatures hear the word of chaos. The banishment effect allows a Will save (at a -4 penalty) to negate.
Creatures whose HD exceed your caster level are unaffected by word of chaos.

\featname{Word of Recall}
\shortability{Conjuration}{(Teleportation)}
\shortability{Level:}{Clr 6, Drd 8}
\shortability{Components:}{V}
\shortability{Casting Time:}{1 standard action}
\shortability{Range:}{Unlimited}
\shortability{Target:}{You and touched objects or other willing creatures}
\shortability{Duration:}{Instantaneous}
\shortability{Saving Throw:}{None or Will negates (harmless, object)}
\shortability{Spell Resistance:}{No or Yes (harmless, object)}
Word of recall teleports you instantly back to your sanctuary when the word is uttered. You must designate the sanctuary when you prepare the spell, and it must be a very familiar place. The actual point of arrival is a designated area no larger than 10 feet by 10 feet. You can be transported any distance within a plane but cannot travel between planes. You can transport, in addition to yourself, any objects you carry, as long as their weight doesn't exceed your maximum load. You may also bring one additional willing Medium or smaller creature (carrying gear or objects up to its maximum load) or its equivalent per three caster levels. A Large creature counts as two Medium creatures, a Huge creature counts as two Large creatures, and so forth. All creatures to be transported must be in contact with one another, and at least one of those creatures must be in contact with you. Exceeding this limit causes the spell to fail.
An unwilling creature can't be teleported by word of recall. Likewise, a creature's Will save (or spell resistance) prevents items in its possession from being teleported. Unattended, nonmagical objects receive no saving throw.

\featname{Zone of Silence}
\shortability{Illusion}{(Glamer)}
\shortability{Level:}{Brd 4}
\shortability{Components:}{V, S}
\shortability{Casting Time:}{1 round}
\shortability{Range:}{Personal}
\shortability{Area:}{5-ft.-radius emanation centered on you}
\shortability{Duration:}{1 hour/level (D)}
By casting zone of silence, you manipulate sound waves in your immediate vicinity so that you and those within the spell's area can converse normally, yet no one outside can hear your voices or any other noises from within, including language-dependent or sonic spell effects. This effect is centered on you and moves with you. Anyone who enters the zone immediately becomes subject to its effects, but those who leave are no longer affected. Note, however, that a successful Spot check to read lips can still reveal what's said inside a zone of silence.

\featname{Zone of Truth}
\shortability{Enchantment}{(Compulsion) [Mind-Affecting]}
\shortability{Level:}{Clr 2, Pal 2}
\shortability{Components:}{V, S, DF}
\shortability{Casting Time:}{1 standard action}
\shortability{Range:}{Close (25 ft. + 5 ft./2 levels)}
\shortability{Area:}{20-ft.-radius emanation}
\shortability{Duration:}{1 min./level}
\shortability{Saving Throw:}{Will negates}
\shortability{Spell Resistance:}{Yes}
Creatures within the emanation area (or those who enter it) can't speak any deliberate and intentional lies. Each potentially affected creature is allowed a save to avoid the effects when the spell is cast or when the creature first enters the emanation area. Affected creatures are aware of this enchantment. Therefore, they may avoid answering questions to which they would normally respond with a lie, or they may be evasive as long as they remain within the boundaries of the truth. Creatures who leave the area are free to speak as they choose.

\end{small}
\end{multicols}


\section{Celestial Spheres}
\quot{``In brightest day, in darkest night, no evil shall escape my sight.''}

Celestial Spheres function like Fiendish Spheres, except that they are associated with not-Evil rather than not-Good. Any Fiendish Spheres which are not associated with Evil also qualify as Celestial Spheres.

\newcommand{\sphere}[9]{\begin{list}{}{\itemspace}\item \textbf{1:} \spell{#1} \item \textbf{3:} \spell{#2} \item \textbf{5:} \spell{#3} \item \textbf{7:} \spell{#4} \item \textbf{9:} \spell{#5} \item \textbf{11:} \spell{#6} \item \textbf{13:} \spell{#7} \item \textbf{15:} \spell{#8} \item \textbf{17:} \spell{#9}}
\newcommand{\spherecont}[1]{\item\textbf{19:} \spell{#1} \end{list}\medskip}

\begin{multicols}{2}
\section{Spheres} \hypertarget{spheres}{}

\newcommand{\sphere}[9]{\begin{list}{}{\itemspace}\item \textbf{1:} \spell{#1} \item \textbf{3:} \spell{#2} \item \textbf{5:} \spell{#3} \item \textbf{7:} \spell{#4} \item \textbf{9:} \spell{#5} \item \textbf{11:} \spell{#6} \item \textbf{13:} \spell{#7} \item \textbf{15:} \spell{#8} \item \textbf{17:} \spell{#9}}
\newcommand{\spherecont}[1]{\item\textbf{19:} \spell{#1} \end{list}\medskip}

Fiends, celestials, and some characters cast magic primarily through spell-like abilities. While many monsters and characters get arbitrary spell lists, spheres present a way to advance spellcasting in a thematic way. When a creature has access to a sphere, she is able to use all of the abilities within that sphere up to her character level. If they gains more levels, more powers of the sphere become available. In this way the spell-like abilities of fiends created with the rules in this tome should always be \ae sthetically and level appropriate.

\ability{Basic Sphere Access:}{When a creature has basic access to a sphere, she can use any of the spells listed in the sphere may be used once per day (each) as spell-like abilities, provided that their listed level is equal or lower to the creature's character level.}

\ability{Advanced Sphere Access:}{When a creature has advanced access to a sphere, she can use any of the spells listed in the sphere may be used 3 times per day (each) as spell-like abilities, provided that their listed level is equal or lower to the creature's character level.}

\ability{Expert Sphere Access:}{When a creature has expert access to a sphere, any spells listed in the sphere may be used at will as spell-like abilities, provided that their listed level is equal or lower to the creature's character level.}

\ability{Creating new spheres:}{The following list of spheres isn't intended to be comprehensive, and we fully expect that some players and DMs will want many more spheres than we have scribed. All new spheres must be approved of by the DM, and should represent some actual (indifferent or evil) trait like ``intoxication" or ``badgers" rather than a game mechanical notion like ``kicking ass and being totally sweet" or something praiseworthy like ``generosity". A good place to start is actually Domains, as these are already a source by which a character gain a spell at every odd-numbered level.}

\ability{Spheres and Spell Levels:}{Spell-like abilities used out of spheres are considered to be cast as a spell level equal to half the minimum needed character level to use the ability (rounded up). The save DC of a spell-like ability granted through Sphere access is Charisma-based. Thus, the save DC for a spell-like ability which becomes available at character level 5 is 13 + Charisma bonus.}
\end{multicols}

\chapter{It's a Miracle!}

Much like \spell{wish}, \spell{miracle} is a problematic spell that makes us sad. Not because it's open-ended in a way that encourages player creativity, we're all for that. It's because it is more accurately titled ``invoke DM fiat''. You are very specifically spending a 9th level spell slot and 5,000 experience points to \emph{trick the DM into breaking the game for you}. And that needs fixing just as much as something which explicitly lets you break the game does.

Our solution is to very explicitly say what a \spell{miracle} is capable of -- and not capable of. The amount of flexibility is still well within what we want from a \spell{miracle}, there are a lot of 7th level or lower spells that do all kinds of things, so the players -- and the DM -- have plenty of room for creativity without it devolving into a mandated Rule 0 fiasco. There's still some room for interpretation, but that's a matter of what's thematic rather than a matter of effect legitimacy.

\subsubsection{Intercession Points}

\desc{Usually, when you do someone a favor, they are expected to give you a personal favor or some sort of physical (or physical-like) reward. Or they try to kill you, or you try to kill them. But if they're a powerful supernatural being, they can also give you Intercession Points with the reasonable expectation that you won't stab them for not paying up. This is a magically binding contract that, at some point in the future, you can use a portion of their power. These Points are treatable as Wish Economy currency (equal in value to a CR 17 soul), and can be used to create magic items or empower the secondary effects of miracle.}

\desc{If you're using Intercession Points to make magic items, you can also use the abilities of the creature who gave them to you as though they were there assisting you in the item creation.}

\desc{Powerful as they are, Intercession Points are costly to generate �- a creature which wishes to grant one must sacrifice Wish Economy currency/magic items worth an amount equal to that of a CR 17 soul. A creature needs to be at least CR 17 itself to grant an Intercession Point, and it cannot grant one to itself. Usually, these are only paid to people you expect to continue working for you.}

\clearpage
\featname{Miracle}
\begin{small}
\shortability{Evocation}{}
\shortability{Level:}{Clr 9}
\shortability{Components:}{V, S; see text}
\shortability{Casting Time:}{1 Standard Action}
\shortability{Range:}{See text}
\shortability{Target, Effect, or Area:}{See text}
\shortability{Duration:}{See text}
\shortability{Saving Throw:}{See text}
\shortability{Spell Resistance:}{As emulated spell or effect}
\end{small}
\quot{``I tap all my White and Red mana, and everything takes infinity damage.\\
...Everything?\\
\emph{Everything}.''}

\noindent\desc{The caster requests a favor from their deity, or philosophy, or some other supernatural superbeing. This lets them do some pretty amazing stuff.}

\noindent\desc{When cast, a miracle can accomplish one of the following:}

\desc{\listone
    \item Duplicate any cleric spell of up to 8th level, or spells to which the caster personally has access (i.e., from Domains, Attune Sphere, or something like that). This includes spells which might otherwise be disallowed due to alignment constraints. 
    \item Duplicate any other spell of up to 7th level. 
    \item Undo the harmful effects of any spell of up to 8th level. This only applies to direct effects of the spell. 
    \item Any effect which is in line with an 8th level spell. 
\end{list}}

\noindent\desc{In addition, by spending an Intercession Point owed by a particular creature or philosophy, the caster may cause a large-scale effect which is associated in some way with that creature or philosophy. IP from a Balor, for example, might drive everyone within the area of effect insane, and IP from the concept of Death might hit everyone in the area with several negative levels (as per enervation).}

\desc{Upon choosing to spend IP on this effect, the caster chooses a spell of 7th level or lower to emulate which is thematically linked to the creature which granted the IP. This spell's effect is then extended to an area centered on the caster, out to a 10-mile radius (which can be reduced if desired).}

\desc{If the spell has an Area field, the Area becomes a sphere centered on the caster within the chosen radius (the caster may exclude his own space if desired). If it has a Target(s) field, this encompasses all valid targets (with specific exceptions if desired) in that area. If it has an Effect which fills a particular area, the Effect is expanded to fill the new area. Spells without any of these qualities are not affected in any way.}

\desc{For example: if the creature in question were a Balor, the area could be a raging inferno (like a \emph{fireball} which was a 10-mile sphere), or drive everyone within the radius insane (as an \emph{insanity} targeted at each creature in the area), or something else that has to do with Balors.}

\chapter{Celestial Superbeings}

\section{Angels}

\section{Exemplar: The Lawful, the Chaotic, and the Furry}

\chapter{Gods and Divinity}

\chapter{Revised Diplomancy Rules}

\chapter{Celestial and Precelestial Economics}

\section{Hope}

\chapter{High Adventure on the Upper Planes}

%\input{highadventure_upper}

\end{document}