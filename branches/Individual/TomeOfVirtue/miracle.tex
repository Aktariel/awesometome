
\subsubsection{Intercession Points}

\desc{Usually, when you do someone a favor, they are expected to give you a personal favor or some sort of physical (or physical-like) reward. Or they try to kill you, or you try to kill them. But if they�re a powerful supernatural being, they can also give you Intercession Points with the reasonable expectation that you won�t stab them for not paying up. This is a magically binding contract that, at some point in the future, you can use a portion of their power. These Points are treatable as Wish Economy currency (equal in value to a CR 17 soul), and can be used to create magic items or empower the secondary effects of miracle.}

If you�re using Intercession Points to make magic items, you can also use the abilities of the creature who gave them to you as though they were there assisting you in the item creation. 

Powerful as they are, Intercession Points are costly to generate � a creature which wishes to grant one must sacrifice Wish Economy currency/magic items worth an amount equal to that of a CR 17 soul. A creature needs to be at least CR 17 itself to grant an Intercession Point, and it cannot grant one to itself. Usually, these are only paid to people you expect to continue working for you. 

\featname{Miracle}
\begin{small}
\shortability{Evocation}{}
\shortability{Level:}{Clr 9}
\shortability{Components:}{V, S; see text}
\shortability{Casting Time:}{1 Standard Action}
\shortability{Range:}{See text}
\shortability{Target, Effect, or Area:}{See text}
\shortability{Duration:}{See text}
\shortability{Saving Throw:}{See text}
\shortability{Spell Resistance:}{As emulated spell or effect}
\end{small}
\emph{``I tap all my White and Red mana, and everything takes infinity damage.\\
...Everything?\\
\emph{Everything}.''}

\desc{The caster requests a favor from their deity, or philosophy, or some other supernatural superbeing. This lets them do some pretty amazing stuff.

When cast, a miracle can accomplish one of the following:}

\listone
    \item Duplicate any cleric spell of up to 8th level, or spells to which the caster personally has access (i.e., from Domains, Attune Sphere, or something like that). This includes spells which might otherwise be disallowed due to alignment constraints. 
    \item Duplicate any other spell of up to 7th level. 
    \item Undo the harmful effects of any spell of up to 8th level. This only applies to direct effects of the spell. 
    \item Any effect which is in line with an 8th level spell. 
\end{list}

\desc{In addition, by spending an Intercession Point owed by a particular creature or philosophy, the caster may cause a large-scale effect which is associated in some way with that creature or philosophy. IP from a Balor, for example, might drive everyone within the area of effect insane, and IP from the concept of Death might hit everyone in the area with several negative levels (as per enervation).}

\desc{Upon choosing to spend IP on this effect, the caster chooses a spell of 7th level or lower to emulate which is thematically linked to the creature which granted the IP. This spell�s effect is then extended to an area centered on the caster, out to a 10-mile radius (which can be reduced if desired).}

\desc{If the spell has an Area field, the Area becomes a sphere centered on the caster within the chosen radius (the caster may exclude his own space if desired). If it has a Target(s) field, this encompasses all valid targets (with specific exceptions if desired) in that area. If it has an Effect which fills a particular area, the Effect is expanded to fill the new area. Spells without any of these qualities are not affected in any way.}

\desc{For example: if the creature in question were a Balor, the area could be a raging inferno (like a \emph{fireball} which was a 10-mile sphere), or drive everyone within the radius insane (as an \emph{insanity} targeted at each creature in the area), or something else that has to do with Balors.}