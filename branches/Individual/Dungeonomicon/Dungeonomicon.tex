\documentclass[10pt]{report}

\usepackage{appendix}
\usepackage{floatflt}
\usepackage{fancyhdr}
\usepackage{textcomp}
\usepackage[usenames]{color}
\usepackage{isoent}    %\sfrac
%\usepackage{palatino} %font
\usepackage{newcent}   %font
\usepackage{sectsty} %custom section headings

\newcommand{\normalsections}{
\sectionfont{\noindent\rule{\textwidth}{0.015in}\\\nohang}
\subsectionfont{\noindent\rule{\textwidth}{0.005in}\\\nohang}
}

\newcommand{\columnsections}{
\sectionfont{\vspace*{-20pt}\noindent\rule{3.5in}{0.015in}\\\nohang}
\subsectionfont{\vspace*{-20pt}\noindent\rule{3.5in}{0.005in}\\\nohang}
}


\sectionfont{\noindent\rule{\textwidth}{0.015in}\\\nohang}
\subsectionfont{\noindent\rule{3.5in}{0.005in}\\\nohang}

\usepackage[Bjarne]{fncychap} %Canned chapter headings


\usepackage{multicol}
\usepackage[bookmarks=true,colorlinks,linkcolor=cyan,breaklinks]{hyperref}



%%%%Margins%%%
\topmargin 0pt
\advance \topmargin by -\headheight
\advance \topmargin by -\headsep
\textheight 8.9in
\oddsidemargin -0.25in
\evensidemargin \oddsidemargin
\textwidth 7in
\oddsidemargin -0.25in
%\setlength {\parindent} {0pt}



%%%Formatting%%%
\newcommand{\ability}[2]{\smallskip \noindent \textbf{#1} #2}
\newcommand{\shortability}[2]{\noindent\textbf{#1} #2\\}
\newcommand{\bolded}[1]{\noindent\textbf{#1}}
\newcommand{\itemability}[2]{\item \textbf{#1} #2}
\newcommand{\featname}[1]{\vspace*{0.1cm plus 0.2cm minus 0.05cm}\noindent\textbf{#1}\\}
\newcommand{\featnamelist}[1]{\vspace*{0.1cm plus 0.2cm minus 0.05cm}\noindent\textbf{#1}}

\newcommand{\descfeat}[2]{\featname{#1}\emph{#2}\\}

\newcommand{\classname}[1]{\subsection{#1}}
\newcommand{\condition}[1]{\emph{#1}}
\newcommand{\quot}[1]{\emph{#1}\medskip}
\newcommand{\desc}[1]{#1 \medskip}
\newcommand{\example}[1]{\emph{#1}}
\newcommand{\magicitem}[1]{\emph{#1}}
\newcommand{\monster}[1]{\subsection{#1} \label{monster:#1}}
\newcommand{\monsterline}[2]{\textbf{#1:} #2\\}
\newcommand{\monstersizetype}[2]{\textbf{#1 #2}\\}
\newcommand{\spell}[1]{\emph{#1}}
\newcommand{\spelllist}[1]{\smallskip \noindent \underline{\textbf{#1}}}

% For the feats -- moved here from feats.tex because we added another 
% section(s) for feats. -Surgo
\newcommand{\minitabular}[1]{\begin{tabular}{p{0.25in}p{2.9in}} #1\\ \end{tabular}}
\newcommand{\babfeat}[7]{
	\noindent\minitabular{\multicolumn{2}{l}{\parbox{3in}{\textbf{#1}}}}
	\minitabular{\multicolumn{2}{l}{\parbox{3in}{\small #2}}}
	\minitabular{\raggedleft\textbf{\small \textbf{+0:}}& {\small #3}}
	\minitabular{\raggedleft\textbf{\small +1:} & {\small #4}}
	\minitabular{\raggedleft\textbf{\small +6:} & {\small #5}}
	\minitabular{\raggedleft\textbf{\small +11:} & {\small #6}}
	\minitabular{\raggedleft\textbf{\small +16:} &{\small #7}}
}
\newcommand{\skillfeat}[8]{
	\noindent\minitabular{\multicolumn{2}{l}{\parbox{3in}{\textbf{#1}}}}
	\minitabular{\multicolumn{2}{l}{\parbox{3in}{\small #2}}}
	\minitabular{\multicolumn{2}{l}{\parbox{3in}{\small\textbf{#3}}}}
	\minitabular{\raggedleft\textbf{\small \textbf{0:}} &{\small #4}}
	\minitabular{\raggedleft\textbf{\small 4:} &{\small #5}}
	\minitabular{\raggedleft\textbf{\small 9:} &{\small #6}}
	\minitabular{\raggedleft\textbf{\small 14:} &{\small #7}}
	\minitabular{\raggedleft\textbf{\small 19:} &{\small #8}}
}

\newcommand{\boxit}[1]{\frame{\parbox{\textwidth}{#1}}}

% A new box command -- the first argument is the box's header, the second the 
% contents of the box. This looks a lot better than \boxit. -Surgo
\newcommand{\abox}[2]{\vspace{5pt}
	\fbox{\begin{minipage}{0.8\linewidth}
	\setlength{\parindent}{0.15in}
	\textbf{#1}

	#2
\end{minipage}}
\vspace{5pt}}

\newcommand{\itemspace}{\setlength{\itemsep}{-1mm}\setlength{\topsep}{-1mm} }

\newcommand{\listone}{\begin{list}{$\bullet$}{\itemspace}}
\newcommand{\listprereq}{\begin{list}{\vspace*{-2pt}}{\itemspace}}
\newcommand{\listtwo}{\begin{list}{$\triangleright$}{\itemspace}}
\newcommand{\listthree}{\begin{list}{--}{\itemspace}}

% Bold the first argument in the listitem
\newcommand{\bolditem}[2]{\item \textbf{#1} #2}

\newcommand{\slashfrac}[2]{${}^{#1}$\hspace*{-1pt}\makebox[2pt]{$\diagup$}${}_{#2}$}
%\newcommand{\half}[0]{\slashfrac{1}{2}}
\newcommand{\half}[0]{\ensuremath{\sfrac{1}{2}} }
\newcommand{\third}[0]{\ensuremath{\sfrac{1}{3}} }
\newcommand{\fourth}[0]{\ensuremath{\sfrac{1}{4}} }
%\newcommand{\half}[0]{\ensuremath{\scriptscriptstyle 1/2}}
%\newcommand{\threefourths}[0]{\ensuremath{\sfrac{3}{4}}}


%\rule{\textwidth}{0.005in}

%\renewcommand{\sectionmark}[1]{\markright{\thesection\ \boldmath\emph{#1}\unboldmath}\\\rule{\textwidth}{0.005in}\\}

\setcounter{tocdepth}{1}

\begin{document}

\pagestyle{plain}
%Cover Page

\begin{center} \Huge

\textsc{Races of War}\end{center}



\vspace{2cm}
\begin{center}\large By Frank Trollman \& K\end{center}


\newpage

\vspace*{4in}

\noindent Please address all complaints and comments about balance to the authors at\\
{\color{blue} \href{http://tgdmb.com/viewforum.php?f=1}{http://tgdmb.com/viewforum.php?f=1}}

\vspace{0.2in}

%\noindent For a hypertext version of some of this information, especially classes, please look at\\
%{\color{blue} \href{http://www.d20ragon.com/frank/}{http://www.d20ragon.com/frank/}}

%\vspace{0.2in}

\noindent Amateur Typesetting by Joshua Middendorf, updated by Morgon ``Surgo'' Kanter, ``Aktariel'', and Stephen ``Quantumboost'' Smith.

\vspace{0.15in}

\noindent Please address all comments regarding the quality (or lack thereof) of the typesetting (that is, formatting of the pdf) to Joshua Middendorf (\href{mailto:middendorfproject@gmail.com}{middendorfproject@gmail.com}), Morgon Kanter (\href{mailto:morgon.kanter@gmail.com}{morgon.kanter@gmail.com}), Aktariel (\href{mailto:aktariel@gmail.com}{aktariel@gmail.com}), or simply comment in the above forum.


%\vspace{0.2in}

%\emph{Enjoy!}


\vspace{1in}
\noindent Published on \today, version 0.6\\
\noindent You may find the most recent version of this document at:\\
{\color{blue} \href{http://www.tgdmb.com/viewtopic.php?t=36046}{http://www.tgdmb.com/viewtopic.php?t=36046}}

\newpage


\pagestyle{fancy}
%Fix the spacing so it's all reasonable
\linespread{.9}  \small  \normalsize \itemspace \normalsections

\tableofcontents


\section{The Socialomicon} %: Heroes in the Greek Sense
\vspace*{-10pt}
\quot{''Can I kill the baby kobolds?"}

When people are asked to name a historical point that D\&D most closely represents, they'll usually say something like ''The Middle Ages," or perhaps a date between 1000 AD and 1500 in Europe. Truth be told, to find a historical period which has a social setup anything like D\&D, you're going to have to go back. Way back. D\&D represents a period in history that is most closely identifiable with the Iron Age: the landscape is dotted with tribes and aspiring empires, the wilderness is largely unexplored, and powerful individuals and small groups can take over an area without having a big geopolitical hubbub about it.

The source material for the social setting of D\&D is not Hans Christian Andersen, it's Homer's The Iliad and Caesar's The Gallic Wars. In the backdrop of early historical empire building, crimes that modern humans shake their heads at the barbarity of are common place -- even among the heroes. D\&D at its core is about breaking into other peoples' homes, possibly killing the residents, and taking their stuff home with you in a sack. And in the context of the period, that is acceptable behavior for a hero.

\subsection{Living With Yourself After a Raid}

The goblins have gone and conducted a raid on your village in full force. They rode in, took a bunch of the sheep, killed some of the people, set fire to some of the cottages, and rode away again with Santa Sacks filled with this year's crop. And they laughed because they thought it was funny. And now that your elder brother has been slain you want to dedicate yourself to the eradication of the Goblin Menace and begin the training necessary to become a Ranger so that you can empty the goblin village from the other side of the valley once and for all.

Par for the course D\&D, right? Wrong! Killing all the goblins isn't just an Evil act, it's unthinkable to most D\&D inhabitants. This is the Classical Era, and actually sowing the fields of Carthage with salt is an atrocity of such magnitude that people will speak of it for thousands of years. In the D\&D world, goblins raid human settlements with raiding parties, humans raid goblin settlements with ''adventuring parties", and like the cattle raiding culture of Scotland, it's simply accepted by all participants as a fact of life.

When your city is raided by other groups of humanoids, it's a bad thing for your city. Orcs may kidnap some of your relatives and use them as slaves (or food), and many of your fellow villagers may lose their lives defending lives and property important to them. But that's part of life in the age, and people just sort of expect that sort of thing.

\subsection{Razing Hell: When Genocide is the Answer}

Sometimes in history there would come a great villain who just didn't get with the program. The Classical example is the Assyrians. Those bastards went around from city to city stacking heads in piles and levying 100\% taxation and such to conquered foes. They became\ldots\  unpopular, and eventually were destroyed as a people. That's the law of the jungle as far back as there are any records: if a group pushes things too far the rules of mercy and raiding simply stop applying. Goblins, orcs, sahuagin\ldots\  these guys generally aren't going to cross that line. But if they do, it's OK for the gloves to come off. In fact, if some group of orcs decides to kill everyone in your village while you're out hunting so that you come home to find that you are the last survivor, other humanoids (even other Evil humanoids like gnolls) will sign up to exterminate the tribe that has crossed the line.

Cultural relativism goes pretty far in D\&D. Acceptable cultural practices include some pretty over-the-top practices such as slavery, cannibalism, and human sacrifice. But genocide is still right out. That being said, some creatures simply haven't gotten with the program, and they are kill-on-sight anywhere in the civilized world or in the tribes of savage humanoids. Mindflayers, Kuo-Toans, and [Monster] simply do not play the same game that everyone else is playing, mostly because their culture simply does not understand other races as having value. And that means that even other Evil races want to exterminate those peoples as a public service. Like the Assyrians, they've simply pushed their luck too far, and the local hobgoblin king will let you marry his daughter if you help wipe them out of an area.

Solitary intelligent monsters often get into the same boat as the Kuo-Toans. Since the Roper really has no society (and possibly the most obscure language in Core D\&D), it's very difficult for it to understand the possible ramifications of offending pan-humanoid society. So now they've done it, and they really haven't noticed the fallout they are receiving from that decision. Ropers pretty much attack anything they see, and now everyone that sees a roper attacks them. In the D\&D worlds, ropers are on the brink of extinction and it probably never even occurs to them that their heavy tendrilled dealings with the other races have pushed them to this state.

\subsection{The Hands of the Divine}

In D\&D the gods are, compared to the wet radish that is your character, unlimited in power. There is no amount of whupass that you could put together that would allow you to triumph over Vecna -- he can cast any (or every) spell as a free action. He can cast ''Kill Drogor the Dwarf Barbarian with no Save", a spell which heretofore had not even been researched by anyone -- as a free action. And he knows many days in advance when he is going to be in danger and who he's going to be in danger from, so that's really not a battle you're going to win. Nevertheless, when adventurers come across a temple to Vecna, they kick everything over, they smash the idol and they steal its ruby eye. And they get away with it.

And that's because when you kick over temple to Vecna, you aren't going against Vecna in any direct sense. Vecna lives on a distant outer plane and has full control over anything that happens in his personal dominion. Anywhere else, and he's essentially playing a game of Populous. If there isn't a knight or prophet of Vecna around, Vecna really can't do much until the ''end-game" scenario in which he starts throwing volcanoes around. And as soon as that starts happening, the best bet is really to try to hop on the first portal out of whatever universe you happen to be in because it's going Armageddon pretty soon. Vecna might encourage some monsters to go look you up, or lower some land in your way, but you're an adventurer -- so that's pretty much what you expected out of life anyway.

\subsection{Temporal Authority in D\&D}
\vspace*{-8pt}
\quot{''Kill the dragon, marry the princess, rule the kingdom."}

D\&D is set in an essentially Iron Age setting. If your group (or even you personally) are known to be hardcore enough, you actually do rule the lands extending as far as you can reach. This doesn't mean that you don't need a bureaucracy, because there's still relatively little that you can do on your own. That administrative staff is necessary, it's there to tell people what you want them to do, and to tell you when they aren't doing it.

In fairy tales, as well as D\&D, the guy (or girl) who saves the kingdom by slaying the big monster marries the child of the local king. This is usually because the current king is himself a powerful dude with a PC class himself. His children may be aristocrats, and by marrying them off to a powerful adventurer who may well be able to take his kingdom by force, he's preserved his own position and kept his family from being set on fire. Nominally in this situation the crown is still in the previous king's family and moving to the next generation normally. You may even get a title like ''Prince Consort" or something -- but everyone knows that you are running the show because you can slay dragons. No one is going to say it, but the princess' only real job in this scenario is to\ldots\  keep you happy. And she's not even the only one that has that job. Surprisingly, the previous king is actually fine with that, because if his daughter has Aristocrat levels, that really is the best he can expect for her.

\subsubsection{The Basis for Hereditary Rule in D\&D}
\vspace*{-8pt}
\quot{''Why would I listen to an Aristocrat? I'm a frickin' Wizard. I can set his whole house on fire with my mind!"}

If being a badass makes you rule the kingdom, and it does, where do aristocrats come from? How could you possibly have anyone in command of anything whose job is to be pretty and otherwise useless? This question comes up in virtually every campaign when the players get to be about 9th level and notice that they can go a whuppin' and a whompin' on the local lords with impunity. The key is not that aristocrats have some sort of shiny pants powers that force people to take their rights of inheritance seriously despite inherent power discrepancy. The secret is the Apprentice and Mentor feats from the DMG2.

Here's the deal: when you have the Mentor Feat, you can teach children how to have whatever class you have, and they gain levels without actually doing anything dangerous themselves. This means that when you're a powerful adventurer and you want to pass on your legacy to your children, you can. They get to majority and they're already a 5th level character (and they can use all the sweet adventuring gear you've accumulated as a powerful adventurer). The son of the man who conquered the Bane Mires is also very hardcore, so having him announce that he should rule when his father passes on is pretty reasonable. Like real human history, this won't always work. Historically, the hand-off of kingship has resulted in open warfare almost half the time. In D\&D, prospective kings actually do have some inherent greatness so monarchies are way more stable in D\&D than they were in the real world -- but there will still be occasional guys who refuse to accept the new kid and try to take things over with their own spells or swords.

\subsubsection{The Place of Aristocrats}

Especially if the local hard-core lord is particularly fecund or has a short attention span, there is ample possibility that there will be more children and nieces and nephews and such in the next generation than the lords with real character classes can train to be hard core. Some lords don't even have the mentorship feat and can't pass on their awesomeness to any part of the next generation. When this happens, there will be one or more children brought up in a courtly manner who have no useful adventuring skills at all -- these unlucky creatures have levels in Aristocrat. They keep their wealth because they are related to people who can kill an elephant with a tea spoon. Their competent family members keep them around because they can be trusted to administer things fairly well on their behalf.

An Aristocrat's primary job in life is to get married to another aristocrat or better yet a hard core adventurer. In this manner they will get more family members who can slay dragons. That's important, because as soon as none of the living members of a house are powerful wizards or warriors -- the house gets its assets liquidated by an adventuring group of orcs or elves (depending upon what kind of house it was) and all the aristocrats in it who weren't killed outright have to run off into the night with whatever wealth they can hide inside their body. It's not pretty, so aristocrats spend a considerable amount of time trying to make themselves as pretty as possible -- anything they can do to make a high level Barbarian want to have children with them is something they'll do without a second thought.


\chapter{Charactonomicon}

\quot{``Honestly, I sort of expected my character to be more awesome than this.''}

Many character archetypes don't fit neatly into the pseudo-European-battlefield that D\&D is often concepted as, but do fit squarely into the genre that Dungeons and Dragons has become. Most notably, the unarmored warriors, swashbucklers, and talkative heroes that are perpetuated in fantasy literature. They do have a place in the D\&D setting, but they really haven't been done well in the D\&D rules. This makes us very sad. What follows is a re-imagining of several classes that have been with us for over 30 years, but which don't have functional game mechanics in the modern era. Some of them (like the Jester) have never had functional game mechanics, while others (like the Monk) have been playable characters at various times but aren't now.

Our goal here is to make these archetypes playable in low-level environments and high level environments without the implementation of Easter Egg class features like DM pity. A monk is an integral portion of D\&D, and there should be a way to participate in combats against monsters of your level other than the current standard:

\begin{enumerate}
\item Massively underperform against enemies of your level.
\item Wait for the DM to recognize this as a systemic problem and throw in campaign specific treasure that will make your character awesome.
\item Use that awesome equipment to pull your weight with the other party members.
\end{enumerate}
\vspace{20pt}

\textbf{Why no Open Lock skills?} Open Lock is a legacy skill that makes no sense. In previous editions of D\&D, a Thief had an ``Open Locks'' skill and a ``Find/Remove Traps'' skill. In 3rd edition. Find/Remove Traps got split into Search and Disable Device. Disable Device is actually capable of bypassing any device or spell-based impediment, not just Traps these days. Heck, it even has bypassing a Lock as an example task! There's a reason that other D20 games have dropped Open Locks altogether, and we strongly support that decision. In that spirit, we've dropped the Open Lock skill from all classes in the Dungeonomicon, and suggest that you allow players to use their Dexterity Modifier in place of their Intelligence Modifier for Disable Device if they want to.

\section{Base Classes}

\classname{Monk} \label{class:monk}
\vspace{-8pt}
\quot{"I am a Grand Master of Flowers. You are not."}

\desc{Fantasy literature's view of the "martial artist" has about as much to do with a real martial artist as its view of salamanders has to do with real salamanders. But let's face the facts: Monks are totally sweet. They flip out and kill people with their hands. A Monk does not practice any "real" martial art; we call those people "Fighters" -- a Monk practices an entirely magical martial art that only works in universes where badgers can talk and winged horses can fly.}

\desc{Every Monk follows a different martial path that involves jumping super high and having glowing things coming off of their hands when they perform their super moves. Some monks use weapons, but most just use their hands and feet to devastating effect. Some Monks shout the names of their techniques in battle to demoralize their opponents, others stay aloof and silent during even the toughest of challenges.}

\ability{Alignment:}{Monks may be of any alignment. Really. If a bar brawl breaks out, some Monks will try to break it up, other Monks will join in. Whatever.}

\ability{Races:}{Because the martial paths of a Monk embrace all manners of comportment, from Stoic Lawfulness to Boisterous Chaos, almost every sapient race has those who take up the monk's path. With its lack of emphasis on ranged weaponry, few of the slower races turn towards these magical combat styles, and halflings and dwarves rarely become monks. The discipline emphasizes physical strength as much as it emphasizes perceptiveness and inner strength, so orcs are as likely to become monks as Kuo-Toa are.}

\ability{Starting Gold:}{2d4x10 gp (50 gold)}

\ability{Starting Age:}{As Monk.}

\ability{Hit Die:}{d8}

\ability{Class Skills:}{The Monk's class skills (and the key ability for each skill) are Balance (Dex), Climb (Str), Concentration (Con), Craft (Int), Diplomacy (Cha), Escape Artist (Dex), Hide (Dex), Jump (Str), Knowledge (all skills individually) (Int), Listen (Wis), Move Silently (Dex), Perform (Cha), Profession (Wis), Sense Motive (Wis), Spot (Wis), Swim (Str), and Tumble (Dex).}

\ability{Skills/Level:}{4 + Intelligence Bonus}

\begin{table}[tbh]
\begin{small}
\begin{tabular}{lp{3cm}p{0.7cm}p{0.7cm}p{0.7cm}p{6.5cm}l}
Level  &Base Attack Bonus &Fort Save &Ref Save &Will Save &Special &AC Bonus\\
1st &+1 &+2 &+2 &+2 &Armored in Life, Fatal Strike, Willow Step, Fighting Style &+4\\
2nd &+2 &+3 &+3 &+3 &Rain of Flowers, Abundant Leap &+5\\
3rd &+3 &+3 &+3 &+3 &Fighting Style &+5\\
4th &+4 &+4 &+4 &+4 &Diamond Soul &+6\\
5th &+5 &+4 &+4 &+4 &Fighting Style &+6\\
6th &+6/+1 &+5 &+5 &+5 &Walk of a Thousand Steps &+7\\
7th &+7/+2 &+5 &+5 &+5 &Fighting Style &+7\\
8th &+8/+3 &+6 &+6 &+6 &Immaculate Diamond Soul &+8\\
9th &+9/+4 &+6 &+6 &+6 &Master Fighting Style &+8\\
10th &+10/+5 &+7 &+7 &+7 &Leap of the Clouds &+9\\
11th &+11/+6/+6 &+7 &+7 &+7 &Master Fighting Style &+9\\
12th &+12/+7/+7 &+8 &+8 &+8 &Master of the Four Winds &+10\\
13th &+13/+8/+8 &+8 &+8 &+8 &Master Fighting Style &+10\\
14th &+14/+9/+9 &+9 &+9 &+9 &Master of the Four Seasons &+11\\
15th &+15/+10/+10 &+9 &+9 &+9 &Grand Master Fighting Style &+11\\
16th &+16/+11/+11/+11 &+10 &+10 &+10 &Master of Diamond Soul &+12\\
17th &+17/+12/+12/+12 &+10 &+10 &+10 &Grand Master Fighting Style &+12\\
18th &+18/+13/+13/+13 &+11 &+11 &+11 &Perfect Mastery &+13\\
19th &+19/+14/+14/+14 &+11 &+11 &+11 &Grand Master Fighting Style &+13\\
20th &+20/+15/+15/+15 &+12 &+12 &+12 &Grand Master of Flowers &+14\\
\end{tabular}
\end{small}
\end{table}

\smallskip\noindent All of the following are Class Features of the Monk class.

\ability{Weapon and Armor Proficiency:}{Monks are proficient with all simple weapons, as well any weapon defined as a special monk weapon, such as the sai, the nunchuka, the kama, the shuriken, and the triple staff. Monks are not proficient with any armor or shields of any kind.}

\ability{Armored in Life (Su):}{A Monk has a special Armor bonus whenever they are not using armor or shields that he is not proficient in. This Armor Bonus applies against Touch Attacks and Incorporeal Touch Attacks, and has a value of +4. Every even numbered class level, the Armored in Life bonus increases by 1. If the Monk wears armor which he is proficient in (for example: normal clothing) that has an enhancement bonus, that enhancement bonus applies to his Armored in Life Armor Bonus.}

\ability{Wilow Step (Su):}{A true monk does not seek to outrun the fist, but to anticipate it. If a Monk would be allowed to add his Dexterity modifier to a Reflex Save or Armor Class, he may add his Wisdom bonus (if positive) instead.}

\ability{Fatal Strike (Su):}{A Monk has a natural weapon Slam in addition to whatever else he is capable of doing. As a natural slam attack, if he uses no other natural or manufactured weapons he adds his Strength and a half to damage and may make iterative attacks if he has sufficient BAB. If the slam is used with other weaponry, it becomes a secondary natural attack, suffers a -5 penalty to-hit, and adds only half his Strength modifier to damage. A monk's slam attack does a base of 1d8 damage for a medium sized monk and does more or less damage as appropriate if the Monk is larger or smaller than medium size.}

\ability{Fighting Style (Su):}{At levels 1, 3, 5, and 7, the Monk learns a Fighting Style. Each Fighting style requires a Swift Action to activate, lasts one round, and is usable at will. Each Fighting Style must have a name (see Naming Your Fighting Style below), and provides two bonuses from the Fighting Style Abilities:}

\begin{itemize}\itemspace\begin{small}
    \item{While Active, your Fighting Style provides a +4 Dodge Bonus to AC.}
    \item{While Active, your Fighting Style provides a +4 Dodge Bonus to Saving Throws.}
    \item{While Active, your Fighting Style forces any opponent struck by your slam attack to make a Fortitude Save (DC 10 + \half\  your character level + your Wisdom Modifier) or become stunned for one round.}
    \item{While Active, your Fighting Style allows you to make an attack of opportunity against any opponent who attacks you. This attack of opportunity must be a trip or disarm attempt.}
    \item{While Active, your Fighting Style provides you with concealment.}
    \item{While Active, your Fighting Style provides a +30' Insight Bonus to your movement rate.}
    \item{While Active, your Fighting Style allows your slam attacks to ignore hardness and DR.}
    \item{While Active, your Fighting Style provides any bonuses it gives to your slam attack to any attack you make with any weapon.}
    \item{While Active, your Fighting Style causes your slam attack to inflict piercing damage and to inflict 2 points of Constitution damage.}
    \item{While Active, your Fighting Style causes your slam attack to inflict slashing damage and to reduce your opponent's movement rate by 10' every time they suffer damage from it. This movement rate reduction can be healed like ability damage (treating 5' of movement as 1 point of ability damage).}
    \item{While Active, your Fighting Style allows you to move through occupied spaces as if they were unoccupied and you provoke no attacks of opportunity for your movement.}
\end{small}\end{itemize}

\ability{Rain of Flowers (Su):}{Any time a 2nd level Monk inflicts lethal damage, he may elect to inflict non-lethal damage instead. Any time a Monk inflicts non-lethal damage, he may elect to inflict lethal damage instead.}

\ability{Abundant Leap (Su):}{At 2nd level, a Monk's ability to jump is unbounded by his height. In addition, the DC for any jump check is divided by two.}

\ability{Diamond Soul (Su):}{At 4th level, the Monk gains Spell Resistance equal to 5 + his character level. At 8th level, his soul becomes immaculate and his Spell Resistance improves to 10 + character level, and at 16th level he masters his diamond soul and his spell resistance improves to 15 + character level.}

\ability{Walk of a Thousand Steps:}{Once per day, a Monk of sixth level or higher may activate a Fighting Style and extend its duration to 1 round/level rather than 1 round. Activating this Fighting Style is still a Swift Action. Other Fighting Styles may be activated during this period, though their duration is normally going to be only 1 round.}

\ability{Master Fighting Style (Su):}{At levels 9, 11, and 13, the Monk learns a Master Fighting Style. Each Master Fighting style requires a Swift Action to activate, lasts one round, and is usable at will. Each Master Fighting Style must have a name (see Naming Your Fighting Style below), and provides two bonuses from the Master Fighting Style Abilities. When a Monk gains a new Master Fighting Style, he may replace one of his Fighting Styles with a different Fighting Style.}



\begin{itemize}\itemspace\begin{small}
    \item{While Active, your Master Fighting Style allows you to \spell{teleport} yourself and everything you are physically carrying 60 feet in any direction as a free action usable once per round.}
    \item{While Active, your Master Fighting Style provides total concealment.}
    \item{While Active, your Master Fighting Style transforms your slam attacks into Force effects that inflict Force damage.}
    \item{While Active, your Master Fighting Style affects any creature struck with your slam attack with a \spell{banishment} effect that transports it back to its home plane unless it succeeds at a Will save (DC 10 + \half\  character level + Wisdom Modifier). Outsiders suffer a -4 penalty to their saving throw. A creature so banished, may not return to the plane it was banished from for a year.}
    \item{While Active, your Master Fighting Style forces any creature struck by your slam attack to make a Reflex Save (DC 10 + \half\  character level + Wisdom Modifier) or be helpless for one round.}
    \item{While Active, your Master Fighting Style provides you the effect of an \spell{air walk} spell, and gives you a +20' Competence bonus to your speed.}
    \item{While Active, your Master Fighting Style affects any opponent you successfully trip or bulrush with the violent thrust version of \spell{telekinesis}, with a caster level equal to your character level. There is no saving throw against this effect.}
    \item{While Active, your Master Fighting Style allows you to shoot fire out of your hands or mouth as a standard action. The fire can be shot out to medium range, requires a ranged touch attack, and inflicts 1d6 of fire damage per character level if it hits.}
    \item{While Active, your Master Fighting Style causes your slam attack to inflict vile damage.}
    \item{While Active, your Master Fighting Style forces every creature within 10 feet of you to make a Will save (DC 10 + \half\  character level + Wisdom Modifier) or become panicked for one minute.}
    \item{While Active, your Master Fighting Style affects any target you strike with your slam attack with a targeted version \spell{greater dispelling} with a caster level equal to your character level.}
    \item{While Active, your Master Fighting Style causes 5d6 of Sonic damage to everything within 30 feet of you when you inflict damage with your slam attack against any target. You are immune to Sonic damage while your Master Fighting Style is active.}
    \item{Instead of gaining a Master Fighting Style Ability, you may choose two regular Fighting Style Abilties.}
\end{small}\end{itemize}




\ability{Leap of the Clouds (Su):}{At 10th level, the DC for any jump check is divided by 5.}

\ability{Master of the Four Winds (Su):}{The Monk's breath of life is carried on the winds of fate. At 12th level, if the monk is restored to life, he doesn't lose a level for doing so.}

\ability{Master of the Four Seasons:}{Time passes relentlessly in the world, but for a monk of 14th level, the change of seasons is as no change at all. He no longer appears to age, never accumulates any additional penalties for growing older and will never die of old age.}

\ability{Grand Master Fighting Style (Su):}{At levels 15, 17, and 19, the Monk learns a Grand Master Fighting Style. Each Grand Master Fighting style requires a Swift Action to activate, lasts one round, and is usable at will. Each Grand Master Fighting Style must have a name (see Naming Your Fighting Style below), and provides two bonuses from the Grand Master Fighting Style Abilities list. When a Monk gains a new Grand Master Fighting Style, he may replace one of his Fighting Styles or Master Fighting Style with a different Style of the same type.

	Grand Master Fighting Style Abilities:}

\begin{itemize}\itemspace \begin{small}
    \item{While Active, your Grand Master Fighting Style makes you and everything you are carrying incorporeal, and your slam attacks are incorporeal touch attacks.}
    \item{While Active, your Grand Master Fighting Style slows down time to the point where you can act twice each round. You do not gain an extra Swift Action during your extra actions.}
    \item{While Active, your Grand Master Fighting Style allows you to punch a hole through space and time, allowing you to open a travel version of \spell{gate} with a slam attack.}
    \item{While Active, your Grand Master Fighting Style prevents all [Teleport] effects from entering or exiting within 1 mile of your location.}
    \item{While Active, your Grand Master Fighting Style causes your slam attacks to reduce the spell resistance of enemies by an equal amount to the damage the slam attack inflicts.}
    \item{While Active, your Grand Master Fighting Style forces every creature struck with your slam attack to make a Fortitude save (DC 10 + \half\  character level + Wisdom Modifier) or die.}
    \item{While Active, your Grand Master Fighting Style affects any target you strike with your slam attack with a \spell{disintegrate} effect, with a caster level equal to your character level (DC 10 + \half\  character level + Wisdom Modifier).}
    \item{While Active, your Grand Master Fighting Style causes you to regenerate. You recover a number of points of nonlethal damage each round equal to your character level. Unarmed or Slam attacks inflict regular damage.}
    \item{While Active, your Grand Master Fighting Style forces any opponent you strike with your slam attack to make a Willpower save (DC 10 + \half\  character level + Wisdom Modifier) or become \spell{feebleminded}.}
    \item{While Active, your Grand Master Fighting Style affects every target you strike with a slam attack with the violent thrust version of \spell{telekinesis}, with a caster level equal to your character level. There is no saving throw against this effect.}
    \item{Instead of gaining a Grand Master Fighting Style Ability, you may choose two Master Fighting Style Abilties.}
\end{small}\end{itemize}

\begin{table}[tbh]
\begin{small}
\begin{center}
\noindent \begin{tabular}{|ll||ll||ll|}
\multicolumn{6}{l}{Naming your Fighting Styles: Roll a d20, or choose}\\
\multicolumn{6}{l}{an adjective, an animal, and a noun:} \\
\hline 1&Running&1&Ox&1&Fist\\
2&Hungry&2&Tiger&2&Stance\\
3&Angry&3&Dragon&3&Spinning Kick\\
4&Naked&4&Crane&4&Attack\\
5&Drunken&5&Monkey&5&Technique\\
6&Fortunate&6&Turtle&6&Style\\
7&Lazy&7&Manticore&7&Dance\\
8&Swift&8&Serpent&8&Movement\\
9&Powerful&9&Hummingbird&9&Touch\\
10&Enlightened&10&Demon&10&Fu\\ \hline
\multicolumn{6}{l}{Note from the authors: Feel free to add any adjectives, animals,}\\
\multicolumn{6}{l}{or nouns that you want.  There's no reason that your character's}\\
\multicolumn{6}{l}{fighting style has to be called "Naked Tiger Stance" rather than}\\
\multicolumn{6}{l}{"Astonished Centaur Defense".}\\
\end{tabular}
\end{center}
\end{small}
\end{table}


\ability{Perfect Mastery:}{Once per day, a Monk of 18th level or higher may activate a Fighting Style, Master Fighting Style, or Grand Master Fighting Style and extend its duration to 1 round/level rather than 1 round. Activating this style is still a Swift Action. Other styles may be activated during this period, though their duration is normally going to be only 1 round.}

\ability{Grand Master of Flowers:}{At 20th level, the Monk becomes an Outsider, and immortal of legend. He gains the augmented subtype of his previous type, and has Damage Reduction of 20/Epic.}


\classname{Jester} \label{class:jester}
\vspace{-8pt}
\quot{"Well no, but if I was doing it to anyone else, you'd think it was funny."}

\desc{To be a Jester is to see the joke in every tragedy. For them, life's a party, and most poor bastards are not invited. They live hard, play hard, and laugh hard knowing that at any moment their life might be cut short by an uncaring world. Jesters may play at being buffoons, but each is a student of life and of people, and they understand not only what makes people laugh, but what makes them cry. As adventurers, they often appreciate baubles and magical trinkets as much as anyone else, but their main goal is to have fun. When fighting enemies, their sense of humor takes a macabre and dark turn, becoming cruel and vicious to better demoralize their foe. As followers of the Laughing God Who Has No Temples, they are generally disrespectful atheists who wander the world looking for excitement and amusement, righting wrongs or committing crimes as the mood takes them.}

\ability{Alignment:}{A Jester may be of any non-Lawful alignment.}

\ability{Races:}{Jesters appear in all cultures and all races have need of buffoons.}

\ability{Starting Gold:}{6d4x10 gp (150 gold)}

\ability{Starting Age:}{As Rogue.}

\ability{Hit Die:}{d6}

\ability{Class Skills:}{The Jester's skills (and the key ability for each skill) are Balance (Dex), Bluff (Cha), Climb (Str), Concentration (Con), Craft (Int), Diplomacy (Cha), Disable Device (Int), Disguise (Cha), Gather Information (Cha), Hide (Dex), Intimidate (Cha), Jump (Str), Listen (Wis), Move Silently (Dex), Perform (Cha), Profession (Wis), Search (Int), Sense Motive (Wis), Sleight of Hand (Dex), Spellcraft (Int), Spot (Wis), Swim (Str), Tumble (Dex), and Use Magic Device (Cha).}

\ability{Skills/Level:}{6 + Intelligence Bonus}

\begin{table}[tbh]
\begin{small}
\begin{tabular}{lp{2cm}p{0.7cm}p{0.7cm}p{0.7cm}p{5cm}lllllll}
Level  &Base Attack Bonus &Fort Save &Ref Save &Will Save &Special &\multicolumn{7}{c}{Spells Per Day}\\
&&&&&&0 &1 &2 &3 &4 &5 &6\\
1st &+0 &+0 &+2 &+0 &Harlequin's Mask, Ignore Components, Poison Use &2 &- &- &- &- &- &-\\
2nd &+1 &+0 &+3 &+0 &Laugh It Off &3 &0 &- &- &- &- &-\\
3rd &+2 &+1 &+3 &+1 &+1d6 Sneak Attack, Power Slide &3 &1 &- &- &- &- &-\\
4th &+3 &+1 &+4 &+1 &Jester's Fient &3 &2 &0 &- &- &- &-\\
5th &+3 &+1 &+4 &+1 &Cruel Comment &3 &3 &1 &- &- &- &-\\
6th &+4 &+2 &+5 &+2 &+2d6 sneak Attack &3 &3 &2 &- &- &- &-\\
7th &+5 &+2 &+5 &+2 &Sight Gag &3 &3 &2 &0 &- &- &-\\
8th &+6/+1 &+2 &+6 &+2 &Low Comedy, Slapstick &3 &3 &3 &1 &- &- &-\\
9th &+6/+1 &+3 &+6 &+3 &+3d6 sneak Attack &3 &3 &3 &2 &- &- &-\\
10th &+7/+2 &+3 &+7 &+3 &Jack-in-the-Box King &3 &3 &3 &2 &0 &- &-\\
11th &+8/+3 &+3 &+7 &+3 &+4d6 sneak Attack &3 &3 &3 &3 &1 &- &-\\
12th &+8/+3 &+4 &+8 &+4 &Killer Clown &3 &3 &3 &3 &2 &- &-\\
13th &+9/+4 &+4 &+8 &+4 &+5d6 sneak Attack &3 &3 &3 &3 &2 &0 &-\\
14th &+10/+5 &+4 &+9 &+4 &Annoy the Gods &4 &3 &3 &3 &3 &1 &-\\
15th &+11/+6/+6 &+5 &+9 &+5 &+6d6 sneak Attack &4 &4 &3 &3 &3 &2 &-\\
16th &+12/+7/+7 &+5 &+10 &+5 &Prat Fall &4 &4 &4 &3 &3 &2 &0\\
17th &+12/+7/+7 &+5 &+10 &+5 &+7d6 sneak Attack &4 &4 &4 &4 &3 &3 &1\\
18th &+12/+7/+7 &+5 &+10 &+5 &Last Trick &4 &4 &4 &4 &4 &3 &2\\
19th &+14/+9/+9 &+6 &+11 &+6 &+8d6 sneak Attack &4 &4 &4 &4 &4 &4 &3\\
20th &+15/+10/+10 &+6 &+12 &+6 &Eternal Trickster &4 &4 &4 &4 &4 &4 &4\\

\end{tabular}
\end{small}
\end{table}

\smallskip\noindent All of the following are Class Features of the Jester class.

\ability{Weapon and Armor Proficiency:}{Jesters are proficient with light armor but not with shields of any kind. A Jester is proficient with no weapons, but suffers no attack penalty for using a weapon with which they are not proficient or which is made for a character of a different size than themselves. Even, perhaps especially, improvised weapons may be used without the usual -4 penalty.}

\ability{Spellcasting:}{The Jester is an Arcane Spellcaster with the same spells per day progression as a Bard. A Jester casts spells from the Jester Spell List (below). A Jester automatically knows every spell on his spell list. He can cast any spell he knows without preparing them ahead of time, provided that spell slots of an appropriate level are still available. To cast a Jester spell, he must have a Charisma at least equal to 10 + the Spell level. The DC of the Jester's spells is Charisma based and the bonus spells are Charisma based.}

\begin{floatingfigure}{2.6in}
\small
\spelllist{Jester Spells:}

\ability{0th Level:}{\emph{Alarm, Detect Magic, Detect Poison, Grease, Unseen Servant, Ventriloquism.}}

\ability{1st Level:}{\emph{Fire Trap, Glitterdust, Magic Mouth, Misdirection, Pyrotechnics, Reduce Person, Sleet Storm, Tasha's Uncontrollable Hideous Laughter, Teleport Trap, Touch of Idiocy.}}

\ability{2nd Level:}{\emph{Baleful Transposition, Explosive Runes, Glyph of Warding, Rage, Rope Trick, Secret Page, Sepia Snake Sigil, Unluck.}}

\ability{3rd Level:}{\emph{Feeblemind, Minor Globe of Invulnerability, Modify Memory, Mordenkainen's Faithful Hound, Nightmare, Servant Horde, Shrink Item.}}

\ability{4th Level:}{\emph{Globe of Invulnerability, Greater Glyph of Warding, Insect Plague, Persistent Image, Sword of Deception, Symbol of Weakness, Tree Shape, Wood Rot.}}

\ability{5th Level:}{\emph{Bigby's Interposing Hand, Energy Immunity, Eyebite, Repulsion, Screen, Symbol of Insanity, Telekinesis.}}

\ability{6th Level:}{\emph{Creeping Doom, Insanity, Refuge, Symbol of Sleep, Symbol of Stunning, Temporal Stasis.}}\\

\normalsize
\end{floatingfigure}


\ability{Poison Use (Ex):}{A Jester may prepare, apply, and use poison without any chance of poisoning himself.}

\ability{Ignore Components:}{A Jester may cast spells from the Jester list without using material components, regardless of whether they are costly or not. This has no effect on any spells that a Jester casts from any other spell-list.}

\ability{Harlequin's Mask (Ex):}{As long as a Jester's face is painted, masked, or adorned in the manner of a harlequin or other comedic figure, he is immune to compulsion effects.}

\ability{Laugh It Off (Ex):}{Fate protects fools and little children, and Jesters certainly adopt the role of fools. At 2nd level, a Jester may add his Charisma modifier as a morale bonus to his saves.}

\ability{Power Slide (Ex):}{If a 3rd level Jester takes damage from an attack, he may allow himself to be flung backwards, thereby lessening the impact. He may make a Balance check with a DC equal to the damage inflicted and if he succeeds, he suffers only half damage. This is a skill check, not a Saving Throw, so abilities such as Evasion do not apply. He is moved away from the source of damage by 5' for every 5 points of damage (or part there of) negated in this way. If there is not enough space for him to move, he suffers a d6 of damage for each square not moved. If he passes through an occupied square, the Jester would have to make a tumble check to avoid attacks of opportunity. \smallskip

If this ability is gained from another class, then the Jester may choose to increase or decrease the total distance moved by 50\% (so a Power Slide that negated 12 points of damage can cause him to move 5', 10', or 15' at his choice).}

\ability{Sneak Attack (Ex):}{At 3rd level, a Jester gains the ability to make sneak attacks as a rogue would. At 3rd level, his sneak attacks inflict 1 extra d6 of damage, and this increases by 1d6 at levels 6, 9, 11, 13, 15, 17, and 19.}

\ability{Jester's Feint (Ex):}{At 4th level, a Jester learns to shock and unnerve his enemies by throwing unexpected objects at them. At a swift action, he may toss a brightly colored object in the square of an enemy with a Sleight of Hand Check opposed by the enemy's Spot check. If it succeeds, the enemy is denied his Dex bonus for the Jester's next attack.

Some Jesters use objects with magical or alchemic effects that act in an enemy's square to use with this ability, while others use colored balls, fruit, pieces of cloth or scarves, or other cast-off materials that fit the requirement of being brightly colored. Wealthy, desperate, or foolish Jesters sometime used coins or gems.}

\ability{Cruel Comment (Ex):}{At 5th level, the Jester has learned to say extremely funny but hurtful things about others. As a swift action, the Jester can make a Bluff check opposed by the target's level plus Charisma check. If the target fails this check, he suffers a -4 to attack rolls, saves, and all other checks. This effect lasts 3 rounds. This is a language-dependant ability.}

\ability{Sight Gag:}{At 7th level, the Jester may apply the Silent Spell and Still Spell metamagics spontaneously to his spells, but only if he casts them as full-round actions. This ability only works with spells on the Jester list, and it does not increase the spell's level or slot used.}

\ability{Low Comedy (Ex):}{By using this ability, a Jester of 8th level or higher can double the armor check penalty of an opponent within 50 feet that he hits with a ranged touch attack. Using this ability is an attack action and counts as a thrown weapon. The penalty can be restored to its normal value with 10 minutes and a bar of soap.}

\ability{Slapstick (Ex):}{At 8th level, any successful sneak attack also inflict a -2 Dex penalty to an enemy for one round.}

\ability{Jack-in-the-Box King (Sp):}{Twice per day, a 10th level Jester may use \spell{fabricate} or \spell{major creation} as a spell-like ability, but only if he is constructing weapons or traps.}

\ability{Killer Clown (Ex):}{At 12th level, so long as he meets the requirements of his Harlequin's face ability, the Jester can make a special Intimidate check as a move action. If successful, this check causes the enemy to suffer the panicked condition for a round per Jester level. This is a mind-effecting fear effect.}

\ability{Annoy the Gods (Su):}{As world-class pranksters, Jesters must learn to avoid the curses and transformations of enemies with a sense of humor. Any time a 14th level Jester has spent at least one round as the victim of an effect that could be removed by a \spell{break enchantment} effect, the effect is removed.}

\ability{Prat Fall (Ex):}{At 16th level, any time a Jester strikes an enemy with a sneak attack, the Jester can make a free Trip attack that does not provoke an Attack of Opportunity. This ability cannot be used on any one enemy more than once a round. The Jester may not be tripped if this fails, and it may be used with ranged sneak attacks. The Jester may substitute his Dexterity modifier for his Strength modifier for the opposed test to trip his foe.}

\ability{Last Trick (Su):}{At 18th level, the Jester can turn even his death into a joke. Any time the Jester is killed or knocked unconscious, one of his spells known is cast as if it were spell in a \spell{contingency} effect.}

\ability{Eternal Trickster (Ex):}{At 20th level, the Jester can become a personification of the Laughing God Who has No Temples. While meeting the requirements of his Harlequin's Mask ability, he does not age and is under the effects of a \spell{mind blank} effect.}


\classname{Assassin} \label{class:assassin}
\vspace{-8pt}
\quot{``I kill people. Individually, you are a person. Collectively, I think you count as people.''}

\desc{An assassin is a master of the art of killing, a vicious weapon honed by experience and inclination to learn the myriad ways to end a life. Unlike common warriors or rogues, an Assassin does not study various fighting arts or muddle his training with martial dirty tricks, he instead studies the anatomy of the various creatures of wildly different anatomies and forms of existence, and he uses this knowledge to place his blows in areas vital for biological or mystical reasons. Stealth and sudden violence are his hallmarks, and various exotic tools and killing methods become his tools.}

\desc{While most societies consider assassination to be a vile art, or at best a dishonorable or unvalorous one, the reasons that drive these killers vary. Cold-hearted mercenaries share a skill set with dedicated demon-hunters, differing only in the application of their skills. Only the most na\"ive student of ethics believes that all killing is evil, or that nobility cannot be found in a mercifully quick death.}

\ability{Alignment:}{An Assassin may be of any alignment.}

\ability{Races:}{Any}

\ability{Starting Gold:}{6d4x10 gp (150 gold)}

\ability{Starting Age:}{As Rogue.}

\ability{Hit Die:}{d6}

\ability{Class Skills:}{The Assassin's skills (and the key ability for each skill) are Balance (Dex), Bluff (Cha), Climb (Str), Concentration (Con), Craft (Int), Diplomacy (Cha), Disable Device (Int), Disguise (Cha), Gather Information (Cha), Hide (Dex), Intimidate (Cha), Jump (Str), Knowledge (all) (Int), Listen (Wis), Move Silently (Dex), Perform (Cha), Profession (Wis), Search (Int), Sense Motive (Wis), Sleight of Hand (Dex), Spellcraft (Int), Spot (Wis), Swim (Str), Tumble (Dex), and Use Magic Device (Cha).}

\ability{Skills/Level:}{6 + Intelligence Bonus}

\begin{table}[htb]
\begin{small}
\begin{tabular}{lp{1.9cm}p{0.7cm}p{0.7cm}p{0.7cm}l}
Level  &Base Attack Bonus &Fort Save &Ref Save &Will Save &Special\\
1st &+0 &+2 &+2 &+0 &Poison Use, Death Attack +3d6, Personal Immunity, Spellcasting\\
2nd &+1 &+3 &+3 &+0 &Uncanny Dodge, Death Attack +4d6\\
3rd &+2 &+3 &+3 &+1 &Hide in Plain Sight, Death Attack +5d6\\
4th &+3 &+4 &+4 &+1 &Cloak of Discretion, Death Attack +6d6\\
5th &+3 &+4 &+4 &+1 &Traps, Trapmaking, Death Attack +7d6\\
6th &+4 &+5 &+5 &+2 &Palm Weapon, Death Attack +8d6\\
7th &+5 &+5 &+5 &+2 &Full Death Attack, Death Attack +9d6\\
8th &+6/+1 &+6 &+6 &+2 &Nerve of the Assassin, Death Attack +10d6\\
9th &+6/+1 &+6 &+6 &+3 &Improved Uncanny Dodge, Death Attack +11d6\\
10th &+7/+2 &+7 &+7 &+3 &Skill Mastery, Death Attack +12d6\\
11th &+8/+3 &+7 &+7 &+3 &Poisonmaster, Death Attack +13d6\\
12th &+8/+3 &+8 &+8 &+4 &Personal Immunity, Death Attack +14d6\\
13th &+9/+4 &+8 &+8 &+4 &Exotic Method, Death Attack +15d6\\
14th &+10/+5 &+9 &+9 &+4 &Personal Immunity, Death Attack +16d6\\
15th &+11/+6/+6 &+9 &+9 &+5 &Killer's Proof, Death Attack +17d6\\
16th &+12/+7/+7 &+10 &+10 &+5 &Exotic Method, Death Attack +18d6\\
17th &+12/+7/+7 &+10 &+10 &+5 &Death by a Thousand Cuts, Death Attack +19d6\\
18th &+13/+8/+8 &+11 &+11 &+6 &Mind Blank, Death Attack +20d6\\
19th &+14/+9/+9 &+11 &+11 &+6 &Exotic Method, Death Attack +21d6\\
20th &+15/+10/+10 &+12 &+12 &+6 &Killing Strike, Death Attack +22d6\\
\end{tabular}
\end{small}
\end{table}

\begin{floatingfigure}{3.9in}
\begin{small}
\noindent\begin{tabular}{lllllllllllllllll}
 & \multicolumn{7}{c}{Assassin Spells Per Day}  &   &\multicolumn{7}{c}{Assassin Spells Known}\\
  &0 &1 &2 &3 &4 &5 &6 &  &  &0 &1 &2 &3 &4 &5 &6\\
1 &2 &- &- &- &- &- &- &  &1 &4 &- &- &- &- &- &-\\
2 &3 &0 &- &- &- &- &- &  &2 &5 &2 &- &- &- &- &-\\
3 &3 &1 &- &- &- &- &- &  &3 &6 &3 &- &- &- &- &-\\
4 &3 &2 &0 &- &- &- &- &  &4 &6 &3 &2 &- &- &- &-\\
5 &3 &3 &1 &- &- &- &- &  &5 &6 &4 &3 &- &- &- &-\\
6 &3 &3 &2 &- &- &- &- &  &6 &6 &4 &3 &- &- &- &-\\
7 &3 &3 &2 &0 &- &- &- &  &7 &6 &4 &4 &2 &- &- &-\\
8 &3 &3 &3 &1 &- &- &- &  &8 &6 &4 &4 &3 &- &- &-\\
9 &3 &3 &3 &2 &- &- &- &  &9 &6 &4 &4 &3 &- &- &-\\
10 &3 &3 &3 &2 &0 &- &- &  &10 &6 &4 &4 &4 &2 &- &-\\
11 &3 &3 &3 &3 &1 &- &- &  &11 &6 &4 &4 &4 &3 &- &-\\
12 &3 &3 &3 &3 &2 &- &- &  &12 &6 &4 &4 &4 &3 &- &-\\
13 &3 &3 &3 &3 &2 &0 &- &  &13 &6 &4 &4 &4 &4 &2 &-\\
14 &3 &3 &3 &3 &3 &1 &- &  &14 &6 &4 &4 &4 &4 &3 &-\\
15 &3 &3 &3 &3 &3 &2 &- &  &15 &6 &4 &4 &4 &4 &3 &-\\
16 &3 &3 &3 &3 &3 &2 &0 &  &16 &6 &5 &4 &4 &4 &4 &2\\
17 &3 &3 &3 &3 &3 &3 &1 &  &17 &6 &5 &5 &4 &4 &4 &3\\
18 &3 &3 &3 &3 &3 &3 &2 &  &18 &6 &5 &5 &5 &4 &4 &3\\
19 &3 &3 &3 &3 &3 &3 &3 &  &19 &6 &5 &5 &5 &5 &4 &4\\
20 &3 &3 &3 &3 &3 &3 &3 &  &20 &6 &5 &5 &5 &5 &5 &4\\
\end{tabular}
\end{small}
\end{floatingfigure}

\smallskip\noindent All of the following are Class Features of the Assassin class.

\ability{Weapon and Armor Proficiency:}{Assassins are proficient with all Light Weapons, as well as simple weapons, repeating crossbows, and hand crossbows. At first level, an Assassin gains proficiency with one Exotic Weapon of her choice. Assassins are proficient with Light Armor but not with shields.}

\ability{Spellcasting:}{The Assassin is an Arcane Spellcaster with the same spells per day and spells known progression as a Bard, except that he gains no more than three spell slots per level. An Assassin's spells known may be chosen from the Sorcerer/Wizard list, and must be from the schools of Divination, Illusion, or Necromancy. To cast an Assassin spell, she must have an Intelligence at least equal to 10 + the Spell level. The DC of the Assassin's spells is Intelligence based and the bonus spells are Intelligence based.}

\ability{Poison Use (Ex):}{An Assassin may prepare, apply, and use poison without any chance of poisoning herself.}

\ability{Death Attack (Ex):}{An Assassin may spend a full-round action to study an opponent who would be denied their Dexterity bonus if she instead attacked that target. If she does so, her next attack is a Death Attack if she makes it within 1 round. A Death Attack inflicts a number of extra dice of damage equal to her Assassin level plus two dice, but only if the target is denied its Dexterity Bonus to AC against that attack. Special attacks such as a coup de grace may be a Death Attack. Assassins are well trained in eliminating magical or distant opponents, and do not have to meet the stringent requirements of a sneak attack, though if a character has both sneak attack and death attack, they stack if the character meets the requirements of both. As long as the victim is denied their dexterity against attacks from the assassin during the study action and the attack itself, it counts as a death attack. An Assassin may load a crossbow simultaneously with his action to study his target if he has a Base Attack Bonus of +1 or more.}

\ability{Personal Immunity (Ex):}{Choose four poisons, an Assassin is immune to all four of those poisons, even if they are made available in a stronger strength. At levels 5, 7, and 12 the Assassin may choose one more type of poison to become immune to. At level 14, an Assassin becomes immune to all poisons.}

\ability{Uncanny Dodge (Ex):}{Starting at 2nd level, an Assassin can react to danger before his senses would normally allow him to do so. He retains her Dexterity bonus to AC (if any) even if she is caught flat-footed or struck by an invisible attacker. However, he still loses her Dexterity bonus to AC if immobilized. If an Assassin already has uncanny dodge from a different class he automatically gains improved uncanny dodge (see below) instead.}

\ability{Hide in Plain Sight (Ex):}{A 3rd level Assassin can hide in unusual locations, and may hide in areas without cover or concealment without penalty. An Assassin may even hide while being observed. This ability does not remove the -10 penalty for moving at full speed, or the -20 penalty for running or fighting.}

\ability{Cloak of Discretion (Su):}{At 4th level, an Assassin is protected by a constant \emph{nondetection} effect, with a caster level equal to his character level.}

\ability{Trapfinding:}{At 5th level, Assassins can use the Search skill to locate traps when the task has a Difficulty Class higher than 20. Finding a nonmagical trap has a DC of at least 20, or higher if it is well hidden. Finding a magic trap has a DC of 25 + the level of the spell used to create it. Assasins can use the Disable Device skill to disarm magic traps. A magic trap generally has a DC of 25 + the level of the spell used to create it. An Assassin who beats a trap's DC by 10 or more with a Disable Device check can study a trap, figure out how it works, and bypass it (with her party) without disarming it.}

\ability{Trapmaking:}{At 5th level, the Assassin learns to build simple mechanical traps in out of common materials. As long as has access to ropes, flexible material like green wood, and weapon-grade materials like sharpened wooden sticks or steel weapons, he can build an improvised trap in 10 minutes. He can build any non-magical trap on the "CR 1" trap list that doesn't involve a pit. These traps have a Search DC equal to 20 + the Assassin's level, have a BAB equal to his own, and are always single-use traps. He may add poison to these traps, if he has access to it, but it will dry out in an hour.}

\ability{Palm Weapon (Su):}{At 6th level, the Assassin learns to conceal weapons with supernatural skill. Any weapon successfully concealed with Sleight of Hand cannot be found with divination magic.}

\ability{Full Death Attack:}{At 7th level, if the Assassin studies an opponent to perform a Death Attack, she can make a full attack during the next round where every attack inflicts Death Attack damage as long as the target was denied their Dexterity bonus to AC against the first attack in the full attack action.}

\ability{Nerve of the Killer:}{At 8th level, an Assassin gains a limited immunity to compulsion and charm effects. While studying a target for a Death Attack, and for one round afterward, he counts as if he were within a \spell{protection from evil} effect. This does not confer a deflection bonus to AC.}

\ability{Improved Uncanny Dodge (Ex):}{An Assassin of 9th level or higher can no longer be flanked. This defense denies another character the ability to sneak attack the character by flanking him, unless the attacker has at least four more levels in a class that provides sneak attack than the target. If a character already has uncanny dodge (see above) from a second class, the character automatically gains improved uncanny dodge instead, and the levels from the classes that grant uncanny dodge stack to determine the minimum level required to flank the character.}

\ability{Skill Mastery (Ex):}{At 10th level, an Assassin becomes so certain in the use of certain skills that she can use them reliably even under adverse conditions. When making a skill check with Climb, Disable Device, Hide, Move Silently, Search, Spellcraft, Use Magic Device, Use Rope, or Swim, she may take 10 even if stress and distractions would normally prevent her from doing so.}

\ability{Poisonmaster:}{At 11th level, the Assassin learns alchemic secrets for creating short-term poisons. By expending an entire healer's kit worth of materials and an hour of time, he can synthesize one dose of any poison in the DMG. This poison degrades to uselessness in one week.}

\ability{Exotic Method:}{At 13th, 16th, and 19th level the Assassin learns an exotic form of killing from the list below. Once chosen, this ability does not change:}
\listone

    \item \ability{Carrier:}{Three times per day, the Assassin can cast \spell{contagion} as a swift action spell-like ability.}
    \item \ability{Poison of the Cockatrice:}{Twice per day, the Assassin can cast \spell{flesh to stone} as a swift action spell-like ability.}
    \item \ability{Killer Faerie Arts:}{Twice per day, the Assassin can cast \spell{polymorph other} as a swift action spell-like ability.}
    \item \ability{Proxy Assassin:}{Twice per day, the Assassin can cast \spell{summon monster VII} as a spell-like ability. This effect lasts 10 minutes.}
    \item \ability{Death By Plane:}{Once per day, the Assassin can cast \spell{plane shift} as a spell-like ability.}
    \item \ability{Dimesional Rip:}{Once per day, the Assassin can cast \spell{implosion} as a spell-like ability. The duration of this effect is three rounds.}
    \item \ability{New School:}{The Assassin may now choose spells known from a new school.}
\end{list}
\vspace{8pt}

\ability{Killer's Proof (Su):}{At 15th level, the Assassin learns to steal the souls of those he kills. If he is holding an onyx worth at least 100 GP when he kills an enemy, he may place their soul within the gem as if he has cast \spell{soul bind} on them at the moment of their death.}

\ability{Death by a Thousand Cuts:}{At 17th level, the assassin has learned to kill even the hardiest of foes by reducing their physical form to shambles. Every successful Death attack inflicts a cumulative -2 Dexterity penalty to the Assassin's victim. These penalties last one day.}

\ability{Mind Blank (Su):}{At 18th level, the Assassin is protected by a constant \spell{mind blank} effect.}

\ability{Killing Strike (Su):}{At 20th level, the Assassin's Death Attacks bypass his victim's DR and hardness.}


\classname{Thief Acrobat} \label{class:thiefacrobat}
\vspace{-8pt}
\quot{"They put their safe on the ceiling, it's like they wanted me to take these scrolls."}

\desc{While the common rogue is a thief, con-man, and scout extraordinaire, the thief acrobat is a highly trained specialist in the art of housebreaking and feats of dexterity and acrobatics. As an adventurer, they are masters of negotiating difficult terrain and situations with flair and panache. Masters of athletics and gymnastics, they hone their art to a level that seems to be magical to the initiated. Most use these skills to gain the easy score or poorly defended hoard, but some take up the life of an adventurer as a chance to test their purely mortals skills against the world full of magic and supernatural creatures.}

\ability{Alignment:}{Any.}

\ability{Races:}{Any}

\ability{Starting Gold:}{4d4x10 gp (100 gold)}

\ability{Starting Age:}{As Rogue}

\ability{Hit Die:}{d6}

\ability{Class Skills:}{The Thief Acrobat's skills (and the key ability for each skill) are Appraise (Int), Balance (Dex), Bluff (Cha), Climb (Str), Craft (Int), Diplomacy (Cha), Disable Device (Int), Escape Artist (Dex), Hide (Dex), Jump (Str), Listen (Wis), Move Silently (Dex), Perform (Cha), Profession (Wis), Search (Int), Sense Motive (Wis), Sleight of Hand (Dex), Spot (Wis), Swim (Str), Tumble (Dex), Use Magic Device (Cha), and Use Rope (Dex).}

\ability{Skills/Level:}{6 + Intelligence Bonus}

\begin{table}[tbh]
\begin{small}
\begin{tabular}{lp{2cm}p{0.7cm}p{0.7cm}p{0.7cm}l}
Level  &Base Attack  Bonus &Fort Save &Ref Save &Will Save &Special\\
1st &+0 &+0 &+2 &+0 &Acrobatic Flair, Trapfinding, Pole Jump\\
2nd &+1 &+0 &+3 &+0 &+1d6 Sneak Attack, Evasion\\
3rd &+2 &+1 &+3 &+1 &Sure Climb, Kip Up\\
4th &+3 &+1 &+4 &+1 &\spell{Detect Magic}, Grapple Line\\
5th &+3 &+1 &+4 &+1 &+2d6 Sneak Attack, Rapid Stealth\\
6th &+4 &+2 &+5 &+2 &Mercurial Charge\\
7th &+5 &+2 &+5 &+2 &+3d6 Sneak Attack, Unsettling Choreography\\
8th &+6/+1 &+2 &+6 &+2 &Improved Evasion\\
9th &+6/+1 &+3 &+6 &+3 &+4d6 Sneak Attack, Athletic Cascade\\
10th &+7/+2 &+3 &+7 &+3 &Skill Mastery\\
11th &+8/+3 &+3 &+7 &+3 &+5d6 Sneak Attack, Aggressive Stealth\\
12th &+8/+3 &+4 &+8 &+4 &Dedicated Evasion\\
13th &+9/+4 &+4 &+8 &+4 &+6d6 Sneak Attack, Power Slide\\
14th &+10/+5 &+4 &+9 &+4 &Shadow Tumble\\
15th &+11/+6/+6 &+5 &+9 &+5 &+7d6 Sneak Attack\\
16th &+12/+7/+7 &+5 &+10 &+5 &Death From Above\\
17th &+12/+7/+7 &+5 &+10 &+5 &+8d6 Sneak Attack\\
18th &+13/+8/+8 &+6 &+11 &+6 &Supreme Skill Mastery\\
19th &+14/+9/+9 &+6 &+11 &+6 &+9d6 Sneak Attack\\
20th &+15/+10/+10 &+6 &+12 &+6 &Supreme Evasion\\
\end{tabular}
\end{small}
\end{table}

\smallskip\noindent All of the following are Class Features of the Thief Acrobat class.

\ability{Weapon and Armor Proficiency:}{Thief Acrobats are proficient with all simple weapons, as well as the sap, the shortsword, the whip, the bolas, the long staff, and the shuriken. Thief Acrobats are proficient with light armor but not with shields of any kind.}

\ability{Trapfinding:}{At 1st level, Thief Acrobats can use the Search skill to locate traps when the task has a Difficulty Class higher than 20. Finding a nonmagical trap has a DC of at least 20, or higher if it is well hidden. Finding a magic trap has a DC of 25 + the level of the spell used to create it. Thief Acrobats can use the Disable Device skill to disarm magic traps. A magic trap generally has a DC of 25 + the level of the spell used to create it. An Thief Acrobat who beats a trap's DC by 10 or more with a Disable Device check can study a trap, figure out how it works, and bypass it (with her party) without disarming it.}

\ability{Acrobatic Flair (Ex):}{A Thief Acrobat may move her full movement while using the Tumble or Balance skill without suffering a penalty or increasing the DC of her check.}

\ability{Pole Jump (Ex):}{If holding a pole, spear, staff, long staff, or other pole-like object in both hands, a Thief Acrobat can add twice her reach to her final distance moved during a Jump check, and in this instance her jump distances are not limited by her height.}

\ability{Sneak Attack (Ex):}{At 2nd level, a Thief Acrobat gains the sneak attack ability as a Rogue. Her sneak attacks inflict an extra d6 of damage at 2nd level. This damage increases by 1d6 at levels 5, 7, 9, 11, 13, 15, 17, and 19.}

\ability{Evasion (Ex):}{If a 2nd level Thief Acrobat succeeds in a Reflex Save to halve damage, she suffers no damage instead.}

\ability{Sure Climb (Ex):}{At 3rd level, a Thief Acrobat gains a climb speed equal to half her land speed.}

\ability{Kip Up (Ex):}{At 3rd level, a Thief Acrobat may stand up from prone as a free action that does not provoke an attack of opportunity.}

\ability{\spell{Detect Magic (Sp)}:}{At 4th level, a Thief Acrobat may use \spell{detect magic} at-will as a spell-like ability. A Thief Acrobat may use her Appraise Skill in place of her Spellcraft in order to glean additional information from her \spell{detect magic}.}

\ability{Grapple Line:}{At 4th level, a Thief Acrobat becomes a master of using grapples and grappling lines. By firing a missile weapon designed as a grappling weapon at an unoccupied square and doing at least 1 point of damage to an object filling that square (wall, ceiling, statue, etc) or a securely affixed object (ceiling post, small statue affixed to floor, etc), a Thief Acrobat can run a rope from his current potion to that location as a full round action. He may then use this rope to make Balance or Climb checks as normal.

Weapons designed as grappling weapons have a simple pulley and loop attached at the end and are balanced for this modification, and have at least a 50' length of strong thread running through it and connected to a rope so that it can be pulled through swiftly. They cost an additional +1 GP each (ammunition costs as much as normal weapons), and suffer a 5 ft reduction in range increment. Many grappling weapons are made out of adamantite in order to better penetrate hard materials like stone.}

\ability{Rapid Stealth (Ex):}{At level 5, the Thief Acrobat does not suffer the -10 penalty to Move Silently or Hide for moving at her full normal speed. She still suffers the normal -20 penalties to hide and move silently for running or fighting if she performs those actions.}

\ability{Mercurial Charge (Ex):}{At level 6, a Thief Acrobat need not move in a straight line to charge, nor must she charge the closest available space. She still may not move back on herself during a charge, and her charge move still ends as soon as she threatens her target.}

\ability{Unsettling Choreography (Ex):}{A Thief Acrobat of 7th level is adept at making other creatures fall down, and may use her Dexterity Modifier in place of her Strength modifier when making a trip or bullrush attempt.}

\ability{Improved Evasion (Ex):}{When a Thief Acrobat of 8th level fails a Reflex Save to halve damage, she takes half damage anyway.}

\ability{Athletic Cascade (Ex):}{At 9th level, if a Thief Acrobat moves before making an attack, for the purposes of flanking she may count any square she has moved through as threatening an opponent, in addition to the space she is actually attacking from. In this manner, she may even flank with herself.}

\ability{Skill Mastery (Ex):}{At 10th level, a Thief Acrobat is able to take 10 on any Appraise, Balance, Disable Device, Jump, Hide, Move Silently, and Tumble checks even in stressful or dangerous situations.}

\ability{Aggressive Stealth (Ex):}{A Thief Acrobat of 11th level does not suffer the -20 penalty to Hide or Move Silently checks for running or fighting.}

\ability{Dedicated Evasion (Ex):}{At 12th level, a Thief Acrobat gains the ability to evade with almost supernatural skill. As a standard action, she can add her Thief Acrobat level as a Dodge bonus to her Reflex Saves and AC for one round.}

\ability{Power Slide (Ex):}{If a 13th level Thief Acrobat takes damage from an attack, she may allow herself to be flung backwards , thereby lessening the impact. She may make a Balance check with a DC equal to the damage inflicted and if she succeeds, she suffers only half damage. This is a skill check, not a Saving Throw, so abilities such as Evasion do not apply. She is moved away from the source of damage by 5' for every 5 points of damage (or part there of) negated in this way. If there is not enough space for her to move, she suffers a d6 of damage for each square not moved. If she passes through an occupied square, the Thief Acrobat would have to make a tumble check to avoid attacks of opportunity.}

\ability{Shadow Tumble (Su):}{At 14th level, a Thief Acrobat has learned to tumble through the Plane of Shadow. She may make a tumble check with a DC equal to 10 plus five for every square she wishes to pass through another plane of existence. Intervening terrain, even walls of force have no effect on movement through the plane of shadow. The Thief Acrobat's total distance moved does not increase, no matter how much of it may be taken through the plane of shadow.}

\ability{Death From Above (Ex):}{At 16th level, the Thief Acrobat has learned to used the energy of a fall to devastating effect. If the Thief Acrobat can fall at least 30' (by falling from a height or by using a Jump check) and end in her enemy's square, any attacks made at the end of that fall do triple damage. Sneak Attack is not multiplied in this way.}

\ability{Supreme Skill Mastery:}{At 19th level, a Thief Acrobat is able to take 20 on any Appraise, Balance, Disable Device, Jump, Hide, Move Silently, and Tumble checks even in stressful or dangerous situations, and does not take twenty times as long as usual for taking 20.}

\ability{Supreme Evasion (Ex):}{At 20th level, a Thief acrobat takes no damage from any effect requiring a Reflex save.}


\section{Prestige Classes}

The prestige class system is as old as Advanced Dungeons \& Dragons, and it has never really lived up to peoples' high expectations of it. From the original Bard on, there has always been an expectation that getting into a Prestige Class somehow \emph{should be} an ordeal where you get less power now and more power later. Others think that you should get more power now and pay for it by getting less power later on. That's crap.

Gaining levels isn't like purchasing a car. You shouldn't be allowed to save up character power in interest drawing accounts. You shouldn't be allowed to borrow power from the future at heinous interest rates. The fact is that every game of D\&D is a \emph{game}. And that doesn't just mean that each campaign is a game, it means that each individual session of D\&D is a game. And games that aren't fair aren't very fun. In the greater scheme of things it is theoretically possible to arrange a situation where one session might be unfair to one player and another session is unfair to the same player in the \emph{opposite fashion}, but the fact is that in practice this is just very mean to all the players all the time. People feel it when they're being screwed far more than when the chips are stacked with them, so this sort of thing is just highly oppressive to everyone involved.

Worse, while you can create some sort of abstract proof about potential long-term balance in either the ``buy now, pay later" or ``save up for the awesome" models, in actual D\&D games this simply does not work. It can't. While D\&D is inherently open ended, each actual game has a beginning, a middle, and an end. And while it would be convenient if every game began at 1st level and ended at 20th, we know that isn't what really happens. Campaigns begin at later levels (after the pay-offs have kicked in or setups have become obsolete), and they end before Epic (before pay-offs or interest payments kick in). And this is \emph{normal}. Any set up in which a character is supposed to have less power at one part of his career and more in another is unenforceable, there's no possible guarantee that both the low power and the high power period will ever actually happen in-game. In fact, in almost all cases it's a pretty good bet that they \emph{won't}.

A character's level determines what they should be able to do. That's their \emph{character level}, not their Class level. When a character is 7th level they should go 50/50 with a Medusa, a Hill Giant, a Spectre, and a Succubus. We know this, because that's what being a 7th level character \emph{means} according to the CR system. If a character lacks the abilities or the numerics to compete evenly against those monsters, then he's underpowered. If a character has the mad skills to consistently crush that kind of opposition, then he's overpowered. And that's where Prestige Classes can come in to patch things up -- because PrCs have a tendency to be available at about 7th level. So if the party Fighter isn't doing well against monsters of his level (and unless he's a pretty min/maxed build, he probably won't be), feel free to throw in PrCs for that character that are much more powerful. And if the party Druid is smacking those opponents down like a line of shots in a red light bar -- then you should consider cutting him off.

What follows are some examples of prestige classes you can introduce into your game to do what Prestige Classes do well -- give characters flavor abilities that they can be proud of and keep underperforming characters on track with the rest of the party.

\classname{Defiler of Temples}\label{class:defiler}
\vspace*{-8pt}
\quot{"That stuff is all sacred to Nerull, so we should desecrate it thoroughly. I'm going to need more beer."}

The gods are jealous creatures and demand that their followers forsake the other gods. More over, they ask their most devoted followers to destroy the icons and worshippers of their peers. A Defiler of Temples is one who has taken up that cause and seeks to destroy the gods that oppose his own.

\ability{Prerequisites:}{}
\listprereq
\itemability{Skills:}{9 ranks in Knowledge (Religion), 4 ranks in Knowledge (Dungeoneering)}
\itemability{Spellcasting:}{Must be able to cast Divine Spells.}
\itemability{Special:}{Must have slain a divine caster following a god other than your own or a cleric dedicated to a philosophy you find abhorrent.}
\end{list}\vspace*{8pt}

\ability{Hit Die:}{d8}

\ability{Class Skills:}{The Defiler of Temples' class skills (and the key ability for each skill) are Appraise (Int), Balance (Dex), Bluff (Cha), Climb (Str), Concentration (Con), Craft (Int), Decipher Script (Int), Diplomacy (Cha), Disable Device (Int), Disguise (Cha), Escape Artist (Dex), Gather Information (Cha), Hide (Dex), Jump (Str), Knowledge (all skills, taken individually) (Int), Listen (Wis), Move Silently (Dex), Profession (Wis), Sense Motive (Wis), Sleight of Hand (Dex), Spellcraft (Int), Swim (Str), and Use Magic Device (Cha).}

\ability{Skills/Level:}{4 + Intelligence Bonus}


\begin{table}[tbh]
\begin{small}
\begin{tabular}{lp{1.9cm}p{0.7cm}p{0.7cm}p{0.7cm}p{6cm}l}
Level&Base Attack Bonus&Fort Save&Ref Save&Will Save&Special&Spellcasting\\
1&+0&+0&+2&+2&\spell{Find traps}, Avoid Divine Wrath&+1 spellcasting level\\
2&+1&+0&+3&+3&Unreproachable Alignment, Divine Spell Resistance&+1 spellcasting level\\
3&+2&+1&+3&+3&\spell{Desecrate}&+1 spellcasting level\\
4&+3&+1&+4&+4&Stolen Power&+1 spellcasting level\\
5&+3&+1&+4&+4&\spell{Mindblank}&+1 spellcasting level\\
\end{tabular}
\end{small}
\end{table}

\smallskip\noindent All of the following are Class Features of the Defiler of Temples class.

\ability{Weapon and Armor Proficiency:}{A Defiler of Temples gains proficiency with no new weapons or armor.}

\ability{Spellcasting:}{Every level, the Defiler of Temples casts spells (including gaining any new spell slots and spell knowledge) as if he had also gained a level in a spellcasting class he had previous to gaining that level.}

\ability{Find Traps (Sp):}{A Defiler of Temples can cast \spell{find traps} as a spell-like ability with a caster level equal to his character level. This ability is usable at will.}

\ability{Avoid Divine Wrath (Ex):}{Magical effects created by divine magic are unable to perceive a Defiler of Temples, so a contingent effect such as a \spell{glyph of warding} will not activate in the Defiler of Temples' presence. A Defiler of Temples can still be affected by a divine spell which targets him or for which he is in the area of effect. Only divine contingent effects are incapable of acting upon him.}

\ability{Unreproachable Alignment (Ex):}{All effects that have any effect dependent upon the alignment of a character affect a 2nd level Defiler of Temples as if he had or lacked whatever alignment would cause the least possible effect. So a \spell{blasphemy} would not affect him as if he was Evil, while an \spell{order's wrath} would not affect him as a non-Chaotic individual.}

\ability{Divine Spell Resistance (Ex):}{The powers of the gods roll off the back of a Defiler of Temples like butter off a warm sofa. At 2nd level a Defiler of Temples gains Spell Resistance of 15 + Character level. This Spell Resistance only applies against Divine spells.}

\ability{Desecrate (Sp):}{A Defiler of Temples loves nothing more than to defile the temples of his enemies. It's what he lives for. At 3rd level, he can do it reflexively with whatever he has on hand. A Defiler of Temples is then able to cast \spell{desecrate} as a spell-like ability with a caster level equal to his character level. This ability is usable at will.}

\ability{Stolen Power:}{By 4th level, the Defiler of Temples has captured enough of the divine power from his enemies that he can use it himself. The Defiler of Temples gains one bonus domain that must be one not offered by his deity or consistent with his own philosophy. Spells can be prepared and cast from this extra domain in precisely the same manner as a normal Cleric's Domains.}

\ability{\spell{Mindblank (Su)}:}{At 5th level, the Defiler of Temples benefits from the effects of the spell \spell{mindblank}. Even the gods themselves cannot find him with their magic.}

\classname{Ninja of Gax}\label{class:ninjaofgax}
\vspace*{-8pt}
\quot{``Of course I'm not a ninja."}

In a world of gates, planar travel, and teleportation, cultural cross-pollination is a sure result. One such product is the Ninja Of Gax, a figure of stealth and deception born of an oriental tradition adopted by outsiders in order to gain access to mystical arts of disguise and obfuscation. Unlike common rogues or assassins, these covert operatives are part of an ancient tradition of ninjas passed from wily teacher to ambitious student, steeped in secrets gained from generations of practice and discipline. Unlike the originators of this tradition, these students do not hold allegiance to lords or owe loyalty to family or dynasty, making them among the most dangerous deceivers and spies. Their skills are their own to command, and they are free to pursue adventure or wealth as they see fit, unbound by the ties that enslave common ninja clans.

\ability{Prerequisites:}
\listprereq
\itemability{Skills:}{9 ranks in Concentration, 4 ranks in Disguise, 4 ranks in Knowledge Dungeoneering}
\itemability{Race:}{Must have the [Human] subtype.}
\itemability{Feat:}{Must have proficiency with at least one exotic weapon.}
\itemability{Spellcasting:}{Must be able to cast Arcane Spells.}
\end{list}\vspace*{8pt}

\ability{Hit Die:}{d4}

\ability{Class Skills:}{The Ninja of Gax's class skills (and the key ability for each skill) are Balance (Dex), Bluff (Cha), Climb (Str), Concentration (Con), Craft (Int), Decipher Script (Int), Disable Device (Int), Disguise (Cha), Escape Artist (Dex), Hide (Dex), Jump (Str), Knowledge (all skills, taken individually) (Int), Listen (Wis), Move Silently (Dex), Profession (Wis), Sense Motive (Wis), Sleight of Hand (Dex), Spellcraft (Int), Swim (Str), Tumble (Dex), and Use Magic Device (Cha).}

\ability{Skills/Level:}{4 + Intelligence Bonus}


\begin{table}[tbh]
\begin{small}
\begin{tabular}{lp{1.9cm}p{0.7cm}p{0.7cm}p{0.7cm}p{7cm}l}
Level&Base Attack Bonus&Fort Save&Ref Save&Will Save&Special&Spellcasting\\
1&+0&+0&+2&+2&Trapfinding, Good in Black, Cover Identity&+1 spellcasting level\\
2&+1&+0&+3&+3&Ki Breath, +1d6 Sneak Attack&+1 spellcasting level\\
3&+1&+1&+3&+3&Touch of the Ring, Ki Stride&+1 spellcasting level\\
4&+2&+1&+4&+4&+2d6 Sneak Attack&+1 spellcasting level\\
5&+2&+1&+4&+4&Ki Transport, Great in Black&+1 spellcasting level\\
\end{tabular}
\end{small}
\end{table}

\smallskip\noindent All of the following are Class Features of the Ninja of Gax class.

\ability{Weapon and Armor Proficiency:}{A Ninja of Gax gains proficiency with the katana, the shuriken, the sai, the long staff, and the nunchaku.}

\ability{Good in Black:}{A Ninja looks good in black, and knows it. While he's wearing only black clothing, he gains a +2 circumstance bonus to his Charisma.}

\ability{Cover Identity:}{A Ninja of Gax rarely admits that he is a member of an ancient Ninja Tradition. The Ninja of Gax gains a +10 bonus to his disguise checks to convince people that he is not a Ninja and is instead an ordinary arcane Spellcaster.}

\ability{Spellcasting:}{Every level, the Ninja of Gax casts spells (including gaining any new spell slots and spell knowledge) as if he had also gained a level in a spellcasting class he had previous to gaining that level.}

\ability{Ki Breath (Su):}{A Ninja of 2nd level gains the ability to hold his breath for an additional minute each day for each level of the Ninja of Gax class.}

\ability{Sneak Attack (Ex):}{At 2nd and 4th level, the Ninja of Gax gains a die of Sneak Attack as a Rogue.}

\ability{Touch of the Ring (Ex):}{A Ninja of Gax of 3rd level suffers no non-lethal damage from the Ring of Gax.}

\ability{Ki Stride (Su):}{At 3rd level, the Ninja of Gax may walk on water for 1 round per class level per day.}

\ability{Ki Transport (Su):}{At 5th level, a Ninja of Gax may walk through walls. By spending three rounds concentrating, the Ninja of Gax may transport himself up to 5 feet in any direction, completely bypassing any intervening obstructions. This ability may not be used if the Ninja of Gax has previously used his ability to hold his breath or walk on water that day.}

\ability{Great in Black:}{At 5th level, the Ninja of Gax looks great in black and his self confidence is bolstered enormously when he is clothed entirely in that color. His circumstance bonus to his Charisma increases to +4.}

\classname{Elothar Warrior of Bladereach}
\vspace*{-8pt}
\quot{"My name is Elothar. Your name is unimportant, for you shall soon be dead."}

The city of Bladereach sits at the mouth of the Typhon River that flows from the Bane Mires into Ferrin's Bay. The elves of Celentian's caravan come every year to trade with the largely human inhabitants of Bladereach and sometimes they leave more than the wares of the Black Orchard Hills when they leave. The results of these dalliances find that they never fit in amongst the people of Bladereach, and are taught the hard secrets of battle that the children of Bladereach have to offer. Often, these half-elven warriors turn to adventuring.

\ability{Prerequisites:}{}
\listprereq
\itemability{Skills:}{Use Rope 9 ranks.}
\itemability{Race:}{Half-elf.}
\itemability{Region:}{Must be from Bladereach.}
\itemability{Special:}{Name must be Elothar.}
\end{list}\vspace*{8pt}

\ability{Hit Die:}{d8}

\ability{Class Skills:}{The Elothar Warrior of Bladereach's class skills (and the key ability for each skill) are Balance (Dex), Climb (Str), Concentration (Con), Craft (Int), Handle Animal (Cha), Heal (Wis), Hide (Dex), Jump (Str), Knowledge (all skills taken individually) (Int), Listen (Wis), Move Silently (Dex), Profession (Wis), Ride (Dex), Search (Int), Spot (Wis), Survival (Wis), Swim (Str), and Use Rope (Dex).}

\ability{Skills/Level:}{4 + Intelligence Bonus}

\begin{table}[tbh]
\begin{small}
\begin{tabular}{lp{1.9cm}p{0.7cm}p{0.7cm}p{0.7cm}l}
Level &Base Attack Bonus &Fort Save &Ref Save &Will Save &Special\\
1st &+1 &+0 &+0 &+2 &Way of Two Swords\\
2nd &+2 &+0 &+0 &+3 &Tommy, Legacy of the Water Stone\\
3rd &+3 &+1 &+1 &+3 &Magic Swords, Immunity to Petrification\\
4th &+4 &+1 &+1 &+4 &I've Got That!\\
5th &+5 &+1 &+1 &+4 &Double Riposte, Fistful of Rubies\\
6th &+6 &+2 &+2 &+5 &Der'renya the Ruby Sorceress\\
7th &+7 &+2 &+2 &+5 &Ways and Paths\\
8th &+8 &+2 &+2 &+6 &Name of the First Eagle\\
9th &+9 &+3 &+3 &+6 &Blessing of the Gnome King\\
10th &+10 &+3 &+3 &+7 &Flying Ship, Your Money is No Good Here\\
11th &+11 &+3 &+3 &+7 &Demesne of Tralathon\\
12th &+12 &+4 &+4 &+8 &Mark of Ruin\\
13th &+13 &+4 &+4 &+8 &Sword of Kas, Dwarf Friend\\
14th &+14 &+4 &+4 &+9 &Happily Ever After, Khadrimarh\\
\end{tabular}
\end{small}
\end{table}


\smallskip\noindent All of the following are Class Features of the Elothar Warrior of Bladereach prestige class.

\ability{Weapon and Armor Proficiency:}{An Elothar Warrior of Bladereach gains proficiency with the Nerra Shard Sword, the Kaorti Ribbon Dagger, and the Shuriken.}

\ability{Way of Two Swords (Ex):}{With a single standard action, an Elothar Warrior of Bladreach may attack with a one-handed or light weapon in each hand at no penalties to-hit or damage for the weapon in his primary or off-hand.}

\ability{Tommy:}{At 2nd level, an Elothar Warrior of Bladereach is joined in his adventures by Tommy, a 5th level Halfling Rogue from Figmountain. Tommy is a loyal cohort and gains levels when the Elothar Warrior of Bladereach does. Other Halflings will be impressed by Tommy's apparent loyalty and the Elothar Warrior of Bladereach gains a +3 bonus to his Diplomacy checks when dealing with Halflings if Tommy is present.}

\ability{Legacy of the Water Stone (Sp):}{An Elothar Warrior of Bladereach of 2nd level has touched the fabled Water Stone, and gleaned a portion of its powers thereby. He may cast \spell{create water} as a spell-like ability at will. The caster level for this ability is 5.}

\ability{Magic Swords (Su):}{Any sword a 3rd level Elothar Warrior of Bladereach holds has an enhancement bonus equal to \third of his character level (round down, no maximum). The enhancement bonus fades one round after the Elothar Warrior of Bladereach stops touching the weapons.}

\ability{Immunity to Petrification (Ex):}{At 3rd level, an Elothar Warrior of Bladereach cannot be petrified.}

\ability{I've Got That! (Sp):}{At 4th level, an Elothar Warrior of Bladereach can mimic the effects of a \spell{drawmij's instant summons} at will. The Elothar Warrior of Bladereach does not need an \spell{arcane mark} on the item, nor does he need a sapphire to call the item in question.}

\ability{Double Riposte (Ex):}{If an opponent provokes an attack of opportunity from a 5th level Elothar Warrior of Bladereach, the Elothar Warrior of Bladereach may attack with a weapon in each hand at no penalty. This is considered a single attack of opportunity for purposes of how many attacks of opportunity the Elothar Warrior of Bladereach is allowed in a turn.}

\ability{Fistful of Rubies:}{At 5th level, an Elothar Warrior of Bladereach finds 10,000 gp worth of rubies.}

\ability{Der'renya the Ruby Sorceress:}{At 6th level, an Elothar Warrior of Bladereach is joined in his travels by Der'renya the Ruby Sorceress, a beautiful Drow magician. She is a Wizard 6/ Seeker of the Lost Wizard Traditions 4, and gains levels when he does. Other dark elves will be angered by Der'renya's betrayal, and will be if anything even less friendly with the Elothar Warrior of Bladereach if encountered with her.}

\ability{Ways and Paths (Su):}{At 7th level, an Elothar Warrior of Bladereach can make his way back to any plane he's ever been to. By wandering around in the wilderness for three days, he can make a Survival check (DC 25) to shift himself and anyone traveling with him to another plane.}

\ability{Name of the First Eagle (Sp):}{At 8th level, an Elothar Warrior of Bladereach can speak the name of the first eagle, which summons a powerful giant eagle that has the attributes of a Roc (though it is only large sized). The eagle appears for one hour, and may be summoned once per day.}

\ability{Blessing of the Gnome King (Su):}{At 9th level, an Elothar Warrior of Bladereach has pleased the king of the Gnomes so thoroughly that he is granted a portion of the gnomish power. The Elothar Warrior of Bladereach can speak with burrowing animals and sees through illusions as if he had \spell{true seeing cast} upon him by a 20th level Sorcerer.}

\ability{Flying Ship:}{At 10th level, an Elothar Warrior of Bladereach finds a Flying Ship from the Eberron setting. And can pilot it around.}

\ability{Your Money is no Good Here:}{An Elothar Warrior of Bladereach of 10th level gets free drinks and food at The Wandering Eye, a tavern in Sigil.}

\ability{Demesne of Tralathon}{At 11th level, an Elothar Warrior of Bladereach gains sole control of Tralathon, a small demiplane that appears to be an abandoned Githyanki outpost. Tralathon has several one-way portals that exit onto places on the Astral Plane, the Prime Material, and Limbo. The Elothar Warrior of Bladereach may planeshift to Tralathon at will as a spell-like ability.}

\ability{Mark of Ruin (Su):}{At 12th level, an Elothar Warrior of Bladereach is permanently marked with the Mark of Ruin, which causes all of his melee attacks to ignore hardness and damage reduction.}

\ability{Sword of Kas:}{An Elothar Warrior of Bladereach finds the Sword of Kas at level 13.}

\ability{Dwarf Friend (Ex):}{The deeds of an Elothar Warrior of Bladereach are well remembered by Dwarves when he reaches level 13. Dwarves he encounters are treated as Friendly.}

\ability{Happily Ever After:}{At 14th level, an Elothar Warrior of Bladereach becomes king of Bladereach with Der'renya as his queen. The castle of Halan Shador, that used to belong to the Lichking Hadrach is his to rule from.}

\ability{Khadrimarh:}{A 14th level, an Elothar Warrior of Bladereach has a young adult white dragon named Khadrimarh as a pet.}

\abox{Elothar Warriors of Bladereach in your campaign}{You may want to adapt this prestige class to the specifics of your campaign. In other campaign worlds, the race, region, and name requirements of this class may need to be changed to fit with the overall narrative.}

\classname{Dungeon Veteran}\label{class:dungeonveteran}
\vspace*{-8pt}
\quot{"No. If the floor looks like a chessboard, don't walk on it."}

Dungeons are dangerous places, and those who venture into them develop a certain paranoia that is readily identifiable. Who else would take a ten foot pole to the privy to prod the ceiling before entering?

\ability{Prerequisites:}{}
\listprereq
\itemability{Skills:}{9 ranks in Climb, 4 ranks in Knowledge (Dungeoneering)}
\itemability{Feat:}{Power Attack, Cleave}
\end{list}\vspace*{8pt}

\ability{Hit Die:}{d8}

\ability{Class Skills:}{The Dungeon Veteran's class skills (and the key ability for each skill) are Balance (Dex), Climb (Str), Concentration (Con), Craft (Int), Disable Device (Int), Handle Animal (Cha), Heal (Wis), Hide (Dex), Jump (Str), Knowledge (dungeoneering) (Int), Knowledge (geography) (Int), Listen (Wis), Move Silently (Dex), Profession (Wis), Ride (Dex), Search (Int), Spot (Wis), Survival (Wis), Swim (Str), and Use Rope (Dex).}

\ability{Skills/Level:}{4 + Intelligence Bonus}


\begin{table}[tbh]
\begin{small}
\begin{tabular}{lp{1.9cm}p{0.7cm}p{0.7cm}p{0.7cm}l}
Level&Base Attack Bonus&Fort Save&Ref Save&Will Save&Special\\
1&+1&+0&+2&+2&Evasion, Dramatic Attack\\
2&+2&+0&+3&+3&Darkvision, Trap Sense\\
3&+3&+1&+3&+3&Loyal Steel, Improved Property Damage\\
4&+4&+1&+4&+4&Looking For Trouble, Exotic Weapon\\
5&+5&+1&+4&+4&\spell{Antimagic Field}, Treasure Sense\\
\end{tabular}
\end{small}
\end{table}

\smallskip\noindent All of the following are Class Features of the Dungeon Veteran class.

\ability{Weapon and Armor Proficiency:}{A Dungeon Veteran gains proficiency with no new weapons or armor.}

\ability{Evasion (Ex):}{If a Dungeon Veteran succeeds in a Reflex Save to halve damage, he suffers no damage instead. If he already has the Evasion class feature, he gains Improved Evasion instead.}

\ability{Dramatic Attack (Ex):}{Dungeon Veterans fight with flair and gusto and take full advantage of the exotic and dangerous surroundings their battles take place in. When a Dungeon Veteran strikes an opponent with a weapon for 10 or more damage, they may elect to perform a Bullrush against that opponent. This Bullrush maneuver does not provoke an attack of opportunity and is considered to automatically touch the opponent. The Dungeon Veteran does not move with this Bullrush.}

\ability{Darkvision (Ex):}{A 2nd level Dungeon Veteran gains Darkvision with a range of 120 feet.}

\ability{Trap Sense (Ex):}{At 2nd level, a Dungeon Veteran gains a Dodge bonus to AC and Saves against Traps equal to his Class Level.}

\ability{Loyal Steel (Ex):}{Every weapon the Dungeon Veteran throws or fires is treated as having the Returning quality once the Dungeon Veteran achieves 3rd level.}

\ability{Improved Property Damage (Ex):}{Sometimes, it's safer just to go through the wall. A 3rd level Dungeon Veteran's attacks ignore the hardness of unattended objects.}

\ability{Looking for Trouble (Ex):}{Dungeon Veterans spend their lives in a constant state of readiness and are unphased by attacks from any direction. When a Dungeon Veteran reaches 4th level he adds his class level to his Spot, Listen, Search, and Initiative checks.}

\ability{Exotic Weapon:}{At 4th level, a Dungeon Veteran gains Exotic Weapon Proficiency in a weapon of his choice as a bonus feat.}

\ability{\spell{Antimagic Field (Sp)}:}{Once per day, a 5th level Dungeon Veteran can cast antimagic field as a spell-like ability. Caster Level is equal to character level.}

\ability{Treasure Sense (Su):}{The greed of those who venture into dungeons is legendary. When a Dungeon Veteran reaches 5th level, he can detect very valuable items. The presence of an item or pile of coins can be felt by the Dungeon Veteran for 5' for every 1,000 gp in value of the item or hoard in question.}

\classname{Master of Snake Mountain}\label{class:masterofsnakemountaint}
\vspace*{-8pt}
\quot{"Now you muscle-bound boobs, prepare to meet your doom. Hahahahaha!"}

Dungeons are hot property. They are enormously expensive to build, and are by their nature defended from magical interventions that would otherwise render their occupants extremely vulnerable. Thus, when a dungeon is over run, it is generally not long before it gains a new occupant. The Master of Snake Mountain is in control of a dungeon, but there's no reason to believe he's the first occupant. He might not even be the second.

A Master of Snake Mountain is one who has taken control of a dungeon and used it as a military staging area to launch grand plans. Such men could politely be described as egomaniacs, and rarely have a kind word to say to anyone that isn't spoken in an extremely sarcastic fashion.

\ability{Prerequisites:}
\listprereq
\itemability{Skills:}{9 ranks in Knowledge Dungeoneering and Perform (oratory)\\
-or-\\
9 ranks in Knowledge Architecture and Engineering and Perform (oratory)}
\itemability{Feat:}{Leadership, Any Item Creation Feat.}
\itemability{Special:}{Must have control of a dungeon, whether by having it built yourself or by taking it from someone else by force.}
\end{list}\vspace*{8pt}

\ability{Hit Die:}{d8}

\ability{Class Skills:}{The Master of Snake Mountain's class skills (and the key ability for each skill) are Appraise (Int), Bluff (Cha), Climb (Str), Concentration (Con), Craft (Int), Decipher Script (Int), Diplomacy (Cha), Disguise (Cha), Escape Artist (Dex), Gather Information (Cha), Jump (Str), Knowledge (all skills, taken individually) (Int), Listen (Wis), Perform (Cha), Profession (Wis), Sense Motive (Wis), Sleight of Hand (Dex), Spellcraft (Int), and Swim (Str).}

\ability{Skills/Level:}{4 + Intelligence Bonus}


\begin{table}[tbh]
\begin{small}
\begin{tabular}{lp{1.9cm}p{0.7cm}p{0.7cm}p{0.7cm}p{6cm}l}
Level&Base Attack Bonus&Fort Save&Ref Save&Will Save&Special&Spellcasting\\
1&+0&+0&+2&+2&Stable of Henchmen, Bardic Music, Code of Conduct&+1 spellcasting level\\
2&+1&+0&+3&+3&Disposable Monstrous Cohort, Speak with Monsters&+1 spellcasting level\\
3&+2&+1&+3&+3&Belittling Tirade&+1 spellcasting level\\
4&+3&+1&+4&+4&Enhance Minions&+1 spellcasting level\\
5&+3&+1&+4&+4&Eyebeams, Wondrous Architect&+1 spellcasting level\\
\end{tabular}
\end{small}
\end{table}

\smallskip\noindent All of the following are Class Features of the Master of Snake Mountain class.

\ability{Weapon and Armor Proficiency:}{A Master of Snake Mountain gains no proficiency with any weapons or armor.}

\ability{Spellcasting:}{Every level, the Master of Snake Mountain casts spells (including gaining any new spell slots and spell knowledge) as if he had also gained a level in a spellcasting class he had previous to gaining that level.}

\ability{Stable of Henchmen:}{A Master of Snake Mountain is a landlord and his dungeon fills up will all manner of ne'er-do-wells and hooligans. Practically this means that a Master of Snake Mountain can swap out his cohort for another cohort of an appropriate level at the beginning of each adventure. This doesn't mean that the Master of Snake Mountain can simply loot a cohort's worth of equipment every adventure, because while the different available cohorts are interchangeable, they actually don't go anywhere special when they are traded out. A cohort that is traded out is not dismissed, he simply doesn't accompany the Master of Snake Mountain on a particular adventure. Such a cohort continues to be available in later adventures if the Master of Snake Mountain decides to swap back. All available cohorts gain levels when the Master of Snake Mountain does, whether they were accompanying his adventures or not.}

\ability{Bardic Music (Su):}{A Master of Snake Mountain can produce Bardic Music effects with his Perform (Oratory) as if he was a Bard with a level equal to his class level. If he actually has Bard levels, the abilities and uses per day stack if he is using Perform (Oratory).}

\ability{Code of Conduct:}{A Master of Snake Mountain must conduct his affairs with senseless, yet restrained villainy. He must abide by the following restrictions:

\listone
    \item A captured or surrendered foe may not be summarily executed, though they may be left in situations almost certain to kill them.
    \item A Master of Snake Mountain boasts constantly and gives believes himself far more accomplished and powerful than he is. He must explain his big plans to anyone who will listen.
    \item A Master of Snake Mountain must behave in a cowardly and villainous fashion. A Master of Snake Mountain may not accept a challenge he regards as fair or sacrifice himself for the good of others.
\end{list}

A Master of Snake Mountain who fails to abide by these restrictions loses his ability to use his Bardic Music abilities until he atones.}

\ability{Disposable Monstrous Cohort:}{At the beginning of every adventure, the Master of Snake Mountain is followed by a monster for no reason once he reaches 2nd level. This monster must be at least 2 CR less than his character level, and will be a Magical Beast, an Aberration, a Plant, a Dragon, or an Ooze. It will follow his orders to the best of its ability, but whether it survives or not it will be replaced just as mysteriously by another monster at the beginning of the next adventure.}

\ability{Speak with Monsters (Ex):}{The general gist of whatever a 2nd level Master of Snake Mountain happens to be ranting about gets across to any Magical Beasts, Plants, Aberrations, Dragons, or Oozes that can hear his tirades but do not have a language.}

\ability{Belittling Tirade (Su):}{At 3rd level, the Master of Snake Mountain can use his Bardic Orations to make people feel bad about themselves. Such creatures receive a -2 morale penalty to attack rolls and saving throws and a -4 morale penalty to their Level Check to oppose intimidate checks. A Master of Snake Mountain can belittle any number of creatures within medium range as a standard action. The feelings of inadequacy last for 1 hour.}

\ability{Enhance Minions (Su):}{At 4th level, the Master of Snake Mountain gains the ability to make grafts. He may supply grafts from any graft list, and may apply grafts from different lists to the same creature (though the maximum of 8 grafts still applies). The costs for applying these grafts are half normal, though he cannot implant grafts into himself.}

\ability{Eyebeams (Su):}{A Master of Snake Mountain of 5th level has the ability to fire painful or deadly rays from his eyes. The Eyebeams are a ray effect with short range, and a creature struck with them (a ranged touch attack) is affected as per a \spell{symbol of pain}. At his option, the Eyebeams may also inflict 4d6 of Force Damage. Once fired, the Eyebeams may not be used again for 1d4+1 rounds.}

\ability{Wondrous Architect:}{At 5th level, a Master of Snake Mountain becomes a master of improving his own pad. He may make Wondrous Architecture in half the normal time at half the normal expense.}

\classname{Seeker of the Lost Wizard Traditions}\label{class:seekerofthelosttraditions}
\vspace*{-8pt}
\quot{``The old ways are the best ways. Magic in the past was capable of things you can't even comprehend."}

Empires have risen and fallen many times in history, and each time new magics are discovered and old magics are lost. The Seeker of the Lost Wizard Traditions is a user of magic who is convinced that the way magic was used in the past is better in some important fashion. Whether they are correct or not is something that the Mages of the Arcane Order would probably be willing to argue for days or weeks. But it is undeniable that much of the magic used by the Seeker are beyond the comprehension of those who have not taken the time to explore its ancient ways.

Previous generations have largely picked the surface clean of ancient magic power, and now those who wish to find the remnants of the ancient civilizations must journey deeper and deeper beneath the earth to find items that are protected from scrying.

\ability{Prerequisites:}
\listprereq
\itemability{Skills:}{Spellcraft 9 ranks; Knowledge (Dungeoneering) 9 ranks}
\itemability{Spellcasting:}{Must be able to prepare arcane spells of at least 2nd level.}
\itemability{Race:}{Human, Elf, or Gnome}
\itemability{Special:}{Must not be specialized in a school of magic other than Illusion.}
\end{list}\vspace*{8pt}

\ability{Hit Die:}{d4}

\ability{Class Skills:}{The Seeker of the Lost Wizard Traditions' class skills (and the key ability for each skill) are Concentration (Con), Craft (Int), Decipher Script (Int), Knowledge (all skills, taken individually) (Int), Profession (Wis), and Spellcraft (Int)).}

\ability{Skills/Level:}{2 + Intelligence Bonus}

\begin{table}[tbh]
\begin{small}
\begin{tabular}{lp{1.9cm}p{0.7cm}p{0.7cm}p{0.7cm}p{6cm}l}
Level&Base Attack Bonus&Fort Save&Ref Save&Will Save&Special&Spellcasting\\
1&+0&+0&+0&+2&Spell Reflection, Scroll Preparation&+1 spellcasting level\\
2&+1&+0&+0&+3&Uncapped Magic&+1 spellcasting level\\
3&+1&+1&+1&+3&Burst Conservancy&+1 spellcasting level\\
4&+2&+1&+1&+4&Harvest Magic&+1 spellcasting level\\
5&+2&+1&+1&+4&Expanse of the Sky&+1 spellcasting level\\
6&+3&+2&+2&+5&Temporary Portal&+1 spellcasting level\\
7&+3&+2&+2&+5&Unbreachable Stone Defense&+1 spellcasting level\\
\end{tabular}
\end{small}
\end{table}

\smallskip\noindent All of the following are Class Features of the Seeker of the Lost Wizard Traditions class.

\ability{Weapon and Armor Proficiency:}{A Seeker of the Lost Wizard Traditions gains no proficiency with any weapons or armor. However, a Seeker of the Lost Wizard Traditions is considered proficient with any magic sword he holds.}

\ability{Spellcasting:}{Every level, the Seeker of the Lost Wizard Traditions casts spells (including gaining any new spell slots and spell knowledge) as if he had also gained a level in a spellcasting class he had previous to gaining that level.}

\ability{Spell Reflection (Su):}{A Seeker of the Lost Wizard Traditions may reflect spells with a line area of effect off of walls. The spell may either bounce off at an appropriate angle (angle of incidence equals angle of refraction) or straight back towards the caster at his whim. Creatures whose spaces are entered twice by a bouncing spell effect are affected twice.}

\ability{Scroll Preparation:}{A Seeker of the Lost Wizard Traditions may prepare his daily spells from any magical writing that he has deciphered without harming himself or the magical writing. Many Seekers take magical scrolls and bind them together into book form because magical scrolls take up less room in a book than do normal pages of spell formulae.}

\ability{Uncapped Magic:}{At 2nd level, spells cast by a Seeker of the Lost Wizard Traditions do not have maximum level-dependent effects.}

\ability{Burst Conservancy (Su):}{At 3rd level, the spells cast by a Seeker of the Lost Wizard Traditions attempt to fill all available space. Every square that a spell with a burst area of effect is prevented from occupying because of a wall or similar obstruction is added to the other side of the effect's area. For example, a fireball takes up 44 squares when used without obstructions. When used in a long, 10' wide hallway by a Seeker of the Lost Wizard Traditions, the fireball would extend to be 110' long.}

\ability{Harvest Magic (Ex):}{A 4th level Seeker of the Lost Wizard Traditions can cut pieces out of recently killed monsters that are useful in item creation. An Aberration, Dragon, Magical Beast, Ooze, or Outsider that has been successfully identified with the appropriate knowledge skill by the Seeker of the Lost Wizard Traditions and killed within the last hour can have one of its organs harvested by the Seeker in a 10 minute procedure that preserves some of the magical power of the creature. The magical portions of such a creature are worth 50 gp and 2 XP towards item creation per CR of the monster.}

\ability{Expanse of the Sky (Su):}{At 5th level, a Seeker of the Lost Wizard Traditions may double the ranges and areas of his spell effects when he is outdoors. As long as the Seeker of the Lost Wizard Traditions has an open sky over his head, every 10' cube in a spell description is a 20' cube, every 30' cone is a 60' cone, and so on and so on. Essentially, all of his spells benefit from Widen and Enlarge Spell}

\ability{Temporary Portal (Su):}{When a 6th level Seeker of the Lost Wizard Traditions casts a [Teleportation] spell that would normally change his own location, he can create a portal from the target location to a location adjacent to himself instead of moving himself. This portal can be seen through and line of effect for spells can be drawn through it. The Seeker of the Lost Wizard Traditions may dismiss the portal at any time as a free action, and it otherwise lasts 1 round per caster level of the Seeker.}

\ability{Unbreachable Stone Defense (Su):}{When a 7th level Seeker of the Lost Wizard Traditions benefits from the spell \spell{stoneskin}, his damage reduction is increased to Unlimited/Adamantine. The hit point reserve of the stone skin is still only reduced by a maximum of 10 points per attack.}



\section{The Economicon} %: Making Sense of the Gold Standard
\vspace*{-10pt}
\quot{"100 pounds of gold for a house? How does anyone make rent without a wheelbarrow?"}

Since time immemorial, D\&D has used the "gold piece" as its primary currency. It is apparently a chunk of reasonably pure gold of vaguely standardized weight that people use fairly interchangeably in different cities populated by different species. In the bad old days, each gold coin was a tenth of a pound, which was hilarious and inane. In the current edition, each gold piece is a fiftieth of a pound. That's 3.43 gp to the Troy Ounce, which means that in the modern economy, each gp is about \$171 worth of gold. Obviously, gold is significantly more common in D\&D than it is on Earth, gold is also undervalued because its status as a currency standard drives it out of industrial uses and causes inflation. Further, populations in D\&D are orders of magnitude smaller than they are in the real world, so the gold per person is higher even with the same amount of gold. So the gold piece is massively less valuable in D\&D economies than it would be in Earth's economies.

Nonetheless, things are really expensive in D\&D, and the high price in gold means that there's a distinct limitation of how much wealth can be transported by any means available. The economies of currency transaction are actually so unfavorable that currency as we understand the term does not exist. Things don't have prices or costs -- all transactions are conducted in barter and a common medium of exchange is heavy lumps of precious metal.

\subsection{Wish and the Economy}

An Efreet can provide a wish for any magical item of 15,000 gp or less. A Balor can greater teleport at will, but can only carry 30 pounds of currency while doing so. Even in platinum pieces, that's 15,000 gp worth of metal. The long and the short of it is that at the upper end of the economy, currency has no particular purchasing power, and magic items of 15,000 gp value or less are viewed as wooden nickels at best. You can spend 15,000 gp and get magic items, but people in the know won't sell you a magic item worth 15,001 gp for money. That kind of item can only be bought for love. Or human souls. Or some other planar currency that is not replicable by chain binding a room full of Efreet to make in bulk.

Powerful characters actually can have bat caves that have sword racks literally covered in 15,000 gp magic items. It's not even a deal because they could just go home and slap some Efreet around and get some more. But even a single major magic item -- that's heavy stuff that such characters will notice. Those things don't come free with hope alone, and every archmage knows that.

\subsection{Wartime Economies Make for Shortages}

Many people wonder why a masterwork dagger goes for more than its weight in gold. That's a pretty valid question to ask; certainly I'm not going to attempt to justify the 600 gp price tag on a masterwork walking stick -- that's just an example of simplistic game mechanics run amok. But to an extent the crazy prices can be justified by the fact that every settlement in every D\&D world is on a war footing all the time. The idea that Peace is somehow a natural state is a fairly recent one, and based on the frequency of wars all over the world -- it's obviously just wishful thinking anyway. War is the default position of every major economy in the world, and that means that weapons have an immediate, and desperate, clientele. Iron is still relatively cheap, because you can't kill people with it right now, but actual weapons and armor are crazy expensive.

That doesn't explain the fact that the PHB charges you over a quarter Oz. of gold just to get a backpack, and it doesn't explain the fact that the markup on masterworking a buckler is the same as the markup on masterworking a breastplate -- that's just a game simplification that makes no real-world sense. But it's a start.

\subsection{Coins are Big and Heavy}
\vspace*{-8pt}
\quot{"How many boards could the Mongols hoard if the Mongol hordes got bored?"}

From the standpoint of the adventurer, the primary difficulty of the D\&D currency system is that the lack of a coherent banking and paper currency system means that there are profound limits to what you could possibly purchase even with platinum. But the currency system hurts on the other end as well. Untrained labor gets a silverpiece a week. That's 500 copper coins a year, which means that no matter how cheap things are they can only make one purchase a day most of the time. That's pretty stifling to the economy, in that however much gets produced, no one can buy it. Demand, from the economics standpoint, is strangled to the point where large production outputs don't even matter (remember that in economics Demand doesn't mean "what people want," it means "what people are willing and able to pay for," so if the average person only has 500 discreet pieces of currency per year, that puts an absolute cap on economic demand, even though the people are of course both needy and greedy enough to want anything you happen to produce).

What's worse, those coins are heavy. For our next demonstration, reach into your change drawer and fish out nine pennies. That's a decent lump in your pocket, neh? That's about one copper piece. Gold pieces are smaller (less than half the size, actually), but weigh the same. D\&D currency, therefore, is more like a Monopoly playing piece than it is like a modern or ancient coin. There's no reason to even believe these things are round, people are seriously marching around gold hats and silver dogs as the basic medium of exchange.

Now, you may ask yourself why these coins are so titanic compared to real coins. The answer is because having piles of coins is awesome. Dragons are supposed to sleep on that stuff, and that requires big piles of coins. Consider my own mattress, which is a "twin-size" (pretty reasonable for a single medium-size creature) and nearly .2 cubic meters. If it was made out of gold, it would be about 3.9 tonnes. That's about eighty-six hundred pounds, and even with the ginormous coins in D\&D, that's four hundred and thirty thousand gold pieces. In previous editions, that sort of thing was simply accepted and very powerful dragons really did have the millions of gold pieces -- which was actually fine. Since third edition, they've been trying to make gold actually equal character power, and the result has been that dragon hoards are\ldots\  really small. None of this "We need to get a wagon team to haul it all away," no. In 3rd edition, hoard sizes have become manageable, even ridiculously tiny. When a 6th level party defeats a powerful and wealthy monster, they can expect to find\ldots\  nearly a liter of gold. That is, the treasure "hoard" of that evil dragon you defeated will actually fit into an Evian bottle.

There are two ways to handle this:
\begin{enumerate}\itemspace
   \item Live with the fact that treasures are small and unexciting in modern D\&D.
   \item Live with the fact that characters who grab a realistic dragon's hoard become filthy stinking rich and this fundamentally changes the way they interact with society.
\end{enumerate}


But once you accept that the realities of the wish based economy, you actually don't have to live with characters unbalancing the game once they find a real mattress filled with gold. That's not even a problem once characters are no longer excited by a +2 enhancement bonus to a stat or a +3 enhancement bonus to Armor. Which means somewhere between 9th and 13th level it's perfectly fine for players to find actual money without unbalancing the game. Really, you can stop worrying about it.

\subsection{Bad Money Drives Out Good: The Penalties of Paper}

People from the modern world are generally pretty perplexed by this idea of handing back and forth actual metal as a medium of exchange. It is an undeniable truth in our lives that the idea of currency is just that: an idea. As long as whatever I'm trading for goods and services can be traded for goods and services, it doesn't actually matter if the exchange commodity has any ascribed intrinsic worth. Paper descriptions of value or even ephemeral electronic representations are not only adequate, they're convenient. But more than that, using valuable commodities as a medium of exchange inhibits the growth of the economy. As long as a certain portion of the wealth is locked up in currency, the economy is strangled coming and going: not only is there a completely arbitrary limit on how many goods and services can be exchanged (the gold supply), but there is also a limit on the kinds of industry and artistic expression that can occur (in that if you use gold for anything but currency you're actually shrinking the money supply and producing negative GDP).

So\ldots\  you're going to solve that by instituting a paper-based exchange system where initially the paper is exchangeable for gold and that eventually gets phased out when the Plebes realize that handing actual gold back and forth is inconvenient and dumb, right? Wrong. Remember that this is the Iron Age, and people haven't invented Nationalism yet. The cornerstone of the Greenback currency is a belief in the nation that prints it -- and nations simply don't exist. You've got empires, and you've got kingdoms, and you've got tribes, and you've got unincorporated villages\ldots\  and that's it as far as civilization goes. When you look at a map in D\&D and a colored region has a name on it, that's the name of the region. Possibly it's even the name of some guy in the region. The point is, that it's not a country in the modern sense of the word, so if some new guy walks in who's bad enough the next cartographer will put his name on the region instead.

And that means that "The Full Faith and Credit of the Kingdom of Daxall" is worth precisely nothing. And while King Daxall can, through force of arms, take all the gold away from all the peasants and get them to trade pieces of paper for goods and services in its place -- no one will actually believe that the paper is currency. They're literally trading promises by King Daxall that he'll let them have their money back if they leave town. And since the serfs can't even leave town, even that promise is meaningless to them. A serf accepts paper for goods and services only because he'll be beheaded if he doesn't. The black market value of these pieces of paper is pretty close to zero. Worse, nearby governments will see this as a blatant attempt to sequester all the gold in King Daxall's pants and will probably declare war (in addition to the fact that no one outside the reach of King Daxall's pikemen will accept Daxall Dollars).

\subsection{Powerful Creatures Have a Powerful Economy}

The amount of gold it takes to get anywhere as a land lord is very large. The question that arises then, is why awesome architecture exists at all. It's a valid question, the listed costs to put things like pit traps and thrones made of bone into your dungeon are stupendously large and actual magical swag can be made available for much less than that. The answer is that:

\begin{enumerate} \itemspace
   \item People don't actually pay all that gold to have their homes remodeled (see the peonomicon below).
   \item Powerful artificers and adventurers don't even want your gold. If something has a value of 100,000 gold pieces, it can't be purchased with gold pieces at all -- because that's an actual ton of gold that you'd have to plop over the counter and the merchant you're dealing with won't take your money even if you have it.
\end{enumerate}

Here we're going to be focusing in on

\listone
    \item Gems
    \item Souls
    \item Concentration
    \item Hope
    \item Raw Chaos
\end{list}

\subsubsection{Gems: Truth or Dare}

Gems are, to the vast majority of participants in the economy, pretty much worthless. A 500 gp diamond is pretty much the same as a gold piece to someone who intends to purchase things with a value of 1 gp or less. And of course, there are a lot more individuals out there who will stab a peasant in the face for a diamond than a gold piece. So why does anyone care?

Well, two reasons: the first is the obvious one that gold is extremely limited in what it can possibly purchase. A +2 sword is worth your weight in gold. Not its weight in gold, your weight in gold. It seriously costs over 166 pounds of gold, and that's just not reasonable for most people to put into their pockets. So people interacting with even the shallow end of the magic trade need there to be some crazily expensive items that have no purpose save to look pretty and be exchangeable for other stuff. But unlike our world gems actually have real value as well: as the fuel for powerful magics.

On Earth, the only reason that a diamond is expensive is because there's an international organization called DeBeers that seriously has actual assassins that will shoot you in the face if you try to sell diamonds for less than the price they've determined that they're supposed to be sold for. D\&D doesn't have that kind of armed monopoly to maintain gem prices, but it does have the fact that people continuously use up gems for spells like raise dead and item creation and the like. So the fact that you can use ruby dust to make continual flames that you can turn around and sell as Everburning Torches means that ruby dust will continue to have value as long as people value light.

The D\&D rules actually only go into the spell component uses of a handful of gems, but rest assured that all the rest are similarly useful when we get into the ephemerals of item creation. A lot of those "components" that cost piles of thousands of gold pieces are actually just piles of gems. Onyx keeps its value based on the needs of necromancers, but amethyst is just as needed to bind illusion magic into a cloak. The exchange rate between gems and magic items is in no danger of going anywhere. Minor magic items and gems are traded avidly by shopkeepers, adventurers, and even powerful outsiders and wizards.

But even so, gems can be simply acquired by the very powerful. The realities of the wish based economy ensure that gems can simply be obtained in large numbers by anyone who really cares enough to dedicate a conjured earth elemental to collecting them. Magical items that cannot be created with the application of spells (that is, magic items valued at more than 15,000 gp) cannot be purchased on the open market with mundane currency, not even gems. That isn't to say that you can't cheat a goblin out of a staff of power with some shiny rocks, you totally can (heck, you could also stab the goblin in the face and take that staff of power), but doing so is not considered a "fair trade" and requires a bluff check on your part.

In addition, many D\&D worlds posit the existence of magic gems, which can be used to make magic items, increase personal power, make a snazzy grill with the bottom row made of gold, and all kinds of stuff. In addition to getting hot women to ask you to smile, these magical gems are magical and are actually considered fair exchange in the near-epic economy. You can't wish for Eberron Dragonshards or Planescape Planar Pearls, so those things have real value to Efreet and other creatures participating in the Big Pond. Rules for using magic gems appear in the Tome of Tiamat.

\subsubsection{Magical Currency}

\listone

\itemability{Souls:}{The souls of powerful creatures are trapped in gems and the trade in them is brisk on the outer planes, especially in the planar metropolis of Finality on Acheron. Once a soul is in a gem, the gem itself is of little or no value, but the soul goes for 100 gp times the square of the CR of the creature whose soul is trapped (see Tome of Fiends for more information on the use of souls).}

\itemability{Concentration:}{Ideas take form on the outer planes, and really pernicious or stellar ideas can be so powerful that they take a while to form. In the before-time, they can be found as an amber-like substance that is extremely valued on Mechanus, and by extension every single other outer plane as well. Concentration is actually made out of ideas, and while it looks like a solid object it is actually a liquid that flows so slowly that you could watch it for a year and only a Modron could tell you have far the flow had taken it. A pound of concentration goes for 50,000 gp to an interested party, and can be used in magical crafting by those with the patience to learn its secrets (see Book of Gears for more information on the use of Concentration).}

\itemability{Hope:}{Hope is funny stuff, it has lots of inertia, but those who carry it are not weighed down in the least. It has mass, but not weight. Even the smallest piece of Hope sheds light like a daylight spell (the effective spell level for this effect is 7, and Hope can overcome almost any darkness). Hope is measured in kilograms rather than pounds, and a kilo of Hope goes for 100,000 gp to those who want it, and it can be used in magical crafting (see Tome of Virtue for more information on the use of Hope).}

\itemability{Raw Chaos:}{The plane of Limbo is filled with possibility and change. Usually this manifests as a continuous creation and destruction that is awe inspiring and terrifying at the same time. Sometimes, for whatever reason this possibility doesn't become anything, and just stays as Raw Chaos. Raw Chaos can have any dimensions and any amount of mass, but from a practical standpoint you either have it or you don't. If you have Raw Chaos and someone else doesn't you can give it to them, and it is generally considered good form for them to give you magical items or planar currency worth 200,000 gp in exchange. Raw Chaos can be transformed into magical items by those with the correct skills (See Tome of Tiamat for more information on the use of Raw Chaos).}
\end{list}

\subsection{The Service Economy: The Profession Rules Don't Work}

The profession rules make us sad. Very sad. Which is unfortunate, because almost everyone in the entire world who isn't an adventurer apparently lives and dies by these things. While the powerful adventurers go off into the planes and exchange Raw Chaos and the Souls of Champions for powerful magical items and favors, your average orc is running around delivering halfling food or joining the army of a powerful warlord for little bits of metal. When the players begin their adventuring careers, they'll be caught up in this economy as well. And even if they eventually become powerful enough to purchase mighty rods with planar currencies they might still be intimately involved in it -- as one of those mighty warlords who throws out tiny pieces of metal to orcish warriors and starting adventurers.

Here's the deal: if your character is a Sailor, that's character flavor. It's not a major portion of your character's power and we really are willing to just give it to you. Having a profession is like knowing a language: sometimes it will come up and sometimes it won't. In that spirit, we suggest that Profession cease being a ranked skill altogether. Just like people don't make "speak dwarvish" checks to have words come out of their mouth, characters should not have to make "Profession: Barkeep" checks to successfully sit behind a bench and hand people ale.

People who have a profession don't make checks to make money, they get a wage if they happen to have a job. The wage will depend on what kind of work they are doing (so no, you can't put 10 ranks into Profession: Janitor and be better paid than the magistrate). Characters are assumed to make a wage approximately similar to the one in the table below if they are working and have an appropriate professional skill. DMs may allow a character to put two ranks into a single Profession skill and be a "master whatever."  Such characters may be able to boast about their skills or perhaps even make more money. The important part is that this means that you can find really good scullery maids who don't have a +5 BAB. Young children can often be drafted to do grown-up jobs, and need only be paid 1/10th the normal rate for whatever it is that you have them doing. Child labor is cheap, but in some ways you get what you pay for and children may become distracted or sick before completing important or dangerous jobs.

\subsubsection{Professions and their Pay Scale}


\featnamelist{Profession Wage/Week}

\begin{multicols}{2}
\begin{small}
\listone
    \item Acolyte 5 GP$^\dagger$
    \item Alchemist 10 GP$^\dagger$
    \item Artisan 5 GP
    \item Bartender/Innkeeper 15 SP
    \item Barrister 8 GP
    \item Butler 2 GP
    \item Clerk 3 GP$^\dagger$ (includes more influential administrators)
    \item Cook 1 GP
    \item Courtesan 5 SP$^\dagger$
    \item Farmer 5 CP (Farmers also feed themselves)
    \item Fisherman 3 SP
    \item Groom 1 GP
    \item Guard 15 SP$^\dagger$
    \item Laborer 1 SP (note: this means no profession at all)
    \item Laborer, Skilled 2 GP
    \item Librarian 3 GP
    \item Janitor/Maid 8 SP
    \item Military Officer 5 GP$^\dagger$
    \item Miner 2 GP
    \item Porter 6 SP
    \item Runner 1 GP
    \item Sage 10 GP$^{\dagger \star}$
    \item Sailor 2 GP
    \item Scribe 2 GP
    \item Servant 8 SP
    \item Shepherd 2 SP
    \item Smith 15 GP
    \item Smith, Master 150 GP
    \item Soldier 15 SP$^\dagger$
    \item Tailor 1 GP
    \item Teamster 2 GP
    \item Torturer 2 GP
    \item Valet 15 SP
    \item Wage Mage 10 GP$^\dagger$
\end{list}

\noindent $^\dagger$: Some professons are actually dependent upon class level and abilities. A 1st level Wage Mage commands a wage of 10 GP a week to sit around and cast 1st level spells and cantrips from time to time, but a 12th level Wizard would command an earnings per week so large that most kingdoms find it more expedient to simply make such magicians part of the government.

\noindent $^\star$: Any skilled profession that is based on one of the ten Knowledge skills in D\&D is a Sage, and is not handled with the Profession skill at all. An Architect does not have "Profession: Sage," he has Knowledge: Architecture and Engineering. The pay scale of a Sage of any kind is extremely dependent upon his skill results. A character with four or five ranks in a couple of knowledges might pull down 10 GP per week, but a character who can regularly make a DC 30 check in any subject no matter how arcane can pull down the big bucks. Assuming of course that he can find someone that actually needs his services.
\end{small}
\end{multicols}


Just because you selected a profession that makes a lot of money doesn't mean that anyone will hire you. Generally only relatively organized areas actually have economies that even can hire Butlers and Clerks. But just because there is work available in an area doesn't mean that there's work available for you. Even in major cities there aren't a whole lot of jobs for a clerk or a barrister, so the competition for those jobs is pretty stiff. Prospective employers are fairly choosy about who they select for such employment, and they'll usually go to guilds (whose reputation is on the line every time they vouch for someone) or their own aristocratic family members rather than hire some random Half-Orc who claims to have the requisite skills.

\subsection{Running a Business}

The rules presented in the DMG2 for running a business make us very sad. Apparently the best way to make money is to run a shop out of a shack in the woods and pour money into it until noble djinni are teleporting to your door to hand over large gems for whatever the heck it is that you're selling. That doesn't make any kind of sense at all. We propose instead that the costs and benefits of running a business should be kind of comparable to those of working for a wage -- since it is essentially exactly the same thing. What we're looking for is rules for running a business that aren't so obviously abusable over time, and which reward various business models rather than finding the killer app that makes the most money (the Shop as it happens) and just using that over and over again.

\subsubsection{Capitalization}

First off: the thing where in the basic DMG2 rules you can capitalize over and over again forever and have the profits go off towards infinity is as abusable as it is dumb. So the very first change that needs to be made is the divine decree that you can't do that. In fact, the concept of recapitalizing just wasn't handled well there at all. It takes money to make money, but investment is not a ladder where you set money on fire until the pyre lights the heavens ablaze and gets you epic items in parcels like clockwork. Instead, starting a business venture costs money -- we call that initial capitalization. That's a one-time cost and the only way you can spend it again is if you start up a second business. After that, you have to supply one-third of the business' expected earnings for each month up front, we call that operational capital. If your business is still running at the end of the month, you get that money back (in addition to the earnings themselves), but if the business venture folds or you get driven off by rampaging monsters, or business events cause the venture to make no money for a month -- that operational capital is gone and you're out a pocket full of shells.

Initial capitalization isn't any cheaper in the wilderness than it is in a big city. Actually, it's more expensive because you have to get goods shipped out into the wilderness to get the whole thing off the ground -- and the wilderness in D\&D is dangerous and teamsters make 2 GP a week each in compensation for that fact. Operational capitalization is cheaper in the wilderness, because expected earnings are less and therefore 1/3 of those earnings is also less. Yes, this means that business owners normally go to the city to conduct business, where there is a whole governmental apparatus to facilitate business dealings and a steady parade of caravans and ships to bring your product or service to the world. The only reasons that anyone does their business outside of major cities is because some particularly risky ventures can only be done far from town (for example: a Larvae Orchard is a high-risk, and therefore high profit enterprise, but it can only be located in the Wastes of Hades).

\subsubsection{Risk}

Risky business ventures make more money. But they also suffer catastrophic mishaps more often. That's what makes them risky. They are not simply an increase to the multiplier on the profit check, because that just makes you more money because player characters don't start businesses that aren't going to have positive profit checks. Maintaining a Risky venture involves you having more challenges to maintain your business -- which in a roleplaying game like D\&D means essentially that you spend more adventures maintaining your business and therefore spend less adventures looting other peoples' dungeons. The extra profit you make from the risky business is offset by the extra challenges you need to overcome. Essentially, taking on a risky business is just like getting the gold from your encounters before you go adventuring.

Risky businesses have a CR and a frequency. The DM is encouraged to send additional problems your way at roughly the frequency of the risk factor, and the ELs of the problems thrown your way should be roughly the same as the CR of the risk factor. Risky businesses also make a lot more money -- roughly the value of an "average" treasure of an encounter of a CR equal to the risk factor every interval of time equal to the risk factor (see the DMG, p. 51). So an onyx mine that had a risk factor of 5/4 months would generate an extra 400gp per month (1600 gp/4) and be plagued with an EL 5 encounter roughly 3 times a year. It's just that easy.

Not all shops are the same. If you're selling burlap clothing, the profits are going to be small and ogre bandits won't even try to take all your stuff. If you're selling weapons of war or magical materials, then you can bet that those ogre mercenaries are going to be a little bit more interested. If you're running a more valuable business (that is, one which makes more money), the villains of the D\&D world will come to take it from you -- the risk factors adjust themselves pretty much automatically when your business improves, making this approximation amazingly accurate in addition to simple.

\subsubsection{Resources}

Resources are like Capitalization that you get to keep. While the presentation in the DMG2 is essentially "something that makes it harder to turn a profit on your business," the fact is that what they actually are is your own private dungeon. While the full rules for actually building your dungeon are going to have to wait until Book of Gears and the advanced crafting rules, for now we're going to assume that the prices in the Stronghold Builder's Guidebook hold up (and yes, we know how silly that is, but we haven't written anything better yet). Essentially, this means that your business needs to be housed in a building, or ship, or cart, or dungeon of some kind. Bigger, more high-scale business ventures are going to need to be housed in more expensive surroundings. That sounds bad, but remember that when business events and risk factors happen to your business, they happen to your business, which means that if you have a ship or a tower to hold your stuff in, you actually get to use it when it gets attacked by gnoll pirates.

Keep in mind that if a business is booming, it may require more resources to house. A shack is all well and good if you plan to sell a couple of pots a month, but if you want to move inventory you've got to have inventory. And that means you need a place to show that inventory. Practically, that means that your projected profits (before calculating Risk-based Profits), can't ever exceed 1/10th the value of your business' resources. Of course, some businesses can only exist with large amounts of resources backing them up. And that's fine, since you really only get the benefits of large resources in large urban areas, this means that in general there are a lot of services that can be found in the big city that can't be found in smaller towns. Which is exactly what you'd want, right?

\subsubsection{Growing the Business}

Characters may outgrow collecting melloweed from the Bane Mires. The occasional hydra they have to defeat to get the goods just doesn't challenge them anymore, and the gold the whole thing takes in every month just doesn't seem worth the hassle. When this happens there are two options: franchise the operation, or grow the business up. A business can be expanded to a larger operation by investing in the next level of resources (causing it to be eligible to make more profits), or by taking on higher value/risk goods and clients (causing the risk factor to increase and profits to increase as well).

Franchising a business simply involves starting up a second (or third) business in another location. Resolve it as a whole new business.

\subsubsection{Profits}

So how much money do these things make? Well, in addition to Resource Limitations, there are demand limitations. That is, the amount of money that people can spend on your goods and services is proportional to how much money they have -- larger communities can spend more money than can smaller communities. The maximum profits per month of any venture are based on the total population that business serves. If you compete with other businesses providing the same goods and services, simply divide the region's population according to market share before you determine maximum profits.

\featnamelist{Population Size / Gold per Month}
\listone
    \item 20-80 -- 4 GP/month
    \item 81-400 -- 10 GP per Month
    \item 401-900 -- 20 GP per Month
    \item 901-2000 -- 80 GP per Month
    \item 2001-5000 -- 300 GP per Month
    \item 5001-12,000 -- 1,500 GP per Month
    \item 12,001-25,000 -- 4,000 GP per Month
    \item 25,001-100,000 -- 10,000 GP per Month
    \item 100,001+ -- 60,000 GP per Month
\end{list}

\vspace*{8pt}

Remember that while this determines the maximum profits, there's no guaranty that your business will actually do as hoped. Things don't always work out as planned, and many business plans aren't good. In order to make your business succeed, you'll have to make a Profit Check. Actually making the projected Profits is a DC 20 check. Every point you fail that DC, reduce your income by 5\%. For every point you exceed 20 on your Profit Check, add 5\% (essentially this just means that you make a 5\% return for every point of Profit Check you make).

The Profit Check itself is simply a straight ability check, using your choice of your Intelligence, Wisdom, or Charisma. Some of the modifiers to Profit Checks from the DMG2 are appropriate, others are not. For your convenience, we're replicating the entire chart with all the needed modifications:

\listone
    \item Owner has appropriate Profession Skill +1
    \item Owner has two appropriate Profession skills +2
    \item Owner is a member of an associated guild +1
    \item Owner spends less than 8 hours per week assisting business operations -8
    \item Owner spends more than 40 hours per week assisting business +1
    \item Business is considered a Monopoly +10
    \item Business is an Oligarchy +4
    \item A Business Partner aids during the term +2
    \item Specialists are on staff +2
    \item Previous Profit Checks "Failed" -1 per consecutive check below 15.
\end{list}


\subsubsection{Command Economies}

Sometimes your "business" is actually just that you run a country, or a guild, or a church, or a criminal organization, or a mercenary command. Or whatever. The point is that your job is to run things, and people pay taxes (or tithes, or protection money, or whatever the kids are calling it these days) to you to make sure that you keep running things in a manner that doesn't involve them being stabbed in the face. The amount of lucre you can squeeze out of these situations has nothing to do with your skill checks or capitalization -- you're essentially stealing from these people so the amount of money you can crank out of them depends largely on how much you're willing to squeeze them and how many people you are squeezing. Taxing a group of people can generate as much money as running a business serving them would. Your "business" in this case is "not stabbing them in the face."

You can be senselessly wicked and punitive on a population and make twice as much gold, but your subjects will hate you. You can also simply sack a region, making ten times as much gold, but driving the remaining population away as refugees. Lawful creatures (such as Hobgoblins and Dwarves) are more likely to pay taxes or save money and taxing or looting them is worth twice as much. Especially impoverished regions (such as one which has labored under a cruel governor for a long time) are worth half as much or less.

\subsection{Bringing the World out of the Dark Ages}

It is historical fact that you can take a ridiculous and crumbling imperium with serfs and horse-drawn carts managed by a tyrannical and squabbling aristocracy and boot strap it into being a technologically sophisticated global power that can win the space race and such in a single generation even while being invaded by an evil and genocidal empire. The people at the top don't even need to be nice or sane, they just have to understand that economics is an entirely voodoo science, and the limits of production can be broken by thousands of percentage points by getting everyone to buy on credit, work on projects that people looking at the big picture tell them to work on, continuously invest in productive capital, and believe in the future.

Right. That's called Communism, and it ends the dark ages immediately even if it isn't run well. Presumably if it was being run by Paladins who actually radiate goodness and Wizards who are inhumanly intelligent and can cast powerful divinations to determine projected needs and goods could be distributed to the masses with teleportals -- it would work substantially better. That sort of thing is not outside the capabilities of your characters in D\&D. It's not outside the capabilities of the people in the village your characters are saving from gnollish invasion. It's not even technically complicated. But it isn't done.

Partly it isn't done because we're playing Dungeons \& Dragons, not Logistics \& Dragons. While it is true that you can fix the world's ills in a much more tangible fashion by industrializing the production of grain and arranging a non-gold based distribution system such that staple food stuffs are available to all, thereby freeing up potential productive labor for use in blah blah blah\ldots\  the fact is that to a very real degree we play this game because telling stories about slaying evil necromancers and swinging on chandeliers is awesome. But the other reason is that the society in D\&D really isn't ready for a modern or futuristic social setup. No one is going to understand how they are supposed to interact with Socialism, Capitalism, or Fascism, things are Feudal and people understand that. Wealth is exchanged for goods and services on the grounds that people on both sides of the exchange aren't sure that they would win the resulting combat if they tried to take the goods or wealth by force of arms.

Rome had steam engines. Actual difference engines that propelled a metal device with the power of a combustion reaction through the medium of the expansion of heated water. Really. They never built rail roads because slaves were cheaper than donkeys and the concept of investing in labor saving devices was preposterous. In D\&D, the idea of having an economy based around trust in the government and labor/wealth equivalencies is similarly preposterous. It's not that the idea wouldn't work, it's that every man, woman, and child in society would simply laugh you out of the room if you tried to explain it to them.


\chapter{The Maginomicon}
\vspace*{-10pt}
\quot{``With great powers come laser eyebeams."}

\section{Easter Egg Class Features: Artifact Swords and Powergloves}

Here's a secret: some characters really can't even play the game at high level. But they do anyway, all the time. Sometimes the players never even realize that their character has no intrinsic capability to play the game at the level he's competing at. And that's because of two things: DMs control the Monsters, and DMs control the Treasure. It is our hope that the Monks and Assassins in this document will be able to hold their own without needing to get Power Gloves that act as magic weapons for their natural weapons or anything else really cheesy like that. That being said, we still haven't covered everything:

\listone
    \item Rogues still need a magical object that allows them to use the Hide skill by about level 9.
    \item Fighters still need their artifact swords at level 10.
    \item Bards still need some completely arbitrary magic item that summons a monster or something so that they can contribute at all past level 12.
    \item Mounted characters need a magical beast or dragon to ride around on by level 7.
\end{list}


And so on. As this series continues, we will attempt to solve some of these outstanding issues.

\section{Your Money is No Good Here}

As described in the Economicon, you can't just throw a walrus' weight in gold on the table and get powerful artifacts in return. You can get powerful magical items in exchange for rare planar currency, but you can only do that in a few planar locations. From the standpoint of the DM this is very convenient, because it means that you can hand out all the opal statues you want without worrying that the players are going to pool it all and get some totally hardcore magic items that will undermine everything. At the same time, it means that you can hand out planar currency and know for a fact that it's going to be used for powerful magical items.

\section{It's not Stupid, it's Advanced!}

The 15,000 gp limit for purchasing equipment can be pretty limiting, but the game works much better once you realize that it's there. Still, while characters can't go out and buy a +4 sword with pieces of gold (all 647 pounds of it), they can purchase a +1 flaming or ghost touch sword with chunks of non-magic metal. You can also pump those up with greater magic weapon to be something level appropriate. This offends some people, but it really is part of the way the D\&D magic item economy is supposed to work. People are supposed to be fighting with weapons that are level appropriate, and people are supposed to be purchasing new weapons for different occasions, and there are not supposed to be stores with racks of powerful swords that would be level appropriate for 12th level characters stacked up in various setups on shelves.

\listone
    \itemability{Bonus Rule:}{The game actually works better if every character of 6th level or higher simply has \spell{greater magic weapon} 1/day as a spell-like ability. Caster level is equal to character level. Try it, it's amazing how many problems are solved by this relatively simple change.}
\end{list}


\section{Material Components: A Joke Gone Way Out of Hand}

Material components are a joke. I'm not saying that they are metaphorically a joke in that they don't act as a consistent or adequate limiting factor to spellcasting, I mean that they are actually a joke. Material components are supposed to be ``ha ha" funny. The fact that even after having this brought to your attention, you still aren't laughing, indicates that this is a failed attempt at humor. Most material components are based on technological gags, when you cast scrying you are literally supposed to grab yourself a ``specially treated" mirror, some wire, and some lemons -- which is to say that you make a TV set to watch your target on and then power it with an archaic battery. When you cast see invisibility you literally blow talc all over the place -- which of course reveals invisible foes. Casting lightning bolt requires you to generate a static charge with an amber rod and some fur, tongues requires that you build a little Tower of Babel, and of course fireball requires that you whip up some actual gunpowder. Get it? You're making the effects MacGuyver style and then claiming that it's ``magic" after the fact. Are you laughing yet?

Of course not, because that joke is incredibly lame and there's no way for it to hold your attention for several months of a continuous campaign.

\section{Some Spells Don't Work}

Many spells are underwhelming for their level or have mechanics that are hard to explain. But first and foremost of all the spells that are bad for the game is Polymorph. That spell is integral to any fantasy setting, but people haven't made it work in 3rd edition. Mostly, this is because people keep writing it long instead of short. Remember, if you can't explain an effect in 2 minutes, everyone else is already confused.

\subsection{Polymorph Version 1: Character Replacement}

If you take part of your character -- any part of your character -- and part of a monster from one of the many monster books in D\&D, and you put them together into a single Voltron-like body, you have broken D\&D. That should be obvious, but since we are over six years into the ridiculous circus that is polymorph in 3rd edition Dungeons and Dragons, apparently it isn't. If it is important to you that you be allowed to dumpster dive through the monster books and find an appropriate to transform into, it is important to D\&D that absolutely no part of your character be mixed and matched during that period. If you want to truly become a monster, you have to actually become that monster. Not ``the monster with all my spell effects running", not ``the monster with my formidable mental attributes. No. You need to become the monster exactly as it appears in the monster book or there's no chance of you getting a balanced result. Some people are going to end up as mediocre monsters with carry-over abilities that happen to synergize well and become tremendously powerful while other people are just as unbalanced in the other direction when they find that drawbacks of their character are carried over and overwrite the abilities of a monster that are supposed to make them any good at all.

And this isn't just hyperbole or doomsday predictions, this is established fact. We've all played with some of the multitude of different versions of Polymorph errata and ``fixes", and the abject horror caused by every single iteration. The idea doesn't work. If you're going to replace any part of the character, you have to replace it all. So here's a version of polymorph that won't make us cry. This ain't rocket science, it just takes a little bit of discipline:

\begin{quote}
\featname{Polymorph Self}
\begin{small}
\shortability{Transmutation}{}
\shortability{Level:}{Sor/Wiz 4}
\shortability{Components:}{V, S}
\shortability{Casting Time:}{1 Standard Action}
\shortability{Range:}{Self}
\shortability{Duration:}{10 minutes/level (D)}
\shortability{Saving Throw:}{Fortitude Negates (Harmless)}
\shortability{Spell Resistance:}{No}
\end{small}
\quot{``A Turtle am I? Let's see how Turtlike I\ldots\  CAN\ldots\  BE!'' And with that, the mage was a giant turtle.}

You vanish and a monster of your choice appears in your place. The creature shares your alignment, personality and goals, and will continue to act as you would within the limits of its intelligence and abilities. The creature must be at least 3 CR less than your character level, may not have the incorporeal or swarm subtype, and is unexceptional for its type. If the monster is killed, the spell is ended. When the spell ends, the monster vanishes and you appear where the monster was with an amount of lethal, nonlethal, and ability damage on you equal to the amount the monster had suffered when the spell ended (this means that if the spell ended because the monster was slain and the monster had an equal or greater number of hit points as you, you may well be dead when you appear).
\end{quote}

\begin{quote}
\featname{Polymorph Other}
\begin{small}
\shortability{Transmutation}{}
\shortability{Level:}{Sor/Wiz 4}
\shortability{Components:}{V, S}
\shortability{Casting Time:}{1 Standard Action}
\shortability{Range:}{Medium}
\shortability{Target:}{One Creature}
\shortability{Duration:}{Permanent (D)}
\shortability{Saving Throw:}{Fortitude Negates}
\shortability{Spell Resistance:}{Yes}
\end{small}
\quot{The witch snarled at the trespasser and pointed her wand vindictively at him. A short incantation later left nothing but a pig in his place.}

Your target vanishes and a creature of your choice appears in its place. The creature shares the alignment, personality and goals of the target, and will continue to act as it would within the limits of its intelligence and abilities. The creature must be at least 5 CR less than your character level, may not have the incorporeal or swarm subtype, and is unexceptional for its type. If the creature is killed, the spell is ended. When the spell ends, the creature vanishes and the target appears where the creature was with an amount of lethal, nonlethal, and ability damage on it equal to the amount the creature had suffered when the spell ended (this means that if the spell ended because the creature was slain and the creature had an equal or greater number of hit points as the original target, it may well be dead when it appears).

\end{quote}


\begin{quote}
\featname{Mass Polymorph}
\begin{small}
\shortability{Transmutation}{}
\shortability{Level:}{Sor/Wiz 7}
\shortability{Components:}{V, S}
\shortability{Casting Time:}{1 Standard Action}
\shortability{Range:}{Medium}
\shortability{Target:}{Any number of creatures within a 20' radius}
\shortability{Duration:}{Permanent (D)}
\shortability{Saving Throw:}{Fortitude Negates}
\shortability{Spell Resistance:}{Yes}
\end{small}
\quot{The crowd looked uncomfortable. They had weapons and were brandishing them in a fashion quite menacing. But the magician was laughing, and that really put a damper on the mood of the entire event. They started to regain their composure and again advance upon him. He snorted and muttered an incantation, and something about swine \ldots}

Each target vanishes and creature of your choice appears in its place. The creatures share the alignment, personality and goals of the targets, and will continue to act as they would within the limits of their intelligence and abilities. The creatures must be at least 7 CR less than your character level, need not be the same for all targets, none may have the incorporeal or swarm subtype, and all are unexceptional for their type. If a creature is killed, the spell is ended for that target only. When the spell ends, the creatures vanish and the targets appear where the creatures were with an amount of lethal, nonlethal, and ability damage on it equal to the amount the creature had suffered when the spell ended (this means that if the spell ended for a target because the creature was slain and the creature had an equal or greater number of hit points as the original target, it may well be dead when it appears).

\end{quote}

\subsection{Polymorph Version 2: Fixed Forms}

The other version is one where transforming leaves you essentially yourself, only with a new hairdo and possibly some bonuses. In this case, you keep everything about yourself and simply get a disguise and some advantages consistent with a buff spell. All of the ``Whatever-Form" spells don't stack with multiple castings or even with each other, because they are considered to be ``one spell makes another spell irrelevant" for purposes of spell stacking.


\begin{quote}\featname{Human Form}
\begin{small}
\shortability{Transmutation}{}
\shortability{Level:}{Brd 1; Sor/Wiz 2}
\shortability{Components:}{V, S}
\shortability{Casting Time:}{1 Standard Action}
\shortability{Range:}{Touch}
\shortability{Target:}{One Willing Creature}
\shortability{Duration:}{10 minutes/level}
\shortability{Saving Throw:}{Fortitude Negates (Harmless)}
\shortability{Spell Resistance:}{Yes}
\end{small}
\quot{The man looked at the fallen prince and smiled. He whispered some eldritch words, and then there were two princes. One living, and one dead. The living prince smiled.}

The target assumes the appearance of a specific individual of medium size or smaller, or of a generic member of a humanoid race. The target is effectively disguised, and gains a +10 bonus on Disguise checks made to impersonate the genuine article. The target suffers no penalties to Disguise for assuming the visage of a different race or sex.
\end{quote}


\begin{quote}
\featname{Lycanthropy}
\begin{small}
\shortability{Transmutation}{}
\shortability{Level:}{Sor/Wiz 3}
\shortability{Components:}{V, S}
\shortability{Casting Time:}{1 Standard Action}
\shortability{Range:}{Touch}
\shortability{Target:}{One Willing Creature}
\shortability{Duration:}{10 minutes/level}
\shortability{Saving Throw:}{Fortitude Negates (Harmless)}
\shortability{Spell Resistance:}{Yes}
\end{small}
\quot{The shaman howled in rage and transformed into a wolverine.}

The target assumes the appearance of a specific or generic animal or magical beast of small, medium, or large size. The target is effectively disguised, and gains a +10 bonus on Disguise checks made to impersonate the genuine article. The target suffers no penalties to Disguise for assuming the visage of a different race or sex. The new form is unable to use normal equipment (all carried or worn items meld into the new form when the spell takes effect), and has whatever natural weapons the caster desires (to a maximum of 1 natural weapon per four levels). These natural weapons inflict an amount of damage appropriate for a magical beast of the new form's size. Any equipment the character had is subsumed into their new form.
\listone
    \item Small, Flying:90' flight speed (good), +4 Dex, -4 strength
    \item Small, Land:+2 Dex
    \item Small, Swimming:60' swim speed
    \item Medium, Flying:60' flight speed (good), +2 Dex
    \item Medium, Land:40' land speed, +2 Strength, +2 Natural Armor
    \item Medium, Swimming: 60' swim speed, +2 Strength, +2 Natural Armor
    \item Large, Flying: 90' flight speed (average), +2 Dex, +4 strength, +1 Natural Armor
    \item Large, Land: +6 Strength, +5 Natural Armor
    \item Large, Swimming: 60' swim speed, +6 Strength, +4 Natural Armor
\end{list}
\end{quote}


\begin{quote}
\featname{Monstrous Form}
\begin{small}
\shortability{Transmutation}{}
\shortability{Level:}{Sor/Wiz 4}
\shortability{Components:}{V, S}
\shortability{Casting Time:}{1 Standard Action}
\shortability{Range:}{Touch}
\shortability{Target:}{One Willing Creature}
\shortability{Duration:}{10 minutes/level (D)}
\shortability{Saving Throw:}{Fortitude Negates (Harmless)}
\shortability{Spell Resistance:}{Yes}
\end{small}
\quot{With a sweep of your cloak you become a creature of nightmare.}

The target assumes a horrific and monstrous countenance of a monster of Medium, Large, or Huge Size. The basic structure can look like pretty much anything, and the descriptions are just guidelines. All of the character's equipment melds into his new form. The character no longer has the ability to use equipment, but has a number of natural weapons appropriate to the new form:

\listone
    \item Yeth Hound (Medium):50' speed, +4 Str, +4 Dex, Bite, Improved Trip
    \item Displacer Beast (Large): +8 Str, +2 Dex, +5 Natural Armor, 1 Primary Bite and 2 secondary Tentacle Whips, Concealment.
    \item Monstrous Spider (Large): 30' Climb Speed, +8 Str, +8 Natural Armor Bonus, 1 natural weapon Bite, Poison (1d6 Con/ 1d6 Con)
    \item Chuul (Large): 60' Swim Speed, +8 Str, +6 Natural Armor, 2 Primary Pinchers, character gains the [Aquatic] Subtype.
    \item Bulette (Large): 20' Burrow Speed, +8 Str, +10 Natural Armor
    \item Manticore (Large): 60' Fly Speed (Average) +8 Str, +6 Natural Armor, 2 natural weapon Claws, 2 natural weapon ranged spikes attacks (1d8 + Str, 19-20 crit, 20' range increment)
    \item Giant Serpent (Huge): +14 Str, +10 Natural Armor Bonus, 1 natural weapon bite, Poison (1d6 Dex damage/1d6 Dex damage), Improved Grab.
\end{list}
\end{quote}


\begin{quote}\featname{Fiend Form}
\begin{small}
\shortability{Transmutation}{}
\shortability{Level:}{Sor/Wiz 5}
\shortability{Components:}{V, S}
\shortability{Casting Time:}{1 Standard Action}
\shortability{Range:}{Touch}
\shortability{Target:}{One Willing Creature}
\shortability{Duration:}{10 minutes/level (D)}
\shortability{Saving Throw:}{Fortitude Negates (Harmless)}
\shortability{Spell Resistance:}{Yes}
\end{small}
\quot{With a foul guttural utterance and a rude gesture, the wizard transforms into a fiend from the lower planes.}

The target assumes the appearance of a specific individual of medium size or smaller, or of a generic member of a fiendish race. The target is effectively disguised, and gains a +10 bonus on Disguise checks made to impersonate the genuine article. The target suffers no penalties to Disguise for assuming the visage of a different race or sex. While in Fiendish form, the target gains two bonus [Fiend] feats of your choice that it would meet the requirements for if it was actually a member of a fiendish race, and gains access to a sphere of your choice. In order to use a spell-like ability from the sphere, the target must expend one spell-slot or prepared spell of an equal or greater spell-level, but there is no other limit to how many times the spell-like abilities can be used. Rules for [Fiend] feats and spheres may be found in the Tome of Fiends.
\end{quote}


\begin{quote}\featname{Dragon Form}
\begin{small}
\shortability{Transmutation}{}
\shortability{Level:}{Sor/Wiz 6}
\shortability{Components:}{V, S}
\shortability{Casting Time:}{1 Standard Action}
\shortability{Range:}{Touch}
\shortability{Target:}{One Willing Creature}
\shortability{Duration:}{10 minutes/level (D)}
\shortability{Saving Throw:}{Fortitude Negates (Harmless)}
\shortability{Spell Resistance:}{Yes}
\end{small}
\quot{The final incantations are completed and you transform into a dragon.}

The target character assumes the form of a huge dragon. The character gets a +14 Strength bonus and a -4 Dexterity penalty. The character gains a +18 Natural Armor Bonus. The character gains immunity to one energy type (which must be Acid, Cold, Electricity, Fire, or Poison), and a breath weapon that inflicts 1d6 per level of the same type of damage. Using the breath weapon is a supernatural ability that requires a standard action and may only be used at most once every 1d4+1 rounds. The character has a flight speed of 120' with poor maneuverability. A character in Dragon Form has three natural attacks: a primary Bite and two secondary claws. Worn equipment is subsumed into the new draconic form.
\end{quote}

%\end{small}
%\end{multicols}

\section{Some Effects Don't work}

\subsection{Stacking Spell Resistance}

Spell Resistance does stack, but it does so in a really weird way that the authors have never actually taken the time to explain. SR is a DC for a level check, and that means that it is actually calculated as the number people are supposed to roll to penetrate your SR plus your CR. A SR of 15 on a CR 1 monster is awesome (it means that people of your level are supposed to roll a 14 to penetrate your SR), while a SR of 15 for a CR 13 monster is a joke (it means that enemies fail to penetrate your SR only on a natural 1). When you have Spell Resistance, and your CR goes up, your Spell Resistance also goes up. An Imp with a CR of 2 and a SR of 6 who takes enough levels of Wizard to gain a CR (2 levels as it happens) gains 1 SR as well and is SR 7.

But what happens if more than one source gives you SR? Well, it still stacks, it just does so in the aforementioned really weird way. First, you take your highest SR, then you start adding very small numbers to it based on what your other sources of SR would give you. If a secondary source of SR is less than 6 + your CR, having it increases your SR by +1. If a secondary source of SR is 6 + your CR or more, but less than 11 + our CR, your primary SR increases by +2. If a secondary source of SR is between 11 + CR and 15 + CR, it increases the primary SR by +3. And finally, a secondary source of SR that is 16 + CR or more adds +4 to the primary SR.

It would be nice if the basic rules ever explained that, but they don't. It doesn't come up all that often, but Drow Monks, for example, don't end up with SR in the high 30s at mid level. Their SR is actually just moderately impressive.

\subsection{Hiding in 3.5 D\&D is Dumb}

OK, we all know that it makes us feel kind of bad when the Rogue sneaks up on people and stabs them in the face without them ever seeing who did it. But you know what? People totally do that crap all the time. It's not even an uncommon occurrence, and there's really no cause to get excited about. The 3.5 rules for hiding, where you need cover or concealment to hide, are retarded. That makes Rogues run around with tower shields so that they can hide themselves and their equipment behind the cover of the tower shield (including the tower shield itself, which makes my brain hurt). Yes, you can totally hide when there are no intervening objects between you and the victim. It's called ``sneaking up behind people" and in a game with no facing it's handled with a hide check opposed by spot.\\

If you attempt to hide in a combat setting, you are under a number of restrictions:
\listone
    \item A character who has been attacked automatically can guess what square you are in. You may retain your invisibility, but that's just Full Concealment, and they could very plausibly hit you.
    \item There is a -20 penalty to Hide for attempting to fight while hidden. The distance penalties on Spot are pretty amazing, but most people can't hide at a -20 penalty.
    \item Once they see you, they see you. If an opponent successfully spots you even once (and they get to try every round while in combat), they just plain see you until you manage to get all the way out of their field of view (generally requiring you to leave the scene or make bluff checks or something).
    \item Spot Bonuses can get quite large. A spotter who knows what he's looking for gets a +4 bonus, and a spotter who is extremely familiar with the target gets a +10 bonus -- these bonuses are weirdly listed under the Disguise skill, but they still apply (so if someone says ``There's a halfling Ninja over there!" every other Guard gets a +4 bonus).
\end{list}

\vspace*{8pt}

But you can do it. Hiding in combat is hard, but it's a thing that powerful characters may be able to do against some opponents. Some of the D\&D authors have an outdated idea that Rogues should be forced to ``hide in shadows" or something. But this is D\&D, and most enemies have Darkvision. There are no shadows. Attempting to force Rogues to hide only in areas that they could plausibly hide in if a suspicious person was looking right at them and knew what they were looking for is incredibly cruel. In any kind of stressful situation that isn't an accurate picture of what is going on.

\subsection{Clerics and Druids get Broken with Supplements}

Sometimes it seems that WotC authors can't even write a supplement without writing a new Cleric spell. Unfortunately, that drives Clerics straight into crazy town because they actually know every spell on their list. So if someone writes 5 new cleric spells for a minor adventure, that's five new options that every Cleric player has for no reason. That has to stop.

Characters like Clerics and Druids are, with few exceptions (*cough*divine power*cough*) pretty much OK with the spells in the Player's Handbook. It's only when we mix in all the crazy options in additional sources that they go over-the-top. It is our contention, then, that such characters continue to be allowed access to all spells in the PHB -- and to only get one bonus known spell from other sourcebooks each level (choose wisely). In this manner, the Clerics and druids of the world will end up having a couple of specific gimmicks, and they won't all just be cookie-cutter copies of each other with an answer for every occasion. Thereafter, such characters could potentially find magical writings with new spells in their discipline that they could learn and use in the same manner as a Wizard. I have nothing against a Druid finding a copy of some ancient text that allows her to call upon the legendary bloodsnow, but it's pretty ridiculous the way in the current rules every Druid can get up one day and decide to have an explosion of bloodsnow.


\chapter{Life Under Ground}

\section{The Thermodynaminomicon: The G of Life Continues}
\vspace*{-10pt}
\quot{``Seriously, this cave is a mile below the surface, what do they eat?"}

When you drop an egg on the ground, it breaks. But when you drop a broken egg it doesn't reform a perfect shell. That's entropy baby, the simple fact that it requires energy inputs to maintain an open system, and a closed system only degrades over time. If there is to be Life, let alone civilization, there has to be some way of getting energy into the system. For the surface worlders, that's not even a problem: the Sun shines energy down on the surface all day. But for those who live in the dark realms: be it sewers, the ocean floor, or the classic dungeon complex; there has got to be a source of energy in the D\&D world that just doesn't exist in ours. You don't see lush forests in marine trenches because the energy inputs just aren't there. But in D\&D neither the ocean floor not the underdark is a desert -- it's a vibrant ecology. Here's why:

%\vspace*{-\lineskip}
\subsection{Those whacky Mushrooms}

Ask any DM what people eat down there in the Underdark and they'll probably say ``Mushrooms" because it is a well known fact that mushrooms are a fungus and not a plant and they don't need sunlight to grow. What they do need, however, is a source of chemical energy. That can be dead bodies or um\ldots\  otherwise digested material, sure, but it still has to come from somewhere. When fungus grabs some organic material and converts it into more fungus, that's an inefficient process. You actually have less energy worth of fungus than you had energy worth of whatever it was that the fungus was eating.

Don't get me wrong -- edible fungus will happily consume things that are inedible (like woodchips) or even poisonous (like waste) and turn it into something you can eat. It just won't make something out of nothing. So mushrooms are an excuse for why Underdark dwellers have something to eat if and only if there is some other way that energy is coming into the system. It could be anything really, just as the trees will turn useless sunlight into tasty peaches, the fungus will turn useless chemical waste into delicious mushrooms. But while mushrooms can handwave the problem of converting energy that you can't use into energy that you can, they don't explain how that energy gets in to the Underdark in the first place.

\subsection{Portals}

The Inner and Outer Planes are infinite in scope, and don't follow the same thermodynamic constraints as Prime worlds do. A portal to the plane of Fire simply heaps energy into whatever cavern it happens to be in, which is like having a little sun right in your closet. Plants and algae can use that energy input to grow, and ooze monsters can eat those plant and civilizations can eat the ooze monsters. As long as those portals stay open, energy can keep entering the system even underground.

\subsection{Magic Deposits}

Magical locations exist all over the place in the D\&D world. Some of them are on the surface, and these become fairy rings or magical castles, or whatever. The point is, almost all of them are put to use, because even the ones in deadly jungles or on top of treacherous mountains simply are not hard to get to. One teleport and you're there. But the world isn't flat (unless you're in Bytopia or the Abyss), and actually there's a lot more volume of the planet that's under the ground than there is on the surface. And that means that the vast majority of magical locations are underground somewhere.

If you consider that in the D\&D world, magic is approximately as important as the Sun is to life on the planet, that means that a significant amount of the total energy inputs in the system are underground to the exclusion of being above ground. The underdark has irregularly spaced Nausica-style gardens down there that are supported by magic upwellings. Each of these locations is massively more productive than a field or forest above ground of similar size, and the immutable fact that the surrounding territory is lifeless barren stone causes fights over these locations to be extremely brutal.

\subsection{Subduction}

The world above is filled with living things, and even in D\&D the vast majority of living things die. When living things die, they leave behind a body that other living things can eat. Some of this actually gets eaten, while other bodies end up sinking into the land. In our world, the organic material sucked into the ground eventually becomes fossil fuels, but in D\&D there's actually stuff down there that will eat it. This is a relatively minor input into the dungeon ecosystem, but it essentially means that there aren't any oil deposits to be found in the D\&D world.

\section{The Bionomicon} %: Biodiversity in Top Predators
\vspace*{-10pt}
\quot{``Where do all these monsters come from? How do they persist generation after generation?"}

Resources are limited, and thus only a finite number of creatures can be supported on any particular diet within any area. D\&D has a biodiversity that would make a modern ecologist sing and dance -- Greyhawk has every single species that Earth has, and then it also has thousands of additional monsters, many of which are technically top predators. That's hard to manage. Remember that to support a single top predator requires a huge amount of energy inputs.

\subsection{The Chicken and the Egg}

In our own world, the question of the chicken and the egg is one put out mostly to confuse the very young. It actually has a definitive answer, the egg came first. It doesn't even matter where you draw the line as to what is a chicken and what is some other creature, because wherever that line is drawn, the creature in question was first an egg and its own parents were not chickens. But in the D\&D world, that's an open question because a lot of creatures are made rather than born and appear in the world fully formed. Golems simply don't need a stable pool of Golems in order to maintain genetic diversity. They don't even have genetic diversity.

Lots of other creatures in D\&D reproduce in a completely magical way. Demons simply spawn out of flaming pits of rebirth, Aleaxes are created from the fact that a god got spent some divine focus (it comes somewhere between ``Place Papal Magnet" and ``Earthquake" I think), and chimeras are assembled out of parts by mad wizards.

\subsection{Small Isolated Populations are Bad}

Bad things happen to a species if there aren't many instances of them in the world. Cheetahs can all accept skin grafts from each other and are all weak to the same diseases. Seriously, all cheetahs could go extinct next year, their existence is that fragile. In D\&D there are numerous species that are less well represented in the world than cheetahs, so why don't they have the same problems?

Truth be told, some of them do. Many times a dungeon will contain a ``unique monster" that the player characters will go and kill. That's an extinction event right there. There will be no more generations of ``five armed fire troll" once you kill Togor the five armed fire troll. But some of them do not, and the reason is that D\&D has a wealth of completely magical ways to keep a lineage from drying up. Even if a creature can't find a mate of its own kind anywhere in the world, there's always the planes and the realms of sorcery. An Archon or Devil can provide a means for any creature to create a new generation. In doing so, the new creature is sometimes pretty much indistinguishable from its mortal parent, and sometimes it shows up as a half-whatever. The point is, no one has to be doing anything particularly weird for the world to have half-fiend dire tigers in it. Which is a load off our minds, because the D\&D world does have half-fiend dire tigers in it.

\subsection{We Eat What We Like}

It has been noted by many an observer that actually humans are a top predator themselves, that it takes nearly 20 years for a human to grow to full size, and they're only 70 kilograms at that point. Thus, the concept of a creature persisting on a diet of human flesh is pretty much absurd. Especially if it lives in an out-of-the-way area like a mountain top or the bottom of a forgotten cave, there's just no way that something can live on manflesh alone.

Some monsters however, get the vast majority of their sustenance through magical means and only need small influxes of real food from their human food. The classic example, of course, is Vampires. They consume much less in food energy than they use up in maintaining their undead existence. Most of their energy is actually siphoned off the negative energy plane, and the drinking of human blood is just a symbolic evil act that they need to perform in order to keep that juice flowing. Similarly, mindflayers are sustained not by the nutritive value of brains, but by their own psychic powers. They need to eat the brains of intelligent creatures to keep their psychic powers sharp, and they need their psychic powers to be sharp in order to survive day to day without eating enough calories to keep themselves alive in the normal way.

Secondarily, lots of creatures have a perfectly fine diet of normal food and simply happen to attack and eat humans if they encounter them. A gelatinous cube, for example, lives just fine off of the lichens and offal that it scrapes off the walls and ceilings with its passing. But it certainly won't turn down a meal of 70 kilos of meat if it comes to that.

\section{Empirinomicon}
\vspace*{-10pt}
\quot{``What difference does it make? We're just going to kill them anyway\ldots"}

The Underdark is filled with empires. It is notable not only that there is civilization down here at all, but that there are actual empires. There's enough space for there to be tribes of creatures that people can unite as part of their nation-building exercises.

But more than that, these disparate tribes are more different than any group of humans have ever been one from another. While there is only one race of humans on Earth, there can be literally dozens of races in a single cavern settlement in a D\&D world.

\subsection{The Myconids: Apathy Writ Large}

The Myconids don't need anybody, have no enemies, and don't care what you do. The first appearance of the Myconid empire was in an adventure otherwise filled with hostile evil humanoids, with the concept being that the Myconids were completely indifferent and would actually return aggression with aggression or soft words with a weird hallucinogenic telepathic mind-meld. The theme was that the Myconids were there as some sort of bizarre intelligence test for your players -- if the players ``figured out" that they could get past the Myconids without resorting to conflict the adventure would be easier and otherwise it would be more difficult.

But the years have passed by, and Myconids are no longer the new kids on the block. Players actually know where they stand with Myconids, and subsequent attempts to write adventures with the same setup have had to make up entirely new Neutral monsters to fill the same role (like the desmodu and their stupid Buck Rogers style fighter planes).

The Myconids have a pretty good world conquest strategy. They don't need anything at all to reproduce themselves and they don't really have to interact with the economies that the other races have bought into. They have an army of the dead and huge piles of crazy potions that they make free of course, but they aren't even interested in fighting the other races. The extent of the Myconid empire and population is limited by the farthest reaches of their mushroom fields, which simply grow a little every year. Eventually the Myconids might push the limits of their fields into areas other races intended to use, and then the Myconids will go ape and start throwing armies of zombies and themselves (every Myconid is completely replaceable) at whatever is in their way -- but for now they just hang out and groove on the telepathy spores and share their dreams.

\subsection{The Aboleth: Inheritance of the Memory Fish}

The signature ability of the Aboleth is that they remember everything that was known to any creature they eat. There's no game mechanics for that ability, it just happens. And while Aboleth can create layered full-sensory illusions whenever they want, and dominate enemies, and turn humanoid opponents into lame deep one clones, the memory devouring ability is the really memorable one. It means that Aboleths remember intimate details about ancient events that go back to the very origins of the Aboleth race, and it means that Aboleths have access to all kinds of special magical sites and gadgets that are not available to other races.

The D\&D world is filled with weird one-off locations that under certain circumstances do potentially awesome things. Over the course of their adventuring lives, a party of adventurers is liable to encounter several of them; and the Aboleth remembers all of the ones found by any of the adventurers that it or any of its ancestors back to the beginning of time ever ate -- which is a tremendously large amount. So any Aboleth plot is going to be facilitated by magical architecture and supernatural convergeances that happen once every hundred years and all kinds of crazy crap. Aboleth periodically plot to take over the world, and otherwise they pretty much sit and fester at the bottom of the nastiest, stinkiest pools in the Underdark.

Keep in mind that even by itself, an Aboleth is badly under CRed. They have actual dominate with a pretty decent DC and they aren't half bad in combat. Also, they have long duration images available as spell-like abilities. So any Aboleth area is going to be covered in layered illusions. If an Aboleth attacks, chances are that it's going to have several turns of doing pretty much anything it wants as the PCs sit shell-shocked on the other side of an illusory wall.

\subsection{The Illithid: Slaves of the Elder Brain}

The illithid have a bad reputation among the other sentient races: the mind flayers see them as food, and most races take offense to that viewpoint. It should be the fuel for a war of extinction on the illithid race, but three things protect the illithid as a race: each is a powerful artillery piece surrounded by hordes of charmed minions; the race is led by powerful elder brains who are each the equal of a powerful sorcerers; and they make good neighbors\ldots\ . that's right, they're good neighbors.

Each mindflayer can potentially control a small army of charmed slaves, can defeat a small army with their powerful stunning blast ability and resistance to magic, and can negotiate any conflict into peace with the ability to telepathically communicate and read minds, but their best ability is the ability to plane shift. As a race that can naturally use this ability, they can hop between planes and be within five miles of any location that they can imagine on their own plane or any other. This key fact means that when they go rampaging for brains and slaves, they not only do it in some place far from their home, but they might do it on some other plane entirely. Due to the fact that few races can mount an extraplanar war, the flayers generally are too far and too difficult to find to ever face retaliation for their acts. Because of the limits of their travel ability, the mind flayers will clear and patrol an area about ten miles from their home, removing any potential threat and keeping dangerous predators away. In this way, they can return to their homes in relative peace, and by scrupulously not preying on their neighbors, they avoid any retaliation on single illithid walking home. Add in their mind-reading and telepathy ability, they are naturally suited to making mutual defense pacts with nearby races so that they can establish a peaceful dominance in their own territory.

The fact that every mind flayer enclave is controlled by a powerful elder brain is another fact that makes their enclaves safe and their culture vital. As a powerful, but generally stationary creature, it has every incentive to make its home as well-defended as possible, drawing on its own powers to equip its home with wondrous architecture and traps befitting a powerful sorcerer (or psionicist). Add in the hordes of slaves and the illithid themselves, this means that even moderately-sized enclaves can bring to bear enough force to make taking the city an extremely unprofitable enterprise.

One final note about the illithid: as planar travelers with an innate ability to travel to any plane, they often gain access to technology and magic from cultures beyond counting. While the mind flayers are geniuses in their own right, they often store knowledge of these devices in the minds of their slaves, a practice that leads them to losing that knowledge when hunger or carelessness takes that slave away. Even so, expect the average illithid to be a font of secrets dredged from dozens of extraplanar cultures, its home filled with odd artifacts and devices culled from those far-off places. If their own powers and hungers weren't so great, they might even be drawn to the exploitation of this knowledge. Luckily for the races of the worlds, the mind flayer's total confidence in their own abilities and need expend time feeding on difficult-to-acquire fare makes them ignore all but the most obviously useful things stolen from other cultures.\\

\classname{Progenitor of the Gith}\label{class:progenitor}
\vspace*{-8pt}
\quot{''I've spent five years as a slave to brain-eating geniuses, fighting every day in the pits for their amusement, killing beings culled from dozens of planes. Do you really think this crap impresses me?"}

The Illithid are slavers extraordinaire, masters of the mind control and capable of traveling far in their search for slaves. To escape their clutches, one must become a creature as powerful as them, and some do so by absorbing the ambient psionic radiations of their cities and becoming a more than mortal creature. In this way, the Githzerai and Githyanki earned their freedom, and this route is still open to those willing and capable of surrendering their essence in exchange for communion with the Astral Plane.

\ability{Prerequisites:}{}
\listprereq
\itemability{BAB:}{+4}
\itemability{Race:}{Human}
\itemability{Special:}{Must have spent at least five years as a slave in an illithid city.}
\end{list}\vspace*{8pt}


\ability{Hit Die:}{d8}

\ability{Class Skills:}{The Progenitor of Gith's skills (and the key ability for each skill) are Balance (Dex), Bluff (Cha), Climb (Str), Concentration (Con), Craft (Int), Handle Animal (Cha), Hide (Dex), Intimidate (Cha), Listen (Wis), Move Silently (Dex), Profession (Wis), Ride (Dex), Search (Int), Sense Motive (Wis), Spot (Wis), Survival (Wis), and Swim (Str).}

\ability{Skills/Level:}{4 + Intelligence Bonus}

\begin{table}[htb]
\begin{small}
\begin{tabular}{lp{1.9cm}p{0.7cm}p{0.7cm}p{0.7cm}l}
Level  &Base Attack Bonus &Fort Save &Ref Save &Will Save &Special\\
1&+1&+2&+2&+2&Thoughtful Warrior, Endurance of the Mind\\
2&+2&+3&+3&+3&Ideas Made Form\\
3&+3&+3&+3&+3&Movement of the Mind\\
4&+4&+4&+4&+4&Astral Strike\\
5&+5&+4&+4&+4&Native of the Silver Sky\\
\end{tabular}
\end{small}
\end{table}



\noindent All of the following are Class Features of the Progenitor of Gith class.

\ability{Weapon and Armor Proficiency:}{Progenitors of Gith gain proficiency in the composite long bow, and gain no armor proficiencies.}

\ability{Thoughtful Warrior (Sp):}{At 1st level, a Progenitor gains the ability to cast \spell{daze}, \spell{mage hand}, and \spell{feather fall} at will as a spell-like ability}

\ability{Endurance of the Mind (Su):}{A Progenitor has likely been mind blasted and charmed many times in his life. If he is currently the subject of an ongoing effect that allows a Willpower save, he may retest that saving throw every round. Success is treated as if he had passed the initial Willpower save.}

\ability{Ideas Made Form (Sp):}{At 2nd level, the Progenitor gains the ability to cast \spell{clairaudience/clairvoyance} and \spell{shatter} at will as a spell-like ability}

\ability{Movement of the Mind (Sp):}{At 3rd level, the Progenitor gains the ability to cast \spell{dimension door} at will as a spell-like ability.}

\ability{Astral Strike (Sp):}{At 4th level, a Progenitor can cast \spell{Telekinesis} at will as a spell-like ability.}

\ability{Native of the Silver Sky (Ex):}{At 5th level, the energies of the Astral Plane now bolster the physical form of the Progenitor, and he gains becomes an Outsider native to the Astral Plane and he gains Spell Resistance equal to his character level +5, a +4 armor bonus to AC, and the ability to cast \spell{plane shift} twice a day as a spell-like ability.

If he breeds with a githzerai or githyanki, any offspring will be of that race.}


\subsection{The Drow: A Higher Technology Setting}
Everyone knows that the dark elves are hardcore. Even in the bad old days of D\&D's conception, dark elves were ``mirror matches'' to the party with class levels of their own, crazy magic items of their own, and good tactics. The real question is: ``why are the dark elves so hardcore?'' The answer is simple. Dark elves are living at a higher technology level than the rest of the D\&D world; their society only exists because, as a society, they cheat. Rather than grow food like surface races, they eat magic mushrooms as the basis of the food chain, they enslave other races for menial positions rather than work, and rather than mine or gather their own resources, they take them from other races. They don't even have to work that hard on defense as Underdark caverns are naturally easy to defend with small numbers of troops stationed at chokepoints

This means that your average dark elf has free time to spare. While some take that time to indulge in the pleasures of their society, most dark elves are the products of a very odd world view: if only dark elves are your peers and everyone else is a slave, then the only real power worth having is power over other dark elves. That being the case, this means that dark elves have both the free time and the inclination to attempt to enslave each other all the time. This breeds great internal strife with each noble house being an armed camp designed to use stealth, power, and manipulation in order to both resist the efforts of other dark elves and attempt to enslave them.

Like any heavily-automated wartime culture, the dark elves spend considerable resources on weapons research, espionage, and cultural misinformation. This means that every noble house or other organization is constantly looking for an ``edge'' in their dealings with other dark elves and other races. This leads them to kidnap experts from other races, engage in spell research, experiment with weird magic or exotic technology, forge partnerships with magically or technologically-advanced races, and otherwise do whatever it takes to grow in power. In any particular drow city you can expect to see dozens of competing forms of magic, odd inventions ranging from mechanical limbs to powered gliders, exotic troops like demon-bred orcs or elite espionage races like skulkers, and constructions with magical architecture or resonances. Since every drow is attempting to master his peers, these magics and technologies are tightly controlled, meaning that when the individual or organization that controls them is killed off, these secrets are often lost, meaning that any particular drow might be using relics from a previous generation (that he may well lack the ability to understand or reproduce).

Other races in the Underdark realize that the dark elves truly only want to control each other, so they allow the occasional resource and slave raids of the dark elves. They know that the dark elves are ill-suited to any form of large-scale conquest due to their particular style of command, so placating the drow is often the best way to conserve the resources of your society. Since the other Underdark races tithe goods to the drow and the drow are smart enough to see the value in trade relationships, Underdark races of note are allowed to use dark elf cities as major trading posts between their own kind and other races. The dark elves see all races as being underneath them, so as long as the other races show deference to them and bring in a profit in trade, they allow this enterprise to continue.

The average drow city is thus a hornet's nest of power, full of indolent, wildly dangerous, and spoiled aristocrats. Even the lowliest of drow lives in a level of luxury suitable to the most powerful of nobles on the surface, and each one of them has reached adulthood in an atmosphere of distrust and manipulation with the weakest dying early. As individuals this makes them powerful and cruel, but as a race it keeps them inwardly looking and less of a threat than more ambitious warrior races, a fact that actually prevents other races from gathering their forces and destroying the drow outright.

\subsection{The Eye Tyrants: Lingering Hatred}

Beholders are among the most magically capable of races; this is a fact that's well accepted. So why don't they own everyone? They can charm people at will, kill people, inspire terror by turning people to stone or just inspire terror with real magical fear. Even the lowliest beholder can attempt to destroy the world one ten foot cube at a time.

The truth? Beholders are paranoid jerks. Beholders recognize that they each have the ability to destroy each other with at least four different effects, and they also have to live with the fact that any beholder can charm his lessers and betters if he happens to get the jump on them. This means that, as a race, the beholders are like gunslingers from the American Old West. They know that to associate with any other beholder is to risk disintegrate or charm rays in the back with the winner taking the loser's treasure and slaves. Actual beholder meetings involve both parties agreeing to aim their anti-magic rays on each other, and only then can negotiations or exchanges can take place. This means that actual organizations of beholders are practically impossible as life breaks down as soon as you can't cover all your enemies in your anti-magic eye (with many eyes, they can spot ambushes by thralls pretty easily, so it's only your peers you fear).

That still doesn't explain why beholders don't go on bloody rampages on the surface races. The reason is simple: longbows. The average beholder is a tough customer that can expect to wreak a bloody swath of destruction if he chooses, but he's painfully weak against long-range weapons. Any race with even a passing knowledge of the beholders knows that they charm people, so they also know that killing the beholder frees the slaves. This is why the beholders prefer underground areas. With ranged weapons blocked by the limits of doors, walls and corridors, beholders can reign as kings in underground or indoor environments.

While the Spelljammer universe posits ``nations'' of beholders held together by racial hatred of other beholders and everyone else, this is really a fallacy. Racial pride or nationalism come to far seconds when you realize that beholder nations are actually held together by single individuals who routinely charm every other beholder on their ship and force them to ``play nice'' with the other beholders. Any ``Hive'' metaphor talking about beholders actually talks about the layers of charm effects building a top-down command structure where the Queen controls everyone, then the second-in-commands has control of everyone else in order to serve as secondary leaders but unable to defy the Queen. Like other kinds of dictatorships, killing the leader figure causes the nation to fall apart into bloody factions headed by second-in-commands attempting to assert control of the others, and these power plays work through the bonds of charm effects and personal charisma. Some Hives are actually controlled by a racial variant called a Hive Mother, but these creatures are merely biological extensions of relationships that already exist within beholder society.

\subsection{The Kuo-Toans: Opportunities Slip By}

The Kuo-Toans are extremely aware that things used to be pretty awesome if you were a Kuo-Toa, and now they suck. They are actually a deep ocean race and they don't live in the ocean at all anymore. That's because long ago they lost a war to the Sahuagin. And they lost it badly. Now they live in pools of water that often as not are fresh water in the bottoms of caves, and they hate it here. The lack of pressure and salinization of the water makes the Kuo-Toans unhealthy and uncomfortable, and they end up stinking of rotting fish as their skin becomes diseased and crumbly.

Every generation of Kuo-Toa is a little sicker than the one before it, and everyone understands and accepts that the race is dying out. Every Kuo-Toa expects the future to be worse than the present, and the Whips (the Clerics of the Kuo-Toa) do nothing to forestall that process or convince their people otherwise. Legends say that the Great Evils they left behind at the bottom of the seas will eventually return to destroy the whole world, but only once they've successfully fed them with enough of the misery of the Kuo-Toan people. No one in Kuo-Toa society wants to become a leader, because the world will become even more unpleasant every year and the leaders are always blamed. A Kuo-Toa gains a position of leadership when the old leader is finally killed and eaten for failure and the Whips draw lots for who has to be the next leader. Most Kuo-Toans believe that these lots are fixed in advance, and they're right.

Despite the utter hatred that all Kuo-Toans hold for all other races, they are perfectly willing to trade with them. The Kuo-Toans are badly out of their element, and need nutritional supplementation from far away just to survive. They need to receive goods from the Drow, and they know it. They hate the Drow, as they hate everyone, but that doesn't stop them from trading. The Kuo-Toans understand that the Aboleth know where every single one of their spawning pools are and that only laziness on the part of the Aboleth has left the Kuo-Toan people with any territory at all. Still, they wait in the darkness for the cataclysm to come that will put them out of their misery and slaughter all the other creatures of the land and the sea. Their one hope is that just before the last Kuo-Toa is finally slain, that they will see with their own eyes the horrible vengeance wreaked on the other empires.\\

\classname{The Monitor}\label{class:monitor}
\vspace*{-8pt}
\quot{"In time\ldots\  even the sun will die. Until then\ldots\  I shall content myself with your demise."}

Kuo-Toans are a depressing group to hang with at the best of times. Their relentless downbeat attitude can turn even the most festive of occasions into a dirge. But perhaps the most depressing of all Kuo-Toa are the Monitors. These monks of Kuo-Toa society are dedicated to a strict regimen of martial training and meditation on the complete futility of all things. Discussions with Monitors have been known to drive even other Kuo-Toa to suicide.




\ability{Prerequisites:}{}
\listprereq
\itemability{Skills:}{9 ranks in Balance}
\itemability{Feat:}{Multiattack}
\itemability{Alignment:}{Any Evil.}
\itemability{Special:}{Must have at least one Fighting Style class feature.}
\itemability{Special:}{Must be trained in the Kuo-Toa Monasteries.}
\end{list}\vspace*{8pt}

\ability{Hit Die:}{d8}

\ability{Class Skills:}{The Monitor's class skills (and the key ability for each skill) are Balance (Dex), Climb (Str), Concentration (Con), Craft (Int), Escape Artist (Dex), Hide (Dex), Jump (Str), Knowledge (Religion) (Int), Knowledge (Dungeoneering) (Int), Listen (Wis), Move Silently (Dex), Perform (Cha), Profession (Wis), Sense Motive (Wis), Spot (Wis), Swim (Str), and Tumble (Dex).}

\ability{Skills/Level:}{4 + Intelligence Bonus}

\ability{BAB:}{Good (1/1), Saves: Fort: Good; Reflex: Good; Will: Good}

\begin{table}[htb]
\begin{small}
\begin{tabular}{lp{1.9cm}p{0.7cm}p{0.7cm}p{0.7cm}l}
Level&Base Attack Bonus&Fort Save&Ref Save&Will Save&Special\\
1&+1&+2&+2&+2&Fighting Style, Armored in Life, Powerful Observation\\
2&+2&+3&+3&+3&Strike the Intangible\\
3&+3&+3&+3&+3& Master Fighting Style\\
4&+4&+4&+4&+4& Wait for Death\\
5&+5&+4&+4&+4& Master Fighting Style\\
6&+6&+5&+5&+5&Depressing Monologue\\
7&+7&+5&+5&+5&Master Fighting Style\\
8&+8&+6&+6&+6&Sticky Hands, Apathy\\
9&+9&+6&+6&+6& Grand Master Fighting Style\\
\end{tabular}
\end{small}
\end{table}



\noindent All of the following are Class Features of the Monitor class.

\ability{Weapon and Armor Proficiency:}{A Monitor gains no proficiency with any weapons or armor.}

\ability{Fighting Style:}{The Monitor gains a Fighting Style as a Monk at first level.}

\ability{Armored in Life:}{Levels of Monitor stack with levels in Monk for the purposes of the Monk's Armored in Life ability.}

\ability{Powerful Observation:}{A Monitor adds his class level as a bonus to his Spot and Sense Motive checks.}

\ability{Strike the Intangible:}{At 2nd level, a Monitor gains the ability to strike the invisible creatures he can see. His natural weapon attacks can hurt incorporeal and ethereal targets without a miss chance related to intangibility.}

\ability{Master Fighting Style:}{At 3rd, 5th, and 7th level a Monitor gains a Master Fighting Style, as a Monk.}

\ability{Wait for Death (Su):}{A Monitor looks forward only to death, but this can be a very long wait indeed. A Monitor of 4th level does not age, sleep, need nutrition, or breathe. Furthermore, a Monitor of 4th level no longer loses hit points when he has 0 hit points or less.}

\ability{Depressing Monologue (Su):}{Any creature that speaks to a 6th level Monitor for more than five minutes must make a Willpower Save (DC 10 + \half hit dice + Charisma Modifier) or be affected by \spell{abject despair} and \spell{curse of crumbling conviction.}}

\ability{Sticky Hands (Ex):}{A Monitor of 8th level makes great use of sticky Kuo-Toa secretions and gains a +4 bonus on Disarm tests, whether he is the attacker or the defender.}

\ability{Apathy (Su):}{At 8th level, a Monitor is able to draw upon supernatural reserves of ennui and ambivalence, rendering him immune to mind affecting effects.}

\ability{Grand Master Fighting Style:}{At 9th level, a Monitor gains a Grand Master Fighting Style as a Monk.}


\subsection{The Troglodytes: Persecution Complex}

Everybody hates Troglodytes. Everybody. They don't necessarily do anything that horrible in the scheme of things, they just happen to stink so bad that they can cause other races to collapse from nausea. So while the dwarves have a very complicated relationship with the hobgoblins where they have long periods of intermittent strife punctuated by flourishing trade relations and shared artistic histories and stuff -- the dwarves literally don't have anything nice to say about the troglodytes at all. Their entire history with the Trogs is one where sometimes they fought and sometimes they didn't fight. There's never been real peace between the Troglodytes and anyone. That's hard on a culture, and their isolation has made them intensely barbaric and xenophobic by the standards of any other race. Troglodytes can't even use the other races as slaves, and open lines of communication do not exist so the Troglodytes can't trade captives back to other races for concessions on the bargaining table. There isn't even a bargaining table at the end of any conflict.

So if you get captured by Troglodytes, you're going to be eaten or sacrificed to their dark gods. The Troglodytes literally have no other use for captives. So the only reason for any of the other races to surrender to Troglodytes is if they think there is a chance they will be rescued. Troglodytes themselves will generally not surrender in battle because they believe that other races will treat them the same way that they treat others.

A natural result of all this, is that the Troglodyte tribes are much lower tech than the rest of the setting. They have no trade in equipment or ideas with the other races, so the only steel equipment that Troglodytes have is what they looted off of fallen enemies. Most troglodyte weapons are just sharp rocks. Troglodytes can be useful to a campaign because they have a legitimate reason to still be ``cave men" even while the rest of the world is putting together portal highways and overshot water mills.


\section{The Constructanomicon}
\vspace*{-10pt}
\quot{''How does that even stay up?"}

Perhaps the most important question surrounding Dungeons and Dragons is the question why there are Dungeons and Dragons. When you think about it, that's pretty weird.

\subsection{Dungeons: By the Gods, Why?}

Alright, we know that you love dungeons. We love them too, despite the fact that we're pretty sure there is no good reason for the silly things. The average D\&D game world is frankly incapable of the technology or manpower needed to build vast underground complexes. I mean, look at our own world history: aside from a single underground city in Turkey and a couple of pyramids and tombs, the ancient world took a pass on underground life. Even the old excuse of ''Wizards can magic it up and they do it because its defensible'' is a bit lame considering that we are talking about a world with teleport and burrowing and ethereal travel; being underground is actually a liability since its harder to escape and people can drop the roof onto you, not to mention the incredible costs involved in doing it even if magic is available.

So here is what we suggest: dungeons have an actual magical purpose. By putting anything behind at least 40' of solid, continuous material (like solid walls of dirt, stone, ice, or whatever, but not a forest of trees or rooms of furniture) the area is immune to unlimited-range or ''longer than Long Range'' spells like Scrying and transportation magic like teleport, greater teleport, the travel version of gate, and other effects. You can use these magics inside a dungeon, but you also stopped by a 40' solid, continuous material in a Line of Effect; this means you can use these effects inside a dungeon to bypass doors and walls, but entering and leaving the dungeon is a problem, and parts of the dungeon that have more than 30' of material in the way between your position and the target of your effect will be effectively isolated from your position.

In summary, in a best-case scenario you can transport yourself to a dungeon, then bust in the entrance and enter the dungeon, then transport yourself to the place you want to be inside the dungeon. In a worse-case scenario, the dungeon designer will have built the dungeon in such a way that only someone aware of the layout can take full advantage of unlimited range or transportation spells like teleports and Scry, or even that most or all areas if the dungeon are inaccessible to these effects.

Of course, there are exceptions. The idea of permanent portals, gates, or teleport circles are just too common in D\&D and too fun to just abandon. Permanent effects will continue to regardless of materials in the way, and will be the premier way to enter and leave dungeons, as well as the best way to move inside a dungeon.

By incorporating these changes in your D\&D world, you are ensuring that players actually explore rooms in your dungeons that you have painstakingly built, you avoid all the problems with Scry-and-Die tactics, and you'll find that players actually care about dungeon geography. It also adds a bit to suspension of disbelief in your setting, which is only good for a cooperative storytelling game.

\subsubsection{Dungeons: building dungeons for fun, profit, and defense}
As an old hand at D\&D, I've seen more dungeons than I can count. Most have followed a ''random generator and a new pad of graph paper'' philosophy to dungeon construction, and frankly that's got to go. Here are a few tips to constructing a dungeon that makes real sense:

\subsubsection{Chokepoints Are Your Friend}
Most dungeons are built like a modern building: ease of use and easy access are emphasized. Don't do that. Remember that a dungeon is built with the idea that it will be invaded at some point by a hostile and possibly supernatural attacker. At the very least, this means that rooms will not have doors to every adjacent room, and single hallways to single rooms will also be avoided.

Chokepoints are your single most important consideration. You want to make sure that attackers get bottled up in them and your forces don't get caught up in them. That's trickier than it sounds. Generally, place your chokepoints at the entrances and exits of your dungeon, and possibly at ''fall back'' positions where troops can make another stand if their position is overrun. Key locations should have their own chokepoints like prisons, treasuries, and quarters for potentially hostile quests. Locations that should not be blocked off by choke points include barracks, armories, and key storage rooms, since you never know when your troops might need some arms or materials to react to a threat.

\subsubsection{The Three ''M''s: Mobility, Manpower, and Morale}

A dungeon is built to house a fighting force, and several considerations come into play in its design. If your dungeon is an abandoned ruin, then the current residents might not exploit these features, but be sure that the original designers had them in mind..

\textbf{Mobility:} Choke points are the first stage in the idea of mobility, as they assume that your enemies will be stuck gathering their forces at once point and behind that chokepoint you are gathering your forces as well; however, that does not need to be true. The designers of a dungeon can easily place one-way secret doors that allow them to get behind an enemy position and outflank an enemy, sending forces from two sides to crush an enemy.

Also, the common feature of long hallways with rooms off to the sides must be avoided. While this is a simple arrangement (and easy to draw on graph paper), it allows attackers to make straight shots toward key areas. It is better to mix-up the layout of non-essential rooms like storerooms so that enemy forces become split as they search rooms and take different routes. A common mistake like a long hallway or a central room with doors allows the enemy to send scouting forces to check rooms, then they can quickly surge forward if one of those forces finds a threat. It is better to split an enemy's forces between several collections of rooms, leaving groups isolated in the event of a counterattack.

\textbf{Manpower:} A well-designed dungeon needs guardians, and there are no solid rules about who you need in your dungeon. Generally, you want troops that are loyal, intelligent, and powerful, but often other considerations come into play. Dumb beasts can be chained at a choke point, and they are perfectly suitable as guardians, and large numbers of weak but smart defenders can set off traps, block passages, or slow the advance of the enemy with caltrops or even their own lifeblood. Depending on the type of guardians the dungeon was intended for, it can have wildly different layouts. For example, a dungeon may have a room that is merely a pit with ladders leading to an entrance and exit, and this room simply houses a dangerous beast like a Dire Bear. Any enemy who wants to take this chokepoint would need to fight the bear. Another example could be a dungeon designed to have kobolds as defenders; this kind of dungeon may have small-sized corridors so that they can move quickly from rooms to room (so that any medium-sized creature must squeeze in) and covered shooting galleries where the kobolds can use crossbows to fire on attackers from relative safety.

\textbf{Morale:} An often overlooked aspect of dungeon construction is morale, which is the simple question of ''are my troops happy enough to stay and confident enough to fight.'' Kitchens and ample food stores are a good first place to start, as are comforts like good barracks or personal rooms, timely payment of salary, and amusements. While a Half-fiend Chimera can be locked in box without food or air, its loyalty and willingness to fight is definitely in question. Some dungeon creators use mindless beasts or unintelligent monsters like oozes, while other creators use controlled monsters like undead, but these troops are generally less effective than dedicated and intelligent troops.

If the dungeon has luxuries like escape routes, common rooms to socialize in, entertainments like gaming rooms, and places to worship gods, troops will be more willing to fight when attackers threaten. Without these things, troops might surrender or flee from a hostile threat, or even turn on the dungeon creator.

\subsubsection{Form Follows Function}

Sometimes, dungeons can be designed in a crazy fashion that is fun to play in, but makes no tactical sense. That's fine, since it can mean that the dungeon was built as part of a magical effect or for some mystical reason. A certain arrangement of rooms may create a dungeon-wide effect that blocks ethereal travel or teleportation, or maybe the fact that the dungeon is arranged like a demon's face means that the dungeon is a giant mystical trap for a bound demon.

The sky is the limit for this kind of thing, and we encourage you to ''go nuts" as it creates flavorful dungeons that you will remember years later. I'm certain people are more likely to remember a dungeon built as a giant hive with hexagonal rooms, honeycombed passages, and undead bees than they are going to remember a standard temple of Orcus.

\subsubsection{Castles and Manors: Taking the Dungeon out of the Dungeon}

Traps, choke-points, humanoid defenders, and monstrous occupants can all be found guarding treasures and lifestyles above ground as well as below. Unfortunately, a building that extends above the surface is inherently more vulnerable than a true Dungeon to the most feared of D\&D tactics: Scry \& Die.

\subsubsection{Unimportance}

While a castle is by definition subject to scry \& die tactics, the number of creatures actually capable of pulling that off is fairly limited and if they don't care enough about your buildings, you're pretty safe. A building doesn't have to be bereft of valuable loot and major players in the game of thrones to avoid teleport assaults -- it just has to look that way. In many ways a run-down shack is safer than a gleaming adamantine fortress. And that means that illusions like hallucinatory terrain and mirage arcana are very valuable to any fortress whose purpose is to keep its occupants and their treasure safe. If no one cares, your swag and your family are safe.

\subsubsection{Magically Foiling Diviners}

When you don't have 40' of solid stone between you and the hostile world outside, scry \& die is a real problem for you. Especially if you're trying to keep order and rule a region, and therefore hiding your fortress really isn't an option, magically protecting yourself from attack magic is going to look pretty tempting. For those of you who are old school, attention has to be drawn to the fact that \spell{nondetection} actually doesn't work at all. It costs you money every day, and the would-be teleport assassins have a chance of spell failure every time they attempt to scry on your location. But nothing happens to them if it doesn't work, so at best \spell{nondetection} makes them try again later. Eventually they're still going to come for you, and you're out a small pile of diamond dust.

The big winners here are \spell{mirage arcana} and \spell{mindblank}. \spell{Mindblank} always wins, even against gods, but it only stops people from pulling a scry \& die on you. Your enemies can still teleport ambush your house, or your butler, and just sort of assume that if your servants are preparing your favorite food in your house that you're probably in there somewhere. This means that if you are living a high profile gangster lifestyle, \spell{mindblank} is of limited utility, but if you are willing to be a shadowy sage who lives on a demiplane somewhere that no one has heard of, it's totally the win. Mirage arcana simply makes a room appear as a different room. This means that when someone attempts a scry \& die, they end up shunted to some completely different room that presumably has deadly magical traps all over it. Unfortunately, there are ways for a clever diviner to bypass that sort of thing, and there's not a whole lot you can do about it. Ultimately, only stupid Wizards lose when they pull Scry \& Die, so based on the Intelligence requirements of Wizarding\ldots\  you pretty much know how this is going to go down. Still, a clever Illusion trap can nab an impatient Wizard, and that's often good enough.

A special shout-out needs to go to \spell{dimensional lock}, because the effects on would-be teleport assassins is hilarious. It doesn't cause the spell to fail, it merely stops dimensional movement into the warded area. So the assassin moves to the Astral Plane, is shifted at high speed over to the segment that corresponds to next to your bed, and then the shift back into the material world fails. This leaves them all buffed up and stranded on the Astral Plane. You can even amuse yourself by putting lethal traps on that portion of the Astral Plane to nail these guys on the way in. The downside of course is that a lock is only 40 feet across, so covering enough of a castle to make teleport ambushes impractical is difficult. Still, if you have enough 8th level spell slots lying around (or less, remember that it's a lower level spell for the Summoner), it provides the basis of some very nice protection. Also good is the fact that since dimensional locks can be tiled, it can also leave spaces that you can use as a means of entrance/egress and which can be potentially defended if they are used as attack points by hostiles.

The \spell{anticipate teleport} line of spells is a cantrip on the Summoner list for a reason. Those spells don't actually stop a scry \& die, and the areas are very small and duration unexceptional. Even if you are a Summoner, defending your house with \spell{anticipate teleport} is probably implausible. The final consideration is the elephant in the room: \spell{Screen}. It's an enormously powerful spell where it fools scrying ''automatically", but unfortunately it is defined so vaguely as to be essentially unusable without creating an argument. Which is really a shame, because it's otherwise the best hope for defending yourself. Your best bet is to make certain key rooms appear like other rooms so that teleport ambushes end up in the wrong areas -- which means that it's basically just \spell{mirage arcane} that's several levels higher.

\subsubsection{The Public Square: When Divination Doesn't Matter}

Sometimes your building is Courthouse, or a Market, or a Factory, and the entire point is that the general public goes in and out of the building all the time. In such a circumstance, all the divination magic in the world doesn't mean anything because your enemies can actually just walk into your building to scout the place for a teleport ambush or even buff themselves up on the outside and then run in while 1/round a level spells are counting down their awesome. In these circumstances, you're going to want a fall-back position to be readily available on little or no notice. Contingent magic and magical traps may well want to pull key personnel out rather than send summoned monsters or impediments in. After all, if you put off the final confrontation for 20 minutes, the teleport ambush has essentially failed.

\subsection{Traps}
\vspace*{-8pt}
\quot{''How did that boulder not crush those displacer beasts?"}

Dungeons are classically filled with monsters and traps. That can be cool, but I'll be the first to admit that it's pretty weird. Traps and monsters are profoundly counter synergistic.

\subsubsection{Designing Traps}

There are numerous collections of devious traps that can easily kill a single character or an entire party. But let's face it: most of them are dumb. Making a trap that will kill or humiliate characters doesn't make you a genius, making traps that kill player characters is easy. Just have the roof cave in to inflict more damage than the PCs have in hit points, it's not even hard. The difficulty is making traps that make sense, as well as traps that will add to the enjoyment of the game rather than paralyze it with a continuous ''I check the banister, Mother May I?" fest.

\subsubsection{Placing Traps}

For a trap to be effective, it has to have essentially no chance of backfiring against its creators. Remember that the dungeon occupants are going to spend a lot more time in the vicinity of any traps than any invading force is, so there has to be a pretty good reason why the trap wouldn't backfire. Traps can cordon off areas that are too big or too small for the normal residents to set them off (Kobolds might put in a collapsing floor that triggered off a weight of over 100 pounds, and Stone Giants might put nasty traps all over any 5' hallways that ran through a dungeon they occupied), or areas that are for whatever other reason off-limits (Dwarves might trap tapped-out shafts in their mines to nail burrowing monsters trying to sneak in the back way). Some traps sound like they'd be plenty selective enough to put everywhere -- like magical symbols that only blast the forces of Good or heretics who don't follow your god. Be careful with those, as just because they won't explode on any of the normal residents doesn't mean that they won't be a liability. After all, what's the point of being a Cleric of Loviatar if you can't have captured Paladins brought to your chambers for interrogation?

Traps can also be left in an ''inactive" state much of the time, and then triggered into activity only when the dungeon's occupants believe that they are under attack. A switch that activates traps in many non-essential areas (like the rec room or the loading dock) is a very real possibility. These can also be activated in layers, a prearranged fallback point might have the mechanisms to activate traps in the outer area that has presumably been compromised by intruders.

Remember that a trap, once active, makes an area more difficult to use. Sometimes that's OK, as is the case when the area in question is being invaded by Bugbears or is itself a tomb prison meant to hold a powerful demon god. But sometimes that's really inconvenient. Active traps just don't make any sense in the mess hall or the barracks. Your own soldiers are going to fall into that pit full of spikes about a thousand times more often than invading adventurers are if you put it right next to the beer kegs.

\subsubsection{Organizational Traps}

The least obvious, but in many ways most useful trap is one which simply allows defenders to respond appropriately to an oncoming attack. An alarm spell is, in the right hands, the most powerful trap in the core rules. You can put it anywhere, and all it does is make a sound when someone enters the area. Like the bell that sounds when you enter a 7-11, the effects of this trap do not meaningfully interfere with the normal operations of the facility they are ensconced in. These traps have as their core utility that they alert the defenders or delay an attacker. Really swank traps will do both.

Obviously, these traps are only worth anything if you have defenders. But remember that a dungeon filled with giant centipedes, or some other mindless monster really isn't going to take full advantage of an alarm system (a ringing bell may wake a sleeping mindless defender up, but it's not going to be able to figure out whether the bell indicates a customer or an invader). Traps designed to misdirect, delay, or otherwise hamper invading forces are only going to appear in unoccupied regions of a dungeon if they are capable of diverting unauthorized entrants into lethal traps. The name for that kind of set-up is a ''Rube Goldberg Mechanism" and it generally has no place in D\&D. Looney Tunes or Mousetrap perhaps, but generally not Dungeons and Dragons.

\subsubsection{Lethal Traps}

Lethal traps are in no way less dangerous to their creators than they are to invaders. Remember always that the creatures in a dungeon intend to live there for perhaps years or even centuries, and the statistics on mine fields just aren't good. The residents of a dungeon have to be completely convinced that a potential trap can't cut off their jangly bits when they are making their way to the privy in the middle of their sleep cycle. That doesn't even mean that lethal traps can be in places that unauthorized residents aren't allowed (like the master's bedchamber) -- that's going to end up beheading servants and guests.

Lethal traps appear in only a couple of kinds of places:

\listone
    \item \textbf{Battlefields:} If an area is contested, right now, having a lethal trap in there is an antisocial but plausible technique.
    \item \textbf{Deserted Regions:} If you leave the dungeon to go on a pilgrimage to a Planar Touchstone that you dig, it's quite thinkable to activate some nasty traps while you're gone.
    \item \textbf{Inaccessible Areas:} If you take over a Brownie hole, there's going to be a lot of crazy hallways that you can't even get into. Filling the mouse holes with mousetraps is fine.
    \item \textbf{Vaults:} If you have something, like a repository of important treasure perhaps, that is really hard to open and is supposed to be used infrequently and possibly only in some sort of crazy ''two guys whip out their keys at the same time" scheme -- trapping that is totally expected.
    \item \textbf{Discerning Traps:} Some magical traps are able to detect certain kinds of creatures and only detonate on specific ones. Unless you're a crazy loner wizard who has no friends and conducts no commerce, those are pretty much a liability. But hey, if you are a Lich-Master Hermit, then those sorts of traps are fine.
\end{list}

\vspace*{8pt}

What this means is that if a dungeon isn't on a war footing right now, any lethal traps in it are probably going to be inactive. If the hobgoblins don't believe that they are under attack right now, the pressure plates all over the dungeon are going to be in their locked position and opening doors is not going to cause poison blades to shoot out. Once they fall back and pull the ''totally being attacked" lever -- then you can go back to worrying if Gygaxian traps lurk behind every door or neck-level tripwires might release torrents of green slime.

\subsubsection{Living Traps}

Some creatures are essentially traps, distinct only in that they have a Wisdom and Charisma score. The monstrous spider, the dire bear in a pit, and the golem are all classics, but the sky is really the limit here. Creatures can act like guard dogs if they are intelligent or magically controlled enough to tell friend from foe. Or they can act like punji sticks at the bottom of a pit if they are uncontrolled.

To be useful to a dungeon's occupants, a living trap has to be unable to turn on its masters. The occupants live in this place so any ''wandering monsters" had best be capable of discerning intruders from VIPs. Any monsters that can't make discernments like that need to be kept in cages or other inaccessible regions of the dungeon until someone specifically unleashes them in the event of a dungeon invasion. What this means for a dungeoneer is that successfully disguising yourself as a Dungeon Resident will keep the trained displacer beasts from attacking you. Furthermore, if you sneak into a dungeon, the untrainable creatures (monstrous vermin, ooze monsters, whatever) are all going to be locked up until an alarm gets sounded. A little discretion can make the dungeon environs a lot safer for the would-be raiders.

\subsubsection{Beneficial Traps}

Game mechanically, any localized triggered magical event is a ''trap". So if you whip out a room that heals everyone in it every round or an immobile pool that you can scry right out of, that's going to be a trap as far as the game is concerned. That means that the residents of a dungeon can shill out surprisingly small amounts of nuyen to get their pads to do all kinds of crazy stuff. Unlimited healing, permanent scrying pools, and more will be a fact of many rooms in virtually any dungeon. Moving these things is impractical, so ownership of a dungeon can be a very lucrative proposition


\section{The Lexiconinomicon} %: Things to Talk About

\subsection{Language in D\&D}
\vspace*{-8pt}
\quot{"Does anyone speak `Roper'? Anyone?"}

The default languages of Dungeons and Dragons (Goblin, Halfling, Giant, Common) assumes a certain level of racial tribalism, where a village is normally expected to be a "Halfling village" or a "Gnoll village," and that was the presentation of the D\&D world -- in 1977. The AD\&D Monster Manual talked about villages of Orcs or Goblins, and you could seriously count on your fingers the number of races that shared living space, and many of those "races" were just leveled versions of normal humanoids (flinds and lizardkings, for example).Thirty years have come and gone since those bad old days, and the modern presentation is much less "genetically isolated tribes" and much more "mixed species regions." Today when an enemy village is written up it has bugbears and orcs, and grimlocks and all kinds of crazy crap in it.

But the languages haven't changed, even though the presented social setup no longer supports that paradigm. A child grows up speaking whatever languages they happen to be exposed to, so when the Orcs were living on their own it was pretty reasonable for the language spoken by Orcs to be a different one from the other tribes and to be identified simply as "The Language that Orcs speak" or simply "Orc." But if any more complicated social system or demographic distribution is posited, that no longer works at all.

\subsubsection{Regional Languages}

The attempt to put Regional Languages into the mix has been a dismal failure. As anyone who has attempted to follow the Forgotten Realms language "system" can attest, that's something that you really have to put up the little finger quotes when you talk about it. A Regional Language is just a tribal language that at some point in time became influential enough that everyone in a region adopted it. That means that a Regional Language actually is "Orcish" -- it just means that the Orcs of that region kicked enough booty that everyone ended up learning Orc, and then in subsequent generations everyone spoke Orc and didn't even think that was weird. Heck, there might not be any Orcs in the area anymore. But everyone in that area will still speak Orc.

\subsubsection{Pidgins: Common and Undercommon}
\vspace*{-8pt}
\quot{"Orc take sword. I own sword. I tell you. I want sword. Orc give sword. I get sword. You tell orc."}

Common is not technically a language, it's a linguistic construct called a Pidgin. A Pidgin is a linguistic amalgamation that combines elements from several languages and has an extremely simple grammatical structure with no iterative capability. Whoa that's a lot of six-dollar words! The point is that all this stuff with relative clauses and structured inheritance that makes D\&D rulebooks read like a legal document is completely absent from a pidgin.

Pidgins form when people from different groups come together for trading purposes. So in the Underdark, Pidgin is pretty much just an extremely simplified version of the Drow version of Elvish. They trade with everyone, and speak in "Tourist Speak" where they speak very loudly and slowly in Drow and everyone has pretty much figured out what that means. Above ground, Common is mostly composed of the Halfling language, with a few loan words from other cultures thrown in.

The only reason that Common and Undercommon stay relatively static in D\&D is because the people actually doing the trading are crazy long lived and do the trading everywhere. The big traders in the D\&D economy are not those stupid caravans who wander around full of swag. No, it's Wizards and Outsiders who teleport expensive and wondrous stuff all over the planet. The reason why you can get a cup of coffee or a bolt of silk in your otherwise European villa is because people with teleportation are moving goods all over the place. So people who speak Common actually do share a common trade language with people clear on the other side of the planet. And they might not even know who the wizards in question are.

Interesting side note: People who grow up speaking a Pidgin as their only language actually speak a Creole, which is a real iterative language just like any other that is made out of the words of the original Pidgin. Human cultures in D\&D apparently default to Common as their primary language. That means that humans presumably speak Common as a language rather than as a pidgin. So the Wizards and Shadow Caravaners come to Human settlements from time to time and regard human speech as being filled with vulgar crazy-talk. The words are all there, but they have extra prepositions and jumble all the thoughts into single sentences.

\subsubsection{Language System I -- High Fantasy}

In the true High Fantasy setting, there are three languages on your continent, and no "Speak Language" skill.

Firstly, there is Common, which is what everyone speaks. Maybe people from far away speak a foreign, incomprehensible tongue, but it's foreign and incomprehensible and your characters don't speak it just because you an Intelligence of 12.

Secondly, there is The Old Tongue, which isn't spoken much, but is used in ancient writings and prophecies and such not. You can't have a speak language for this, to read it (or understand it), you need Decipher Script. This is what Decipher Script is for, since ancient script is generally in The Old Tongue. If you are a big bad ass elf you show off your many ranks in Decipher Script by peppering your speech with Old Tongue terms. If anyone asks, The Old Tongue is so complicated, full of subtle meaning and generally awesome that it can never be used for reliable communication.

Thirdly, there is The Dark Tongue, which is just like The Old Tongue from a game mechanical standpoint. To speak The Dark Tongue, you take The Old Tongue, change every other vowel into a hard consonant (a$\rightarrow$k, e$\rightarrow$t, i$\rightarrow$p, o$\rightarrow$g, u$\rightarrow$ch, y$\rightarrow$q), and all of the pauses (') become glottal stops (`). If you are a member of the evil political party, you pepper your speech with Dark Tongue words and phrases to prove how cool you are.
Ex.: L'rihylya'anyur cescelenti $\rightarrow$ L`rphylqa`knychr cesctlentp

The High Fantasy language system is about what you get from books like Shanarra or the Wheel of Endlessness cycle. It's also really easy, which is why it is in use by lazy authors. It also has the advantage that the Decipher Script skill has an obvious and explicable use (which let's face it: in standard D\&D it does not have, even deciphering magical writing is a Spellcraft check). People pretty much talk in English except when it's plot important that they be incomprehensible and everyone knows where everything stands. It's even less realistic than the basic rules, but it's closer to a lot of the important source material.

\subsubsection{Language System II -- Remotely Realistic}

Each major cultural group (e.g. Europe, China, India) has a language which they will call, more or less, "Classical" (e.g. Latin, Classical Han Chinese, Sanskrit, respectively.) This is the language that people will use for writing, it is also the language of discourse for travelers and the like. Classical is the only language with a meaningful written form, although you might find some scribbled notes or poems (e.g. the Golliard Poems, found at Carmina Burana) in a local, or vulgar, language. There are also pre-classical writings (e.g. Greek, Cuneiform, not to mention Old Slaan and Aboleth) which you will need Decipher Script to read.

You may or may not have mystical languages (Terran, Aquan, Celestial, etc.), if you do, it might be a good idea to have one of those serve in place of classical for one major cultural grouping or another. To save yourself trouble, assume that your world contained four great civilizations -- Northern, Southern, Eastern and Western. Each of these civilizations left behind a classical language, which is used for academic and administrative discourse in that region.

In this model, there is no "common" that is spoken by commoners. The tongue of the ancient Dwarvish Empire will be spoken by everyone in the Northern countries who is educated, but the uneducated commoners will speak all kinds of crazy local tongues (Wenn, Lapp, Prussian, etc.) and you may well have to turn to magical translation or local educated characters (such as the town wizard or a local aristocrat) in order to get your point across to the Plebes. This closely approximates the position that Latin had in medieval Europe or the position that Han Chinese had in medieval China.

\subsection{Spellbooks}
\vspace*{-8pt}
\quot{"Warning: may contain Explosive Runes."}

Long ago, a spellbook was an actual magical object. Magic Users could pop their book open, rip pages out and blast the contents out as magic scrolls. Their very method of spell preparation was to open this pile of dangerous magic items and concentrate on creating copies of the scrolls in their minds to be released later on as powerful magic. For those of you who are new to 3rd edition, that statement seems pretty weird, because it doesn't work that way at all anymore.

\subsubsection{Using Other Peoples' Spellbook}

You can pick up some other guy's spellbook and prepare spells out of them once you've deciphered the spell in question. The DC isn't even hard -- it's only 15+Spell Level (and you can take 10), so a high level character can't even fail. And by a "high level character," I actually mean a first level character if he has an Int Bonus of at least +2, which he does. With the extreme ease of using other peoples' spellbooks, one is tempted to ask why anyone ever makes a full scale copy of a spell -- that's crazy expensive.

The answer, of course, is that they don't. In reality all spells copied by powerful wizards are created with the secret page spell. That spell allows you to "hide" whatever is on a page with any writing you want -- even (specifically) spells. So the "fake page" is actually part of a spellbook, and the "real page" is probably just doodles of horses or tallies of wins and losses in Backgammon. Secret Page comes online for Wizards at level 5, so any Wizard of even modest power should be able to construct spellbooks in hours rather than days for zero gold pieces.

If you want, you can "master" another wizard's spellbook, at which point everything in it becomes just like you wrote it. The DC is 25 + Spell level

\subsubsection{Getting by Without a Spellbook}

People often assume that wizards carry their spellbooks with them at all times and that taking these books away from them will cripple their character beyond redemption. For low level adventuring wizards, this is essentially true. But for high level Wizards and wizards who don't adventure, nothing could be further from the truth.

Copies of spellbooks are astoundingly expensive -- but once characters enter the fabricate or wish economies of the upper levels, that cost is either meaningless or can be bypassed entirely (thanks secret page). Any decently high level wizard may well have dozens or hundreds of copies of that precious manuscript. That's why characters with a Wizard Mentor can copy spells for free -- the high level wizard literally just hands them a copy of any spell they figure out how to master.

\subsubsection{Arcane Magical Writings}

As written, a Wizard can learn a spell from any spellbook page or scroll she has deciphered. Deciphering a page or scroll is a spellcraft check that, among other things, tells you whether it is arcane or divine. That means that under the rules as written, a Wizard can take Cleric Scrolls and copy them into her spellbook and then they become Wizard spells of the same level. Honestly\ldots\  most DMs will not let you do that even though the PHB is extremely specific that that is exactly what you can do. But if it's really important to you to learn Cleric spells, you still can.

Many DMs put in the additional restriction that to learn a spell it must be Arcane, or even that it must be a Sor/Wiz spell. That's actually fine, because the world of D\&D includes Nagas, who cast Cleric spells as Sorcerer spells. They can make scrolls (or you can make a scroll with a Naga), and then you can learn those precious Cleric spells if you really care. Chances are, though, that you don't care. Clerics are much better than Wizards in every single aspect of their characters except in their spell-list. And while there are certainly some gems on the Cleric list as far as spells go, chances are if you wanted to build a character who casts those spells you'd be better off having been a Cleric in the first place. Have better hit points, Saves, and BAB. So while learning Cleric spells is probably a pretty stupid goal, it is definitely achievable no matter how strict your DM is.



\chapter{Dungeons of Note}

Sure, you've been to the sewers under the town, and the maze in the wilderness, and the cave that opens up into the Underdark, but when was the last time you went into a dungeon that you cared about in any way? Which was the last one that had some traction, some \emph{pizzazz}? Here are some sample dungeons that will stick in the players' minds long after they leave them:

\section{The Hall of Records}

It's where information goes to die, except that it never dies. Located in a distant corner of Baator, the Hall of Records is a timeless library that contains a wealth of information dating back to when only the Aboleth had an empire in the mortal world. The filing system is intensely baroque, and it requires more than good searching skills to find the document that you need. The layout of the complex is inherently evil and unhelpful, designed to hamper and ruin those who need its services. The extradimensional floor plan is highly confusing and moreover the noneuclidian geometry is run with substantial changes on each layer. The index can tell you approximately where you need to be, and the only way in or out is teleportation.

Every visit to The Hall of Records is unique, and the players don't really need to map it all out. The really neat part about the place though is that it is strongly opposed to Divination magic and timeless. This means that creatures can (and do) hide out in here for hundreds of years when they make a lot of enemies. Many of these vagabonds make permanent camps in various parts of the Hall of Records. They live a limited, hermit-like existence and react strongly when other creatures enter the areas they have claimed as their own.

\section{The Tomb of Iuchiban}

The world's greatest blood mage made a quite credible attempt to gain godhood and nearly succeeded. Actually killing him permanently was essentially out of the question (and completely pointless for a being of such incredible and unethical power), so he was imprisoned into a block of jade. That block of jade is further suspended in a lake of mercury in the center of a lattice of tunnels filled with the most dangerous traps that the greatest architect magician of the time could create. The nature of the construction suppresses and confuses shadowlands creatures, as well as conjuration and divination magic, making them more and more unreliable as you get farther towards the center. The original architect set the last traps from the very center of the tomb and could not himself escape, so he committed suicide right there next to the final prison. If you can get to the middle, you'll see him there and get to read his last thoughts, still preserved after all these years.

And while crawling your way through metal lined tunnels (to stop burrowing creatures) filled with imaginative lethal traps might seem like a bad thing, remember that your progress through the Tomb is essentially timed. Guards patrol the outside of the Tomb constantly, and the Empire will send people into the tomb if and only if Blood Speakers have broken that perimeter in an attempt to revive their lord. So whether your party is composed of Blood Speakers or Imperial Agents, the other team is also making its way through the maze, and if you don't get to the center in time, things will go badly for you. Taking 20 on Search may not be possible.

\section{The Garden of the Gretel the Snowshaper}

Long ago there was a 15th level Illusionist with access to several of the effects that increase the reality of shadow spells, allowing her to make 90\% real simulacrums of herself with 13 levels and some spare XP, who were also able to make simulacrums of her, which were therefore also able to do so, and so on and so on. When she was finally slain, she had already amassed an army of approximately 100 13th level copies of herself in her workshop located in a valley blanketed in a constant layer of snow. And each of the simulacrums is unable to gain levels, so they have nothing better to do with XP than just make magic items, constructs, and wondrous architecture with it. Each simulacrum is completely aware that it must follow the orders of another simulacrum farther up the chain towards the original Snowshaper, so each takes great pains to avoid talking to other simulacrums lest they be forced to follow potentially self-destructive commands.

The Garden today is so wrapped in Illusions that it appears to be a garden in truth. Fountains, hedges, and colorful birds stand in stark contrast with the icy and forbidding mountains that surround the valley. Thermal Illusions make the region feel balmy and warm, but in truth the area is so cold that exposed skin will become frostbitten in a short period of time (noticing this is happening requires a successful DC 24 Willpower Save to disbelieve the affected temperature). Gretel's palace appears as a fancy pagoda made of paper and wood, but in truth it is an edifice of ice carved through with tunnels. About 80 Gretel Simulations persist to this day, and they are still under orders to remain in the valley and make things. Each of them has hidden herself in sections of the castle or the surrounding gardens, attempting to fend off other lower numbered Gretels who could command them. Reactions to non-Gretel characters entering the valley are highly mixed, often constrained by the last orders they received when the Garden was still functioning properly.

\section{The Closed Shafts}

Dwarves and Kobolds dig tunnels deep into the roots of the mountains in an attempt to get access to the veins of gold and mithril that run through the earth's rocky heart. Particularly deep shafts often yield the best results, so the different teams sometimes have been known to sacrifice a bit of safety to push down as far as possible. Rivalry between the dig teams of the different races is intense and when the mine shafts break through into one another, battles often erupt over mined out territory. Shafts compromised by enemy forces are sometimes boobytrapped by either (or both) of the races, and the maps of the shafts become confused for both sets of foremen. Such was the case in the section now simply called ``The Closed Section" by Dwarf and Kobold alike.

Both the Dwarves and Kobolds had been digging into what promised to be an exceptionally rich vein of mithril ore, and had been playing the territorial control game against each other, breaking shafts through into the other's territory and trapping it. The dangerous, yet not unusual game was upset when the Dwarves hit water, flooding the lower sections and threatening to terminate the entire project. Undaunted, the Dwarves began setting up machinery to pump the water out. Once that started coming online, the Aboleth attacked. Having massively more power than the Dwarven miners, they quickly overwhelmed the lowest worker teams and shut the machines off. The rest of the Dwarves, seeing their compatriots converted into Skum, quickly withdrew.

The Kobolds, seeing the Dwarven presence weaken in the mines (and not knowing about the Aboleth forces), quickly moved in to secure territory, moving throughout the mine and setting up make-shift traps all along the route in order to damage the Dwarves' ability to move back into position. When they encountered the Aboleth territory, they too were turned into Skum and slaves, and the Kobolds relinquished their claims on the shafts as well.

There's a lot of mithril down there, but even the partial maps of the shafts that were possessed by the foremen of the Dwarf and Kobold teams have been lost. And now, the Aboleth's Skum forces are moving up into the other territory. Both the Dwarves and the Kobolds want someone to go down there and overwhelm the hostile forces long enough to get those machines back up and pumping.


\end{document}