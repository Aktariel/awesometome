\chapter{The Socialomicon : Heroes in the Greek Sense}
\vspace*{-10pt}
\quot{``Can I kill the baby kobolds?"}

When people are asked to name a historical point that D\&D most closely represents, they'll usually say something like ``The Middle Ages", or perhaps a date between 1000 AD and 1500 in Europe. Truth be told, to find a historical period which has a social setup anything like D\&D, you're going to have to go back. Way back. D\&D represents a period in history that is most closely identifiable with the Iron Age: the landscape is dotted with tribes and aspiring empires, the wilderness is largely unexplored, and powerful individuals and small groups can take over an area without having a big geopolitical hubbub about it.

The source material for the social setting of D\&D is not Hans Christian Andersen, it's Homer's The Iliad and Caesar's The Gallic Wars. In the backdrop of early historical empire building, crimes that modern humans shake their heads at the barbarity of are common place -- even among the heroes. D\&D at its core is about breaking into other peoples' homes, possibly killing the residents, and taking their stuff home with you in a sack. And in the context of the period, that is acceptable behavior for a hero.

\section{Living With Yourself After a Raid}

The goblins have gone and conducted a raid on your village in full force. They rode in, took a bunch of the sheep, killed some of the people, set fire to some of the cottages, and rode away again with Santa Sacks filled with this year's crop. And they laughed because they thought it was funny. And now that your elder brother has been slain you want to dedicate yourself to the eradication of the Goblin Menace and begin the training necessary to become a Ranger so that you can empty the goblin village from the other side of the valley once and for all.

Par for the course D\&D, right? Wrong! Killing all the goblins isn't just an Evil act, it's unthinkable to most D\&D inhabitants. This is the Classical Era, and actually sowing the fields of Carthage with salt is an atrocity of such magnitude that people will speak of it for thousands of years. In the D\&D world, goblins raid human settlements with raiding parties, humans raid goblin settlements with ``adventuring parties", and like the cattle raiding culture of Scotland, it's simply accepted by all participants as a fact of life.

When your city is raided by other groups of humanoids, it's a bad thing for your city. Orcs may kidnap some of your relatives and use them as slaves (or food), and many of your fellow villagers may lose their lives defending lives and property important to them. But that's part of life in the age, and people just sort of expect that sort of thing.

\section{Razing Hell: When Genocide is the Answer}

Sometimes in history there would come a great villain who just didn't get with the program. The Classical example is the Assyrians. Those bastards went around from city to city stacking heads in piles and levying 100\% taxation and such to conquered foes. They became\ldots\  unpopular, and eventually were destroyed as a people. That's the law of the jungle as far back as there are any records: if a group pushes things too far the rules of mercy and raiding simply stop applying. Goblins, orcs, sahuagin\ldots\  these guys generally aren't going to cross that line. But if they do, it's OK for the gloves to come off. In fact, if some group of orcs decides to kill everyone in your village while you're out hunting so that you come home to find that you are the last survivor, other humanoids (even other Evil humanoids like gnolls) will sign up to exterminate the tribe that has crossed the line.

Cultural relativism goes pretty far in D\&D. Acceptable cultural practices include some pretty over-the-top practices such as slavery, cannibalism, and human sacrifice. But genocide is still right out. That being said, some creatures simply haven't gotten with the program, and they are kill-on-sight anywhere in the civilized world or in the tribes of savage humanoids. Mindflayers, Kuo-Toans, and [Monster] simply do not play the same game that everyone else is playing, mostly because their culture simply does not understand other races as having value. And that means that even other Evil races want to exterminate those peoples as a public service. Like the Assyrians, they've simply pushed their luck too far, and the local hobgoblin king will let you marry his daughter if you help wipe them out of an area.

Solitary intelligent monsters often get into the same boat as the Kuo-Toans. Since the Roper really has no society (and possibly the most obscure language in Core D\&D), it's very difficult for it to understand the possible ramifications of offending pan-humanoid society. So now they've done it, and they really haven't noticed the fallout they are receiving from that decision. Ropers pretty much attack anything they see, and now everyone that sees a roper attacks them. In the D\&D worlds, ropers are on the brink of extinction and it probably never even occurs to them that their heavy tendrilled dealings with the other races have pushed them to this state.

\section{The Hands of the Divine}

In D\&D the gods are, compared to the wet radish that is your character, unlimited in power. There is no amount of whupass that you could put together that would allow you to triumph over Vecna -- he can cast any (or every) spell as a free action. He can cast ``Kill Drogor the Dwarf Barbarian with no Save", a spell which heretofore had not even been researched by anyone -- as a free action. And he knows many days in advance when he is going to be in danger and who he's going to be in danger from, so that's really not a battle you're going to win. Nevertheless, when adventurers come across a temple to Vecna, they kick everything over, they smash the idol and they steal its ruby eye. And they get away with it.

And that's because when you kick over temple to Vecna, you aren't going against Vecna in any direct sense. Vecna lives on a distant outer plane and has full control over anything that happens in his personal dominion. Anywhere else, and he's essentially playing a game of Populous. If there isn't a knight or prophet of Vecna around, Vecna really can't do much until the ``end-game" scenario in which he starts throwing volcanoes around. And as soon as that starts happening, the best bet is really to try to hop on the first portal out of whatever universe you happen to be in because it's going Armageddon pretty soon. Vecna might encourage some monsters to go look you up, or lower some land in your way, but you're an adventurer -- so that's pretty much what you expected out of life anyway.

\section{Temporal Authority in D\&D}
\vspace*{-8pt}
\quot{``Kill the dragon, marry the princess, rule the kingdom."}

D\&D is set in an essentially Iron Age setting. If your group (or even you personally) are known to be hardcore enough, you actually do rule the lands extending as far as you can reach. This doesn't mean that you don't need a bureaucracy, because there's still relatively little that you can do on your own. That administrative staff is necessary, it's there to tell people what you want them to do, and to tell you when they aren't doing it.

In fairy tales, as well as D\&D, the guy (or girl) who saves the kingdom by slaying the big monster marries the child of the local king. This is usually because the current king is himself a powerful dude with a PC class himself. His children may be aristocrats, and by marrying them off to a powerful adventurer who may well be able to take his kingdom by force, he's preserved his own position and kept his family from being set on fire. Nominally in this situation the crown is still in the previous king's family and moving to the next generation normally. You may even get a title like ``Prince Consort" or something -- but everyone knows that you are running the show because you can slay dragons. No one is going to say it, but the princess' only real job in this scenario is to\ldots\  keep you happy. And she's not even the only one that has that job. Surprisingly, the previous king is actually fine with that, because if his daughter has Aristocrat levels, that really is the best he can expect for her.

\subsection{The Basis for Hereditary Rule in D\&D}
\vspace*{-8pt}
\quot{``Why would I listen to an Aristocrat? I'm a frickin' Wizard. I can set his whole house on fire with my mind!"}

If being a badass makes you rule the kingdom, and it does, where do aristocrats come from? How could you possibly have anyone in command of anything whose job is to be pretty and otherwise useless? This question comes up in virtually every campaign when the players get to be about 9th level and notice that they can go a whuppin' and a whompin' on the local lords with impunity. The key is not that aristocrats have some sort of shiny pants powers that force people to take their rights of inheritance seriously despite inherent power discrepancy. The secret is the Apprentice and Mentor feats from the DMG2.

Here's the deal: when you have the Mentor Feat, you can teach children how to have whatever class you have, and they gain levels without actually doing anything dangerous themselves. This means that when you're a powerful adventurer and you want to pass on your legacy to your children, you can. They get to majority and they're already a 5th level character (and they can use all the sweet adventuring gear you've accumulated as a powerful adventurer). The son of the man who conquered the Bane Mires is also very hardcore, so having him announce that he should rule when his father passes on is pretty reasonable. Like real human history, this won't always work. Historically, the hand-off of kingship has resulted in open warfare almost half the time. In D\&D, prospective kings actually do have some inherent greatness so monarchies are way more stable in D\&D than they were in the real world -- but there will still be occasional guys who refuse to accept the new kid and try to take things over with their own spells or swords.

\subsection{The Place of Aristocrats}

Especially if the local hard-core lord is particularly fecund or has a short attention span, there is ample possibility that there will be more children and nieces and nephews and such in the next generation than the lords with real character classes can train to be hard core. Some lords don't even have the mentorship feat and can't pass on their awesomeness to any part of the next generation. When this happens, there will be one or more children brought up in a courtly manner who have no useful adventuring skills at all -- these unlucky creatures have levels in Aristocrat. They keep their wealth because they are related to people who can kill an elephant with a tea spoon. Their competent family members keep them around because they can be trusted to administer things fairly well on their behalf.

An Aristocrat's primary job in life is to get married to another aristocrat or better yet a hard core adventurer. In this manner they will get more family members who can slay dragons. That's important, because as soon as none of the living members of a house are powerful wizards or warriors -- the house gets its assets liquidated by an adventuring group of orcs or elves (depending upon what kind of house it was) and all the aristocrats in it who weren't killed outright have to run off into the night with whatever wealth they can hide inside their body. It's not pretty, so aristocrats spend a considerable amount of time trying to make themselves as pretty as possible -- anything they can do to make a high level Barbarian want to have children with them is something they'll do without a second thought.
