\chapter{Character Advancement: Power and Wealth}
\vspace{-12pt}

\quot{``Assuming that I make the use of most of our spells, I should be able to advance a circle of magic every week or so, which essentially means that the optimum solution to this difficulty to simply scare up minor tangential difficulties in the woods for two months so that I can go back in time and solve this problem retroactively.''}
\vspace{12pt}

\desc{While we're talking about magical items, we really have to talk about XP at the same time. And that's not just because the DMG asks us to pay small amounts of XP to create them. D\&D is based on two kinds of advancement: XP and GP. Both of them have failed, because we're actually playing a cooperative storytelling game and not Diablo multiplayer. We know that a high level guy can whack low level stuff again and again at virtually no risk, and that this can be repeated endlessly for levels. We know that people can take off downtime to just plain \emph{farm} to get GP endlessly. Seriously, ``XP Grind'' is extremely boring and players should not be exposed to it under any circumstances.}

\desc{Noone wants to hear about the time you threw a \emph{cloudkill} into a Satyr tavern and then teleported home so that you could try out the new spells that appeared in your book because you just dinged to 10th level. That's a story that is \emph{dumb}, and the current rules pretty much expect you to do it over and over again. If we're going to have a rational system for magic items, we can't have it work that way.}

\section{XP: Beer Me.}
\quot{``Boil an Anthill: Go Up One Level.''}

\desc{The rubrics for challenge and advancement as depicted in the DMG have to go. We've looked at them from every direction, and they don't work. At all. And no, I'm not talking about the classic problems like the variable difficulty inherent in fighting a giant scorpion (an interesting intellectual exercise for a 4th level horse archer or a brutal melee slugfest for a 14th level swordsman). That's a real problem, but we are talking about the basic structure of fighting monsters of increasing CR, getting increased piles of XP, and moving on with your life. That's got to end.}

\desc{Here's why: according to the DMG you are supposed to face about 4 equal-level challenges per day of adventuring. Further, going by the XP chart, your 4-person party will go up a level every time you defeat 13.3 of those encounters -� which is less than 4 days worth of encounters according to the first idea. So if you adventure ``like you're supposed to'' � you'll go up 2 levels a week. And of course, if you encounter less than 4 enemies a day, spell-slot characters like Wizards and Druids are crazy good. Essentially, this means that D\&D characters go from 1st level to 20th level in half the time as it takes to bring a pregnancy to term.}

\desc{Indeed, D\&D society is essentially impossible. Not because Wizards are producing expensive items with their minds or because high level Clerics can raise the dead � but because the character advancement posited in the DMG is so fast that it is literally impossible for anyone to keep tabs on what the society even is. High level characters are the military, economic, and social powerbases of the world. And they apparently rise from nothing in� about 2 \half months. That means that if a peasant goes home to plant his crops, then when he gets back to the city with his harvest in the fall the city will have seen the rise of a group of hearty adventurers who attempt to conquer the world and achieve godhood four times while he's gone. The city will have been conquered by a horde of Dao and sucked into the Elemental Plane of Earth and then returned to the prime material as a group of escaped Dao slaves achieved their freedom and themselves became powerful plane hopping adventurers who graduated to the Epic landscape. Then a team of renegade soldiers from the Dao army will have run off into the countryside and survived in the Spider Woods long enough to return with the Spear of Ankhut to return the city to the Dao Sultan in exchange for a gravy train of concubines and wishes. Then a squad of frustrated concubines will have turned on their masters and engaged in a web of intrigue culminating in the poisoning of the Dao Sultan with Barghest Bile and ultimately turned the city into a matriarchal magocracy run by ex-concubine sorceresses. So when the peasant returns with his harvest of wheat, he returns to� a black edifice of magical stone done up in Arabian styles and bedecked with weaponry from Olympus that is all controlled by epically subtle and powerful wizards who are themselves the masters of a setting created from the fallout of the destruction of a setting that is itself the fallout of the destruction of a setting that was in turn created out of the destruction of the setting that our peasant walked away from with a bag of grain come planting time last year.}

\desc{And while purely intellectual exercises in a universe that is essentially a giant lava lamp of crazy can be interesting, satisfying storytelling is impossible. If the players can't make lasting impact, the game has no meaning. And if players are seriously going from 1st to 20th in a single season, lasting impact of any kind is absurd to even contemplate. It behooves players and DMs to come to a consensus about how they want their campaign to be structured. There is no single best way to handle character advancement in a cooperative storytelling game, and there are a lot of ways to really piss off the other players at the table if you aren't all on the same page to begin with.}

\section{Reach for the Stars: Character Advancement}

\desc{All classic fantasy adventures take place in D\&D terms somewhere between 1st and 10th level. Seriously. Conan is like 3rd level, Theseus is about 3rd level too. Adventures for 13th level in literature of any kind are hard to come by and generally involve wearing capes or being a god. However, D\&D is not a game about modeling tales of legendary knights, skilled samurai, or barbarian chiefs � it is a game about adventuring in the world of D\&D. And in D\&D, characters do become 20th level, at which point they either become honorary Olympians or join the Justice League. Within that context, character advancement should follow a few basic principles:}

\desc{\listone
    \item\ability{Stagnant Characters are frustrating.}{That is, in a game which offers so much potential for advancement, it is frustrating to be in the position where you don't actually get to do any of it. Sure, in a game like Shadowrun there's no disappointment to be had from not being able to achieve godhood and in a game like Champions you don't need to advance your character at all to have a good time. But D\&D is a leveled system and not getting those levels makes us sad.}
    \item\ability{Advancement of Characters shouldn't destroy the setting.}{If you're playing a ``pirate game'' then you shouldn't get to the point where there is no longer a purpose served in piracy as long as you still play that game. Furthermore, you shouldn't be adverse to downtime on the grounds that waiting a month or two for a storm to go by will leave your enemies driving air cars powered by t-rexes on bicycles.}
    \item\ability{Players should be able to play with their toys.}{Too often, a character will get a shiny new trick only to go up in level and have no further use for it long before he has had a chance to actually use it. And that defeats the entire purpose of leveling up in the first place.}
    \item\ability{Characters should not be rewarded for doing stupid crap.}{Seriously. Your goal is to rescue the princess, so what should you do? Rescue the princess, or� run around the compound she's being held in punching out the baron's attack dogs? An army is heading for your city, should you sneak in and kill the enemy general or should you try to wrestle the army's horses one at a time?}
\end{list}}

\vspace{12pt}

\noindent \desc{This leads us to several conclusions of varying palatability:}

\subsubsection{Wealth By Level Has Got to Go.}

\desc{This hurts a lot of people, but it's true. If you can turn a pile of silver into increases to your natural armor bonus, the setting is going to be destroyed. Quite literally, and with crowbars. Fantasy settings are filled with bridges made of opal and castles faced with blue ice that sty forever cold and stuff. This fantastic scenery is awesome, and it contributes to the feel of fantasy that should permeate the cooperative stories we tell within a D\&D game. If player character power is determined by ``wealth'' in any directly measurable fashion, you can bank on PCs ripping all the expensive facing off the castles they conquer � and then we all lose.}

\desc{See, it's pragmatic and even sort of reasonable to rip the marble off the Great Pyramid at Giza and use it to build fancy houses in Cairo. But for all the future generations, it sucks. There really is a correlation here: if we don't allow people to trade blocks of marble for extra spells per day and more powerfully magical swords, then people will leave our pyramids alone. Otherwise, future generations will look at another unfaced ziggurat and wonder what wonders the ancient battlefields possessed before vandals came and destroyed our fantasy world.}

\subsubsection{Encounter XP Has Got to Go.}

\desc{XP rewards are a form of incentive towards heroic behavior. The problem is that individual challenges don't make things more heroic, they just make things more time consuming. By parting out XP per \emph{encounter} rather than per \emph{quest} the game is actually discouraging intelligent play. Avoiding difficulties is supposed to get you XP according to the DMG but we all know that doesn't actually happen in any game or published module.}

\desc{Adventurers respond very rapidly to incentives. If you give incentives for painstakingly stabbing minotaur after minotaur in the face the players will do that. If you incentivize running past the horde of minotaurs and rescuing the princess the players will do that instead. So if the XP comes from quest completion, players will \emph{complete quests}. If XP comes from Final Fantasy style XP dancing in the woods � the players will do that instead. Since one of them makes for awesome stories, and the other is a rote repetition of the worst kind of World of Warcraft nonsense, we know what has to be done.}

\begin{list}{}{\itemspace}
\item \subsubsection{A Little Note on XP Costs}
\item \desc{I know that you're probably thinking ``If XP rewards are handed out in a less per-diem manner, doesn't that mean that XP costs would be more noticeable and even actually have meaning?'' And of course the answer is ``yes''.... Sort of.}
\item \desc{The problem with XP costs isn't just that they don't really cost anything ``in the long run'' (which they don't), the problem is that they are bad for the game. Like Age increases before them, an XP cost is essentially running up a credit card bill. You get whatever it is that you were buying with the XP cost \desc{now}, and you pay \desc{later} (by death from old age or not going up in level when you otherwise would). That's never balanced, because there's no guaranty that the character in question will still be being played when that credit card comes due.}
\item \desc{So even though staggering XP gains out longer as suggested in this book [i]would[/i] make XP costs more meaningful than the hoax they are in the basic rules, we still strongly aadvise you to do away with them in your home games as we have in ours.}
\end{list}

\section{Strategies of Advancement}

\desc{Having determined the core problems with advancement in the manner described in the DMG, let's talk about some of the ways you could do it that might be satisfying. Like the handling of alignment and necromancy that we're talked of in the past, there really is no right answer -- it really depends upon what your group wants to do.}

\subsection{Steady State 1: Serial Heroism}

\quot{``We have another mission for you.....''}

\desc{Let's face it: in a lot of fictional source material, the characters don't really change between their adventures at all. In fact, that's kind of the \emph{point} of a lot of stories. The hero is the one fixed point and the story is just the fixed character reacting to different situations. You read about Conan or Hercules fighting the Moon Men or the Ice Jarls, but you don't really read the story set after Hercules got a laser gun and grew wings. Even the books where Conan is an old man rarely reference specific events from previous books.}

\desc{In the serial heroism campaign, characters begin play at the level that depicts their abilities appropriately. Characters have signature equipment and a collection of levels and skills that are integral to their character. Over the course of the adventure, the characters may well find new equipment and learn special crap and be blessed by Nymph Pools and whatever -� but at the beginning of the next adventure they will be back to exactly the same place they were last time. Even characters getting married or having limbs whacked off doesn't have any effect on the next episode.}

\desc{There are a lot of ways to explain this. Adventurers spend money profligately and put equipment into bat caves and bequeath magic swords to temples and favored wenches. Major wounds can be healed, and we all know how rarely things work out between men and women �- especially when one is a halfling rogue and the other is a giant iguana. You can either begin each episode by coming up with an amusing off-the-cuff answer to why you begin the next adventure just like you began the last one or you can just ignore it the way Saturday morning cartoons do. It's not a big problem.}

\desc{There are a lot of advantages to this sort of thing. If the characters already do what they are supposed to (generally about level 6 or so with a couple of standard magic items and an artifact), then advancement of any kind just makes the character less like himself. He Man didn't become a better show when Prince Adam got a plane -� it just lost focus. But there are pitfalls as well. Certainly it is the case that games like World of Warcraft or Everquest can be remarkably unsatisfying \emph{precisely} because no real accomplishment can occur. It is a fine line between a character not changing and a character's actions not mattering �- walking that line is sometimes quite difficult. Certainly, before such a sweeping change is implemented, very frank discussions must be had between players and the DM. The game is essentially now a series of once off adventures that happen to have the same characters in them.}

\desc{In the Serial Heroism game the character's core abilities are the same in every tale. That can be mythic. Like the Robin Hood songs. But it can also be retarded. Like the Smurfs.}

\subsection{Steady State 2: Trophy Hunting}

\quot{``So� where \emph{are} we putting the giant penny?''}

\desc{Characters like Conan and He-Man are pretty much the same between issues or episodes. But what of characters like Angel and Buffy who really do pick up and use equipment found in previous episodes? This is also a very plausible setup of ``nearly steady state'' storytelling with limited character development. The character stays relatively recognizable one adventure to another. Chapter after chapter goes by without the player ever growing wings, learning to fly, shooting laser eye beams or in any other way having obviously gained a level of Bard. Important plot points and devices are referenced in later installments, allowing the characters to use the Doom Glaive after they took it off the cooling body of Bruc Avec Piti'e both immediately in that adventure and subsequently in later adventures as well. While in the true Serial the characters would have destroyed the Doom Glaive at the end of the adventure, in the Trophy Hunting model it stays in the Bat Cave only until it is needed for a later adventure.}

\desc{In this model of steady state dynamics, the players gradually increase in power � though they do so in an asymmetric fashion that is not level dependent. This means that the amount of Ogres that the party can successfully dispatch \bolded{will} increase considerably over time. But it won't increase \emph{dramatically} and the players may never be able to take on a really hardcore monster like a Cranium Rat Swarm or a Pit Fiend.}

\desc{In this model then, it is expected that even the \emph{idea} of ``Wealth by Level'' be tossed in the trash. The players are literally gaining as many as infinity magic items per level because by and large they aren't going up levels \emph{at all}, while magic items are accumulating slowly. Characters can bathe in magic puddles that increase their stats or find statues that transform into giant frogs; but this can also happen pretty slowly and still be fine because players aren't being forced into situations where they necessarily face higher leveled opposition all the time.}

\subsection{Rapid Advancement: Level a Session}

\quot{``That was last week. This week I am a master of fire.''}

\desc{It is entirely plausible to play a game where the characters go up a level every adventure or even every session. While this sort of rapid advancement scenario is often dismissed as ``munchkin'', it actually does capture the feel of many stand-alone books and movies quite well. There are a lot of stories like The Wheel of Time or The Matrix which are actually ruined by having sequels at all �- they are much better as a single progression where the characters begin as youngsters who don't even know about the major Evil that threatens the world and progress briskly into becoming world straddling badasses who control reality with willpower alone.}

\desc{In this set up it is highly recommended that the DM hand out magic items like candy. After all, while the players are fighting hill giants today, they'll be up against a swarm of bloodfiend locusts next week and a rogue deva the week after that. The players will need new swag to face their new enemies just as they'll need new class abilities.}

\desc{Many players feel that this sort of play environment is simple minded, but really nothing could be further from the truth. In fact, players have no chance to get acquainted with their new abilities before they are laden with even newer abilities. With only a single adventure to make use of each new level of powers it is entirely possible that the Wizard will \emph{never} get a chance to use one or even both of the shiny new spells he picks up each level. Indeed, since both the characters and the opposition is coming up with more power and options each week, the game is actually \emph{really hard}.}

\desc{And that, ironically, is the most major drawback of this gaming style. Some of the players who gravitate the most towards this advancement system are actually the least able to successfully juggle a new class level and two new magic items every week. Sure, there are difficulties to be had in this scenario when players miss a session or three (nothing says ``suck'' like finding out that Fighter's girlfriend the sorceress cohort is actually a more powerful magician than you are). But that can be worked around in a number of ways: the DMG suggests giving out experience bonuses to people who fall behind until they catch up in level and that works well enough. Of course, to actually make use of that you'd have to chuck the idea of not being able to level more than once per session -� which makes characters even more confusing -� but there you go.}

\subsection{Attenuating Advancement: Diminished Returns.}

\quot{``You youngsters have no concept of how difficult it was to get the Doom Glaive.''}

\desc{If one considers advancement at face value: a direct method to prevent adventuring from becoming ``stale'', then it is entirely reasonable to question its inclusion in the game at all. After all, a sixth level party could very plausibly encounter a manticore, a summoning ooze, a dragon, a war party of ogres, a troll, an evil wizard, a dinosaur, a nymph, a mud slaad, a nerra facechanger, a medusa, a circle of myconid, a cathedral protected by a stained glass golem, a cadre of yak folk, an infestation of ash rats, a room full of hammerers, a spawn of Kyuss, or a dreadful cleric with some orcish minions. Or whatever. The point is, you could very plausibly face different opposition every week until half the players move out of town before you ever run out of monsters to fight. The staleness then, comes not at the hands of the players in any case, but for the DM. After all, once the DM has thrown the adventure where an ancient cathedral of Pelor has been taken over by an evil group of Yak Folk who have bound a Janni and forced her to tell them the secret password that allows them to break into the inner cloister without having the stained glass tear itself out of the wall and attack them in order to conduct a foul ritual to transform the daughter of the old king into a medusa and set up some zombie ogres to protect themselves while the mighty ritual commences � that leaves some of the DM's favorite monsters used up out of that level. More importantly however, the players are presenting essentially the same skill set so long as their skill set doesn't change � meaning that the DM can become bored finding challenges for the PCs unless the PCs demonstrably change over time.}

\desc{Be that as it may, the fact is that higher level characters with more magical swag have more abilities than do lower level characters and quite definitely present a face to team monster with more attachments on their Swiss Army knives.}