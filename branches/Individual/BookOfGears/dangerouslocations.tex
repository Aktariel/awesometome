\chapter{Dangerous Locations: When the Floor has a CR}

\desc{It is an undeniable truth that hunting goblins in a dank warren filled with dead falls and snares is both more exciting and more dangerous than hunting the same goblins in an open field. However, it must be stressed that the way 3rd edition D\&D has traditionally dealt with this -- to give CRs to individual traps as if they were enemy monsters in their own right -- is both unsatisfying and unplayable. The fact is that you probably are \emph{never} going to tell a story about the time your party was shot at by an arrow trap, it just isn't interesting in the same way that overcoming an evil necromancer or slaying a greedy dragon is.}

\desc{And why is that? It's essentially because an arrow trap is not an encounter, it's an \emph{attack}. Just a single salvo in an ongoing battle between you and the dungeon, not a battle in and of itself. And when looked at in that manner, the problem becomes obvious: a \spell{glyph of warding} is a single spell. Overcoming it is like making your saving throw against the same Cleric's \spell{hold person}, it's simply wildly inappropriate to stop the action and play the battle complete music at that point.}

So what do we do about it? Well, just as one does not stop and record a victory every time you bypass a summoned monster or overcome an opponent's thrown javelin, we shouldn't be worrying about the CR of individual traps. No, we should be concerned only with the CR of \emph{areas} that have traps in them. For one thing, this means that we don't have to have endless arguments over whether people should get the XP for bypassing the \spell{symbol of pain} on the door if they came in through the floor. For another, the act of avoiding \emph{that} stupid argument really helps to encourage characters to play things a bit smarter and not simply run through the ``Hallway of Leveling'' where they go up a level every thirteen traps and get to 20th level in less than 150 doors.

\section{Location CRs: Quality and Quantity}
\vspace*{-8pt}
\quot{``Why check the door? Maybe because there was a trap on \textbf{every single other door in this entire complex?!}''}

\desc{To a limited extent an area can become more dangerous by making traps more ubiquitous. We say a ``limited'' extent because there is a profound sense of diminishing returns when the chance of encountering a trap equals one. Our classic example is the Citadel of Fire, the castle that is the home of the Efreeti King. It's \emph{on fire}. Every square is \emph{on fire}. Every door is \emph{on fire}. And if you go there, \emph{you'll be on fire}. To an extent, that means that the kind of dangerous area that you might have seen in the Lizard Temple when you were 4th level is now every square on the battlemat. That's bad. But it's not unconquerably bad. It doesn't take a whole lot of Fire Resistance to survive in that kind of environment, and you don't have to be amazingly high level to get your grubby mitts on that kind of fire resistance. The fact that every single doorknob and chair is on fire in the Citadel essentially just means ``Only Adventurers with Fire Resistance can Adventure here'' or even ``You must be at least as tall as this sign to attack the Citadel.''}

\desc{And the effect would be pretty much the same if you just had to wade through a \emph{moat} of Fire. There are literally dozens of rooms in the Citadel of Fire that are on fire without this increasing the difficulty of your assault in any way. And that's OK. In fact, people would be slightly offended if large amounts of the Citadel of Fire were not in fact on fire, which would be the logical way to do it if you were handing out XP or construction costs on a per flaming room basis. It adds to the immersion to have some relatively homogenous fantasy environments.}

Practically speaking, this means that by the time you have put in enough of a single type of difficulty that the players will not plausibly be able to complete their quest without taking appropriate precautions, the CR of the location shouldn't rise any more by adding more of the same difficulty. And that goes for more than just places being on fire. If there are enough pressure plates linked to arrows that the PCs aren't going to get through alive without the Rogue taking 20 on her Search checks, throwing in some more arrow traps (or tripwires, or anything else that the Rogue can find and bypass by just taking the time to search thoroughly) doesn't make the area any more difficult. A cave at the bottom of the sea isn't any more difficult when it's \emph{completely} full of water than when it's \emph{mostly} full of water -- you still need \spell{water breathing} just to get there.

\section{WWMD? Disabling Traps.}
\vspace*{-8pt}
\quot{``A paperclip can be a wondrous thing. More times than I can remember, one of these has gotten me out of a tight spot.''}

\desc{The Disable Device Skill is extremely powerful and amazingly bizarre. You don't need it to bypass a trap, there are dungeons full of Kuo Toans who have no more Disable Device than you do who bypass traps every day. What Disable Device \emph{does} do is allow you to interfere with the mechanisms of mechanical and magical devices such that they don't get in your stuff when you \emph{don't} have access to the special catch or magic word or whatever it is that you're supposed to have. In short, any fool can press an off switch or simply not step on an on-switch; Disable Device allows you to shut things down \emph{without} access to those things.}

\desc{Once you have found a trap with the Spot skill, it requires no skill roll at all to simply walk around it. If you discover a pressure plate, you can normally expect to simply step or jump over it without even making a Disable Device check. What Disable Device let's you do is set the plate to not trigger if you do walk on it. Often that's pretty pointless, but sometimes it's pretty useful, especially if you're up against a "trap" that is a siege defense or hostile spell (such that its normal deactivation trigger is far away). Remember however, that you can still activate traps by any of a number of means without actually being in harm's way. Summoned monsters, tossed barrels and the ubiquitous 10' pole have been used by generations of adventurers to activate traps from 10' or more away. Again, that totally works and requires \emph{zero} ranks in disable device. However, sometimes you don't want a trap to go off at all or a trap can go off virtually limitless numbers of times -- that's where disable device comes in.}

\desc{So what counts as a device? Well\ldots\ \emph{everything}. Every mechanical or magical effect is a device. A \spell{Wall of Force} is a device as is a giant stone block that is set to fall down on a foolish intruder who breaks a trip wire. A character with sufficient Disable Device can successfully turn off any magical effect or prevent virtually any cause and effect chain from occurring. You can stop an avalanche (DC 15) even after it has begun (DC 35). You can remove any permanent magic effect, even curses like \spell{Cause Blindness} (DC 32). What you \emph{can't} do is disable instantaneous effects. \spell{Flesh to Stone}, therefore, is out of bounds for disabling, as is \spell{Wall of Stone}. Sorry, once an instantaneous effect has gone off, there's nothing left to disable.}

\desc{How does that work? I have no frickin idea. Rogues, Thief Acrobats, Ninjas, and Gadgeteers are capable of simply turning off \spell{Geas} and there's no physical explanation for how it is that they do it. The fact is that most of the devices in D\&D are beyond my understanding. I don't know how a \spell{symbol of death} works, I don't know how the magical energies stay in place for weeks or years until activated, so I don't know how a Ninja goes about making those magical energies dissipate harmlessly without entering the kill zone. I do know that he can do it, and if required I can make something up that sounds cool. That's a DM's job, after all.}

\abox{Item Spotlight: Bag of Flour}{The bag of flour can be used to disable any rune or sigil without meaningful risk. A magical rune can only detonate if it is uncovered. So if you throw some flour on it, the symbol can't ever explode and is now completely safe. You may want to put the flour on the end of a pole because moving your hand \emph{close} to a rune may trigger it before the flour lands.}

\section{I \emph{live} here: Setting off Traps}
\vspace*{-8pt}
\quot{``How did those gnolls run through that hallway if the whole thing collapses when people are in it?''}

\desc{The common conceit of trap placement is that they automatically go off against player characters who don't find them and automatically don't go off against Team Monster. Needless to say, that's ridiculous, and it actually harms the game when you implement it. While there are magical traps that are virtually guaranteed to go off against certain kinds of creatures and are nonetheless bypassable with something as simple as a command word, those are not PC/NPC selective. A command word bypassed Symbol will go off against any creature that doesn't say the magic word. That means that creatures without language capabilities like bears holding sharks or remorhazz will set those traps exactly as PCs who don't know any better would. It also means that any player character in the correct position can simply \emph{listen} for the command words that Goblins use when safely passing over the danger zone and use it themselves. The base DC is only 15 so the challenge here is actually getting into position to observe enemies bypassing magical traps rather than the replication of the technique itself. The bypass words on magical symbols are pretty forgiving, they can be spoken by blink dogs, Sahuagin and Xorn without serious risk of misunderstanding.}

But what of other traps? Mechanical traps go off mechanically, which means that to make them go off you have to \emph{do} something to make it go off. And that means that there is a chance that even someone who doesn't have a clue what they are doing might simply happen to not set off the trap. Life is filled with Mr. McGoos and if there is \emph{any} path to walk across an area without setting off a pressure plate there is a chance that people will happen to do so. And yet, if there isn't a way to move past a trap, there's a whole area that the residents of an area have to avoid altogether (or just be immune to the effect of the trap). Here are some common trap triggers:

\listone
	\bolditem{Opening a Door:}{This is a common and fun one because unless someone decides to go through the wall (and sometimes even then) the trap will go off any time the door is opened. This can either be placed on "fake" doors that the occupants have no intention of ever opening, or it can be put on doors that are used frequently if there is a separate switch to deactivate the trap (be sure to get buzzed in). The important part about this is that an opening trigger will go off any time the door is opened normally. If you cut a hole in the middle of the door and squeeze through it, you're probably safe. After all, the door itself is acting as a switch in this case, methods of entrance that don't literally involve turning that hinge often don't involve pulling the switch.}
	\bolditem{Tripping a Wire:}{Strings and wires can be strung in walkways at anything from ground to eye level. A trip wire sets off a trap when it is broken or pulled upon, and thus won't go off at all if creatures shorter than the wire run underneath it (barring polearms and the like). A tripwire lower to the ground is more likely to be randomly stepped over than is a higher tripwire, but less likely to be seen. Several trip wires can be run in tandem across a walkway to virtually guaranty that a passerby will sever them, but in doing so they become a lot more visible. In general, a trip wire can go off 25\% of the time when someone moves through its space and have a spot DC of 20, go off 50\% of the time and have a spot DC of 15, or go off 100\% of the time and have a spot DC of only 10. A trip wire can be severed without triggering the trap by holding both ends of the wire and slicing out the middle -- but this requires a Disable Device check (DC 20). Failure triggers the trap. A tripwire can be triggered from range by throwing a chair at the problem, or with an arrow (against projectile weapons a tripwire has an AC of 13, against a larger object such as a barrel or a couple of cabbages tied together the AC is negligible).}
	\bolditem{Pressing a Plate:}{Bizarrely complex mechanisms can be hidden inside of walls and a pressure plate is as good a manner as any to get those mechanisms up and working. I seriously don't have any idea what the mechanical pieces under the floor look like, and neither do you. And that's generally OK. Mostly players won't respond to pressure plates by breaking the floor or walls open to get at the clockwork (though that is a viable option), mostly players will gamely accept whatever fate the pressure plate has in store for them. Without tearing up the scenery, characters can disable a pressure plate with a Disable Device check (generally DC 20, though more awesome plates exist). Pressure plates can be disguised as regular floor and are often quite difficult to spot (DC 16-30). A pressure plate can be as small as a single out of place brick or floorboard and may go off quite rarely (1-5 times out of 20 when someone moves through the space), this has the advantage that characters ``in the know'' can step over it (though enemies are presented with the same option). Alternately, pressure plates can cover entire squares, being triggered automatically if any creature heavier than a specific cutoff enters the square. In any case, characters can fly over a pressure plate or climb along the wall and simply never activate it.}
	\bolditem{Getting Stabbed:}{The old ones are the good ones, and many a trap has been simply to put pointy bits on areas that a character might step on, touch, or fall into. One can with exaggerated care simply step over such things, but in the heat of battle this may be pretty difficult. A single caltrop or blade is rather unlikely for someone to step on (a 1 on a d20 unless the character is crawling or otherwise stepping on more of the square than one might expect), and can be quite difficult to find unless one is specifically looking for it (DC 18 to spot). An area covered with spikes, caltrops, or blades is generally pretty obvious (DC 5 to spot), but it is generally assumed that anyone who moves into a covered square will step on one unless they take some sort of precautions. Caltrop covered terrain is difficult terrain, and characters who move through it at faster than a \half speed walk are going to step on something they'd rather not unless they make a Reflex Save (DC 20). Characters standing in an area covered with caltrops or the like are denied their Dex bonus to AC unless they have 5 ranks in Balance or allow themselves to step on something every time they are attacked.}
	\bolditem{Offending a Glyph:}{Magical runes have at times been implied to have the power to determine a character's alignment, their level, their class, even what they've eaten recently. That's not good for anyone, and we cannot suggest that it be allowed. So here's what Runes do: first, they are constantly taking 20 on a Listen check. That means that you need to make a Stealth check DC 21 to sneak past one. It also means that they will, generally speaking, hear a command word to turn off or turn on. A Magic Rune can also have a detection spell imbedded in them, which last until the rune triggers. So a rune might be set to go off as soon as a source of ``Good" was brought to within 10 feet of the Rune. A Rune might also simply be set to go off whenever any creature moves through its area while it is active (being activated and deactivated with command words set when the rune is). The parameters of a rune can be determined with a DC 20 + Spell Level Knowledge (Arcana) check.}
\end{list}

\section{Facing the Architect: The CR of Locations}

\desc{When you adventure in a dangerous or exotic location you are essentially encountering the architect of that location. Each trap, obstacle, and danger of the region can be looked at as the contingent spells and attacks of the force that put that together. Sometimes a devious maze is engineered by a mad architect or fabricated by an elusive wizard and this is in fact literally true. Other times the Forest of Dread is just really dangerous on its own lookout and the only ``architect" involved is just the DM.}

\desc{The importance here is that an individual \spell{fire trap} isn't really an encounter. It's a single attack, and a pretty ineffective one at that. When the wizard tries to soften you up with his \spell{explosive runes}, that's a lot like the same wizard softening you up by conjuring some celestial badgers and sending them around the corner to engage your forces.}

So while we definitely do not suggest doing something dumb like giving out XP for each trap bypassed, we do encourage you to consider the traps in an area to collectively be an opponent. An opponent that spends a lot of time hiding and taking opportunistic attacks. The Kobold Warrens, for example, have a number of trip wires set to launch crossbow bolts at anyone tall enough to pass through them. In an ideal world, the trip wires would be fairly visible, but in the heat of battle characters may feel compelled to chase after kobolds through the strings.

\subsection{Structuring Encounters in a Day}

\desc{Challenge Ratings have a real utility as a DM, but do not substitute for having a decent idea of what your party is capable of. We're going to go back to the Giant Scorpion a few times, because it's a very poignant example, but we could just as easily be talking about Fairies or Elementals. The Monstrous Scorpion comes in a variety of CRs based on its size and overall awesomeness. Don't be fooled: in reality a monstrous scorpion is essentially of identical difficulty regardless of size based entirely upon what the players are capable of tactically. The Monstrous Scorpion has no intelligence, no ranged attacks, and no interesting abilities -- it's just a biological construct that happens to be exceptionally tough in its one-dimensional way. If you can simply get to longish range (or \emph{fly}) and use ranged attacks, you win. It'll take a while, but you will win. It doesn't really matter what level you are, or how strong your ranged attacks are, victory will be yours. On the other hand, if the Scorpion is presented as a closet troll, it'll mess you right up.}

What the CR grants you as DM then is a basic idea of how much ``resources'' an encounter is liable to use up. The Scorpion, for example, will use up a lot of arrows and not a small amount of time. It probably won't cause any damage if the players play it smart, but it will drag things out for a bit. Higher CRs will take a bite out of the arrows of higher level parties and so on. Still, the fact is that in no way will facing an appropriately CRed monster use up the 20\% of your resources specified by the DMG. Not at any level. What kinds of resources will be used up will depend upon the types of opposition:

\begin{list}{}{\itemspace}
\bolditem{Traps:} Trapped locations of an appropriate CR are generally speaking time sinks more than anything else. At levels 1-6, the characters will normally Search regions that are known to contain traps, which reduces the character's speed through the area to 5' per 6 seconds (about \half MPH or 0.9 KPH).
\item
\item So even though we're looking to completely toss the idea that players should actually \emph{get} anything for necessarily killing ``Ogre Thug \#2'' that doesn't mean that he shouldn't be there.
\item
\item As player characters become higher level they can take on more opposition. This does not necessarily mean they should be confronted with \emph{more powerful} opposition, but they should certainly encounter more of it. A Lunar Ravager and a Sand Giant are basically two large sized men with funny colored skin and a bad attitude. The fact that one is massively more powerful than the other is a staple of the D\&D system, but doesn't make an extremely exciting story. Having just looked up the stats of a Lunar Ravager and a Sand Giant I am confident that defeating a Sand Giant is a more difficult feat -- though of course it is not a more \emph{impressive} feat since as previously described both opponents are just 3 meter tall dudes with funny colored skin and a sword. Taking on 45 bug bears, which is something the stronger party could easily accomplish is however much more impressive than defeating 15 gnolls, as would be a light romp for the party who might otherwise face the Lunar Ravager.
\item
\item It is therefore important to note that parties should generally speaking not run into level appropriate opposition until quite late in an adventure. It's fine for a boss to be a True Fiend, Wizard, or Androsphix who is 2 or 3 CRs higher than the average character level in the party, but the vast majority of opposition should be several levels lower and a crap tonne more numerous than the PCs. This isn't just because this sort of thing keeps cleaving and \spell{fireballs} as reasonably viable tactics, but because high level combats really do involve lots of participants on both sides of the combat kicked out of the battle from time to time and if there's only one enemy it gets really anticlimactic.
\end{list}