\section{[Metamagic] Feats} \label{feats:metamagic}

As a spellcaster's knowledge of magic grows, she can learn to cast spells in ways slightly different from the ways in which the spells were originally designed or learned. Preparing and casting a spell in such a way is harder than normal but, thanks to metamagic feats, at least it is possible.  Spells modified by a metamagic feat use a spell slot higher than normal. This does not change the level of the spell, so the DC for saving throws against it does not go up.

%Wizards and Divine Spellcasters: Specific class reference, changed to Preparation Based Casters
\ability{Preparation Based Spellcasters:}{Wizards and divine spellcasters must prepare their spells in advance. During preparation, the character chooses which spells to prepare with metamagic feats (and thus which ones take up higher\textendash level spell slots than normal).}

%Sorcerors and Bards: Specific class reference, changed to Spontaneous Casters
\ability{Spontaneous Casters:}{Sorcerers and bards choose spells as they cast them. They can choose when they cast their spells whether to apply their metamagic feats to improve them. As with other spellcasters, the improved spell uses up a higher\textendash level spell slot. But because the sorcerer or bard has not prepared the spell in a metamagic form in advance, he must apply the metamagic feat on the spot. Therefore, such a character must also take more time to cast a metamagic spell (one enhanced by a metamagic feat) than he does to cast a regular spell. If the spell's normal casting time is 1 action, casting a metamagic version is a full\textendash round action for a sorcerer or bard. (This isn't the same as a 1\textendash round casting time.)

For a spell with a longer casting time, it takes an extra full\textendash round action to cast the spell.
Spontaneous Casting and Metamagic Feats: A cleric spontaneously casting a cure or inflict spell can cast a metamagic version of it instead. Extra time is also required in this case. Casting a 1\textendash action metamagic spell spontaneously is a full\textendash round action, and a spell with a longer casting time takes an extra full\textendash round action to cast.}

\ability{Effects of Metamagic Feats on a Spell:}{In all ways, a metamagic spell operates at its original spell level, even though it is prepared and cast as a higher\textendash level spell. Saving throw modifications are not changed unless stated otherwise in the feat description.

The modifications made by these feats only apply to spells cast directly by the feat user. A spellcaster can't use a metamagic feat to alter a spell being cast from a wand, scroll, or other device.

Metamagic feats that eliminate components of a spell don't eliminate the attack of opportunity provoked by casting a spell while threatened. However, casting a spell modified by Quicken Spell does not provoke an attack of opportunity.

Metamagic feats cannot be used with all spells. See the specific feat descriptions for the spells that a particular feat can't modify.}

\ability{Multiple Metamagic Feats on a Spell:}{A spellcaster can apply multiple metamagic feats to a single spell. Changes to its level are cumulative. You can't apply the same metamagic feat more than once to a single spell.}

\ability{Magic Items and Metamagic Spells:}{With the right item creation feat, you can store a metamagic version of a spell in a scroll, potion, or wand. Level limits for potions and wands apply to the spell's higher spell level (after the application of the metamagic feat). A character doesn't need the metamagic feat to activate an item storing a metamagic version of a spell.}

\ability{Counterspelling Metamagic Spells:}{Whether or not a spell has been enhanced by a metamagic feat does not affect its vulnerability to counterspelling or its ability to counterspell another spell.}

\begin{multicols}{2}

\metafeat{Empower Spell [Metamagic]}
{All variable, numeric effects of an empowered spell are increased by one\textendash half.
Saving throws and opposed rolls are not affected, nor are spells without random variables. An empowered spell uses up a spell slot two levels higher than the spell's actual level.}\\

\metafeat{Enlarge Spell [Metamagic]}
{You can alter a spell with a range of close, medium, or long to increase its range by 100\%. An enlarged spell with a range of close now has a range of 50 ft. + 5 ft./level, while medium\textendash range spells have a range of 200 ft. + 20 ft./level and long\textendash range spells have a range of 800 ft. + 80 ft./level. An enlarged spell uses up a spell slot one level higher than the spell's actual level.
Spells whose ranges are not defined by distance, as well as spells whose ranges are not close, medium, or long, do not have increased ranges.}\\

\metafeat{Extend Spell [Metamagic]}
{An extended spell lasts twice as long as normal. A spell with a duration of concentration, instantaneous, or permanent is not affected by this feat. An extended spell uses up a spell slot one level higher than the spell's actual level.}\\

\metafeat{Heighten  Spell [Metamagic]}
{A heightened spell has a higher spell level than normal (up to a maximum of 9th level). Unlike other metamagic feats, Heighten Spell actually increases the effective level of the spell that it modifies. All effects dependent on spell level (such as saving throw DCs and ability to penetrate a lesser globe of invulnerability) are calculated according to the heightened level. The heightened spell is as difficult to prepare and cast as a spell of its effective level.}\\

\metafeat{Maximize Spell [Metamagic]}
{All variable, numeric effects of a spell modified by this feat are maximized. Saving throws and opposed rolls are not affected, nor are spells without random variables. A maximized spell uses up a spell slot three levels higher than the spell's actual level.
An empowered, maximized spell gains the separate benefits of each feat: the maximum result plus one\textendash half the normally rolled result.}\\

\metafeat{Quicken  Spell [Metamagic]}
{Casting a quickened spell is a free action. You can perform another action, even casting another spell, in the same round as you cast a quickened spell. You may cast only one quickened spell per round. A spell whose casting time is more than 1 full round action cannot be quickened. A quickened spell uses up a spell slot four levels higher than the spell's actual level. Casting a quickened spell doesn't provoke an attack of opportunity.
\shortability{Special:}This feat can't be applied to any spell cast spontaneously (including sorcerer spells, bard spells, and cleric or druid spells cast spontaneously), since applying a metamagic feat to a spontaneously cast spell automatically increases the casting time to a full\textendash round action.}\\

\metafeat{Silent  Spell [Metamagic]}
{A silent spell can be cast with no verbal components. Spells without verbal components are not affected. A silent spell uses up a spell slot one level higher than the spell's actual level.
\shortability{Special:}{Bard spells cannot be enhanced by this metamagic feat.}}\\

\metafeat{Still  Spell [Metamagic]}
{A stilled spell can be cast with no somatic components.
Spells without somatic components are not affected. A stilled spell uses up a spell slot one level higher than the spell's actual level.}\\

\metafeat{Widen  Spell [Metamagic]}
{You can alter a burst, emanation, line, or spread shaped spell to increase its area. Any numeric measurements of the spell's area increase by 100\%.A widened spell uses up a spell slot three levels higher than the spell's actual level.
Spells that do not have an area of one of these four sorts are not affected by this feat.}\\

\end{multicols}