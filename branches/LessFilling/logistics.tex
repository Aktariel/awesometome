
\section{Logistics and Dragons}
\vspace*{-10pt}
\quot{``A tiger fights with claws, a dragon fights with fire. An army fights with rice."}

Does your character have chalk? Is it written on his character sheet?\\

The level of detail given over to what characters have in their pockets and saddle bags varies tremendously from game to game. And that's fine. There is nothing objectively wrong with characters keeping track of every single quart of oil that passes through your character's hands, just as there is nothing wrong with hand waving all non-magical equipment. In fact, when characters start interacting with the wish economy it is perfectly OK to handwave a character's minor magic items (we'll just assume that a 16th level Ranger has a wand of cure light wounds in his boot -- it's seriously not worth keeping track of).

So with such a wide array of perfectly reasonable and enjoyable ways to play the game, why bring it up at all? Well, the fact is that ultimately you need to find out what level of detail your DM wants to deal in. To be honest, I find that I usually don't even use notes and simply keep salient campaign information in my head. So I don't require players to write down how many pitons they have. Other DMs write it all down and have an index card that states how many towels are in individual laundry hampers when the players burst in the door. Both work.

The point is that this can cause very real arguments between people if they aren't on the same page. Like most other aspects of role playing gaming, it should be hammered out exactly what you're doing before you start playing.

\subsection{The Demographics of D\&D}
\vspace*{-8pt}
\quot{``If you can bludgeon an elephant to death with a teddy bear on a stick, fill in the bubble labeled `yes'."}

The breakdown of characters with class levels and their levels in society at large in the DMG is almost exactly wrong in every respect. Think about it: when you think of the powerful people in the world, how many of them are Wizards? Now, how many of them are Fighters? The truth is, that everyone who is 15th level is actually of roughly equal power and capable of influencing the world to a roughly similar extent. The reason that the high-end world is shaped so much more by Wizardly activity than it is by powerful swordsmen is because the vast majority of high level characters are spellcasters!

The reason for this is simple: NPCs go up levels when they are in situations appropriate to their class, not for overcoming challenges like Player Characters do. NPC dragons go up levels by just living a long time, NPC necromancers can go up levels by sitting around a musty tomb reading ancient tomes, and NPC Fighters go up levels by Participating in Major Wars. One of these things is not like the other, and the end result is that the high end of the NPC world is primarily populated by Dragons and Wizards -- NPC Fighters can become high level, but not by doing incredibly safe things so most of the time they don't. So when the eight most powerful NPCs come together and form a council for world rulership or something, chances are very good that every single one of them is some kind of spellcaster. This isn't because a 20th level Fighter isn't a hardcore dude, it's because NPC Fighters rarely survive in the environment required to become 20th level, while NPC Wizards often do.

Furthermore, the population density overall has little relevance to the number or level of powerful characters in a region. Indeed, some of the harshest environments have only highly leveled characters in them. The deeper you go into Moil or the Banemires, the less likely you are to run into a humanoid, and the more likely any humanoid you do meet is to be a total badass. So I'm sorry, there isn't a simple rubric to determine the highest level character in a region or the level spread of said characters (indeed, Necromancers persist notably longer when they become more powerful and the level distribution is a reverse bell-curve with a local minimum at 6th level). It would be nice to say that there was, but that just isn't so.

\subsection{Leaders of Men}
\vspace*{-8pt}
\quot{``You can only breathe fire every couple of seconds, I have so many tiny men that you cannot win."}

Not all campaigns will want to deal with a character's baggage train and camp followers. Certainly it can be quite a pain to try to keep track of a small army of soldiers in the middle of a continuous dungeon crawl. As such, any [Leadership] feat is completely optional. Some games simply won't use Leadership feats in any capacity, and that's fine.

Further, there are a lot of potential ways to get your army on (Influence is based on your Diplomacy, Artifice is based on your Craft skill, Command is based on your BAB, Necromancy and Summoning are both based on your highest castable spell level), and there's no specific reason that you wouldn't be able to have more than one. Except of course, that it can be extremely confusing to try to play with large sources of PC-led armies. So there is another common house rule that limits each character to no more than one Leadership feat.

In any case, if Leadership is allowed at all, there are some ground rules. First of all, no Cohort should ever be more or less than 2 levels lower than the PC. Ever. So if someone has a cohort that's something dumb like an Iron Golem, it's got to advance so that its CR advances in line with the character's level. Cohorts that can't be excused doing that aren't appropriate cohorts. Secondly, followers are traditionally of the crappy classes (Warrior, Expert, Aristocrat), and that's why followers are given appropriate CRs like \half rather than levels like ``1".

\begin{table}[tbh]
\begin{small}
\begin{tabular}{lr|rrrrrrrrrrrrr}
\multicolumn{2}{r}{Leadership} & \multicolumn{12}{c}{Followers by CR}\\
\multicolumn{2}{r}{Score}&\half&1&1\half&2&2\half&3&4&5&6&7&8&9&10\\\hline
&1&1&&&&&&&&&&&&\\
&2&1&&&&&&&&&&&&\\
&3&1&&&&&&&&&&&&\\
&4&2&&&&&&&&&&&&\\
&5&2&1&&&&&&&&&&&\\
&6&3&1&&&&&&&&&&&\\
&7&4&2&1&&&&&&&&&&\\
&8&6&3&1&&&&&&&&&&\\
&9&8&4&2&1&&&&&&&&&\\
&10&12&6&3&2&1&&&&&&&&\\
&11&16&8&4&3&2&1&&&&&&&\\
&12&20&10&5&4&3&2&1&&&&&&\\
&13&30&15&9&8&7&6&3&1&&&&&\\
&14&42&21&10&9&8&7&3&1&&&&&\\
&15&56&28&14&13&12&11&5&2&1&&&&\\
&16&70&35&17&16&15&14&7&3&1&&&&\\
&17&80&40&20&19&18&17&8&4&2&1&&&\\
&18&90&45&22&21&20&19&9&4&2&2&&&\\
&19&100&50&25&24&23&22&11&6&3&2&1&&\\
&20&120&60&30&28&26&24&12&6&4&2&2&&\\
&21&150&75&35&30&28&26&13&7&5&3&2&1&\\
&22&175&80&40&35&30&28&14&7&5&3&2&2&\\
&23&200&100&50&40&35&30&15&8&5&4&3&2&1\\
&24&250&125&60&50&40&35&17&8&6&4&3&2&2\\
&25&275&130&65&60&50&40&20&10&6&4&4&3&2\\
&26&300&150&75&65&60&50&25&12&6&5&4&3&3\\
&27&350&175&80&75&65&60&30&15&7&5&4&3&3\\
\end{tabular}
\end{small}
\end{table}



So those tiny men could just as easily be CR \half trained dogs or CR \half Kobold Warriors. Whatever.

\subsection{Administering your People}
\vspace*{-8pt}
\quot{``Alright, let's hear it\ldots\  for Me!"}

So you've slain the dragon and the local hobgoblin clan has arranged an elegant wedding between you and the most beautiful daughter of the daimyo\ldots\  and then what? Most of the people you just became the ruler of are commoners. That doesn't mean that they have the Commoner class -- holy crap was that thing a bad idea all around. No, in fact, we're phasing that class out completely. No, it means that your people mostly come with a humanoid hit die that if they ever tried hard enough would be replaced by a character class of some kind.

But while these people can end up with a character class, they aren't going to. Your peeps are pretty much useless, and you've got to accept that. Some of them will have their humanoid hit dice transferred out to be Experts or Warriors -- but those classes only go up to 5th level and aren't good. These guys pay taxes and need to get rescued. Really, that's why they are there.

\subsection{Conquered People}
\vspace*{-8pt}
\quot{``Spare the man in the kangaroo suit, he amuses me."}

Sometimes you just don't get along with people at all. Sure, if you kill the local Remorhazz, the locals will probably make you the lord. This is basically short hand for the fact that most people realize that you could overpower society, but right now at least you're fighting for society and they want to make sure it stays that way. But sometimes they don't. Either they already have a lord or they just really don't like you.

But you can still make yourself master of these ingrates by actually overpowering their society. Smack their lords around, beat their guards in combat, and crush their fortifications and temples beneath your sandaled feat. And then\ldots\  you have a bunch of people that resent you. Sure, you can buy some of them over with promises of power over their own people, and if you rule a land for a generation the children will grow up not even knowing the taste of any lash but yours. And you could even be a kind king and make your people prosper and such, and some people will respect you for that.

But policing and improving an economy filled with people that resent your presence is expensive. Whether you're trying to bribe them into loyalty or just sending guards to execute trouble makers this is simply much less efficient than getting occasional voluntary taxation. In fact, it's roughly half as lucrative as administering non-confrontational civilians.

Furthermore, regardless of how nice you're being or how hard you come down on potential troublemakers, there will be heroes who come to kick you out. Maybe they just refuse to take your filthy halfling lucre, maybe they object to your daily executions for failure to work hard enough. Whatever, the actual injustice of your regime has little to do with how often it is overthrown in the D\&D world. The point is that in addition to getting less taxes, you have to deal with a stream of hostile adventurers. And that's why so many groups just don't bother -- after beating the lizardfolk in the field, most adventurers just sack the temple and move on.

\subsection{Why We Fight}
\vspace*{-8pt}
\quot{``People on the left\ldots\  we hate the people on the right."}

There are lots of reasons to kill other people, and all of them can seem like a good idea. You can raise the sword for religious differences (anything from ``they eat people" to ``those people give Pelor's blessing with the wrong number of fingers"), conquering territory, stealing swag, killing peoples you don't like, etc.

The key here is that no matter what you're fighting for, it's a lot easier to get an army together if you can put a good spin on whatever it is that you're doing. Sure, some creatures will fight for small piles of gold -- but most want either really large piles of gold or even better -- assurances that they are doing the best possible thing by potentially throwing their lives away to kill other people. Heck, most creatures that would be willing to fight for gold alone would just as soon attack a creature offering to pay them gold to get the gold as fight on that creature's behalf to get the gold as payment.

What this means is that bribing creatures to switch sides is generally really hard. It's not just that creatures are generally adverse to switching sides, it's that for a creature to fight on any side they probably already rationalized killing other creatures for that team and it's going to take a lot to change their mind
