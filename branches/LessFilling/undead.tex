\section{New Rules for Undead}

The interaction of Undead with the rest of the rules is often less than satisfactory. Part of this is that the undead type itself is extremely overzealous in the game effects it provides. The fact that all undead don't need sleep means that vampires don't have to sleep in their coffins. The fact that undead don't have a Constitution score means that Ghouls can run for exactly zero rounds before they have to make a Con check (that they automatically fail) to continue (and also says it can ``run on indefinitely", a base contradiction that makes us sad). The fact that undead are immune to critical hits means that a vampire can't be staked through the heart (even if it was sleeping, which it isn't). But even beyond that fundamental error, the multitude of authors that compromise the Dungeons and Dragons design staff never seemed to get on the same page as to exactly what being undead means -- so a surprisingly large number of contradictory statements pepper the products.

And I'm not just talking about how they made an entire Deathless Type when there's already Ghosts (Alignment: Any) right in the core rules.

\subsection{Subtypes}

The obvious, and slickest, way to handle the excesses of the Undead type would be to simply rewrite the Undead type with a lot less in it and throw down a number of subtypes (mindless for skeletons, amorphous for shadows, and ponderous for zombies) to put in the abilities that each type of undead is supposed to have. But polls have shown that people aren't willing to play with optional rules that do that -- but perversely they are willing to add new subtypes to monsters to remove rules instituted by the base template. I don't know why, but DMs are honestly more likely to use an additional subtype that removes an inappropriate game effect from a monster than they are to use a modified base type that doesn't have the inappropriate effect in the first place. So that's how we're going to do it here.


\subsubsection{Dark Minded (subtype)} \label{undead:darkminded}

Undead creatures with an intelligence score have an intelligence that can be influenced, though they are dead and cannot be influenced by appeals to emotion. A dark minded creature has the following traits:
\listone
\item Not immune to mind affecting affects.
\item Immune to morale and fear effects.
\item Heals normally
\item Any Bluff, Diplomacy, or Intimidate attempts to influence a dark minded creature are made with a -10 penalty.
\item A Dark Minded creature continues to advance in age categories, growing older and wiser over time. It does not accrue any penalties to its attributes for advancing in age categories, and a Dark Minded creature has no maximum age.
\item Sample creatures with the [Dark Minded] subtype:
Liches,
Nightshades, and
Vampires
\end{list}

\subsubsection{Unliving (subtype)} \label{undead:unliving}

An Unliving creature is an undead that mimics many of the capacities of a living creature without truly being alive. An unliving creature has the following game effects:

\listone \item Unliving creatures have a metabolism of sorts, and thus have a Constitution score.
\item Unliving creatures require food (often blood or flesh) and sleep, and are vulnerable to magical sleep effects even if they are otherwise immune to mind affecting effects.
\item Unliving creatures have at least one vital organ, and are subject to critical hits from attackers with at least one rank in Knowledge (Religion).
\item Not destroyed upon reaching 0 it points, though its existence still ends if it reaches -10 as normal.
\item Subject to subdual damage, but can benefit from the Regeneration ability as normal.
\item Sample creatures with the [Unliving] subtype:
Ghouls,
Necropolitan, and
Vampires
\end{list}


\subsection{Undead and Aging}


Undead don't age. They don't get any older or more decrepit over time, that's the whole point. A creature with the undead type does not grow older at all, unless further modified by the Dark Minded subtype. This probably should have been in the Monster Manual.


\subsection{Becoming Undead}

The basic rules for transforming into Undead were never intended to be playable by player characters. And thus it is unsurprising that the legions of the damned are not only unsatisfying, but actually unplayable when placed in a game. The following are templates that can be added to a character to make them into an Undead without actually changing their Level Adjustment. If a player wants to explore the legendary powers available to some of these creatures, they are encouraged to take Prestige Classes available to undead or to take one or more [Undead] feats that can grant the character these abilities within the normal level progression context. Each undead creature type has access to a special class that characters may take to advance their special abilities.

\subsubsection{Revenants}\label{undead:revenant}
\vspace*{-8pt}
\quot{``Fear me first before all other evils under the heavens. Before even Death, for I am hatred and do not die."}

A revenant is the victim of a murder driven to avenge their own death. A game master might allow a character to return from the dead as a revenant if their character died in a particularly unfair fashion or if their character had a lot left to do.

\listone
    \itemability{Type:}{The character's type changes to Undead and the character's former type becomes a subtype with the ``augmented" modifier. The character also gains the Dark Minded subtype.}
    \itemability{Hit Dice:}{The character's BAB, Saves, and skills are all unaffected. The character must reroll his Hit Points, but every hit die is a d12.}
    \itemability{Ability Scoress:}{The character loses his Constitution score.}
    \itemability{Alignment:}{The character's alignment changes to Lawful.}
    \itemability{Special Qualities:}{The character cannot be turned, but may be rebuked. The character heals completely at the setting of the sun, unless he is in a Tomb or hallowed area. This healing can even bring him back from destruction, but if his body is nailed to the ground (or in a Tomb or hallowed area), he can never come back from the dead by any means.}
    \itemability{Level Adjustment:}{+0}
\end{list}

\subsubsection{Vampires}\label{undead:vampire}
\vspace*{-8pt}
\quot{``An eternity of loneliness and betrayal is, ultimately, an eternity."}

A vampire is an unliving mockery of life that lives by cruelly consuming the blood of the innocent. Only characters slain by a vampire's Constitution Drain rise as vampires, and even then only if they have 5 hit dice or more. Characters with less hit dice become monstrous vampire spawn and do not retain their abilities.

\listone
    \itemability{Type:}{The character's type changes to Undead and the character's former type becomes a subtype with the ``augmented" modifier. The character also gains the Dark Minded and Unliving subtypes.}
    \itemability{Hit Dice:}{The character's Hit Dice, BAB, Saves, and skills are all unaffected.}
    \itemability{Ability Scoress:}{The character gains a +2 bonus to his Strength and Charisma.}
    \itemability{Alignment:}{The character's alignment changes to Evil.}
    \itemability{Special Attacks:}{The character can drain blood from a helpless or willing victim, inflicting 2 points of Constitution Drain per round. The character heals 5 points for each point of Constitution drain in this way, and consuming 4 points of Constitution from intelligent creatures is considered enough ``food" for one day (and the vampire gains no sustenance from any other food). Humanoids slain by this Constitution Drain may rise as vampires or vampire spawn (though the character has no control over them unless granted by another ability).}
    \itemability{Special Qualities:}{The character gains Turn Resistance +2. The character suffers 2d6 damage and is considered staggered every round he is exposed to direct sunlight. This damage cannot be healed by any means until the character is in a place with no light at all (such as a coffin). A vampire character is vulnerable to Light effects.}
    \itemability{Level Adjustment:}{+0}
\end{list}\vspace*{8pt}

Vampires may take levels in \hyperref[class:vampireparagon]{Vampire Paragon}.

\subsubsection{Ghouls}\label{undead:ghoul}
\vspace*{-8pt}
\quot{``The flesh of heroes reeks of their strength in death even as it is embodied in life. The taste is exquisite beyond description. As you quiver there and watch my meal, I want you to know that I allow you to live only in the hope that you can get word to more who think they have the strength to end my reign of terror."}

Ghoul Fever is a horrifying illness that incites an almost insatiable craving for the flesh of humanoids. Characters with at least 2 class levels brought to zero Constitution by Ghoul Fever find their constitution restored and begin their unlife as Ghouls. Characters with less than 2 class levels simply die and rot.

\listone
    \itemability{Type:}{The character's type changes to Undead and the character's former type becomes a subtype with the ``augmented" modifier. The character also gains the Dark Minded and Unliving subtypes.}
    \itemability{Hit Dice:}{The character's Hit Dice, BAB, Saves, and skills are all unaffected.}
    \itemability{Ability Scores:}{The character's Dexterity increases by +2.}
    \itemability{Alignment:}{The character's alignment changes to Evil.}
    \itemability{Special Attacks:}{The character gains a bite attack that inflicts an amount of damage appropriate to her size. She also is a carrier of Ghoul Fever.}
    \itemability{Special Qualities:}{The character gains Turn Resistance of +2. The character cannot eat anything other than raw meat (vegetables or cooked foods are forcefully vomited up, leaving the character sickened for an hour), and her total dietary requirements are not reduced.}
    \itemability{Level Adjustment:}{+0}
\end{list}\vspace*{8pt}


Ghouls may take levels in \hyperref[class:ghoulparagon]{Ghoul Paragon}.

\subsubsection{Swordwraith} \label{undead:swordwraith}
\vspace*{-8pt}
\quot{``I remain\ldots\ because I like to kill."}

Mercenaries devoted strongly enough to a life of war that they carry on in death their endless campaign of destruction. A character slain in battle may return as a Swordwraith if his services were hired under false pretenses or if his exploits were particularly impressive before his life finally ended (at the discretion of the DM).

Swordwraiths appear somewhat insubstantial and have faintly glowing eyes, but they are not truly incorporeal and their eyes do not produce enough light to modify vision penalties.

\listone
    \itemability{Type:}{The character's type changes to Undead and the character's former type becomes a subtype with the ``augmented" modifier. The character also gains the Dark Minded and Unliving subtypes.}
    \itemability{Hit Dice:}{The character's Hit Dice, BAB, Saves, and skills are all unaffected.}
    \itemability{Skills:}{The character gains a +2 bonus to his Hide and Move Silently checks.}
    \itemability{Alignment:}{The character's alignment is unchanged.}
    \itemability{Special Qualities:}{The character gains Turn Resistance +2.}
    \itemability{Level Adjustment:}{+0}
\end{list}\vspace*{8pt}

Swordwraiths may take levels in \hyperref[class:swordwraithparagon]{Swordwraith Paragon}.
