%The entry for some of the special abilities, especially ones that are also conditions or class features, read more like a damn list than a series of paragraphs.  So thats what I turned them into.  It's easier to read that way anyway.

\section{Special Abilities}

A special ability is either extraordinary, spell-like, or supernatural in nature.

\ability{Extraordinary Abilities (Ex):}{Extraordinary abilities are nonmagical. They are, however, not something that just anyone can do or even learn to do without extensive training. Effects or areas that negate or disrupt magic have no effect on extraordinary abilities.}

\ability{Spell-Like Abilities (Sp):}{Spell-like abilities, as the name implies, are spells and magical abilities that are very much like spells. Spell-like abilities are subject to spell resistance and dispel magic. They do not function in areas where magic is suppressed or negated (such as an antimagic field).}

\ability{Supernatural Abilities (Su):}{Supernatural abilities are magical but not spell\textendash like. Supernatural abilities are not subject to spell resistance and do not function in areas where magic is suppressed or negated (such as an antimagic field). A supernatural ability's effect cannot be dispelled and is not subject to counterspells. See the table below for a summary of the types of special abilities.}

\begin{table}
\begin{tabular}[h!]{l|ccc}
\multicolumn{4}{c}{\textbf{Table: Special Ability Types}} \\
				& Ex & Sp & Su \\ \hline
Dispel & No & Yes & No \\
Spell resistance & No & Yes & No \\
Antimagic field & No & Yes & Yes \\
Attack of opportunity & No & Yes & No \\ \hline
\multicolumn{4}{p{3in}}{\textit{Dispel:} Can dispel magic and similar spells dispel the effects of abilities of that type?} \\
\multicolumn{4}{p{3in}}{\textit{Spell Resistance:} Does spell resistance protect a creature from these abilities?} \\
\multicolumn{4}{p{3in}}{\textit{Antimagic Field:} Does an antimagic field or similar magic suppress the ability?} \\
\multicolumn{4}{p{3in}}{\textit{Attack of Opportunity:} Does using the ability provoke attacks of opportunity the way that casting a spell does?} \\
\end{tabular}
\end{table}

\subsection{Ability Score Loss}

Various attacks cause ability score loss, either ability damage or ability drain. Points lost to ability damage return at the rate of 1 point per day (or double that if the character gets complete bed rest) to each damaged ability, and the spells lesser restoration and restoration offset ability damage as well. Ability drain, however, is permanent, though restoration can restore even those lost ability score points.

\vspace*{10pt}

\noindent While any loss is debilitating, losing all points in an ability score can be devastating.

\listone
	\item Strength 0 means that the character cannot move at all. He lies helpless on the ground.
	\item Dexterity 0 means that the character cannot move at all. He stands motionless, rigid, and helpless.
	\item Constitution 0 means that the character is dead.
	\item Intelligence 0 means that the character cannot think and is unconscious in a coma\textendash like stupor, helpless.
	\item Wisdom 0 means that the character is withdrawn into a deep sleep filled with nightmares, helpless.
	\item Charisma 0 means that the character is withdrawn into a catatonic, coma\textendash like stupor, helpless.
\end{list}

\vspace*{10pt}

Keeping track of negative ability score points is never necessary. A character's ability score can't drop below 0. Having a score of 0 in an ability is different from having no ability score whatsoever.

Some spells or abilities impose an effective ability score reduction, which is different from ability score loss. Any such reduction disappears at the end of the spell's or ability's duration, and the ability score immediately returns to its former value.

If a character's Constitution score drops, then he loses 1 hit point per Hit Die for every point by which his Constitution modifier drops. A hit point score can't be reduced by Constitution damage or drain to less than 1 hit point per Hit Die.

The ability that some creatures have to drain ability scores is a supernatural one, requiring some sort of attack. Such creatures do not drain abilities from enemies when the enemies strike them, even with unarmed attacks or natural weapons.

\subsection{Antimagic}

An antimagic field spell or effect cancels magic altogether. An antimagic effect has the following powers and characteristics.

\listone
	\item No supernatural ability, spell\textendash like ability, or spell works in an area of antimagic (but extraordinary abilities still work).
	\item Antimagic does not dispel magic; it suppresses it. Once a magical effect is no longer affected by the antimagic (the antimagic fades, the center of the effect moves away, and so on), the magic returns. Spells that still have part of their duration left begin functioning again, magic items are once again useful, and so forth.
	\item Spell areas that include both an antimagic area and a normal area, but are not centered in the antimagic area, still function in the normal area. If the spell's center is in the antimagic area, then the spell is suppressed.
	\item Golems and other constructs, elementals, outsiders, and corporeal undead, still function in an antimagic area (though the antimagic area suppresses their spellcasting and their supernatural and spell\textendash like abilities normally). If such creatures are summoned or conjured, however, see below. 
	\item Summoned or conjured creatures of any type, as well as incorporeal undead, wink out if they enter the area of an antimagic effect. They reappear in the same spot once the field goes away.
	\item Magic items with continuous effects do not function in the area of an antimagic effect, but their effects are not canceled (so the contents of a bag of holding are unavailable, but neither spill out nor disappear forever).
	\item Two antimagic areas in the same place do not cancel each other out, nor do they stack.
	\item Wall of force, prismatic wall, and prismatic sphere are not affected by antimagic. Break enchantment, dispel magic, and greater dispel magic spells do not dispel antimagic. Mage's disjunction has a 1\% chance per caster level of destroying an antimagic field. If the antimagic field survives the disjunction, no items within it are disjoined.
\end{list}

\subsection{Blindsight and Blindsense}

Some creatures have blindsight, the extraordinary ability to use a nonvisual sense (or a combination of such senses) to operate effectively without vision. Such sense may include sensitivity to vibrations, acute scent, keen hearing, or echolocation. This ability makes invisibility and concealment (even magical darkness) irrelevant to the creature (though it still can't see ethereal creatures). 

\listone
	\item Blindsight operates out to a range specified in the creature description.
	\item Blindsight never allows a creature to distinguish color or visual contrast. A creature cannot read with blindsight.
	\item Blindsight does not subject a creature to gaze attacks (even though darkvision does).
	\item Blinding attacks do not penalize creatures using blindsight. 
	\item Deafening attacks thwart blindsight if it relies on hearing.
	\item Blindsight works underwater but not in a vacuum.
	\item Blindsight negates displacement and blur effects.
\end{list}
	
\ability{Blindsense:}{Other creatures have blindsense, a lesser ability that lets the creature notice things it cannot see, but without the precision of blindsight. The creature with blindsense usually does not need to make Spot or Listen checks to notice and locate creatures within range of its blindsense ability, provided that it has line of effect to that creature. Any opponent the creature cannot see has total concealment (50\% miss chance) against the creature with blindsense, and the blindsensing creature still has the normal miss chance when attacking foes that have concealment. Visibility still affects the movement of a creature with blindsense. A creature with blindsense is still denied its Dexterity bonus to Armor Class against attacks from creatures it cannot see.}

\subsection{Breath Weapon}

A creature attacking with a breath weapon is actually expelling something from its mouth (rather than conjuring it by means of a spell or some other magical effect). Most creatures with breath weapons are limited to a number of uses per day or by a minimum length of time that must pass between uses. Such creatures are usually smart enough to save their breath weapon until they really need it.

\listone
	\item Using a breath weapon is typically a standard action.
	\item No attack roll is necessary. The breath simply fills its stated area.
	\item Any character caught in the area must make the appropriate saving throw or suffer the breath weapon's full effect. In many cases, a character who succeeds on his saving throw still takes half damage or some other reduced effect.
	\item Breath weapons are supernatural abilities except where noted.
	\item Creatures are immune to their own breath weapons.
	\item Creatures unable to breathe can still use breath weapons. (The term is something of a misnomer.)
\end{list}

\subsection{Charm and Compulsion}

Many abilities and spells can cloud the minds of characters and monsters, leaving them unable to tell friend from foe�or worse yet, deceiving them into thinking that their former friends are now their worst enemies. Two general types of enchantments affect characters and creatures: charms and compulsions.

Charming another creature gives the charming character the ability to befriend and suggest courses of actions to his minion, but the servitude is not absolute or mindless. Charms of this type include the various charm spells. Essentially, a charmed character retains free will but makes choices according to a skewed view of the world.

\listone
	\item A charmed creature doesn't gain any magical ability to understand his new friend's language.
	\item A charmed character retains his original alignment and allegiances, generally with the exception that he now regards the charming creature as a dear friend and will give great weight to his suggestions and directions.
	\item A charmed character fights his former allies only if they threaten his new friend, and even then he uses the least lethal means at his disposal as long as these tactics show any possibility of success (just as he would in a fight between two actual friends).
	\item A charmed character is entitled to an opposed Charisma check against his master in order to resist instructions or commands that would make him do something he wouldn't normally do even for a close friend. If he succeeds, he decides not to go along with that order but remains charmed.
	\item A charmed character never obeys a command that is obviously suicidal or grievously harmful to her.
	\item If the charming creature commands his minion to do something that the influenced character would be violently opposed to, the subject may attempt a new saving throw to break free of the influence altogether.
	\item A charmed character who is openly attacked by the creature who charmed him or by that creature's apparent allies is automatically freed of the spell or effect.
\end{list}

\vspace*{10pt}

Compulsion is a different matter altogether. A compulsion overrides the subject's free will in some way or simply changes the way the subject's mind works. A charm makes the subject a friend of the caster; a compulsion makes the subject obey the caster.

Regardless of whether a character is charmed or compelled, he won't volunteer information or tactics that his master doesn't ask for.

\subsection{Cold Immunity}

A creature with cold immunity never takes cold damage. It has vulnerability to fire, which means it takes half again as much (+50\%) damage as normal from fire, regardless of whether a saving throw is allowed, or if the save is a success or failure.

\subsection{Damage Reduction}

Some magic creatures have the supernatural ability to instantly heal damage from weapons or to ignore blows altogether as though they were invulnerable.

The numerical part of a creature's damage reduction is the amount of hit points the creature ignores from normal attacks. Usually, a certain type of weapon can overcome this reduction. This information is separated from the damage reduction number by a slash. Damage reduction may be overcome by special materials, by magic weapons (any weapon with a +1 or higher enhancement bonus, not counting the enhancement from masterwork quality), certain types of weapons (such as slashing or bludgeoning), and weapons imbued with an alignment. If a dash follows the slash then the damage reduction is effective against any attack that does not ignore damage reduction.

Ammunition fired from a projectile weapon with an enhancement bonus of +1 or higher is treated as a magic weapon for the purpose of overcoming damage reduction. Similarly, ammunition fired from a projectile weapon with an alignment gains the alignment of that projectile weapon (in addition to any alignment it may already have).

Whenever damage reduction completely negates the damage from an attack, it also negates most special effects that accompany the attack, such as injury type poison, a monk's stunning, and injury type disease. Damage reduction does not negate touch attacks, energy damage dealt along with an attack, or energy drains. Nor does it affect poisons or diseases delivered by inhalation, ingestion, or contact. 

Attacks that deal no damage because of the target's damage reduction do not disrupt spells.
Spells, spell\textendash like abilities, and energy attacks (even nonmagical fire) ignore damage reduction.
Sometimes damage reduction is instant healing. Sometimes damage reduction represents the creature's tough hide or body,. In either case, characters can see that conventional attacks don't work.

If a creature has damage reduction from more than one source, the two forms of damage reduction do not stack. Instead, the creature gets the benefit of the best damage reduction in a given situation. 

\subsection{Darkvision}

Darkvision is the extraordinary ability to see with no light source at all, out to a range specified for the creature. Darkvision is black and white only (colors cannot be discerned). It does not allow characters to see anything that they could not see otherwise�invisible objects are still invisible, and illusions are still visible as what they seem to be. Likewise, darkvision subjects a creature to gaze attacks normally. The presence of light does not spoil darkvision.

\subsection{Death Attacks}

In most cases, a death attack allows the victim a Fortitude save to avoid the affect, but if the save fails, the character dies instantly.

\listone
	\item Raise dead doesn't work on someone killed by a death attack. 
	\item Death attacks slay instantly. A victim cannot be made stable and thereby kept alive.
	\item In case it matters, a dead character, no matter how she died, has \textendash 10 hit points.
	\item The spell death ward protects a character against these attacks.
\end{list}

\subsection{Disease}

When a character is injured by a contaminated attack touches an item smeared with diseased matter, or consumes disease\textendash tainted food or drink, he must make an immediate Fortitude saving throw. If he succeeds, the disease has no effect�his immune system fought off the infection. If he fails, he takes damage after an incubation period. Once per day afterward, he must make a successful Fortitude saving throw to avoid repeated damage. Two successful saving throws in a row indicate that he has fought off the disease and recovers, taking no more damage.

These Fortitude saving throws can be rolled secretly so that the player doesn't know whether the disease has taken hold.
Disease Descriptions

Diseases have various symptoms and are spread through a number of vectors. The characteristics of several typical diseases are summarized on Table: Diseases and defined below.

\ability{Disease:}{Diseases whose names are printed in italic in the table are supernatural in nature. The others are extraordinary.}

\ability{Infection:}{The disease's method of delivery�ingested, inhaled, via injury, or contact. Keep in mind that some injury diseases may be transmitted by as small an injury as a flea bite and that most inhaled diseases can also be ingested (and vice versa).}

\ability{DC:}{The Difficulty Class for the Fortitude saving throws to prevent infection (if the character has been infected), to prevent each instance of repeated damage, and to recover from the disease.}

\ability{Incubation Period:}{The time before damage begins.}

\ability{Damage:}{The ability damage the character takes after incubation and each day afterward.}

\ability{Types of Diseases:}{Typical diseases include the following:

\listone
	\item \ability{Blinding Sickness:}{Spread in tainted water.}
	\item \ability{Cackle Fever:}{Symptoms include high fever, disorientation, and frequent bouts of hideous laughter. Also known as ``the shrieks.''}
	\item \ability{Demon Fever:}{Night hags spread it. Can cause permanent ability drain.}
	\item \ability{Devil Chills:}{Barbazu and pit fiends spread it. It takes three, not two, successful saves in a row to recover from devil chills.}
	\item \ability{Filth Fever:}{Dire rats and otyughs spread it. Those injured while in filthy surroundings might also catch it.}
	\item \ability{Mindfire:}{Feels like your brain is burning. Causes stupor.}
	\item \ability{Mummy Rot:}{Spread by mummies. Successful saving throws do not allow the character to recover (though they do prevent damage normally).}
	\item \ability{Red Ache:}{Skin turns red, bloated, and warm to the touch.}
	\item \ability{The Shakes:}{Causes involuntary twitches, tremors, and fits.}
	\item \ability{Slimy Doom:}{Victim turns into infectious goo from the inside out. Can cause permanent ability drain.}
\end{list}}
 
\begin{table}
\begin{tabular}[h!]{l|cccc}
\multicolumn{5}{c}{\textbf{Table:} Diseases} \\
Disease                         & Infection & DC & Incubation & Damage                     \\ \hline
Blinding sickness               & Ingested  & 16 & 1d3 days   & 1d4 Str\textsuperscript{1} \\
Cackle fever                    & Inhaled   & 16 & 1 day      & 1d6 Wis                    \\
Demon fever                     & Injury    & 18 & 1 day      & 1d6 Con2                   \\
Devil chills\textsuperscript{3} & Injury    & 14 & 1d4 days   & 1d4 Str                    \\
Filth fever                     & Injury    & 12 & 1d3 days   & 1d3 Dex, 1d3 Con           \\
Mindfire                        & Inhaled   & 12 & 1 day      & 1d4 Int                    \\
Mummy rot\textsuperscript{4}    & Contact   & 20 & 1 day      & 1d6 Con                    \\
Red ache                        & Injury    & 15 & 1d3 days   & 1d6 Str                    \\
Shakes                          & Contact   & 13 & 1 day      & 1d8 Dex                    \\
Slimy doom                      & Contact   & 14 & 1 day      & 1d4 Con\textsuperscript{2} \\ \hline
\multicolumn{5}{p{7in}}{\textsuperscript{1} Each time the victim takes 2 or more damage from the disease, he must make another Fortitude save or be permanently blinded.} \\
\multicolumn{5}{p{7in}}{\textsuperscript{2} When damaged, character must succeed on another saving throw or 1 point of damage is permanent drain instead.} \\
\multicolumn{5}{p{7in}}{\textsuperscript{3} The victim must make three successful Fortitude saving throws in a row to recover from devil chills.} \\
\multicolumn{5}{p{7in}}{\textsuperscript{4} Successful saves do not allow the character to recover. Only magical healing can save the character.} \\
\end{tabular}
\end{table}

\subsubsection{Healing a Disease}

Use of the Heal skill can help a diseased character. Every time a diseased character makes a saving throw against disease effects, the healer makes a check. The diseased character can use the healer's result in place of his saving throw if the Heal check result is higher. The diseased character must be in the healer's care and must have spent the previous 8 hours resting.

Characters recover points lost to ability score damage at a rate of 1 per day per ability damaged, and this rule applies even while a disease is in progress. That means that a character with a minor disease might be able to withstand it without accumulating any damage.

\subsection{Energy Drain and Negative Levels}

Some horrible creatures, especially undead monsters, possess a fearsome supernatural ability to drain levels from those they strike in combat. The creature making an energy drain attack draws a portion of its victim's life force from her. Most energy drain attacks require a successful melee attack roll�mere physical contact is not enough. Each successful energy drain attack bestows one or more negative levels on the opponent. A creature takes the following penalties for each negative level it has gained.

\listtwo
	\item \textendash 1 on all skill checks and ability checks.
	\item \textendash 1 on attack rolls and saving throws.
	\item \textendash 5 hit points.
	\item \textendash 1 effective level (whenever the creature's level is used in a die roll or calculation, reduce it by one for each negative level).
\end{list}
	
\vspace*{10pt}
	
If the victim casts spells, she loses access to one spell as if she had cast her highest\textendash level, currently available spell. (If she has more than one spell at her highest level, she chooses which she loses.) In addition, when she next prepares spells or regains spell slots, she gets one less spell slot at her highest spell level. 

Negative levels remain for 24 hours or until removed with a spell, such as restoration. After 24 hours, the afflicted creature must attempt a Fortitude save (DC 10 + 1/2 attacker's HD + attacker's Cha modifier). (The DC is provided in the attacker's description.) If the saving throw succeeds, the negative level goes away with no harm to the creature. The afflicted creature makes a separate saving throw for each negative level it has gained. If the save fails, the negative level goes away, but the creature's level is also reduced by one.

A character with negative levels at least equal to her current level, or drained below 1st level, is instantly slain. Depending on the creature that killed her, she may rise the next night as a monster of that kind. If not, she rises as a wight. A creature gains 5 temporary hit points for each negative level it bestows (though not if the negative level is caused by a spell or similar effect).

\subsection{Etherealness}

Phase spiders and certain other creatures can exist on the Ethereal Plane. While on the Ethereal Plane, a creature is called ethereal. Unlike incorporeal creatures, ethereal creatures are not present on the Material Plane.

Ethereal creatures are invisible, inaudible, insubstantial, and scentless to creatures on the Material Plane. Even most magical attacks have no effect on them. See invisibility and true seeing reveal ethereal creatures.
An ethereal creature can see and hear into the Material Plane in a 60\textendash foot radius, though material objects still block sight and sound. (An ethereal creature can't see through a material wall, for instance.) An ethereal creature inside an object on the Material Plane cannot see. Things on the Material Plane, however, look gray, indistinct, and ghostly. An ethereal creature can't affect the Material Plane, not even magically. An ethereal creature, however, interacts with other ethereal creatures and objects the way material creatures interact with material creatures and objects.

Even if a creature on the Material Plane can see an ethereal creature the ethereal creature is on another plane. Only force effects can affect the ethereal creatures. If, on the other hand, both creatures are ethereal, they can affect each other normally.
A force effect originating on the Material Plane extends onto the Ethereal Plane, so that a wall of force blocks an ethereal creature, and a magic missile can strike one (provided the spellcaster can see the ethereal target). Gaze effects and abjurations also extend from the Material Plane to the Ethereal Plane. None of these effects extend from the Ethereal Plane to the Material Plane.

Ethereal creatures move in any direction (including up or down) at will. They do not need to walk on the ground, and material objects don't block them (though they can't see while their eyes are within solid material).
Ghosts have a power called manifestation that allows them to appear on the Material Plane as incorporeal creatures. Still, they are on the Ethereal Plane, and another ethereal creature can interact normally with a manifesting ghost. Ethereal creatures pass through and operate in water as easily as air. Ethereal creatures do not fall or take falling damage. 

\subsection{Evasion and Improved Evasion}

These extraordinary abilities allow the target of an area attack to leap or twist out of the way. Rogues and monks have evasion and improved evasion as class features, but certain other creatures have these abilities, too.

\listone
	\item If subjected to an attack that allows a Reflex save for half damage, a character with evasion takes no damage on a successful save. 
	\item As with a Reflex save for any creature, a character must have room to move in order to evade. A bound character or one squeezing through an area cannot use evasion.
	\item As with a Reflex save for any creature, evasion is a reflexive ability. The character need not know that the attack is coming to use evasion.
	\item Improved evasion is like evasion, except that even on a failed saving throw the character takes only half damage.
\end{list}

\subsection{Fast Healing}

A creature with fast healing has the extraordinary ability to regain hit points at an exceptional rate. Except for what is noted here, fast healing is like natural healing. 

\listone
	\item At the beginning of each of the creature's turns, it heals a certain number of hit points (defined in its description).
	\item Unlike regeneration, fast healing does not allow a creature to regrow or reattach lost body parts.
	\item A creature that has taken both nonlethal and lethal damage heals the nonlethal damage first.
	\item Fast healing does not restore hit points lost from starvation, thirst, or suffocation.
	\item Fast healing does not increase the number of hit points regained when a creature polymorphs.
\end{list}

\subsection{Fear}

Spells, magic items, and certain monsters can affect characters with fear. In most cases, the character makes a Will saving throw to resist this effect, and a failed roll means that the character is shaken, frightened, or panicked.

\ability{Shaken:}{Characters who are shaken take a \textendash 2 penalty on attack rolls, saving throws, skill checks, and ability checks.}

\ability{Frightened:}{Characters who are frightened are shaken, and in addition they flee from the source of their fear as quickly as they can. They can choose the path of their flight. Other than that stipulation, once they are out of sight (or hearing) of the source of their fear, they can act as they want. However, if the duration of their fear continues, characters can be forced to flee once more if the source of their fear presents itself again. Characters unable to flee can fight (though they are still shaken).}

\ability{Panicked:}{Characters who are panicked are shaken, and they run away from the source of their fear as quickly as they can. Other than running away from the source, their path is random. They flee from all other dangers that confront them rather than facing those dangers. Panicked characters cower if they are prevented from fleeing.}

\ability{Becoming Even More Fearful:}{Fear effects are cumulative. A shaken character who is made shaken again becomes frightened, and a shaken character who is made frightened becomes panicked instead. A frightened character who is made shaken or frightened becomes panicked instead.}

\subsection{Fire Immunity}

A creature with fire immunity never takes fire damage. It has vulnerability to cold, which means it takes half again as much (+50\%) damage as normal from cold, regardless of whether a saving throw is allowed, or if the save is a success or failure.

\subsection{Gaseous Form}

Some creatures have the supernatural or spell\textendash like ability to take the form of a cloud of vapor or gas.
Creatures in gaseous form can't run but can fly. A gaseous creature can move about and do the things that a cloud of gas can conceivably do, such as flow through the crack under a door. It can't, however, pass through solid matter. Gaseous creatures can't attack physically or cast spells with verbal, somatic, material, or focus components. They lose their supernatural abilities (except for the supernatural ability to assume gaseous form, of course).

\listone
	\item Creatures in gaseous form have damage reduction 10/magic. Spells, spell\textendash like abilities, and supernatural abilities affect them normally. Creatures in gaseous form lose all benefit of material armor (including natural armor), though size, Dexterity, deflection bonuses, and armor bonuses from force armor still apply.
	\item Gaseous creatures do not need to breathe and are immune to attacks involving breathing (troglodyte stench, poison gas, and the like).
	\item Gaseous creatures can't enter water or other liquid. They are not ethereal or incorporeal. They are affected by winds or other forms of moving air to the extent that the wind pushes them in the direction the wind is moving. However, even the strongest wind can't disperse or damage a creature in gaseous form.
	\item Discerning a creature in gaseous form from natural mist requires a DC 15 Spot check. Creatures in gaseous form attempting to hide in an area with mist, smoke, or other gas gain a +20 bonus.
\end{list}
	
\subsection{Gaze Attacks}

While the medusa's gaze is well known, gaze attacks can also charm, curse, or even kill. Gaze attacks not produced by a spell are supernatural.

\listone
	\item Each character within range of a gaze attack must attempt a saving throw (which can be a Fortitude or Will save) each round at the beginning of his turn.
	\item An opponent can avert his eyes from the creature's face, looking at the creature's body, watching its shadow, or tracking the creature in a reflective surface. Each round, the opponent has a 50\% chance of not having to make a saving throw. The creature with the gaze attack gains concealment relative to the opponent. An opponent can shut his eyes, turn his back on the creature, or wear a blindfold. In these cases, the opponent does not need to make a saving throw. The creature with the gaze attack gains total concealment relative to the opponent.
	\item A creature with a gaze attack can actively attempt to use its gaze as an attack action. The creature simply chooses a target within range, and that opponent must attempt a saving throw. If the target has chosen to defend against the gaze as discussed above, the opponent gets a chance to avoid the saving throw (either 50\% chance for averting eyes or 100\% chance for shutting eyes). It is possible for an opponent to save against a creature's gaze twice during the same round, once before its own action and once during the creature's action.
	\item Looking at the creature's image (such as in a mirror or as part of an illusion) does not subject the viewer to a gaze attack.
A creature is immune to its own gaze attack.
	\item If visibility is limited (by dim lighting, a fog, or the like) so that it results in concealment, there is a percentage chance equal to the normal miss chance for that degree of concealment that a character won't need to make a saving throw in a given round. This chance is not cumulative with the chance for averting your eyes, but is rolled separately.
	\item Invisible creatures cannot use gaze attacks.
	\item Characters using darkvision in complete darkness are affected by a gaze attack normally.
	\item Unless specified otherwise, a creature with a gaze attack can control its gaze attack and ``turn it off'' when so desired.
\end{list}

\subsection{Incorporeality}

Spectres, wraiths, and a few other creatures lack physical bodies. Such creatures are insubstantial and can't be touched by nonmagical matter or energy. Likewise, they cannot manipulate objects or exert physical force on objects. However, incorporeal beings have a tangible presence that sometimes seems like a physical attack against a corporeal creature. 

\listone
	\item Incorporeal creatures are present on the same plane as the characters, and characters have some chance to affect them. 
	\item Incorporeal creatures can be harmed only by other incorporeal creatures, by magic weapons, or by spells, spell\textendash like effects, or supernatural effects. They are immune to all nonmagical attack forms. They are not burned by normal fires, affected by natural cold, or harmed by mundane acids.
	\item Even when struck by magic or magic weapons, an incorporeal creature has a 50\% chance to ignore any damage from a corporeal source�except for a force effect or damage dealt by a ghost touch weapon.
	\item Incorporeal creatures are immune to critical hits, extra damage from being favored enemies, and from sneak attacks. They move in any direction (including up or down) at will. They do not need to walk on the ground. They can pass through solid objects at will, although they cannot see when their eyes are within solid matter. 
	\item Incorporeal creatures hiding inside solid objects get a +2 circumstance bonus on Listen checks, because solid objects carry sound well. Pinpointing an opponent from inside a solid object uses the same rules as pinpointing invisible opponents (see Invisibility, below). 
	\item Incorporeal creatures are inaudible unless they decide to make noise.
	\item The physical attacks of incorporeal creatures ignore material armor, even magic armor, unless it is made of force (such as mage armor or bracers of armor) or has the ghost touch ability. 
	\item Incorporeal creatures pass through and operate in water as easily as they do in air.
	\item Incorporeal creatures cannot fall or take falling damage.
	\item Corporeal creatures cannot trip or grapple incorporeal creatures. 
	\item Incorporeal creatures have no weight and do not set off traps that are triggered by weight.
	\item Incorporeal creatures do not leave footprints, have no scent, and make no noise unless they manifest, and even then they only make noise intentionally.
\end{list}

\subsection{Invisibility}

The ability to move about unseen is not foolproof. While they can't be seen, invisible creatures can be heard, smelled, or felt. 

\listone
	\item Invisibility makes a creature undetectable by vision, including darkvision.
	\item Invisibility does not, by itself, make a creature immune to critical hits, but it does make the creature immune to extra damage from being a ranger's favored enemy and from sneak attacks.
	\item A creature can generally notice the presence of an active invisible creature within 30 feet with a DC 20 Spot check. The observer gains a hunch that ``something's there'' but can't see it or target it accurately with an attack. A creature who is holding still is very hard to notice (DC 30). An inanimate object, an unliving creature holding still, or a completely immobile creature is even harder to spot (DC 40). It's practically impossible (+20 DC) to pinpoint an invisible creature's location with a Spot check, and even if a character succeeds on such a check, the invisible creature still benefits from total concealment (50\% miss chance).
	\item A creature can use hearing to find an invisible creature. A character can make a Listen check for this purpose as a free action each round. A Listen check result at least equal to the invisible creature's Move Silently check result reveals its presence. (A creature with no ranks in Move Silently makes a Move Silently check as a Dexterity check to which an armor check penalty applies.) A successful check lets a character hear an invisible creature ``over there somewhere.'' It's practically impossible to pinpoint the location of an invisible creature. A Listen check that beats the DC by 20 pinpoints the invisible creature's location.

\begin{table}
\begin{tabular}[h!]{l|c}
\multicolumn{2}{c}{Listen Check DCs to Detect Invisible Creatures}                        \\
Invisible Creature Is \ldots                   & DC                                       \\ \hline
In combat or speaking                          & 0                                        \\
Moving at half speed                           & Move Silently check result               \\
Moving at full speed                           & Move Silently check result \textendash 4 \\
Running or charging	Move Silently check result & \textendash 20                           \\
Some distance away                             & +1 per 10 feet                           \\
Behind an obstacle                             & (door) +5                                \\
Behind an obstacle (stone wall)                & +15                                      \\
\end{tabular}
\end{table}

	\item A creature can grope about to find an invisible creature. A character can make a touch attack with his hands or a weapon into two adjacent 5\textendash foot squares using a standard action. If an invisible target is in the designated area, there is a 50\% miss chance on the touch attack. If successful, the groping character deals no damage but has successfully pinpointed the invisible creature's current location. (If the invisible creature moves, its location, obviously, is once again unknown.)
	\item If an invisible creature strikes a character, the character struck still knows the location of the creature that struck him (until, of course, the invisible creature moves). The only exception is if the invisible creature has a reach greater than 5 feet. In this case, the struck character knows the general location of the creature but has not pinpointed the exact location.
If a character tries to attack an invisible creature whose location he has pinpointed, he attacks normally, but the invisible creature still benefits from full concealment (and thus a 50\% miss chance). A particularly large and slow creature might get a smaller miss chance.
	\item If a character tries to attack an invisible creature whose location he has not pinpointed, have the player choose the space where the character will direct the attack. If the invisible creature is there, conduct the attack normally. If the enemy's not there, roll the miss chance as if it were there, don't let the player see the result, and tell him that the character has missed. That way the player doesn't know whether the attack missed because the enemy's not there or because you successfully rolled the miss chance.
If an invisible character picks up a visible object, the object remains visible. One could coat an invisible object with flour to at least keep track of its position (until the flour fell off or blew away). An invisible creature can pick up a small visible item and hide it on his person (tucked in a pocket or behind a cloak) and render it effectively invisible. 
Invisible creatures leave tracks. They can be tracked normally. Footprints in sand, mud, or other soft surfaces can give enemies clues to an invisible creature's location.
	\item An invisible creature in the water displaces water, revealing its location. The invisible creature, however, is still hard to see and benefits from concealment.
	\item A creature with the scent ability can detect an invisible creature as it would a visible one.
	\item A creature with the Blind\textendash Fight feat has a better chance to hit an invisible creature. Roll the miss chance twice, and he misses only if both rolls indicate a miss. (Alternatively, make one 25\% miss chance roll rather than two 50\% miss chance rolls.)
	\item A creature with blindsight can attack (and otherwise interact with) creatures regardless of invisibility.
	\item An invisible burning torch still gives off light, as does an invisible object with a light spell (or similar spell) cast upon it.
	\item Ethereal creatures are invisible. Since ethereal creatures are not materially present, Spot checks, Listen checks, Scent, Blind\textendash Fight, and blindsight don't help locate them. Incorporeal creatures are often invisible. Scent, Blind\textendash Fight, and blindsight don't help creatures find or attack invisible, incorporeal creatures, but Spot checks and possibly Listen checks can help.
	\item Invisible creatures cannot use gaze attacks.
	\item Invisibility does not thwart detect spells.
	\item Since some creatures can detect or even see invisible creatures, it is helpful to be able to hide even when invisible.
\end{list}

\subsection{Level Loss}

A character who loses a level instantly loses one Hit Die. The character's base attack bonus, base saving throw bonuses, and special class abilities are now reduced to the new, lower level. Likewise, the character loses any ability score gain, skill ranks, and any feat associated with the level (if applicable). If the exact ability score or skill ranks increased from a level now lost is unknown (or the player has forgotten), lose 1 point from the highest ability score or ranks from the highest\textendash ranked skills. If a familiar or companion creature has abilities tied to a character who has lost a level, the creature's abilities are adjusted to fit the character's new level. The victim's experience point total is immediately set to the midpoint of the previous level.

\subsection{Low\textendash Light Vision}

Characters with low\textendash light vision have eyes that are so sensitive to light that they can see twice as far as normal in dim light. Low\textendash light vision is color vision. A spellcaster with low\textendash light vision can read a scroll as long as even the tiniest candle flame is next to her as a source of light. Characters with low\textendash light vision can see outdoors on a moonlit night as well as they can during the day.

\subsection{Paralysis}

Some monsters and spells have the supernatural or spell\textendash like ability to paralyze their victims, immobilizing them through magical means. (Paralysis from toxins is discussed in the Poison section below.)

\listone
	\item A paralyzed character cannot move, speak, or take any physical action. He is rooted to the spot, frozen and helpless. Not even friends can move his limbs. He may take purely mental actions, such as casting a spell with no components.
	\item A winged creature flying in the air at the time that it becomes paralyzed cannot flap its wings and falls. A swimmer can't swim and may drown.
\end{list}

\subsection{Poison}

When a character takes damage from an attack with a poisoned weapon, touches an item smeared with contact poison, consumes poisoned food or drink, or is otherwise poisoned, he must make a Fortitude saving throw. If he fails, he takes the poison's initial damage (usually ability damage). Even if he succeeds, he typically faces more damage 1 minute later, which he can also avoid with a successful Fortitude saving throw.

One dose of poison smeared on a weapon or some other object affects just a single target. A poisoned weapon or object retains its venom until the weapon scores a hit or the object is touched (unless the poison is wiped off before a target comes in contact with it). Any poison smeared on an object or exposed to the elements in any way remains potent until it is touched or used.
Although supernatural and spell\textendash like poisons are possible, poisonous effects are almost always extraordinary.

Poisons can be divided into four basic types according to the method by which their effect is delivered, as follows.

\ability{Contact:}{Merely touching this type of poison necessitates a saving throw. It can be actively delivered via a weapon or a touch attack. Even if a creature has sufficient damage reduction to avoid taking any damage from the attack, the poison can still affect it. A chest or other object can be smeared with contact poison as part of a trap.}

\ability{Ingested:}{Ingested poisons are virtually impossible to utilize in a combat situation. A poisoner could administer a potion to an unconscious creature or attempt to dupe someone into drinking or eating something poisoned. Assassins and other characters tend to use ingested poisons outside of combat.}

\ability{Inhaled:}{Inhaled poisons are usually contained in fragile vials or eggshells. They can be thrown as a ranged attack with a range increment of 10 feet. When it strikes a hard surface (or is struck hard), the container releases its poison. One dose spreads to fill the volume of a 10\textendash foot cube. Each creature within the area must make a saving throw. (Holding one's breath is ineffective against inhaled poisons; they affect the nasal membranes, tear ducts, and other parts of the body.)}

\ability{Injury:}{This poison must be delivered through a wound. If a creature has sufficient damage reduction to avoid taking any damage from the attack, the poison does not affect it. Traps that cause damage from weapons, needles, and the like sometimes contain injury poisons.}

The characteristics of poisons are summarized on Table: Poisons. Terms on the table are defined below.

\ability{Type:}{The poison's method of delivery (contact, ingested, inhaled, or via an injury) and the Fortitude save DC to avoid the poison's damage.}

\ability{Initial Damage:}{The damage the character takes immediately upon failing his saving throw against this poison. Ability damage is temporary unless marked with an asterisk (*), in which case the loss is a permanent drain. Paralysis lasts for 2d6 minutes.
Secondary Damage: The amount of damage the character takes 1 minute after exposure as a result of the poisoning, if he fails a second saving throw. Unconsciousness lasts for 1d3 hours. Ability damage marked with an asterisk is permanent drain instead of temporary damage.}

\ability{Price:}{The cost of one dose (one vial) of the poison. It is not possible to use or apply poison in any quantity smaller than one dose. The purchase and possession of poison is always illegal, and even in big cities it can be obtained only from specialized, less than reputable sources.}

\subsubsection{Perils of Using Poison}

A character has a 5\% chance of exposing himself to a poison whenever he applies it to a weapon or otherwise readies it for use. Additionally, a character who rolls a natural 1 on an attack roll with a poisoned weapon must make a DC 15 Reflex save or accidentally poison himself with the weapon.

\subsubsection{Poison Immunities}

Creatures with natural poison attacks are immune to their own poison. Nonliving creatures (constructs and undead) and creatures without metabolisms (such as elementals) are always immune to poison. Oozes, plants, and certain kinds of outsiders are also immune to poison, although conceivably special poisons could be concocted specifically to harm them.

\begin{table}
\begin{tabular}[h!]{l|cccc}
\multicolumn{5}{c}{\textbf{Table: Poisons}} \\
Poison                 & Type          & Initial Damage & Secondary                    & Damage Price \\ \hline
Nitharit               & Contact DC 13 & 0              & 3d6 Con                      & 650 gp   \\
Sassone leaf residue   & Contact DC 16 & 2d12 hp        & 1d6 Con                      & 300 gp   \\
Malyss root paste      & Contact DC 16 & 1 Dex          & 2d4 Dex                      & 500 gp   \\
Terinav root           & Contact DC 16 & 1d6 Dex        & 2d6 Dex                      & 750 gp   \\ 
Black lotus extract    & Contact DC 20 & 3d6 Con        & 3d6 Con                      & 4,500 gp \\
Dragon bile            & Contact DC 26 & 3d6 Str        & 0	                           & 1,500 gp \\
Striped toadstool      & Ingested DC 11& 1 Wis          & 2d6 Wis + 1d4 Int            & 180 gp   \\
Arsenic                & Ingested DC 13& 1 Con          & 1d8 Con                      & 120 gp   \\
Id moss                & Ingested DC 14& 1d4 Int        & 2d6 Int                      & 125 gp   \\
Oil of taggit          & Ingested DC 15& 0              & Unconsciousness              & 90 gp    \\
Lich dust              & Ingested DC 17& 2d6 Str        & 1d6 Str                      & 250 gp   \\
Dark reaver powder     & Ingested DC 18& 2d6 Con        & 1d6 Con + 1d6 Str            & 300 gp   \\
Ungol dust             & Inhaled  DC 15& 1 Cha          & 1d6 Cha + 1 Cha*             & 1,000 gp \\
Insanity mist          & Inhaled DC 15 & 1d4 Wis        & 2d6 Wis                      & 1,500 gp \\
Burnt othur fumes      & Inhaled DC 18 & 1 Con*         & 3d6 Con                      & 2,100 gp \\
Black adder venom      & Injury DC 11  & 1d6 Con        & 1d6 Con                      & 120 gp   \\
Small centipede poison & Injury DC 11  & 1d2 Dex        & 1d2 Dex                      & 90 gp    \\
Bloodroot              & Injury DC 12  & 0              & 1d4 Con + 1d3 Wis            & 100 gp   \\
Drow poison            & Injury DC 13  & Unconsciousness&	Unconsciousness for 2d4 hours& 75gp     \\
Greenblood oil         & Injury DC 13  & 1 Con          & 1d2 Con                      & 100 gp   \\
Blue whinnis           & Injury DC 14  & 1 Con          & Unconsciousness              & 120 gp   \\
Medium spider venom    & Injury DC 14  & 1d4 Str        & 1d4 Str                      & 150 gp   \\
Shadow essence         & Injury DC 17  & 1 Str*         & 2d6 Str                      & 250 gp   \\
Wyvern poison          & Injury DC 17	 & 2d6 Con        & 2d6 Con                      & 3,000 gp \\
Large scorpion venom   & Injury DC 18	 & 1d6 Str        & 1d6 Str                      & 200 gp   \\
Giant wasp poison      & Injury DC 18	 & 1d6 Dex        & 1d6 Dex                      & 210 gp   \\
Deathblade             & Injury DC 20  & 1d6 Con        & 2d6 Con                      & 1,800 gp \\
Purple worm poison     & Injury DC 24	 & 1d6 Str        & 2d6 Str                      & 700 gp   \\ \hline
\multicolumn{5}{p{7in}}{*Permanent drain, not temporary damage.} \\	 
\end{tabular}
\end{table}

\subsection{Polymorph}

Magic can cause creatures and characters to change their shapes�sometimes against their will, but usually to gain an advantage. Polymorphed creatures retain their own minds but have new physical forms.
The polymorph spell defines the general polymorph effect.

Since creatures do not change types, a slaying or bane weapon designed to kill or harm creatures of a specific type affects those creatures even if they are polymorphed. Likewise, a creature polymorphed into the form of a creature of a different type is not subject to slaying and bane effects directed at that type of creature. 

A ranger's favored enemy bonus is based on knowing what the foe is, so if a creature that is a ranger's favored enemy polymorphs into another form, the ranger is denied his bonus.

A dwarf 's bonus for fighting giants is based on shape and size, so he does not gain a bonus against a giant polymorphed into something else, but does gain the bonus against any creature polymorphed into a giant.

\subsection{Psionics}

Telepathy, mental combat and psychic powers�psionics is a catchall word that describes special mental abilities possessed by various creatures. These are spell\textendash like abilities that a creature generates from the power of its mind alone�no other outside magical force or ritual is needed. Each psionic creature's description contains details on its psionic abilities.

Psionic attacks almost always allow Will saving throws to resist them. However, not all psionic attacks are mental attacks. Some psionic abilities allow the psionic creature to reshape its own body, heal its wounds, or teleport great distances. Some psionic creatures can see into the future, the past, and the present (in far\textendash off locales) as well as read the minds of others. 

\subsection{Rays}

All ray attacks require the attacker to make a successful ranged touch attack against the target. Rays have varying ranges, which are simple maximums. A ray's attack roll never takes a range penalty. Even if a ray hits, it usually allows the target to make a saving throw (Fortitude or Will). Rays never allow a Reflex saving throw, but if a character's Dexterity bonus to AC is high, it might be hard to hit her with the ray in the first place.

\subsection{Regeneration}

Creatures with this extraordinary ability recover from wounds quickly and can even regrow or reattach severed body parts. Damage dealt to the creature is treated as nonlethal damage, and the creature automatically cures itself of nonlethal damage at a fixed rate.
Certain attack forms, typically fire and acid, deal damage to the creature normally; that sort of damage doesn't convert to nonlethal damage and so doesn't go away. The creature's description includes the details.
Creatures with regeneration can regrow lost portions of their bodies and can reattach severed limbs or body parts. Severed parts die if they are not reattached.

\listone
	\item Regeneration does not restore hit points lost from starvation, thirst, or suffocation.
	\item Attack forms that don't deal hit point damage ignore regeneration.
	\item An attack that can cause instant death only threatens the creature with death if it is delivered by weapons that deal it lethal damage.
\end{list}

\subsection{Resistance to Energy}

A creature with resistance to energy has the ability (usually extraordinary) to ignore some damage of a certain type each round, but it does not have total immunity. Each resistance ability is defined by what energy type it resists and how many points of damage are resisted. It doesn't matter whether the damage has a mundane or magical source. When resistance completely negates the damage from an energy attack, the attack does not disrupt a spell. This resistance does not stack with the resistance that a spell might provide.

\subsection{Scent}

This extraordinary ability lets a creature detect approaching enemies, sniff out hidden foes, and track by sense of smell.

\listone
	\item A creature with the scent ability can detect opponents by sense of smell, generally within 30 feet. If the opponent is upwind, the range is 60 feet. If it is downwind, the range is 15 feet. Strong scents, such as smoke or rotting garbage, can be detected at twice the ranges noted above. Overpowering scents, such as skunk musk or troglodyte stench, can be detected at three times these ranges.
	\item The creature detects another creature's presence but not its specific location. Noting the direction of the scent is a move action. If it moves within 5 feet of the scent's source, the creature can pinpoint that source.
	\item A creature with the Track feat and the scent ability can follow tracks by smell, making a Wisdom check to find or follow a track. The typical DC for a fresh trail is 10. The DC increases or decreases depending on how strong the quarry's odor is, the number of creatures, and the age of the trail. For each hour that the trail is cold, the DC increases by 2. The ability otherwise follows the rules for the Track feat. Creatures tracking by scent ignore the effects of surface conditions and poor visibility.
	\item Creatures with the scent ability can identify familiar odors just as humans do familiar sights.
Water, particularly running water, ruins a trail for air\textendash breathing creatures. Water\textendash breathing creatures that have the scent ability, however, can use it in the water easily.
	\item False, powerful odors can easily mask other scents. The presence of such an odor completely spoils the ability to properly detect or identify creatures, and the base Survival DC to track becomes 20 rather than 10.
\end{list}

\subsection{Spell Resistance}

Spell resistance is the extraordinary ability to avoid being affected by spells. (Some spells also grant spell resistance.)

To affect a creature that has spell resistance, a spellcaster must make a caster level check (1d20 + caster level) at least equal to the creature's spell resistance. (The defender's spell resistance is like an Armor Class against magical attacks.) If the caster fails the check, the spell doesn't affect the creature. The possessor does not have to do anything special to use spell resistance. The creature need not even be aware of the threat for its spell resistance to operate.

Only spells and spell\textendash like abilities are subject to spell resistance. Extraordinary and supernatural abilities (including enhancement bonuses on magic weapons) are not. A creature can have some abilities that are subject to spell resistance and some that are not. Even some spells ignore spell resistance; see When Spell Resistance Applies, below. 
A creature can voluntarily lower its spell resistance. Doing so is a standard action that does not provoke an attack of opportunity. Once a creature lowers its resistance, it remains down until the creature's next turn. At the beginning of the creature's next turn, the creature's spell resistance automatically returns unless the creature intentionally keeps it down (also a standard action that does not provoke an attack of opportunity).

\listone
	\item A creature's spell resistance never interferes with its own spells, items, or abilities.
	\item A creature with spell resistance cannot impart this power to others by touching them or standing in their midst. Only the rarest of creatures and a few magic items have the ability to bestow spell resistance upon another.
	\item Spell resistance does not stack. It overlaps. 
\end{list}

\subsubsection{When Spell Resistance Applies}

Each spell includes an entry that indicates whether spell resistance applies to the spell. In general, whether spell resistance applies depends on what the spell does:

\ability{Targeted Spells:}{Spell resistance applies if the spell is targeted at the creature. Some individually targeted spells can be directed at several creatures simultaneously. In such cases, a creature's spell resistance applies only to the portion of the spell actually targeted at that creature. If several different resistant creatures are subjected to such a spell, each checks its spell resistance separately.}

\ability{Area Spells:}{Spell resistance applies if the resistant creature is within the spell's area. It protects the resistant creature without affecting the spell itself.}

\ability{Effect Spells:}{Most effect spells summon or create something and are not subject to spell resistance. Sometimes, however, spell resistance applies to effect spells, usually to those that act upon a creature more or less directly, such as web.
Spell resistance can protect a creature from a spell that's already been cast. Check spell resistance when the creature is first affected by the spell.}

Check spell resistance only once for any particular casting of a spell or use of a spell\textendash like ability. If spell resistance fails the first time, it fails each time the creature encounters that same casting of the spell. Likewise, if the spell resistance succeeds the first time, it always succeeds. If the creature has voluntarily lowered its spell resistance and is then subjected to a spell, the creature still has a single chance to resist that spell later, when its spell resistance is up.

Spell resistance has no effect unless the energy created or released by the spell actually goes to work on the resistant creature's mind or body. If the spell acts on anything else and the creature is affected as a consequence, no roll is required. Creatures can be harmed by a spell without being directly affected. 

Spell resistance does not apply if an effect fools the creature's senses or reveals something about the creature.

Magic actually has to be working for spell resistance to apply. Spells that have instantaneous durations but lasting results aren't subject to spell resistance unless the resistant creature is exposed to the spell the instant it is cast. 

When in doubt about whether a spell's effect is direct or indirect, consider the spell's school:

\ability{Abjuration:}{The target creature must be harmed, changed, or restricted in some manner for spell resistance to apply. Perception changes aren't subject to spell resistance.

Abjurations that block or negate attacks are not subject to an attacker's spell resistance�it is the protected creature that is affected by the spell (becoming immune or resistant to the attack).}

\ability{Conjuration:}{These spells are usually not subject to spell resistance unless the spell conjures some form of energy. Spells that summon creatures or produce effects that function like creatures are not subject to spell resistance.}

\ability{Divination:}{These spells do not affect creatures directly and are not subject to spell resistance, even though what they reveal about a creature might be very damaging.}

\ability{Enchantment:}{Since enchantment spells affect creatures' minds, they are typically subject to spell resistance.}

\ability{Evocation:}{If an evocation spell deals damage to the creature, it has a direct effect. If the spell damages something else, it has an indirect effect.}

\ability{Illusion:}{These spells are almost never subject to spell resistance. Illusions that entail a direct attack are exceptions.}

\ability{Necromancy:}{Most of these spells alter the target creature's life force and are subject to spell resistance. Unusual necromancy spells that don't affect other creatures directly are not subject to spell resistance.}

\ability{Transmutation:}{These spells are subject to spell resistance if they transform the target creature. Transmutation spells are not subject to spell resistance if they are targeted on a point in space instead of on a creature. Some transmutations make objects harmful (or more harmful), such as magic stone. Even these spells are not generally subject to spell resistance because they affect the objects, not the creatures against which the objects are used. Spell resistance works against magic stone only if the creature with spell resistance is holding the stones when the cleric casts magic stone on them.}

\subsubsection{Successful Spell Resistance}

Spell resistance prevents a spell or a spell\textendash like ability from affecting or harming the resistant creature, but it never removes a magical effect from another creature or negates a spell's effect on another creature. Spell resistance prevents a spell from disrupting another spell.

Against an ongoing spell that has already been cast, a failed check against spell resistance allows the resistant creature to ignore any effect the spell might have. The magic continues to affect others normally.

\subsection{Tremorsense}

A creature with tremorsense automatically senses the location of anything that is in contact with the ground and within range.

\listone
	\item If no straight path exists through the ground from the creature to those that it's sensing, then the range defines the maximum distance of the shortest indirect path. It must itself be in contact with the ground, and the creatures must be moving. 
	\item As long as the other creatures are taking physical actions, including casting spells with somatic components, they're considered moving; they don't have to move from place to place for a creature with tremorsense to detect them.
\end{list}

\subsection{Turn Resistance}

Some creatures (usually undead) are less easily affected by the turning ability of clerics or paladins.

\listone
	\item Turn resistance is an extraordinary ability.
	\item When resolving a turn, rebuke, command, or bolster attempt, added the appropriate bonus to the creature's Hit Dice total. 
\end{list}