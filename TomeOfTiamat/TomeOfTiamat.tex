\documentclass[10pt]{report}

\usepackage{appendix}
\usepackage{floatflt}
\usepackage{fancyhdr}
\usepackage{textcomp}
\usepackage[usenames]{color}
\usepackage{isoent}    %\sfrac
\usepackage{newcent}   %font
\usepackage{sectsty} %custom section headings

\newcommand{\normalsections}{
\sectionfont{\noindent\rule{\textwidth}{0.015in}\\\nohang}
\subsectionfont{\noindent\rule{\textwidth}{0.005in}\\\nohang}
}

\newcommand{\columnsections}{
\sectionfont{\vspace*{-20pt}\noindent\rule{3.5in}{0.015in}\\\nohang}
\subsectionfont{\vspace*{-20pt}\noindent\rule{3.5in}{0.005in}\\\nohang}
}


\sectionfont{\noindent\rule{\textwidth}{0.015in}\\\nohang}
\subsectionfont{\noindent\rule{3.5in}{0.005in}\\\nohang}

\usepackage[Bjarne]{fncychap} %Canned chapter headings


\usepackage{multicol}
\usepackage[bookmarks=true,colorlinks,linkcolor=cyan,breaklinks]{hyperref}



%%%%Margins%%%
\topmargin 0pt
\advance \topmargin by -\headheight
\advance \topmargin by -\headsep
\textheight 8.9in
\oddsidemargin -0.25in
\evensidemargin \oddsidemargin
\textwidth 7in
\oddsidemargin -0.25in
%\setlength {\parindent} {0pt}



%%%Formatting%%%
\newcommand{\ability}[2]{\smallskip \noindent \textbf{#1} #2}
\newcommand{\shortability}[2]{\noindent\textbf{#1} #2\\}
\newcommand{\bolded}[1]{\noindent\textbf{#1}}
\newcommand{\itemability}[2]{\item \textbf{#1} #2}
\newcommand{\featname}[1]{\vspace*{0.1cm plus 0.2cm minus 0.05cm}\noindent\textbf{#1}\\}
\newcommand{\featnamelist}[1]{\vspace*{0.1cm plus 0.2cm minus 0.05cm}\noindent\textbf{#1}}

\newcommand{\descfeat}[2]{\featname{#1}\emph{#2}\\}

\newcommand{\classname}[1]{\section{#1}}
\newcommand{\condition}[1]{\emph{#1}}
\newcommand{\quot}[1]{\emph{#1}\medskip}
\newcommand{\desc}[1]{#1 \medskip}
\newcommand{\example}[1]{\emph{#1}}
\newcommand{\magicitem}[1]{\emph{#1}}
\newcommand{\monster}[1]{\subsection{#1} \label{monster:#1}}
\newcommand{\monsterline}[2]{\textbf{#1:} #2\\}
\newcommand{\monstersizetype}[2]{\textbf{#1 #2}\\}
\newcommand{\spell}[1]{\emph{#1}}
\newcommand{\spelllist}[1]{\smallskip \noindent \underline{\textbf{#1}}}

% For the feats -- moved here from feats.tex because we added another 
% section(s) for feats. -Surgo
\newcommand{\minitabular}[1]{\begin{tabular}{p{0.25in}p{2.9in}} #1\\ \end{tabular}}
\newcommand{\babfeat}[7]{
	\noindent\minitabular{\multicolumn{2}{l}{\parbox{3in}{\textbf{#1}}}}
	\minitabular{\multicolumn{2}{l}{\parbox{3in}{\small #2}}}
	\minitabular{\raggedleft\textbf{\small \textbf{+0:}}& {\small #3}}
	\minitabular{\raggedleft\textbf{\small +1:} & {\small #4}}
	\minitabular{\raggedleft\textbf{\small +6:} & {\small #5}}
	\minitabular{\raggedleft\textbf{\small +11:} & {\small #6}}
	\minitabular{\raggedleft\textbf{\small +16:} &{\small #7}}
}
\newcommand{\skillfeat}[8]{
	\noindent\minitabular{\multicolumn{2}{l}{\parbox{3in}{\textbf{#1}}}}
	\minitabular{\multicolumn{2}{l}{\parbox{3in}{\small #2}}}
	\minitabular{\multicolumn{2}{l}{\parbox{3in}{\small\textbf{#3}}}}
	\minitabular{\raggedleft\textbf{\small \textbf{0:}} &{\small #4}}
	\minitabular{\raggedleft\textbf{\small 4:} &{\small #5}}
	\minitabular{\raggedleft\textbf{\small 9:} &{\small #6}}
	\minitabular{\raggedleft\textbf{\small 14:} &{\small #7}}
	\minitabular{\raggedleft\textbf{\small 19:} &{\small #8}}
}

\newcommand{\boxit}[1]{\frame{\parbox{\textwidth}{#1}}}

% A new box command -- the first argument is the box's header, the second the 
% contents of the box. This looks a lot better than \boxit. -Surgo
\newcommand{\abox}[2]{\vspace{5pt}
	\fbox{\begin{minipage}{0.8\linewidth}
	\setlength{\parindent}{0.15in}
	\textbf{#1}

	#2
\end{minipage}}
\vspace{5pt}}

\newcommand{\itemspace}{\setlength{\itemsep}{-1mm}\setlength{\topsep}{-1mm} }

\newcommand{\listone}{\begin{list}{$\bullet$}{\itemspace}}
\newcommand{\listprereq}{\begin{list}{\vspace*{-2pt}}{\itemspace}}
\newcommand{\listtwo}{\begin{list}{$\triangleright$}{\itemspace}}
\newcommand{\listthree}{\begin{list}{--}{\itemspace}}

% Bold the first argument in the listitem
\newcommand{\bolditem}[2]{\item \textbf{#1} #2}

\newcommand{\slashfrac}[2]{${}^{#1}$\hspace*{-1pt}\makebox[2pt]{$\diagup$}${}_{#2}$}
%\newcommand{\half}[0]{\slashfrac{1}{2}}
\newcommand{\half}[0]{\ensuremath{\sfrac{1}{2}} }
\newcommand{\third}[0]{\ensuremath{\sfrac{1}{3}} }
\newcommand{\fourth}[0]{\ensuremath{\sfrac{1}{4}} }
%\newcommand{\half}[0]{\ensuremath{\scriptscriptstyle 1/2}}
%\newcommand{\threefourths}[0]{\ensuremath{\sfrac{3}{4}}}


%\rule{\textwidth}{0.005in}

%\renewcommand{\sectionmark}[1]{\markright{\thesection\ \boldmath\emph{#1}\unboldmath}\\\rule{\textwidth}{0.005in}\\}

\setcounter{tocdepth}{1}

\begin{document}

\pagestyle{plain}
%Cover Page

\begin{center} \Huge

\textsc{Races of War}\end{center}



\vspace{2cm}
\begin{center}\large By Frank Trollman \& K\end{center}


\newpage

\vspace*{4in}

\noindent Please address all complaints and comments about balance to the authors at\\
{\color{blue} \href{http://tgdmb.com/viewforum.php?f=1}{http://tgdmb.com/viewforum.php?f=1}}

\vspace{0.2in}

%\noindent For a hypertext version of some of this information, especially classes, please look at\\
%{\color{blue} \href{http://www.d20ragon.com/frank/}{http://www.d20ragon.com/frank/}}

%\vspace{0.2in}

\noindent Amateur Typesetting by Joshua Middendorf, updated by Morgon ``Surgo'' Kanter, ``Aktariel'', and Stephen ``Quantumboost'' Smith.

\vspace{0.15in}

\noindent Please address all comments regarding the quality (or lack thereof) of the typesetting (that is, formatting of the pdf) to Joshua Middendorf (\href{mailto:middendorfproject@gmail.com}{middendorfproject@gmail.com}), Morgon Kanter (\href{mailto:morgon.kanter@gmail.com}{morgon.kanter@gmail.com}), Aktariel (\href{mailto:aktariel@gmail.com}{aktariel@gmail.com}), or simply comment in the above forum.


%\vspace{0.2in}

%\emph{Enjoy!}


\vspace{1in}
\noindent Published on \today, version 0.6\\
\noindent You may find the most recent version of this document at:\\
{\color{blue} \href{http://www.tgdmb.com/viewtopic.php?t=36046}{http://www.tgdmb.com/viewtopic.php?t=36046}}

\newpage


\pagestyle{fancy}
%Fix the spacing so it's all reasonable
\linespread{.9}  \small  \normalsize \itemspace \normalsections

\tableofcontents

\chapter{Power Sources}

\chapter{The Physics of Magic}

%\input{stuff}

\chapter{Classes}

\classname{Warlock} \label{comm:class:spherelock}
\vspace*{-8pt}
\quot{``I have all the powers of Hell at my disposal. Who are you to question me?''}

\desc{Some people want power, and are crazy enough to offer their soul to various not nice people to get it. Others are simply (un)lucky enough to be descendants of those same people. However they managed to get their abilities, Warlocks wield the powers of the lower planes, which include powerful magic spells and the ability to shoot hellfire out of their hands.}

\ability{Alignment:}{Warlocks who bargain away their souls for power tend to be Evil, though nothing requires them to be. In fact, demons and devils will jump on the chance to corrupt someone Good to Team Evil, but such instances are rare. People whose power comes from their blood can be any alignment.}

\ability{Races:}{Warlocks can be any non-Outsider. Actual Outsiders don't bargain away their souls because they already have the ability to use the powers they would get, and that is represented by them having access to classes like True Fiend or Conduit of the Lower Planes. However, a Warlock who later becomes an Outsider can still be a Warlock.}

\ability{Starting Gold:}{4d4*10 gp (100 gold)}

\ability{Starting Age:}{As Rogue.}

\ability{Hit Die:}{d6}

\ability{Class Skills:}{The Warlock's class skills (and the key ability for each skill) are Bluff (Cha), Concentration (Con), Disguise (Cha), Hide (Dex), Intimidate (Cha), Knowledge (all, taken individually) (Int), Listen (Wis), Move Silently (Dex), Sense Motive (Wis), Sleight of Hand (Dex), Spellcraft (Int), Spot (Wis), Use Magic Device (Cha).}

\ability{Skills/Level:}{4 + Intelligence Bonus}


\begin{table}[tbh]
\begin{small}
\begin{tabular}{lp{3cm}p{0.7cm}p{0.7cm}p{0.7cm}l}
Level &Base Attack Bonus &Fort Save &Ref Save &Will Save &Special\\
1st &+0 &+0 &+2 &+2 &Sphere, Eldritch Blast\\
2nd &+1 &+0 &+3 &+3 &Fiendish Similarities\\
3rd &+2 &+1 &+3 &+3 &Sphere\\
4th &+3 &+1 &+4 &+4 &Call Fiends\\
5th &+3 &+1 &+4 &+4 &Hellfire Blast\\
6th &+4 &+2 &+5 &+5 &Sphere\\
7th &+5 &+2 &+5 &+5 &Damage Reduction\\
8th &+6/+1 &+2 &+6 &+6 &Bonus Feat\\
9th &+6/+1 &+3 &+6 &+6 &Sphere\\
10th &+7/+2 &+3 &+7 &+7 &Dark Blast\\
11th &+8/+3 &+3 &+7 &+7 &Fiendish Servant\\
12th &+9/+4 &+4 &+8 &+8 &Sphere\\
13th &+9/+4 &+4 &+4 &+8 &Energy Resistance\\
14th &+10/+5 &+4 &+9 &+9 &Fear Aura\\
15th &+11/+6/+6 &+5 &+9 &+9 &Sphere, Fiendish Apotheosis
\end{tabular}
\end{small}
\end{table}

\smallskip\noindent All of the following are Class Features of the Warlock class.

\ability{Weapon and Armor Proficiency:}{Warlocks are proficient in light armor, shields (but not great shields), and simple weapons.}

\ability{Sphere:}{A Warlock gains basic access to a sphere at every third level. If the Warlock selects a sphere that he already has basic access to, he upgrades it to advanced access. If he already had advanced access, he gains expert access.}

\ability{Eldritch Blast (Su):}{As an attack action, a Warlock may fire a blast of fire at his foes. This has a range of Close (25 feet +5 ft./2 levels), does 1d6 damage per level of Warlock, and requires a ranged touch attack to hit.}

\ability{Fiendish Similarities}{A 2nd level Warlock may qualify for and take feats with the [Fiend] or [Necromantic] tag, as long as he meets the other requirements. For [Necromantic] feats, he may use his character level as his caster level.}

\ability{Call Fiends (Sp):}{At 4th level, a Warlock can Summon an Outsider with the [Evil] subtype once per day, as long as the Outsider's Challenge Rating is 3 less than his character level or lower. He can also choose to double the number of creatures summoned by reducing the max CR of the creatures by 2 per doubling (a 9th level Warlock could summon one CR 6 fiend, two CR 4 fiends, 4 CR 2 fiends, or 8 CR 1 fiends). This is treated as a spell of one half the Warlock's level, rounded down, with a caster level equal to his levels in Warlock.}

\ability{Hellfire Blast (Su):}{At 5th level, a Warlock may choose to fire a blast of hellfire instead of normal fire, at the cost of 2d6 points of damage (i.e. a 5th level Warlock could use a 5d6 fire blast or a 3d6 hellfire blast). This blast bypasses fire resistance and deals half damage to creatures with fire immunity.}

\ability{Damage Reduction (Ex):}{At 7th level, a Warlock gains damage reduction equal to one half his class level, rounded up. This damage reduction is bypasses by whatever material is baneful to the Warlock's fiendish patron or ancestor (silver for Baatezu, wood for Yugoloths, stone for Demodands, and iron for Tanar'ri) or Good aligned weapons. At 13th level, it is bypassed only by one of those, which the Warlock chooses upon taking the level. If a Warlock later takes levels in True Fiend, the damage reduction stacks and, at the third level of True Fiend, is bypassed only by weapons that are both Good and made of a baneful substance.}

\ability{Bonus Feat:}{At 8th level, a Warlock gains one bonus feat, which can be any [Fiend] feat he qualifies for.}

\ability{Dark Blast (Su):}{At 10th level, a Warlock may choose to change the damage of his Eldritch Blast to Unholy damage by reducing the damage it deals by 4d6 (a 10th level Warlock could opt to fire a 10d6 fire blast, 8d6 hellfire blast, or 6d6 unholy blast).}

\ability{Fiendish Servant:}{An 11th level Warlock is followed by a cohort whose CR is 2 less than his level, but the cohort can only have levels in True Fiend, Fiendish Brute, or Conduit of the Lower Planes (they can also have racial hit dice).}

\ability{Energy Resistance:}{At 13th level, a Warlock gains resistance 10 to two energy types of his choice, and resistance 5 to a third.}

\ability{Fear Aura (Su):}{At 14th level, a Warlock can radiate a 20 foot aura of \spell{fear} (as per the spell) at will with a caster level equal to his character level. Save DC is 10+1/2 HD+Cha modifier.}

\ability{Fiendish Apotheosis:}{A 15th level Warlock becomes an Outsider with the [Evil] subtype, even if his own alignment is not evil. He is immortal and does not age. This allows him to qualify for the Fiend classes, which can be used to fill out his last 5 levels.}
\classname{Warmage}
\vspace{-8pt}
\quot{TACTICAL GENIUS!}

\desc{Wizards in general are pretty good on the battlefield. When it comes to killing lots of people at a time, casters excel at that if they feel like it. Not that a single great warrior can't hack through an army by the time everyone else has done their shoes up, but it's always the explosions of flame that people remember.}

\desc{But some casters are specifically designed for the battlefield. Not only do they call ordinance down, annihilating large numbers of people at a time, but they can also take care of the stuff that only the good commanders think of: food, drink, digging trenches, building fortifications, gathering intel, making the battlefield conditions perfect. When you need someone like that, you turn to the Warmage.}

\ability{Hit Die:}{d6}

\ability{Class Skills:}{}

\ability{Skills/Level:}{4 + Int modifier.}

\begin{table}[htb]
\begin{small}
\begin{tabular}{lp{3cm}p{0.7cm}p{0.7cm}p{0.7cm}l}
Level  &Base Attack Bonus &Fort Save &Ref Save &Will Save &Special\\
1st &+0 &+0 &+2 &+2 &Elemental Exchange (Fire and Cold), Spellcasting, Armored Casting\\
2nd &+1 &+0 &+3 &+3 &Create Water, Create Food\\
3rd &+1 &+1 &+3 &+3 &Purify Food and Drink\\
4th &+2 &+1 &+4 &+4 &Energy Effect\\
5th &+2 &+1 &+4 &+4 &Swift Cast 1/day, Elemental Exchange (Acid and Electricity)\\
6th &+3 &+2 &+5 &+5 &Limitless Spell Force\\
7th &+3 &+2 &+5 &+5 &Explosive Spell\\
8th &+4 &+2 &+6 &+6 &Energy Surge\\
9th &+4 &+3 &+6 &+6 &Weaken Defences, Elemental Exchange (Sonic)\\
10th &+5 &+3 &+7 &+7 &Swift Cast 2/day, Chain Spell\\
11th &+5 &+3 &+7 &+7 &Rallying Spell\\
12th &+6/+1 &+4 &+8 &+8 &Energy Blitz\\
13th &+6/+1 &+4 &+8 &+8 &Heroes Feast, Elemental Exchange (Force and Negative)\\
14th &+7/+2 &+4 &+9 &+9 &Instant Fortress\\
15th &+7/+2 &+5 &+9 &+9 &Swift Cast 3/day, Arcane Ordinance\\
16th &+8/+3 &+5 &+10 &+10 &Energy Mastery\\
17th &+8/+3 &+5 &+10 &+10 &Dimensional Fortress\\
18th &+9/+4 &+6 &+11 &+11 &Hellstorm\\
19th &+9/+4 &+6 &+11 &+11 &Dragonflight Bombardment\\
20th &+10/+5 &+6 &+12 &+12 &Swift Cast 4/day, Arcane Siege, Win\\
\end{tabular}
\end{small}
\end{table}

\begin{table}[htb]
\begin{small}
\noindent\begin{tabular}{lllllllllll}
 & \multicolumn{10}{c}{Warmage Spells Per Day}\\
 &0 &1 &2 &3 &4 &5 &6 &7 &8 &9\\
1 &4 &2 &- &- &- &- &- &- &- &-\\
2 &5 &3 &- &- &- &- &- &- &- &-\\
3 &6 &4 &2 &- &- &- &- &- &- &-\\
4 &6 &5 &3 &- &- &- &- &- &- &-\\
5 &6 &6 &4 &2 &- &- &- &- &- &-\\
6 &6 &6 &5 &3 &- &- &- &- &- &-\\
7 &6 &6 &6 &4 &2 &- &- &- &- &-\\
8 &6 &6 &6 &5 &3 &- &- &- &- &-\\
9 &6 &6 &6 &6 &4 &2 &- &- &- &-\\
10 &6 &6 &6 &6 &5 &3 &- &- &- &-\\
11 &6 &6 &6 &6 &6 &4 &2 &- &- &-\\
12 &6 &6 &6 &6 &6 &5 &3 &- &- &-\\
13 &6 &6 &6 &6 &6 &6 &4 &2 &- &-\\
14 &6 &6 &6 &6 &6 &6 &5 &3 &- &-\\
15 &6 &6 &6 &6 &6 &6 &6 &4 &2 &-\\
16 &6 &6 &6 &6 &6 &6 &6 &5 &3 &-\\
17 &6 &6 &6 &6 &6 &6 &6 &6 &4 &2\\
18 &6 &6 &6 &6 &6 &6 &6 &6 &5 &3\\
19 &6 &6 &6 &6 &6 &6 &6 &6 &6 &4\\
20 &6 &6 &6 &6 &6 &6 &6 &6 &6 &5\\
\end{tabular}
\end{small}
\end{table}

\smallskip\noindent All of the following are Class Features of the Warmage class.

\ability{Weapon and Armor Proficiency:}{Warmages are proficient with all simple and martial weapons. Warmages are proficient with light armor and medium armor but not with shields of any kind.}

\ability{Spellcasting:}{The Warmage automatically knows every spell on their class list that they are a high enough level to cast. They cast spells spontaneously, without preparation. The ability score that determines their spellcasting is Intelligence. Unless stated otherwise, any Warmage class features that affect her spells will only affect those gained from Warmage levels, not other classes.}

\ability{Armored Casting:}{A Warmage casts arcane spells, but she is not affected by the arcane spell failure of any armor or shield she is proficient with. This ability only applies to her Warmage spells, if she is able to cast any other arcane spells, they are affected by arcane spell failure normally.}

\ability{Elemental Exchange:}{If a spell deals a specific type of energy damage (Acid, Cold, Electricity, Fire, Sonic, Force, Negative, Dessication), the Warmage can elect to swap it to Fire or Cold damage - later, more energy types become available. This is not an action and does not affect the spell in any other way, unless the spell has an energy descriptor, in which case it changes to match the new one (so you won't have a Cold Fireball being a [Fire] spell). This applies to any spells the Warmage can cast, not just her Warmage ones.}

\ability{Create Water (Sp):}{The Warmage can Create Water as a Spell-like ability at will. The Caster Level is equal to her class level.}

\ability{Create Food (Sp):}{The Warmage can Create Food as a Spell-like ability at will. This creates one pound of nutritious, if bland, non-descript food per class level and uses a Standard Action. Consider it the equivalent of a military MRE designed to keep for long times. Yeah, I can see you salivating already.}

\ability{Purify Food and Drink (Sp):}{The Warmage can purify food and drink as a Spell-like ability at will.}

\ability{Energy Effect:}{When casting spells that deal energy damage, a special effect occurs, based on the energy type:

\noindent\begin{tabular}{|l|l|}
\hline
Energy: &Effect:\\
\hline
Acid &the target becomes Sickened for 1d4 rounds\\
\hline
Cold &the target becomes numb, dropping whatever they are holding\\
\hline
Electricity &the target is momentarily shocked, becoming Staggered for 1 round\\
\hline
Fire &the target catches fire\\
\hline
Sonic &the target is Deafened for 1 minute\\
\hline
\end{tabular}

The effects only occur if the target is damaged. If a spell causes multiple damage types, only one effect can be chosen per target. Targets are entitled to a Fortitude save to negate this effect (DC 10 + half the Warmage's HD + her Int modifier). If the original spell allows a saving throw, then passing that save negates the effect instead of making someone roll two saves.}

\ability{Swift Cast:}{Once per day per five class levels, the Warmage can cast a spell as a Swift Action. This only applies to spells that can be cast as a full round or less, however it does not change the spell level or any other aspect. It DOES change the casting time, in the exact manner that I just stated.}

\ability{Limitless Spell Force:}{If a spell has effects that increase with caster level (one ray per 4 levels beyond 3rd, 1d6 per level, 2 targets per level etc.) and usually has a limit on this increase, ignore the limit: the effects increase with the Warmage's caster level indefinitely.}

\ability{Explosive Spell:}{When casting a damaging spell with an area of effect, the Warmage may elect to make it an Explosive spell. All targets who take damage must succeed on a Fortitude save or be sent flying. The DC is 10 + half the Warmage's HD + her Int modifier. They are hurled in the direction the spell struck them from, moving to the outer edge, and half the distance travelled again (so if they would need to move 10' to leave the area of effect they would be moved 10+5 = 15'. If they had to move 100' they would end up being knocked 150'), then land prone. If a solid barrier prevents their movement, they take 1d6 Bludgeoning damage for every 10' of movement denied. Yes, the Warmage may pinball people with a bouncing Lightning Bolt.}

\ability{Energy Surge:}{Even greater effects apply to Warmage spells that deal energy damage.

\begin{tabular}[h]{|l|l|}
\hline
Energy: &Effect:\\
\hline
Acid &The target takes 1d4 Str damage as their muscle tissue is eaten away\\
\hline
Cold &The target becomes Slowed for 2d4 rounds as their joints freeze\\
\hline
Electricity &The target is Stunned for 1 round\\
\hline
Fire &The target becomes Exhausted for 2d4 rounds\\
\hline
Sonic &The target is knocked prone and Dazed for 1 round\\
\hline
\end{tabular}

The Warmage may elect to apply Energy Effect or Energy Surge, but not both at once. As with Energy Effect, there is a saving throw.}

\ability{Weaken Defences:}{Any foe who suffers damage from a spell cast by a Warmage takes a penalty on their Armour Class, Damage Reduction, Spell Resistance and Energy Resistances until the beginning of the Warmage's next turn. This penalty is equal to the spell level.}

\ability{Chain Spell:}{When casting a single-target spell (whether a single target is designated or it is a ray, orb etc. - even if multiple targets can be selected but only one is chosen, but not for area-of-effect spells) with a casting time of one round or less, the Warmage can make it chain out to other targets. This increases the casting time to take up one round and then a full round action on the following turn. The spell will then chain out, with half of its original effect, to a number of secondary targets up to half her caster level. They are still entitled to any saving throws, and no-one may be targeted multiple times (additional chains are wasted if there are too few targets). The maximum distance from primary to secondary target is equal to half the original range.

By adding yet another full round to the casting time, the secondary targets will then chain out to tertiary targets (but they can't all chain back to each other. Again, any given person is effected only once per casting).}

\ability{Rallying Spell:}{Whenever the Warmage casts a spell that successfully takes a foe out of combat, she may elect for it to become a rallying spell. There is a blast of loud noise and she lights up (suppressing any Darkness effects) for one round, and all allies within 30' gain a morale bonus on Attack rolls and Will saves equal to the spell level, lasting for one round.}

\ability{Energy Blitz:}{Incredible effects apply to Warmage spells that deal energy damage.

\begin{tabular}[h]{|l|p{12cm}|}
\hline
Energy: &Effect:\\
\hline
Acid &All of the target's non-magical equipment on their person is dissolved immediately, and they are Blinded for the rest of the encounter\\
\hline
Cold &The target is Paralysed for 1 round, then Slowed for 2 more\\
\hline
Electricity &The target becomes Confused and Entangled for the rest of the encounter\\
\hline
Fire &The target becomes a raging inferno, as though set on fire for 5d6 Fire damage per round, setting adjacent subjects on fire as well\\
\hline
Sonic &The target is knocked prone and rendered Staggered and unable to stand for 4 rounds\\
\hline
\end{tabular}

The Warmage may elect to apply Energy Effect, Energy Surge or Energy Blitz, but no combination of the three at a time. As with Energy Effect and Surge, there is a saving throw.}

\ability{Heroes Feast:}{Once per day, the Warmage may cast Heroes Feast as a Spell-like Ability, except that it can affect three times as many people. This may be cast again if a major victory is scored (such as the defeat of an enemy army or capture of a large castle. Ask your DM).}

\ability{Instant Fortress:}{The Warmage may summon a Fortress as a Spell-like Ability. The fortress appears after three consecutive rounds of concentration, and then functions like the item of the same name. If packed up, then it regenerates all damage within 24 hours. If destroyed, it takes a week for another to be called forth. If the concentration time is extended out to one minute, then the fortress appears as two towers connected by a wall 25' tall, 10' thick and 50' long, with arrow slits and battlements at the top. If extended out to ten minutes, it becomes four towers, each connected by one such wall in a square formation, with a 60' tall tower in the centre, connected to the walls by 10' thick, 10' tall, 20' long corridors.

The fortress requires as much time to pack up as to set up.}

\ability{Arcane Ordinance:}{By focusing for two full rounds in addition to the original casting time, the Warmage may quadruple the area of effect of an area spell. By focusing for five rounds in addition to the original casting time, she may multiply the area by ten. However, either benefit will only apply to the first round (unless Instantaneous), unless the Warmage maintains concentration on the spell.}

\ability{Energy Mastery:}{Whenever the Warmage kills a target with a spell that deals a type of Energy damage that she could change a spell to do (Acid, Cold, Electricity, Fire, Force, Negative, Sonic), there is an explosion of energy. Everyone adjacent to the target takes 1d4 damage of the same type per caster level (Reflex half, DC 10 + half the Warmage's HD + her Int modifier), with no Effects, Surges or Blitzes applied to this. This is not a spell, so anyone slain by this does [b]not[/b] also explode, creating an infinite peasant chain of ordinance.}

\ability{Dimensional Fortress:}{Once per day, the Warmage may cast Mordenkainen's Magnificent Mansion. Anyone who attempts to gain entry uninvited triggers an alarm and  a Disintegrate effect. The DC of this effect is 10 + half the Warmage's hit dice + her Caster Level.}

\ability{Hellstorm:}{Once per day, the Warmage may call a devastating blitz of arcane power upon her foes, designed to force everyone to stay in cover, but also doing a reasonable job of tearing buildings apart so the battle can truly begin. 

The Warmage must spend a full round action concentrating, and then designate the area of effect: two 10x10' squares per caster level. After this, the effect is out of her hands - even if she wants to, she cant stop it. The area is struck by an Earthquake that lasts for one minute, and every round, a single meteor (as per Meteor Swarm) strikes the location of her choosing (if she does not make a selection, a random area is picked, generally large or tall targets such as buildings go first) until the meteor is up. Anyone inside a structure that is not destroyed by the earthquake/meteor is perfectly safe from this damage.

Additionally, anyone who is not enjoying physical cover of at least 50\% takes Acid damage and Fire damage, each equal to the Warmage's class level, every round. The only way to avoid this (aside from energy resistance/immunity) is to gain cover or leave the area of effect.

At the end of the minute, all effects stop, except for the pits and difficult ground, which is instantaneous and thus doesn't go away, and everyone within the area must make a Fort save (DC 10 + half the Warmage's hit dice + her Int modifier) or be Stunned for 3 rounds, allowing the invading army to rush in and start the violence.}

\ability{Dragonflight Bombardment:}{The Warmage gains a loyal ally: a Dragon with a true CR that must be at least 3 less than the Warmage's character level. If it dies, another turns up in 1d6+6 days. This dragon acts as a mount, and will gladly fight in melee or rain its breath weapon down upon the battlefield. Additionally, as long as both are in physical contact, the Warmage may cast any damaging spell through the dragon, affecting everyone in the Area of Effect of the breath weapon. If the duration was more than Instantaneous, it becomes ``one round''.}

\ability{Arcane Siege:}{Three times per day, the Warmage may target a building she can see and, as a Supernatural Ability, cause it to crack open. If the building fails a Fortitude save (use the best save of the inhabitants, if none it automatically fails) it splits apart, tumbling to the ground and dealing 20d6 damage to all of the inhabitants who are likely trapped and unable to escape. Clouds of dust equal to a Sandstorm are churned up for 1 minute, as well.}

\ability{Win:}{The Warmage wins the game. There is no saving throw for this. Note that this doesn't actually affect the game.}

\spelllist{Warmage Spell List:}

\small\ability{0th level:}{\emph{Dancing Lights, Darkness, Detect Magic, Detect Poison, Disrupt Undead, Ghost Sound, Light, Magic Missile, Mending, Message, Purify Food and Drink, Rouse, Stand}}

      \ability{Level 1:}{\emph{Alarm, Detect Scrying, Detect Secret Doors, Disguise Self, Endure Elements, Entangle, Firespray, Grease, Mass Rouse, Mass Stand, Mount, Obscuring Mist, Shocking Grasp, Silent Image, Ventriloquism}}

      \ability{Level 2:}{\emph{Arcane Lock, Continual Flame, Fireball, Jet of Steam, Gust of Wind, Incendiary Slime, Locate Object, Mass Enlarge Person, Mass Reduce Person, Misdirection, Protection From Arrows, Pyrotechnics, Resist Energy, Scare, Scorching Ray, See Invisibility, Whispering Wind}}

      \ability{Level 3:}{\emph{Caustic Mire, Caustic Smoke, Deep Slumber, Dispel Magic, Flame Arrow, Fly, Horrid Sickness, Invisibility Sphere, Lightning Bolt, Major Image, Mass Resist Energy, Mordenkainen's Faithful Hound, Nondetection, Protection From Energy, Rage, Sleet Storm, Stinking Cloud, Wall of Fire, Water Breathing, Wind Wall}}

      \ability{Level 4:}{\emph{Confusion, Dimension Door, Dimensional Anchor, Evard's Black Tentacles, Fear, Fire Shield, Fire Trap, Hallucinatory Terrain, Ice Storm, Move Earth, Scrying, Shout, Solid Fog, Stone Shape, Wall of Fire, Wall of Ice}}

      \ability{Level 5:}{\emph{Cloudkill, Cone of Cold, Deltane's Fiery Tentacles, Fire and Brimstone, Fire Seeds, Greater Dispel, Dismissal, Fabricate, Lightning Leap, Mind Fog, Mirage Arcana, Nightmare Terrain, Prying Eyes, Seeming, Sending, Tactical Teleportation, Telepathic Bond, Transmute Mud to Rock, Transmute Rock to Mud, Wall of Force, Wall of Stone, Waves of Fatigue}}

      \ability{Level 6:}{\emph{Acid Fog, Chain Lightning, Contingency, Guards and Wards, Mass (Bull's Strength, Cat's Grace, Bear's Endurance), Mass Suggestion, Storm of Fire and Ice, Sunburst, True Seeing, Veil, Wall of Iron, Wall of Thorns}}

      \ability{Level 7:}{\emph{Banishment, Control Weather, Deadly Lahar, Earthquake, Forcecage, Greater Teleport, Incendiary Cloud, Mass Hold Person, Mass Invisibility, Mordenkainen's Sword, Phase Door, Planeshift, Prismatic Spray, Reverse Gravity, Waves of Exhaustion, Whirlwind}}

      \ability{Level 8:}{\emph{Deadly Sunstroke, Dimensional Lock, Greater Prying Eyes, Greater Shout, Horrid Wilting, Mass Charm Monster, Mass Polymorph, Prismatic Wall, Storm of Vengeance}}

      \ability{Level 9:}{\emph{Astral Projection, Elemental Swarm, Etherealness, Firestorm, Foresight, Mass Hold Monster, Meteor Swarm, Prismatic Deluge, Prismatic Sphere, Timestop, Towering Thunderhead, Wail of the Banshee}}
\normalsize

\spelllist{Altered Spells:}
\ability{Chain Lightning:}{This acts as a Lightning Bolt (and may be bounced, as below), except that every time a target takes damage from it (whether they make the save or not), another smaller bolt shoots out to another target of your choice within 50 feet. This requires a Ranged Touch Attack, and if the target is struck, they take half the original damage. If a bolt bounces over someone multiple times, multiple secondary bolts are launched. An individual can be hit by the primary bolt and one or more secondary bolts (though not secondary bolts that are caused by them getting struck) in the same casting.}

\ability{Cone of Cold:}{This works like normal, except the entire area of effect is coated in ice (functioning like a Grease spell that is automatically Dispelled, square-by-square, by Fire effects) and anyone who fails the save takes 1d6 Dexterity damage.}

\ability{Firestorm:}{This spell functions as normal, except that it lasts for as long as the Warmage concentrates. The area of effect cannot be moved, however.}

\ability{Flame Arrow:}{When cast, all arrows fired from within 50' of the caster during the next round deal an additional amount of Fire Damage equal to 2d6 plus her Caster Level.}

\ability{Lightning Bolt:}{When cast by a Warmage, Lightning Bolts may bounce off surfaces, either at a 90 degree angle or 180 degrees straight back at the caster, caster's choice.}

\ability{Mass X:}{The spell works just as normal, except it affects one subject per caster level.}

\ability{Mass Polymorph:}{This works only on willing targets, affecting two subjects per caster level. The subjects are transformed into Trolls, effectively replacing their character sheets with the MM entry. They can think like themselves, retaining their alignments and memories, but don't actually retain their skills, feats, ability scores (even mental ones) or anything else. You don't get troll wizards and troll knights and troll thief-acrobats, you get trolls.

These trolls may, however, utter ``Hurr hurr, I'm a (their original class)'' as a free action. Troll wizards can be said to cast sleep (range: touch, material focus: a club, somatic component: make an attack roll) and Knock (range: touch, material focus: a club, somatic component: make a break check), likewise Troll Rogues can be said to be able to pick locks (with a club) and disable devices (with a club).}

\ability{Meteor Swarm:}{this causes four meteors to rain down from the sky, all at different points. Make a ranged touch attack against four different targets (including sections of ground). The targets, if hit, take 10d6 Bludgeoning Damage. Then, all within a 20' radius of each meteor takes 1d6 Fire damage per caster level (if multiple blasts hit the same person, they are damaged multiple times) with a Ref save for half (each time, if struck multiple times). If the targets of the meteors were hit by the attacks, they automatically fail the save.

The area then becomes difficult ground, with the 5' square points of impact becoming deep craters (10' deep pits).}


\chapter{Bonus Core Classes}

\classname{Elementalist}
\vspace{-8pt}
\quot{``Feel the wrath of the natural order you have angered!"}

\ability{Alignment:}{An prospective Elementalist must be of a neutral alignment (N, NG, NE, LN, or CN) in order to learn the secrets of Elementalism. Once a character already has at least one level of Elementalist, nothing prevents her from changing alignment.}

\ability{Races:}{Every race has elementalists, but races that have more draconic heritage mixed in have decidedly more elementalists. Dwarves have a natural affinity for stone and often choose the route of the Elementalist. Kobolds are naturally inclined towards elementalism and only jealous guarding of the secrets of elementalism by}

\ability{Starting Gold:}{4d4x10 gp (100 gold)}

\ability{Starting Age:}{As Druid.}

\ability{Hit Die:}{d6}

\ability{Class Skills:}{The Elementalist's class skills (and the key ability for each skill) are Concentration (Con), Craft (Int), Decipher Script (Int), Diplomacy (Cha), Disable Device (Int), Escape Artist (Dex), Handle Animal (Cha), Heal (Wis), Intimidate (Cha), Knowledge (arcana) (Int), Knowledge (nature) (Int), Knowledge (the planes) (Int), Profession (Wis), Ride (Dex), Search (Int), and Spellcraft (Int).}

\ability{Skills/Level:}{4 + Intelligence Bonus}


\begin{table}[htb]
\begin{small}
\begin{tabular}{lp{1.9cm}p{0.7cm}p{0.7cm}p{0.7cm}l}
&   Base Attack Bonus&  Fort Save&  Ref Save&   Will Save&  Special\\
1&  +0& +0& +0& +2&  Armored Casting, \spell{Create Water} \\
2&  +1& +0& +0& +3&  Elemental Survival\\
3&  +1& +1& +1& +3&  Advanced Learning\\
4&  +2& +1& +1& +4&  \spell{Create Air}\\
5&  +2& +1& +1& +4&  Elemental Faminilar, Advanced Learning\\
6&  +3& +2& +2& +5&  \\
7&  +3& +2& +2& +5&  Resistance to Energy: 5, Advanced Learning\\
8&  +4& +2& +2& +6&   \\
9&  +4& +3& +3& +6&  \spell{Create Fire}, Advanced Learning\\
10& +5& +3& +3& +7&  Timelessness  \\
11& +5& +3& +3& +7&  Advanced Learning, Elemental Traits   \\
12& +6& +4& +4& +8&   \\
13& +6& +4& +4& +8&  Resistance to Energy: 10, Advanced Learning   \\
14& +7& +4& +4& +9&  \spell{Create Wood}   \\
15& +7& +5& +5& +9&  Advanced Learning, Improved Summoning \\
16& +8& +5& +5& +10&     Elemental Wildshape 1/day \\
17& +8& +5& +5& +10&     \spell{Create Earth}, Advanced Learning   \\
18& +9& +6& +6& +11&     Elemental Wildshape 2/day \\
19& +9& +6& +6& +11&     Resistance to Energy: 15, Advanced Learning   \\
20& +10&    +6& +6& +12&     Elemental Wildshape 3/day \\

\end{tabular}
\end{small}
\end{table}



\smallskip\noindent All of the following are Class Features of the Elementalist class.

\ability{Weapon and Armor Proficiency:}{Elementalists are proficient with all simple weapons, as well as the scimitar, the battle axe, the trident, the pick (heavy and light), and the longbow (including composite longbows). Elementalists are proficient with light armor but not with shields of any kind.}

\ability{Spellcasting:}{The Elementalist is an Arcane Spellcaster with the same spells per day progression as a Sorcerer. An Elementalist casts spells from the Elementalist Spell List (below). An Elementalist automatically knows every spell on her spell list. She can cast any spell she knows without preparing them ahead of time, provided that spell slots of an appropriate level are still available.

To cast an Elementalist spell, she must have an Intelligence at least equal to 10 + the Spell level. The DC of the Elementalist's spells is Wisdom based and the bonus spells are Intelligence based.}

\ability{\spell{Create Water (Su)}:}{An Elementalist can create water as a standard action at will as the spell create water with a caster level equal to her character level.}

\ability{Armored Casting:}{An Elementalist casts arcane spells, but she is not affected by the arcane spell failure of any armor or shield she is proficient with. This ability only applies to her Elementalist spells, if she is able to cast any other arcane spells, they are affected by arcane spell failure normally.}

\ability{Elemental Survival (Ex):}{An Elementalist of 2nd level or higher survives in elemental planes as easily as on the prime. Whenever on any elemental, paraelemental, or energy plane, she is able to ignore any harmful planar traits and moves through any of these planes without impediment.}

\ability{Advanced Learning:}{At 3rd level and every two levels afterwards, the Elementalist may permanently add one spell to her spell list. This spell must be of a level she can already cast, and may not be of the Illusion or Necromancy school. Only spells from the Druid or Wu Jen spell list may be added in this way.}

\ability{\spell{Create Air (Su)}:}{At 4th level an Elementalist can create air at will as if she was an open Bottle of Air.}

\ability{Elemental Familiar:}{At 5th level, an Elementalist can acquire a familiar in the same manner as a Sorcerer. Unlike a Sorcerer, the Elementalist has only 5 choices for her familiar:}

\listone
    \item \ability{Small Air Elemental:}{+2 to Tumble and Jump Checks.}
    \item \ability{Small Earth Elemental:}{+3 on Bullrush checks, whether the attacker or defender.}
    \item \ability{Small Fire Elemental:}{+3 to Intimidate checks.}
    \item \ability{Small Water Elemental:}{+4 bonus to Swim checks, you may take 10 on swim checks at any time.}
    \item \ability{Small Wood Elemental:}{+2 to Survival and Climb checks.}
\end{list}


\ability{Resistance to Energy (Ex):}{At 7th level, an Elementalist has accumulated an inherent resilience in the face of all manners of elemental adversity. She has an Energy Resistance of 5 against any form of energy damage she is exposed to. At 13th level, this general resistance increases to 10 points. At 19th level, the resistance increases to 15.}

\ability{\spell{Create Fire (Su)}:}{At 9th level, an Elementalist can set a creature or object on fire at will as a standard action. The target must be within short range, and suffers 2d6 of fire damage every round until the fire is extinguished. A victim can attempt to extinguish itself as a full-round action by making a Reflex Save (DC 15). The flames, once begun, are non-magical.}

\ability{Timeless:}{At 10th level, an Elementalist is infused with the uncompromising nature of the raw elements themselves. Se stops aging and never dies of old age.}

\ability{Elemental Traits:}{At 11th level, an Elementalist attunes herself to a specific element, becoming in some way like the element of her choice:}

\listone
    \item \ability{Air Elemental:}{Gains Air Mastery and a Flight Speed (perfect maneuverability) equal to her walking speed.}
    \item \ability{Earth Elemental:}{Gains Earth Mastery, a 30' Tremorsense, and a Burrowing Speed equal to half her walking speed.}
    \item \ability{Fire Elemental:}{Gains Immunity to Fire, and her body immolates whenever desired inflicting an additional 4d6 of fire damage on any creature struck with her unarmed strikes or which strike her with an unarmed strike.}
    \item \ability{Water Elemental:}{Gains Water Mastery, the [Aquatic] subtype, and a swim speed equal to her walking speed.}
    \item \ability{Wood Elemental:}{Gains Immunity to Polymorphing, a 60 foot Woodsense, and a Climb Speed equal to her walking speed.}
\end{list}


\ability{\spell{Create Wood (Su)}:}{At 14th level, an Elementalist can create a full sized tree as a standard action, as if using a Quaal's Feather Token (Tree). This is an at-will ability.}

\ability{Elemental Wildshape (Su):}{At 16th level, the Elementalist can actually become an Elemental in a manner simply to wildshape. Once per day, an Elementalist can assume an Alternate Form of an Air Elemental, an Earth Elemental, a Fire Elemental, a Water Elemental, or a Wood Elemental as a standard action. The alternate form may be dismissed at will, but otherwise persists for 24 hours. Only True Elemental forms may be assumed, and the forms in question must be smaller than Elder (Small to Huge Size is acceptable). Every 2 levels, the Elemental Wildshape may be activated an extra time each day.}

\ability{\spell{Create Earth (Su)}:}{At 17th level, an Elementalist can create a wall of stone at any time as a standard action.}

\spelllist{Elementalist Spell List:}

\small\ability{0th level:}{\emph{Attune Form, Caltrops, Detect Magic, Detect Poison, Light}}

      \ability{1st level:}{\emph{Air Breathing, Entangle, Gust of Wind, Obscuring Mist, Pass Without Trace, Produce Flame, Speak With Plants, Stone Shatter, Summon Elemental I, Wall of Smoke, Water Breathing}}

      \ability{2nd level:}{\emph{Binding Winds, Command Plants, Creeping Cold, Earth Bind, Earthen Grace, Fog Cloud, Heat Metal, Protection From Arrows, Soften Earth and Stones, Summon Elemental II*, Wall of Sand, Warp Wood, Wood Shape}}

      \ability{3rd level:}{\emph{Blight, Control Water, Earth Reaver, Fire Shield, Fly, Plant Growth, Stone Shape, Stone Skin, Summon Elemental III*, Wall of Fire, Wall of Water, Wind Wall}}

      \ability{4th level:}{\emph{Briar Web, Greater Stone Shape, Scry, Summon Elemental IV*, Wall of Stone}}

      \ability{5th level:}{\emph{Animate Plants, Move Earth, Stone Tell, Summon Elemental V*, Wall of Thorns}}

      \ability{6th level:}{\emph{Control Plants, Energy Immunity, Flesh to Stone, Stone to Flesh, Summon Elemental VI*}}

      \ability{7th level:}{\emph{Greater Scrying, Summon Elemental VII*, Transmute Rock to Lava}}

      \ability{8th level:}{\emph{Summon Elemental VIII*}}

      \ability{9th level:}{\emph{Elemental Swarm, Summon Elemental IX*, Summon Elemental Monolith}}
\normalsize

\classname{Fire Mage} \label{class:firemage}
\vspace*{-8pt}
\quot{"Yes, fire \emph{is} cool."}

\ability{Alignment:}{Fire is a destructive force, and a lot of Fire Mages are Chaotic. But they don't have to be.}

\ability{Races:}{Fire Mages appear in all races, though significant portions of many races live in areas where being a Fire Mage is illegal.}

\ability{Starting Gold:}{6d6x10 gp (210 gold)}

\ability{Starting Age:}{As Rogue.}

\ability{Hit Die:}{d8}

\ability{Class Skills:}{The Fire Mage's class skills (and the key ability for each skill) are Bluff (Cha), Climb (Str), Craft (Int), Concentration (Con), Disguise (Cha), Escape Artist (Dex), Handle Animal (Cha), Intimidate (Cha), Jump (Str), Listen (Wis), Move Silently (Dex), Profession (-), Ride (Dex), Search (Int), Spellcraft (Int), Spot (Wis), Survival (Wis), and Use Rope (Dex).}

\ability{Skills/Level:}{4 + Intelligence Bonus}


\begin{table}[htb]
\begin{small}
\begin{tabular}{lp{3cm}p{0.7cm}p{0.7cm}p{0.7cm}l}
Level &Base Attack Bonus &Fort Save &Ref Save &Will Save &Special\\
1st &+0 &+2 &+2 &+2 &Fire Resistance, Fire Burst, Fire Bolts, Impress Flames, Fire Magic\\
2nd &+1 &+3 &+3 &+3 &Ignite\\
3rd &+2 &+3 &+3 &+3 &Piercing Flames, Hand of Fire\\
4th &+3 &+4 &+4 &+4 &Fire Immunity, Smokeless Flame\\
5th &+3 &+4 &+4 &+4 &Fireballs\\
6th &+4 &+5 &+5 &+5 &Mindfire\\
7th &+5 &+5 &+5 &+5 &Visions of Flame\\
8th &+6/+1 &+6 &+6 &+6 &Soul of Cinders\\
9th &+6/+1 &+6 &+6 &+6 &Sculpt Flames\\
10th &+7/+2 &+7 &+7 &+7 &Conflagration\\
11th &+8/+3 &+7 &+7 &+7 &Beacon, Firewalk\\
12th &+9/+4 &+8 &+8 &+8 &Bond of Fire\\
13th &+9/+4 &+8 &+8 &+8 &Fire Clouds\\
14th &+10/+5 &+9 &+9 &+9 &Searing Light, Ray of Light\\
15th &+11/+6/+6 &+9 &+9 &+9 &Sending, Rain of Fire\\
\end{tabular}
\end{small}
\end{table}

\smallskip\noindent All of the following are Class Features of the Fire Mage class.

\ability{Weapon and Armor Proficiency:}{Fire Mages are proficient with all simple weapons, as well as the whip, all martial axes, and all sizes and varieties of scimitar (including falchions). Fire Mages are proficient with light armor but not with shields of any kind.}

\ability{Fire Resistance (Ex):}{A Fire Mage has a Resistance to Fire equal to twice his level.}

\ability{Fire Burst (Sp):}{As a standard action, a Fire Mage can emit a burst of flame from his body, striking all creatures and objects within 10' of his position except himself. This burst of flames inflicts 1d6 of fire damage, with an allowed Reflex Save for half (DC 10 + \half Level + Charisma Modifier).}

\ability{Fire Bolts (Sp):}{A Fire Mage can throw bolts of fire as an attack action. A Fire Bolt travels out to short range, and inflicts 1d6 of Fire damage per level. A Fire Bolt strikes its target with a ranged touch attack.}

\ability{Impress Flames (Ex):}{Every time a Fire Mage inflicts Fire damage on any target, whether with his class abilities or another source of fire, he inflicts an amount of extra Fire Damage equal to his class level or his Charisma modifier, whichever is less.}

\ability{Fire Magic (Ex):}{A Fire Mage is considered to have every spell with the Fire Descriptor on his spell list for the purpose of activating magic items.}

\ability{Ignite (Sp):}{As a standard action, a 2nd level Fire Mage can cause any creature or object to burst into flame. A creature on fire suffers 1d6 of Fire damage per round (the Mage's Impress Flames ability applies to each round of course), and the creature can attempt to put itself out with a DC 15 Reflex save (see the DMG, p. 303). This ability can be used out to Medium range, and it always hits.}

\ability{Piercing Flames (Ex):}{From 3rd level on, a Fire Mage's Fire cuts through Fire Resistance, hardness, and Immunity. No more than \half of the damage inflicted by his fire damage can be negated by hardness or immunity or resistance to Fire. In addition, the Fire Mage ignores the first 5 points of Fire Resistance that a target has.}

\ability{Hand of Fire (Su):}{A 3rd level Fire Mage can set fire to their own body, causing them to count as armed at all times, even with unarmed attacks. The Fire Mage also causes an extra 1d6 of Fire damage with all melee attacks.}

\ability{Fire Immunity (Ex):}{A 4th level Fire Mage is immune to Fire.}

\ability{Smokeless Flames (Sp):}{A 4th level Fire Mage can create fires that produce no heat and do not burn. These fires can be anything from the size of a torch to a bonfire, and produce light accordingly. Each lasts until the next time the sun rises. Smokeless Flame can be created anywhere within Medium range.}

\ability{Fireballs (Sp):}{A 5th level Fire Mage can hurl explosive fire anywhere within Long Range as a Full Round Action. This Fire explodes into a 20' radius burst and inflicts 1d6 of Fire Damage per level. All creatures within the area are entitled to a Reflex save to halve damage (DC 10 + \half Level + Charisma Modifier).}

\ability{Mindfire (Sp):}{A 6th level Fire Mage can start a Fire in a creature's mind, duplicating the effects of \spell{rage} or \spell{confusion} for a number of minutes equal to his Level. The victim must be within Medium Range, and is entitled to a Will Save to negate this effect (DC 10 + \half Level + Charisma Modifier). This is a Mind influencing Compulsion effect.}

\ability{Visions of Flames (Sp):}{A 7th level Fire Mage can \spell{contact other plane} to communicate with the denizens of the Elemental Plane of Fire. A Fire Mage is in no danger of becoming insane or damaged by this experience.}

\ability{Soul of Cinders (Sp):}{An 8th level Fire Mage has burnt his soul to ash, and is no longer susceptible to Energy Drain or Fear.}

\ability{Sculpt Flames (Sp):}{A 9th level Fire Mage can create delicate shapes and walls made of fire. The Fire is fully shapeable, but cannot pass through more than 2 squares per level. Any creature passing through a square with fire in it suffers 1d6 of fire damage per level. A creature which is in a square that is being filled with fire is entitled to a Reflex Save (DC 10 + \half Level + Charisma Modifier) to move to the nearest non-flaming square as an immediate action. These fires persist for 1 round per level. Alternately, the Fire Mage can replicate a \spell{wall of fire} which persists for 1 minute per level.}

\ability{Conflagration (Sp):}{At 10th level, a Fire Mage can surround himself with a nimbus of flames that extends for 10' in all directions from his person. All other targets in this area suffer a d10 of Fire Damage per level, but are entitled to a Reflex Save (DC 10 + \half Level + Charisma Modifier). In addition, a Fire Mage can cast \spell{fireshield} at will (Hot Shield only).}

\ability{Beacon (Sp):}{An 11th level Fire Mage can create a magically permanent bonfire as a standard action. He always knows exactly where each Beacon he has created is and will know if it is put out by any means.}

\ability{Firewalk (Sp):}{At 11th level a Fire Mage can walk into any fire large enough to fit his person and appear in any other fire that is likewise of sufficient size anywhere on any plane of existence. The Fire Mage must know where the target fire is. The Fire Mage can take any number of willing creatures or carried objects that are also able to fit in both flames.}

\ability{Bonds of Fire (Sp):}{A 12th level Fire Mage can craft solid fire and entrap a victim in it. The bonds will immobilize a creature which fails a Reflex Save (DC 10 + \half Level + Charisma Modifier), and will \condition{entangle} the creature unless it succeeds in its save by more than 5. A creature can attempt to escape by taking a Full round action to make a Strength or Escape Artist test with a DC equal to the Use Rope Skill Result of the Fire Mage. The victim suffers 20 points of Fire Damage per round, and the bonds of fire last until the victim escapes or the Fire Mage dismisses them.}

\ability{Fire Clouds (Sp):}{As a Full Round Action, a 13th level Fire Mage can create huge billowing clouds of Fire. The Fire Clouds must be created within Long range, and persist for 3 rounds whether they are still in range or not. The cloud is shapeable, and covers at most 3 10' cubes per Level. Each round, everyone and everything inside the cloud suffers 1d6 of Fire damage per level, but is entitled to a Reflex save for half damage (DC 10 + \half Level + Charisma Modifier).}

\ability{Searing Light (Sp):}{A 14th level Fire Mage can call levels of illumination that are painful and destructive as the unmitigated baleful glare of the sun itself. All darkness within 5 miles is dispelled, and everything is illuminated. All undead suffer a 10 points of damage per round. All creatures specifically vulnerable to light suffer 10 damage per round (thus, vampires suffer 20 damage per round). All creatures are \condition{dazzled}. Creatures must pass a Fortitude save (DC 10 + \half Level + Charisma Modifier) every minute or become \condition{blind} for the remainder of the effect. Creatures that are blinded when the effect ends are entitled to another Fort save to get their vision back, but if they fail this save the blinding is permanent. This effect lasts until the Fire Mage dismisses it or he is incapacitated.}

\ability{Ray of Light (Sp):}{As an attack action, a 14th level Fire Mage can fire a ray of Light at any target within Short Range. It inflicts 1d6 of Light Damage per level if it hits with a Ranged Tuuch Attack. Undead take 10 extra damage. Creatures specifically vulnerable to Light suffer an additional 10 damage.}

\ability{Sending (Sp):}{A 15th level Fire Mage can send a message, as the \spell{sending} spell to any creature on any plane of existence with a standard action and receive a reply even if they are on different planes of existence.}

\ability{Rain of Fire (Sp):}{At 15th level, the Fire Mage can open the skies and dump raw inferno upon all who would oppose him. The fires inflict 1d6 of Fire Damage per level, and victims are permitted a Reflex save (DC 10 + \half level + Charisma Modifier). The Fire Mage chooses which squares are struck with fire, and the only limits to how many squares can burn is how many squares the Fire Mage can see. There are no range limits to this power save line of sight.}

\classname{Puppeteer} \label{class:puppeteer}
\vspace*{-8pt}
\quot{"This time, I think they'll stay up longer - I used more juice."}

\desc{It is well known to those who investigate such things that electrical energy can bring animation to the freshly dead in much the same way as positive or negative energy can. Those who have a natural inclination towards commanding the lightning can live out their life in obscurity or they can investigate their own abilities.

Those of a particularly investigative bent can accomplish much towards animating the dead and even creating new life. The puppeteer is one such person.}

\ability{Alignment:}{Electricity is a destructive force, but it is also the source of life. The Puppeteer focuses on the animating aspects, and a lot of them are Lawful. But they don't have to be.}

\ability{Races:}{Puppeteers appear in all races, though significant portions of many races live in areas where being a Puppeteer is illegal.}

\ability{Starting Gold:}{6d6x10 gp (210 gold)}

\ability{Starting Age:}{As Rogue.}

\ability{Hit Die:}{d8}

\begin{table}[htb]
\begin{small}
\begin{tabular}{lp{3cm}p{0.7cm}p{0.7cm}p{0.7cm}p{9cm}}
Level &Base Attack Bonus &Fort Save &Ref Save &Will Save &Special\\
1st &+0 &+2 &+2 &+2 &Electricity Resistance, Jolt, Electric Bolts, Puppet the Dead, Electric Magic\\
2nd &+1 &+3 &+3 &+3 &Disrupting Shock, Familiar\\
3rd &+2 &+3 &+3 &+3 &Greased Lightning, Repair Construct\\
4th &+3 &+4 &+4 &+4 &Electricity Immunity, Arc Light, Devastating Thunder\\
5th &+3 &+4 &+4 &+4 &Perpetual Storm, Corpse Quickening\\
6th &+4 &+5 &+5 &+5 &Persistent Puppets\\
7th &+5 &+5 &+5 &+5 &Lightning Bolts\\
8th &+6/+1 &+6 &+6 &+6 &Create Golem\\
9th &+6/+1 &+6 &+6 &+6 &Army of Puppets\\
10th &+7/+2 &+7 &+7 &+7 &Tunneling\\
11th &+8/+3 &+7 &+7 &+7 &Magnetism\\
12th &+9/+4 &+8 &+8 &+8 &Life Anew\\
\end{tabular}
\end{small}
\end{table}

\ability{Class Skills:}{The Puppeteer's class skills (and the key ability for each skill) are Bluff (Cha), Craft (Int), Concentration (Con), Diplomacy (Cha), Disable Device (Int), Disguise (Cha), Handle Animal (Cha), Intimidate (Cha), Knowledge (Each skill individually, Int), Listen (Wis), Move Silently (Dex), Profession (-), Ride (Dex), Search (Int), Spellcraft (Int), Spot (Wis), Survival (Wis), and Use Rope (Dex).}

\ability{Skills/Level:}{ 4 + Intelligence Bonus}

\smallskip\noindent All of the following are Class Features of the Puppeteer class:

\ability{Weapon and Armor Proficiency:}{Puppeteers are proficient with all simple weapons, as well as the whip, all martial spears, and all sizes and varieties of chain (including spiked chains). Puppeteers are proficient with light armor but not with shields of any kind.}

\ability{Electricity Resistance (Ex):}{A Puppeteer has a Resistance to Electricity equal to twice his level.}

\ability{Jolt (Su):}{As a standard action, a Puppeteer can electrify his body, shocking the next creature which he touches or which touches him during the next minute. This shock inflicts 1d6 of electricity damage, with an allowed Fortitude Save for half (DC 10 + 1/2 Level + Charisma Modifier).}

\ability{Electric Bolts (Sp):}{A Puppeteer can throw bolts of electricity as an attack action. A Lightning Bolt tavels out to short range, and inflicts 1d6 of Fire damage per level. A Lightning Bolt strikes its target with a ranged touch attack.}

\ability{Puppet the Dead (Su):}{If the Puppeteer can inflict electricity damage on a corpse, he can can it to rise as a zombie. This zombie can't have more than 4 hit dice for every level of puppeteer he possesses, and it immediately collapses if it has been active for more than 10 minutes or if the puppeteer animates a second corpse.
\begin{list}{}{\itemspace}\item These zombies are of the construct type rather than being true undead, and are healed by electricity damage. Otherwise use the normal zombie template\end{list}}

\ability{Electric Magic (Ex):}{A Puppeteer is considered to have every spell with the Electricity Descriptor on his spell list for the purpose of activating magic items.}

\ability{Disrupting Shock (Sp):}{As a standard action, a 2nd level Puppeteer can create an electrical discharge inside another creature's body. This effect causes a d6 of damage and stuns the target for one round. The victim is entitled to a Fortitude save (DC 10 + 1/2 Level + the Puppeteer's Intelligence bonus) to halve the damage and negate the stunning effect. This ability can be used out to Medium range, and it always hits.}

\ability{Familiar:}{At 2nd level, a Puppeteer is entitled to a familiar. They may choose a corpse familiar or a construct familiar, but not a normal living animal.}

\ability{Greased Lightning (Ex):}{From 3rd level on, a Puppeteer's Electricity cuts through Electricity Resistance, hardness, and Immunity. No more than 1/2 of the damage inflicted by his electrical damage can be negated by hardness or immunity or resistance to electricity. In addition, the Puppeteer ignores the first 5 points of Electricity Resistance that a target has.}

\ability{Repair Construct (Sp):}{A 3rd level Puppeteer can energize a construct with a touch. This touch heals 2d8+Level hit points, and is be usable at any time.}

\ability{Electricity Immunity (Ex):}{A 4th level Puppeteer is immune to Electricity.}

\ability{Arc Light (Sp):}{A 4th level Puppeteer may shed light like a \emph{daylight} spell from his own body. The clearly electrical light emanates from any portion of the character's body and can be begun or ended as a move action.}

\ability{Devastating Thunder (Ex):}{When a 4th level Puppeteer inflicts electrical damage on any target, he inflicts an additional amount of that damage equal to his Intelligence modifier or his class level, whichever is less.}

\ability{Perpetual Storm (Sp):}{A 5th level Puppeteer benefits from \emph{call lightning} at all times.}

\ability{Corpse Quickening (Ex):}{When a 5th level Puppeteer animates a corpse, it is not limited to a single standard action.}

\ability{Persistent Puppets (Su):}{A 6th level Puppeteer can create lightning zombies which last an entire day before falling apart on their own.}

\ability{Lightning Bolts (Sp):}{At 7th level, the Puppeteer can send forth a \emph{lightning bolt} as the sorcerer/wizard spell, at will. This spell-like ability has a save DC of 10 + 1/2 Level + Intelligence Modifier. Unlike the normal spell, a Puppeteer's Lightning Bolt has no damage cap.}

\ability{Create Golem (Su):}{An 8th level Puppeteer can create Flesh Golems. These do not require the expenditure of XP. The latest creation, and \emph{only} the last one created by the Puppeteer is immune to the berserking trait as it is fully under his control.}

\ability{Army of Puppets (Sp):}{A 9th level Puppeteer's animated corpses no longer collapse when he raises another puppet, so long as his total number of puppet's is less than his class level.}

\ability{Tunneling (Sp):}{At 10th level, a Puppeteer can teleport short distances, as per \emph{dimension door}. This ability is usable at will.}

\ability{Magnetism (Sp):}{An 11th level Puppeteer can hurl metal objects around at high speed for no discernible reason. This acts like \emph{telekinesis}, which is usable at will, save that only creatures and objects made primarily of ferrous metals may be lifted and thrown.}

\ability{Life Anew (Su):}{A 12th level Puppeteer can create Corpse Creatures with lightning. This is like using \emph{create undead} save that the creatures are Constructs instead of Undead, are healed by Electrical damage, and don't have any action reduction.}
\classname{Snowscaper}
\quot{''I think you need to chill out.''}

\desc{Snowscapers tap into the Plane of Ice, and may create ice and cold and snow when they want to. They're better on their home turf (cold snowfields), but the mark of a powerful snowscaper is being able to make their own snow and ice.}

\ability{Starting Gold:}{6d6x10 gp (210 gold)}

\ability{Starting Age:}{As Rogue.}

\ability{Hit Die:}{d8}

\begin{table}[htb]
\begin{small}
\begin{tabular}{lp{3cm}p{0.7cm}p{0.7cm}p{0.7cm}l}
Level &Base Attack Bonus &Fort Save &Ref Save &Will Save &Special\\
1st &+0 &+2 &+2 &+2 &Cold Resistance, Cold Magic, Pall of Frost, Frozen Heart\\
2nd &+1 &+3 &+3 &+3 &Brittling, Create Ice, Skate\\
3rd &+2 &+3 &+3 &+3 &Frost�s Bite, Ice Skating, Ground Freeze.\\
4th &+3 &+4 &+4 &+4 &Cold Immunity, Encumber\\
5th &+3 &+4 &+4 &+4 &Icebeam, Let It Snow\\
6th &+4 &+5 &+5 &+5 &Freeze, Never-melt Ice\\
7th &+5 &+5 &+5 &+5 &Skate on Air, Create More Ice\\
8th &+6/+1 &+6 &+6 &+6 &Wall of Ice\\
9th &+6/+1 &+6 &+6 &+6 &Blizzard, Mirror Mirror On The Wall\\
10th &+7/+2 &+7 &+7 &+7 &Through the Looking Glass\\
11th &+8/+3 &+7 &+7 &+7 &Animate Snow\\
12th &+9/+4 &+8 &+8 &+8 &Create Tons of Ice\\
13th &+9/+4 &+8 &+8 &+8 &Wintersmith, The Great Blizzard of '52.\\
\end{tabular}
\end{small}
\end{table}

\ability{Class Skills:}{The Snowscaper's class skills (and the key ability for each skill) are Bluff (Cha), Climb (Str), Craft (Int), Concentration (Con), Disguise (Cha), Escape Artist (Dex), Handle Animal (Cha), Intimidate (Cha), Jump (Str), Listen (Wis), Move Silently (Dex), Profession (-), Ride (Dex), Search (Int), Spellcraft (Int), Spot (Wis), Survival (Wis), and Use Rope (Dex).}

\ability{Skills/Level:}{4 + Intelligence Bonus}

\ability{Weapon and Armor Proficiency:}{The Snowscaper is proficient with all simple weapons, all martial swords and piercing weapons, and any three exotic piercing or slashing weapons he wants. The Snowscaper is proficient with light armor but not with shields.}

\ability{Cold Resistance (Ex):}{At level 1, the Snowscaper gains Cold Resistance equal to her character level.}

\ability{Cold Magic (Ex):}{All Cold spells are considered spells known for the purposes of magic item activation.}

\ability{Coldfire (Su):}{At level 1, the Snowscaper may form a semi-solid ball of pure cold energy and then throw it at an enemy as an standard action, where it'll burst upon impact. This is a ranged touch attack with a Short range, and does 1d6 cold damage per level.}

\ability{Pall of Frost (Su):}{At level 1, the Snowscaper may frost herself over and chill the air around herself within 10 feet, inflicting 1d6 cold damage to everyone within the radius. While frosted, the Snowscaper is considered armed, and all of her melee attacks do 1d6 extra cold damage.}

\ability{Frozen Heart (Su):}{A Snowscaper gets her Charisma bonus or her character level (whichever is lower) to Cold damage.}

\ability{Brittling (Su):}{At level 2 the Snowscaper may concentrate her will upon a person or object within medium range, and chill the target. This always hits, and does 1d6 cold damage, and makes the target lose their Dex bonus to AC for one round (two if the target is Cold-vulnerable). Objects have their hardness halved for 1d4 rounds; this goes for any creature which also has a Hardness score.}

\ability{Create Ice Object (Su):}{At level 2 a Snowscaper may use a standard action to create any object or objects she's seen seen before--out of ice. They may be created in the air or in a square of your choice (within range), 10 lbs per character level, short range, a number of objects equal to your character level. Hardness equal to 10 + � Character level + Charisma Modifier, but takes 1d6 damage each round it�s in non-freezing temperature (which the Snowscaper can get around). Treat it as having the HP of a material of a similar hardness, if you care that much about sundering. Also, creating your maximum amount is a full-round action.}

\ability{Skate (Su):}{At level 2, The Snowscaper may [i]skate[/i] at will, as per the psionic power, on a line of ice she creates ahead of her as she moves. The trail remains iced over for one round. This may also be used on natural ice.}

\ability{Frost�s Bite (Su):}{At level 3, the Snowscaper�s ice abilities generate a bitter cold. Her cold abilities penetrate Cold immunity, resistance, and hardness.}

\ability{Ice Skating (Su):}{At level 3, you may use Skate to cross liquid surfaces. The ice trail still disappears after a round, so you'd better keep moving, especially if you're crossing lava or acid.}

\ability{Ground Freeze (Su):}{At level 3, the Snowscaper may freeze four 5-foot squares per character level as a standard action. The ice in any square can be thawed using a fire spell. Also, the Snowscaper may now fix Ice objects onto horizontal and vertical surfaces and be sure they'll support a decent amount of weight.}

\ability{Cold Immunity (Su):}{At level 4, the Snowscaper does not fear cold, and is immune to it.}

\ability{Encumber (Su):}{At level 4, the Snowscaper�s creation abilities are getting better, allowing her to do more and more things. She may spend a standard action to attempt to wrap someone in heavy, restricting ice�as much as she can make with Create Ice--encumbering them with the weight, with a Reflex save for half the weight.}

\ability{Icebeam (Su):}{The Snowscaper's mastery of coldfire has expanded to allow her to fire a ray of it. Long-range ray as a standard action, 1d6 Cold Damage per character level.}

\ability{Let It Snow (Su):}{As a standard action, the Snowscaper may it snow in a medium-range radius, in as wide or as small an area as you want (within the radius). It starts off with a foot of snow, and goes up a foot every round until you tell it to stop.}

\ability{Freeze (Su):}{At level 6, the Snowscaper may bind someone or something up with ice. Treat as a Hold Monster, but with a Reflex save instead.}

\ability{Never-melt Ice (Su):}{At level 6, the Snowscaper�s ice creations are immune to all fire and won�t melt unless the Snowscaper allows it or the Snowscaper dies.}

\ability{Skate In Air (Su):}{At level 7, the Snowscaper may now skate through the air by creating a sheet of ice to travel across. The largest angle at which the Snowscaper can travel upwards is 45 degrees (as per Air Walk).}

\ability{Create More Ice (Su):}{At level 7, the Snowscaper may create 100 lbs of ice per character level within medium range. She may now make three separate objects per character level.}

\ability{Wall of Ice (Sp):}{At level 8, the Snowscaper gains Wall of Ice as an at-will spell-like ability.}

\ability{Blizzard (Sp):}{At level 9 the Snowscaper may use a standard action to produce a howling gale of freezing wind carrying shards of ice and snow. It does 1d6 Cold damage per character level, and also does 1d6 Slashing and Piercing damage/three character levels. A Blizzard is a short-range Cone.}

\ability{Mirror Mirror on the Wall (Su):}{All of the Snowscaper�s ice creations come from the Plane of Ice, and, as such, are linked and can be linked. At level 9, a Snowscaper has gained enough mastery to link any two smooth reflective surfaces made from her ice like a window as standard action.

\noindent That is to say, the Snowscaper stands in front of a mirror, chooses one of her other mirrors, wherever it may be, and then may see out the other mirror as if looking through a window.

\noindent Anyone or anything present on the other side can likewise see and communicate through their mirror. This effect lasts until the ice mage dismisses it as a free action (or until they're knocked out or killed or what-have-you). Only two surfaces may be linked as such at a time.}

\ability{Through the Looking Glass (Su):}{At level 9, a Snowscaper using her [i]Mirror Mirror On The Wall[/i] ability may pass through the mirrors as easily as climbing through a window, stepping through an open door, or falling down a hole. Others may also come, as long they form a chain by holding hands and the first person through is the Snowscaper. It is not advisable to let go of the chain when you're halfway through the mirror, because now your body parts will be separated by the distance between the mirrors.}

\ability{Animate Snow (Sp):}{At level 11, the Snowscaper may use Animate Snow, as the spell, at will.}

\ability{Create Tons of Ice (Su):}{At level 12, the Snowscaper can create 1,000 lbs of ice per character level as full-round action. It can be created within long range, but must be created on the ground.}

\ability{Wintersmith (Su):}{At level 13, The Snowscaper may now plunge an area into winter. When she first gets this ability, she may create large amounts of snow and icicles and all that, out to Long Range, for one day per character level. This can have effects like Let it Snow and Ground Freeze. Unlike most of her creations, this is not Never-melt Ice. At character level 15, she may do it as far as she can see. At level 17, the wintery conditions she creates do not melt normally and last until she relents and lets them thaw away, or until she is killed.}

\ability{The Great Blizzard of '52 (Sp):}{The Snowscaper's blizzards are now of legendary proportions. She may create a blizzard which is a Medium-ranged Cone. It inflicts (Character Level + Cha Modifier)d6 in Cold damage, and (1/3 Character Level + Cha Modifier) in Slashing and Piercing damage.}

\chapter{Prestige Classes}

\classname{Arcane Strategist} \label{comm:prestige:arcanestrategist}
\vspace*{-8pt}
\quot{``Did they think they could fool a strategist like me? I have specialised in the use of fire all my life!"}

You plan things out ahead of time. In excruciating detail. And then kill people. With your plans.

\ability{Requirements:}{}
\listprereq
\itemability{Skills:}{Knowledge (Engineering) 8 ranks, Knowledge (Nature) 8 ranks and Knowledge (Tactics) 8 ranks }
\itemability{Spells:}{must be able to cast 3rd level Arcane spells}
\itemability{Special:}{must have followers or an army or a [Leadership] feat.}
\end{list} \vspace*{8pt}

\ability{Hit Die:}{d6}

\ability{Class Skills:}{Whatever you want, since Koumei doesn't believe in cross-class skills.}

\ability{Skills/Level:}{6 + Intelligence Bonus}

\begin{table}[tbh]
\begin{small}
\begin{tabular}{lp{1.9cm}p{0.7cm}p{0.7cm}p{0.7cm}p{6cm}l}
Level  &Base Attack Bonus &Fort Save &Ref Save &Will Save &Special &Spellcasting\\
1st &+0 &+0 &+0 &+2 &Move Earth, Expert Tactician&+1 spellcaster level\\
2nd &+1 &+0 &+0 &+3 &Delay Spell, Mass Spell 1/Day &+1 spellcaster level\\
3rd &+1 &+1 &+1 &+3 &Signal&+1 spellcaster level\\
4th &+2 &+1 &+1 &+4 &Spell Trap, Mass Spell 2/Day &+1 spellcaster level\\
5th &+2 &+1 &+1 &+4 &Retributive Spell &+1 spellcaster level\\
6th &+3 &+2 &+2 &+5 &Spell Beacon, Mass Spell 3/Day &+1 spellcaster level\\
7th &+3 &+2 &+2 &+5 &Eight Trigrams Formation &+1 spellcaster level\\
8th &+4 &+2 &+2 &+6 &Artillery, Mass Spell 4/Day &+1 spellcaster level\\
9th &+4 &+3 &+3 &+6 &Conjure Battlefield &+1 spellcaster level\\
10th &+5 &+3 &+3 &+7 &TACTICAL GENIUS!, Mass Spell 5/Day &+1 spellcaster level\\
\end{tabular}
\end{small}
\end{table}


\smallskip\noindent All of the following are Class Features of the Arcane Strategist prestige class.

\ability{Weapon and Armor Proficiency:}{Arcane Strategists gain no proficiency with any weapon or armor.}

\ability{Spellcasting:}{Every level, the Arcane Strategist casts spells (including gaining any new spell slots and spell knowledge) as if they had also gained a level in an arcane spellcasting class they had previous to gaining that level.}

\ability{Move Earth:}{The Arcane Strategist may cast Move Earth as a spell-like ability a number of times per day equal to her Int modifier.}

\ability{Expert Tactician:}{This feat is gained as a bonus feat.}

\ability{Delay Spell:}{The Arcane Strategist may Delay any spell they cast, except for the spells of their highest level available. A spell may be delayed by up to 3 rounds, chosen at the time of casting. After this delay, the spell goes off, just as if it had been cast only then.}

\ability{Mass Spell:}{Once per day, the Arcane Strategist may cast a Mass version of a spell, using the same spell slot as the original. This takes a full round action if it would normally take less time, or an extra full round action if it would take one round or more. The spell must be one that affects one or more targets, not an area of effect. It now affects one target per caster level. Every 2 levels, another daily use of this is granted.}

\ability{Signal Spell:}{With a full round action, the Arcane Strategist may declare a special strategy. All allies within hearing distance can be blessed with a Contingency for when the Arcane Strategist casts a specific spell, causing them to perform a Standard, Move or Full Round Action (stated at time of preparation) as an Immediate Action. This lasts until activated, the upcoming Dawn/Dusk/Noon/Midnight or the ability is used again (it overwrites current contingencies). \smallskip

An example could be ``When I cast a Mass Haste, everyone make a Full Attack Action against an enemy you can reach." or ``When I Fireball the enemies, everyone Charge them".}

\ability{Spell Trap:}{The Arcane Strategist may cast a spell as a Trap, as long as the spell usually affects an area of effect. The Trap is placed onto a 5' square and lasts until the upcoming Dawn/Dusk/Noon/Midnight, or until triggered, or until the maximum allowed is exceeded, where the oldest are removed first. The maximum allowed is the Arcane Strategist's Intelligence modifier. \smallskip

Enemies can detect the trap with a Search check equal to the save DC of the spell (if it doesn't have one normally, it is 10 + spell level + Int modifier). If someone steps into the square without Disabling it first, the spell goes off, only affecting the one square. All other effects are the same, and the square being trapped has to be adjacent to the Arcane Strategist.}

\ability{Retributive Spell:}{The Arcane Strategist may cast any Single Target spell as a Retributive spell, but may only have one up at a time. She becomes wreathed in magic for 1 minute or until struck in melee combat. If the latter occurs, the spell activates, affecting the attacker as though they were the original target of the spell.}

\ability{Spell Beacon:}{The Arcane Strategist may cast a spell as a Spell Beacon at a range of up to 20' per class level. The square targeted glows with a pillar of light, helping to direct allies to it. Whenever one of her allies touches the square, the spell is activated. This may be any multiple target, single target or area of effect spell, and is usually used to place buffs at vital objectives to help allies hold the positions. Some strategists are not above making the spells harmful, however, effectively setting orbital bombardments on their allies.}

\ability{Eight Trigrams Formation:}{With a Full Round Action, the Arcane Strategist can create a defensive barrier for her allies. Draw an imaginary line between every pair of allies, including the Arcane Strategist. These form Walls of Force for one round, and are filled with Acid Fog that does not affect her or her allies. \smallskip

Additionally, the Arcane Strategist and her allies gain the benefits (but not the drawbacks) of a Stoneskin effect for one round. Any foe who attacks the Arcane Strategist or any of her allies during this time but fails to deal any damage automatically becomes Exhausted for one minute. In this case, allies are only those within 50' of the Arcane Strategist, not those back home in Kansas.}

\ability{Artillery:}{Damaging spells cast by the Arcane Strategist begin to have tell-tale signs of their destructive capacity. When the Arcane Strategist casts Area of Effect spells that allow a save and primarily deal damage, her enemies may elect to dive for cover, becoming Prone and Cowering for 1 round, but automatically passing all saving throws. They can do this after rolling and failing, but doing so extends the Cowering to 2 rounds. While they remain Prone and Cowering, they continue to automatically pass saving throws against such spells cast by her.}

\ability{Conjure Battlefield:}{Once per day, the Arcane Strategist may transform the landscape into a favourable battlefield. One square mile gains the effects of Nightmare Terrain (the Arcane Strategist and her allies ignore this effect), and a trench network covers a third of the battlefield - the third closest the Strategist, providing Full Cover. Additionally, a full fortress set (as per the Warmage "Instant Fortress" ability) is conjured on her side, and spits out one Fireball per round. Finally, the middle third of the area is filled with a Stone Spikes effect. Should a major victory be scored, this ability may be used once more on that day.}

\ability{TACTICAL GENIUS!}{Once per day, the Arcane Strategist may perform a great tactical move, selecting one of two effects:
\begin{itemize}
  	\item{1) She and her allies are instantly Teleported Without Error or Gated to the precise point they wish to be, with a Timestop effect taking place upon arrival.}

	\item{2) A powerful magical ally is summoned, appearing anywhere the Arcane Strategist can see. Typical creatures include:}
		\begin{itemize}
  			\item{a Remorhaz with additional HD}
			\item{a Purple Worm (no sniggering there!) with additional HD}
			\item{a Colossal Adamantine Animated Siege Tower with a Catapult on top}
			\item{a Colossal Monstrous Vermin with additional HD}
		\end{itemize} 
\end{itemize}

The CR may be any amount up to your character level -2. The creature hangs around until one minute has passed, and usually just goes on a rampage, trying to deal as much damage to the enemy forces as possible.

This ability may be used once more in the day if a great victory is achieved.}

\ability{Great Victories:}{A decisive martial victory on the scale of annexing a decently-sized kingdom, or in lieu of conquest, a terror that was plaguing aforementioned decently-sized kingdom and threatened it with destruction/enslavement/more than just the heebiejeebies.

Optionally anything that involves great tactical coordination and represents the culmination of months if not years of planning and maneuvering might also qualify (for instance, courting, seducing and having sex with eight princesses at once, or convincing the entire planet to play a game of The Floor Is Lava (this can of course be really simple by actually turning the floor into lava)).}
\classname{Big Nob}  \label{comm:prestige:bignob}
\vspace*{-8pt}
\quot{``MY NAME IS HUGE!"}


In some cultures, being the biggest seriously makes you the best. You are viewed as the leader, and if someone smaller disagrees, they have to kill you. Even then, everyone is cheering for you, because you're bigger.

\ability{Prerequisites:}{}
\listprereq
\itemability{BAB:}{+1}
\itemability{Size:}{Large or larger}
\itemability{Feats:}{Leadership}
\itemability{Special:}{Must lead an army, or have led an army, in which you are/were the biggest.}
\end{list}\vspace*{8pt}

\ability{Hit Die:}{d12}

\ability{Class Skills:}{The Big Nob's class skills (and the key ability for each skill) have not yet been written}

\ability{Skills/Level:}{4 + Intelligence Bonus}

\begin{table}[tbh]
\begin{small}
\begin{tabular}{lp{1.9cm}p{0.7cm}p{0.7cm}p{0.7cm}p{10cm}l}
Level&Base Attack Bonus&Fort Save&Ref Save&Will Save&Special\\
1&+1&+2&+0&+0&Command Rating +1, Look Out Sir!\\
2&+2&+3&+0&+0&+1 Natural Armour, Intimidating Battlecry\\
3&+3&+3&+1&+1&Big Morale +1, Warstride\\
4&+4&+4&+1&+1&+1 Natural Armour, Minion See Minion Do\\
5&+5&+4&+1&+1&Command Rating +1, WAAAAAGH!\\
6&+6&+5&+2&+2&+1 Natural Armour, Get Over Here!\\
7&+7&+5&+2&+2&Big Morale +2, More WAAAAAGH!\\
8&+8&+6&+2&+2&+1 Natural Armour, Who's Scarier?\\
9&+9&+6&+3&+3&Command Rating +1, Worthy Sacrifice, My Name Is Huge\\
10&+10&+6&+3&+3&+1 Natural Armour, The Most WAAAAAGH!\\
\end{tabular}
\end{small}
\end{table}

\smallskip\noindent All of the following are Class Features of the Big Nob class.

\ability{Weapon and Armor Proficiency:}{You gain proficiency with all armour and shields.}

\ability{Look Out, Sir!:}{As long as you have at least one minion, cohort or follower within your reach, you gain the benefits of Improved Evasion. However, if you elect to use it, one minion, cohort or follower within reach is killed. This does not result in a penalty or reduction to your Leadership score.}

\ability{Intimidating Battlecry:}{As a Swift action, when charging you can let out a battlecry. This allows you to make a demoralise attempt against all foes within 60' who can hear you. You receive a +2 bonus for every size category above Medium. Additionally, all allies within 60' gain a +20 ' bonus to their speed for one turn.}

\ability{Warstride:}{You can ignore all difficult terrain, and are in no way impeded by the corpses of your allies and/or enemies. You never slip on blood, or are blinded by sprays of blood, should such things ever be relevant (see: Kobold feats).}

\ability{Minion See, Minion Do:}{Whenever you attack a foe, all allies who could reach the foe by making a charge attack may do so as an Immediate action.}

\ability{WAAAAAGH!:}{You have such legend surrounding you, and your minions have so much faith in you, that you grow one size category, gaining all relevant bonuses and penalties.}

\ability{Get Over Here!:}{When you bellow at a foe, issuing a challenge, you scare them into approaching you so you can hit them. As a move action, you may issue this challenge to a foe who can see and hear you. If they fail a Will save (10 + \half HD + Cha) then they are Shakened and have to approach you (spending at least one move action getting nearer. They can totally charge you to do this, though). If they pass, they are not subject to the fear effect and can elect to pass on your challenge. Doing this, however, gives all of their underlings a -2 morale penalty for 1 minute because their leader is a chickenshit.}

\ability{More WAAAAAGH!:}{Your Leadership rating is unable to be lowered by anything, even if you personally elect to slaughter half your followers. Additionally, every follower within 30' may spend a standard action praising you, empowering your blows with the power of WAAAAAGH! This grants you +1d6 to all melee damage per simpering minion on your next turn.}

\ability{Who's Scarier?:}{You are immune to fear, as you are the scariest thing you know of. Your minions are also immune to fear from any source other than you.}

\ability{Worthy Sacrifice:}{Any time you take damage, you may elect for a minion, follower or cohort within reach to take this damage for you. This only works once per round, however. If ever you are rendered prone or shoved to another square, you can sacrifice a minion or follower within reach to negate this effect, squishing them flat.}

\ability{My Name Is Huge:}{You may elect to now have a very large, impressive name and title that all feel obliged to say in full. Any ability that could involve speaking your name, such as a Knight's challenge, any Truenaming effect, or similar, takes longer to perform, along this chart:
Less than Swift $\Rightarrow$ Swift $\Rightarrow$ Move $\Rightarrow$ Standard $\Rightarrow$ Full $\Rightarrow$ Minute $\Rightarrow$ *2}

\ability{The Most WAAAAAGH!:}{You gain another size category, along with all relevant bonuses and penalties. }
\classname{Crusader of the Elemental Forces} \label{comm:prestige:crusaderelemental}
\vspace*{-8pt}
\quot{``GOOOOOO, PLANET!''}

You have devoted your Knightly pursuits to the study of how to kill people with fire. And water. And air. And sometimes, for something different, with earth.

\ability{Requirements:}{}
\listprereq
\itemability{BAB:}{+10}
\itemability{Skills:}{Knowledge (the Planes) 13 ranks, Knowledge (Nature) 13 ranks}
\itemability{Special:}{Must have some form of attack that deals Fire, Electricity, Cold or Acid damage. Elemental Knights qualify by dint of existing.}
\end{list}\vspace*{8pt}

\ability{Hit Die:}{d10}

\ability{Class Skills:}{Whatever you want, since Koumei doesn't believe in cross-class skills.}

\ability{Skill Points at Each Level:}{4 + Int modifier.}

\begin{table}[tbh]
\begin{small}
\begin{tabular}{lp{1.9cm}p{0.7cm}p{0.7cm}p{0.7cm}l}
Level  &Base Attack  Bonus &Fort Save &Ref Save &Will Save &Special\\
1st &+1 &+2 &+2 &+2 &Elemental Resistances, Elemental Strike\\
2nd &+2 &+3 &+3 &+3 &Heart of Water, Drench, Elemental Mount\\
3rd &+3 &+3 &+3 &+3 &Heart of Earth, Earthen Grasp, Elemental Rage\\
4th &+4 &+4 &+4 &+4 &Heart of Air, Whirlwind, Storm of Elemental Fury\\
5th &+5 &+4 &+4 &+4 &Heart of Fire, Fiery Skin, Storm of Vengeance\\
\end{tabular}
\end{small}
\end{table}

\smallskip\noindent All of the following are Class Features of the Crusader of the Elemental Forces prestige class.

\ability{Weapon and Armor Proficiency:}{Crusader of the Elemental Forces gain no proficiency with any weapon or armor.}

\ability{Elemental Resistances:}{The Crusader of the Elemental Forces receives Energy Resistance equal to 5 times their class level against Fire, Acid, Cold and Electricity. They are also unaffected by reasonably strong wind, fog, rain, muddy ground, quicksand, sandstorms, random lightning strikes, particularly hot weather or particularly cold weather.}

\ability{Elemental Strike:}{When dealing additional damage to a Designated Opponent, the Crusader may, on a successful hit, elect to either Soak the target (making them count as Entangled until dried), set them on fire, knock them prone with earthly might or cause the wind to carry them into the air, moving them up to 50' away. If multiple hits are made, multiple effects may be caused in the same round.}

\ability{Heart of Water:}{The Crusader has a permanent Heart of Water effect (Swim speed equal to Land Speed, Breathe Underwater, +5 enhancement bonus to Escape Artist). At any rime, the Crusader can elect to be affected by Freedom of Movement instead. This is a Swift Action to decide, and lasts until they change it back with another Swift Action. This is a Supernatural Ability.}

\ability{Drench:}{The Crusader gains the ability to drench others (and flames), just like a Water Elemental. This is a Supernatural ability.}

\ability{Elemental Mount:}{The Crusader gains a Mount to ride. This mount is loyal, and if it dies, another appears a day later. It can be any Elemental with a CR at least 3 less than the Crusader's character level. It can be ridden even if logic says otherwise, and causes no harm at all to the Crusader.}

\ability{Heart of Earth:}{The Crusader has a permanent Heart of Earth effect (+8 bonus to resist Trip/Overrun/Bull Rush and increased maximum HP of 30). Alternatively, with a Swift Action they may lose these effects and instead gain a Stoneskin effect that lasts until they use a Swift Action to change back. As long as both original Heart effects are active (not the alternate spell effects), the Crusader gains Light Fortification. This is a Supernatural Ability.}

\ability{Earthen Grasp:}{The Crusader can make the ground grab people who are standing on it. This is a Supernatural Ability that requires a Standard Action. She may use her own BAB and Strength to make a grapple attempt at a target within 50', but counts as a Colossal creature. If the foe is grappled, the Crusader can Concentrate to sustain this effect, constricting them and dealing 10d6 Crushing damage every round until the foe escapes.}

\ability{Elemental Rage:}{The Elemental Strike ability is enhanced. If the Crusader sets an enemy on fire, they burn brilliantly, taking 5d6 Fire damage per round and the DC to put the flames out becomes 10 + half the Crusader's Hit Dice + her Charisma modifier. If she soaks them, water floods their lungs and they drown enough to become Exhausted. If she knocks them prone they are Stunned for 1 round, and if she knocks them into the air, they are held in place as though by a Telekinetic Sphere, air whirling all about them.}

\ability{Heart of Air:}{The Crusader of the Elements gains a permanent Heart of Air effect (+10 enhancement bonus to Jump checks, 30' Flight (Average)). She may at any time unleash a Gust of Wind as a Swift action, but doing so deactivates the Heart of Air until her next turn. This is a Supernatural ability.}

\ability{Whirlwind:}{The Crusader may, with a Standard Action, transform into a tornado of sorts. Her ability scores and attributes all remain the same, however she has a constant 50\% Concealment and gains a Slam attack that deals 1d8+Str*1.5 Bludgeoning for a Medium Creature and buffets the target 20' away. She may also trap enemies inside the vortex like an Air Elemental. However, she does lose the ability to cast any spells and loses the benefits of Heart of Earth, Fire and Water. Changing back just requires a Swift Action.}

\ability{Storm of Elemental Fury:}{The Crusader may cast Storm of Elemental Fury three times per day as a Spell-like Ability, as long as she designated an opponent who did not strike her in the last turn. The opponent must be in the area of effect. The DC is 10 + half her hit dice + her Charisma modifier.}

\ability{Heart of Fire:}{The Crusader gains a permanent Heart of Fire effect (Fire Immunity, +10' enhancement bonus to speed). Alternatively, with a Swift Action she may swap it out for a Fire Shield, and change it back with another Swift action. If all four Hearts are active, she becomes immune to critical hits.}

\ability{Fiery Skin:}{Anyone who Grapples with the Crusader or strikes her with a melee weapon catches fire instantly.}

\ability{Storm of Vengeance:}{Once per day, the Crusader may cast Storm of Vengeance as a Supernatural Ability. It will last until the next dawn, dusk, noon or midnight (whichever comes first). Seriously.}
Dragoon
� 	"You can fly?!?" 	�
�A blue dragon simultaneously discovering the existance and capabilities of Dragoons

The Dragoons are a group of warriors dedicated to slaying large monsters. Of course, they are most famous for taking on dragons, to the point that many of them have a definite dragon motif to their armor. Armed with a spear to be able to reach vital points on barn-sized creatures with hides like rock, Dragoons leap fearlessly into battle. And I do mean leap.

It is hypothesized that the amazing physical abilities shown by a Dragoon are the result of natural talent for magic combined it with a strict training regimen, and channeled towards something other than spells.
[edit] Becoming a Dragoon

Generally, Dragoons are recruited and trained as a sort of public service; even Evil civilizations have a problem with big monsters, so Dragoons can hail from those places, too.


Entry Requirements Skills: 	Jump 8 Ranks
Feats: 	Giant Slayer for those using Races of War (DnD Other) rules, Skill Focus: Jump for those who aren't.
Special: 	Heavy armor proficiency, and proficiency with a spear, lance, or similar weapon.

Hit Die: d10

Level 	Base
Attack Bonus 	Saving Throws 	Special
Fort 	Ref 	Will
1st 	+1 	+2 	+2 	+0 	Evasion, Jump Attack, Jump Good
2nd 	+2 	+3 	+3 	+0 	Spear Mastery
3rd 	+3 	+3 	+3 	+1 	Dragonbreath
4th 	+4 	+4 	+4 	+1 	Killer Jump
5th 	+5 	+4 	+4 	+1 	Dragonbreath
6th 	+6 	+5 	+5 	+2 	Giantkiller
7th 	+7 	+5 	+5 	+2 	Dragonbreath
8th 	+8 	+6 	+6 	+2 	Dragonheart, Improved Evasion
9th 	+9 	+6 	+6 	+3 	Dragonbreath
10th 	+10 	+7 	+7 	+3 	Double Jump

Class Skills (4 + Int modifier per level)
Balance, Climb, Intimidate, Jump, Knowledge (all), Listen, Spot, Survival, Swim, Tumble

Overtrained: While not exactly a Code of Conduct, Dragoons do have a few things drilled into their head by their teacher: Defend the People From Monsters, Especially Big Monsters.

The majority of their training is concerned with finding ways to stab monsters in the face when it seems improbable that the Dragoon would be able to reach that high. Naturally, a Dragoon first reaction upon seeing a monster chewing on a few cows in the street and hearing people scream, would be to heft his spear and then endeavor to stab the monster in the face. I'm not saying they *must* do it, but it will be their first, second, and third thought.

Jump Attack (Ex): At first level Dragoons learn a specialized mode of attack�death from above. As long as one falls at least thirty feet and attacks at the end of the fall, he does double damage on a successful hit; if he's wielding a spear or a lance of some kind, he does triple damage and ignores either armor or natural armor (whichever is higher). Should he score a critical hit with a Jump Attack, damage is caculated by adding 1 onto the weapon's critical multiplier, or 2 if using a spear or a lance or similar weapon.

Jump Good (Su): At 1st level, A Dragoon's Jump DCs are halved, and the formula for figuring vertical jump height is the same as for horizontal jump height (one foot per point on the check). Dragoons jump extraordinarily quickly, letting them complete an entire jump, no matter how far it is, as a move action, and their Armor Check penalty does not apply to their Jump checks; they are always considered to benefit from a running start on Jump checks. They also gain immunity to falling damage as long as they�re conscious and always land on their feet.

Evasion (Ex): Really big creatures can often unleash really big blasts of fire or acid or some other unpleasantness. For this reason, Dragoons are trained in the arts of evasion. Should a Dragoon already have Evasion, this stacks to Improved Evasion.

Spear Mastery (Ex): Dragoons like spears. They�re simple to use, and the reach makes it easier to damage vital points on their chosen foes�big, big monsters. At 2nd level, A Dragoon may use a reach weapon as if it were not a reach weapon with no penalty and the critical threat range with spears and lances and other piercing polearms is increased by 1 (this stacks with anything that doubles the critical threat range, but is added after the doubling).

Dragonbreath (Su): Dragoons have a career fighting monsters who tend to have breath weapons or some ability to turn an entire area into pain. Naturally, there is a desire to get even, and at 3rd level it's become strong enough to take over more of the Dragoon's natural magical talent and grant him a breath weapon known as Dragonbreath.

At levels 3, 5, 7, and 9, select an energy type out of fire, electricity, cold, acid, and sonic. The Dragoon can use his Dragonbreath with any of his repertoire of energy types. Dragonbreath is a 15-foot cone which inflicts 1d8 per character level of the energy type (selected when used), Reflex save DC 10 + 1/2 Character level + Con mod for half. After Dragonbreath is used, it may not be used again for 1d4 rounds.

Killer Jump (Su): If a Dragoon has learned anything by 4th level, it�s that there is a huge amount of energy to be gained from falling.

Killer Jump is full-round action. It begins when the Dragoon makes a Jump check to land on a target, and with this particular check he may travel travel up to 10 feet (vertically and/or horizontally) for every two points on the check.

When he lands on the target, he makes a touch attack roll. It is like a normal Jump Attack (does double damage on a successful hit (triple if using a spear)) but an additional d6 of damage for every 10 feet he fell, and is considered a critical threat and resolved as such; the bonus falling damage is not multiplied. He may do this every 1d4 rounds.

Giantkiller (Ex): A 6th-level Dragoon has some experience with killing big creatures, and has discovered a very interesting fact�big creatures have big vital areas. Of course, having a long weapon helps in exploiting these areas.

A Dragoon gets a doubled critical range for spear-like weapons, as long as the target is Large size or larger. He also gets a +2 on his attack rolls to confirm critical hits for each size the creature is above medium.

Improved Evasion (Ex): As the Rogue. If the Dragoon already has Improved Evasion, he gets a bonus feat for which he meets the prerequisites.

Dragonheart (Su): A Dragoon who hits his stride will enter a state of mind. It's a good place for him, and a bad place for his enemies.

If a Dragoon succeeds on a saving throw or confirms a natural critical hit (the auto-threat from Killer Jump does not count, unless the attack roll was naturally in the threat range), Dragonheart activates as a free action. The Dragoon gains Fast Healing equal to 1/2 his character level + his Con modifier, and becomes immune to mind-affecting effects, ability damage or drain, and morale and fear effects. Dragonheart lasts for a number of rounds equal to his class level.

Double Jump (Su): The Dragoon is now a master of jumping on things to kill them. When he connects with a Jump Attack or a Killer Jump, he may make an swift action to perform another Jump Attack on the same target.
[edit] Ex-Dragoon

Generally, Dragoons only retire when old age makes fighting giant monsters risky. Most of them, though, default to being Dragoons and, as such, seek out, recruit, and train suitable people to replace themselves in the ranks.
\classname{Maelstrom of Fiery Ki} \label{comm:prestige:maelstrom}
\vspace*{-8pt}
\quot{``It was a mistake to make me unhappy.... SHI SHI HOUKUDAN!"}

You get very tense. It helps you fight better. Sometimes, you get so tense you explode and everyone dies.

\ability{Prerequisites:}{}
\listprereq
	\itemability{BAB:}{+7}
	\itemability{Special:}{Must posses a Super Gauge.}
\end{list}\vspace*{8pt}

\ability{Hit Die:}{d10}

\ability{Class Skills:}{Whatever you want, since Koumei doesn't believe in cross-class skills.}

\ability{Skills/Level:}{4 + Intelligence Bonus}

\begin{table}[tbh]
\begin{small}
\begin{tabular}{lp{1.9cm}p{0.7cm}p{0.7cm}p{0.7cm}p{6cm}l}
Level&Base Attack Bonus&Fort Save&Ref Save&Will Save&Special\\
1&+1&+2&+2&+2&Trouble Brewing, Rageblade\\
2&+2&+3&+3&+3&Ki Burst, Taste Your Own Blood\\
3&+3&+3&+3&+3&Tempest in a Tea Cup, Shadowrun\\
4&+4&+4&+4&+4&Flow of Violence, Overflowing Cup of Ki\\
5&+5&+4&+4&+4&Deadly Finishing Move, Ki Eruption\\
\end{tabular}
\end{small}
\end{table}

\smallskip\noindent All of the following are Class Features of the Maelstrom of Fiery Ki prestige class.

\ability{Weapon and Armor Proficiency:}{The Maelstrom of Fiery Ki gains no proficiency with armor or weapons.}

\ability{Trouble Brewing:}{The Maelstrom of Fiery Ki builds up a lot of Tension, the kind that can only be eased off by applying violence to other people. In the face. Whenever struck, or striking an enemy, they gain 2 Tension instead of 1.}

\ability{Rageblade:}{When striking the subject of their Tension, the Maelstrom adds their current Tension to melee damage rolls. This is multiplied by critical hits, Leap Attack and so on.}

\ability{Ki Burst:}{As a Standard Action, the Maelstrom may unleash a wave of burning Ki energy at her foes. This has a radius of 5' per 2 points of Tension spent, and everyone in the area takes 1d6 Fire damage per Tension spent, up to a maximum of the Maelstrom's hit dice. Only half of this may be resisted/negated. Everyone in the area is subject to a Ref save for half damage (DC 10 + 1/2 the Maelstrom's HD + her Wis mod). Spending an additional 10 Tension transforms this into a Swift action.}

\ability{Taste Your Own Blood:}{The Maelstrom practically enjoys getting smacked around now. When suffering a critical hit, a sneak attack, any [Pain] effect, or any hit that reduces her to 25\% or less (including multiple little hits when already in the danger zone), she gains another 5 Tension.}

\ability{Tempest in a Tea Cup:}{The Maelstrom of Fiery Ki may now ignore attacks from others, focusing exclusively on the one building her Super Gauge. By doing so, she gains no Tension against anyone other than the designated foe, but as a result does not lose her Tension just because someone else slaps her up. If she turns and attacks them, the Tension is lost as normal. She can opt not to use this ability at any time. \smallskip

One side effect of this ability is she no longer loses her Tension after attacking a different target - the Tension is lost when the decision is made, thus preventing her from possibly being able to charge up against one person and release against another.}

\ability{Shadowrun:}{The Maelstrom can step onto the Ethereal Plane, and thus may walk on water or, for that matter, thin air as long as she ends her turn standing on something solid. It can be a flimsy tree branch or a bamboo pole, mind you, as long as it is a solid object. Furthermore, she may step through Walls of Force, treating them merely as 10' movement per wall, and can add her Intelligence modifier to her Initiative. \smallskip

If she already possesses this ability or later on gets it from continuing their life as a Sohei, she instead gains Poetry in Motion. If she manages to get that as well, then she gets a round of applause.}

\ability{Flow of Violence:}{Pain is a great teacher: it teaches us not to get hurt, usually. With a Swift Action, the Maelstrom may spend 10 Tension to gain the benefits of a single [Combat] feat for a number of rounds equal to her Wisdom modifier. Only one feat may be gained at a time in this manner.}

\ability{Overflowing Cup of Ki:}{The Maelstrom is such a tornado of fury that she gains twice as much Tension from all sources (10 from critical hits, sneak attacks, [Pain] effects and dangerously low health, 4 from hitting or getting hit).}

\ability{Deadly Finishing Move:}{By spending 20 Tension as a Standard Action, the Maelstrom may unleash a killer Destroyer move. This attack deals an additional 2d6 damage per Hit Die of the Maelstrom, and bypasses all Damage Reduction (including /-) and Regeneration. \smallskip

It also deals Strength Damage equal to the Maelstrom's Wisdom modifier, and the target must make either a Fortitude save or a Will save (Maelstrom's choice) or be slain instantly and completely destroyed. Not even ashes remain, and the soul is transported to the centre of the planet, on another plane (probably Hell). \smallskip

It should be mandated that the player name this move and either shout it out before unleashing it, or quietly state the name after the effects take place. And adopts a cool pose, with the name written in Kanji on pieces of paper and held up next to their head by another player. \smallskip

The target, if slain, does gain enough time to compose and recite a poem, however they cannot use this time for anything else. It's a poem and death, or death without sauce.}

\ability{Ki Eruption:}{At the cost of 20 additional Tension, Ki Burst may be used as an Immediate Action, interrupting enemy actions. If an interrupted enemy takes damage from this, their action automatically fails.}


\chapter{Feats/Spells/Spheres}

\section{Feats}

%\newcommand{\babfeat}[7]{\noindent\textbf{#1} \\ \emph{#2} \\ \textbf{Benefit:}#3 \\ \textbf{+1:}#4  \\ \textbf{+6:}#5 \\ \textbf{+11:}#6 \\ \textbf{+16:}#7 \bigskip }
%\newcommand{\skillfeat}[7]{\noindent\textbf{#1} \\ \emph{#2} \\ \textbf{Benefit:} #3 \\ \textbf{4:} #4  \\ \textbf{9:} #5 \\ \textbf{14:} #6 \\ \textbf{19:} #7 \bigskip }

\section{The Failure of Feats}
\vspace*{-10pt}
\quot{``How about instead of being able to travel anywhere in the multiverse, transform yourself into anything you can think of, stop time, and slay everyone you can see, we just give a nice +1 to hit with your secondary weapon? Deal?''}

\noindent\desc{Feats were an interesting idea when they were ported to 3rd edition D\&D. But let's face it; they don't go nearly far enough. Feats were made extremely conservative in their effects on the game because the authors didn't want to offend people with too radical a change. Well, now we've had third edition for 6 years, and we're offended. Feats are an interesting and tangible way to get unique abilities onto a character, but they have fallen prey to two key fallacies that has ended up turning the entire concept to ashes in our mouths. The first is the idea that if you think of something kind of cool for a character to do, you should make it a feat. That sounds compelling, but you only get 7 feats in your whole life. If you have to spend a feat for every cool thing you ever do, you're not going to do very many cool things in the approximately 260 encounters you'll have on your way from 1st to 20th level. The second is the idea that a feat should be equivalent to a cantrip or two. This one is even less excusable, and just makes us cry. A +1 bonus is something that you seriously might forget that you even have. Having one more +1 bonus doesn't make your character unique, it makes you a sucker for spending one of the half dozen feats you'll ever see on a bonus the other players won't even mention when discussing your character.}

\noindent\desc{We all understand this problem, what do we do about it? Well, for starters, Feats have to do more things. Many characters are 5th level or so and they only have 2 feats. Those feats should describe their character in a much more salient way than ``I'm no worse shooting into melee than I am shooting at people with cover that isn't my friends.'' This was begun with the tactical feats, but it didn't go far enough. It's not enough to add additional feats that do something halfway interesting for high level characters to have -- we actually have to replace the stupid one dimensional feats in the PHB with feats that rational people would care about in any way. Spending a single feat should be enough to make you a ``sniper character'' because for a substantial portion of your life you only get one feat. Secondly, we have to clear away feats that don't provide numeric bonuses large enough to care about. The minimum bonus you'll ever notice is +3, because that's actually larger than the difference between having rolled well and having rolled poorly on your starting stats. Numeric bonuses smaller than that are actually insulting and need to be removed from the feats altogether. 3.5 Skill Focus was a nice start, but that's all it was -- a start.}

\noindent{Furthermore, the fundamental structure of feats has been a disaster. The system of prerequisites often ensures that characters won't get an ability before it would be level appropriate for them to do so, but actually does nothing to ensure that such characters are in fact getting level appropriate abilities. Indeed, if a 12th level character decides that they want to pursue a career in shooting people in the face, they have to start all over gaining an ability that is supposed to be level appropriate for a 1st level character. Meanwhile, when a wizard of 12th level decides to pursue some new direction in spellcasting -- he learns a new 6th level spell right off -- and gets an ability that's level appropriate for a 12th level character.}

\subsection{Exploits}

\noindent\desc{Getting proficiency with a weapon isn't worth a feat. They hand that crap out with your character class for free. Seriously, even exotic weapon proficiencies aren't a big deal. Therefore, we're instituting Exploits as something that can be acquired in-game. These are for any of the binary abilities that simply don't have a massive impact on your character's performance at any level.}

\noindent\desc{If you have Martial Weapon Proficiency, it's really unreasonable for it to be that hard to learn how to use a new weapon, whether it's exotic or not. If you spend a week training with a weapon, you can make an Int check (DC 10) to simply gain the Exploit of Exotic Weapon Proficiency. And no, you can't take 10 on that.}

\noindent{If you don't have Martial Weapon Proficiency and you want to use a new weapon, that's touchier. But if you have a weapon for an entire level, you should just gain proficiency in it when you gain your next level whatever level you happen to select.}

\subsection{The New Feat System}

\noindent{So where are we going with this? First of all, feat chains are gone. That seemed like a good idea, but it wasn't. Secondly, the vast majority of feats don't have prerequisites at all, they scale. A [Combat] feat scales to your Base Attack Bonus, a [Skill] feat scales to your ranks in a skill, and a [Metamagic] feat scales to the highest level spell you can cast. And that's because those are the only things in the game that actually have anything to do with the level your character is in any way that we feel good about.}

\section{The New Combat Ready Feats} \label{feats:combat}

\begin{multicols}{2}
\hypertarget{feat:blindfighting}{}\babfeat{Blind Fighting [Combat]}{
You don't have to see to kill.}{
You may reroll your miss chances caused by concealment.}{
While in darkness, you may move your normal speed without difficulty.}{
You have Blindsense out to 60', this allows you to know the location of all creatures within 60'.}{
You have Tremorsense out to 120', this allows you to ``see'' anything within 120' that is touching the earth.}{
You cannot be caught flat footed.}

\hypertarget{feat:blitz}{}\babfeat{Blitz [Combat]}{
You go all out and try to achieve goals in a proactive manner.}{
While charging, you may opt to lose your Dexterity Bonus to AC for one round, but inflicting an extra d6 of damage if you hit.}{
You may go all out when attacking, adding your Base Attack Bonus to your damage, but provoking an Attack of Opportunity.}{
Bonus attacks made in a Full Attack for having a high BAB are made with a -2 penalty instead of a -5 penalty.}{
Every time you inflict damage upon an opponent with your melee attacks, you may immediately use an Intimidate attempt against that opponent as a bonus action.}{
You may make a Full Attack action as a Standard Action.}

\hypertarget{feat:combatlooting}{}\babfeat{Combat Looting [Combat]}{
You can put things into your pants in the middle of combat.}{
You may sheathe or store an object as a free action.}{
You get a +3 bonus to  \hyperlink{combat:disarm}{Disarm} attempts. Picking up objects off the ground does not provoke an attack of opportunity.}{
As a Swift action, you may take a ring, amulet/necklace, headband, bracer, or belt from an opponent you have successfully \hyperlink{combat:grapple}{grappled}. You may pick up an item off the ground in the middle of a move action.}{
If you are grappling with an opponent, you may activate or deactivate their magic items with a successful Use Magic Device check. You may make Appraise checks as a free action.}{
You can take 10 on Use Magic Device and Sleight of Hand checks.}

\vspace{100pt}
\hypertarget{feat:combatschool}{}\babfeat{Combat School [Combat]}{
You are a member of a completely arbitrary fighting school that has a number of recognizable signature fighting moves.}{
First, name your fighting style (such as ``Hammer and Anvil Technique'' or ``Crescent Moon Style'', or ``Way of the Lightning Mace''). This fighting style only works with a small list of melee weapons that you have to describe the connectedness to the DM in a half-way believable way. Now, whenever you are using that technique in melee combat, you gain a +2 bonus on attack rolls.}{
Your immersion in your technique gives you great martial prowess, you gain a +2 to damage rolls in melee combat.}{
When you strike your opponent with the signature moves of your fighting school in melee, they must make a Fortitude Save (DC 10 + 1/2 your level + your Strength bonus) or become dazed for one round.}{
You may take 10 on attack rolls while using your special techniques. The DC to disarm you of a school-appropriate weapon is increased by 4.}{
You may add +5 to-hit on any one attack you make after the first each turn. If you hit an opponent twice in one round, all further attacks this round against that opponent are made with The Edge.}

\hypertarget{feat:command}{}\babfeat{Command [Combat] [Leadership]}{
You lead tiny men.}{
You have a Command Rating equal to your Base Attack Bonus divided by five (round up).}{
You can muster a group of followers. Your leadership score is your Base Attack Bonus plus your Charisma Modifier.}{
You are able to delegate command to a loyal cohort. A cohort is an intelligent and loyal creature with a CR at least 2 less than your character level. Cohorts gain levels when you do.}{
With a Swift Action you may rally troops, allowing all allies within medium range of yourself to reroll their saves vs. Fear and gain a +2 Morale Bonus to attack and damage rolls for 1 minute. This is a language-dependent ability that may be used an unlimited number of times.}{
Your allies gain a +2 morale bonus to all saving throws if they can see you and you are within medium range.}


\hypertarget{feat:dangersense}{}\babfeat{Danger Sense [Combat]}{
Maybe Spiders tell you what's up. You certainly react to danger with uncanny effectiveness.}{
You get a +3 bonus on Initiative checks.}{
For the purpose of Search, Spot, and Listen, you are always considered to be ``actively searching''. You also get Uncanny Dodge.}{
You may take 10 on Listen, Spot, and Search checks.}{
You may make a Sense Motive check (opposed by your opponent's Bluff check) immediately whenever any creature approaches within 60' of you with harmful intent. If you succeed, you know the location of the creature even if you cannot see it.}{
You are never surprised and always act on the first round of any combat.}


\hypertarget{feat:elusivetarget}{}\babfeat{Elusive Target [Combat]}{
You are very hard to hit when you want to be.}{
You gain a +2 Dodge bonus to AC.}{
Your opponents do not gain flanking or higher ground bonuses against you.}{
Your opponents do not inflict extra damage from the \hyperlink{combat:powerattack}{Power Attack} option.}{
Diverting Defense -- As an immediate action, you may redirect an attack against you to any creature in your threatened range, friend or foe. You may not redirect an attack to the creature making the attack.}{
As an immediate action, you may make an attack that would normally hit you miss instead.}


\hypertarget{feat:experttactician}{}\babfeat{Expert Tactician [Combat]}{
You benefit your allies so good they remember you long time.}{
You gain a +4 bonus when flanking instead of the normal +2 bonus. Your allies who flank with you gain the same advantage.}{
You may \hyperlink{combat:feint}{Feint} as an Immediate action.}{
As a move action, you may make any 5' square adjacent to yourself into difficult ground.}{
For determining flanking with your allies, you may count your location as being 5' in any direction from your real location.}{
You ignore Cover bonuses less than full cover.}


\hypertarget{feat:ghosthunter}{}\babfeat{Ghost Hunter [Combat]}{
You smack around those folks in the spirit world.}{
Your attacks have a 50\% chance of striking incorporeal opponents even if they are not magical.}{
You can hear incorporeal and ethereal creatures as if they lacked those traits (note that shadows and the like rarely bother to actively move silently).}{
You can see invisible and ethereal creatures as if they lacked those traits.}{
Your attacks count as if you had the Ghost Touch property on your weapons.}{
Any Armor or shield you use benefits from the Ghost Touch quality.}

\vspace{100pt}
\hypertarget{feat:giantslayer}{}\babfeat{Giant Slayer [Combat]}{
Everyone has a specialty. Yours is miraculously finding ways to stab creatures in the face when it seems improbable that you would be able to reach that high.}{
When you perform a \hyperlink{combat:grabon}{grab on} Grapple maneuver, you do not provoke an attack of opportunity.}{
You gain a +4 Dodge bonus to your AC and Reflex Saves against attacks from any creature with a longer natural reach than your own.}{
You have The Edge against any creature you attack that is larger than you. Also, an opponent using the Improved Grab ability on you provokes an attack of opportunity from you. You may take this attack even if you do not threaten a square occupied by your opponent.}{
When you attempt to trip an opponent, you may choose whether your opponent resists with Strength or Dexterity.}{
When involved in an opposed bull rush, grapple, or trip check as the attacker or defender, you may negate the size modifier of both participants. You may not choose to negate the size modifier of only one character.}


\hypertarget{feat:greatfortitude}{}\babfeat{Great Fortitude [Combat]}{
You are so tough. Your belly is like a prism.}{
You gain a +3 bonus to your Fortitude Saves.}{
You die at -20 instead of -10.}{
You gain 1 hit point per level.}{
You gain DR of 5/-.}{
You are immune to the fatigued and exhausted conditions. If you are already immune to these conditions, you gain 1 hit point per level for each condition you were already immune to.}


\hypertarget{feat:hordebreaker}{}\babfeat{Horde Breaker[Combat]}{
You kill really large numbers of people.}{
You gain a number of extra attacks of opportunity each round equal to your Dexterity Bonus (if positive).}{
Whenever you drop an opponent with a melee attack, you are entitled to a bonus ``cleave'' attack against another opponent you threaten. You may not take a 5' step or otherwise move before taking this bonus attack. This Cleave attack is considered an attack of opportunity.}{
You may take a bonus 5' step every time you are entitled to a Cleave attack, which you may take either before or after the attack.}{
You may generate an aura of fear on any opponents within 10' of yourself whenever you drop an opponent in melee. The save DC is 10 + the Hit Dice of the dropped creature.}{
Opponents you have the Edge against provoke an attack of opportunity from you by moving into your threatened area or attacking you.}

\vspace{30pt}
\hypertarget{feat:hunter}{}\babfeat{Hunter [Combat]}{
You can move around and shoot things with surprising effectiveness.}{
The penalties for using a ranged weapon from an unstable platform (such as a ship or a moving horse) are halved.}{
Shot on the Run -- you may take a standard action to attack with a ranged weapon in the middle of a move action, taking some of your movement before and some of your movement after your attack. That still counts as your standard and move action for the round.}{
You suffer no penalties for firing from unstable ground, a running steed, or any of that.}{
You may take a full round action to take a double move and make a single ranged attack from any point during your movement.}{
You may take a full round action to run a full four times your speed and make a single ranged attack from any point during your movement. You retain your Dexterity modifier to AC while running.}


\hypertarget{feat:insightfulstrike}{}\babfeat{Insightful Strike [Combat]}{
You Hack people down with inherent awesomeness.}{
You may use your Wisdom Modifier in place of your Strength Modifier for your melee attack rolls.}{
Your attacks have The Edge against an opponent who has a lower Wisdom and Dexterity than your own Wisdom, regardless of relative BAB.}{
Your melee attacks have a doubled critical threat range.}{
You make horribly telling blows. The extra critical multiplier of your melee attacks is doubled (x2 becomes x3, x3 becomes x5, and x4 becomes x7).}{
Any Melee attack you make is considered to be made with a magic weapon that has an enhancement bonus equal to your Wisdom Modifier (if positive).}


\hypertarget{feat:ironwill}{}\babfeat{Iron Will [Combat]}{
You are able to grit your teeth and shake off mental influences.}{
You gain a +3 bonus to your Willpower saves.}{
You gain the slippery mind ability of a Rogue.}{
If you are stunned, you are dazed instead.}{
You do not suffer penalties from pain and fear.}{
You are immune to compulsion effects.}

\vspace{300pt}
\hypertarget{feat:juggernaut}{}\babfeat{Juggernaut [Combat]}{
You are an unstoppable Juggernaut.}{
You may be considered one size category larger for the purposes of any size dependant roll you make (such as a Bull Rush, Overrun, or Lift action).}{
You do not provoke an attack of opportunity for entering an opponent's square.}{
You gain a +4 bonus to attack and damage rolls to destroy objects. You may shatter a Force Effect by inflicting 30 damage on it.}{
When you successfully \hyperlink{combat:bullrush}{bullrush} or overrun an opponent, you automatically Trample them, inflicting damage equal to a natural slam attack for a creature of your size.}{
You gain the Rock Throwing ability of any standard Giant with a strength equal to or less than yourself.}


\hypertarget{feat:lightningreflexes}{}\babfeat{Lightning Reflexes [Combat]}{
You are fasty McFastFast. It helps keep you alive.}{
You gain a +3 bonus to your Reflex saves.}{
You gain Evasion, if you already have Evasion, that stacks to Improved Evasion.}{
You may make a Balance Check in place of your Reflex save.}{
You gain a +3 bonus to your Initiative.}{
When you take the Full Defense Action, add your level to your AC.}


\hypertarget{feat:mageslayer}{}\babfeat{Mage Slayer [Combat]}{
You have trained long and hard to kill magic users. Maybe you hate them, maybe you just noticed that most of the really dangerous creatures in the world use magic.}{
You gain Spell Resistance of 5 + Character Level.}{
Damage you inflict is considered ``ongoing damage'' for the purposes of concentration checks made before the beginning of your next round. All your attacks in a round are considered the same source of continuing damage.}{
Creatures cannot cast defensively within your threat range.}{
Your attacks ignore Deflection bonuses to AC.}{
When a creature uses a [Teleportation] effect within medium range of yourself, you may choose to be transported as well. This is not an action.}


\hypertarget{feat:mountedcombat}{}\skillfeat{Mounted Combat [Skill]}{
You are at your best when fighting with an ally that you are sitting on.}{
Ride Ranks:}{
Once per turn, you may attempt to negate an attack that hits your mount by making a Ride skill check with a DC equal to the AC that the attack hit. Attacks that do not require an attack roll cannot be negated in this way.}{
While Mounted, you may take a charge attack at any point along your mount's movement, so long as your mount is moving in a straight line up to the point of your attack.}{
You suffer no penalty to your ride or handle animal skill checks when training or riding unusual mounts such as magical beasts or dragons.}{
You may use your Ride Check in place of your mount's Balance, Jump, Climb, or Reflex Saving Throws.}{
Any time a spell effect would target your mount, you may elect to have it target you instead. Any time a spell effect would target you, you may elect to have it affect your Mount instead.}


\hypertarget{feat:murderousintent}{}\babfeat{Murderous Intent [Combat]}{
You stab people in the face.}{
You may make a \hyperlink{combat:coupdegrace}{Coup de Grace} as a standard action.}{
When you kill an opponent, you gain a +2 Morale Bonus to your attack and damage rolls for 1 minute.}{
Once per round, you may take an attack of opportunity against an opponent who is denied their Dexterity bonus to AC.}{
You may take a \hyperlink{combat:coupdegrace}{Coup de Grace} action against opponents who are stunned.}{
You may take a \hyperlink{combat:coupdegrace}{Coup de Grace} action against opponents who are dazed.}


\hypertarget{feat:phalanxfighter}{}\babfeat{Phalanx Fighter[Combat]}{
You fight well in a group.}{
You may take attacks of opportunity even while flat footed.}{
Any Dodge bonus to AC you gain is also granted to any adjacent allies for as long as you benefit from the bonus and your ally remains adjacent.}{
Charging is an action that provokes an attack of opportunity from you. This attack is considered to be a ``readied attack'' if it matters for purposes like setting against a charge.}{
You may attack with a reach weapon as if it was not a reach weapon. Thus, a medium creature would normally threaten at 5' and 10' with a reach weapon.}{
You may take an Aid Another action once per round as a free action. You provide double normal bonuses from this effect.}

\vspace{300pt}
\hypertarget{feat:pointblankshot}{}\hypertarget{feat:pbshot}{}\babfeat{Point Blank Shot [Combat]}{
You are crazy good using a ranged weapon in close quarters.}{
When you are within 30' of your target, your attacks with a ranged weapon gain a +3 bonus to-hit.}{
You add your base attack bonus to damage with any ranged attack within the first range increment.}{
You do not provoke an attack of opportunity when you make a ranged attack.}{
When armed with a Ranged Weapon, you may make attacks of opportunity against opponents who provoke them within 30' of you. Movement within this area does not provoke an attack of opportunity.}{
With a Full Attack action, you may fire a ranged weapon once at every single opponent within the first range increment of your weapon. You gain no additional attacks for having a high BAB. Make a single attack roll for the entire round, and compare to the armor class of each opponent within range.}


\hypertarget{feat:sniper}{}\babfeat{Sniper [Combat]}{
Your shooting is precise and dangerous.}{
Your range increments are 50\% longer than they would ordinarily be. Any benefit of being within 30' of an opponent is retained out to 60'.}{
Precise Shot -- You do not suffer a -4 penalty when firing a ranged weapon into melee and never hit an unintended target in close combats or grapples.}{
Sharp Shooting -- Your ranged attacks ignore Cover Bonuses (total cover still bones you).}{
Opponents struck by your ranged attacks do not automatically know what square your attack came from, and must attempt to find you normally.}{
Any time you hit an opponent with a ranged weapon, it is counted as a critical threat. If your weapon already had a 19-20 threat range, increase its critical multiplier by 1.}


\hypertarget{feat:subtlecut}{}\babfeat{Subtle Cut [Combat]}{
You cut people so bad they have to ask you about it later.}{
Any time you damage an opponent, that damage is increased by 1.}{
As a standard action, you can make a weapon attack that also reduces a creature's movement rate. For every 5 points of damage this attack does, reduce the creature's movement by 5'. This penalties lasts until the damage is healed.}{
As a standard action, you may make a weapon attack that also does 2d4 points of Dexterity damage.}{
Any weapon attack that you make at this level acts as if the weapon had the wounding property.}{
As a standard action, you may make an attack that dazes your opponent. This effect lasts one round, and has a DC of 10 + half your level + your Intelligence bonus.}

\vspace{10pt}
\hypertarget{feat:twoweaponfighting}{}\babfeat{Two Weapon Fighting [Combat]}{
When armed with two weapons, you fight with two weapons rather than picking and choosing and fighting with only one. Kind of obvious in retrospect.}{
You suffer no penalty for doing things with your off-hand. When you make an attack or full-attack action, you may make a number of attacks with your off-hand weapon equal to the number of attacks you are afforded with your primary weapon.}{
While armed with two weapons, you gain an extra Attack of Opportunity each round for each attack you would be allowed for your BAB, these extra attacks of opportunity must be made with your off-hand.}{
You gain a +2 Shield Bonus to your armor class when fighting with two weapons and not flat footed.}{
You may Feint as a Swift action.}{
While fighting with two weapons and not flat footed you may add the enhancement bonus of either your primary or your off-hand weapon to your Shield Bonus to AC.}


\hypertarget{feat:weaponfinesse}{}\babfeat{Weapon Finesse [Combat]}{
You are incredibly deft with a sword.}{
You may use your Dexterity Modifier instead of your Strength modifier for calculating your melee attack bonus.}{
Your special attacks are considered to have the Edge when you attack an opponent with a Dexterity modifier smaller than yours, even if your Base Attack Bonus is not larger.}{
You may use your Dexterity modifier in place of your Strength modifier when attempting to trip an opponent.}{
You may use your Dexterity modifier in place of your Strength modifier for calculating your melee damage.}{
Opportunist -- Once per turn, when an opponent is struck, you may take an attack of opportunity on that opponent.}

\vspace{300pt}
\hypertarget{feat:whirlwind}{}\hypertarget{feat:whirlwindattack}{}\babfeat{Whirlwind [Combat]}{
You are just as dangerous to everyone around you as to anyone around you.}{
As a full round action, you may make a single attack against each opponent you can reach. Roll one attack roll and compare to each available opponent's AC individually.}{
You gain a +3 bonus to Balance checks.}{
As a full round action, you may take a regular move action and make a single attack against each opponent you can reach at any point during your movement. Roll one attack roll and compare to each available opponent's AC individually.}{
Until your next round after making a whirlwind attack, you may take an attack of opportunity against any opponent that enters your threatened area.}{
As a full round action, you take a charge action, overrunning any creature in your path, and may make a single attack against each opponent you can reach at any point during your movement. Roll one attack roll and compare to each available opponent's AC individually.}


\hypertarget{feat:zenarchery}{}\babfeat{Zen Archery [Combat]}{
You are very calm about shooting people in the face. That's a good place to be.}{
You may use your Wisdom Modifier in place of your Dexterity Modifier on ranged attack rolls.}{
Any opponent you can hear is considered an opponent you can see for purposes of targeting them with ranged attacks.}{
If you cast a Touch Spell, you can deliver it with a ranged weapon (though you must hit with a normal attack to deliver the spell).}{
As a Full Round Action, you may make one ranged attack with a +20 Insight bonus to hit.}{
As a Full Round Action, you may make one ranged attack with a +20 Insight bonus to hit. If this attack hits, your attack is automatically upgraded to a critical threat. If the threat range of your weapon is 19-20, your critical multiplier is increased by one.}

\end{multicols}

\section{Spells}

\section{Elemental Spheres}
\quot{``Let our powers combine.''}

Elemental Spheres function like Fiendish Spheres, except that each one is associated with one or more of the Elements. An Elemental Weird taking an Elemental Sphere must be a Weird of the same element as the Sphere. It is reccomended that Genies follow these same rules (yes, this is a nerf, to keep them thematically appropriate. While it is a cute trick to take the Cold sphere on your Efreeti to gain immunity to Cold and offset your Vulnerability with already being immune, it's not really thematically appropriate to have an Efreeti better at Cone of Cold than it is at Fireball). Note that Magma has no spheres listed for it. This is intentional; it can draw Spheres from both the Earth and Fire lists. Spheres listed with Fiend as an element are new, and can be taken as a Fiendish Sphere.

\newcommand{\sphere}[9]{\begin{list}{}{\itemspace}\item \textbf{1:} \spell{#1} \item \textbf{3:} \spell{#2} \item \textbf{5:} \spell{#3} \item \textbf{7:} \spell{#4} \item \textbf{9:} \spell{#5} \item \textbf{11:} \spell{#6} \item \textbf{13:} \spell{#7} \item \textbf{15:} \spell{#8} \item \textbf{17:} \spell{#9}}
\newcommand{\spherecont}[1]{\item\textbf{19:} \spell{#1} \end{list}\medskip}

Any unlisted parts of a Sphere (such as its granted ability or spell list) use the Fiendish Sphere of the same name. 

\begin{multicols}{2}
\section{Spheres} \hypertarget{spheres}{}

\newcommand{\sphere}[9]{\begin{list}{}{\itemspace}\item \textbf{1:} \spell{#1} \item \textbf{3:} \spell{#2} \item \textbf{5:} \spell{#3} \item \textbf{7:} \spell{#4} \item \textbf{9:} \spell{#5} \item \textbf{11:} \spell{#6} \item \textbf{13:} \spell{#7} \item \textbf{15:} \spell{#8} \item \textbf{17:} \spell{#9}}
\newcommand{\spherecont}[1]{\item\textbf{19:} \spell{#1} \end{list}\medskip}

Fiends, celestials, and some characters cast magic primarily through spell-like abilities. While many monsters and characters get arbitrary spell lists, spheres present a way to advance spellcasting in a thematic way. When a creature has access to a sphere, she is able to use all of the abilities within that sphere up to her character level. If they gains more levels, more powers of the sphere become available. In this way the spell-like abilities of fiends created with the rules in this tome should always be \ae sthetically and level appropriate.

\ability{Basic Sphere Access:}{When a creature has basic access to a sphere, she can use any of the spells listed in the sphere may be used once per day (each) as spell-like abilities, provided that their listed level is equal or lower to the creature's character level.}

\ability{Advanced Sphere Access:}{When a creature has advanced access to a sphere, she can use any of the spells listed in the sphere may be used 3 times per day (each) as spell-like abilities, provided that their listed level is equal or lower to the creature's character level.}

\ability{Expert Sphere Access:}{When a creature has expert access to a sphere, any spells listed in the sphere may be used at will as spell-like abilities, provided that their listed level is equal or lower to the creature's character level.}

\ability{Creating new spheres:}{The following list of spheres isn't intended to be comprehensive, and we fully expect that some players and DMs will want many more spheres than we have scribed. All new spheres must be approved of by the DM, and should represent some actual (indifferent or evil) trait like ``intoxication" or ``badgers" rather than a game mechanical notion like ``kicking ass and being totally sweet" or something praiseworthy like ``generosity". A good place to start is actually Domains, as these are already a source by which a character gain a spell at every odd-numbered level.}

\ability{Spheres and Spell Levels:}{Spell-like abilities used out of spheres are considered to be cast as a spell level equal to half the minimum needed character level to use the ability (rounded up). The save DC of a spell-like ability granted through Sphere access is Charisma-based. Thus, the save DC for a spell-like ability which becomes available at character level 5 is 13 + Charisma bonus.}
\end{multicols}

\chapter{Elemental Denizens}

\section{Dragons}

\classname{Half Dragon}
\vspace*{-8pt}
\quot{''Also, dragons are pretty! Very pretty.''}

Half Dragon is a character class that can be used to make a monster into a more ''dragon-like'' monster. It can also be taken by player characters who want to be more dragon-like, for whatever reason. Maybe they got spilled with dragon blood and became dragon-like themselves. Maybe their mother slept with a dragon in disguise. Maybe the dragon wasn't in disguise. I don't want to hear about it.

\ability{Prerequisites:}
\listprereq
\itemability{Special:}{Must have some reason why you're more draconic than everyone else. Maybe it's inherited, maybe it's not. Something. Just don't go into too much detail, I probably don't even want to know.}
\itemability{Special:}{Must choose one appropriate dragon type to be more like.}
\end{list}

\ability{Hit Die:}{d12}

\ability{Class Skills:}{The Half Dragon's class skills (and the key ability for each skill) are Bluff (Cha), Climb (Str), Concentration (Con), Craft (Int), Diplomacy (Cha), Hide (Dex), Intimidate (Cha), Jump (Str), Knowledge (all skills individually) (Int), Listen (Wis), Move Silently (Dex), Move Silently (Dex), Search (Int), Spot (Wis), Swim (Str).}

\ability{Skills/Level:}{6 + Intelligence Bonus}

\begin{table}[tbh]
\begin{small}
\begin{tabular}{lp{1.9cm}p{0.7cm}p{0.7cm}p{0.7cm}l}
Level  &Base Attack Bonus &Fort Save &Ref Save &Will Save &Special\\
1  & +1 & ++2 & +2 & +2 & Breath Weapon, Scales of the Dragon, Dragon Sight\\
2  & +2 & +2 & +2 & +2 &  Wings of the Dragon, Subtypes, Immunities, Strength of the Dragon\\
3  & +3 & +3 & +3 & +3 &  Sphere, Spell Resistance\\
\end{tabular}
\end{small}
\end{table}

\smallskip\noindent All of the following are Class Features of the Half Dragon prestige class.

\ability{Weapon and Armor Proficiency:}{The Half Dragon gains no proficiency with armor or weapons.}

\ability{Breath Weapon (Su):}{A Half Dragon is able to emit a deadly breath weapon. The energy type of the weapon is appropriate to whatever dragon type is involved in the dragon-half. As appropriate, it will either come out as a cone with short range, or a line which is twice that long. The breath weapon does 1d6 of damage per character level (or CR) of the Half Dragon and the Reflex save DC is 10 + \half Hit Dice + Con Modifier. Using this breath weapon is a Standard action, and once used it cannot be used again for 2d4 rounds.}

\ability{Scales of the dragon (Ex):}{Half Dragons gain an Enhancement Bonus equal to \third of their Hit Dice (rounded up) to their Natural Armor Bonus to AC.}

\ability{Dragon Sight (Ex):}{Half Dragons gain 60' of Darkvision and see 3 times as far in limited light conditions as per Low Light Vision.}

\ability{Wings of the Dragon (Ex):}{At second level, a half dragon grows dragon wings. They may be small and cute or huge and cloak-like. But regardless of size or mass ratios, they allow the character to fly half again as fast as their land movement rate with average maneuverability.}

\ability{Subtypes:}{A Half Dragon of second level has the Dragon Subtype, as well as whatever elemental or energy subtypes is appropriate for whatever kind of dragon is manifest within her. For example: a Half Black Dragon would gain subtypes of [Dragon, Water].}

\ability{Immunities (Ex):}{2nd level Half Dragons are immune to sleep and paralysis effects, as well as the energy type that they themselves breathe on people.}

\ability{Strength of the Dragon (Ex):}{Upon achieving 2nd level, a Half Dragon is blessed with increased Strength. She gains an Enhancement Bonus equal to \third of her hit dice (rounded up) to her Strength score.}

\ability{Sphere:}{A 3rd level Half Dragon gains a single Sphere.}

\ability{Spell Resistance (Ex):}{3rd level Half Dragons have Spell Resistance of 2 + Character Level (yes, only 2 + Character Level, I'm not even kidding).} 

\section{Elementals}

\subsection{Elemental-Bodied}

\noindent\desc{Elemental-Bodied are humanoid figures born from the essense of one of the elemental planes, solidified into something fleshlike.  While not entirely bright, Elemental-Bodied are more than capable of becoming both powerful creatures and adventurers.}

\noindent\desc{Elemental-Bodied is a general term; an Elemental-Bodied character with the (Earth) subtype, for instance, is called an Earthbodied.}

\noindent\desc{Elemental-Bodied have the following racial traits:}

\desc{\listone
    \item Medium Size: Elemental-bodied average about 6' tall.  They have no sex or gender.
    \item 20' movement, plus additional movement depending on type.
    \item Outsider Type, appropriate subtype (Air, Earth, Fire, or Water).  Elemental-Bodied are native to the appropriate elemental plane, and are (Extraplanar) on the Material Plane.
    \item Elemental Flesh: Elemental-Bodied are counted as Elementals, not Outsiders, for effects that target specific creature types.
    \item +2 Str (Earth, Water), +2 Dex (Air, Fire), -2 Int
    \item Darkvision 60'
    \item Elemental-Bodied do not sleep, and are immune to magic sleep effects.
    \item +4 to saves against Poison, Disease, and Paralysis.  Elemental-Bodied have an unusual physiology that is not subject to the same constraints as mortals.
    \item Light Fortification
\end{list}}

\noindent\desc{Elemental-Bodied gain more abilities, dependent on their specific type:}

\desc{\listone
    \item Airbodied gain a 15' Fly speed with Perfect maneuverability, a 30' base land speed, and a +2 bonus on Athletics checks.
    \item Earthbodied gain a 20' Burrow speed, and a +2 bonus on Survival checks.
    \item Firebodied have a 20 Climb speed.  They also shed light, brightly illuminating between up 40', and shedding shadowy illumination  over twice that distance, or a minimum of 5'.  They may change brightness or douse thesmelves to smoulder (5' shadow illumination) as a swift action.  They also gain a +2 bonus on Acrobatics checks (Balance and Tumble, under PHB skills).
    \item Waterbodied gain a 60' Swim Speed and breathe both water and air.  They also gain a +2 bonus on Escape Artist checks.
    \item Favored Class: Elemental Brute and Elemental Weird
    \item Automatic Languages: Appropriate elemental language
    \item Bonus Languages: Abyssal, Aquan, Auran, Celestial, Common, Draconic, Formian, Ignan, Infernal, Slaad, Terran.
\end{list}}

\classname{Elemental Brute}
\vspace*{-8pt}
\quot{"Wind etches cliff walls.\\
A landslide falls to the sea.\\
Raindrops douse spring fires."}

\desc{The powers of the elements are more than just a source for mages to tap, but the building blocks out of which everything else is made.  By combining more of the elements onto one's self, an elemental can become a being of immense physical power.  Elemental Brutes not only grow this power, but also use it to control the battlefield around them.}

\desc{Elemental Brutes are heavy hitters in melee, and use their abilities to make the battlefield a more difficult place for their opponents and to protect their allies.}

\ability{Alignment:}{Elemental Brutes can be any alignment, and often are.}

\ability{Races:}{Only Outsiders and Elementals from the Elemental Planes may take levels in Elemental Brute.  Those from the Material plane whose ancestors come from the Elemental Planes may take levels in Elemental Brute, but they must have the Outsider or Elemental type.}

\ability{Starting Gold:}{4d4x10 GP}

\ability{Starting Age:}{As Rogue.}

\ability{Hit Die:}{d10}

\ability{Base Attack Bonus:}{Medium (as Cleric)}

\ability{Saving Throws:}{Good Fortitude and either Reflex (Air, Fire) or Will (Earth, Water)}

\ability{Class Skills:}{Balance (Dex), Climb (Str), Hide (Dex), Jump (Str), Listen (Wis), Move Silently (Dex), Sense Motive (Wis), Spot (Wis), Survival (Wis), and Swim (Str).}

\ability{Nice Skills:}{Acrobatics (Dex), Athletics (Con), Jump (Str), Perception (Wis), Sense Motive (Wis), Survival (Wis), and Stealth (Dex)}

\ability{Skill Points per Level:}{4 + Int Mod (x4 at 1st level)}

\begin{table}[htb]
\begin{small}
\begin{tabular}[h]{lp{1.9cm}p{0.7cm}p{1.5cm}p{0.7cm}p{9cm}}
&   Base Attack Bonus&  Fort Save&  Ref Save&   Will Save&  Special\\
1&  +0& +2& +2/+0& +0/+2&  Elemental Origin, Natural Armor, Natural Weapon, Bonus Feat, Combat Talent +1\\
2&  +1& +3& +3/+0& +0/+3&  Attribute Boost\\
3&  +2& +3& +3/+1& +1/+3&  Elemental Mastery\\
4&  +3& +4& +4/+1& +1/+4&  Attribute Boost\\
5&  +3& +4& +4/+1& +1/+4&  Bonus Feat, Combat Talent +2\\
6&  +4& +5& +5/+2& +2/+5&  Attribute Boost\\
7&  +5& +5& +5/+2& +2/+5&  Horde Breaker\\
8&  +6/+1& +6& +6/+2& +2/+6&  Attribute Boost\\
9&  +6/+1& +6& +6/+3& +3/+6&  Bonus Feat, Combat Talent +3\\
10& +7/+2& +7& +7/+3& +3/+7&  Attribute Boost\\
\end{tabular}
\end{small}
\end{table}

\ability{Weapon and Armor Proficiency:}{Elemental Brutes are proficient only with their natural weapons, and with no armor.}

\ability{Elemental Origin:}{An Elemental Brute grows from the material of one of the elemental planes.  It has the most obvious connection available (Genasi (except Material Genasi) and Elemental-Bodied all indicate a connection with one of the planes, as do all elemental monsters in the Monster Manual).  If none suggests itself, such as for a Material Genasi, the Elemental Brute gets to pick one of the four elemental planes.  Certain abilities come in different forms depending on elemental origin.  Air, Fire, and Shadow Brutes gain the General Feat version of Weapon Finesse as a bonus feat.}

\ability{Natural Weapon:}{An Elemental Brute gains a Slam attack for 1d8 damage (if medium), plus strength and a half.}

\ability{Natural Armor:}{Elemental Brutes gain a natural armor bonus to AC equal to their level.}

\ability{Bonus Feat:}{At 1st, 5th, and 9th level, the Elemental Brute gains a bonus [Elemental], [Monstrous], or [General] feat that it qualifies for.  At 7th level it gains Horde Breaker.  If it already has Horde Breaker, it may gain another bonus feat, or another Combat feat that it qualifies for.}

\ability{Combat Talent (ex):}{You may add the listed number to your base attack bonus for purposes of unlocking abilities of [Combat] Feats.  If playing with fractional BAB, instead count the Elemental Brute as providing full BAB for purposes of unlocking [Combat] Feat abilities}

\ability{Attribute Boost:}{At 2nd level and every even numbered level afterward, the Elemental Brute's physical attributes improve, as if it had gained several character levels. Every time the Elemental Brute gains an attribute boost, its Constitution and one other physycial attribute dependent on Elemental Origin irrevocably increase by 1.

\noindent\begin{tabular}[h]{ll}
\emph{Strength}: &Earth, Water, Ice, Magma, Wood\\
\emph{Dexterity:} &Air, Fire, Shadow\\
\end{tabular}}

\ability{Elemental Mastery (ex):}{An Elemental Brute of 3rd level or higher gains The Edge against opponents in its element.

\noindent\begin{tabular}[h]{ll}
\emph{Air:} &An Air Brute gains The Edge against flying opponents.\\
\emph{Earth:} &An Earth Brute gains The Edge as long as its weight and its opponent's weight are both on the ground.\\
\emph{Fire:} &A Fire Brute gains The Edge against any opponent that is presently on fire or otherwise taking continuous fire damage when it makes its attack.\\
\emph{Water:} &A Water Brute gains The Edge against any opponent immersed at least halfway in water.\\
\end{tabular}

Exotic elemental types also gain this ability:

\noindent\begin{tabular}[h]{ll}
\emph{Ice:} &An Ice Brute gains The Edge against any opponent standing on ice.\\
\emph{Magma:} &A Magma Brute behaves like a Fire Brute.\\
\emph{Shadow:} &A Shadow Brute gains The Edge against any opponent against whose attacks it has concealment or total concealment.\\
\emph{Wood:} &A Wood Brute gains The Edge against any opponent who ended their last turn adjacent to it without attacking it.\\
\end{tabular}}

\subsubsection*{Playing an Elemental Brute}
\ability{Religion:}{Elemental Brutes are seldom particularly religious, although they may follow an Elemental deity of their appropriate kind.  Recently, a cult of evil Elemental Brutes was found worshipping the Elder Elemental Eye.}

\ability{Other Classes:}{Elemental Brutes get along well with Elemental Weirds of like element, as their abilities complement eachother well.  They do not necessarily get along well with Genies, as many Genies expect to be lords over the Elemental Brutes.  Elemental Brutes get along well with Druids, and with properly respectful Elementalists, whose magic complements the Elemental Brute's physical power.  Stealthy types often appreciate the distraction that an Elemental Brute can provide, and warriors may appreciate having an Elemental Brute to spar with.  Casters may view an Elemental Brute as little higher than their summoned minion, earning the Brute's wariness.}

\ability{Combat:}{Elemental Brutes are effective at controlling the battlefield to force enemies into melee with them. Air Brutes often take an air superiority role, tripping flying opponents to cause them to stall and drop them to where their allies can reach them, while Earth and Water Brutes control a section of battlefield with their bulk. Fire Brutes make effective skirmishers, especially with the Whirlwind feat to allow them to attack several creatures at once, and Burn to allow them the Edge against any creature they've already hit. Exotic Brutes gain other abilities to match.}

\ability{Advancement:}{An Elemental Brute, after completing the class, may take levels in Elemental Weird, or in another combat-focused class, or may become a Genie.  A few levels of Elemental Brute can add quite a bit to many melee Genasi or Elemental-Bodied builds, especially to gain the Large Size and Huge Size feats, granting the character a greater reach and the ability to use bigger weapons.  Other melee classes grant the Elemental Brute the ability to use better weapons and shields, and multiclassing into Monk would give the Brute the ability to use fighting styles with its elemental Slams.}

\classname{Elemental Weird}
\vspace*{-8pt}
\quot{''The guardian of the Temple of the Seas shall not allow you to pass''}

\desc{The Elements are known as one of the many primal sources of magic.  Mortal mages and even beings of the outer planes draw on the energies of the Inner Planes to fuel their magic, but Elemental Weirds, as beings of the elements themselves, surpass them all at drawing on raw elemental power.  For an elemental to become an Elemental Weird is to drink deep from the wellspring of magic, soar on the winds of fate, cast sorceries as the roots of the mountains, and burn with the fires of power, fully seizing their birthright as elementals.  Those Elementals and Genasi who walk this path gain a terrifying mastery of the magics of the elements.}

\ability{Alignment:}{Most Elemental Weirds tend toward at least some form of Neutrality, but it takes all kinds.}

\ability{Races:}{Only Outsiders and Elementals from the Elemental Planes may take levels in Elemental Weird.  Those from the Material Plane whose ancestors come from the Elemental Planes may take levels in Elemental Weird, but they must have the Outsider or Elemental type.}

\ability{Starting Gold:}{4d4 x 10 GP (100 GP)}

\ability{Starting Age:}{Complex (as Wizard)}

\ability{Hit Die:}{d6}

\ability{Class Skills:}{Concentration (Con), Craft (Int), Diplomacy (Cha), Knowledge (All skills, taken individually) (Int), Spellcraft (Int)}
\ability{Skill Points per Level:}{2 + Int Mod}

\begin{table}[htb]
\begin{small}
\begin{tabular}[h]{lp{1.9cm}p{0.7cm}p{0.7cm}p{0.7cm}l}
&   Base Attack Bonus&  Fort Save&  Ref Save&   Will Save&  Special\\
1&  +0& +0& +0& +2&  Elemental Origin, Sphere\\
2&  +1& +0& +0& +3&  Enhanced Sphere Access\\
3&  +1& +1& +1& +3&  Sphere\\
4&  +2& +1& +1& +4&  Hardiness of the Elements\\
5&  +2& +1& +1& +4&  Sphere\\
6&  +3& +2& +2& +5&  Elemental Skills\\
7&  +3& +2& +2& +5&  Sphere\\
8&  +4& +2& +2& +6&  Unstoppable Force\\
9&  +4& +3& +3& +6&  Sphere\\
10& +5& +3& +3& +7&  Magical Training\\
\end{tabular}
\end{small}
\end{table}

\ability{Weapon and Armor Proficiency:}{An Elemental Weird is proficient with all Simple weapons, and a martial weapon dependent on Elemental Origin, and Light armor, but not with shields of any kind.
Origin Proficiency:
\begin{tabular}[h]{ll}
\emph{Air} &Bolas, Throwing Axe\\
\emph{Earth} &Light and Heavy pick\\
\emph{Water} &Trident, Net\\
\emph{Fire} &Spiked Chain\\
\end{tabular}
Exotic origins also grant a weapon proficiency:
\begin{tabular}[h]{ll}
\emph{Ice} &Handaxe, Battleaxe\\
\emph{Magma} &Light Hammer, Warhammer\\
\emph{Shadow} &Shortsword, Kukri\\
\emph{Wood} &Glaive, Greatclub\\
\end{tabular}}

\ability{Elemental Origin:}{An Elemental Weird's power originates with one of the elemental planes, typically one of the four classical planes.  They have the most obvious connection available (Genasi (except Material Genasi) and Elemental-Bodied all indicate a connection with one of the planes, as do all elemental monsters in the Monster Manual). If none suggests itself, such as for a Material Genasi, the Elemental Brute gets to pick one of the four elemental planes.  The Elemental Weird's origin determines the Spheres to which it gains access.}

\ability{Sphere:}{The Elemental Weird gains basic access to a Sphere at every odd numbered level. If the Elemental Weird selects a sphere that it already has basic access to, it upgrades its access to advanced access. If it already had advanced access, it gains expert access.}

\ability{Enhanced Sphere Access:}{At 2nd level, the Elemental Weird gains extra uses of the spell-like abilities that it gains from it Spheres. The Elemental Weird gains a number of extra uses of any spell-like ability equal to half the number by which its character level exceeds the minimum character level to use the spell-like ability (rounded up). So if the Elemental Weird has a character level of 4, it would gain 1 extra use of a spell-like ability that is granted by one of it spheres at character level 3 and 2 extra uses of any spell-like from one of its spheres with a minimum level of 1.  The Elemental Weird gains a +1 bonus to caster level for all spell-like abilities cast from Spheres to which it has Expert acess, and saving throws against such abilities are made against a DC of 11 + 1/2 the Weird's character level (rounded up) + the Weird's charisma modifier (the highest save DC it would have, plus one).}

\ability{Bonus Feats:}{An Elemental Weird gains the Hardiness of the Elements feat as a bonus feat at level 4, and Unstoppable Force as a bonus feat at level 8.  If it already has Hardiness of the Elements at level 4, it gains Unstoppable Force then instead.  If it already has Unstoppable Force when it is given that as a bonus feat, it gains its choice of any [ Elemental ] or [ Item Creation ] feat instead.}

\ability{Elemental Skills:}{An Elemental Weird gains a +10 Competence bonus to a skill depending on their elemental origin:
\begin{tabular}[h]{ll}
\emph{Air} &Tumble\\
\emph{Earth} &Knowledge (Dungeoneering)\\
\emph{Fire} &Jump\\
\emph{Water} &Escape Artist\\
\end{tabular}
Exotic origins gain the following:
\begin{tabular}[h]{ll}
\emph{Ice} &Sense Motive\\
\emph{Magma} &Intimidate\\
\emph{Shadow} &Sleight of Hand\\
\emph{Wood} &Search\\
\end{tabular}}

\ability{Magical Training:}{An Elemental Weird of 10th level is able to cast magic in a more traditional fashion.  It has the Spells per day and spells known (including Advanced Learning) of a 6th level Elementalist, and a caster level of 10.  At its option, it may use Charisma instead of Intelligence to determine the highest level of spells it may cast and its bonus spells per day, or instead of Wisdom to determine spell save DCs, but not both.  It may take classes that improve spellcasting of existing classes in order to advance its spellcasting ability.}

\section{Genies}

\subsection{Genasi}
\vspace*{-8pt}
\quote {"Fire burns in my veins. And soon, in yours."}

Genasi are the descendents of mortals and genies of various kinds.  Mortals with minor infusions of elemental essense before birth, or descended from those with stronger infusions, also appear as Genasi.  Their physical traits are influenced by the elements they are descended from.

Depending on their elemental influences, Genasi are likely to become every kind of adventurer.

Genasi have the following racial traits:
\listone
    \item Medium Size. Genasi fall into the human height and weight ranges, although Earth Genasi tend to be squat and Air Genasi willowy.
    \item 30' movement
    \item Outsider Type (Native and Human subtypes)
    \item Elemental Subtype of the appropriate element: (Air), (Earth), (Fire), or (Water).
    \item Darkvision 60'
    \item +2 Strength
    \item +1 to the DC to save against any spells or spell-like abilities cast by the Genasi with a descriptor matching the Genasi's elemental subtype.
    \item Favored Class: Genie and by element: Air: Thief-Acrobat.  Earth: Knight.  Fire: Elementalist.  Water: Monk.
    \item Automatic Languages: Common
    \item Bonus Languages: Abyssal, Aquan, Auran, Celestial, Draconic, Formian, Ignan, Infernal, Slaad, Terran
\end{list}

Genasi also gain another ability that reflects their elemental origin:
\listone
    \item Air: Fly speed of 10' with Good maneuverability
    \item Earth: Burrow speed of 10'
    \item Fire: \spell{Produce flame} as a spell-like ability three times per day, cast at the Genasi's character level.  
    \item Water: Swim speed of 30' and ability to breathe water as easily as air.
\end{list}

\textbf{Fire Genasi Genie Substitution Class Feature:}

\desc{Instead of getting Immunity to Fire as a Genie at 1st level, the Fire Genasi instead loses their Cold Vulnerability.}

\textbf{Material and Shadow Genasi:}

\textbf{Material Genasi} are the descendents of Jann and mortals.  A Material Genasi does not get an elemental subtype, but instead receives a +1 to their saving throw against all spells of any elemental subtype.  Their favored classes are Druid and Genie.

\textbf{Shadow Genasi} are the descendents of Khayal Genies (see the Tome of Magic) and mortals.  A Shadow Genasi gains a bonus to Constitution instead of Strength, no elemental subtype, Spell Focus (Illusion) as a bonus feat, which also applies to any spell-like abilities it may have, and the ability to cast \spell{invisibility} once per day as a spell-like ability with a caster level equal to their character level).  Their favored classes are Beguiler and Genie.
\classname{Genie} \label{class:genie}
\vspace*{-8pt}
\quot{"Who controls the past, controls the future. Who controls the \spell{wishes}, controls the past."}

\desc{Djinn, Efreet, Marid, Dao\ldots these are the names that inspire terror and greed throughout the planes, and with good reason. These Genies are far more powerful than the other denizens of the Elemental Sultanates, and it is for this reason that they rule them. They control the wishes, and for many they may as well control the \textit{universe}.}

\desc{The Genies are universally accomplished, but this doesn't make them more powerful at any particular level than any other character. Indeed, level is a measure of power. The most powerful denizens of the Elemental Planes are Genies \textit{and higher level} than mere elementals. The Genie advances in everything all at once, and thus gains new abilities relatively slowly compared to what other, more focused Outsider progressions are capable of.}

\ability{Alignment:}{While the Elemental Planes are Neutral, the denizens often are not. Genies can be of any alignment and often are.}

\ability{Races:}{The Genie is \textit{only} available to Outsiders with a plane of origin in the Elemental Planes. Creatures from the prime material plane whose ancestors were from an Elemental Plane may take this class, but they must have the Outsider type.}

\ability{Starting Gold:}{6d6x10 gp (210 gold)}

\ability{Starting Age:}{Since a Genie is immortal and never ages, a character may claim any starting age she wishes.}

\ability{Hit Die:}{d8}

\ability{Class Skills:}{The Genie's class skills (and the key ability for each skill) are Appraise (Int, Bluff (Cha), Climb (Str), Concentration (Con), Craft (Int), Diplomacy (Cha), Disguise (Cha), Escape Artist (Dex), Hide (Dex), Intimidate (Cha), Knowledge (all skills taken individually) (Int), Listen (Wis), Move Silently (Dex), Profession (-), Search (Int), Sense Motive (Wis), Sleight of Hand (Dex), Speak Language (-), Spellcraft (Int), Spot (Wis), Survival (Wis), Use Magic Device (Cha), and Use Rope (Dex).}

\ability{Skills/Level:}{8 + Intelligence Bonus}

\begin{table}[tbh]
\begin{small}
\begin{tabular}{lp{3.5cm}p{0.7cm}p{0.7cm}p{0.7cm}l}
Level &Base Attack Bonus &Fort Save &Ref Save &Will Save &Special\\
1st &+1 &+2 &+2 &+2 &Immortality, Planar Traits\\
2nd &+2 &+3 &+3 &+3 &Telepathy, Lesser Genie Powers\\
3rd &+3 &+3 &+3 &+3 &Genie Powers\\
4th &+4 &+4 &+4 &+4 &Sphere\\
5th &+5 &+4 &+4 &+4 &Greater Planar Traits\\
6th &+6/+1 &+5 &+5 &+5 &\\
7th &+7/+2 &+5 &+5 &+5 &Greater Genie Powers\\
8th &+8/+3 &+6 &+6 &+6 &Sphere\\
9th &+9/+4 &+6 &+6 &+6 &\textit{Summon}\\
10th &+10/+5 &+7 &+7 &+7 &\\
11th &+11/+6/+6 &+7 &+7 &+7 &Grant Wishes\\
12th &+12/+7/+7 &+8 &+8 &+8 &Sphere\\
13th &+13/+8/+8 &+8 &+8 &+8 &Damage Reduction\\
14th &+14/+9/+9 &+9 &+9 &+9 &Awesome Planar Traits\\
15th &+15/+10/+10 &+9 &+9 &+9 &Greater Summoning\\
16th &+16/+11/+11/+11 &+10 &+10 &+10 &Sphere\\
17th &+17/+12/+12/+12 &+10 &+10 &+10 &Elemental Power\\
18th &+18/+13/+13/+13 &+11 &+11 &+11 &\textit{Gate}\\
19th &+19/+14/+14/+14 &+11 &+11 &+11 &Epic Damage Reduction\\
20th &+20/+15/+15/+15 &+12 &+12 &+12 &Sphere\\
\end{tabular}
\end{small}
\end{table}

\smallskip\noindent All of the following are Class Features of the Genie class.

\ability{Weapon and Armor Proficiency:}{Genies are proficient with all simple and martial weapons, as well as the whip, the net, and the lajatang. Genies are proficient with light armor but not with shields of any kind.}

\ability{Immortality (Ex):}{Ageless as the earth and endless as the sky, the True Genie never ages and retains a youthful appearance unto a thousand thousand generations.}

\ability{Planar Traits:}{A Genie is a member of one of the iconic aristocracies of the Elemental Conflux. Starting at first level she may travel on any elemental plane without suffering the baleful effects of those extreme environments, in addition she gains access to the distinctive abilities of her race, as befits her plane of origin:}

\listone
	\itemability{Fire:}{Efreet Traits:}
	\listtwo
		\itemability{Heat(Ex):}{Any time an Efreet hits an opponent with a melee attack or she is struck with a natural weapon she inflicts her Constitution Modifier in Fire Damage in addition to whatever else she does. This ability may be suppressed as a standard action and resumed as a Swift action.}
		\itemability{Immunity to Fire:}{An Efreet takes no damage from fire of any kind.}
	\end{list}
	\itemability{Air:}{Djinn Traits:}
	\listtwo
		\itemability{Air Mastery (Ex):}{Airborne creatures suffer a -1 penalty to attack and damage rolls against a Djinn.}
		\itemability{Immunity to Acid:}{A Djinn takes no damage from Acid of any kind.}
	\end{list}
	\itemability{Water:}{Marid Traits:}
	\listtwo
		\itemability{Water Mastery (Ex):}{A Marid gains a +1 bonus to attack and damage rolls against opponents touching water.}
		\itemability{Water Breathing (Ex):}{A Marid benefits as per \spell{water breathing}, but non-magical and all the time.}
		\itemability{Immunity to Cold:}{A Marid takes no damage from cold of any kind.}
	\end{list}
	\itemability{Earth:}{Dao Traits:}
	\listtwo
		\itemability{Earth Mastery (Ex):}{A Dao recieves a +1 bonus to attack and damage rolls if both it and its opponent are touching the ground.}
		\itemability{Immunity to Electricity:}{A Dao takes no damage from electricity of any kind.}
	\end{list}
\end{list}

\ability{Telepathy (Su):}{At 2nd level, a Genie gains the ability to communicate telepathically with any creature that speaks a language within 100 feet.}

\ability{Genie Powers (Sp):}{At 2nd level, a Genie may cast \spell{create food and water} once per day. At 3rd level, the Genie can \spell{planeshift} at will. The only planes which can be accessed in this manner are the Elemental planes and the prime material. Only willing creatures may be transported. At 7th level, a Genie may cast \spell{major creation} 3 times a day. Any objects created which last more than 12 hours are permanent.}

\ability{Sphere:}{The Genie gains basic access to a sphere at 4th level, and gains a new sphere at every fourth level afterwards. If the Genie selects a sphere that she already has basic access to, she upgrades it to advanced access. If she already had advanced access, she gains expert access.}

\ability{Greater Planar Traits:}{A Genie of 5th level or more gains access to more of the distinctive abilities of her race, as befits her plane of origin:}

\listone
	\itemability{Fire:}{Efreet Traits:}
	\listtwo
		\itemability{Size Changing (Sp):}{An Efreet can change a creature's size up or down one size category for an hour, and can do this twice per day. This can be used offensively, and the save DC is Charisma based. This is the equivalent of a 2nd level spell.}
		\itemability{Start Fires (Su):}{An Efreet can set anything she can see on fire as a standard action.}
	\end{list}
	\itemability{Air:}{Djinn Traits:}
	\listtwo
		\itemability{Whirlwind (Ex):}{A Djinn can assume the form of a whirlwind, as described in the description of the Air Elemental.}
		\itemability{Gust of Wind (Sp):}{A Djinn can use \spell{gust of wind} at will.}
	\end{list}
	\itemability{Water:}{Marid Traits:}
	\listtwo
		\itemability{Drench (Ex):}{A Marid can extinguish normal or magical fires with a touch. This always works.}
		\itemability{Resistances:}{A Marid has Sonic, Fire, and Acid Resistance 10.}
	\end{list}
	\itemability{Earth:}{Dao Traits:}
	\listtwo
		\itemability{Earth Glide:}{At 5th level, a Dao is able to move through solid stone as if it were open space. She may take any non-living objects she can carry with her.}
	\end{list}
\end{list}

\ability{Summon (Sp):}{At 9th level, a Genie can attempt to \spell{summon} vassals and others of its kind (for example: a Djinn could summon other denizens of the Plane of Air). Summoning another creature of the same character level has a 40\% chance of success, and summoning a creature of a lower level increases the chances of success by 10\% for every level the summoner's level exceeds the CR of the target.}

\ability{Grant Wishes (Sp):}{At 11th level, a Genie bcomes a steward of the \spell{wish} economy. She may grant up to three mortal wishes each day. Doing so takes a few minutes to word the \spell{wish} properly and any costs are paid by the recipient (remember that many wishes do not have a special cost).}

\ability{Damage Reduction (Su):}{A 13th level Genie has Damage Reduction of Level/Adamantine. At 19th level this becomes DR of Level/Epic.}

\ability{Awesome Planar Traits:}{A Genie of 14th level or more gains access to more of the distinctive abilities of her race, as befits her plane of origin:}

\listone
	\itemability{Fire:}{Efreet Traits:}
	\listtwo
		\itemability{Firestorm (Sp):}{An Efreet can use \spell{firestorm} at will.}
	\end{list}
	\itemability{Air:}{Djinn Traits:}
	\listtwo
		\itemability{Telekinesis (Sp):}{A Djinn can use \spell{telekinesis} at will.}
	\end{list}
	\itemability{Water:}{Marid Traits:}
	\listtwo
		\itemability{Acid Fog (Sp):}{A Marid can use \spell{acid fog} once per hour.}
	\end{list}
	\itemability{Earth:}{Dao Traits:}
	\listtwo
		\itemability{Transmute Rock and Mud (Sp):}{A Dao can use \spell{transmute rock to mud} and \spell{transmute mud to rock} at will.}
	\end{list}
\end{list}

\ability{Greater Summoning:}{A Genie of 15th level may attempt to use her summon power to summon a creature of a level higher than her own, though doing so carries only a 30\% chance of success.}

\ability{Elemental Power:}{The powers of the lower planes are awesome to behold. At 17th level, the True Fiend gains a +10 bonus to defeating Spell Resistance with the spell-like abilities granted by her spheres.}

\ability{Gate (Sp):}{At 18th level, a Genie can open a \spell{gate} (transport version) whenever she wants.}


\chapter{Evocation Overhaul}

\chapter{High Adventure on the Inner Planes}

\section{High Adventure in the Inner Planes}

The Inner Planes are, for the most part, hostile to Material life and not all that feasible as adventure locations for a low-level party of elves, dwarves and humans. However, to a large extent that doesn't really matter since the sort of adventurers who can \emph{get} to the Plane of Fire are actually the ones who are perfectly adapted to it. So you could totally have a party of genasi and elementals and half-dragons, and they could totally run around the Elemental Plane of their choice and be perfectly fine. You pretty much are guaranteed that your party will only be able to even live in whichever plane they chose and the Plane of Air - but really that's just better than it is for humanoids.

Like the Lower Planes, the Inner Planes are infinite in size, and there are places with more high-level activity than others. You'll probably start off just fighting element humanoids or badgers or something like that well before the Genie overlords eve come into the picture, much like you'll be doing on the Prime before fighting Dragons or whatever.

The Inner Planes are infinite in size, and this is often taken as meaning that they are somehow filled with infinite power. This is essentially completely false. Remember that the Primes are essentially infinite in scope as well, and while there are ancient dragons and even Xixicals\ldots \textit{somewhere}, the fact is that you could adventure your whole life and never ever meet one. The world is mostly filled with forests, and mountains, and little river valleys, and most of the time the villains you encounter are going to be rabid dire weasels and bugbear junkies who will try to resell your shoes for a hit of mordayn vapor. Gehena is actually just like that, except that instead of you never seeing powerful dragons in your day to day life, you never see Arcanoloths. The bad guys you encounter may well be a \textit{fiendish} dire weasel and a bugbear junkie \textit{petitioner}, but the essential threat level is pretty much the same.

Low level adventuring, thus, is extremely plausible in the lower planes. It's not advisable for low level characters to go running around Tiamat's lair or anything, but the fact that the Elder Brain Pool is somewhere in the Underdark hasn't stopped \textit{any} low level campaigns from tunnel crawling as far as I can recall. What follows is some wilderness adventure seeds from the lower planes for low (1-5), medium (6-10), and high (11-15) level. Players who want to adventure at near epic levels (16+) don't even need adventure seeds of this sort because they actually can just take on The Dark Eight or whatever. For whatever reason, lots of ink has been spilled on near epic adventuring in the lower planes, and I have every confidence in a decent DM's ability to throw a Balor at a party and make a rollicking and dangerous encounter.


\subsection{High Adventure in\ldots Acheron!}

The first thing to realize about Acheron is that it really isn't a bad place to be. It's not even Evilly Aligned, so even campaigns using The Face of Horror have no reason to play up the terror of being here -- the sand of Acheron is not Evil. But it \textbf{is} made out of steel. Characters who are going to go adventuring will do so in Avalas, because that's the part of the plane that doesn't \textit{turn you to stone}.

\subsubsection{Campaign Seed: The Tunnel to Pandemonium}

Here's a little piece of D\&D history for you -- In AD\&D, Orcs were \textit{Lawful Evil}, so the Orcish pantheon lives in Acheron to war eternally with the Goblin pantheon \textit{even though Orcs are Chaotic now}. That means that the cube of Nishrek, where Gruumsh calls his most favored and despised for Gruumshian Justice when they have passed on -- is itself a bubble of Pandemonium found far from its place in the Wheel. There are, therefore, numerous portals to Pandemonium all over Clangor and Nishrek, so characters who wish to fight Orcs and Goblins in the lower planes can do so to an unlimited degree by portal hopping through the Pandemonium and Acheron layers. As an agent of Gruumsh or Maglubiet, characters can fight their way through savage humanoids, savage humanoid armies, savage humanoids with fiendish allies, savage humanoid war machines, and even powerful outsiders aligned with savage humanoids \textit{well into epic}. You can also use this rivalry as the backdrop for any of a number of ``find the artifact before it falls into seriously the wrong hands'' type adventures, with the characters switching sides repeatedly based on who seems to have the artifact now.

\subsubsection{Campaign Seed: You're in the Army Now}

Cities and castles populate the lands of Acheron without number, and all of them are on a war footing at all times. Characters can travel generally without molestation throughout this area and conduct a fairly profitable bit of trading and scavenging if they do things right. But if they do things wrong, they may end up drafted into some local militia or imperial army. Characters can have substantial numbers of adventures as part of a military force, or they can attempt to resist being drafted by any of a number of means. Unfortunately, the laws of Acheron being what they are, once the characters impress their will by force of personality or arms enough to avoid the draft, they'll find themselves as a \textit{side} -- which means that they'll be treated as a hostile army all themselves by other forces. At that point they can try to stick it out alone, or try to get some help, of course almost every empire in Acheron started the same way. So the players can progress smoothly from the ``chased by bad guys'' scenarios to the ``forge an empire in blood'' scenarios to the ``marry the princess, design your castle'' scenarios.

\subsubsection{Ten Low Level Adventures in Acheron}

You pull into the hamlet's bar, and see what they have to offer a stranger. It isn't good. After a brief set of questions to make sure you aren't going to burn the place down, the bartender tells you\ldots

\listone
	\item The town is infested with fiendish rats. Beer just isn't safe until their gone, sorry.
	\item A rival faction as poisoned the well, and someone needs to find a new source of water.
	\item Brigands are holding the pass. I hear one of them is an Ogre.
	\item The man you are looking for\ldots he was taken away by the Scarthian Army.
	\item That signet ring is part of King Imag's royal accoutrements. If someone could get all of them together\ldots it could spell big changes for the County of Yevekh.
	\item Orcs have come through the tunnel, their leader has a silver sword and noone dares to stand against him.
	\item After the Citadel of Zor fell, bodies were piled as high as your arm pit. I hear someone is making them all into zombies now, it's a shame really.
	\item I'd love to give you change, but after Sir Garreth set the taxes to 100\%, I'm afraid I have no coins to give you.
	\item In this town, either you're for Sheriff Braxton, or you're dead. This town, we like to have choices.
	\item It's free drinks here if you can get Clarrissa the hobgoblin matron to allow her daughters to marry.
\end{list}

\subsubsection{Ten Mid Level Adventures in Acheron}

An emissary of hoary Surog, the ice count, contacts you. He has (the ring, the antidote, the code) you need, and he'll give it you, but first\ldots

\listone
	\item One of his lieutenants has betrayed him; since you are random strangers, he can trust you to find out which one.
	\item His daughter has run off with the blue falcon, that accursed do-gooder. Bring her home, do with him as you wish.
	\item His daughter is the blue falcon. Stop her, but don't kill her.
	\item His daughter is the blue falcon, and Surog's rival, Cardinal Valgos, has put-her-in-a-death-trap. Rescue her, without letting on that Surog knows her identity.
	\item Cardinal Valgos has found some route to smuggle forces into Yevekh. Find how they're getting in.
	\item Cardinal Valgos is planning an attack, and Surog is not prepared. Infiltrate his mercenary forces and cause as much delay as possible.
	\item Cardinal Valgos has placed Surog in some kind of suspended animation! You have to lift the curse before one of Surog's underlings makes a play for power.
	\item A blue crossbow bolt with a head shaped like a stylized raptor strikes the emissary from nowhere, killing him before he can deliver your mission! Who is trying to stop you, and why?
	\item Cardinal Valgos has Imag's heir, or so he claims. Prove the heir is false, or steal him away.
	\item Cardinal Valgos has tricked the fox of the mountains, Dagipert, into allying with him. Break up this alliance one way or another.
\end{list}

\subsubsection{Ten High Level Adventures in Acheron}

You stand at the front of your army, triumphant over every foe the Lichking has sent against you, over the next hill you see\ldots

\listone
	\item The Lichking's vampire sister, all alone with a white flag.
	\item A pile of bodies impaled to the top of a 200 foot metal rod.
	\item A stampede of zombie elephants.
	\item A chasms cleaved into the side of the cube burbling with lava.
	\item A portal opening up upon an army of orcs in Pandemonium, easily the equal of your own.
	\item A huge pile of what appears to be gold.
	\item A huge pile of what appears to be skulls on fire.
	\item A wyvern bearing a message in its claws.
	\item The daughter of King Zormmund, tied to an elder earth elemental.
	\item Your grand vizier, who has apparently betrayed you again.
\end{list}


\subsection{High Adventure in\ldots Pandemonium!}

Pandemonium is a victim of the terrible confusion that permeates Law and Chaos in D\&D literature, and its inhabitants are portrayed in a number of improbable lights. Pandemonium is not an Evil plane, but it's fairly wicked and it's inherently Chaotic. How it and the people who live there appear in your game is entirely dependent upon how your game ends up handling Chaos in general. Pandemonium might be extremely disorganized, or inherently deceitful, or starkly unhelpful, or simply a lawless wilderness. But what it almost certainly \textit{isn't} is a source of low comedy where people do ``whacky stuff'' because they are so ``crazy''. That's the kind of thing that makes us cry.

Pandemonium can be a source of classic D\&D adventure at its finest -- the towns of Pandemonium are located right next to twisting tunnels through the stone and loud noises sound continuously through the warrens. So characters can go right from the town to the dungeon crawl without any explanation or overland travel, and those dungeon encounters are inherently episodic because nothing can hear your combats.

Pandemonium is dark and loud, and filled with confused people. At its best, Pandemonium is basically a huge rave. At its worst, Pandemonium is a huge rave. Like every part of the D\&D afterlife, Pandemonium can be a punishment or a reward. And like Acheron, this place isn't inherently Evil. So even if you are using The Face of Horror, the Eternal Rave isn't that bad of a place.

\subsubsection{Campaign Seed: Life in the Big City}

Welcome to The Madhouse. It's one of the largest planar metropolises in D\&D, and unlike places like the City of Brass or Sigil, it really \textit{doesn't} have some group of powerful outsiders ruling it with an iron fist. In fact, The Madhouse has no rulership of any kind. The place is dark, and loud, and the only light comes from naked women with glow sticks. Essentially, you can get away with pretty much anything without interference from opponents significantly outside your level range. You can keep having urban adventures continuously from 1st to 20th without ever getting seriously derailed by concerns of DM self-insertion characters coming over to knock over your house of cards. At the same time, there really \textit{are} Balors in this complex, so if you actually want to \textit{seek out} higher-powered enemies, that's doable.

\subsubsection{Campaign Seed: The Largest Dungeon}

Tunnels crisscross Pandemonium all over the place, and they are completely stable because the way gravity works there actually can't be a cave-in. But the place is dark and windy, and filed with tunnels that move around for no reason. The caverns are filled with monsters, traps, and treasure. It's all there, from shambling zombies to ninja temples, the low level areas cross seamlessly into the higher level ones. Oddly, this is the only place in the entire multiverse of D\&D where the old Gygaxian standby of having deeper and deeper levels of the dungeon filled with nastier and nastier monsters and traps actually makes sense. There's a town nearby, and the map doesn't have to make any sense at all. If you're looking for Nethack style adventuring, Pandemonium delivers.

\subsubsection{Ten Low Level Adventures in Pandemonium}

You lean over the counter to the waitress, not because she's so beautiful, but because you can barely hear her over the din. Honest. You're pretty sure she said\ldots

\listone
	\item WE DON'T SERVE YOUR KIND HERE. THE MILLER ONLY SENDS US BASALT FLOUR NOW.
	\item WE GOT AN ORDER OF APRICOTS IN THIS WEEK, THE CRAZ NAKED MAN CLAIMS TO MAKE IT HIMSELF.
	\item THE TUNNELS ON THE WEST SIDE, NO ONE COMES BACK FROM THOSE. NOT EVEN THOSE NICE MEN FROM LAST MONTH WITH ALL THE WEAPONS.
	\item IF KELLIGAN SEES YOU LEANING ON ME LIKE THIS, HE'LL KILL US BOTH.
	\item THERE WAS A MAN LOOKING FOR YOU. HE SAID HE OWED YOU MONEY.
	\item DO I KNOW YOU? AFTER THE WATER TURNED BLACK, I'VE HAD TO ASK EVERYONE THAT.
	\item I HAVE THE CURSE. YOU SHOULDN'T STAND SO CLOSE.
	\item I CAN'T FEEL MY MIND. STOP TAUNTING ME!
	\item THE BEER IS FREE TODAY. IT'S A LONG STORY.
	\item DON'T UNCOVER THOSE LIGHTS! THERE'S A WIGHT IN THE BUILDING.
\end{list}

\subsubsection{Ten Mid Level Adventures in Pandemonium}

You've found the sage you were looking for, but it looks like he's dead. His corpse is torn apart and lying on a heap against the part of the floor that's the ceiling to you. Droplets of congealing blood rotate slowly in the la grange points between ceiling and floor. He's got a piece of parchment in his cold hands, and it says\ldots

\listone
	\item wights have found me kill me kill me kill me
	\item I think this corpse will fool the howlers. At least for a while. If you wanted some water it's become more dangerous.
	\item NWNENWWS
	\item This man is an example. If Hruggek's Ninja Temple requests taxes, pay them.
	\item It's written in an old Orcish tongue. You'll have to find an Orc slain on the Prime at least a thousand years ago.
	\item The man's name is Gregor.
	\item Orcs! How I hate them! Their scimitars open the way!
	\item This is a ruse. The sage has escaped.
	\item Go back. Erythnul is not to be mocked.
	\item Itchy. Tasty.
\end{list}

\subsubsection{Ten High Level Adventures in Pandemonium}

The gates of the building have been torn asunder, as the characters run in, it seems that they're too late because\ldots

\listone
	\item Wights swarm over the insides, covering every piece of furniture with writhing limbs and moaning incessantly.
	\item Neogi great old masters hang from the ceiling, affixed by strands of hardened mucous.
	\item The pews stand empty as dust sweeps through the ancient church propelled by powerful winds.
	\item Hruggekian throwing stars are imbedded in virtually every wooden surface.
	\item A gaping planar rift hovers in the middle of the room, the winds of Pandemonium hurtling small objects into the void.
	\item The red dragon is already here, the hobgoblin princess is in his grasp.
	\item Black fires lick the insides of the room, the tomes are most likely destroyed!
	\item A tremendous serpent creeps over the tattered carpet.
	\item The winds howl even louder in here. Or maybe\ldots there are air elementals!
	\item A friendly and purring kitten is tossed back and forth by the terrible winds.
\end{list}


\subsection{High Adventure in\ldots Carceri!}

Point of fact: being in Carceri sucks. It's hard to leave, and it's an unpleasant place to be. That's the whole point. But believe it or not, those who please Nerull sufficiently are \textit{rewarded} with an eternity in Carceri. Now some of these people are just sadists -- creatures who enjoy the suffering of others so much that being able to assist in the degradation of others is payment and more for having to live in a hell dimension in the Prison Plane. But for others, life in Carceri is just genuinely pretty good. Some of these prison dimensions are minimum security white people jail -- there's a golf course and your ``guards'' are attractive women. It's still a prison of course, but if someone doesn't \textit{want} to leave, are they really a prisoner?

Anywhere you go in Carceri, it's all Evil, and people normally only go here if they are themselves Evil. That means that the people who are being punished here are being punished for \textit{failure}, not wickedness. The most successfully wicked individuals actually are rewarded here. Carceri can be a great place to introduce horrific elements into your story because by its nature anything that happens in Carceri, \textit{stays} in Carceri. Horrifying and depraved elements you introduce in a Carceri adventures don't have to apply to any subsequent adventures if you don't want them to.

\subsubsection{Campaign Seed: A Ring of Keys}

Carceri is a never ending parade of pocket dimensions filled with punishments and rewards that are both cruel and ironic. Travel between these cells is almost impossible, but there are ways. Most notably, there are maps that can tell you a secret path to get from one prison to the next; and there are adjustable rings that can transport a character directly from one prison to another depending upon how it is adjusted. Either can make for unlimited hours of enjoyment as players hop from one piece of episodic turmoil to the next. The maps work just like the map from \underline{Time Bandits}, and the rings work just like the devices from \underline{Sliders}. Really. Furthermore, those objects are authorized personnel \textit{only}, so if the players have one they are going to be hunted by Demodands with a new wacky scheme to catch them every adventure.

\subsubsection{Campaign Seed: Escape from Tartarus}

Just because you have been placed in a prison plane doesn't mean you deserved this punishment, or even that you committed a crime. The plane itself will punish impersonally, hiding its portals behind elaborate stages designed to elicit suffering.

Fight your way our of Tartarus, and no prison in any plane will every hold you\ldots

\subsubsection{Ten Low Level Adventures in Carceri}

You pass through the portal and find yourself in a new prison dimension. This one is\ldots

\listone
	\item Filled with thick, thorny foliage. Also it smells like boar and the thorns splinter and get into your armor.
	\item A town where the streets are filled with fighting.
	\item An expansive desert. Vultures fly overhead, but the scorpions seem unwilling to wait for you to die.
	\item A foul sewer. The water is waste deep. At least, you hope it's water.
	\item A scrubland with rusted iron spikes jutting out of the ground. Cages filled with starving madmen top some of the spikes, while other cages have long since fallen to the ground.
	\item A banquet hall stacked with delicious looking food. Haggard goblins look at the food with longing, but nothing seems to stand between them\ldots
	\item A windswept glacier. Far beneath you, there is a shadow in the ice. Far in the distance, a wolf howls.
	\item A stark stone room, where light filters oddly through a great number of spider webs and a dusty stained glass window.
	\item An earthy sinkhole. Worms poke through the topsoil everywhere around you, their eyeless heads wriggling like mad.
	\item A garden maze under an overcast sky. Fantastic shapes are cut into the hedges, and some ever seem to watch you.
\end{list}

\subsubsection{Ten Mid Level Adventures in Carceri}

If you could figure out the secret of this prison, you could escape\ldots

\listone
	\item The labyrinth seems to have four spatial dimensions\ldots
	\item The land shakes with earthquakes, but they still try to build houses.
	\item That eagle keeps eating that guy's entrails\ldots hey wait, I have entrails\ldots
	\item Why does that sanitarium seem to be inside-out?
	\item Why does everyone here wear a mask?
	\item Criminals in this put themselves into prison cells?
	\item The ghosts don't die when we kill them, and if we can't kill them we can't leave this building.
	\item It looks like a brothel, but who are the petitioners? The clients or the girls?
	\item The portal has a gold lock on it, and I was sure I saw a glint of gold in one of those oozes.
	\item An endless desert of white sand\ldots Or is it bone dust? 
\end{list}

\subsubsection{Ten High Level Adventures in Carceri}

If you just had it, then you'd be free\ldots

\listone
	\item A ship of chaos passes this way every day at the same time. If I could only make it notice me\ldots
	\item I almost have enough money to bribe the demodands into releasing me.
	\item That demon is a master of planar magic, and its said that his enemies get tossed to other planes.
	\item The fiends involved in the Blood War come from other planes. If I had an army large enough to impress them, they might show me a way out.
	\item If I could remember my home, I could just cast a spell and go home.
	\item The sage knows a way out, but he's so crazy that he'll only tell the secret to someone he considers a peer. What do I have to learn to do that?
	\item I can't believe that she's here. Do you think she'll forgive me?
	\item That war machine that looks like a bug the size of a mountain\ldots I hear its powered by a portal to the Astral Plane.
	\item I could open this portal, but I need the Blessing of Nerull.
	\item A wizard has been traveling Carceri for rare components, and it's said that he has access to plane-hopping effects.
\end{list}


\subsection{High Adventure in\ldots Hell!}

The Infernal Realm of Baator is essentially 9 infinitely large regions that happen to have a big pit that acts as a portal to the other 8 somewhere in them. So while the gods (and official publications) spend a lot of time worrying about that big pit in the middle, the fact is that the vast majority \textit{of Baatorian residents} don't even know it exists. Near epic play will spend an inordinate amount of time worrying about the goings-on around The Pit, and send the heroes off to go siege the fortresses around the ledge and such, but for the rest of your character's life the Nine Hells of Baator are just some inhospitable terrain filled with level-appropriate monsters.

\subsubsection{Campaign Seed: A Kafkaesque Nightmare}

Baator is home to one of the multiverse's most pervasive, efficient, and \textit{evil} bureaucracies. They don't lose your documents, they don't forget to mail things to you when they said they were going to, they simply have a set of rules that is at once awe-inspiringly complex and actually \textit{designed} to cause suffering to those who need to use its services. A campaign set around the backdrop of filling out forms sounds about as entertaining as doing your taxes in Hell, but there's ample opportunity for comedy, horror, and adventure in such a scenario, as well as ample prospect for character growth. The action starts when the characters need to change their registered employment, or want to protest their home getting knocked over to build a throughway, or perform some other completely mundane bureaucratic task. Unfortunately, the form they need to begin this process is clearly on display downstairs in the room marked ``Beware of Leopard''.

Surfing bureaucracy in Baator is about the only place where that makes for exciting D\&D adventures. The challenges to be overcome are social, mental, physical, and magical and efficient bureaucrats will tell you \textit{exactly what you need to do} to get things accomplished. This isn't like a Kafkaesque Nightmare on Earth, where you'll get stonewalled or your papers will just get lost, this is completely efficient and functional -- but designed by super geniuses to make your character uncomfortable. At lower levels there's a fiendish leopard in the room with the papers you need. At higher levels there's a golem that's supposed to stop people from entering the office where you need to convince a Gelugon to stamp your form. As the characters push their way to the top, they will find themselves in the position of being able to create their own red tape\ldots

On a side note: I just want to point out that my spell-checker recognizes ``Kafkaesque'' as a word. Sweet.

\subsubsection{Campaign Seed: Law of the West}

The great cities of Baator are infinitely far away from some of the nether regions of the plane. But the Law (and the Evil) still needs to be maintained. If you get far enough out into the boonies, Pit Fiends and the like just can't be bothered to show up and solve problems. So when Chaos (or Good) comes in to assault a frontier town, it falls to hard boiled individuals like the Player Characters to set things right. There's a new sheriff in town, and he's got levels in a PC class. This is your chance to use all your Western clichéin a fantasy setting, when you can turn Cowboy Movies into Kurosawa flicks.

Once the players beat back the gnolls who have come in at the behest of hyena ranchers trying to drive the gloom farmers off the land, the place is going to be a nicer place and attract Ogre Duelists or dishonest bankers. When it becomes known that the portal nexus is coming through, suddenly all that property is going to shoot up in value. And suddenly the pit fiends \textit{do} care what goes on in your sleepy neck of the woods.

\subsubsection{Ten Low Level Adventures in Hell}

It's a dusty little town, like an infinite number of others just like it both functionally and aesthetically. You don't know what makes this town special, and with the number of horrors you've seen on the plains -- you're not sure you want to. Still, this is a place it doesn't pay to break the rules when it isn't important, so the first thing you to is walk in through the curtain they hung up on the door to the Town Hall. Inside you see\ldots

\listone
	\item A dried out sahuagin sits behind the desk. He's mumbling about how the water is all gone.
	\item An officious imp attempts to shoo you right back out the door.
	\item Five corpses in fancy clothes lay strewn about the entrance hall.
	\item Putrid husks of humans in cages hang from the ceiling while a ghoul repeatedly jumps up trying to get at the rotting morsels.
	\item A mountain of papers covers the desk. From somewhere behind them a voice tells you that it is busy.
	\item A hobgoblin sits with his feet on the desk. As you enter, he stands up smartly and asks your business.
	\item Long lines of petitioners block off any hope of registering an time soon.
	\item Zombies shamble around the insides of the building and an imp is attempting to complete its paperwork while flying around the ceiling.
	\item The floor has collapsed entirely
	\item The front counter has been smashed and the interior smells like hyena urine.
\end{list}

\subsubsection{Ten Mid Level Adventures in Hell}

At last! You stand before the magistrate, it seems like you've been waiting for an eternity. You state your case, and he tells you\ldots

\listone
	\item ``You have the choice of death by platricorn or death by fire. Choose!''
	\item ``I grant you writ of ownership of Gelzugh's Tavern. You have the full backing of Hell in taking control of it from Gelzugh. Way back.''
	\item ``Your circlet is not \textit{jade}, it's \textit{malachite}, which is totally different. You're going to have to go back into the mines and find a \textit{jade} circlet.''
	\item ``Every one of you are sentenced to clean the sewers of Leng of the crawlers or die in the attempt.''
	\item ``It is Tuesday, so you're going to have to travel to Chitterport to have this taken care of.''
	\item ``Actually, this contract looks legitimate to me; Baelphor is legally the child's father.''
	\item ``I find nothing in this documentation to lead me to believe that these passports have been stamped correctly. Deport everyone.''
	\item ``You can't be serious. These swords aren't even magical.''
	\item ``Foolhardy mortals! You have wasted my valuable time and now I shall waste yours!''
	\item ``Raelzella's marriage is now void, the ownership of the larvae will be decided by combat.''
\end{list}

\subsubsection{Ten High Level Adventures in Hell}

Sorting through the ancient paperwork in the forgotten tower, you've found\ldots

\listone
	\item Documentation that proves that you personally are descended from an Erinyes.
	\item A small plush doll of a petrified Pit Fiend. It appears to be a \spell{shrunk item}.
	\item Spellbooks belonging to an evil lich.
	\item A map of a mighty fortress that appears to have stood where the shard spires stand now.
	\item Proof that a powerful Gelugon is not entitled to his position.
	\item A recipe for a dish now famous throughout the plane.
	\item Tongues of an ancient beast in a box. When the box is opened, the tongues speak of a fortress filled with giants.
	\item A portal to a deeper Hell in between the pages of a book.
	\item Poetry thought lost for a thousand years.
	\item Prophecies that mention you by name.
\end{list}


\subsection{High Adventure in\ldots The Abyss!}

The Abyss is well known for being infinitely big and infinitely bad in all directions, and it is. If there is some hellscape in your nightmares, its probably somewhere in the Abyss and there is someone there waiting to hurt you. The only thing it has going for it is that its very unorganized, meaning that the endless evil is only rarely directed enough the threaten other planes and planar oasis tend to places of great turmoil, meaning that small groups can easily blend in and ingratiated themselves amid the variety of beings that call these planes home.

Unlike other planes, there is no ``standard'' Abyssal Plane, aside from the top level called the Plane of Infinite Portals. These planes may be set up like a deck of cards, but they only share the chaos and evil traits, any particular plane can have any elemental or magic traits in the book and have geography ranging from the mundane mountains, forests, and plains to fantastic locations harmful to all but the most exotic forms of life. The only thing one can depend on is that pits and holes in the Abyss are often planar portals, and they only lead to deeper and wilder layers of the Abyss. Climbing back out of the Abyss is a much more difficult task, one that requires knowledge of planar pathways like the River Styx or powerful magic.

\subsubsection{Campaign Seed: We're the Exotic Products Trading Company (Abyssal Branch!)}

``We are here to serve your needs, and we offer a range of services ranging from capture of exotic lifeforms to collection of unique minerals and lore! We even have an on-call Search and Recovery Team available to recover lost individuals, `bargain' with demon governments, or protect important trade shipments! Contact one of our offices in Sigil or our home office on the Plane of Infinite Portals!''

\subsubsection{Campaign Seed: Pirates of the River Styx!}

``Yo ho, me hearties! The River Styx be vast and mysterious and its waters kiss the Abyssal planes like a cheating lover! Why set sail in the other Lower Planes when the Abyss is infinite and lawlessness is a virtue of its people? The good boat The Groping Marilith has room for any brave soul whose handy with steel or spell and has an eye for exotic and demonic beauties in every port and magic and jewels hidden in the nether regions of every fiend. Come ply the Abyss with us, and forget your troubles on the River Styx!''

\subsubsection{Ten Low Level Adventures in The Abyss}

\listone
	\item Food Run! Demon weevils have infected an Abyssal Town on the river Styx, and the first group to bring untainted food for them will earn a valuable ally.
	\item Race! A Nalfeshness ruler of miles-long city straddling the River Styx on the 33th level of the Abyys has decided to host a riverboat race to please his unruly people. There's big money to be made in this no holds barred sailing race through an Abyssal city!
	\item The good ship Lollyjaws is plying the River Styx with its zombie crew, and they've decided that you've hit a big score and you need help ``investing'' it.
	\item Message in a bottle. A map written in Celestial has been found in a blottle on the River Styx. Its this a map to a treasure, some poor soul's hope for rescue, or a clever trap to capture well equipped adventure seekers?
	\item Run aground! A chaos ship containing mysterious spices and drugs and run aground near a port town, and its bedlam as psychotropic clouds spew forth to wreak chaos in the town. Loot the vessel before the helplessness of the townsfolk attracts powerful fiends who'll sweep up the any booty.
	\item A dark, beautiful, and mysterious stranger decides that only your organization can retrieve a packet of information from the 411th plane.
	\item Mapquest! Map a planar route to an exotic locale in the Abyss, and return to collect your reward.
	\item ``There's an emergency! Deliver this call for help to the 911th plane!''
	\item Worm farmer! Travel to Noisesome Vale on the 489th layer of the Abyss and collect samples of the worms that eat sulfer gas and exhale breathable air for a Fiendish Gnome client with ideas for a Styxian submarine.
	\item An erratic portal between the 1st and 239th planar has started functioning properly again, and the Lost of City of Azzabanazanazan has been found (much to the inhabitants surprise). A little clever negotiating between this city and a few of the more popular demon cities could mean big profit.
\end{list}

\subsubsection{Ten Mid Level Adventures in The Abyss}

\listone
	\item Naval vessels of the Nine Hells have made serious incursions along the river Styx, and a clever ``privateer'' can make a little coin by signing up with a demon lord to resist these salty devils.
	\item Smiley Tom, the infamous Incubus captain of the legendary Slippery Cat has been imprisoned in Graz'zt realm for unknown crimes. Rescue him to gain his legendary gratitude, or use this opportunity to steal the Slippery Cat, the greatest ship to ever sail the River Styx.
	\item The Forgetful Fog Technique. Some clever pirate has figured out a way to create fog on the waters of the River Styx, then push these vapors onto towns and cities, looting them silly while the inhabitants are blissfully unaware. Catch these clever thieves to stop their amnesiac attacks, or perhaps gain a monopoly on this tactic yourself.
	\item One of your mates have finally bedded one lass too many\ldots she's been granted a wish by a glabrezu, and ill-luck follows your mate and his friends(which is unfortunately you). Win her affections back or find her a new romance in the Abyss, or else the curse will be the end of you.
	\item Ever hear of the sea elves living in a city hidden under the River Styx on the 356th plane? Their touch steal memories and they sell them on the demonic market and\ldots what was I saying? Hey, who are you? Who am I?
	\item A lazy balor chief running the glorious demon city of Belzasharazar on the 45th layer wants a new pleasure palace constructed, but his succubus consort has other ideas. Burn the construction often enough and he'll lose interest, and you'll earn a powerful patron in the demon city.
	\item The latest fad in Sigil is the practice of keeping glowing dragonflies as party lighting, but these exotic insects are found only on the 232nd layer of the Abyss, a plane suddenly caught in a vicious conflict between two barely-known demon lords. Deliver a shipment of these blinky bugs to Sigil and you'll be invited to all the best parties, opening up other pecuniary possibilities.
	\item You've been approached by a cabal of wizard from the Prime, and they want information on the Black Tower. Infiltrate the Black Tower to steal their secrets, or turn sides and lead a strike force to the Prime to nip these nosy wizards in the bud.
	\item A cargo box shows up on your door with a valuable, but difficult-to-sell and dangerous product (like a shipment of souls), and several parties seem to think that you are the owner. Find a way to sell the cargo to a more powerful individual or else these parties will take it from you with extreme prejudice.
	\item An old associate has deeded you a confectionary in the City of Brass that specializes in demon chocolates and sweets. The Sultan has decreed that if you don't pay back taxes in city of Brass currency that he'll foreclose on the property (and your soul). Go on a whirlwind tour of the Abyss to collect enough stock to make enough quick cash to save the shop (and your hide).
\end{list}

\subsubsection{Ten High Level Adventures in The Abyss}

\listone
	\item Over a dozen pirate ships working the River Styx have been declaring that they are part of an Armada in order to pass along blame, and they are saying that you are the Admiral! Find and smash these lying upstarts or ``gently convince'' them to actually accept your command.
	\item A general in the Blood War has found a way to divert the River Styx and he is using these pathways to strike key demon and devil armies, killing both his enemies and competitors. Both sides are willing to handsomely reward the party capable of ending this maritime terrorism.
	\item Rumors and hints point to a powerful artifact being transported along the River Styx in a vessel of unusual design, and factions vie to be one to seize this powerful item.
	\item An island has appeared in a notoriously wide section of the River Styx, and dragons have been leaving the island to raid vessels. By your estimation, they should have amassed a horde that is fantastically large, even by the standards of dragons.
	\item The Mask of the Captain has resurfaced, a powerful artifact that creates and closes permanent gateways between the River Styx and the Prime Material Plane, and a powerful Prime nation has decided that they will increase the wealth of their people by plundering the cities of the Abyss.
	\item A trading vessel of unusual design flies into the Abyss, avoiding known planar pathways. It is crewed by a race that planar sages have never seen, and they offer trade goods of exotic and powerful design. Is this a simple trade mission, or an incursion from another plane by a new planar power?
	\item Orcus's agents have begun purchasing magic items related to planar travel, hinting at an invasion of enormous proportions.
	\item A demon lord of waning power has declared that his power and command over his layer of the Abyss will pass onto the individual to defeat him in single combat, and contestants have gathered at his fortress. Is this a ruse to gather the equipment and souls of powerful individuals, or is he truly offering a chance at the title of demon lord?
	\item An old friend brings news of the discovery of an empty city found in perfect condition in the Abyss full of trade goods and magic, but without a single living or undead soul. To take control of this city is to learn its secrets, and possibly gain its enemies\ldots enemies unconcerned with wealth or magical power.
	\item Yeenoghu has decided that you are a demon lord in disguise who is pretending at weakness as a ruse, and he is sparing no cost to send agents to test this theory. Convince him that you are a mortal, or strike him so hard that he ceases his attacks.
\end{list}


\subsection{High Adventure in\ldots Gehenna!}

First, it's the home of the Yugoloths. These outsiders are the dealmakers and compromisers of the fiendish world, so they might be involved in any plot or any scheme that makes its way across the planes. The land itself is series of volcanic mountains where sentients have forced their own existence into, jammed between the Hades and Hell and connect to the River Styx, so it is well situated between several of the Lower Planes. The works of mortals and immortals alike are eventually destroyed by tremors in this architect's nightmare of a plane and only the works of the gods last here. That being said, the entire plane has an angle that ranges from inconvenient (45 degrees) to unlivable (straight up), meaning life in Gehenna is far more socially dependant than other Lower Planes due to the fact that the only place to live is in the cubbies, caves, boltholes and settlements that litter this plane. It's not that you can't live in on the slopes and are forced to cooperate and co-oexist and you are forced to compete for space like in Hades, its just that life in Gehenna without a clique \textit{sucks}.

What do all of these things mean? It means that Gehenna is a realm for movers and shakers, a place where ``the deal'' and ``the juice'' matters more than any ideals or hopes. Even the petitioners of this plane are only concerned with power, and only the cruel nature of this plane keeps them chained here. Brinksmanship and counting coup and favors are the symbols of power here, and mere physical might or magical power take a backseat to one's ability to \textit{manipulate people with physical power and magical might}.

\subsubsection{Campaign Seed: The Yugoloths Want You!}

While Tanar'ri generals are known the power and might of their hordes and Baatezu armies are know for their frightening disciple and efficiency, it is the Yugoloth forces that are know for their subtlety and tactical elegance. They don't fight for reputation or honor; they fight to fulfill a contract and make a profit, making them among the deadliest generals in the Lower Planes.

You've joined that organization now, and the Yugoloths have need for elite squads of problem-solvers with a propensity for violence and a capability for discretion.

\subsubsection{Campaign Seed: The Grand Game in the Crawling City}

In the Crawling City, you've got to be useful or you're dead. You attached yourself to a minor Yugoloth noble, and he's begun using you as behind the scenes agents in the Lower Planar courts. With skill and nerve, one day you might earn the fear and respect of the fiends and become a player in your own right.

\subsubsection{Ten Low Level Adventures in Gehenna}

\listone
	\item A famous Yugoloth tactician is taking new students, and he's set a distinctly fiendish entry requirement: interested students publicly apply, and one week later the first to present themselves is accepted. The last time he took new students, no applicant ended the week alive enough to show up\ldots
	\item Small bands of petitioners have been gathering under the banner of a charismatic profit and raiding minor settlements in the night. Eliminate the threat by assassination or counterattack.
	\item Tremors! Minor rumblings and a trusted fiendish seer predict a major lava spout in a small settlement, destroying it, and several interestied parties want to loot it or the refugees in the final hour. Intercept these rogues, or plunder the settlement for yourselves.
	\item A minor Baatezu noble has been spotted in the Crawling City, and it's suspected that he's trying to hire away an elite group of Baatezu mercenaries when their current contract expires. Find and interrogate him, and the Yugoloths will repay this little favor. Whether he returns to his home plane with his life and valuables is your own business.
	\item The Double ``H'' Run. Despite the Blood War, some trade does exist between the Baatezu and the certain Tanar'ri, and the Yugoloths have their hand in it. Escort a package between the Nine Hells and Hades, avoiding agents from both fiendish factions who would use it to discredit their countrymen.
	\item The Masked Ball is next week, and a clever soul capable of learning the identities of several indiscrete parties can earn a few coins with the information brokers of Gehenna.
	\item A tiefling fop of a swordsman has defeated several prominent Yugoloth blademasters in mostly fair duels, despite his obvious lack of skill. Several persons of note would like to know his secret, and would pay even more to have that secret removed at an opportune moment.
	\item A mortal Sorceress of rare skill and infamous carnal desires has come to Crawling City, and entities of power are jostling to be known as one of her clients. Secure her cooperation for a client and win wealth; secure it for yourselves and win power and danger.
	\item A Tanar'ri of an unusually Lawful bent has entered the service of a Yugoloth of middling power. Discover the secret of his service, and that service can be passed on to a more worthy fiend, or kept as secret weapon for yourself.
	\item A Yugoloth of some influence has secured the services of an unusual household staff of famous, though powerless, Prime mortals. Spoil his coup by tempting, tricking, or intimidating these mortals into committing terrible blunders during the next power meeting, and you can harvest some amount of his influence.
\end{list}

\subsubsection{Ten Mid Level Adventures in Gehenna}

\listone
	\item A powerful Tanar'ri fortress has been bidded for destruction, and the Yugoloths will pay well for the group that finds an exploitable weakness.
	\item Several subcommanders have been bickering over the right to extract a powerful dragon of a military bent from Carceri, and rewards will fall upon anyone capable of securing this beast's services for the Yugoloth.
	\item A key planar touchstone in Hades will prove the key to an isolated fraction of the Blood War, insuring victory for one side or the other. Destroy this site, or profits for the Yugoloth in this conflict will fall dramatically. Secure it for yourself and turn it against both armies to secure a stalemate, and some fraction of the increased profits will fall your way.
	\item A powerful Yugoloth well- known for patronizing up-and-coming allies has declared that you are his protege, making you a target for his enemies Punish these enemies, and you might secure his patronage for real.
	\item A small army in the Blood War has wandered into Gehenna and is a threat to the Yugoloths. Destroy its leadership and loot its paymaster, and the Yugoloths will see that you are amply rewarded.
	\item A band of thieves have turned the Crawling City upside-down. Recover and return the valuable objects and win influence. Hold the objects hostage for future favors, and gain power that money can't buy.
	\item An unknown magical effect has stopped the feet of the Crawling City, and the first to discover the cause will win no small amount of gratitude from the ultraloth ruler of the city
	\item A series of businesses across Gehenna have been vandalized, an obvious turf war between two competing interests, and the first group to discover the identity of either player can earn a contract to accelerate or reverse the destruction.
	\item A spellbook of unique magics useful to a courtly mage has been found, and the owner of such magics would pay handsomely to not have his secrets revealed.
	\item A Baatezu diplomat has come to Crawling City, and he has decided that you will become his agent. Avoid a diplomatic incident without betraying the Yugoloths, and the powers that be may reward your ability to resolve such a conflict.
\end{list}

\subsubsection{Ten High Level Adventures in Gehenna}

\listone
	\item A cabal of liches have a sudden need for several rare components, and they are willing to trade battlefield magic for the first party to collect their list.
	\item It has come to your attention that several key subcommanders are plotting a coup over the control of the Crawling City. Shatte this conspiracy, or risj all and become its ringleader.
	\item The Yugoloths are looking to subcontract a dangerous mission on the prime against a noble house of demon-hunters. Get the contract and eliminate the hunters, or accept a greater bribe from the them to hold the contract long enough for them to counterattack.
	\item Key contracts for the Blood War have been stolen, and the first person to recover them will control a Yugoloth army of immense proportions.
	\item A war machine of great size and terrible power has been spotted in Mechanus, and such a device would fetch a king's ransom in the war markets of the Crawling City.
	\item A clique of fiendish spellcasters has set a challenge: the first entity to scour the planes for a specific but almost unique spell will earn a tome of their greatest spells. They expect one of their members to win and then resolve a dispute about claims of leadership of the clique, but an indiscrete servant blabbed the rules of the contract and now several interests seek to win the contest.
	\item A mortal noble of rare talents has entered the Crawling City and is recruiting agents for one goal: recover the contract that dooms his soul to property after death. To help him is to defy Yugoloth tradition, but the rewards might just be right.
	\item For some unknown reason, Inevitables stalk the Crawling City, and a clever stagemen might just be able to divert them towards one's enemies.
	\item The ruler of the Crawling city is missing, and chaos rules as several factions make a bid for power.
	\item Negative energy has begun to permeate the Crawling City and undead powerful enough to challenge of Yugoloth leadership have begun to rise. Is this an attack by a god whose portfolio is death, or some ruse to put the Yugoloth against an enemy they cannot defeat.
\end{list}


\subsection{High Adventure in\ldots Hades!}

One would think that Hades is among the worst Lower Planes to adventure in\ldots and they'd be right. The plane itself has the two nasty qualities: it poisons you with the Grays until you become a depressed Goth, and the Entrapping trait takes your memories and makes you want to never leave like a bad house guest. That being said, adventure is still possible, even for the least powerful adventurer.

It works like this: think of Hades as an unforgiving desert. Travel in this ``desert'' is only done by moving from oasis to oasis. These oases are towns and settlements that are built in such a way to resist the Grays and the Entrapping trait (see the Handbook of the Planes for an example of such a place). The only things that permanently live in the desert are creatures who are both immune to the Entrapping trait (like outsiders) or who have already succumbed to it (which has no other game effect other than ``become an NPC who doesn't want to leave''); these creatures also have some way of dealing with the Grays, and so they are creatures with SR 10 or better or are immune to Wis damage (like undead). This generally means that the ``desert'' that is Hades is filled with wild-eyed hermits and bandits and other forlorn spirits (which might be actual undead) living in the blasted and ruined geography of Hades, or creatures of some special power who skirt the edges of civilizations. Some NPCs you meet might just be Entrapped, but enter an oasis once in a while to recover from the Grays; other such characters might have ways to cure the Wis damage that the Grays cause, thus they are entrapped by Hades, but have no reason to enter an oasis, and some powerful creatures can resist The Grays almost indefinitely due to their high Saves.

Hades also has a few other features of note: It's the ultimate source of Evil of all types, and all of the evil outsiders are equally (un)welcome there. You could easily see a Yugoloth, a Devil, or a Demon without that being part of a plot device. Since Hades is the creation place for larva, the serving-sized petitioner souls of very evil people, the big evils of the multiverse have taken to fighting and brokering for this natural resource full time, and it all starts here. Night Hags and Liches are other players in this economy, but they are the freelancers in the publication of evil.

\subsubsection{Campaign Seed: The End of Oasis}

You've lived in the town all your life, and you know that only madmen and the 'loths live beyond the walls, but now you must travel the wasted plains to find your destiny.

\subsubsection{Campaign Seed: A World At War}

The Blood War wages endlessly and pointlessly across the Gray Wastes, with most territory never held or even claimed. The only things that have value in this whole plane are the occasional portal, oasis, or larva vein. Every other patch of land is a liability and \textit{no one} wants it.

\subsubsection{Ten Low Level Adventures in Hades}

\listone
	\item A Yugoloth has died while on a trading mission to your town, leaving behind a shipment of larva. To prevent your town from falling under the 'Loths gaze, you must take them to the nearest Yugoloth city for sale.
	\item A battle in the Blood War was fought near your town, and the undead fodder from that battle now terrorize the countryside.
	\item The leader of your town wants it to become a waypoint for message delivery, and he hires you to delivery the first messages.
	\item Something has been coming in from the wilderness to stalk the townsfolk. Will you track it back to its lair outside of town?
	\item The well has been poisoned, and you must find a new source of water for the town deep underground, far from the protective influence of your home.
	\item A terrible new disease has been ravaging all the nearby towns, and the Oinoloth has decreed that the town with the best gift will be spared.
	\item Devil agents want to construct a supply depot far from their own infernal realm, and will pay well for the location of new oasis(minus any current inhabitants).
	\item The nearest town has its eye on the riches of your town, and now has agents and a small force scouting for weak points and key personality to kidnap.
	\item Two caravans have entered your town at the same time, and now they have begun attacking and sabotaging each other at night in an effort to be the only one to leave.
	\item It's Election Day! Factions in town work against each other in an effort to become the new Mayor, and everyone knows that the loser will end up exiled to the wastes.
\end{list}

\subsubsection{Ten Mid Level Adventures in Hades}

\listone
	\item For some, mere death is not a real revenge. A powerful leader hires the party to defend a prison built in order to entrap entities in Hades in a spot unprotected from the effects of the plane.
	\item A legion of elemental soldiers have been led through a Gate, and they have succumbed to the effects of the plane. The first town leader to convince them to join him will gain a powerful fighting force.
	\item The Yugoloths have decided to annex your township, and only a show of overwhelming force or a high bribe will convince them to leave your town alone.
	\item Something is destroying oasis after oasis, isolating your town from the trade paths.
	\item A Gate has been opened to Celestia, and celestials have offered asylum to your township. Is this an opportunity to evacuate your town, or is this a fiendish trick to destroy your town?
	\item during a battle in an unfamiliar oasis, your party is trasnported to an unknown location in Hades, far from any oasis. Can you find your way home, or even to a safe location before you succumb to the planes traits.
	\item A series of Gates have opened up to a distant region in Hades, and townships now vie to control the altered landscape.
	\item The river Styx is flooding, and threatens to wipe out several cities built on its waters, including your town's primary trade partner.
	\item A caravan of bioloths has entered your town, beginning a carnival that threatens to enslave everyone.
	\item A powerful Yugoloth has been working against the Oinoloth, and your town is caught in the cross-fire. Will you work against it, or for it?
\end{list}

\subsubsection{Ten High Level Adventures in Hades}

\listone
	\item Rumor hint that your town holds a mystical font that can make anyone bathing in its waters immune to Entrapping and the Grays, and several powerful forces vie to control this wonder.
	\item The Blood War has boiled up in your region, and a clever party could benefit from working with one side or the other, or even both.
	\item A powerful devil decides that he needs more exotic troops, and he is willing to extend his protection to your town if you can capture powerful creatures from several legendary parts of Hades.
	\item Angels have gained a foothold into Hades, and have decided that your town is the first to be ``purified.''
	\item During a particularly brutal battle in the Blood War, a powerful artifact has been lost. The first to regain such an artifact might be a threat to even the Yugoloths.
	\item A cabal of Night Hag Sorcerers have decided to harvest your town, and the only way to catch them is to breach the barrier between your plane and theirs.
	\item A powerful outsider offers his services to your town, saying that he can create planar gates. Such a resource would transform your town into a planar metropolis, but can it survive the attention it will attract?
	\item A powerful Warlord has taken over rulership of several towns, attempting to build an empire in Hades, and your town must either gather the forces of the surrounding towns to fight this menace, or usurp rulership for yourselves.
	\item A dangerous wizard has found a way to concentrate the evil of the plane, and he is using this evil as weapon that can corrupt even the Yugoloths to his person brand of evil.
	\item Strange and terrible diseases are taking their roll on all the inhabitants of Hades, and the only way to stop these plagues is to assume the mantle of the Oinoloth.
\end{list}



\end{document}