\noindent{The Inner Planes are mostly made of more-or-less pure elemental matter, and many of them are composed of elemental matter that is incredibly hostile to Material beings. Adventuring in the Plane of Fire, Negative Energy Plane, or Positive Energy Plane is just outright infeasible for living things from the Prime without magical protection, and the Planes of Air and Water require you to be able to breathe the local matter for much of their bulk. However, to a large extent that doesn't matter, since much of the Inner Planes is profoundly dull even for the natives. The really interesting locales are where the planar traits are different from the plane at large. These places are called ``planar bubbles'' in existing literature, and they usually look like giant blobs of some sort of other material floating in the otherwise uniform matter of the plane. This includes Material Plane bubbles, which are essentially like a normal Prime world aside from any encroaching traits from the rest of the plane, and are about as inhabitable by normal people as, say, volcanic areas or floodplains. They aren't particularly common in terms of \emph{density}, but since they're the actual places of interest and your players can actually survive there, there's good odds that they'll actually go to these places at some point during their stay on the Inner Planes.}

\section{High Adventure in\dots The Plane of Air!}

\subsection*{Campaign Seed: }

\subsection*{Campaign Seed: }

\section{High Adventure in\dots The Plane of Earth!}

\noindent\desc{The Elemental Plane of Earth is a lot like Pandemonium, except quieter and with heavier, monodirectional gravity.  There are a number of portals between the two, in fact, due to their similar environments, which provide a needed respite to people on both sides escaping the gravity on the Plane of Earth or the screaming on Pandemonium.  If you look at the entire plane as a group of dungeons that never open to the surface except for through portals, you wouldn't be wrong.}

\noindent{One important thing to remember on the Plane of Earth: spells with longer-than-long range rarely work \emph{at all}, since they almost invariably have to go through 40' of solid rock to get anywhere. So any long-distance travel or communication is going to be routed through another plane, and few groups have the resources to manage this, especially with any kind of consistency. So while the Dao control a Great Dismal Delve the size of a continent and a whole lot of smaller outposts -- they aren't actually at all unified, since they have to route their communications through the Material Plane, and \emph{plane shift}ing is incredibly inaccurate; it can take days to walk to where you're going after you arrive and takes an actual genie to manage it, plus there's the risk of interception. Also, because of this, places are more connected to the Sevenfold Mazework than they are to another colony ten miles away, since they're both equally unreachable, but the Mazework has the resources to send postgenies out occasionally.  So one outpost seriously doesn't care if you topple the next one over. The two could even fight an entire war before the Caliph hears about it.}

\subsection*{Campaign Seed: Slave Revolt}

\noindent{The Great Dismal Delve is run on slavery and is constantly eroded by earthquakes, elementals bent on collapsing it, and the like.  But it still has a number of still-standing abandoned areas, uninhabited or inhabited by squatters, where the Dao don't know all the paths.  It's into these paths that escaped slaves run. Since the Dao often advance by class levels and can call in favors from across the cosmos, a slave revolt or party of escaped slaves can easily have enemies through to the high levels, and naturally progress from running and hiding to an insurgency to building their own empire and toppling the Caliph. A slave revolt also provides an excellent incentive for a party of extremely different alignments to work together.}

\subsection*{Campaign Seed: Freeholds}

\noindent{The Elemental Plane of Earth is one of the most invaded elemental planes, because of its vast mineral wealth. Many of these mines are far from the mazes and mines of the Great Dismal Delve, and so have little need to care about the might of the genies being focused on them. Nearby genies are another matter, and their relations with freeholds can vary.  Some are in a state of constant war, while some even manage to trade with the genies.  Being hired as guards or negotiators at such a mine is an entirely likely PC occupation.  This gives PCs a rare window into how the monster tribes they find in dungeons actually interact with each other, since these freeholds are kinda exactly the same thing.  And it's not like the time between attacks is boring, either.  A freehold can be anything from the domain of a tyrant with its own associated plots, to an old west mining town crawling with prospectors looking for their big score, to a cursed necropolis held together by a single necromancer. Any one of these schemes is loaded with its own possibilities for town adventures -- even in times of relative peace.}

\subsection*{Ten Low-Level Adventures in The Plane of Earth}

The escaped slave tells you:

\listone
\item She found a silver vein in her flight, but the tunnel caved in behind her. If you'll feed her, she'll show you the way.
\item He's the vanguard of a large revolt. Give him the supplies they need, or they'll take them by force.
\item A cave-in destroyed a small palace a short distance away.
\item A new mine tunnel has breached your home, and there's a force of dao on the way.
\item She knows of a group of escaped slaves lost in a nearby tunnel complex who will be thankful for any rescue.
\item He's found a portal to another plane, but he can't tell which, in a cavern complex nearby.
\item She saw a group of fire mephit traders lost in some tunnels behind her.
\item Nothing, as he is stuck down by a curse as soon as his babbling becomes comprehensible. What was he stopped from saying?
\item The dao are beginning excavations of a millenia-old fossil vault.
\item There is a decree from the Ataman freeing the slaves of dao involved in a coup attempt. Infiltrating the palace could then have rich rewards.
\end{list}

\subsection*{Ten Mid-Level Adventures in The Plane of Earth}

The freeholder wizard's emissary stands before you.  His master has the map you need, and asks, in return:

\listone
\item You must keep the dao away from his gem mines until he gets the stone he needs.
\item That you assassinate her former business partner, a politically-connected dao, without pointing back to her.
\item His son has been enslaved by the dao. Free him.
\item His son has run off with a dao princess. Bring him back.
\item Her daughter has been leading raids against the dao. Make sure that the dao reprisal doesn't affect her mining.
\item Her daughter has been leading raids against the dao while disguised, and is rushing into an unwinnable ambush. Prevent her capture without letting on as to her identity.
\item That you retrieve a McGuffin from the Xorn Tunnels.
\item That you retrieve the heart of an elder earth elemental.
\item The next cavern complex over from his freehold has a rich gem seam, but also has stone giants. Relocate them so that he can begin mining.
\item The stone giants have agreed to serve her as mercenaries in exchange for a relic stolen by pandemonic bugbears. Procuring said relic falls to you.
\end{list}

\subsection*{Ten High-Level Adventures in The Plane of Earth}

The vanquished Hetman lies at your feet, and says:

\listone
\item The Caliph's tumens are already on their way.
\item His palace covered a vault containing an imprisoned magical being of immense power, cracked in the fighting.
\item The entire palace is rigged to collapse and warded against dimensional travel.
\item Her defeat was prophecied by a weird in a distant cavern complex, with knowledge of the fate of her destroyer.
\item He just said the command word to unbind a number of sleeping elementals.
\item Of her role in a plot for a coup against the Caliph, spoiled by her defeat.
\item Nothing you can understand but prayers to Erythnul.
\item That he will trade the location an ancient demonic artifact to be allowed to escape.
\item That her spies know who among your lieutenants you can trust better than you do.
\item A promise of vengeance, then is cut off by being pulled through some kind of calling spell.
\end{list}

\section{High Adventure in\ldots The Plane of Fire!}

\noindent\desc{More than any other Inner Plane, adventures in the Plane of Fire tend to take place in planar bubbles. If you can breathe water, the majority of the Plane of Water is basically just a lukewarm benthic zone, and it's the kind of place that Sahuagin might live without even realizing that they weren't on the Prime. But the archetypical expanse of the Plane of Fire is just, well, \emph{fire}. It's like the churning surface of a sun that extends in all directions for eternity. And while it is colder and less destructively melty than the all-consuming plasma of an actual star, it's still basically just an endless expanse of fluid, dangerous, \emph{useless} fire. Did I say useless? You bet, because heat engines actually work by heat \emph{difference}, so from the standpoint of residents of the Plane of Fire it is actually \emph{cold} that you use to run a power plant. The fire in between everything is just like the worthless emptiness of deep space except that it will also catch you on fire. Forget Carceri or the Gray Wastes -- the Elemental Plane of Fire is the worst place in the D\&D multiverse.}

\noindent\desc{But just because it's a horrible place, even the \emph{worst} place, doesn't mean that there isn't stuff you want there. And just because it is the most inhospitable place imaginable, doesn't mean that low level characters can't adventure there. The key is the planar bubbles exist. That is basically the only reason that anyone gives the Plane of Fire the time of day. The most important bubbles are Prime Bubbles. These are areas of land and sea with atmospheres, that happen to be shaped like a Ptolemic world -- a circle of land and sea with a hemisphere of atmosphere above. And of course, outside that is endless roiling fire. So the ground gets kind of rocky and parched, what with the sky being a never-ending holocaust without reason or respite - essentially it's like living in a Dragonforce video.}

\noindent\desc{Those Bubbles aren't just the only place your characters can survive, they are the only places that \emph{any} of the residents give a damn about. Remember that even if you happen to be a fire elemental, you still eat ``flammable'' materials if you want to grow any larger, and those only come from the ``cold'' spots. So not only is the practically usable terrain in the Plane of Fire very small compared to the plane's total volume, but the space between is inhospitable void. And not just inhospitable void -- it's \emph{opaque} inhospitable void. Standing on one of the floating islands, you can't even \emph{see} the other islands. When you look into the inferno you have no way of knowing whether the next place of value or substance is a few centimeters or a few parsecs of burning emptiness in any particular direction.}

\noindent\desc{So what does that mean for the low level adventurer? It means that practically speaking, no one \emph{expects} your character to want to go anywhere that would cause them to actually catch fire. No one else does, not even the planar residents who are \emph{actually made out of fire}. So it's totally workable as an adventure locale at any level. The Plane of Fire is run by the Efreet Sultans, and that gives the entire place a very fantasy-Arabic feel. Ignan, the approved lingua franca of the universe, is explicitly based on Arabic. That thing where Arabic calligraphy kind of looks like living flame? Yeah, they went there. While the Djinn have a \emph{presence} in the Plane of Air and the Dao have their own Caliphate in the Plane of Earth, the Sultan of Fire \emph{owns} the Plane of Fire. Because there is hardly any real estate, and finding or getting to it is in most cases a Wish Economy proposition.}

\noindent{The Plane of Fire is your chance not only to throw out every Arabian Nights clich\'{e} you know, it's also a place to throw in 1950s sci-fi left and right. Basically everywhere that anyone lives is one of those bubble colonies or asteroid mining facilities from the Heinlein juveniles. To get from one planetoid to another requires getting into a heat protected shell and then throwing yourself from one to the other. Once you leave a Planar Bubble, there's no gravity or wind, so it's basically \emph{exactly} like one of those personal space ships that were talked about in the old Republic Serials. Some of them are even saucer shaped.}

\subsection*{Campaign Seed: Conquest of the New World}

\noindent\desc{Even beings of pure fire cannot see far into the firmament, and so it is that new places of interest are ``discovered'' all the time in the most surprising of places. The iron ships that travel between bubbles need exacting angles of departure, because once they are off course, there's really no measurement you could take to figure that out (and often nothing you could do about it if you did). So a new island might well be just 1 degree off an established trade route. And once a new land is discovered, it's Columbian Conquest all over again. This new world may well have occupants that object to being ``discovered'' let alone colonized, but on the other hand they could seriously have fountains of youth or cities of gold.}

\noindent{Exploring a new Planar Bubble in The Plane of Fire is a good way to bring out any kind of D\&D adventure you want. The PCs have literally \emph{no} idea what they might find there, and there's a very great incentive to keep exploring since even \emph{wood} and \emph{water} are hugely valuable resources once you get off this gravity well and back to a more civilized one. You don't just get to loot the temples of stone using pyramids, you also get to confront their heathen demon gods, find relics of fallen ancient civilizations or the secrets of long forgotten wizards. A Planar Bubble that ``no one'' knew about on The Plane of Fire is about the safest place in the entire damn multiverse, so anyone who \emph{did} know about it could have stored or imprisoned, well, \emph{anything} there.}

\subsection*{Campaign Seed: Janissaries of the Fire Sultan}

\noindent{The Efreeti sultan is cruel, but he is not stupid. And he is well aware of the limitations of being a guy who is \emph{on fire} all the time when the only things in the entire universe that have any value are things that are \emph{not} on fire. And so it is that the Fire Sultan has children of non-flaming races raised in his employ. These children grow up to be \emph{Janissaries}: creatures who act as agents for the Efreeti and build their empire without incidentally burning it down. There is a lot of room for advancement in the Janissaries, the Sultan genuinely values your skills \emph{more} than he values the skills of the other Efreeti. First of all, there is basically no chance of you ever actually becoming Sultan (you just don't have the right fire in your blood), and secondly, unlike a real Efreet, you can do stuff that the Sultan cannot. There are a lot of politics that go in court, and the rest of the Efreeti have a tendency to rather \emph{resent} Janissaries; while at the same time doing their damnedest (literally) to avoid any direct confrontation with something the Sultan considers to be ``his.'' Do the Sultan proud, and you can have your every wish granted (as long as that wish doesn't include becoming Sultan or leaving the Sultan's employ). Fail him sufficiently, and he may allow the more jealous members of the court to take their frustrations out on you.}

\subsection*{Ten Low Level Adventures in The Plane of Fire}

You're getting the report from the overseer of the pipeline workers. The Kobold tells you that they aren't getting as much water as expected because....

\listone
\item A group of Firenewts has claimed that the pipeline runs through their tribal lands and have begun monkey wrenching.
\item The water reserves aren't as extensive as hoped near the surface, and the pipeline will have to be extended into the caverns.
\item Superstitious fears have broken out among the workers, they speak of burning snakes.
\item Drilling has broken through to inferno before expected, this rock isn't as stable as we'd hoped.
\item The water has some kind of creatures living in it. Creatures that live in water.
\item Some creatures have been bringing clouds of smoke with them when they crawl over the pipeline.
\item A rival mining group is siphoning water from our reserves.
\item Some guy who looked like a Yak has paid more than enough money for the land to get the crew to drill elsewhere.
\item Everyone who touches the water seems to forget what they were doing.
\end{list}

\subsection*{Ten Mid Level Adventures in The Plane of Fire}

Laughing, the Efreet relays the news. It's never a good thing when an Efreet is happy to tell you something, and this is no exception because...

\listone
\item Some group of xorn came in with a load of opals just two days ago. You're going to have to go farther afield if you want to liquidate those gems.
\item It seems that while you were out, they've made a new appointment of Sheriff.
\item The land title has been revoked and given to Hakim.
\item Surtyr wants his money back. Now.
\item Yak Men have taken over the entire city.
\item A Red Dragon has claimed the water reserves.
\item The Bubble has begun wobbling, the only way home is by wish.
\item The princess is in another palace.
\item The gnomes have themselves a Frost Salamander that they are keeping alive somehow, and mere flammables are virtually worthless here.
\item The great astrolabe has been shattered.
\end{list}

\subsection*{Ten High Level Adventures in The Plane of Fire}

The Iron Flask isn't completely inscrutable, and your research indicates that it contains...

\listone
\item One of the Sultan's uncles.
\item A potion of Immortality.
\item A gate to a deep layer of Baator.
\item The heart's blood of Baphomet.
\item The phylactery of a powerful Lich.
\item A decree from the previous Sultan.
\item A heretical Genie who was imprisoned for predictions that appear to have come true.
\item The crown of Pyriria.
\item The condensed gaseous form of a Chaos Roc. One of several, if the accompanying letter is to be believed.
\item The laughter of Queen Chandra.
\end{list}

\section{High Adventure in\dots The Plane of Water!}

%The Elemental Plane of Water is actually a lot like Air, in that you really can just play ordinary humanoids and get along fine. The difference, of course, is that if you want to be able to play outside of planar bubbles those humanoids need to breathe water. This is actually a good place to set a campaign off the Prime Material if your players want to be Merfolk, Locathah, or Sea Elves. The only problem with such a campaign is that most Aquatic creatures can't actually breathe air most other places -- but by that point you've probably accepted those restrictions already, and it's not much different than playing a non-aquatic character in an ocean campaign.

\subsection*{Campaign Seed: }

\subsection*{Campaign Seed: }

\subsection*{Ten Low Level Adventures in The Plane of Water}

\subsection*{Ten Mid Level Adventures in The Plane of Water}

\subsection*{Ten High Level Adventures in The Plane of Water}

\section{High Adventure in\dots The Positive Energy Plane!}

High Adventure in... The Positive Energy Plane!

Adventuring on the Positive Energy Plane, much like the Plane of Fire or Hades, is generally confined to planar bubbles. This isn't because the plane is innately inimical to life -- indeed, in a very real sense it's [i]composed[/i] of life and will pour itself into any living creatures on the plane, invigorating them and healing any wounds. The problem with spending time on the plane is that it doesn't actually have much air, and that the portions of the plane with the Major Positive-dominant trait don't actually [i]stop[/i] pouring energy into living things. Even after any wounds have healed and living beings are fully restored to health and vigor, the plane keeps pouring in healing energy until its "patients" explode.

So if your players are here, it's very likely that they either have some sort of special protection from positive energy, are sticking to the portions of the plane which only have the minor positive-dominant trait, or continually hack at themselves with their own attacks, and that they have some source of air - either from some sort of equipment or being in a planar bubble. These minor positive areas aren't actually harmful to life, and still heal damaged lifeforms.

Due to the nature of the plane, a common tactic for those traveling from one bubble to another is to continually maim oneself, taxing their bodies just enough that they can survive the occasional spike in energy levels. Natives of the plane who don't have protective magic can often be distinguished by "scars" of newly-regrown flesh. This is even more common among planar travelers, who may very well perform this scarification for hours on end due to actually traveling through major positive sections of the plane.

Most communities on the Positive Energy Plane are severely separated from each other, both because travel is actually dangerous and because it's hard to actually [i]see[/i] things on the Positive Energy Plane. Major positive areas are bright enough and emit enough light that seeing anything in their direction - let alone past them - is unlikely to work too well. And really, they make up most of the plane, so there's a good chance that there's going to be one in front of anything interesting. Other sensory inputs are amplified as well, so overall everything's a lot more loud, hot, and bright than you'd probably enjoy.

[b]Campaign Seed: Nomads of the Energy Storms[/b]

It's a fact of life on the Positive Energy Plane that if you're healthy and hit a major positive area or the energy levels spike, you're as good as dead. To avoid this happening, many societies live in places where they know of a few different "safe zones" and use divinations to tell them when to move. But since they also need to be sure that the area they want to migrate to is safe to move to, they [i]also[/i] employ scouts to periodically check on known bubbles and search for new ones. Sometimes these scouting parties encounter dumb monstrous threats they need to clear out, sometimes they encounter other settlements or squatters (possibly also monsters) which they need to either kill or make trade or land deals with.

[b]Campaign Seed: Graves of Steel[/b]

There is plenty of energy on the Positive Energy Plane, and to an inventor or golem-crafter that means free power sources. You can make a very serviceable engine just by using a Ravid's [i]animate objects[/i] power or by extracting the healing energies of the plane via a simple lifeform like oozes, and so plenty of inventors will come here to work on various projects that need plenty of energy. But the plane is a very bad place to stay put for long periods of time, so these inventors will sometimes abandon their work when a major positive trait decides to express itself. And so there are places where the Positive Energy Plane is littered with a lot of mostly intact mechanical devices just laying around.

Every so often, a wandering planar effect or a Ravid will wander into one of these graveyards, so it's also the case that a lot of these have machines that are either still running or recently reactivated. You can then toss in almost any "machines run amok" or "giant monster attack" plot you like; these things really do wander off and start terrorizing villages, and not much less often than anything happening on the plane.

Ten Low-Level Adventures in The Positive Energy Plane

[list]The nomad looks both in pain and gleeful at the same time. He walks over with a bundle on his shoulder and inside you see...
[*]a 

"Our automatic back-flayers are damaged, and we're need you to recover some components from the nearest scrapyard."
[*]"Have you seen my son? We lost him during an energy storm near a water bubble."
[*]"We have the cogs for the Forgelord, do you have the payment?"

Ten Mid Level Adventures in The Positive Energy Plane

You're pretty sure these schematics will tell you how to control the biomechanical golem. When you begin reading them, you realize...

[list][*]The opening in its chest cavity is a negative energy portal, and it's starting to open.
[*]This signature belongs to the Forgelord who sent you here in the first place.
[*]It's not a golem at all... it's a suit of armor. And it's occupied.
[*]The planar energy control rods aren't there to suppress the positive energy. They're there to amplify it.
[*]The inventor's notes indicate he may still be alive in here somewhere.
[*]An air element bubble is what's protecting you from the energy storm outside, and 
[*]
[*]
[*]
[*]
[/list]

Ten High Level Adventures in The Positive Energy Plane


\section{High Adventure in\dots The Negative Energy Plane!}

\noindent\desc{If you're even \emph{considering} running a game in the Negative Energy Plane, it is very probable that you are using Playing With Fire morality for your necromancy. This is in large part because every writeup of the NEP ever made has \emph{assumed} Playing With Fire, and that indeed it is precisely these descriptions that give people the best scriptural ammunition against Crawling Darkness. But also because if Negative Energy is inherently evil, the plane becomes incredibly \emph{boring}. We already \emph{have} the Gray Wastes of Gehenna, so there's no real point in having \emph{another} gray desert made out of ultimate evil.}

\noindent\desc{The game provides two supposedly different Negative Energy Planes for you to consider. One is made out of Major Negative Energy Dominant with patches that are Minor Negative Energy Dominant, and the other is made out of Minor Negative Energy Dominant with patches of Major Energy Dominant. Well, anyone who has ever looked at a splotchy cow knows that whether you have a black cow with white spots or a white cow with black spots is entirely a matter of perspective. Since the NEP is infinite, both Major and Minor patches are infinite in size and in scope, so it really makes no difference at all which one you are nominally using. From a practical standpoint, either way you're going to be in either a Major or Minor Negative Energy area, the adventure location you are going to next will either be in the same area or a different one, and if you go far enough in any direction you will go from one to the other. And anyway, both Minor and Major Negative Dominant ares are totally fatal to living creatures, and completely harmless to undead and constructs, and the baleful effects are completely negated by \emph{negative energy protection} or \emph{attune plane}. So seriously: who cares? Since the only actual difference is the unprotected living creatures crumble to ash in Minor Dominant and are transformed into wraiths in Major Dominant, our suggestion would be to go with Major Dominant most of the time. It's largely academic, because outside the planar bubbles \emph{there is no air} (so without some sort of magical attunement, every living creature is just going to die of asphyxiation, negative energy or no).}

\noindent\desc{The Negative Energy Plane hates life. It hates the good, and it hates the wicked both the same. It does not condone or aid harm or murder, it simply greedily and expeditiously extinguishes any life exposed to it. But if you're \emph{alive} that's basically no worse than the vacuum of space, and if you're \emph{not} alive it's a whole lot better. For those who are undead, non-living, or have the right kind of protections, the Negative Energy Plane is a lot like any other void plane of the D\&D cosmology save that there is no ambient light source. Comparisons can be made to Limbo, the Astral Plane, and of course: the Elemental Plane of Air. The difference is just the fact that it is unlit, and therefore looks like the night sky rather than extending out to a gray fog where the soft glow of the ambient light eventually wipes out anything you could see.}

\noindent\desc{Once you factor in the Planar Bubbles (which as an ironic statement, are called ``doldrums'' by Negative Energy inhabitants), the Negative Energy Plane is basically exactly the same as our universe. If you were on a prime bubble, you pretty much would only with difficulty be able to know that you weren't on a Prime. There's a dark hostile, airless void outside your planet, and there's absolutely nothing stopping any light source of any distance from eventually sending its ray to you. So the sky above you is black and full of tiny lights. Well, it wouldn't really be \emph{that} difficult to figure it out, because absolutely everyone can \emph{fly} just by thinking about it. And the lights in the sky are just like what ancient people thought about them: some of them are very large and far away (like Elemental Fire bubbles that function as stars), and others are more modest light sources that are more reasonable distances. The intrinsic flight includes not only hovering, but also \emph{acceleration} that is only relativistically limited. You can accelerate at 1G or more by sheer willpower as long as you want without energy expenditure. So a trip from the Earth to Mars would take less than 5 days even at its most distant point (assuming that they were both on the Negative Material Plane). So titanic, even \emph{solar} distances are quite reachable. Also of note is that the directions to Neverland (Third star on the left, and straight on 'til morning) are completely reasonable directions, and represent another planar bubble that is about 2 million kilometers away. Like all regions of subjective gravity, going ``towards'' a point will automatically have you accelerate continuously to the halfway mark and then have acceleration away from it for the rest of the journey, so you never ram into anything at relativistic speed.}

\noindent\desc{The distances between things in the Negative Elemental Plane are truly vast, but travel is so \emph{easy} that from a practical standpoint, things in the Negative Energy Plane are actually kind of ``happening.'' The exception of course, is unlit structures. These are called ``Castles Perilous'' by the locals, and making one is pretty much a declaration that you under no circumstances want visitors. After all, without giving off any light, you're basically about as findable as any rock out in deep space is in the real world. The only ways to find one are to happen to see them passing in front of a light source or to shoot one's self off into the void looking for the automatic deceleration that accompanies moving towards a real object -- and even knowing that second one is an option requires the kind of math you'd need a Knowledge (Planes or Engineering) DC 25 test to do.}

\noindent\desc{An important thing to consider is the presence of Voidstone. It's a special material that will destroy and absorb any creature (even undead creatures) if they come into contact with it for a few seconds. Truly badass creatures like dragons and gods \emph{might} be able to hold it for a minute or two before being eradicated from existence, but as you might imagine, that stuff is still in huge demand for making into weaponry. Since it doesn't do anything to other inert elemental material like, say \emph{metal tools}, it ends up being quite workable and incredibly valuable. Voidstone is planar currency for obvious reasons -- but finding it is very difficult because it's not very large, pure black, and forms in the middle of large sections of empty void.}

\noindent{But perhaps the most important point about the Negative Energy Plane is that the parity with the Positive Energy Plane is not complete. Living creatures are natural, so they have no protection from being exposed to ``too much'' positive energy -- and they can totally explode. Undead creatures are \emph{unnatural} and only exist at all because they are supported by magic to siphon off a specific and measured quantity of negative energy. So they don't ever ``explode'' in Negative Dominant areas, whether they have ``protection'' or not. As such, groups of intelligent undead often make homes out of Castles Perilous in the middle of strong Negative Energy Vortices. Because seriously: why not?}

\subsection*{Campaign Seed: Death World}

\noindent{A Doldrum region in the Negative Energy Plane is a lot like Neverland if it was made by American McGee. Everyone can fly like Peter Pan, and each region fills up with weird crap from all over the planes like tribes of Indians, mermaids, and pirates. However, these places are also constantly under assault by a low level rain of \emph{zombies from space}. That's not a joke, undead beasts literally float around in the void and choose to fall towards points of light. So if you're running around Pixie Hollow, there is a not insignificant chance that some undead monster is going to fall out of the sky and go on a rampage. This setup allows for very reasonably scaling D\&D adventuring. After all, if the PCs become masters of their surroundings and conquer the Maze of Regrets, you have a totally reasonable excuse to have a level appropriate undead army fall \emph{from space} and start causing havoc. In the meantime, even though the levels of Negative Energy aren't high enough to snuff the life out of anything, they \emph{are} leaking into Doldrums enough to make things \emph{subtly} creepy and unpleasant. Feel free to use any Ravenloft clich\'{e}s you want. Or just American McGee it up -- people live on a fricking \emph{Death World}, so have just messed up stuff happen all the time. Have cats croak out ``help... me...'' for no reason. Have thorns drip unexplained blood. Have trees inexplicably drain of color. Inhabitants go crazy and start eating pieces of themselves. Go nuts.}

\subsection*{Campaign Seed: Welcome to the Void Heart}

\noindent{There is a city built into the inside of a one-mile diameter iron Dyson Sphere which is called ``Heart of the Void'' or ``Deathheart'' depending on who you ask. Some sages built a city there a long time ago and eventually an army of the undead broke in and murdered everyone. Tonight it's a minor necropolis that is broken up into factions that fight each other for domination. And I know what you're thinking: \emph{so what?} I mean, that's only 3.14 square miles of city, and even though it has the population density of New York, it still only has 70,000 inhabitants, and a lot of them are ghouls. But the really important thing is what the sages used to \emph{do}, which was to track all the objects in the Negative Energy Plane. All the rocks of Voidstone, all the Castles Perilous, \emph{everything}. No one knows how they did it, because some vampiric minotaur killed the last of them a few hundred years back and feasted on her heart -- but they \emph{did} leave notes. All over the city, there are books filled with page after page of descriptions of the size, shape, and location of various objects in the void. There are a lot of adventures there: some books are useless without other books in the same series; some books are the possessions of hostile undead gangs that either do or do not know how valuable they are; and many books detail the locations of items and structures that are themselves interesting and valuable adventuring locales.}

\subsection*{Ten Low-Level Adventures in The Negative Energy Plane}

The ghoul chitters and licks his parched lips. Seemingly reluctant to proceed, he whispers...

\listone
\item ``You may have defeated me, but there are a \emph{dozen} more on their way...''
\item ``Fellnax wants his coins. He wants them bad...''
\item ``You can kill me, I'll never tell you were the diadem is.''
\item ``I knew someone would find me. I didn't know who, but after the Hellmire job, I knew it was only a matter of time...''
\item ``These bones... these bones are mine...''
\item ``You traitors! I'll feast on you!''
\item ``Do you have the \emph{scrolls}? My master said you would have the \emph{scrolls}...''
\item ``You don't look like Fellnax's men.''
\item ``Fellnax sent me to tell you, to tell you that he is going to kill all of you...''
\item ``We still have the girl, please don't do anything we'd both regret.''
\end{list}

\subsection*{Ten Mid-Level Adventures in The Negative Energy Plane}

It's good to meet another outworlder. But there's something weird about this guy...

\listone
\item There are faint sobs coming from his backpack.
\item He casts no reflection.
\item Everytime he mentions the Castle Perilous he came from, he looks over his shoulder.
\item There are the scars of bite marks all over his arm.
\item When he talks about his family getting eaten, it's like he doesn't even care.
\item When he mentions the golden statues of Kath, it's like he doesn't even care.
\item He seems genuinely relieved to be \emph{here}.
\item He steps right over the ghoul corpses as if that was a normal thing.
\item He has one of Fellnax's amulets. Or something that looks just like one...
\item There is a wraith following behind him, one that looks just like he does...
\end{list}

\subsection*{Ten High-Level Adventures in The Negative Energy Plane}

You've got a fix on the Voidstone you were looking for. Unfortunately it's...

\listone
\item Suspended inside the chest cavity of a dracolich.
\item Worshiped by a death cult of Kuo Toa.
\item Inside a Castle Perilous named ``Doom Watch''
\item Been made into a sword by a mad Duergar.
\item Guarded by a Void Shadow.
\item Guarded by a Shadow Dragon
\item The Tomb of a fallen god.
\item Locked in Lethe Ice.
\item On the far side of an Allip Belt
\item In the workshop of a Master Skincrafter.
\end{list}